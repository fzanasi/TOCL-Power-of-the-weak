

Consider a one-step logic $\llang_1(A)$ and formula $\varphi \in \llang_1(A)$.
%
We say that $\varphi$ is \emph{co-continuous in $a\in A$} if $\varphi$ is monotone in $a$ and,
for all $(D,\val)$ and $\ass:\fovar\to D$,
$$
\text{if } (D,\val),\ass \not\models \varphi \text{ then } \exists U \subseteq_\omega \val(a) \text{ such that } (D, \val[a \mapsto D\setminus U]),\ass \not\models \varphi.
$$

Observe that, already with the abstract definition of Boolean dual given in Definition~\ref{d:bdual1} we can prove the expected relationship between the notions of continuity and co-continuity.

\begin{proposition}\label{prop:contdualcocont}
	Let $\varphi \in \llang_1(A)$, $\varphi$ is continuous in $a\in A$ iff $\varphi^\delta$ is co-continuous in $a$.
\end{proposition}
% \fbox{$\Leftarrow$} 
% (D,V) \not\models phi 
% iff [dual, cf.Def 5 in LICS]
% (D,V^c) \models \phi^dual
% iff [cont in a]
% (D, V^c[a->F]) \models phi^dual    [F is finite]
% iff [dual]
% (D, V[a->D\F]) \not\models phi
\begin{proof}
Let $\varphi$ be continuous in $a$, we prove that $\varphi^\delta$ is co-continuous in $a$:
\begin{align*}
(D,\val) \not\models \varphi^\delta
& \text{ iff } (D,\val^c) \models \varphi & \tag{Definition~\ref{d:bdual1}} \\
& \text{ iff } \exists U \subseteq_\omega \val^c(a) \text{ such that } (D,\val^c[a\mapsto U]) \models \varphi & \tag{$\varphi$ continuous} \\
& \text{ iff } \exists U' \subseteq_\omega \val(a) \text{ such that } (D,\val[a\mapsto D\setminus U']) \not\models \varphi & \tag{Definition of $\val^c$}
\end{align*}
%
The proof of the other direction is analogous.
\end{proof}

To define a syntactic notion of co-continuity we first give a concrete definition of the dualization operator of Definition~\ref{d:bdual1} and then show that the one-step language $\olque$ is closed under Boolean duals.

\begin{definition}\label{DEF_dual} 
Let $\varphi \in {\olque}(A)$. 
The \emph{dual} $\varphi^{\delta} \in {\olque}(A)$ of $\varphi$ is defined 
as follows.
\begin{align*}
% \nonumber to remove numbering (before each equation)
 (a(x))^{\delta} & :=  a(x) 
\\ (\top)^{\delta} & :=  \bot 
  & (\bot)^{\delta} & :=  \top 
\\  (x \approx y)^{\delta} & :=  x \not\approx y 
  & (x \not\approx y)^{\delta}& :=  x \approx y 
\\ (\varphi \wedge \psi)^{\delta} &:=  (\varphi)^{\delta} \vee (\psi)^{\delta} 
  &(\varphi \vee \psi)^{\delta}& :=  (\varphi)^{\delta} \wedge (\psi)^{\delta}
\\ (\exists x.\psi)^{\delta} &:=  \forall x.(\psi)^{\delta} 
  &(\forall x.\psi)^{\delta} &:=  \exists x.(\psi)^{\delta} 
\\ (\exists^{\infty} x.\psi)^{\delta} &:= \forall^{\infty} x.(\psi)^{\delta} 
  &(\forall^{\infty} x.\psi)^{\delta} &:=  \exists^{\infty} x.(\psi)^{\delta}
\end{align*} %\hfill $\lhd$
\end{definition}

\begin{remark}
	Observe that if $\varphi \in {\ofo}(A)$ then $\varphi^{\delta} \in {\ofo}(A)$ and that the operator preserves positivity of the predicates. That is, if $\varphi \in {\olque}^+(A)$ then $\varphi^{\delta} \in {\olque}^+(A)$ and the same occurs with $\ofo^+(A)$.
\end{remark}

The proof of the following Proposition is a routine check.

\begin{proposition}\label{prop:duals}
The sentences $\phi$ and $\phi^{\delta}$ are Boolean duals, for every $\phi 
\in \olque(A)$.
\end{proposition}

We are now ready to give the syntactic definition of a co-continuous fragment for the one-step logics into consideration.

\begin{definition}\label{def:cocontfrag}
	Let $A$ be a set of names. The syntactic fragments of $\olque(A)$ and $\ofo(A)$ which are \emph{co-continuous} in $a\in A$ are given by
	%
	\begin{align*}
		\cocont{\olque}{a}(A) &:= \{\varphi \mid \varphi^\delta \in \cont{\olque}{a}(A)\} &
		\cocont{\ofo}{a}(A) &:= \{\varphi \mid \varphi^\delta \in \cont{\ofo}{a}(A)\} .
	\end{align*}
	%
\end{definition}

\begin{proposition}
	A formula $\varphi \in \olque(A)$ is co-continuous in $a\in A$ iff $\varphi \in \cocont{\olque}{a}(A)$.
\end{proposition}

\begin{proof} This is a consequence of Proposition~\ref{prop:contdualcocont}, Theorem~\ref{thm:olquecont} and Definition~\ref{def:cocontfrag}.
\end{proof}