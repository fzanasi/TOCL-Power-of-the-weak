\documentclass[prodmode,acmtecs]{acmsmall} % Aptara syntax

% Metadata Information
\acmVolume{0}
\acmNumber{0}
\acmArticle{0}
\acmYear{2017}
\acmMonth{0}

% Copyright
%\setcopyright{acmcopyright}
%\setcopyright{acmlicensed}
%\setcopyright{rightsretained}
%\setcopyright{usgov}
%\setcopyright{usgovmixed}
%\setcopyright{cagov}
%\setcopyright{cagovmixed}

% DOI
\doi{0000001.0000001}

%ISSN
\issn{1234-56789}

\usepackage{amsfonts}
\usepackage{xspace}
\usepackage{amssymb}
\usepackage{amsmath}
\usepackage[T1]{fontenc}

\newtheorem{fact}{Fact}
\newtheorem{claim}{Claim}

\usepackage[hidelinks]{hyperref}

\usepackage{enumerate}
\usepackage{multicol}
\usepackage{array}
\usepackage{eqparbox}
\usepackage{tkz-graph}


\usepackage{enumerate,xspace,stmaryrd,xcolor}
\usepackage[shortlabels]{enumitem}
\usepackage[e]{esvect}

\usepackage{tikz}
\usetikzlibrary{arrows,positioning}
\usetikzlibrary{automata,backgrounds,petri}

\usepackage{fixme}
%\usepackage[author=anonymous]{fixme}
\FXRegisterAuthor{af}{aaf}{AF}
\FXRegisterAuthor{fc}{afc}{FC}
\FXRegisterAuthor{fz}{afz}{FZ}
\FXRegisterAuthor{yv}{ayv}{YV}


%%%%%%%%%%%%%%%%%%%%%%%%%%%NEXT


% claims
\makeatletter
\newcounter{ourclaimcounter}
\newenvironment{ourclaim}{%
 	\stepcounter{ourclaimcounter}%
	\def\@currentlabel{\theourclaimcounter}%
	\begin{trivlist}\item[]{\sc Claim~\arabic{ourclaimcounter}.~}
	}
	{\end{trivlist}
        }
\newenvironment{claimfirst}{%
	\setcounter{ourclaimcounter}{0}%
	\begin{ourclaim}}{\end{ourclaim}}
\makeatother

% to be supplied
\newenvironment{tbs}{%
   \small\tt
   \begin{enumerate}[$\blacktriangleright$]}{\end{enumerate}}
\newcommand{\btbs}{\begin{tbs}}
\newcommand{\etbs}{\end{tbs}}
\newcommand{\shtbs}[1]{\begin{tbs} \item #1 \end{tbs}}

\newcommand{\todo}{\marginpar{$\bullet$}}

%%%%%%%%%% LEFTOVERS %%%%%%%%

%\newcommand*\sepsection{\bigskip}

% \newcommand{\umods}{\mathfrak{M}_1}
% \newcommand{\sumods}{\umods^s}
% \newcommand{\rest}{|}
% \newcommand{\iso}{\simeq}
% \newcommand{\resto}{{\upharpoonright}}
% \newcommand{\dcup}{\uplus}
% \newcommand{\stl}{\rightarrowtriangle} %\prec}
% \newcommand{\tc}{\mathsf{TC}}
% \newcommand{\st}{\mathrm{ST}}
% \newcommand{\ot}{\mathrm{OT}}
% \newcommand{\sorts}{\mathcal{S}}
% \newcommand{\asort}{\mathtt{s}}
% \newcommand{\aSort}{\mathtt{S}}
% \newcommand{\pto}{\rightharpoonup}
% \newcommand{\llet}{\subseteq}
% \newcommand{\lletneq}{\subsetneq}
% \newcommand{\ffunc}[2]{F^{#1}_{#2}}
% \newcommand{\somemod}{\heartsuit}
% \newcommand{\gsubmodel}[2]{{#1}^{\downarrow}_{#2}}
% \newcommand{\gsubaut}[2]{{#1}^{\blacktriangle}_{#2}}
% \newcommand{\pleadsto}[1]{R^{\shortrightarrow}_{#1}}

% \newcommand{\omls}{\oml^\flat}
% \newcommand{\oca}{\ensuremath{\mathrm{ADD}_1}\xspace}
% \newcommand{\ocm}{\ensuremath{\mathrm{MUL}_1}\xspace}
% \newcommand{\ocam}{\ensuremath{\oca^s}\xspace}
% \newcommand{\ocmm}{\ensuremath{\ocm^s}\xspace}
% \newcommand{\ogcam}{\ensuremath{\textup{g}\oca^s}\xspace}
% \newcommand{\ogcmm}{\ensuremath{\textup{g}\ocm^s}\xspace}

% mu-calculi
% \newcommand{\mucaML}{\ensuremath{\mu_{a}\ML}\xspace}
% \newcommand{\nmucaML}{\ensuremath{\mu_{na}\ML}\xspace}
% \newcommand{\rmucaML}{\ensuremath{\mu_{a}^{-}\ML}\xspace}
% \newcommand{\rnmucaML}{\ensuremath{\mu_{na}^{-}\ML}\xspace}
% \newcommand{\lmuML}{\ensuremath{\mu\ML^{\lor}}\xspace}
% \newcommand{\lmucML}{\ensuremath{\mu_{C}\ML^{\lor}}\xspace}
% \newcommand{\lnmucML}{\ensuremath{\mu_{nc}\ML^{\lor}}\xspace}
% \newcommand{\mufoe}{\ensuremath{\mu\foe}\xspace}
% \newcommand{\mufoei}{\ensuremath{\mufoe^\infty}\xspace}
% \newcommand{\mucfoe}{\ensuremath{\mu_{c}\foe}\xspace}
% \newcommand{\mucfoei}{\ensuremath{\mu_{c}\foe^\infty}\xspace}
% \newcommand{\mucafoe}{\ensuremath{\mu_{a}\foe}\xspace}
% \newcommand{\so}[1]{\mathrm{SO}(#1)}
% \newcommand{\afmc}{\ensuremath{\mathrm{AFMC}}\xspace}
% \newcommand{\muMC}{\ensuremath{\Sigma\mu\ML}}    %least fixpoint-fragment of Mu-calculus
% \newcommand{\muffoe}{\ensuremath{\mufoe^\ff}\xspace}
% \newcommand{\muffoei}{\ensuremath{{\mufoei}^\ff}\xspace}
% \newcommand{\mucaffoe}{\ensuremath{\mucafoe^\ff}\xspace}
% \newcommand{\mucffoei}{\ensuremath{{\mucfoei}^\ff}\xspace}

% \newcommand{\matches}{\mathcal{B}}
% \newcommand{\match}{\pi}
% \newcommand{\fwa}[2]{{#1}^{\!\div}_{#2}} %{\mathsf{\fconst}^{#2}_{wa}(#1)}
% \newcommand{\fwc}[2]{{#1}^{\!\div}_{#2}} %{\mathsf{\fconst}^{#2}_{wa}(#1)}

% \newcommand{\sgame}{\mathcal{S}}
% \newcommand{\agamesym}{\mathcal{A}_\textrm{s}}
% \newcommand{\agamenfa}{\underline{\mathcal{A}}} %{\mathcal{A}_\textrm{FA}}
% \newcommand{\xgame}{\mathcal{X}}

% \newcommand{\aut}{\ensuremath\mathbb{A}\xspace}
% \newcommand{\baut}{\ensuremath\mathbb{B}\xspace}
% \newcommand{\iaut}{\ensuremath\mathbb{X}\xspace}
% \newcommand{\efgame}{\mathrm{EF}}
% \newcommand{\game}{\mathcal{G}}

% \newcommand{\toprops}[1]{\varpi_c}
% \newcommand{\compset}[1]{\{#1\}}



%% version of May 12, 2004 by YV
%preamble

\usepackage{amssymb}
\usepackage{amsmath}
\usepackage{latexsym}

%%%%%%%%%%%%%%% SIZES %%%%%%%%%%%%%%

\setlength{\textwidth}{6.125in}
\setlength{\textheight}{210mm}
\setlength{\topmargin}{5mm}
\setlength{\headsep}{7mm}
\setlength{\marginparsep}{2mm}
\setlength{\marginparwidth}{0.75in}
\setlength{\oddsidemargin}{0.125in}
\setlength{\evensidemargin}{0.125in}

%%%%%%%%%%%%%%% SIZES %%%%%%%%%%%%%%

\newcommand{\emdef}[1]{\textsl{#1}}

%%%%%%%%%%%%%%%  TBS  %%%%%%%%%%%%%%

%\newcommand{\todo}{}
\newcommand{\todo}{\marginpar[$\bullet$]{$\bullet$}}
\newcommand{\hbml}{\textsc{hbml}\marginpar[$\blacktriangleleft$]
{$\blacktriangleleft$}}
%\newcommand{\hbml}{\textsc{hbml}}
\newenvironment{tbs}{%
   \small\tt
   \renewcommand{\labelitemi}{$\blacktriangleright$}%
   \begin{itemize}}{\end{itemize}}
\newcommand{\btbs}{\begin{tbs}}                                                                      
\newcommand{\etbs}{\end{tbs}}                                                                      
\newcommand{\shtbs}[1]{\begin{tbs} \item #1 \end{tbs}}

\newcommand{\hide}[1]{}

%%%%%%%%%%%%%%%  TBS  %%%%%%%%%%%%%%

%%%%%%%%%%%%%%% ENVIRONMENTS %%%%%%%%%%%%%%

%\newtheorem{theorem}{Theorem}[section]
%\newtheorem{fewtheorem}{Theorem}
%\newtheorem{fact}[theorem]{Fact}
%\newtheorem{proposition}[theorem]{Proposition}
%\newtheorem{lemma}[theorem]{Lemma}
%\newtheorem{corollary}[theorem]{Corollary}
%\newtheorem{defi}[theorem]{Definition}
%\newtheorem{conjecture}[theorem]{Conjecture}
%\newtheorem{conv}[theorem]{Convention}
%\newtheorem{rema}[theorem]{Remark}
%\newtheorem{exam}[theorem]{Example}
\newtheorem{observation}[theorem]{Observation} %%niet in LLNCS style

%\newenvironment{definition}{\begin{defi}\rm}{\hfill $\lhd$\end{defi}}
%\newenvironment{definition}{\begin{defi}\rm}{\end{defi}}
%\newenvironment{convention}{\begin{conv}\rm}{\end{conv}}
%\newenvironment{remark}{\begin{rema}\rm}{\hfill $\lhd$\end{rema}}
%\newenvironment{example}{\begin{exam}\rm}{\hfill $\lhd$\end{exam}}

%% proof
%\newenvironment{proof}{\begin{trivlist}\item[]{\bf
%Proof.}}{\hfill {\sc qed}\end{trivlist}}
\newenvironment{proofsketch}{\begin{trivlist}\item[]{\bf
Proof sketch.}}{\hfill {\sc qed}\end{trivlist}}
%\newenvironment{proofof}[1]{\begin{trivlist}\item[\hskip\labelsep{\bf
%Proof~of~{#1}.\ }]}{\hspace*{\fill} {\sc qed}\end{trivlist}}

%claims
%Use claim only inside a proof and start always with claimfisrt. 
%Then the claims are numbered inside the proof.

\newtheorem{claim2}{\sc Claim}
%\newenvironment{claim}{\begin{claim2}\rm}{\end{claim2}\rm}
\newenvironment{claimfirst}{\setcounter{claim2}{0}
               \begin{claim2}\rm}{\end{claim2}\rm}

\newenvironment{pfclaim}{\begin{trivlist}\item[]{\sc Proof of
Claim.}}{\hfill {\mbox{$\blacktriangleleft$}}\end{trivlist}}

\newtheorem{exer}{Exercise}[section]      
\newenvironment{exercise}{\begin{exer}\rm}{\end{exer}}
\newenvironment{exenumerate}{%
  \renewcommand\theenumi{\alph{\value{enumi}}}
  \renewcommand\theenumi{\@alph{\c@enumi}}
  \renewcommand\theenumii{\@roman\c@enumii}
  \renewcommand\theenumiii{\@Alph\c@enumiii}
  \renewcommand\theenumiv{\@Roman\c@enumiv}
  \renewcommand\labelenumi{(\theenumi)}
  \renewcommand\labelenumii{(\theenumii)}
  \renewcommand\labelenumiii{(\theenumiii)}
  \renewcommand\labelenumiv{(\theenumiv)}
  \begin{enumerate}}{\end{enumerate}}
\newcommand{\points}[1]{\marginpar{\fbox{\texttt{#1 pt}}}}

\newenvironment{opgenumerate}{%
  \renewcommand{\theenumi}{\alph{enumi}}%
  \renewcommand\theenumii{\roman{enumiii}}%
  \renewcommand{\labelenumi}{(\theenumi)}%
  \renewcommand{\labelenumii}{(\theenumii)}%
  \begin{enumerate}}{\end{enumerate}}


%%%%%%%%%%%%%%% ENVIRONMENTS %%%%%%%%%%%%%%

%%%%%%%%%%%%%% STRUCTURES %%%%%%%%%%%%%% 

\newcommand{\mathstr}[1]{\mathbb{#1}}
\newcommand{\la}{\langle}
\newcommand{\ra}{\rangle}
\newcommand{\struc}[1]{(#1)}

%objects
\newcommand{\bbA}{\mathstr{A}}
\newcommand{\bbB}{\mathstr{B}}
\newcommand{\bbC}{\mathstr{C}}
\newcommand{\bbD}{\mathstr{D}}
\newcommand{\bbE}{\mathstr{E}}
\newcommand{\bbF}{\mathstr{F}}
\newcommand{\bbG}{\mathstr{G}}
\newcommand{\bbK}{\mathstr{K}}
\newcommand{\bbL}{\mathstr{L}}
\newcommand{\bbM}{\mathstr{M}}
\newcommand{\bbN}{\mathstr{N}}
\newcommand{\bbO}{\mathstr{O}}
\newcommand{\bbP}{\mathstr{P}}
\newcommand{\bbQ}{\mathstr{Q}}
\newcommand{\bbR}{\mathstr{R}}
\newcommand{\bbS}{\mathstr{S}}
\newcommand{\bbT}{\mathstr{T}}
\newcommand{\bbX}{\mathstr{X}}
\newcommand{\bbZ}{\mathstr{Z}}


% relating objects
\newcommand{\iso}{\cong}
\newcommand{\inj}{\rightarrowtail}
\newcommand{\injr}[1]{\stackrel{#1}{\rightarrowtail}}
\newcommand{\injl}[1]{\stackrel{#1}{\leftarrowtail}}
\newcommand{\suj}{\twoheadrightarrow}
\newcommand{\sujr}[1]{\stackrel{#1}{\twoheadrightarrow}}
\newcommand{\sujl}[1]{\stackrel{#1}{\twoheadleftarrow}}
\newcommand{\ti}[1]{\widetilde{#1}}
\newcommand{\wh}[1]{\widehat{#1}}
\newcommand{\DR}[1]{R_{#1}}

\newcommand{\bis}{\mathrel{\mathchoice%
{\raisebox{.3ex}{$\,
  \underline{\makebox[.7em]{$\leftrightarrow$}}\,$}}%
{\raisebox{.3ex}{$\,
  \underline{\makebox[.7em]{$\leftrightarrow$}}\,$}}%
{\raisebox{.2ex}{$\,
  \underline{\makebox[.5em]{\scriptsize$\leftrightarrow$}}\,$}}%
{\raisebox{.2ex}{$\,
  \underline{\makebox[.5em]{\scriptsize$\leftrightarrow$}}\,$}}}}
\newcommand{\bisg}{\mathrel{\mathchoice%
{\raisebox{.3ex}{$\,
  \underline{\makebox[.7em]{$\Leftrightarrow$}}\,$}}%
{\raisebox{.3ex}{$\,
  \underline{\makebox[.7em]{$\Leftrightarrow$}}\,$}}%
{\raisebox{.2ex}{$\,
  \underline{\makebox[.5em]{\scriptsize$\Leftrightarrow$}}\,$}}%
{\raisebox{.2ex}{$\,
  \underline{\makebox[.5em]{\scriptsize$\Leftrightarrow$}}\,$}}}}


%other
\newcommand{\Paths}[1]{\mathit{Paths}_{#1}}
%\newcommand{\omunr}[1]{\overline{#1}}
\newcommand{\unrav}[2]{\mathbb{E}_{#1}(#2)}
\newcommand{\omunr}[1]{\unrav{\om}{#1}}
\newcommand{\ufr}[1]{#1_{\bullet}}

%%%%%%%%%%%%%% CLASSES %%%%%%%%%%%%%% 

\newcommand{\class}[1]{\mathsf{#1}}

\newcommand{\clC}{\class{C}}
\newcommand{\clD}{\class{D}}
\newcommand{\clE}{\class{EG}}
\newcommand{\clK}{\class{K}}
\newcommand{\clL}{\class{L}}
\newcommand{\clV}{\class{V}}

\newcommand{\Lcl}[1]{\class{L}(#1)}

\newcommand{\BA}{\class{BA}}
\newcommand{\BAO}[1]{\class{BAO}_{#1}}
\newcommand{\BAOt}{\BAO{\tau}}
\newcommand{\BAOp}[1]{\class{BAO}^{+}_{#1}}
\newcommand{\BAOpt}{\BAOp{\tau}}
\newcommand{\BAE}[1]{\class{BAE}_{#1}}
\newcommand{\BAEt}{\BAE{\tau}}
\newcommand{\BAM}[1]{\class{BAM}_{#1}}
\newcommand{\BAMt}{\BAM{\tau}}
\newcommand{\MA}{\class{MA}}

\newcommand{\Fram}[1]{\class{Fr}_{#1}}
\newcommand{\Framt}{\Fram{\tau}}
\newcommand{\FS}{\class{FS}}
\newcommand{\DGF}[1]{\class{DGF}_{#1}}
\newcommand{\DGFt}{\DGF{\tau}}

%class operations
\newcommand{\Oop}{\class{Op}}
\newcommand{\Cm}{\class{Cm}}
\newcommand{\Cst}{\class{Cst}}
\newcommand{\Hom}{\class{H}}
\newcommand{\Ob}{\class{Ob}}
\newcommand{\Prod}{\class{P}}
\newcommand{\Sir}{\class{Sir}}
\newcommand{\Str}{\class{Str}}
\newcommand{\Sub}{\class{S}}
\newcommand{\Subf}{\class{S}_{\class{f}}}
\newcommand{\Sum}{\class{\Sigma}}
\newcommand{\UProd}{\class{Pu}}
\newcommand{\UPow}{\class{Pw}}
%\newcommand{\Var}{\class{Var}}
\newcommand{\Covar}{\class{Covar}}

%%%%%%%%%%%%%%% SYNTAX %%%%%%%%%%%%%%

% Languages
%\newcommand{\Prop}{\ensuremath{\mathsf{Prop}}}        %PROP
\newcommand{\Prop}{\ensuremath{\mathsf{P}}}        %PROP
\newcommand{\Propa}{\ensuremath{\mathsf{Q}}}        %PROP
%\newcommand{\PropQ}{\ensuremath{\mathsf{Q}}}        %PROP
\newcommand{\Propb}{\ensuremath{\mathsf{R}}}        %PROP
\newcommand{\Propq}{\ensuremath{\mathsf{Pq}}}        %PROP

%%\newcommand{\Act}{\ensuremath{\mathsf{Act}}}       %Act
%\newcommand{\Act}{\ensuremath{\mathsf{D}}}       %Act
%\newcommand{\var}{\textsc{var}}                    %VAR
%\newcommand{\ML}{\ensuremath{\mathrm{ML}}}       %BML
%\newcommand{\BML}{\ensuremath{\mathrm{BML}}}       %BML
%\newcommand{\CML}{\ensuremath{\mathrm{CML}}}       %CML
%\newcommand{\PML}{\ensuremath{\mathrm{PML}}}       %PML
%\newcommand{\MFL}{\ensuremath{\mathrm{MFL}}}       %PML
%\newcommand{\MSL}{\ensuremath{\mathrm{MSL}}}       %PML
%\newcommand{\muML}{\ensuremath{\mu\ML}}    %muML
%\newcommand{\muCML}{\ensuremath{\mu\CML}}  %muPML
%\newcommand{\muPML}{\ensuremath{\mu\PML}}  %muPML
%\newcommand{\muMFL}{\ensuremath{\mu\MFL}}  %muMFL
%\newcommand{\muMSL}{\ensuremath{\mu\MSL}}       %PML
%\newcommand{\StwoS}{\ensuremath{\mathrm{S2S}}}       %PML
%\newcommand{\msol}{\mathrm{MSOL}}       
%\newcommand{\mso}{\mathrm{MSO}}       
%\newcommand{\sfor}{\unlhd}                                %sfor
%\newcommand{\psfor}{\lhd}                                 %psfor
%\newcommand{\Sfor}[1]{\ensuremath{\mathit{Sfor}(#1)}}     %Sfor
%\newcommand{\MLatt}{\mathit{MLatt}}
%\newcommand{\SLatt}{\mathit{SLatt}}
%\newcommand{\TLatt}{\mathit{TLatt}}
%\newcommand{\Latt}{\mathit{Latt}}
%\newcommand{\Bool}{\mathit{Bool}}
%\newcommand{\Boolt}{\mathit{Bool}_{\tau}}
%\newcommand{\MFO}{\mathit{FO}}
%\newcommand{\BF}{\mathit{BF}}
%\newcommand{\SBF}{\mathit{SBF}}

\newcommand{\yvF}{\contAFMC}
\newcommand{\yvFOE}{\mathit{FOE}}
\newcommand{\yvFO}{\mathit{FO}}
\newcommand{\yvAut}{\mathit{Aut}}
\newcommand{\yvwAut}{\mathit{Aut}_{w}}
\newcommand{\yvcwAut}{\mathit{Aut}_{cw}}

\newcommand{\yvMSO}{\ensuremath{\mathrm{MSO}}}
\newcommand{\yvWMSO}{\ensuremath{\mathrm{WMSO}}}
\newcommand{\yvWFMSO}{\ensuremath{\mathrm{WFMSO}}}
\newcommand{\yvL}{\mathcal{L}}
\newcommand{\yvLo}{\yvL_{1}}
\newcommand{\yvmLo}{\yvL_{1}^{+}}
\newcommand{\yvdual}[1]{#1^{\delta}}


\newcommand{\yvmonot}[1]{{#1}^{+}}


\newcommand{\FV}[1]{\mathit{FV}(#1)}
\newcommand{\BV}[1]{\mathit{BV}(#1)}

\newcommand{\lang}{\mathcal{L_\mu}}              %Mu Language

\newcommand{\Ebis}{\exists}
\newcommand{\UI}[2]{I_{#1}(#2)}


% boolean symbols
\newcommand{\dto}{\leftrightarrow}
\newcommand{\muneg}{{\sim}}
\newcommand{\autneg}[1]{\widetilde{#1}}
\newcommand{\bv}{\bigvee}
\newcommand{\bw}{\bigwedge}
\newcommand{\bwsmall}{\textstyle{\bigwedge}}

%modal symbols
% use "\Box" for the modal box
\newcommand{\dia}{\Diamond}
% use the following for indexed diamonds as in PDL
\newcommand{\mop}[1]{\dia_{#1}}
\newcommand{\dmop}[1]{\Box_{#1}}
%\newcommand{\mop}[1]{\la #1 \ra}
%\newcommand{\dmop}[1]{[#1]}
\newcommand{\nxt}[1]{\raisebox{.3ex}{$\scriptstyle \bigcirc$}_{#1}}
\newcommand{\nb}{\nabla}

\newcommand{\hs}{\heartsuit}
\newcommand{\cbox}{\blacksquare}
\newcommand{\cdia}{\blacklozenge}
% \newcommand{\mopastar}{\langle d^{*} \rangle}
\newcommand{\mopastar}{\langle * \rangle}

\newcommand{\ssse}{\sqsubseteq}
%\newcommand{\Rso}{\rl{R}}
\newcommand{\Rso}{R}
\newcommand{\act}[1]{{\Downarrow}#1}


%various
\renewcommand{\phi}{\varphi} % nicer \phi
\newcommand{\isbnf}{\mathrel{::=}}
\newcommand{\divbnf}{\mathrel{|}}
\newcommand{\is}{\approx}
\newcommand{\issm}{\preceq}
\newcommand{\len}[1]{|#1|}
\newcommand{\diff}[1]{\mathtt{diff}(#1)}
\newcommand{\emppred}[1]{\mathtt{empty}(#1)}
\newcommand{\sing}[1]{\mathtt{sing}(#1)}

%logics
\newcommand{\logK}{\mathbf{K}}
%\newcommand{\ext}{.}
\newcommand{\Kmu}{\mathbf{K\mu}}

%%%%%%%%%%%%%%  SEMANTICS  %%%%%%%%%%%%%% 

\newcommand{\waar}{\Vdash}
\newcommand{\forces}{\Vdash}
\newcommand{\gwaar}{\forces_{g}}
\newcommand{\mng}[1]{[\![ #1 ]\!]}
\newcommand{\meq}{\leftrightsquigarrow}

\newcommand{\exop}[1]{\langle #1 \rangle}
\newcommand{\unop}[1]{[#1]}

\newcommand{\Bhv}{\mathit{Bhv}}
\newcommand{\Cov}{\class{Cov}}
\newcommand{\Mod}{\class{Mod}}
\newcommand{\Equ}{\mathit{Equ}}
\newcommand{\Log}{\mathit{Log}}
\newcommand{\NExt}[1]{\mathrm{NExt}(#1)}
\newcommand{\Th}{\mathit{Log}}
\newcommand{\Fr}{\class{Fr}}

%%%%%%%%%%%%%% COALGEBRA %%%%%%%%%%%%%% 

\newcommand{\Set}{\mathsf{Set}}
\newcommand{\Rel}{\mathsf{Rel}}
\newcommand{\SET}{\mathsf{SET}}
\newcommand{\Stone}{\mathsf{Stone}}
\newcommand{\Alg}{\mathsf{Alg}}
\newcommand{\Coalg}{\mathsf{Coalg}}

\newcommand{\dueq}{\rightleftharpoons}
\newcommand{\pb}{\mathit{pb}}
%\newcommand{\opp}[1]{#1^{\mathit{op}}}
\newcommand{\idar}{\mathit{id}}
\newcommand{\incl}{\hookrightarrow}

%\newcommand{\F}{\mathcal{F}}
\newcommand{\F}{\Omega}
%\newcommand{\G}{\Psi}
\newcommand{\FC}{\F_{C}}
\newcommand{\FX}{\F_{X}}
%\newcommand{\funG}{\mathcal{G}}
\newcommand{\funG}{\Xi}
\newcommand{\Viet}{\mathcal{V}}
\newcommand{\funId}{\mathcal{I}}
%\newcommand{\funC}[1]{\mathcal{C}_{#1}}
\newcommand{\funC}[1]{#1}
%\newcommand{\funP}{\mathcal{P}}
\newcommand{\funP}{\pw}
\newcommand{\funPom}{\funP_{\omega}}
\newcommand{\olP}{\breve{\mathcal{P}}}
\newcommand{\funPP}{\olP \circ \olP}
\newcommand{\UpP}{\mathcal{U}_{\olP}}
\newcommand{\FiP}{\mathcal{F}_{\olP}}

\newcommand{\btC}{\mathsf{B}_{C}}
\newcommand{\btCone}{\mathsf{B}_{C_{1}}}
\newcommand{\Kr}{\mathsf{K}}
%\newcommand{\Kr}{\mathcal{\Lambda}}
\newcommand{\Hr}{\Lambda}
%\newcommand{\Hr}{\Xi}
\newcommand{\Ing}{\mathit{Ing}}
\newcommand{\FmK}{\Fm_{\Kr}}
%\newcommand{\next}{\odot}

\newcommand{\rl}[1]{\overline{#1}}
\newcommand{\beh}{{!}}
\newcommand{\Beh}{\mathit{beh}}


%%%%%%%%%%%%%%  AUTOMATA   %%%%%%%%%%%%%% 


\newcommand{\q}{a}
\newcommand{\Q}{A}
\newcommand{\ai}{\q_{I}}
\newcommand{\at}{a_{\top}}
\newcommand{\af}{a_{\bot}}
\newcommand{\qi}{q_{I}}
\newcommand{\bi}{b_{I}}
\newcommand{\zi}{z_{I}}
\newcommand{\Acc}{\mathit{Acc}}
\newcommand{\cM}{\mathcal{M}}
\newcommand{\Lom}{L_{\om}}

%\newcommand{\ttr}[1]{\twoheadrightarrow^{#1}}
\newcommand{\tr}[1]{\stackrel{#1}{\rightarrow}}
\newcommand{\ttr}[1]{\stackrel{#1}{\twoheadrightarrow}}
\newcommand{\ttrF}[1]{\stackrel{#1}{\twoheadrightarrow_{F}}}
%\newcommand{\bttr}[2]{\stackrel{#1}{\twoheadrightarrow}_{#2}}
\newcommand{\bttr}[2]{\overset{#1}{\underset{#2}{\twoheadrightarrow}}}

\newcommand{\lar}{\triangledown}

\newcommand{\sh}[1]{{#1}^{\sharp}}
\newcommand{\rA}{\sh{A}}
\newcommand{\Ri}{R_{I}}
\newcommand{\NOT}{\mathrm{NBT}}
\newcommand{\SCC}{\mathrm{SCC}}
\newcommand{\ind}{\mathit{ind}}

\newcommand{\bbAun}{\bbA_{\cup}}
\newcommand{\bbAit}{\bbA_{\cap}}
\newcommand{\bbAneg}{\autneg{\bbA}}
\newcommand{\Deun}{\De_{\cup}}
\newcommand{\Deit}{\De_{\cap}}
\newcommand{\Deneg}{\De_{\sim}}

\newcommand{\gprod}{\rtimes}

%%%%%%%%%%%%%%  AUTOMATA   %%%%%%%%%%%%%% 

\newcommand{\rstaut}[1]{\!\upharpoonright_{#1}\,}
\newcommand{\prj}[2]{\pi_{#1}{\bbA}}

%%%%%%%%%%%%%%   GAMES     %%%%%%%%%%%%%% 

\newcommand{\AG}{\mathcal{A}}
\newcommand{\BG}{\mathcal{B}}
\newcommand{\EG}{\mathcal{E}}
\newcommand{\GG}{\mathcal{G}}
%\newcommand{\EmG}{\mathcal{G}_{\nada}}
\newcommand{\EmG}{\mathcal{S}}
\newcommand{\RG}{\mathcal{R}}
\newcommand{\UG}{\mathcal{U}}
%\newcommand{\NGm}{\mathcal{N}}
\newcommand{\cG}{\mathcal{G}}
\newcommand{\eloi}{\exists}
\newcommand{\abel}{\forall}
%\newcommand{\winner}{\mathrm{winner}}
\newcommand{\winner}{W}
\newcommand{\Win}{\mathrm{Win}}
%\newcommand{\plr}{P}
\newcommand{\plr}{\sigma}
\newcommand{\Plr}[1]{P_{#1}}
\newcommand{\opp}[1]{\bar{#1}}
\newcommand{\Unfinf}[1]{\mathit{Unf}^{\infty}(#1)}
\newcommand{\OOccinf}[2]{\mathit{Inf}_{#2}(#1)}
\newcommand{\Occ}[1]{\mathit{Occ}(#1)}
\newcommand{\Occinf}[1]{\mathit{Inf}(#1)}
\newcommand{\Occfin}[1]{\mathit{Occ}^{<\om}(#1)}
\newcommand{\PM}[1]{\mathrm{PM}_{#1}}

\newcommand{\Attr}{\mathit{Attr}}
\newcommand{\attr}{\mathit{attr}}

%%%%%%%%%%%%%%   GAMES     %%%%%%%%%%%%%% 

%%%%%%%%%%%%%% MATHEMATICS %%%%%%%%%%%%%% 

%various
\newcommand{\bpr}{{\upharpoonright}}
\newcommand{\rst}[1]{{\upharpoonright}_{#1}\,}
%\newcommand{\pw}{\mathit{Sb}}
\newcommand{\nada}{\varnothing}
%\newcommand{\cpl}[1]{{-}_{#1}}
\newcommand{\cpl}{{-}}
\newcommand{\rcpl}[1]{{\sim}_{#1}}
\newcommand{\yiff}{\mbox{ iff }}
\newcommand{\paar}[2]{\langle #1, #2 \rangle}
\newcommand{\haak}[1]{[\![#1]\!]}
\newcommand{\Sb}[1]{\mathcal{P}(#1)}
\newcommand{\sse}{\subseteq}
\newcommand{\quot}{/\!\!}
\newcommand{\Diag}{\Delta}
\newcommand{\Id}[1]{\mathit{Id}_{#1}}
\newcommand{\Un}{\Upsilon}
\newcommand{\size}[1]{|#1|}
\newcommand{\first}{\mathit{first}}
\newcommand{\last}{\mathit{last}}

\newcommand{\Bdual}[1]{\tilde{#1}}

%\newcommand{\pw}{\mathcal{P}}
%\newcommand{\pw}{\wp}
%\newcommand{\apw}{\mathstr{P}}
\newcommand{\PE}{\pw_{\eloi}}
\newcommand{\PA}{\pw_{\abel}}


%%%%%%%%%%%%%% VARIOUS ABBREVIATIONS %%%%%%%%%%%%%% 

%Greek letters
\newcommand{\Si}{\Sigma}
\newcommand{\Ga}{\Gamma}
\newcommand{\De}{\Delta}
\newcommand{\Om}{\Omega}
\newcommand{\al}{\alpha}
\newcommand{\be}{\beta}
\newcommand{\de}{\delta}
\newcommand{\ga}{\gamma}
\newcommand{\si}{\sigma}
\newcommand{\om}{\omega}


%various
\newcommand{\isdef}{\mathrel{:=}}
\newcommand{\isbydef}{\;:\!\iff\;}
\newcommand{\ol}[1]{\overline{#1}}
\newcommand{\Succ}[1]{\mathit{Succ}_{#1}}
\newcommand{\Ord}{\mathcal{O}}
%Fixed points
\newcommand{\pre}[1]{\ensuremath{\mathrm{PRE}({#1})}}           %POS(phi)
\newcommand{\pos}[1]{\ensuremath{\mathrm{POS}({#1})}}           %POS(phi)
\newcommand{\fix}[1]{\ensuremath{\mathrm{FIX}({#1})}}           %FIX(phi)
\newcommand{\lfp}[1]{\ensuremath{\mathrm{LFP.}#1}}             %LFP.phi
\newcommand{\gfp}[1]{\ensuremath{\mathrm{GFP.}#1}}             %GFP.phi
%\newcommand{\alfp}[2]{\ensuremath{\mathrm{LFP}^{#1}.}#2}     %LFP.phi
%\newcommand{\agfp}[2]{\ensuremath{\mathrm{GFP}^{#1}.}#2}     %LFP.phi
\newcommand{\alfp}[2]{\ensuremath{{#2}^{#1}_{\mu}}}     %LFP.phi
\newcommand{\agfp}[2]{\ensuremath{{#2}^{#1}_{\nu}}}    %LFP.phi
\newcommand{\Dom}{\mathsf{Dom}}
\newcommand{\Ran}{\mathsf{Ran}}
\newcommand{\eps}{\epsilon}
\newcommand{\lts}{LTS}

%\newcommand{\nrpageone}{\setcounter{page}{1}}
\newcommand{\nrpageone}{}
\newcommand{\coloneqq}{\mathrel{:=}}
\newcommand{\ovl}[1]{\overline{#1}}
\newcommand{\start}{\Rightarrow}

\newtheorem{maintheorem}{Theorem}

\newcommand {\scc}{\mathsf{scc}}

\renewcommand{\sc}{\scshape}
\newenvironment{Iff-RL}{\textbf{($\Rightarrow$)} }{\bigskip}
\newenvironment{Iff-LR}{\textbf{($\Leftarrow$)} }{}
%\newcommand{\m}{\mathit}
\newcommand{\mb}{\mathbb}
\def \: {\colon}
\def \p {\wp}
%\newcommand{\lift}[1]{{#1}^{\wp}}

\newcommand{\mrg}{\hspace{-0.15mm}} % margin correction of references for theorems and definitions


%Notation
%\newcommand{\shA}{\mb{A}^{\wp}}

\newcommand{\mbshA}{\mb{A}^{\sharp}}
%\newcommand{\shA}{A^{\sharp}}
\newcommand{\shai}{a_{I}^{\sharp}}
\newcommand{\shDe}{\Delta^{\sharp}}
\newcommand{\bmDe}{\Delta^{\flat}}
\newcommand{\shf}{f^{\sharp}}      % Strategy associated with the ref pow constr
\newcommand{\shm}{\pi^{\sharp}}    % Match associated with the ref pow constr
\newcommand{\shG}{\mc{G}^{\sharp}} % Game associated with the ref pow constr
\newcommand{\NBT}{\m{NBT}}
\newcommand{\Up}[1]{{\Uparrow}{#1}}
\newcommand{\WC}[1]{N#1}
\newcommand{\V}{\tscolors}
\newcommand{\R}[1]{R[#1]}
%\newcommand{\tmapProj}{\widetilde{\Delta}} % Transition function of the projection automaton ALREADY DEFINED IN COMMANDS.TEX
\newcommand{\MSO}{\mso}
%\newcommand{\WFMSO}{\mathrm{WFMSO}}
\newcommand{\NDB}{\mathrm{NDB}}

%\newcommand{\f}{F} % SUPSCRIPT of the finitary construction

\newcommand{\lng}{\mathcal{L}} % language of transition systems accepted by a parity automaton

\newcommand{\mgFOETrsym}{\star} % Translation symbol
\newcommand{\mgFOETr}[1]{(#1)^\mgFOETrsym} % Translation from mgFOE into WMSO (Section 6)


\newcommand{\noet}{{\scriptscriptstyle N}}
\newcommand{\fin}{{\scriptscriptstyle F}}
\newcommand{\buc}{{\scriptscriptstyle B}}
\newcommand{\noetexists}{\exists_{\noet}}
\newcommand{\finexists}{\exists_{\fin}}
\newcommand{\df}{\mathrel{: =}}


% DA FACUNDO

\newcommand{\mlque}{\ensuremath{\mu\lque}\xspace}%\omega}\xspace}%(\qu)}\xspace}
\newcommand{\clque}{\ensuremath{\mu_{c}\lque}\xspace}%\omega}\xspace}%(\qu)}\xspace}
\newcommand{\lque}{\ensuremath{{\foe}^\infty}\xspace}%\omega}\xspace}%(\qu)}\xspace}
\newcommand{\mondbnfolque}[4]{\dbnf^{#4}_{\lque}(#1,#2,#3)}		% disjunct of basic normal form monotone LQUE
\newcommand{\posdbnfolque}[3]{\dbnf^+_{\lque}(#1,#2,#3)}		% disjunct of basic normal form positive LQUE
\newcommand{\contAFMC}{\ensuremath{\mu_{c}\ML}\xspace}%{\ensuremath{\mathsf{F}}}

% DA YDE
\newcommand{\om}{\omega}
\newcommand{\al}{\alpha}


%

% macros
\newcommand{\nat}{\ensuremath{\mathbb{N}}\xspace}
\newcommand{\tup}[1]{\langle{#1}\rangle}
\newcommand{\ext}[1]{\llbracket#1\rrbracket}
\newcommand{\prop}{\mathsf{P}}
\newcommand{\qprop}{\mathsf{Q}}
\newcommand{\acts}{\mathsf{D}}
\newcommand{\model}{\mathbb{T}}
\newcommand{\osmodel}{\mathbf{D}}
\newcommand{\compset}[1]{\{#1\}}
\newcommand{\mc}{\mathcal}
\newcommand{\mmodels}{\Vdash}
\newcommand{\umods}{\mathfrak{M}_1}
\newcommand{\rest}{|}
\newcommand{\omegaunrav}[1]{{#1}^\omega}
\newcommand{\unravel}[1]{{#1}^e}


\newcommand{\MC}{\ensuremath{\mu\ML}\xspace}
\newcommand{\AFMC}{\ensuremath{\mathrm{AFMC}}\xspace}
\newcommand{\contAFMC}{\ensuremath{\mu_{c}\ML}\xspace}%{\ensuremath{\mathsf{F}}}
\newcommand{\nmso}{\ensuremath{\mathrm{NMSO}}\xspace}
\newcommand{\NMSO}{\nmso}
\newcommand{\WMSO}{\wmso}
\newcommand{\wmso}{\ensuremath{\mathrm{WMSO}}\xspace}
\newcommand{\owmso}{\ensuremath{\mathrm{WMSO}_1}\xspace}
\newcommand{\wmsoe}{\ensuremath{\mathrm{WMSO}^=}\xspace}
\newcommand{\owmsoe}{\ensuremath{\mathrm{WMSO}^=_1}\xspace}
\newcommand{\fo}{\ensuremath{\mathrm{FO}}\xspace}
\newcommand{\gfo}{\ensuremath{\mathrm{gFO}}\xspace}
\newcommand{\mgfo}{\ensuremath{\mu\mathrm{gFO}}\xspace}
\newcommand{\ofo}{\ensuremath{\fo_1}\xspace}%\mathrm{FO}_1}\xspace}
\newcommand{\foe}{\ensuremath{\mathrm{FOE}}\xspace}
\newcommand{\gfoe}{\ensuremath{\mathrm{gFOE}}\xspace}
\newcommand{\mgfoe}{\ensuremath{\mu\mathrm{gFOE}}\xspace}
\newcommand{\ofoe}{\ensuremath{\foe_1}\xspace}%\mathrm{FO}^=_1}\xspace}
\newcommand{\cont}[2]{#1\mathrm{C}_{#2}}
\newcommand{\cocont}[2]{#1\overline{\mathrm{C}}_{#2}}
\newcommand{\monot}[2]{#1\mathrm{M}_{#2}}
\newcommand{\mclass}{K}%{\ensuremath{\mathsf{C}}\xspace}
\newcommand{\llang}{\ensuremath{\mathcal{L}}\xspace}
\newcommand{\trees}{\ensuremath{\mathcal{T}}}
\newcommand{\autlang}{\ensuremath{L}}

\newcommand{\qu}{\ensuremath{\exists^\infty}\xspace}%{\mathcal Q}_0}\xspace}
\newcommand{\fqu}{\ensuremath{\exists^{<\omega}}\xspace}%{\mathcal Q}_0}\xspace}
\newcommand{\dqu}{\ensuremath{\forall^\infty}\xspace}%\overline{{\mathcal Q}_0}}\xspace}
\newcommand{\wqu}{\ensuremath{\mathbf{W}}\xspace}
\newcommand{\lqu}{\ensuremath{\fo^\infty}\xspace}%\omega}\xspace}%(\qu)}\xspace}
%\newcommand{\olqu}{\ensuremath{\lqu_1}\xspace}
\newcommand{\lque}{\ensuremath{{\foe}^\infty}\xspace}%\omega}\xspace}%(\qu)}\xspace}
\newcommand{\mlque}{\ensuremath{\mu\lque}\xspace}%\omega}\xspace}%(\qu)}\xspace}
\newcommand{\clque}{\ensuremath{\mu_{c}\lque}\xspace}%\omega}\xspace}%(\qu)}\xspace}
\newcommand{\olque}{\ensuremath{\lque_1}\xspace}

\newcommand{\dbnf}{\nabla}										% disjunct of basic normal form
\newcommand{\dbnfofo}[1]{\dbnf_{\fo}(#1)}						% disjunct of basic normal form FO
\newcommand{\dgbnfofo}[2]{\dbnf_{\fo}(#1,#2)}					% disjunct of generalized basic normal form FO
\newcommand{\dbnfofoe}[2]{\dbnf_{\foe}(#1,#2)}					% disjunct of basic normal form FOE
\newcommand{\dbnfolque}[3]{\dbnf_{\lque}(#1,#2,#3)}				% disjunct of basic normal form LQUE
\newcommand{\dbnfinf}[1]{\dbnf_\infty(#1)}

\newcommand{\mondbnfofo}[2]{\dbnf^{#2}_\fo(#1)}					% disjunct of basic normal form monotone FO
\newcommand{\mondgbnfofo}[3]{\dbnf^{#3}_\fo(#1,#2)}				% disjunct of generalized basic normal form monotone FO
\newcommand{\mondbnfofoe}[3]{\dbnf^{#3}_{\foe}(#1,#2)}			% disjunct of basic normal form monotone FOE
\newcommand{\mondbnfolque}[4]{\dbnf^{#4}_{\lque}(#1,#2,#3)}		% disjunct of basic normal form monotone LQUE
\newcommand{\mondbnfinf}[2]{\dbnf^{#2}_\infty(#1)}

\newcommand{\posdbnfofo}[1]{\dbnf^+_{\fo}(#1)}					% disjunct of basic normal form positive FO
\newcommand{\posdgbnfofo}[2]{\dbnf^+_{\fo}(#1,#2)}				% disjunct of generalized basic normal form positive FO
\newcommand{\posdbnfofoe}[2]{\dbnf^+_{\foe}(#1,#2)}				% disjunct of basic normal form positive FOE
\newcommand{\posdbnfolque}[3]{\dbnf^+_{\lque}(#1,#2,#3)}		% disjunct of basic normal form positive LQUE
\newcommand{\posdbnfinf}[1]{\dbnf^+_\infty(#1)}

\newcommand{\sovar}{\mathsf{Var}}
\newcommand{\fovar}{\mathsf{iVar}}
\newcommand{\cov}{\mathsf{cov}}
\newcommand{\ass}{g}
\newcommand{\val}{V}
\newcommand{\tscolors}{\sigma}
\newcommand{\tsval}{\tscolors^\flat}

\newcommand{\aut}{\ensuremath\mathbb{A}\xspace}
\newcommand{\tmap}{\Delta}
\newcommand{\pmap}{\Omega}
\newcommand{\ord}{\preceq}
\newcommand{\reach}{\leadsto}
\newcommand{\lthen}{\to}

\newcommand{\inc}{\sqsubseteq}
\newcommand{\here}[1]{{\Downarrow}#1}
\newcommand{\arediff}[1]{\mathrm{diff}(#1)}
\newcommand{\foeq}{\approx}
\newcommand{\tcont}{\vartriangle}
\newcommand{\tmono}{\circ}
\newcommand{\vlist}[1]{\vv{#1}}

\newcommand{\efgame}{\mathrm{EF}}
\newcommand{\egame}{\mathcal{E}}
\newcommand{\eloise}{\ensuremath{\exists}\xspace}
\newcommand{\abelard}{\ensuremath{\forall}\xspace}
\newcommand{\pmatches}[2]{\mathsf{PM}^{#1}_{#2}}
\newcommand{\win}{\text{\sl Win}} 
% Document starts
\begin{document}

% Page heads
\markboth{F. Carreiro, A. Facchini, Y. Venema, F. Zanasi}{The Power of the Weak}

% Title portion
\title{The Power of the Weak}

\author{
Facundo Carreiro
\affil{XXX}
Alessandro Facchini
\affil{IDSIA, Switzerland}
Yde Venema
\affil{University of Amsterdam}
Fabio Zanasi
\affil{University College London}
}

\begin{abstract}
BLABLA
\end{abstract}


%
% The code below should be generated by the tool at
% http://dl.acm.org/ccs.cfm
% Please copy and paste the code instead of the example below. 
%
 \begin{CCSXML}
<ccs2012>
<concept>
 <concept_id>10003752.10003790</concept_id>
 <concept_desc>Theory of computation~Logic</concept_desc>
 <concept_significance>500</concept_significance>
 </concept>
 <concept>
 <concept_id>10003752.10003766</concept_id>
 <concept_desc>Theory of computation~Formal languages and automata theory</concept_desc>
 <concept_significance>500</concept_significance>
</ccs2012>
\end{CCSXML}
% \begin{CCSXML}
% <ccs2012>
% <concept>
% <concept_id>10003752.10003766</concept_id>
% <concept_desc>Theory of computation~Formal languages and automata theory</concept_desc>
% <concept_significance>500</concept_significance>
% </concept>
% <concept>
% <concept_id>10003752.10003790</concept_id>
% <concept_desc>Theory of computation~Logic</concept_desc>
% <concept_significance>500</concept_significance>
% </concept>
% </ccs2012>
% \end{CCSXML}

\ccsdesc[500]{Theory of computation~Logic}
\ccsdesc[500]{Theory of computation~Formal languages and automata theory}

\keywords{Modal $\mu$-Calculus, Weak Monadic Second Order Logic, Tree Automata, Bisimulation}

\acmformat{Facundo Carreiro, Alessandro Facchini, Yde Venema, Fabio Zanasi, 2017. The Power of the Weak.}


%\begin{bottomstuff}
%The third author has been supported by Poland's National Science Centre (decision DEC-2012/05/N/ST6/03254).
%\end{bottomstuff}

\maketitle


%%%%

\clearpage

\tableofcontents

\clearpage


%%%%
%%%% INTRODUCTION
%%%%

\section{Introduction}\label{sec:intro}
% \section{Introduction}\label{sec:intro}

\subsection{Expressiveness modulo bisimilarity.}
%
This paper concerns the relative expressive power of some languages used for
describing properties of pointed labelled transitions systems, or Kripke
models.
The interest in such expressiveness questions stems from applications where
these structures model computational processes, and bisimilar pointed
structures represent the \emph{same} process.
Seen from this perspective, properties of transition structures are relevant
only if they are invariant under bisimilarity.
This explains the importance of bisimulation invariance results of the form
\begin{equation*}
%\label{eq:1}
%\eqno{(*)}
M \equiv L / {\bis} \text{ (over $K$)}
\end{equation*}
stating that,  if one restricts attention to a certain class $K$ of transition
structures, one language $M$ is expressively complete with respect to the
relevant (i.e., bisimulation-invariant) properties that can be formulated in
another language $L$.
In this setting, generally $L$ is some rich yardstick formalism such as
first-order or monadic second-order logic, and $M$ is some modal-style
fragment of $L$, usually displaying much better computational behavior
than the full language $L$.

A seminal result in the theory of modal logic is van Benthem's Characterisation
Theorem~\cite{vanBenthemPhD}, stating that every bisimulation-invariant
first-order formula $\alpha(x)$ is actually equivalent to (the standard
translation of) a modal formula:
\begin{equation*}
%\label{eq-vB}
\ML \equiv \fo/{\bis} \text{ (over the class of all LTSs)}.
\end{equation*}
Over the years, a wealth of variants of the Characterisation Theorem have been
obtained.
For instance, Rosen proved that van Benthem's theorem is one of the few
preservation results that transfers to the setting of finite
models~\cite{rose:moda97}; for a recent, rich source of van Benthem-style
characterisation results, see Dawar \& Otto~\cite{DawarO09}.
In this paper we are mainly interested is the work of Janin \&
Walukiewicz~\cite{Jan96}, who extended van Benthem's result to the setting
of fixpoint logics, by proving that the modal $\mu$-calculus ($\MC$) is the
bisimulation-invariant fragment of monadic second-order
logic ($\mso$):
\begin{equation*}
%\label{eq-JW}
\MC \equiv \mso/{\bis} \text{ (over the class of all LTSs)}.
\end{equation*}

\subsection{Bisimulation invariance for $\wmso$.}
The yardstick logic that we consider in this paper is \emph{weak} monadic
second-order logic ($\wmso$), a variant of monadic second-order logic where
the second-order quantifiers range over \emph{finite} subsets of the
transition structure rather than over arbitrary ones.
Our target will be to identify the bisimulation-invariant fragment of this
logic $\wmso$.

Before moving on, we should stress the role of the ambient class $K$ in
bisimulation-invariance results.
Of particular importance in the setting of weak monadic second-order logic is
the difference between structures of finite versus arbitrary branching degree.
In the case of finitely branching models, it is not very hard to show that
$\wmso$ is a (proper) fragment of $\mso$, and it seems to be folklore that
$\wmso/{\bis}$ corresponds to $\afmc$, the alternation-free fragment of the
modal $\mu$-calculus.
For binary trees, this result was proved by Arnold \& Niwi{\'n}ski in
\cite{ArnoldN01}.
In the case of structures of arbitrary branching degree, however, $\wmso$
and $\mso$ have \emph{incomparable} expressive power.
The fact that, in particular, $\wmso$ does not correspond to a fragment of
$\mso$, is witnessed by the class of infinitely branching structures, which
is clearly $\wmso$-definable, cannot be defined in $\mso$, since every
$\mso$-definable class of trees contains a finitely branching
tree.\footnote{As remarked in \cite{CateF11}, this follows from the automata characterisation of MSO given in \cite{Walukiewicz96}.} %\afnote{This follows implicitly from Igor's automata characterization of MSO, cf. props. 2 in my paper with Balder.}
For this reason, the relative expressive power of $\wmso/{\bis}$ and
$\mso/{\bis}$ is not a priori clear.
However, it is reasonable to think that $\wmso/{\bis}$ is strictly \emph{weaker} than \afmc: the class of well-founded trees, which is definable in $\afmc$ by
the simple formula $\mu p. \Box p$, is not definable in \wmso.\footnote{This follows from the fact that $\wmso$ can only define properties of trees that, from a topological point of view, are Borel, which is not the case of the class of trees defined by $\mu p. \Box p$--- see e.g. \cite{CateF11}.}
% \begin{comment}
% What is clear, however, is that $\wmso/{\bis}$ is strictly \emph{weaker}
% \fzwarning{How do we know even that the bis. inv. fragment of WMSO is weaker than $\MC$?}
% than \afmc: the class of well-founded trees, which is definable in \afmc by
% the simple formula $\mu p. \Box p$, is not definable in \wmso.
% (CITATION NEEDED)\fznote{Alessandro, was the (CITATION NEEDED) first stated in your thesis \cite{FacchiniPhD}?}.
% \end{comment}
Incidentally, the question whether, conversely, there is a natural logic of
which the
bisimulation-invariant fragment corresponds to $\afmc$ was answered positively
by three of the present authors in~\cite{DBLP:conf/lics/FacchiniVZ13}, where they introduced
another variant of $\mso$, called well-founded $\mso$ (\nmso), and proved
that $\nmso/{\bis} \equiv \afmc$ (over the class of all LTSs).

The main result that we shall prove in this paper states that the
bisimulation-invariant fragment of $\wmso$ is equivalent to a certain,
fragment $\mucML$ of the modal $\mu$-calculus.
\begin{equation}
\label{eq-main}
\mucML \equiv \wmso/{\bis}  \text{ (over the class of all LTSs)}.
\end{equation}
This fragment $\mucML$, which is strictly weaker than the alternation-free fragment
of $\muML$, is characterised by a certain restriction on the application of
fixpoint operators, which involves the notion of \emph{(Scott) continuity}.

Continuity, an interesting property that features naturally in the semantics
of many (fixpoint) logics, in fact plays a key role throughout this paper.
For its definition, we consider how the meaning $\haak{\phi}^{\model} \sse T$
of a formula $\phi$ in some structure $\model$ (with domain $T$) depends on the meaning of a fixed
proposition letter or monadic predicate symbol $p$.
This dependence can be formalised as a map $\phi^{\model}_{p}: \wp(T) \to
\wp (T)$, and if this map satisfies the condition
\begin{equation}
\label{eq-Sc}
\phi^{\model}_{p}(X) = \bigcup \Big\{ \phi^{\model}_{a}(X') \mid X'
\text{ is a finite subset of } X \Big\},
\end{equation}
we say that $\phi$ is \emph{continuous in $p$}.
The topological terminology stems from the observation that \eqref{eq-Sc}
expresses the continuity of the map $\phi^{\model}_{p}$ with respect to the Scott
topology on $\pw(T)$. % and $\pw(T)$.
If we look at concrete cases, this definition can be given a different reading:
if $\phi$ is a formula of the
modal $\mu$-calculus, \eqref{eq-Sc} means that $\phi$ holds at some state $s$
of $\model$ iff we can shrink the interpretation of the proposition letter $p$
to some finite subset of the original interpretation, in such a way that
$\phi$ holds at $s$ in the modified version of $\model$.

%\btbs
%\item
%mention nice properties?
%\etbs

A syntactic \emph{characterization} of this property for the modal $\mu$-calculus
was obtained by Fontaine~\cite{Fontaine08,FV12}, and the definition of our fragment $\mucML$
uses this characterization as follows:
whereas in the full language of $\muML$ the only syntactic condition on the
formation of a formula $\mu p. \phi$ is that $\phi$ is \emph{positive} in $p$,
for the fragment $\mucML$ this condition is strengthened to the requirement that
$\phi$ is (syntactically) \emph{continuous} in $p$.
More precisely, the fragment $\mucML$ is defined as follows:

\begin{definition}
For each set $\qprops$ of
proposition letters, the fragment $\cont{\MC}{\qprops}$ of $\MC$ which is \emph{continuous in $\qprops$}
is given by the simultaneous induction
\begin{equation*}
   \varphi ::= q
   \mid \psi
   \mid \varphi \lor \varphi
   \mid \varphi \land \varphi
   \mid \Diamond \varphi
   \mid \mu p.\alpha
\end{equation*}
where $p\in\props$, $q \in \qprops$, $\psi$ is a $\qprops$-free $\MC$-formula, and
$\alpha \in \cont{\MC}{\qprops\cup\{p\}}$.
%
The formulas of the fragment $\mucML$ are then given by the following induction:
\begin{equation*}
   \varphi ::= p \mid \lnot \varphi
    \mid \varphi \lor \varphi
    \mid  \Diamond \varphi
    \mid \mu p.\alpha
\end{equation*}
where $p \in \props$, and $\alpha \in \cont{\MC}{p}$.
\end{definition}

In fact we will prove, analogous to the result by Janin \& Walukiewicz,
the following strong version of the characterization result~\eqref{eq-main},
which provides an explicit translation, mapping any
bisimulation-invariant formula $\phi$ in $\wmso$ to an equivalent formula
$\phi^{\bullet}$ in $\mucML$.

%\afnote{Enumerate in the theorem to be more readable?}
\begin{theorem}
\label{t:m1}
There are effective translations $(-)^{\bullet}: \wmso \to \mucML$ and
$(-)_{\bullet}: \mucML \to \wmso$ such that 
\begin{enumerate}
\item A formula $\phi$ of $\wmso$ is
bisimulation invariant if and only if $\phi \equiv \phi^{\bullet}$, and
\item $\psi \equiv \psi_{\bullet}$ for every formula $\psi \in \mucML$.
\end{enumerate}
\end{theorem}

To see how this theorem implies \eqref{eq-main}, observe that part (i)
shows that $\wmso/{\bis} \leq \mucML$.
Part (ii) states that $\mucML \leq \wmso$, so combined with the fact that
every formula in $\mucML \sse \MC$ is bisimulation invariant, this gives the
converse, $\mucML \leq \wmso/{\bis}$.



\subsection{Automata for $\wmso$.}
%
As usual in this research area, our proof will be automata-theoretic in
nature.
More specifically, as the second main contribution of this paper, we
introduce a new class of parity automata that exactly captures the expressive
power of $\wmso$ over the class of tree models of arbitrary branching degree.

Before we turn to a description of these automata, we first have a look at the
automata, introduced by Walukiewicz~\cite{Walukiewicz96}, corresponding to $\mso$
(over tree models).
Fixing the set of proposition letters of our models as $\props$, we think of
$\wp(\props)$ as an \emph{alphabet} or set of \emph{colors}.
We can then define an $\mso$-automaton as a tuple $\aut = \tup{A,\tmap,\pmap,a_I}$, where $A$ is a finite set of states, $a_I$ an initial state, and $\pmap:
A \to \bbN$ is a parity function.
%
The transition function $\tmap$ maps a pair $(a,c) \in A \times \pw(\props)$ to a
sentence in the first-order language (with equality) $\ofoe(A)$, of which the
state space $A$ provides the set of (monadic) predicates.
%
For a more precise definition, let $\ofoe^+(A)$ denote the set of
those sentences in $\ofoe(A)$ where all predicates in $A$ occur only positively;
% the formulas of this language can be given by the following syntax:
% \[
% \varphi \mathrel{::=}
% a(x)
% %\mid \neg a(y)
% \mid x \foeq y
% \mid \neg (x \foeq y)
% %\mid \neg \varphi
% \mid \varphi \lor \varphi
% \mid \varphi \land \varphi
% \mid \exists x.\varphi
% \mid \forall x.\varphi
% \]
% where $a \in A$ and $x,y$ represent individual first-order variables.
we require that $\tmap: A \times \wp(\props) \to \ofoe^+(A)$.

We shall refer to $\ofoe$ as the \emph{one-step language} of $\mso$-automata,
and denote the class of $\mso$-automata with $\Aut(\ofoe)$.
%, because of the key role of $\ofoe(A)$ in
The automata that we consider in this article run on labelled transition systems
and decide wether to accept or reject them. To take such decision we associate
an acceptance game for an
$\mso$-automaton $\bbA$ and a transition system $\model$.
A match of this game consists of two players, $\eloise$ and $\abelard$, moving a
token from one position to another.
When such a match arrives at a so-called \emph{basic} position, i.e., a
position of the form $(a,t) \in A \times T$, the players consider the
sentence $\tmap(a,c_{t}) \in \ofoe^+(A)$, where $c_{t} \in \wp(\props)$ is the color
of $t$ (that is, the set of proposition letters true at $t$).
At this position $\eloise$ has to turn the set $R[t]$ of successors of
$s$ into a \emph{model} for the formula $\tmap(a,c_{t})$ by coming up with an
interpretation $I$ of the monadic predicates $a \in A$ as subsets of $R[s]$,
so that the resulting first-order structure $(R[s],I)$ makes the
formula $\tmap(a,c_{t})$ true.

% \btbs
% \item
% explain role of $\abelard$?
% \etbs

Walukiewicz's key result linking $\mso$ to $\Aut(\ofoe)$ states that
\begin{equation}
\mso \equiv \Aut(\ofoe)
 \text{ (over tree models)},
\end{equation}
and the proof of this result proceeds by inductively showing that every formula
$\phi$ in $\mso$ can be effectively transformed into an equivalent
automaton $\bbA_{\phi} \in \Aut(\ofoe)$.
For the details of this construction, a fairly intricate analysis of the
one-step logic $\ofoe$ is required, crucially involving various normal
forms of the sentences of $\ofoe(A)$.

In order to adapt this approach to the setting of \wmso, observe that by
K\"onig's lemma, a subset of a tree $\model$ is finite iff it is both a subset of
a finitely branching subtree of $\model$ and \emph{noetherian}, that is, a subset
of a subtree of $\model$ that has no infinite branches.
This suggests that we may change the definition of $\mso$-automata into one
of $\wmso$-automata via two kinds of modifications, roughly speaking
corresponding to a horizontal and a vertical `dimension' of trees.

For the `vertical modification' we may turn to the literature on weak automata~\cite{MullerSaoudiSchupp92}.
The acceptance condition $\pmap$ of a parity automaton $\bbA =
\tup{A, \tmap, \pmap, a_I}$ is \emph{weak} if $\pmap(a) = \pmap(a')$ whenever
the states $a$ and $a'$ belong to the same strongly connected component
(SCC) of the automaton. To see that the notion of connected component is well-defined
observe that for $\bbA$ we can associate a directed graph on $A$
such that $a,b \in A$ are connected iff $b$ occurs in $\tmap(a,c)$ for some $c \in \wp(\props)$.
%
Let $\AutW(\ofoe)$ denote the set of $\mso$-automata with a weak parity
condition.
It was proved in \cite{Zanasi:Thesis:2012} (see also \cite{DBLP:conf/lics/FacchiniVZ13}) that
\begin{equation*}
%\label{eq-wfmso}
\nmso \equiv \AutW(\ofoe) \text{ (over the class of all trees)},
\end{equation*}
with $\nmso$ denoting the earlier mentioned variant of $\mso$ %/$\wmso$
where second-order quantification is restricted to noetherian subsets of trees.
From this it easily follows that
\begin{equation*}
%\label{eq-wfmso2}
\wmso \equiv \AutW(\ofoe) \text{ (over the class of finitely
branching trees)},
\end{equation*}
since the noetherian subsets of a finitely branching trees correspond to the
finite ones.
Over the class of all tree models, however, $\wmso$ is \emph{not} equivalent
to $\AutW(\ofoe)$, %$ \equiv \nmso$
as is witnessesed by the earlier mentioned
class of well-founded trees, which can be defined in $\afmc \leq \nmso$,
but not in $\wmso$.

The hurdle to take, in order to find automata for WMSO on trees of
\emph{arbitrary} branching degree, concerns the horizontal dimension; the
main problem lies in finding the right one-step language for
$\wmso$-automata.
An obvious candidate for this language would be weak monadic second-order logic
itself, or more precisely, its variant $\owmso$ over the signature of monadic
predicates (corresponding to the automata states).
A very helpful observation, made by V\"a\"an\"anen~\cite{vaananen77}, states that
\[
\owmso \equiv \ofoei,
\]
where $\ofoei$ is the extension of $\ofoe$ with the generalized quantifier
$\qu$, where $\qu x. \phi$ meaning that there are \emph{infinitely} many
objects satisfying $\phi$.
Taking the \emph{full} language of $\owmso$ or $\ofoei$ as our one-step language
would give too much expressive power: since $\ofoei$ extends $\ofoe$,
we would find that, over tree models, $\AutW(\ofoei)$ extends
$\AutW(\ofoe)$, whereas we already saw that $\AutW(\ofoe) \equiv
\nmso$ is incomparable to $\wmso$.
%
It is here that we will crucially involve the notion of \emph{continuity}.
The automata corresponding to $\wmso$ will be of the form $\bbA = \tup{A, \tmap, \pmap, a_I}$,
where the transition map $\tmap: A \times \wp(\props) \to
{\ofoei}^+(A)$ is subject to the following two constraints, for all $a,a' \in A$
belonging to the same strongly connected component of $A$:
\begin{description}
\itemsep 0 pt
\item[(weakness)] $\pmap(a) = \pmap(a')$, and
\item[(continuity)]
if $\pmap(a)$ is odd (resp. even), then for each colour $c\in \wp(\props)$,
   $\tmap(a,c)$ is continuous (resp. co-continuous) in $a'$,
\end{description}
where co-continuity is a dual notion to continuity.
The class of these automata is denoted by $\AutWC(\ofoei)$.
%
Consequently, for a proper definition of these automata we need a
\emph{syntactic} characterization of the $\ofoei(A)$-sentences that are
(co-)continuous in one (or more) monadic predicates of $A$.

For this purpose, we conduct a fairly detailed model-theoretic study of the
logic $\ofoei$ which we consider to be the third main
contribution of our work.
Similar to the results for $\ofoe$, we provide normal forms for the
sentences of $\ofoei(A)$, and syntactic characterizations of the fragments
whose sentences are monotone (respectively continuous) in some monadic predicate $a \in A$.

% \begin{theorem}
% \label{t:m2}
% A sentence of $\ofoei(A)$ is monotone in ${a\in A}$ iff it is equivalent to
% a sentence in the language given by the following syntax:
% \begin{equation}
% \label{eq-pofoei}
% \varphi ::= \psi \mid a(x) \mid \exists x.\varphi(x) \mid \forall x.\varphi(x)
% \mid \varphi \land \varphi \mid \varphi \lor \varphi
% \mid \qu x.\varphi(x) \mid \dqu x.\varphi(x)
% \end{equation}
% where $\psi \in \ofoei(A\setminus \{a\})$.

% A sentence of $\ofoei(A)$ is continuous in ${a\in A}$ iff it is equivalent to
% a sentence in the language given by the following syntax:
% \begin{equation}
% \label{eq-cofoei}
% \varphi ::=
% \psi \mid a(x) \mid \exists x.\varphi(x) \mid \varphi \land \varphi
%    \mid \varphi \lor \varphi \mid \wqu x.(\varphi,\psi)
% \end{equation}
% where $\psi \in \ofoei(A\setminus \{a\})$ and
% $\wqu x.(\varphi,\psi) :=
%    \forall x.(\varphi(x) \lor \psi(x)) \land \dqu x.\psi(x)$.
% \end{theorem}

% Using these characterizations, we can now provide the definition of a
% \wmso-automaton.

% \begin{definition}
% We denote by $\yvmonot{\ofoei}(A)$ the set of sentences that are positive
% in each $a \in A$ (i.e., belong to \eqref{eq-pofoei} for all $a \in A$), and
% we let $\AutW{\ofoei}$ denote the class of structures of the form
% $\bbA = \struc{A, a_I, \tmap, \pmap}$ such that $A$ is a finite set of states,
% $a_I \in A$ is the initial state of $\bbA$, $\tmap: A \times C \to
% \yvmonot{\ofoei}{a}(A)$ is the transition map, and $\pmap: A\to \bbN$ is the
% parity map of $\bbA$.

% Such a structure is a \emph{\wmso-automaton} if $\tmap$ and $\pmap$ satisfy,
% for all states $a,a'$ belonging
% to the same strongly connected component, the following two
% conditions:
% \begin{itemize}
% \item
% (weakness) $\pmap(a) = \pmap(a')$, and
% \item
% (continuity)
% if $\pmap(a')$ is even, then $\tmap(a',c)$
% belongs to the fragment \eqref{eq-cofoei}, for each $c \in \wp(\props)$
% \\ (and a dual condition applies in case $\pmap(a')$ is odd).
% \end{itemize}
% The class of these automata is denoted by $AutWC(\ofoei)$.
% \end{definition}

To finish, we give constructions transforming \wmso-formulas to
\wmso-automata and vice-versa, witnessing that
%
\begin{equation}
\label{eq:m3}
\wmso \equiv \AutWC(\ofoei) \text{ (over tree models)}.
\end{equation}

\subsection{Proof of main result.}

To conclude our introduction we briefly sketch the proof of our main result, 
Theorem~\ref{t:m1}(1).
Roughly speaking, we follow the bisimulation-invariance proof by Janin \& 
Walukiewicz, which revolves around relating two distinct types of automata, 
which correspond, respectively, to the logics $\mso$ and $\MC$.
More precisely, these two automaton types are given as $\AutWC(\ofoe)$ and
$\AutWC(\ofo)$, where the one-step languages are first-order 
logic respectively with and without equality.
What we will add to their proof is the insight from~\cite{Venxx} that the
required relation between $\AutWC(\ofoe)$ and $\AutWC(\ofo)$ already follows
from results relating the respective one-step languages.

In our setting, we need to identify automata corresponding to the fragment
$\mucML$.
For this purpose we introduce the class $\AutWC(\ofo)$ consisting of those
automata in $\AutWC(\ofo)$ that satisfy similar weakness and continuity 
conditions as the ones in $\AutWC(\ofoei)$:
\begin{equation}
\label{eq:autF}
\mucML \equiv \AutWC(\ofo) \text{ (over the class of all LTSs)}.
\end{equation}

As the key step in our proof then, we will provide a translation 
$(-)^{\bullet}: \ofoei \to \ofo$ which naturally induces a transformation 
$(-)^{\bullet}: \AutWC(\ofoei) \to \AutWC(\ofo)$.
As a consequence of the nice model-theoretic properties of the translation at 
the one-step level, the automaton transformation satisfies, for all 
transition systems $\bbT$:
\begin{equation}
\label{eq:crux-i}
\bbA^{\bullet} \text{ accepts } \bbT \text{ iff } \bbA \text{ accepts 
} \omegaunrav{\bbT}
\end{equation}
where $\omegaunrav{\bbT}$ is the $\omega$-unravelling of $\bbT$.
It easily follows from \eqref{eq:crux-i} that a \wmso-automaton $\bbA$
is bisimulation invariant iff $\bbA\equiv \bbA^{\bullet}$, and so 
Theorem~\ref{t:m1}(1) follows by \eqref{eq:m3} and \eqref{eq:autF}.

\subsection{Overview of paper.}
In the next section we give a precise definition of the preliminaries required to understand this article. In Section~\ref{sec:onestep} we define the one-step logics that will be used through the paper and give normal forms and syntactic characterizations of their monotone and (co-)continuous fragments. In Section~\ref{sec:aut} we formally define \wmso-automata and show that from every \wmso-formula we can construct an equivalent \wmso-automaton. In Section~\ref{sec:aut-to-formula_wmso} we prove the converse, that is, for every \wmso-automaton we can construct an equivalent \wmso-formula, this finishes the automata characterization of \wmso over tree models. Finally, in Section~\ref{sec:char} we prove the main result of the paper, namely that the fragment $\mucML$ is the bisimulation-invariant fragment of \wmso.




\clearpage


%%%%
%%%% PRELIMINARIES
%%%%
\section{Preliminaries}\label{sec:prel}

\subsection{Transition systems and trees} \label{ssec:prelim_trees}
Throughout this article we fix a set $\props$ of elements that will be called
\emph{proposition letters} and denoted with small Latin letters $p, q, \ldots$ .
We denote with $C$ the set $\wp (\props)$ of \emph{labels} on $\props$; it will be
convenient to think of $C$ as an \emph{alphabet}.
Given a binary relation $R \subseteq X \times Y$, for any element $x \in X$,
we indicate with $R[x]$ the set $\compset{ y \in Y \mid (x,y) \in R}$ while $R^+$
and $R^{*}$ are defined respectively as the transitive closure of~$R$ and
the reflexive and transitive closure of~$R$. The set $\Ran(R)$ is defined as $\bigcup_{x\in X}R[x]$.

A \emph{$C$-labeled transition system} (LTS) is a tuple $\model = \tup{T,R,\tscolors,s_I}$ where
$T$ is the universe or domain of $\model$, $\tscolors:T\to\wp(\props)$ is a marking,
$R\subseteq T^2$ is the accessibility relation and $s_I \in T$ is a distinguished node.
We use $|\model|$ to denote the domain of $\model$.
%
% We use $\tscolors^\flat$ to denote the map $\tscolors^\flat:T\to\wp(\props)$ defined as
% $\tscolors^\flat(s) = \{p \in \props \mid s\in \tscolors(p)\}$ and $|\model|$ to denote the domain of $\model$.
Observe that the marking ${\tscolors:T\to\wp(\props)}$ can be seen as a valuation $\tsval:\props\to\wp (T)$ given by $\tsval(p) = \{s \in T \mid p\in \tscolors(s)\}$.

%A pointed LTS is a tuple $(\model,s)$ where $s\in T$ is a distinguished element.
%Pointed LTSs may also be written as $(\model,s) = \tup{T,R,\tscolors,s}$.
%We use $|\model|$ to denote the domain of $\model$.
%
A \emph{$C$-tree} is a LTS in which every node can
be reached from $s_I$, and every node except $s_I$ has a unique predecessor;
the distinguished node $s_I$ is called the \emph{root} of $\model$.
Each node $s \in T$ uniquely defines a subtree of $\model$ with carrier
$R^{*}[s]$ and root $s$. We denote this subtree by ${\model.s}$.
We use the term \emph{tree language} as a synonym of class of $C$-trees.

The tree unravelling of an LTS $\model$ is given by $\unravel{\model} := \tup{T_P,R_P,\tscolors',s_I}$ where $T_P$ is the set of finite paths in $\model$ stemming from $s_I$, $R_P(t,t')$ iff $t'$ is an extension of $t$ and the color of a path $t\in T_P$ is given by the color of its last node in $T$. The $\omega$-unravelling $\omegaunrav{\model}$ of $\model$ is an unravelling which has $\omega$-many copies of each node different from the root.

A \emph{$p$-variant} of a transition system $\model = \tup{T,R,\tscolors,s_I}$
is a $\wp (\props\cup\{p\})$-transition system $\tup{T,R,\tscolors',s_I}$
such that $\tscolors'(s)\setminus\{p\} = \tscolors(s)$ for all $s \in T$.
Given a set $S \subseteq T$, we let $\model[p\mapsto S]$ denote the $p$-variant
%$\tup{T,R,\tscolors',s_I}$ of $\model$
where $p \in \tscolors'(s)$ iff $s \in S$.
%A \emph{path} through $\model$ is a sequence $\pi = (s_i)_{i< \alpha}$ of
%elements of $T$, where $\alpha$ is either $\omega$ or a natural number,
%and $(s_i,s_{i+1}) \in R$ for all $i$ with $i+1 < \alpha$. \fznote{do we need the definition of path?}

Let $\varphi \in \llang$ be a formula of some logic $\llang$,
we use $\ext{\varphi} = \compset{\model \mid \model \models \varphi}$ to denote the class
of transition systems that make $\varphi$ true.
%The subscript $P$ is omitted when the set $P$ of proposition letters is clear from the context.
A class $\mclass$ of transition systems is said to be \emph{$\llang$-definable} if there
is a formula $\varphi \in \llang$ such that $\ext{\varphi} = \mclass$.
We use the notation $\varphi \equiv \psi$ to mean that $\ext{\varphi} = \ext{\psi}$ and given two logics
$\llang, \llang'$ we use $\llang \equiv \llang'$ when the $\llang$-definable and $\llang'$-definable
classes of models coincide.

\textit{Convention.}
Throughout this paper, we will only consider transition systems $\model$
in which $R[s]$ is non-empty for every node $s \in T$.
In particular this means that every tree we consider is \emph{leafless}.
All our results, however, can easily be lifted to the general case.
\subsection{Games}
We introduce some terminology and background on infinite games.
All the games that we consider involve two players called \emph{\'Eloise}
($\exists$) and \emph{Abelard} ($\forall$).
In some contexts we refer to a player $\Pi$ to specify a
a generic player in $\{\exists,\forall\}$.
%
Given a set $A$, by $A^*$ and $A^\omega$ we denote respectively the set of
words (finite sequences) and streams (or infinite words) over $A$.

A \emph{board game} $\mc{G}$ is a tuple $(G_{\exists},G_{\forall},E,\win)$,
where $G_{\exists}$ and $G_{\forall}$ are disjoint sets whose union
$G=G_{\exists}\cup G_{\forall}$ is called the \emph{board} of $\mc{G}$,
$E\subseteq G \times G$ is a binary relation encoding the \emph{admissible
moves}, and $\win \subseteq G^{\omega}$ is a \emph{winning condition}.
An \emph{initialized board game} $\mc{G}@u_I$ is a tuple
$(G_{\exists},G_{\forall},u_I, E,\win)$ where
%$(G_{\exists},G_{\forall},E,\win)$ is a board game and
$u_I \in G$ is the
\emph{initial position} of the game.
When $\win$ is  given by a parity function
$\pmap: G \to \omega$ we say that $\mc{G}$ is a parity game and sometimes
simply write $\mc{G}=(G_{\exists},G_{\forall},E,\pmap)$.

Given a board game $\mc{G}$, a \emph{match} of $\mc{G}$ is simply a path
through the graph $(G,E)$; that is, a sequence $\pi = (u_i)_{i< \alpha}$ of
elements of $G$, where $\alpha$ is either $\omega$ or a natural number,
and $(u_i,u_{i+1}) \in E$ for all $i$ with $i+1 < \alpha$.
A match of $\mc{G}@u_{I}$ is supposed to start at $u_{I}$.
Given a finite match $\pi = (u_i)_{i< k}$ for some $k<\omega$, we call
$\m{last}(\pi) := u_{k-1}$ the \emph{last position} of the match; the
player $\Pi$ such that $\m{last}(\pi) \in G_{\Pi}$ is supposed to move
at this position, and if $E[\m{last}(\pi)] = \emptyset$, we say that
$\Pi$ \emph{got stuck} in $\pi$.
%
A match $\pi$ is called \emph{total} if it is either finite, with one of the
two players getting stuck, or infinite. Matches that are not total are called
\emph{partial}.
Any total match $\pi$ is \emph{won} by one of the players:
If $\pi$ is finite, then it is won by the opponent of the player who gets stuck.
Otherwise, if $\pi$ is infinite, the winner is $\exists$ if $\pi \in
\win$, and $\forall$ if $\pi \not\in \win$.

Given a board game $\mc{G}$ and a player $\Pi$, let $\pmatches{G}{\Pi}$ denote
the set of partial matches of $\mc{G}$ whose last position belongs to player
$\Pi$.
A \emph{strategy for $\Pi$} is a function $f:\pmatches{G}{\Pi}\to G$.
A match $\pi  = (u_i)_{i< \alpha}$ of $\mc{G}$ is
\emph{$f$-guided} if for each $i < \alpha$ such that $u_i \in G_{\Pi}$ we
have that $u_{i+1} = f(u_0,\dots,u_i)$.
%
Let $u \in G$ and a $f$ be a strategy for $\Pi$.
We say that $f$ is a \emph{surviving strategy} for $\Pi$ in $\mc{G}@u$ if
%
\begin{enumerate}
  \item[(i)] For each $f$-guided partial match $\pi$ of $\mc{G}@u$, if $\m{last}(\pi)$
  is in $G_{\Pi}$ then $f(\pi)$ is legitimate, that is,
  $(\m{last}(\pi),f(\pi)) \in E$.
\end{enumerate}
%
We say that $f$ is a \emph{winning strategy} for $\Pi$ in $\mc{G}@u$ if, additionally, %the following condition is met
%We say that $u$ is a \emph{winning position} for $\Pi$ in $\mc{G}$ if, additionally, the following condition is met
%
\begin{enumerate}
  \item[(ii)] $\Pi$ wins each $f$-guided total match of $\mc{G}@u$.
\end{enumerate}
%
If $\Pi$ has a winning winning strategy for $\mc{G}@u$ then $u$ is called a \emph{winning position} for $\Pi$ in $\mc{G}$.
The set of positions of $\mc{G}$ that are winning for $\Pi$ is denoted by $\win_{\Pi}(\mc{G})$.
A strategy $f$ is called \emph{positional} if $f(\pi) = f(\pi^{\prime})$ for each $\pi,\pi^{\prime} \in \Dom(f)$ with $\m{last}(\pi) = \m{last}(\pi^{\prime})$.
A board game $\mc{G}$ with board $G$ is \emph{determined} if $G = \win_{\exists}(\mc{G}) \cup \win_{\forall}(\mc{G})$, that is, each $u \in G$ is a winning position for one of the two players.

The next result states that parity games are positionally determined.
\begin{fact}[\cite{EmersonJ91,Mostowski91Games}]
\label{THM_posDet_ParityGames}
For each parity game $\mc{G}$, there are positional strategies $f_{\exists}$
and $f_{\forall}$ respectively for player $\exists$ and $\forall$, such that
for every position $u \in G$ there is a player $\Pi$ such that $f_{\Pi}$ is a
winning strategy for $\Pi$ in $\mc{G}@u$.
\end{fact}
%
From now on, we always assume that each strategy we work with in parity games
is positional. Moreover, we think of a positional strategy $f_\Pi$ for player $\Pi$
as a function $f_\Pi:G_\Pi\to G$.


\subsection{The Modal $\mu$-Calculus and its fragments.}\label{subsec:mu}

The language of the modal $\mu$-calculus ($\MC$) on $\props$ is given by the following grammar:
%
\begin{equation*}
    \varphi\ ::= q \mid \neg q \mid \varphi \land \varphi \mid
    \varphi \lor \varphi \mid  \Diamond \varphi \mid \Box \varphi \mid
    \mu p.\varphi \mid \nu p.\varphi
\end{equation*}
% \begin{equation*}
%     \varphi\ ::= q \mid \varphi \land \varphi \mid \lnot\varphi
%     \mid  \Diamond \varphi \mid
%     \mu p.\varphi
% \end{equation*}
%
where $p,q \in \props$ and $p$ is positive in $\varphi$ (i.e., $p$ is not negated).
We use the standard convention that no variable is both free and bound in a formula and that every bound variable is fresh.
%
Let $p$ be a bound variable occuring in some formula $\varphi \in \MC$, we use $\delta_p$ to denote the binding definition of $p$, that is, the formula such that either $\mu p.\delta_p$ or $\nu p.\delta_p$ are subformulas of $\varphi$.

The semantics of this language is completely standard. Let $\model = \tup{T,R,\tscolors, s_I}$ be a transition system and $\varphi \in \MC$. We inductively define the \emph{meaning} $\ext{\varphi}^{\model}$ which includes the following clauses for the least $(\mu)$ and greatest ($\nu$) fixpoint operators:
%
\begin{align*}
  \ext{\mu x.\psi}^{\model}  & :=   \bigcap \{S \subseteq T \mid S \supseteq \ext{\psi}^{\model[x\mapsto S]} \}  \\
  \ext{\nu x.\psi}^{\model}  & :=   \bigcup \{S \subseteq T \mid S \subseteq \ext{\psi}^{\model[x\mapsto S]} \}
\end{align*}
%
We say that $\varphi$ is \emph{true} in $\model$ (notation $\model \mmodels \varphi$) iff $s_I \in \ext{\varphi}^{\model}$.% As for the case of $\wmso$, $\ext{\varphi}$ denotes the class of transition systems where $\varphi$ is true.\fcwarning{Pointed or not?}

We will now describe the semantics defined above in game-theoretic terms. That is,
we will define the evaluation game $\egame(\varphi,\model)$ associated with a formula $\varphi \in \MC$ and a transition system $\model$. This game is played by two players (\eloise and \abelard) moving through positions $(\xi,s)$ where $\xi$ is a subformula of $\varphi$ and $s \in T$.
%

\begin{table}
\centering
\begin{tabular}{|l|c|l|c|}
  \hline
  Position & Player & Admissible moves\\
  \hline
  $(\psi_1 \vee \psi_2,s)$ & $\exists$ & $\{(\psi_1,s),(\psi_2,s) \}$ \\
  $(\psi_1 \wedge \psi_2,s)$ & $\forall$ & $\{(\psi_1,s),(\psi_2,s) \}$ \\
  $(\Diamond\varphi,s)$ & $\exists$ & $\{(\varphi,t)\ |\ t \in R[s] \}$ \\
  $(\Box\varphi,s)$ & $\forall$ & $\{(\varphi,t)\ |\ t \in R[s] \}$ \\
  $(\mu p.\varphi,s)$ & $-$ & $\{(\varphi,s) \}$ \\
  $(\nu p.\varphi,s)$ & $-$ & $\{(\varphi,s) \}$ \\
  $(p,s)$ with $p$ bound in $\varphi$ & $-$ & $\{(\delta_p,s) \}$ \\
  $(\lnot q,s) \in \props \times T$, $q$ free in $\varphi$ and $q \notin \tscolors(s)$ & $\forall$ & $\emptyset$\\
  $(\lnot q,s) \in \props \times T$, $q$ free in $\varphi$ and $q \in \tscolors(s)$ & $\exists$ & $\emptyset$\\
  $(q,s) \in \props \times T$, $q$ free in $\varphi$ and $q \in \tscolors(s)$ & $\forall$ & $\emptyset$\\
  $(q,s) \in \props \times T$, $q$ free in $\varphi$ and $q \notin \tscolors(s)$ & $\exists$ & $\emptyset$\\
  \hline
\end{tabular}
\caption{Evaluation game for the modal $\mu$-calculus}
\label{egame_mucalc}
\end{table}
%

In an arbitrary position $(\xi,s)$ it is useful to think of
\eloise trying to show that $\xi$ is true at $s$, and of \abelard of trying to convince her that $\xi$ is false at $s$. The rules of the evaluation game are given in Table~\ref{egame_mucalc}.
Every finite match of this game is lost by the player that got stuck. To give a winning condition for an infinite match let $p$ be, of the bound variables of $\varphi$ that get unravelled infinitely often, the one that is the highest in the syntactic tree of $\varphi$. The winner of the match is \abelard if $p$ is a $\mu$-variable and \eloise if $p$ is a $\nu$-variable. We say that $\varphi$ is true in $\model$ iff \eloise has a winning strategy in $\egame(\varphi,\model)$.

\bigskip
Formulas of the modal $\mu$-calculus are classified according to their
\emph{alternation depth}, which roughly is given as the maximal length of
a chain of nested alternating least and greatest fixpoint operators~\cite{Niwinski86}.
The \emph{alternation-free fragment} of the modal $\mu$-calculus~($\AFMC$) is the collection of
$\MC$-formulas without nesting of least and greatest fixpoint operators.

\begin{definition}
  Let $\varphi$ be a formula of the modal $\mu$-calculus. We say that $\varphi\in\AFMC$ iff for all subformulas $\mu p.\psi_1$ and $\nu q.\psi_2$ we have that $p$ is not free in $\psi_2$ and $q$ is not free in $\psi_1$.
\end{definition}

Over arbitrary transition systems, this fragment is
less expressive than the whole $\MC$~\cite{Park79}. %That is, there is a $\MC$-formula $\varphi$ such that $\ext{\varphi}$ is not $\AFMC$-definable~\cite{Park79}.

In order to properly define the fragment $\mucML \subseteq \AFMC$ which is of critical importance in this article, we are particularly interested in the \emph{continuous} fragment of the modal $\mu$-calculus. As observed in Section~\ref{sec:intro}, the abstract notion of continuity can be given a concrete interpretation in the context of $\mu$-calculus.
%
\begin{definition}
Let $\varphi \in \MC$, and $q$ be a propositional variable. We say that \emph{$\varphi$ is continuous in $q$} iff for every transition system $\model$ there exists some finite $S \subseteq_\omega \tsval(q)$ such that
$$
\model \mmodels \varphi \quad\text{iff}\quad \model[q \mapsto S] \mmodels \varphi .
$$
\end{definition}

We can give a syntactic characterization of the fragment of $\MC$ that captures this property. Given a set $\qprops$ of propositional variables, we define the fragment of \MC \emph{continuous} in $\qprops$, denoted by $\cont{\MC}{\qprops}$, by induction in the following way
\begin{equation*}
   \varphi ::= q
   \mid \psi
   \mid \varphi \lor \varphi
   \mid \varphi \land \varphi
   \mid \Diamond \varphi
   \mid \mu p.\alpha
\end{equation*}
%
where $q \in \qprops$, $p \in \props \setminus \qprops$, $\alpha \in \cont{\MC}{\qprops\cup\{p\}}$, and $\psi$ is a $\qprops$-free $\MC$-formula.

\begin{proposition}[\cite{Fontaine08,FV12}]\label{prop:FVcont}
A $\MC$-formula is continuous in $q$ iff it is equivalent to a formula in the fragment $\cont{\MC}{q}$.
\end{proposition}

Finally, we define $\mucML$ to be the fragment of $\MC$ where the use of the least fixed point operator is restricted to the continuous fragment. Formally, it is defined as follows.

\begin{definition}
Formulas of the fragment $\mucML$ are given by:% the following induction:
\begin{equation*}
   \varphi ::= q \mid \lnot \varphi
    \mid \varphi \lor \varphi
    \mid \Diamond \varphi
    \mid \mu p.\alpha
\end{equation*}
%
where $p,q \in \props$, and $\alpha \in \cont{\MC}{p} \cap \mucML$.
\end{definition}

\begin{proposition}
Let $\varphi \in \mucML$, the following hold
\begin{enumerate}[(1)]
\itemsep 0pt
\item $\varphi$ is an $\AFMC$-formula,
\item Every $\mu$-variable in $\varphi$ is existential (i.e., is only in the scope of diamonds), and dually every $\nu$-variable in $\varphi$ is universal (i.e., is only in the scope of boxes).
\end{enumerate}
\end{proposition}
\begin{proof}
Both points are proved by an easy induction on the complexity of a formula. For the first one,  it is enough to notice that if $\varphi \in \cont{\MC}{q} \cap \AFMC$, then $\mu q. \varphi \in \AFMC$ by definition of $\cont{\MC}{q} $.
\end{proof}

As an immediate consequence of Proposition \ref{prop:FVcont} we have the following:

\begin{corollary}\label{cor:cont}
For every $\mucML$-formula $\mu p. \varphi$, $\varphi$ is continuous in $p$.
\fznote{can we say also that $\nu p. \varphi$ is co-continuous in $p$, on the base of the second part of Prop. 2.14(2)?}
\end{corollary}


%\myparagraph{Finite approximants of monotone maps.}
%\index{approximant}
%\index{$F^\alpha$}
%Let $F:\wp(\moddom) \to \wp(\moddom)$ be a monotone map. The approximants of the least fixpoint of $F$ are the sets ${F^\alpha(\nada) \subseteq \moddom}$, where $\alpha$ is an ordinal. The map $F^\alpha$ is intuitively the $\alpha$-fold composition of $F$. Formally,% it is defined as
%
%\begin{itemize}
%	\itemsep 0pt
%	\item $F^0(X) := \nada$,
%	\item $F^{\alpha+1}(X) := F(F^\alpha(X))$,
%	\item $F^\lambda(X) := \bigcup_{\alpha<\lambda} F^\alpha(X)$ for limit ordinals $\lambda$.
%\end{itemize}
%
%\noindent The sets $F^\alpha(\nada)$ are called approximants because of the following fact.
%
%\begin{fact}
%	For every $s\in\moddom$ we have that $s\in \lfp(F)$ if and only if $s\in F^\beta(\nada)$ for some ordinal $\beta$.
%\end{fact}
%
%Moreover, this approximation starts at $F^0(\nada) = \nada$ and grows strictly until it stabilizes for some ordinal $\beta$. This ordinal is called the closure or unfolding ordinal of $F$.



\subsection{Bisimulation}
Bisimulation is a notion of behavioral equivalence between processes.
For the case of  transition systems, it is formally defined as follows.

\begin{definition}
Let $\model = \tup{T, R, \tscolors, s_I}$ and
$\model' = \tup{T', R', \tscolors', s'_I}$ be transition systems.
A \emph{bisimulation} is a relation $Z \subseteq T \times T'$
such that for all $(t,t^{\prime}) \in Z$ the following holds:
\begin{description}
  \itemsep 0 pt
  \item[(atom)] $p \in \tscolors(t)$ iff $p \in \tscolors^{\prime}(t')$ for all $p\in\prop$;
  \item[(forth)] for all $s \in R[t]$ there is $s^{\prime} \in R^{\prime}[t^{\prime}]$ such that $(s,s^{\prime}) \in Z$;
  \item[(back)] for all $s^{\prime} \in R^{\prime}[t^{\prime}]$ there is $s \in R[t]$ such that $(s,s^{\prime}) \in Z$.
\end{description}
%
Two pointed transition systems $\model$ and $\model^{\prime}$ are
\emph{bisimilar} (denoted $\model \bis \model^{\prime}$) if there is a
bisimulation $Z \subseteq T \times T^{\prime}$ containing $(s_I,s'_I)$.
\end{definition}

The following fact about tree unravellings will allow us to provide a proof of
Theorem~\ref{t:m1} by just proving it for tree languages.

\begin{fact}\label{prop:tree_unrav}
$\model$ and $\model^e$ are bisimilar, for every transition system $\model$.
\end{fact}

A class of transition systems $\mclass$ is \emph{bisimulation closed} if $\model
\bis \model^{\prime}$ implies that $\model \in \mclass$ iff $\model^{\prime}
\in \mclass$, for all $\model$ and $\model^{\prime}$.
A formula $\varphi \in \llang$ is \emph{bisimulation-invariant} if $\model \bis
\model^{\prime}$ implies that $\model \mmodels \varphi$ iff $\model^{\prime}
\mmodels \varphi$, for all $\model$ and $\model^{\prime}$.
%An analogous definition can be given for $\wmso$.

\begin{fact}
Each $\MC$-definable class of transition systems is bisimulation closed.
\end{fact}


\subsection{First-order logic and extensions}


We start by introducing the syntax and semantics of first-order logic and then discuss some extensions that will be used in this dissertation.

\begin{definition}
The language of \emph{first-order logic with equality} ($\foe$) on a set of predicates $\props$, and individual variables $\fovar$ is given by:\[
\varphi ::= q(x) \mid R(x,y) \mid x \foeq y \mid \exists x.\varphi \mid \lnot\varphi \mid \varphi \lor \varphi
\]
where $p,q\in\props$ and $x,y\in\fovar$. We use $\fo$ to denote \emph{first-order logic without equality}, which is defined as $\foe$ but without the $\foeq$ predicate.
\end{definition}


The free variables $\FV(\varphi)$ of a formula $\varphi\in\foe$ are the \emph{individual} variables which are not bound by a quantifier. The inductive definition of $\FV(\varphi)$ is standard.

\begin{remark}
	Every logic in this paper will be function-free. That is, we will \emph{not} consider logics with function symbols in their signature.
\end{remark}


Formulas of $\foe$ will be interpreted over models $\npmodel = \tup{\npmoddom, R, \tscolors}$ with an assignment $\ass: \fovar \to \npmoddom$. Usually, first-order structures (for the signature that we use) are given as tuples $\tup{\npmoddom,R, P_1, \dots}$ where $P_i \subseteq \npmoddom$. However, this same information is encoded in $\npmodel$, since $P_i = \tsval(p_i)$. The semantics of $\foe$ (and also $\fo$) is standard, given as follows:
%

\begin{align*}
\npmodel,\ass \models p_i(x) & \quad\text{ iff }\quad  p_i \in \tscolors(\ass(x)) \\
\npmodel,\ass \models \mid x \foeq y & \quad\text{ iff }\quad \ass(x) = \ass(y) \\
\npmodel,\ass \models R(x,y) & \quad\text{ iff }\quad {R}(\ass(x),\ass(y)) \\
\npmodel,\ass \models \lnot\varphi & \quad\text{ iff }\quad  \npmodel,\ass \not\models \varphi \\
\npmodel,\ass \models \varphi\lor\psi & \quad\text{ iff }\quad  \npmodel,\ass \models \varphi \text{ or } \npmodel,\ass \models \psi \\
\npmodel,\ass \models \exists x.\varphi & \quad\text{ iff }\quad  \text{there is $s \in \npmoddom$ such that $\npmodel, \ass[x\mapsto s] \models \varphi$}.
\end{align*}

\subsubsection{First-order logic with generalized quantifiers}


In this subsection we introduce an extension of first-order logic with so called generalised quantifiers.
% Our interest in this extension stems from the fact that it will allow us to define a variant of first-order logic that is expressively equivalent to weak monadic second order (see Section~\ref{sec:onestep}) and has nice technical features such as a normal form theorem.
%
Mostowski~\cite{Mostowski1957} defined unary generalised quantifiers as follows: a unary generalised quantifier ${\mathcal Q}$ is a collection of pairs $(J, X)$ with $X \subseteq J$, and satisfying the following condition
%
\[
\text{If } \big( (J,X)\in {\mathcal Q}, \ |X|=|Y| \ \land \ | J \setminus X|=|K \setminus Y|\big) \text{ then } (K,Y)\in {\mathcal Q}.
\]

\noindent The semantics of ${\mathcal Q}$ is then defined by the following condition
\[
\npmodel,\ass \models {\mathcal Q}x. \phi(x) \quad\text{iff}\quad (\npmoddom,\{s\in\npmoddom \mid \npmodel,\ass[x\mapsto s] \models \phi(x) \}) \in {\mathcal Q},
\]
%
for every model $\npmodel$ and assignment $\ass$.

In this work we will only focus on the generalised quantifier $\qu$ expressing that there exist infinitely many elements satisfying a certain condition. Formally, it is defined as:
%
\[ \qu := \{(J,X) \mid |X| \geq \aleph_0\}.\]
%
The dual of $\qu$ is $\dqu =\{(J,X) \mid |J\setminus X| < \aleph_0\}$. It is worth observing what is the intended meaning of this quantifier: $\dqu x.\varphi$ expresses that there are \emph{at most finitely many} elements \emph{falsifying} the formula $\varphi$.


\begin{definition}
The extension of first-order logic with equality ($\foe$) 
obtained by adding $\qu$ to the corresponding first-order language is denoted $\foei$.
\end{definition}

\subsubsection{Fixpoint extension of first-order logic}


In this subsection we give an extension of $\foe$ with a unary fixed point operator. This extension is known in the literature as FO(LFP$^1$) but we will call it $\mufoe$ %. In this dissertation we focus on \unfolfp, which we denote as $\mufoe$
to keep a consistent notation for fixpoint extensions (e.g., we use $\mu\llang$ for a base logic $\llang$) and fragments thereof (e.g.~$\mu_X\llang$ where $X$ is some restriction on the fixpoint).
%
% \shtbs{PTIME by Immerman, import more things from the PDLJW paper}
% Arity hierarchy of LFP is strict~\cite{grohefphierarchy}.

%As usual with (extensions of) first-order logic, $\mufoe$ will be interpreted over models with an assignment. See Section~\ref{sec:wcl} ($2\wcl$ \textsl{vs.} $\wcl$) for a discussion on how languages with individual variables fit in our setting.
Because of the presence of individual variables, the syntax and semantics of the fixpoint operator is considerably more involved than for the modal $\mu$-calculus.


\begin{definition}
The language of \emph{first-order logic with equality and unary fixpoints} ($\mufoe$) on a set of predicates $\props$, actions $\acts$ and individual variables $\fovar$ is given by:\[
\varphi ::= q(x) \mid R_\aact(x,y) \mid x \foeq y \mid \exists x.\varphi \mid \lnot\varphi \mid \varphi \lor \varphi \mid [\lfp_{p{:}x}.\varphi(p,x)](z)
\]
where $p,q\in\props$, $\aact\in\acts$ and $x,y\in\fovar$.
%The formula $\varphi(p,x)$ should also satisfy that $p$ occurs only positively and if $p(y)$ is a free occurrence then $x=y$.
Observe that $z$ is free in the fixpoint clause and the fixpoint operator binds the designated variables $x$ and $p$.
\end{definition}

The free variables $\FV(\varphi)$ of a formula $\varphi\in\mufoe$ are obtained by extending the standard definition of $\FV$ for $\foe$ with the clause 
\[\FV([\lfp_{p{:}x}.\varphi(p,x)](z)) := (\FV(\varphi) \setminus \{x\}) \cup \{z\}.\]


The semantics of the fixpoint formula $[\lfp_{p{:}x}.\varphi(p,x)](z)$ is the expected one (as introduced in~\cite{Chandra1982,Moschovakis2008,MoschovakisOrig}). First we give a slightly more general definition than we need right now (which will be useful later). For every model $\npmodel$, assignment $\ass$, and predicates (propositions) $\qprops$, the map $\ffunc{\varphi}{\qprops{:}x}:\wp(\npmoddom)\to \wp(\npmoddom)$ is given as:
\[
% \ffunc{\varphi}{p{:}x}(Y) := \{t \in \npmoddom \mid \npmodel[p \mapsto Y],\ass[x\mapsto t] \models \varphi(p, x) \}.
\ffunc{\varphi}{\qprops{:}x}(\vlist{Y}) := \{t \in \npmoddom \mid \npmodel[\qprops \mapsto \vlist{Y}],\ass[x\mapsto t] \models \varphi(\qprops, x) \}.
\]
%
The formula $\npmodel,\ass \models [\lfp_{p{:}x}.\varphi(p,x)](z)$ is then defined to hold iff $\ass(z) \in \lfp(\ffunc{\varphi}{p{:}x})$. That is, if $\ass(z)$ is in the least fixpoint of the map $\ffunc{\varphi}{p{:}x}$.

\begin{remark}\label{rem:parameters}
	
	Suppose that $\varphi \in \mufoe$ has free variables $FV(\varphi) = \{x,\vlist{y}\}$. If we consider the fixpoint formula $\psi := [\lfp_{p{:}x}.\varphi(p,x)](z)$ then $\psi$ would have as free variables $FV(\psi) = \{z,\vlist{y}\}$. The free variables of $\varphi$ which are not bound by the fixpoint (in this case~$\vlist{y}$) are called the \emph{parameters} of the fixpoint.

	Parameters can always be avoided at the expense of increasing the arity of the fixpoint~\cite[p.~184]{Libkin2004}. That is, for example, taking the fixpoint over a relation $P(x_1,\dots,x_n)$ instead of just a predicate $p$. However, in this dissertation we will only consider fixpoints over unary predicates, and therefore we will allow the use of parameters unless explicity stated.
\end{remark}

The language of $\mufoe$ can also be further extended with a \emph{greatest} fixpoint operator $[\gfp_{p{:}x}.\varphi(p,x)](z)$ whose semantics are given by $\npmodel,\ass \models [\gfp_{p{:}x}.\varphi(p,x)](z)$ iff $\ass(z) \in \gfp(\ffunc{\varphi}{p{:}x})$. However, this extension does not add expressive power, since it is possible to prove that $[\gfp_{p{:}x}.\varphi(p,x)](z) \equiv \lnot[\lfp_{p{:}x}.\lnot\varphi(\lnot p,x)](z)$.


\subsection{Monadic second-order logics}\label{sec:prel-so}
Given a set of propositional letters, or predicates, $\prop$, we define three variants of monadic second-order logic on it:
\emph{(standard) monadic second-order logic} ($\mso(\prop)$),
\emph{weak monadic second-order logic} ($\wmso(\prop)$) and
\emph{noetherian monadic second-order logic} ($\nmso(\prop)$).
We omit  $\prop$ when the set of proposition letters is clear from context. 
These logics share the same syntax.
\begin{definition}\label{def:mso}
The formulas of the \emph{monadic second-order
language} on a set of predicates $\prop$ are defined by the following grammar:
%
\begin{eqnarray*}\label{EQ_mso}
  \varphi ::= \here{p} \mid p \inc q \mid R(p,q) \mid \lnot\varphi \mid \varphi\lor\varphi \mid \exists p.\varphi,
\end{eqnarray*}
where $p$ and $q$ are letters from $\prop$.
We  adopt the standard convention that no letter is both free and bound in
$\varphi$.
\end{definition}

The three logics are distinguished by their semantics.
Let  $\model = \tup{T,R,\tscolors, s_I}$ be a LTS, the interpretation of the
atomic formulas and the boolean connectives is fixed and standard, e.g.:
\begin{align*}
\model \models \here{p} & \quad\text{ iff }\quad  \tsval(p) = \compset{s_I} \\
\model \models p \inc q & \quad\text{ iff }\quad  \tsval(p) \subseteq \tsval(q) \\
\model \models R(p,q) & \quad\text{ iff }\quad  \text{for every $s\in \tsval(p)$ there exists $t\in \tsval(q)$ such that $sRt$} 
\end{align*}

The interpretation of the existential quantifier is that

\begin{align*}
\model \models\ \exists p. \varphi  & \quad\text{ iff }\quad  \model[p \mapsto X] \models \phi \,
\left.\begin{cases}
 \text{for some }   & {\bf (\mso)} \\
  \text{for some \emph{finite} }   & {\bf (\wmso)} \\
    \text{for some \emph{noetherian} }   & {\bf (\nmso)} 
 \end{cases}\right\}\,
 X \subseteq T.
\end{align*}
%if and only if
%\begin{description}
%%[\IEEEsetlabelwidth{$\alpha\omega \pi\theta\mu\varphi$}\IEEEusemathlabelsep]
%\item[$(\Wmso)$] $\model[p \mapsto X] \models \phi$ for some finite $S \subseteq_\omega T$
%\item[$(\Nmso)$] $\model[p \mapsto X] \models \phi$ for some noetherian
%    $X \subseteq T$.
%\end{description}

Let $\varphi\in \mso$ be a formula. We denote with $\|\varphi \|_P$ the set
of $C$-transition structures $\model$ such that $\model\models \varphi$.
The subscript $P$ is omitted when the set $P$ of proposition letters is clear
from the context.
A class $\mc{L}$ of transition systems is $\mso$\emph{-definable} if there
is a formula $\varphi \in \mso$ such that $\| \varphi \| = \mc{L}$.
We define the analogous notions for $\wmso$ and $\nmso$ in the same way.



\begin{remark}
The reader may have expected a more standard two-sorted language for second-order logic, for example given by
%
$$
\varphi ::= p(x)
%\mid X(y)
\mid R(x,y)
\mid x \foeq y
\mid \neg \varphi
\mid \varphi \lor \varphi
\mid \exists x.\varphi
\mid \exists p.\varphi
$$%
where $p \in \prop$, $x,y \in \fovar$ (individual variables), %$X \in \sovar$ (second-order variables)
and $\foeq$ is the symbol for equality.
Both definitions can be proved to be equivalent, however, we choose to keep Definition~\ref{def:mso} as it will be better suited to work with in the context of automata.
\end{remark}


\clearpage





%%%%
%%%% ONE STEP LOGICS
%%%%
\section{One-step logics, normal forms and continuity}\label{sec:onestep}

\begin{definition}
Given a finite set $A$
we define a \emph{one-step model} to be a tuple $\osmodel = (D,\val)$
consisting of a set $D$,  which we call the domain of $\osmodel$, 
and a valuation $\val: A \to \wp D$.

Depending on context, elements of $A$ will be called \emph{monadic predicates}, \emph{names}
or \emph{propositional variables}. The class of all one-step models will be denoted by $\umods$.


A \emph{one-step language} is a map $\llang$ assigning to each finite set $A$, a set $\llang(A)$ of objects called \emph{one-step formulas} over $A$.
We require that $\llang(\bigcap_{i} A_{i}) = \bigcap_{i} \llang(A_{i})$,
so that for each $\varphi \in \llang(A)$ there is a smallest $A_{\varphi} \subseteq A$ such
that $\varphi \in \llang(A_{\varphi})$; this $A_{\varphi}$ is the set of names that \emph{occur} in $\varphi$.

We assume that one-step languages come with a \emph{truth}
relation: given a one-step model $\osmodel$, a formula $\varphi \in \llang$
is either \emph{true} or \emph{false} in $\osmodel$, denoted by,
respectively, $\osmodel \models \varphi$ and $\osmodel \not\models \varphi$.
%

We also assume that $\llang$ has a \emph{positive fragment} $\llang^+$
characterising monotonicity. We say that a formula $\varphi \in \llang(A)$ is
monotone in $a\in A$ iff $(\osmoddom,\val) \models \varphi$ implies $(\osmoddom,\val[a\mapsto E]) \models \varphi$ whenever $\val(a) \subseteq E$. Hence, we require that $\varphi \in \llang(A)$ is
monotone in all $a\in A$ iff it is equivalent to a formula $\varphi' \in \llang^+(A)$.
\end{definition}


Observe that every valuation $\val: A \to \wp (D)$ can equivalently be seen as a marking (or coloring) $\val^\natural:D \to \wp(A)$ given by $\val^\natural(d) := \{a \in A \mid d \in \val(a)\}$ and as a relation $Z_\val := \{ (a,d) \mid d\in \val(a)\}$.
We will use these perspectives interchangeably.


\begin{definition}
The set $\ofoe(A)$ of one-step first-order sentences (with equality) is given by the sentences formed by
\[
\varphi ::=
\top \mid \bot 
\mid a(x)
%\mid \neg a(x)
\mid x \foeq y
%\mid \neg x \foeq y
\mid \neg \varphi
\mid \varphi \lor \varphi
%\mid \varphi \land \varphi
\mid \exists x.\varphi
%\mid \forall x.\varphi
\]
where $x,y\in \fovar$, $a \in A$. The one-step logic $\ofo(A)$ is as $\ofoe(A)$ but without equality. 
The set $\ofoei(A)$ of one-step first-order sentences with generalized quantifier $\qu$ (with equality)
%and the set $\olqu(A)$ of one-step first-order senteces with generalized quantifier $\qu$ (without equality)
is defined analogously by just adding the clauses for the generalised quantifiers $\qu x. \varphi$ and $\dqu x. \varphi$.
\end{definition}

\begin{remark}
	The elements $\top$ and $\bot$ are added for technical reasons. Even though they are already definable in $\ofoe(A)$, this will not necessarily be the case in other fragments that will be defined later.
\end{remark}


\begin{definition}
	Let $\varphi \in \ofoei(A)$ be a formula, $\osmodel = (\osmoddom,\val)$ be a one-step model and $\ass:\fovar\to \wp(D)$ be an assignment. The semantics of $\ofoei(A)$ is given as follows:
	%
	\begin{align*}
	    \osmodel,\ass \models a(x) & \quad\text{iff}\quad \ass(x) \in \val(a),\\
	    \osmodel,\ass \models x \foeq y & \quad\text{iff}\quad \ass(x) = \ass(y),\\
	    %
	    \osmodel,\ass \models \exists x.\varphi & \quad\text{iff}\quad \osmodel,\ass[x\mapsto d] \models \varphi \text{ for some $d\in D$},\\
	    %
	    \osmodel,\ass \models \qu x.\varphi & \quad\text{iff}\quad \osmodel,\ass[x\mapsto d] \models \varphi \text{ for infinitely many distinct $d\in D$},
	\end{align*}
	%
	while the Boolean connectives are defined as expected.
\end{definition}

Recall that $\dqu x.\varphi$ expresses that there are \emph{at most finitely many} elements \emph{falsifying} the formula $\varphi$.

\index{rank, quantifier}
% \index{$qr$}
\begin{definition}
The quantifier rank $qr(\varphi)$ of a formula $\varphi \in \ofoei$ (hence also for $\ofo$ and $\ofoe$) is defined as follows
%
\begin{itemize}
	\itemsep 0 pt
	\item If $\varphi$ is atomic then $qr(\varphi) = 0$,
	\item If $\varphi = \lnot\psi$ then $qr(\varphi) = qr(\psi)$,
	\item If $\varphi = \psi_1 \land \psi_2$ or $\varphi = \psi_1 \lor \psi_2$ then $qr(\varphi) = \max\{qr(\psi_1),qr(\psi_2)\}$,
	\item If $\varphi = Qx.\psi$ for $Q \in \{\exists,\forall,\qu,\dqu\}$ then $qr(\varphi) = 1+qr(\psi)$.
\end{itemize}
%
% \index{$\equiv_k^{\llang}$}
Given a one-step logic $\llang$ we write $\osmodel \equiv_k^{\llang} \osmodel'$ to indicate that the one-step models $\osmodel$ and $\osmodel'$ satisfy exactly the same formulas $\varphi \in \llang$ with $qr(\varphi) \leq k$. The logic $\llang$ will be omitted when it is clear from context.
\end{definition}

% The following observation will allow us to
% work with the (single-sorted) language $\ofoe$ instead of the (two-sorted) language $OWMSO$.\fcwarning{owmso?}

% \begin{fact}[\cite{vaananen77}]
% $OWMSO(A) \equiv \ofoei(A)$.
% \end{fact}

% In the following subsections we provide a detailed model theoretic analysis of the one-step logics that we use in this article, specifically, we give
% %
% \begin{itemize}
% 	\itemsep 0 pt
% 	\item Normal forms for arbitrary formulas of $\ofo$, $\ofoe$ and $\ofoei$.
% 	%
% 	\item Strong forms of syntactic characterizations for the monotone and continuous fragments of several of the mentioned logics. Namely, for $\llang \in \{\ofo,\ofoe,\ofoei\}$ we provide
% 		\begin{enumerate}[(a)]
% 			%\itemsep 0 pt
% 			\item A fragment $\monot{\llang}{a}$ and a translation $(-)^\tmono:\llang(A)\to\monot{\llang}{a}(A)$ such that for every $\varphi \in\llang$ we have $\varphi\equiv\varphi^\tmono$ iff $\varphi$ is monotone in $a \in A$,
% 			%
% 		\end{enumerate}
% 		%
% 		for $\llang \in \{\ofo,\ofoei\}$ we provide
% 		%
% 		\begin{enumerate}[(a)]
% 			\item[(b)] A fragment $\cont{\llang}{a}$ and a translation $(-)^\tcont:\llang(A)\to\cont{\llang}{a}(A)$ such that for every $\varphi \in\llang$ we have $\varphi\equiv\varphi^\tcont$ iff $\varphi$ is continuous in $a \in A$.\fcwarning{The real cont-translation is from $\monot{\llang}{a}(A)$}
% 		\end{enumerate}
% 		%
% 		Moreover, we show that the latter translation also restricts to the fragment $\llang^+_1$, i.e.,
% 		%
% 		\begin{enumerate}[(a)]
% 			\item[(c)] The restriction $(-)^\tcont_{+}:\llang^+_1(A)\to\cont{\llang^+_1}{a}(A)$ of $(-)^\tcont$ is such that for every $\varphi \in\llang^+_1$ we have $\varphi\equiv\varphi^\tcont_+$ iff $\varphi$ is continuous in $a \in A$.
% 		\end{enumerate}
% 	%
% 	\item Syntactic characterizations of the co-continuous fragments of $\ofo$ and $\ofoei$.
% 	%
% 	\item Normal forms for the monotone and continuous fragments.
% \end{itemize}

\index{isomorphism, partial}
% \index{$f: [d_1,\dots,d_k] \mapsto [d'_1,\dots,d'_k]$}
\index{$f:\vlist{d} \mapsto \vlist{d'}$}
A \emph{partial isomorphism} between two one-step models %$\osmodel = (D,\val)$ and $\osmodel' = (D',\val')$
$(D,\val)$ and $(D',\val')$ is a \emph{partial} function $f: D \to D'$ which is injective and satisfies that $d \in \val(a) \Leftrightarrow f(d) \in \val'(a)$ for all $a\in A$ and $ d\in \Dom(f)$.

Given two sequences $\vlist{d} \in D^k$ and $\vlist{d'} \in {D'}^k$ 
we use $f:\vlist{d} \mapsto \vlist{d'}$ to denote the partial function $f:D\pto D'$ defined as $f(d_i) := d'_i$. We explicitly avoid cases where there exist $d_i,d_j$ such that $d_i = d_j$ but $d'_i \neq d'_j$.

% Given two sequences $[d_1,\dots,d_k] \in D^k$ and $[d'_1,\dots,d'_k] \in {D'}^k$ we use $f: [d_1,\dots,d_k] \mapsto [d'_1,\dots,d'_k]$ to denote the partial function $f:D\to D'$ defined as $f(d_i) := d'_i$. We explicitly avoid situations where there exist $d_i,d_j$ such that $d_i = d_j$ but $d'_i \neq d'_j$.

\subsection{Normal forms}\label{subsec:normalforms}


In this section we provide normal forms for the single-sorted one-step logics $\ofo$, $\ofoe$ and $\ofoei$. These normal forms will be pivotal to characterize the different fragments of these logics, in later sections.

\subsubsection{Normal form for $\ofo$}

We start by stating a normal form for one-step first-order logic without equality. A formula in \emph{basic form} gives a complete description of the types that are satisfied in a one-step model.

\index{form, basic!$\ofo$}
\index{$\dbnfofo{\Sigma}$}
\begin{definition}\label{def:bfofo}%[Basic form for \ofo]
A formula $\varphi \in \ofo(A)$ is in \emph{basic form} if $\varphi = \bigvee \dbnfofo{\Sigma}$
where each disjunct is of the form
\[
%\dbnfofo{\Sigma}{\Pi} = \bigwedge_{S\in\Sigma} \exists x. \tau_S(x) \land \forall x. \bigvee_{S\in\Pi} \tau_S(x)
\dbnfofo{\Sigma} = \bigwedge_{S\in\Sigma} \exists x. \tau_S(x) \land \forall x. \bigvee_{S\in\Sigma} \tau_S(x)
\]
%for some sets of types $\Sigma \subseteq \Pi \subseteq \wp A$ (observe the first $\subseteq$!).
for some set of types $\Sigma\subseteq \wp A$.
\end{definition}

It is easy to prove, using Ehrenfeucht-Fra\"iss\'e games, that every formula of first-order logic without equality over a unary signature (i.e., $\ofo$) is equivalent to a formula in basic form. Proof sketches can be found in~\cite[Lemma 16.23]{Graedel2002} and~\cite[Proposition 4.14]{Venema2014}. We omit a full proof because it is very similar to the following more complex cases.

\begin{fact}\label{fact:ofonormalform}
Every formula of $\ofo(A)$ is equivalent to a formula in basic form.
\end{fact}

%%%%
\subsubsection{Normal form for $\ofoe$}

When considering a normal form for $\ofoe$, the fact that we can `count types' using equality yields a more involved basic form.

\index{form, basic!$\ofoe$}
\index{$\dbnfofoe{\vlist{T}}{\Pi}$}
\begin{definition}%[Basic form for \ofoe]
We say that a formula $\varphi \in \ofoe(A)$ is in \emph{basic form} if $\varphi = \bigvee \dbnfofoe{\vlist{T}}{\Pi}$ where each disjunct is of the form
%
\begin{equation*}%\label{eq:normalformofoe}
\dbnfofoe{\vlist{T}}{\Pi} = \exists \vlist{x}.\big(\arediff{\vlist{x}} \land \bigwedge_i \tau_{T_i}(x_i) \land \forall z.(\arediff{\vlist{x},z} \lthen \bigvee_{S\in \Pi} \tau_S(z))\big)
\end{equation*}
%
such that $\vlist{T} \in \wp(A)^k$ for some $k$ and $\Pi \subseteq \vlist{T}$.  The predicate $\arediff{\vlist{y}}$, stating that the elements $\vlist{y}$ are distinct, is defined as $\arediff{y_1,\dots,y_n} := \bigwedge_{1\leq m < m^{\prime} \leq n} (y_m \not\approx y_{m^{\prime}})$.
\end{definition}

We prove that every formula of monadic first-order logic with equality (i.e., $\ofoe$) is equivalent to a formula in basic form. This result seems to be folklore, however, we provide a detailed proof because some of its ingredients will be used later, when we give a normal form for $\ofoei$. We start by defining the following relation between one-step models.

% \index{$\sim^=_k$}
\begin{definition}
	Let $\osmodel$ and $\osmodel'$ be one-step models. For every $k \in \nat$, the relation $\osmodel \sim^=_k \osmodel'$ is defined as
\begin{eqnarray*}
	\osmodel \sim^=_k \osmodel' & \Longleftrightarrow & \forall S\subseteq A \ \big(
	   |S|_\osmodel = |S|_{\osmodel'} < k \\
	&& \qquad\qquad \text{or } |S|_\osmodel,|S|_{\osmodel'} \geq k \big)
\end{eqnarray*}
\end{definition}

Intuitively, two models are related by $\sim^=_k$ when their type information coincides `modulo~$k$'. Later we will prove that this is the same as saying that they cannot be distinguished by a formula of $\ofoe$ with quantifier rank lower or equal to $k$. For the moment, we prove the following properties of $\sim^=_k$.

\begin{proposition}\label{props:eqrelofoe} The following hold:
	\begin{enumerate}[(i)]
		\itemsep 0 pt
		\item $\sim^=_k$ is an equivalence relation,
		\item $\sim^=_k$ has finite index,
		\item Every $E \in \umods/{\sim^=_k}$ is characterized by a formula $\varphi^=_E \in \ofoe(A)$ with $qr(\varphi^=_E) = k$.
	\end{enumerate}
\end{proposition}
\begin{proof}
	We only prove the last point. Let $E \in \umods/{\sim^=_k}$ and let $\osmodel \in E$ be a representative. Call $S_1,\dots,S_n \subseteq A$ to the types such that $|S_i|_\osmodel = n_i < k$ and $S'_1,\dots,S'_m \subseteq A$ to those satisfying $|S'_i|_\osmodel \geq k$. Now define
	%
	\begin{align*}
	\varphi^=_E := & \bigwedge_{i\leq n} \big(\exists x_1,\dots,x_{n_i}.\arediff{x_1,\dots,x_{n_i}} \ \land \\
		& \qquad\bigwedge_{j\leq n_i} \tau_{S_i}(x_j) \land \forall z. \arediff{x_1,\dots,x_{n_i},z} \lthen \lnot\tau_{S_i}(z)\big)\ \land \\
	    & \bigwedge_{i\leq m} \big(\exists x_1,\dots,x_k.\arediff{x_1,\dots,x_k} \land \bigwedge_{j\leq k} \tau_{S'_i}(x_j) \big)
	            %& \bigwedge_{i\leq m} \big(\forall x_1,\dots,x_{k-1}.\exists x_k.\arediff{x_1,\dots,x_{k}} \land \bigwedge_{j\leq k} \tau_{S'_i}(x_j) \big)
	\end{align*}
	%
	First note that the union of all the $S_i$ and $S'_i$ yields all the possible $A$-types, and that if a type is not realized at all, then it will correspond to some $S_j$ with $n_j = 0$. It is easy to see that $qr(\varphi^=_E) = k$ and that $\osmodel' \models \varphi^=_E$ iff $\osmodel' \in E$. Observe that $\varphi^=_E$ gives a specification of $E$ ``type by type''.
\end{proof}

Next we recall a (standard) notion of Ehrenfeucht-Fra\"iss\'e game for $\ofoe$ which will be used to establish the connection between ${\sim^=_k}$ and $\equiv_k^\foe$.

% \index{$\efgame^=_k(\osmodel_0,\osmodel_1)$}
\begin{definition}
	Let $\osmodel_0 = (D_0,\val_0)$ and $\osmodel_1 = (D_1,\val_1)$ be one-step models. We define the game $\efgame^=_k(\osmodel_0,\osmodel_1)$ between \abelard and \eloise. If $\osmodel_i$ is one of the models we use $\osmodel_{-i}$ to denote the other model. A position in this game is a pair of sequences $\vlist{s_0} \in D_0^n$ and $\vlist{s_1} \in D_1^n$ with $n \leq k$. The game consists of $k$ rounds where in round $n+1$ the following steps are made
	%
	\begin{enumerate}[1.]
		\itemsep 0 pt
		\parsep 0 pt
		\item \abelard chooses an element $d_i$ in one of the $\osmodel_i$,
		\item \eloise responds with an element $d_{-i}$ in the model $\osmodel_{-i}$.
		\item Let $\vlist{s_i} \in D_i^n$ be the sequences of elements chosen up to round $n$, they are extended to ${\vlist{s_i}' := \vlist{s_i}\cdot d_i}$. Player \eloise survives the round iff she does not get stuck and the function $f_{n+1}: \vlist{s_0}' \mapsto \vlist{s_1}'$ is a partial isomorphism of one-step models.
	\end{enumerate}
	%
	Player \eloise wins iff she can survive all $k$ rounds.
	%
	%At the end of the game we have a sequence of $k$ pairs $(d_0^i,d_1^i)$. The match is won by \eloise iff she never gets stuck and the function $f: d_0^i \mapsto d_1^i$ is a partial isomorphism of one-step models, that is, $f$ is bijective and $d \in \val_0(a) \Leftrightarrow f(d) \in \val_1(a)$ for all $a\in A$.
	%
	Given $n\leq k$ and $\vlist{s_i} \in D_i^n$ such that $f_n:\vlist{s_0}\mapsto\vlist{s_1}$ is a partial isomorphism, we use $\efgame_{k}^=(\osmodel_0,\osmodel_1)@(\vlist{s_0},\vlist{s_1})$ to denote the (initialized) game where $n$ moves have been played and $k-n$ moves are left to be played.
\end{definition}

\begin{lemma}\label{lem:connofoe}
	The following are equivalent
	\begin{enumerate}
		\itemsep 0 pt
		\item\label{lem:connofoe:i} $\osmodel_0 \equiv_k^\foe \osmodel_1$,
		\item\label{lem:connofoe:ii} $\osmodel_0 \sim_k^= \osmodel_1$,
		\item\label{lem:connofoe:iii} \eloise has a winning strategy in $\efgame_k^=(\osmodel_0,\osmodel_1)$.
	\end{enumerate}
\end{lemma}
\begin{proof}
	Step~(\ref{lem:connofoe:i}) to~(\ref{lem:connofoe:ii}) is direct by Proposition~\ref{props:eqrelofoe}. For~(\ref{lem:connofoe:ii}) to~(\ref{lem:connofoe:iii}) we give a winning strategy for \eloise in $\efgame_k^=(\osmodel_0,\osmodel_1)$. We do it by showing the following claim
	%
	\begin{claimfirst}
	Let $\osmodel_0 \sim_k^= \osmodel_1$ and $\vlist{s_i} \in D_i^n$ be such that $n<k$ and $f_n:\vlist{s_0}\mapsto\vlist{s_1}$ is a partial isomorphism; then \eloise can survive one more round in $\efgame_{k}^=(\osmodel_0,\osmodel_1)@(\vlist{s_0},\vlist{s_1})$.
	\end{claimfirst}
	%
	\begin{pfclaim}
		Let \abelard pick $d_i\in D_i$ such that the type of $d_i$ is $T \subseteq A$. If $d_i$ had already been played then \eloise picks the same element as before and $f_{n+1} = f_n$. If $d_i$ is new and $|T|_{\osmodel_i} \geq k$ then, as at most $n<k$ elements have been played, there is always some new $d_{-i} \in D_{-i}$ that \eloise can choose that matches $d_i$. If $|T|_{\osmodel_i} = m < k$ then we know that $|T|_{\osmodel_{-i}} = m$. Therefore, as $d_i$ is new and $f_n$ is injective, there must be a $d_{-i} \in D_{-i}$ that \eloise can choose. %This shows how to extend the partial isomorphism given by the $(k-1)$-round game to a partial isomorphism for the $k$-round game.
	\end{pfclaim}
	
	Step~(\ref{lem:connofoe:iii}) to~(\ref{lem:connofoe:i}) is a standard result~\cite[Corollary 2.2.9]{fmt} which we prove anyway because we will need to extend it later. We prove the following loaded statement.
	\begin{claim}
		Let $\vlist{s_i} \in D_i^n$ and $\varphi(z_1,\dots,z_n) \in \ofoe(A)$ be such that $qr(\varphi) \leq k-n$. If \eloise has a winning strategy in the game $\efgame_k^=(\osmodel_0,\osmodel_1)@(\vlist{s_0},\vlist{s_1})$ then $\osmodel_0 \models \varphi(\vlist{s_0})$ iff $\osmodel_1 \models \varphi(\vlist{s_1})$.
	\end{claim}
	%
	\begin{pfclaim}
		If $\varphi$ is atomic the claim holds because of $f_n:\vlist{s_0}\mapsto \vlist{s_1}$ being a partial isomorphism. Boolean cases are straightforward.
		%
		Let $\varphi(z_1,\dots,z_n) = \exists x. \psi(z_1,\dots,z_n,x)$ and suppose $\osmodel_0 \models \varphi(\vlist{s_0})$. Hence, there exists $d_0 \in D_0$ such that $\osmodel_0 \models \psi(\vlist{s_0},d_0)$.
		%
		By hypothesis we know that \eloise has a winning strategy for $\efgame_k^=(\osmodel_0,\osmodel_1)@(\vlist{s_0},\vlist{s_1})$. Therefore, if \abelard picks $d_0\in D_0$ she can respond with some $d_1\in D_1$ and has a winning strategy for $\efgame_{k}^=(\osmodel_0,\osmodel_1)@(\vlist{s_0}{\cdot}d_0,\vlist{s_1}{\cdot}d_1)$.
		%
		By induction hypothesis, because $qr(\psi) \leq k- (n+1)$, we have that $\osmodel_0 \models \psi(\vlist{s_0},d_0)$ iff $\osmodel_1 \models \psi(\vlist{s_1},d_1)$ and hence $\osmodel_1 \models \exists x.\psi(\vlist{s_1},x)$. The opposite direction is proved by a symmetric argument. %and the case where $\varphi$ is $\forall x. \psi(z_1,\dots,z_n,x)$ are very similar.
		\end{pfclaim}
		%
		Combining these claims finishes the proof of the lemma.
\end{proof}

\begin{theorem}\label{thm:bnfofoe}
Every $\psi \in \ofoe(A)$ is equivalent to a formula in basic form.
\end{theorem}
\begin{proof}
	Let $qr(\psi) = k$ and let $\ext{\psi}$ be the class of models satisfying $\psi$. As $\umods/{\equiv_k^\foe}$ is the same as $\umods/{\sim_k^=}$ by Lemma~\ref{lem:connofoe}, it is easy to see that $\psi$ is equivalent to $\bigvee \{ \varphi^=_E \mid E \in \ext{\psi}/{\sim_k^=} \}$. Now it only remains to see that each $\varphi^=_E$ is equivalent to $\dbnfofoe{\vlist{T}}{\Pi}$ for some $\Pi \subseteq \wp A$ and $T_i \subseteq A$.

	The crucial observation is that we will use $\vlist{T}$ and $\Pi$ to give a specification of the types ``element by element''. Let $\osmodel \in E$ be a representative. Call $S_1,\dots,S_n \subseteq A$ to the types such that $|S_i|_\osmodel = n_i < k$ and $S'_1,\dots,S'_m \subseteq A$ to those satisfying $|S'_i|_\osmodel \geq k$. The size of the sequence $\vlist{T}$ is defined to be $(\sum_{i=1}^n n_i) + k\times m$ where $\vlist{T}$ is contains exactly $n_i$ occurrences of type $S_i$ and $k$ occurrences of each $S'_j$. On the other hand $\Pi = \{S'_1,\dots,S'_m\}$. It is straightforward to check that $\varphi^=_E$ is equivalent to $\dbnfofoe{\vlist{T}}{\Pi}$, however, the quantifier rank of the latter is only bounded by $k\times 2^{|A|} + 1$.
\end{proof}

%%%%
\subsubsection{Normal form for $\ofoei$}

The logic $\ofoei$ extends $\ofoe$ with the capacity to tear apart finite and infinite sets of elements. This is reflected in the normal form for $\ofoei$ by adding extra constraints to the normal form of $\ofoe$.

\index{form, basic!$\ofoei$}
\index{$\dbnfofoei{\vlist{T}}{\Pi}{\Sigma}$}
\index{$\dbnfinf{\Sigma}$}
\begin{definition}\label{def:basicform_fofoei}
We say that a formula $\varphi \in \ofoei(A)$ is in \emph{basic form} if $\varphi = \bigvee \dbnfofoei{\vlist{T}}{\Pi}{\Sigma}$ where each disjunct is of the form
%
\[
\dbnfofoei{\vlist{T}}{\Pi}{\Sigma} = \dbnfofoe{\vlist{T}}{\Pi \cup \Sigma} \land \dbnfinf{\Sigma}
\]
where
\[
\dbnfinf{\Sigma} := \bigwedge_{S\in\Sigma} \qu y.\tau_S(y) \land \dqu y.\bigvee_{S\in\Sigma} \tau_S(y)
\]
%
for some set of types $\Pi,\Sigma \subseteq \wp A$ and each $T_i \subseteq A$.
\end{definition}

Intuitively, the formula $\dbnfinf{\Sigma}$ says that (1) for every type $S\in\Sigma$, there are infinitely many elements satisfying $S$ and (2) only finitely many elements do not satisfy any type in $\Sigma$.

A short argument reveals that, intuitively, every disjunct expresses that each one-step model satisfying it admits a partition of its domain in three parts:
\begin{enumerate}[(i)]
\itemsep 0 pt
\item distinct elements $t_1,\dots,t_n$ with type $T_1,\dots,T_n$,
\item finitely many elements whose types belong to $\Pi$, and
\item for each $S\in \Sigma$, infinitely many elements with type $S$.
\end{enumerate}

In the same way as before, we define a relation $\sim^\infty_k$ which refines $\sim^=_{k}$ by adding information about the (in-)finiteness of the types.

% \index{$\sim^\infty_k$}
\begin{definition}
	Let $\osmodel$ and $\osmodel'$ be one-step models. For every $k\in\nat$, the relation $\osmodel \sim^\infty_k \osmodel'$ is defined as follows:
\begin{eqnarray*}
	\osmodel \sim^\infty_0 \osmodel' & \Longleftrightarrow & \text{always}\\
	\osmodel \sim^\infty_{k+1} \osmodel' & \Longleftrightarrow & \osmodel \sim^=_{k+1} \osmodel' \text{ and }\\
	&& \forall S\subseteq A \ \big(
	|S|_\osmodel,|S|_{\osmodel'} < \omega \text{ or } |S|_\osmodel,|S|_{\osmodel'} \geq \omega \big)%\\
	%&& \qquad\qquad \text{or } |S|_\osmodel,|S|_{\osmodel'} \geq \omega \big)
\end{eqnarray*}
\end{definition}

\begin{proposition}\label{props:eqrelolque} The following hold:
	\begin{enumerate}[(i)]
		\itemsep 0 pt
		\item $\sim^\infty_k$ is an equivalence relation,
		\item $\sim^\infty_k$ has finite index,
		\item $\sim^\infty_k$ is a refinement of $\sim^=_k$,
		\item Every $E \in \umods/{\sim^\infty_k}$ is characterized by a formula $\varphi^\infty_E \in \ofoei(A)$ with $qr(\varphi) = k$.
	\end{enumerate}
\end{proposition}
\begin{proof}
	We only prove the last point, for $k>0$. Let $E \in \umods/{\sim^\infty_k}$ and let $\osmodel \in E$ be a representative of the class. Let $E' \in \umods/{\sim^=_k}$ be the equivalence class of $\osmodel$ with respect to $\sim^=_k$.
	%
	Let $S_1,\dots,S_n \subseteq A$ be all the types such that $|S_i|_\osmodel \geq \omega$. % and $S'_1,\dots,S'_m \subseteq A$ to those satisfying $|S'_i|_\osmodel < \omega$. Now define
	%
	\[
	\varphi^\infty_E := \varphi^=_{E'} \land
		% \bigwedge_{i\leq n} \qu x.\tau_{S_i}(x) \land \bigwedge_{i\leq m} \dqu x.\lnot\tau_{S'_i}(x) .
		\dbnfinf{\{S_1,\dots,S_n\}} .
	\]
	%
	It is not difficult to see that $qr(\varphi^\infty_E) = k$ and that $\osmodel' \models \varphi^\infty_E$ iff $\osmodel' \in E$. %Observe again, that $\varphi^\infty_E$ gives a specification of $E$ ``type by type''.
\end{proof}

Now we give a notion of Ehrenfeucht-Fra\"ss\'e game for $\ofoei$. In this case the game extends $\efgame^=_k$ with a move for $\qu$.

% \index{$\efgame^\infty_k(\osmodel_0,\osmodel_1)$}
\begin{definition}
	Let $\osmodel_0 = (D_0,\val_0)$ and $\osmodel_1 = (D_1,\val_1)$ be one-step models. We define the game $\efgame^\infty_k(\osmodel_0,\osmodel_1)$ between \abelard and \eloise. A position in this game is a pair of sequences $\vlist{s_0} \in D_0^n$ and $\vlist{s_1} \in D_1^n$ with $n \leq k$. The game consists of $k$ rounds, where in round $n+1$ the following steps are made. First \abelard chooses to perform one of the following types of moves:
	%
	\begin{enumerate}[(a)]
		\itemsep 0 pt
		\parsep 0 pt
		%
		\item Second-order move
		\begin{enumerate}[1.]
			\itemsep 0 pt
			\parsep 0 pt
			\item \abelard chooses an infinite set $X_i \subseteq D_i$,
			\item \eloise responds with an infinite set $X_{-i} \subseteq D_{-i}$,
			%\item {\color{red} This is probably not needed anymore (according to~\cite{Kolaitis199523}):} Given a fresh name $s \notin A$ the models are extended to $\osmodel_i = (M_i, \val_i[s \mapsto X_i])$,
			\item \abelard chooses an element $x_{-i} \in X_{-i}$,
			\item \eloise responds with an element $x_i \in X_i$.
		\end{enumerate}
		%
		\item First-order move
		\begin{enumerate}[1.]
			\itemsep 0 pt
			\parsep 0 pt
			\item \abelard chooses an element $d_i \in D_i$,
			\item \eloise responds with an element $d_{-i} \in D_{-i}$.
		\end{enumerate}
	\end{enumerate}
	%
	Let $\vlist{s_i} \in D_i^n$ be the sequences of elements chosen up to round $n$, they are extended to ${\vlist{s_i}' := \vlist{s_i}\cdot d_i}$. \eloise survives the round iff she does not get stuck and the function $f_{n+1}: \vlist{s_0}' \mapsto \vlist{s_1}'$ is a partial isomorphism of one-step models.
\end{definition}

This game can be seen as an adaptation of the Ehrenfeucht-Fra\"ss\'e game for monotone generalized quantifiers found in~\cite{Kolaitis199523} to the case of full monadic first-order logic. %see Def 3.2, 3.8 and Thm 3.9.

\begin{lemma}\label{lem:connolque}
	The following are equivalent:
	\begin{enumerate}
		\itemsep 0 pt
		\item\label{lem:connolque:i} $\osmodel_0 \equiv_k^{\foei} \osmodel_1$,
		\item\label{lem:connolque:ii} $\osmodel_0 \sim_k^\infty \osmodel_1$,
		\item\label{lem:connolque:iii} \eloise has a winning strategy in $\efgame_k^\infty(\osmodel_0,\osmodel_1)$.
	\end{enumerate}
\end{lemma}

%{\color{red} The $\equiv_k$ is probably not ok. For example, we have much more succinctness to count in WMSO: we can say ``there are $2n$ elements'' with a formula of quantifier rank $n+1$ using $\exists X. (\exists x_1\dots x_n. x_i\in X) \land (\exists x_1\dots x_n. x_i\notin X)$. Should it be $2^k$ somewhere?}

\begin{proof}
	Step~(\ref{lem:connolque:i}) to~(\ref{lem:connolque:ii}) is direct by Proposition~\ref{props:eqrelolque}. For~(\ref{lem:connolque:ii}) to~(\ref{lem:connolque:iii}) we show
	%
	\begin{claimfirst}
	Let $\osmodel_0 \sim_k^\infty \osmodel_1$ and $\vlist{s_i} \in D_i^n$ be such that $n<k$ and $f_n:\vlist{s_0}\mapsto\vlist{s_1}$ is a partial isomorphism; then \eloise can survive one more round in $\efgame_{k}^\infty(\osmodel_0,\osmodel_1)@(\vlist{s_0},\vlist{s_1})$.
	% Let $\osmodel_0 \sim_k^\infty \osmodel_1$, $n<k$ and $\vlist{r_n} \in D_0^n$, $\vlist{s_n} \in D_1^n$ be  such that $f_n:\vlist{r_n}\mapsto\vlist{s_n}$ is a partial isomorphism; then \eloise can survive one more round in $\efgame_{k}^\infty(\osmodel_0,\osmodel_1)@(\vlist{r_n},\vlist{s_n})$.
	\end{claimfirst}
	%
	\begin{pfclaim}
		We focus on the second-order moves because the first-order moves are the same as in the corresponding Claim of Lemma~\ref{lem:connofoe}. Let \abelard choose an infinite set $X_i \subseteq D_i$, we would like \eloise to choose a set $X_{-i} \subseteq D_{-i}$ such that the following conditions hold:
		%
		\begin{enumerate}[(a)]
			\parskip 0pt
			%
			\item\label{it:piso} %For $j\in\{0,1\}$ let $\vlist{r_j} := \vlist{s_j}\cap X_j$ be the restriction of $\vlist{s_j}$ to the elements of $X_j$. We want 
			The map $f_n$ is a well-defined partial isomorphism between the restricted one-step models $\osmodel_0{\rest}X_0$ and $\osmodel_1{\rest}X_1$,
			%
			\item\label{it:equiv}
			For every type $S$ we have that there is an element $d\in X_i$ of type $S$ which is \emph{not} connected by $f_n$ iff there is such an element in $X_{-i}$,
			% $\osmodel_0{\rest}X_0 \sim_{m+1}^= \osmodel_1{\rest}X_1$ where $m = |r_i|$. That is, for all $S \subseteq A$,
			% \begin{enumerate}[(i)]
			% 	\itemsep 0pt
			% 	\item if $|S|_{X_i} < m+1$ then $|S|_{X_{-i}} = |S|_{X_i}$,
			% 	\item if $|S|_{X_i} \geq m+1$ then $|S|_{X_{-i}} \geq m+1$,
			% 	%\item if $|S|_{X_i} \geq \omega$ then $|S|_{X_{-i}} \geq \omega$,
			% \end{enumerate}
			%
			\item\label{it:inf} $X_{-i}$ is infinite.
		\end{enumerate}
		%
		First we prove that such a set exists. % and then we will use it to prove the claim.
		%
		To satisfy item~\eqref{it:piso} she just needs to add to $X_{-i}$ the elements connected to $X_i$ by $f_n$; this is not a problem.

%		\begin{figure}
%			\centering
%			\includegraphics[scale=0.9]{chapter5/fig-efinf.pdf}
%			% \caption{$k=10, n=5, m=2, |S|_{X_i}=3$ and $|S|_{D_i}=|S|_{D_{-i}}=6$.}
%			\caption{Elements of type $S$ have coloured background.}
%			\label{fig:efinf}
%		\end{figure}

		For item~\eqref{it:equiv} we proceed as follows: for every type $S$ such that there is an element $d\in X_i$ of type $S$, we add a new element $d'\in D_{-i}$ of type $S$ to $X_{-i}$. To see that this is always possible, observe first that $\osmodel_0 \sim_k^\infty \osmodel_1$ implies $\osmodel_0 \sim_k^= \osmodel_1$. Using the properties of this relation, we divide in two cases:
		%
		\begin{itemize}
			\item If $|S|_{D_i} \geq k$ we know that $|S|_{D_{-i}} \geq k$ as well. From the elements of $D_{-i}$ of type $S$, at most $n<k$ are used by $f_n$. Hence, there is at least one $d'\in D_{-i}$ of type $S$ to choose from.
			%
			\item If $|S|_{D_i} < k$ we know that $|S|_{D_{i}} = |S|_{D_{-i}}$. From the elements of $D_{i}$ of type $S$, at most $|S|_{D_{i}}-1$ are used by $f_n$. The reason for the $-1$ is that we are assuming that we have just chosen a $d\in X_i$ which is not in $f_n$. Using that $|S|_{D_{i}} = |S|_{D_{-i}}$ and that $f_n$ is a partial isomorphism we can again conclude that there is at least one $d'\in D_{-i}$ of type $S$ to choose from.
		\end{itemize}
		%
		For item~\eqref{it:inf} observe that as $X_{i}$ is infinite but there are only finitely many types, there must be some $S$ such that $|S|_{X_i} \geq \omega$. It is then safe to add infinitely many elements for $S$ in $X_{-i}$ while considering point~\eqref{it:equiv}. Moreover, the existence of infinitely many elements satisfying $S$ in $D_{-i}$ is guaranteed by $\osmodel_0 \sim_k^\infty \osmodel_1$.

		Having shown that \eloise can choose a set $X_{-i}$ satisfying the above conditions, it is now clear that using point~\eqref{it:equiv} \eloise can survive the ``first-order part'' of the second-order move we were considering. This finishes the proof of the claim.
		% we continue the proof of the claim as follows: as $\osmodel_0{\rest}X_0 \sim_{m+1}^= \osmodel_1{\rest}X_1$ and $f':\vlist{r_0}\mapsto\vlist{r_1}$ is a partial isomorphism between them (with sequences of length $m$) then by the first claim of Lemma~\ref{lem:connofoe}, we know that \eloise can survive one round in $\efgame_{m+1}^=(\osmodel_0{\rest}X_0,\osmodel_1{\rest}X_1)@(\vlist{r_0},\vlist{r_1})$. In particular, she can survive the ``first-order part'' of the second-order move we were considering. This finishes the proof of the claim.
	\end{pfclaim}
	%
	Going back to the proof of Lemma~\ref{lem:connolque}, for step~(\ref{lem:connolque:iii}) to~(\ref{lem:connolque:i}) we prove the following.
	%
	\begin{claim}
		Let $\vlist{s_i} \in D_i^n$ and $\varphi(z_1,\dots,z_n) \in \ofoei(A)$ be such that $qr(\varphi) \leq k-n$. If \eloise has a winning strategy in $\efgame_k^\infty(\osmodel_0,\osmodel_1)@(\vlist{s_0},\vlist{s_1})$ then $\osmodel_0 \models \varphi(\vlist{s_0})$ iff $\osmodel_1 \models \varphi(\vlist{s_1})$.
	\end{claim}
	%
	\begin{pfclaim}
		All the cases involving operators of \ofoe are the same as in Lemma~\ref{lem:connofoe}. We prove the inductive case for the generalized quantifier. Let $\varphi(z_1,\dots,z_n)$ be of the form $\qu x.\psi(z_1,\dots,z_n,x)$ and let $\osmodel_0 \models \varphi(\vlist{s_0})$. Hence, there is an \emph{infinite} set $X_0 \subseteq D_0$ such that
		%
		\begin{equation}\label{eq:ind0ok}
		\osmodel_0 \models \psi(\vlist{s_0},x_0) \text{ if and only if } x_0\in X_0 .
		\end{equation}
		%
		By hypothesis we know that \eloise has a winning strategy for $\efgame_k^\infty(\osmodel_0,\osmodel_1)@(\vlist{s_0},\vlist{s_1})$. Therefore, if \abelard plays a second-order move by picking $X_0 \subseteq D_0$ she can respond with some infinite set $X_1 \subseteq D_1$. We claim that $\osmodel_1 \models \psi(\vlist{s_1},x_1)$ for every $x_1\in X_1$. First observe that if this holds then the set $X'_1 := \{ d_1 \in D_1 \mid \osmodel_1 \models \psi(\vlist{s_1},d_1)\}$ must be infinite, and hence $\osmodel_1 \models \qu x.\psi(\vlist{s_1},x)$.% and we are done.
		
		Assume, for a contradiction, that $\osmodel_1 \not\models \psi(\vlist{s_1},x'_1)$ for some $x'_1\in X_1$. Let \abelard play that $x'_1$ as the second part of his move. Then, as \eloise has a winning strategy, she will respond with some $x'_0 \in X_0$ such that she has a winning strategy for $\efgame_{k}^\infty(\osmodel_0,\osmodel_1)@(\vlist{s_0}{\cdot}x'_0,\vlist{s_1}{\cdot}x'_1)$. By induction hypothesis, as $qr(\psi) \leq k-(n+1)$, this means that $\osmodel_0 \models \psi(\vlist{s_0},x'_0)$ iff $\osmodel_1 \models \varphi(\vlist{s_1},x'_1)$ which contradicts~(\ref{eq:ind0ok}). The other direction is symmetric.
	\end{pfclaim}
	%
	Combining the claims finishes the proof of the lemma.
\end{proof}


\begin{theorem}\label{thm:bfofoei}
Every formula $\varphi \in \ofoei(A)$ is equivalent to a formula in basic form.
\end{theorem}
\begin{proof}
	This can be proved using the same technique as in Theorem~\ref{thm:bnfofoe}. Hence we only focus on showing that $\varphi_E^\infty \equiv \dbnfofoei{\vlist{T}}{\Pi}{\Sigma}$ for some $\Pi,\Sigma \subseteq \wp A$ and $T_i \subseteq A$. Recall that
	\[
	\varphi^\infty_E = \varphi^=_{E'} \land \dbnfinf{\{S_1,\dots,S_n\}}
	\]
	where $\{S_1,\dots,S_n\}$ are all the types that should be satisfied by infinitely many elements.
	Using Theorem~\ref{thm:bnfofoe} on $\varphi^=_{E'}$ we know that this is equivalent to
	\[
	\varphi^\infty_E = \dbnfofoe{\vlist{T}'}{\Pi'} \land \dbnfinf{\{S_1,\dots,S_n\}}
	\]
	for some $\Pi' \subseteq \wp A$ and $T'_i \subseteq A$. Now separate $\Pi'$ as $\Pi' = \Pi \uplus \Sigma$ where $\Sigma:=\{S_1,\dots,S_n\}$ is composed of the infinite types and $\Pi := \Pi'\setminus\Sigma$ is composed of the finite types. After a minor rewriting, we get that
	%
	\[
	\varphi^\infty_E \equiv \dbnfofoe{\vlist{T}'}{\Pi\cup\Sigma} \land \dbnfinf{\Sigma}.
	\]
	%
	Therefore, we can conclude that $\varphi^\infty_E \equiv \dbnfofoei{\vlist{T}'}{\Pi}{\Sigma}$.
\end{proof}

\noindent The following stronger normal form will be useful in later chapters.

\begin{proposition}\label{prop:bfofoei-sigmapi}
	For every formula in the basic form $\bigvee \dbnfofoei{\vlist{T}}{\Pi}{\Sigma}$ it is possible to assume, without loss of generality, that $\Sigma \subseteq \Pi$.
\end{proposition}
\begin{proof}
	This is direct from observing that $\dbnfofoei{\vlist{T}}{\Pi}{\Sigma}$ is equivalent to $\dbnfofoei{\vlist{T}}{\Pi\cup\Sigma}{\Sigma}$. To check it we just unravel the definitions and observe that
	$\dbnfofoe{\vlist{T}}{\Pi \cup \Sigma} \land \dbnfinf{\Sigma}$ is equivalent to $\dbnfofoe{\vlist{T}}{\Pi \cup \Sigma \cup \Sigma} \land \dbnfinf{\Sigma}$.
\end{proof}

\subsection{One-step monotonicity}\label{subsec:one-stepmonot}
% !TEX root = ../main.tex

\index{formula!one-step!monotone}
Given a one-step logic $\llang(A)$ and formula $\varphi \in \llang(A)$.
%
We say that $\varphi$ is \emph{monotone in $\{\vlist{a}\} \subseteq A$} if for every one step model $(D,\val)$ and assignment $\ass:\fovar\to D$,
\[
\text{if } (D,\val),\ass \models \varphi \text{ and } \val(\vlist{a}) \subseteq \vlist{E} \text{ then } (D,\val[\vlist{a}\mapsto \vlist{E}]),\ass \models \varphi.
\]
%
% We use $\llang^+(A)$ to denote the fragment of $\llang(A)$ composed of formulas monotone in all $a\in A$.

\begin{remark}\label{rem:monotprodeach}
	It is easy to prove that a formula is monotone in $\{\vlist{a}\} \subseteq A$ iff it is monotone in every $a_i$. Therefore, in the following proofs we will, in general, consider monotonicity in every single $a_i$ instead of in the full $\vlist{a}$. This is equivalent, and only done to avoid an even more complex notation.
\end{remark}

Monotonicity is usually tightly related to positivity. If the quantifiers are well-behaved (i.e., monotone) then a formula $\varphi$ will usually be monotone in $a \in A$ iff $a$ has positive polarity in $\varphi$, that is, if it only occurs under an even number of negations. This is the case for all one-step logics considered in this dissertation. In this section we give a syntactic characterization of monotonicity for several one-step logics.

\index{$\tau^{A'}_S$}
\index{type!$A'$-positive}
\begin{definition}
	Given $S \subseteq A$ and $A' \subseteq A$ we use the following notation
	\[
	\tau^{A'}_S(x) := \bigwedge_{b\in S} b(x) \land \bigwedge_{b\in A\setminus (S\cup A')}\lnot b(x) ,
	\]
	for what we call the \emph{$A'$-positive} $A$-type $\tau^{A'}_S$.
	Intuitively, $\tau^{A'}_S$ works almost like the $A$-type $\tau_S$, but discarding the negative information for the names in $A'$.
	If $A' = \{a\}$ we write $\tau^a_S$ instead of $\tau^{\{a\}}_S$. Observe that with this notation, $\tau^+_S$ is equivalent to $\tau^A_S$.
	%Moreover, we use $\tau^+_S$ to denote the \emph{positive} $A$-type defined as $\tau^+_S(x) := \bigwedge_{b\in S} b(x)$.
\end{definition}

%	For every one-step logic $\llang$ for which we have a basic form based on disjuncts $\dbnf_{\llang}$ we use $\dbnf^p_{\llang}$ to denote the basic form that we get by replacing every type $\tau_S$ with $\tau^a_S$; analogously, we use $\dbnf_{\llang}^+$ to denote the replacement of every type $\tau_S$ with $\tau^+_S$.

%%
\subsubsection{Monotone fragment of $\ofo$}

\index{fragment!monotone!$\ofo$}
\index{$\ofo$!$\monot{}{A'}$}
\begin{theorem}\label{thm:ofomonot}
A formula of $\ofo(A)$ is monotone in $A' \subseteq A$ iff it is equivalent to a sentence given by:
\[
\varphi ::= \psi \mid a(x) \mid \exists x.\varphi \mid \forall x.\varphi \mid \varphi \land \varphi \mid \varphi \lor \varphi
\]
where $a \in A'$,
$\psi \in \ofo(A\setminus A')$. We denote this fragment as $\monot{\ofo}{A'}(A)$.
\end{theorem}
%
The result will follow from the following two lemmas and Remark~\ref{rem:monotprodeach}.

\begin{lemma}\label{lem:monofoismonot}
Every $\varphi \in \monot{\ofo}{a}(A)$ is monotone in $a$.
\end{lemma}
\begin{proof}
	The proof is a routine argument by induction on the complexity of $\varphi$.
\end{proof}
% \begin{proof}
% We show, by induction, that any one-step formula $\varphi$ in the fragment (which may not be a sentence) satisfies, for every one-step model $(D,\val:A\to\wp D)$, assignment ${\ass:\fovar\to D}$, 
% %
% \[(D,\val),\ass \models \varphi \text{ and } \val(a) \subseteq E \text{ then } (D,\val[a\mapsto E]),\ass \models \varphi.\] 
% %
% \begin{enumerate}[\textbullet]
% \item If ${\varphi = \psi \in \ofo(A\setminus \{a\})}$ changes in the $a$ part of the valuation will make no difference and hence the condition is trivial. %The same happens if $\varphi = q(x)$ with $q\in P$ but $q \neq a$.

% \item Case $\varphi = a(x)$: if $(D,\val),\ass \models a(x)$ then $\ass(x)\in \val(a)$. Clearly, $\ass(x) \in \val[a\mapsto E](a)$ because $\val(a) \subseteq E$. Therefore $(D, \val[a\mapsto E]),\ass \models a(x)$. %For the other direction assume $U \subseteq_\omega \val(a)$ and $(D, \val[a\mapsto U]),\ass \models a(x)$. This means that $\ass(x) \in U \subseteq \val(a)$, hence $(D, \val),\ass \models a(x)$.

% \item Case $\varphi = \varphi_1 \lor \varphi_2$: assume $(D,\val),\ass \models \varphi$. Without loss of generality we can assume that $(D,\val),\ass \models \varphi_1$ and hence by induction hypothesis we have that $(D,\val[a\mapsto E]),\ass \models \varphi_1$ which clearly satisfies $(D,\val[a\mapsto E]),\ass \models \varphi$. %For the other direction let $U \subseteq_\omega \val(a)$ and assume wlog that $(D,\val[a\mapsto U]),\ass \models \varphi_1$. By induction hypothesis $(D,\val),\ass \models \varphi_1$ which entails $(D,\val),\ass \models \varphi$.

% \item Case $\varphi = \varphi_1 \land \varphi_2$: assume $(D,\val),\ass \models \varphi$. As we have $(D,\val),\ass \models \varphi_i$, by induction hypothesis we get $(D,\val[a\mapsto E]),\ass \models \varphi_i$. Therefore $(D,\val[a\mapsto E]),\ass \models \varphi$. %The other direction is very similar to the case of disjunction.

% \item Case $\varphi = \exists x.\varphi'(x)$ and $(D,\val),\ass \models \varphi$. By definition there exists $d\in D$ such that $(D,\val),\ass[x\mapsto d] \models \varphi'(x)$. By induction hypothesis $(D,\val[a\mapsto E]),\ass[x\mapsto d] \models \varphi'(x)$ and hence $(D,\val[a\mapsto E]),\ass \models \exists x.\varphi'(x)$.

% \item Case $\varphi = \forall x.\varphi'(x)$ and $(D,\val),\ass \models \varphi$. By definition $(D,\val),\ass[x\mapsto d] \models \varphi'(x)$ for all $d\in D$. By induction hypothesis $(D,\val[a\mapsto E]),\ass[x\mapsto d] \models \varphi'(x)$ for all $d\in D$. Hence $(D,\val[a\mapsto E]),\ass \models \forall x.\varphi'(x)$.
% \end{enumerate}
% %
% This finishes the proof.
% \end{proof}

Before going on we need to introduce a bit of new notation. In Section~\ref{subsec:normalforms} we introduced the formula $\dbnfofo{\Sigma}$. We now give a few variants of this notation, which will be crucial to build the normal forms of the fragments discussed in this dissertation.

\index{$\mondbnfofo{\Sigma}{A'}$}
\index{$\mondgbnfofo{\Sigma}{\Pi}{A'}$}
\begin{definition}
Let $A'\subseteq A$ be a finite set of names. The $A'$-positive variant of $\dbnfofo{\Sigma}$ is given as follows:
% \begin{align*}
% 	\mondbnfofo{\Sigma}{A'} &:= \bigwedge_{S\in\Sigma} \exists x. \tau^{A'}_S(x) \land \forall x. \bigvee_{S\in\Sigma} \tau^{A'}_S(x) \\
% 	\posdbnfofo{\Sigma} &:= \bigwedge_{S\in\Sigma} \exists x. \tau^+_S(x) \land \forall x. \bigvee_{S\in\Sigma} \tau^+_S(x) .
% \end{align*}
\[
	\mondbnfofo{\Sigma}{A'} := \bigwedge_{S\in\Sigma} \exists x. \tau^{A'}_S(x) \land \forall x. \bigvee_{S\in\Sigma} \tau^{A'}_S(x).
\]
We also introduce the following generalized forms of the above notation:
% \begin{align*}
% 	\mondgbnfofo{\Sigma}{\Pi}{A'} &:= \bigwedge_{S\in\Sigma} \exists x. \tau^{A'}_S(x) \land \forall x. \bigvee_{S\in\Pi} \tau^{A'}_S(x) \\
% 	\posdgbnfofo{\Sigma}{\Pi} &:= \bigwedge_{S\in\Sigma} \exists x. \tau^+_S(x) \land \forall x. \bigvee_{S\in\Pi} \tau^+_S(x) .
% \end{align*}
\[
	\mondgbnfofo{\Sigma}{\Pi}{A'} := \bigwedge_{S\in\Sigma} \exists x. \tau^{A'}_S(x) \land \forall x. \bigvee_{S\in\Pi} \tau^{A'}_S(x) .
\]
The \emph{positive} variants of the above notations are defined as $\posdbnfofo{\Sigma} := \mondbnfofo{\Sigma}{A}$ and $\posdgbnfofo{\Sigma}{\Pi} := \mondgbnfofo{\Sigma}{\Pi}{A}$.
\end{definition}

To prove that the fragment $\monot{\ofo}{a}$ is complete for monotonicity in $a$, we need to show that every formula which is monotone in $a$ is equivalent to some formula in $\monot{\ofo}{a}$. We prove a stronger result: we give a translation that constructively maps arbitrary formulas into $\monot{\ofo}{a}$. The interesting observation is that the translation will preserve truth iff the given formula is monotone in $a$.

\begin{lemma}
There exists a translation $(-)^\tmono:\ofo(A) \to \monot{\ofo}{a}(A)$ such that
a formula ${\varphi \in \ofo(A)}$ is monotone in $a\in A$ if and only if $\varphi\equiv \varphi^\tmono$.
\end{lemma}
%
\begin{proof}
To define the translation we assume, without loss of generality, that $\varphi$ is in the normal form $\bigvee \dbnfofo{\Sigma}$ given in Definition~\ref{def:bfofo}, that is:
%
\[
\dbnfofo{\Sigma} := \bigwedge_{S\in\Sigma} \exists x. \tau_S(x) \land \forall x. \bigvee_{S\in\Sigma} \tau_S(x).
\]
%
% for some types $\Sigma \subseteq \wp A$.
%
We define the translation as
$(\bigvee \dbnfofo{\Sigma})^\tmono:= \bigvee \mondbnfofo{\Sigma}{a}$.
%  where
% %
% \[
% \mondbnfofo{\Sigma}{A'} := \bigwedge_{S\in\Sigma} \exists x. \tau^{A'}_S(x) \land \forall x. \bigvee_{S\in\Sigma} \tau^{A'}_S(x)
% \]

From the construction it is clear that $\varphi^\tmono \in \monot{\ofo}{a}(A)$ and therefore the right-to-left direction of the lemma is immediate by Lemma~\ref{lem:monofoismonot}. For the left-to-right direction assume that $\varphi$ is monotone in $a$, we have to prove that $(D,\val) \models \varphi$ if and only if $(D,\val) \models \varphi^\tmono$.

\bigskip
\noindent \fbox{$\Rightarrow$} This direction is trivial.

\bigskip
\noindent \fbox{$\Leftarrow$} Assume $(D,\val) \models \varphi^\tmono$ and let $\Sigma$ be such that $(D,\val) \models \mondbnfofo{\Sigma}{a}$. Because of the universal part of $\mondbnfofo{\Sigma}{a}$, it is safe to assume that the \emph{only} ($a$-positive) types realized in $(D,\val)$ are exactly those in $\Sigma$; moreover, it is also safe to assume that every type has a (single) distinct witness (this is because $(D,\val)$ can be proved to be $\ofo$-equivalent to such a model).
%
%We have to prove that $(D,\val) \models \varphi$. It is easy to prove that $(D,\val) \equiv_\fo (D\times \{0,1\},\val_\pi)$ where $D\times\{0,1\}$ has $2$ copies of each element of $D$ and $\val_\pi((d,i)) := \val(d)$. Hence, it is enough to prove that $(D\times \{0,1\},\val_\pi) \models \varphi$.
%
For every $S \in \Sigma$, let $d_S$ be the witness of the $a$-positive type $\tau^a_S(x)$ in $(D,\val)$. Let $U := \{d_S \mid S\in \Sigma, a\notin S\}$ and $\val' := \val[a \mapsto \val(a) \setminus U]$. 

\begin{claimfirst}
	$(D,\val') \models \dbnfofo{\Sigma}$.
\end{claimfirst}

\begin{pfclaim}
First we show that the existential part of the normal form is satisfied. That is, that for every $S\in \Sigma$ we have a witness for the \emph{full} type $\tau_S(x)$. If $a\in S$ the witness is given by $\varphi^\tmono$, that is, $d_S$. If $a \notin S$ then we specially crafted $d_S$ to be a witness. The universal part is clearly satisfied.
\end{pfclaim}
%
To finish observe that, by monotonicity of $\varphi$, we get $(D,\val) \models \varphi$. %This finishes the proof of the lemma.
\end{proof}

Putting together the above lemmas we obtain Theorem~\ref{thm:ofomonot}. Moreover, a careful analysis of the translation gives us the following corollary, providing normal forms for the monotone fragment of $\ofo$.

\begin{corollary}\label{cor:ofopositivenf} Let $\varphi \in \ofo(A)$, the following hold:
	\begin{enumerate}[(i)]
		\item The formula $\varphi$ is monotone in $A' \subseteq A$ iff it is equivalent to a formula in the basic form $\bigvee \mondbnfofo{\Sigma}{A'}$ for some types $\Sigma \subseteq \wp A$.
		%
		\item The formula $\varphi$ is monotone in every $a\in A$ (i.e., $\varphi\in\ofo^+(A)$) iff $\varphi$ is equivalent to a formula $\bigvee \posdbnfofo{\Sigma}$ for some types $\Sigma \subseteq \wp A$. %, where
		%
		%For the translation, let
		%$(\bigvee \dbnfofo{\Sigma})^\tmono:= \bigvee \mondbnfofo{\Sigma}{a}$ and
		%
		% \[
		% \posdbnfofo{\Sigma} := \bigwedge_{S\in\Sigma} \exists x. \tau^+_S(x) \land \forall x. \bigvee_{S\in\Sigma} \tau^+_S(x) .
		% \]
	\end{enumerate}
\end{corollary}

% The following stronger normal form will be useful. Intuitively, it says that every set $\Sigma$ is comprised of types that are incomparable (in terms of containment) between each other.

% \begin{proposition}\label{props:strongmonofo}
% In the above normal form we can assume that every disjunct $\mondbnfofo{\Sigma}{A'}$ is such that for every pair of distinct $S,S'\in \Sigma$ at least one of the following conditions hold:
% \begin{itemize}
% 	\itemsep 0 pt
% 	\item $S \cap (A\setminus A') \neq S' \cap (A\setminus A')$, or
% 	\item $S \cap A' \not\subseteq S' \cap A'$ and $S' \cap A' \not\subseteq S \cap A'$.
% \end{itemize}
% \end{proposition}
% %
% \begin{proof}
% 	Assume that for some distinct $S,S'\in\Sigma$ neither of the conditions hold. That is,
% 	%
% 	$S \cap (A\setminus A') = S' \cap (A\setminus A')$ and
% 	either (1) $S \cap A' \subseteq S' \cap A'$ or (2) $S' \cap A' \subseteq S \cap A'$.
% 	%
% 	It is easy to observe that if (1) holds then $\tau^{A'}_{S'}(x) \models \tau^{A'}_{S}(x)$ and if (2) holds then $\tau^{A'}_{S}(x) \models \tau^{A'}_{S'}(x)$.

% 	\begin{claimfirst}
% 		In both cases $\mondbnfofo{\Sigma}{A'}\equiv \mondbnfofo{\Sigma\setminus \{S\}}{A'} \lor \mondbnfofo{\Sigma\setminus \{S'\}}{A'}$.
% 	\end{claimfirst}
% 	%
% 	\begin{pfclaim}
% 		We only prove case (1) since the cases are symmetric. Let $(D,\val) \models \mondbnfofo{\Sigma}{A'}$. Either (a) every element satisfying $\tau^{A'}_{S}(x)$ also satisfies $\tau^{A'}_{S'}(x)$ or; (b)~there are witnesses for $\tau^{A'}_{S}(x)$ which do not satisfy $\tau^{A'}_{S'}(x)$. In case~(a) we clearly have that $(D,\val) \models \mondbnfofo{\Sigma\setminus \{S\}}{A'}$. For case~(b), it is trivial to see that $(D,\val)$ satisfies the existential part of $\mondbnfofo{\Sigma\setminus \{S'\}}{A'}$ since it poses less constraints. For the universal part, suppose that some element satisfies $\tau^{A'}_{S'}(x)$, then in particular (by hypothesis) it also satisfies $\tau^{A'}_{S}(x)$. Since $S\in \Sigma\setminus \{S'\}$ this does not carry any problem. We can conclude that $(D,\val) \models \mondbnfofo{\Sigma\setminus \{S'\}}{A'}$.
% 	\end{pfclaim}
% 	%
% 	This finishes the proof.
% \end{proof}

%%
\subsubsection{Monotone fragment of $\ofoe$}

\index{fragment!monotone!$\ofoe$}
\index{$\ofoe$!$\monot{}{A'}$}
\begin{theorem}\label{thm:ofoemonot}
A formula of $\ofoe(A)$ is monotone in ${A'\subseteq A}$ iff it is equivalent to a sentence given by:
\[
\varphi ::= \psi \mid a(x) \mid \exists x.\varphi \mid \forall x.\varphi \mid \varphi \land \varphi \mid \varphi \lor \varphi
\]
where $a\in A'$ and $\psi \in \ofoe(A\setminus A')$. We denote this fragment as $\monot{\ofoe}{A'}(A)$.
\end{theorem}

Observe that, in this definition, the equality predicate is taken into account by the $\psi$ clause.
Before going on we need to introduce a bit of new notation. 

\index{$\mondbnfofoe{\vlist{T}}{\Pi}{A'}$}
\index{$\posdbnfofoe{\vlist{T}}{\Pi}$}
\begin{definition}
Let $A'\subseteq A$ be a finite set of names. The monotone variant of $\dbnfofoe{\vlist{T}}{\Pi}$ is given as follows:
\begin{align*}
	\mondbnfofoe{\vlist{T}}{\Pi}{A'} &:= \exists \vlist{x}.\big(\arediff{\vlist{x}} \land \bigwedge_i \tau^{A'}_{T_i}(x_i) \land \forall z.(\arediff{\vlist{x},z} \lthen \bigvee_{S\in \Pi} \tau^{A'}_S(z))\big). 
	% \\
	% \posdbnfofoe{\vlist{T}}{\Pi} &:= \exists \vlist{x}.\big(\arediff{\vlist{x}} \land \bigwedge_i \tau^{+}_{T_i}(x_i) \land \forall z.(\arediff{\vlist{x},z} \lthen \bigvee_{S\in \Pi} \tau^{+}_S(z))\big) .
\end{align*}
When the set $A'$ is a singleton $\{a\}$ we will write $a$ instead of $A'$. The positive variant of $\dbnfofoe{\vlist{T}}{\Pi}$ is defined as above but with $+$ in place of $A'$.
\end{definition}

\noindent The result follows from the following lemma.

\begin{lemma}
The following hold:
\begin{enumerate}
	\itemsep 0pt
	\item Every $\varphi \in \monot{\ofoe}{A'}(A)$ is monotone in $A'$.
	\item There exists a translation $(-)^\tmono:\ofoe(A) \to \monot{\ofoe}{A'}(A)$ such that
a formula ${\varphi \in \ofoe(A)}$ is monotone in $A'$ if and only if $\varphi\equiv \varphi^\tmono$.
\end{enumerate}
\end{lemma}
\begin{proof}
	In Theorem~\ref{thm:ofoeimonot} this result is proved for \ofoei (i.e., \ofoe extended with generalized quantifiers). It is not difficult to adapt the proof for $\ofoe$. Intuitively, the translation is defined as $\varphi^\tmono := \varphi[\lnot a(x) \mapsto \top \mid a\in A']$ for $\varphi$ in negation normal form.
\end{proof}

Combining the normal form for $\ofoe$ and the above lemma, we obtain the following corollary providing a normal form for the monotone fragment of $\ofoe$.

\begin{corollary}\label{cor:ofoepositivenf}
	Given $\varphi \in \ofoe(A)$, the following hold:
	\begin{enumerate}[(i)]
		\item The formula $\varphi$ is monotone in $A'\subseteq A$ iff it is equivalent to a formula in the basic form $\bigvee \mondbnfofoe{\vlist{T}}{\Pi}{A'}$
		where
		%
		% \[
		% 	\mondbnfofoe{\vlist{T}}{\Pi}{A'} := \exists \vlist{x}.\big(\arediff{\vlist{x}} \land \bigwedge_i \tau^{A'}_{T_i}(x_i) \land \forall z.(\arediff{\vlist{x},z} \lthen \bigvee_{S\in \Pi} \tau^{A'}_S(z))\big)
		% \]
		% and
		for each disjunct we have $\vlist{T} \in \wp(A)^k$ for some $k$ and $\Pi\subseteq\vlist{T}$,
		%
		\item The formula $\varphi$ is monotone in all $a\in A$ (i.e., $\varphi\in \ofoe^+(A)$) iff it is equivalent to a formula in the basic form $\bigvee \posdbnfofoe{\vlist{T}}{\Pi}$
		where
		%
		% \[
		% 	\posdbnfofoe{\vlist{T}}{\Pi} := \exists \vlist{x}.\big(\arediff{\vlist{x}} \land \bigwedge_i \tau^{+}_{T_i}(x_i) \land \forall z.(\arediff{\vlist{x},z} \lthen \bigvee_{S\in \Pi} \tau^{+}_S(z))\big)
		% \]
		% and
		for each disjunct we have $\vlist{T} \in \wp(A)^k$ for some $k$ and $\Pi\subseteq\vlist{T}$.
	\end{enumerate}
\end{corollary}

%%
\subsubsection{Monotone fragment of $\ofoei$}

\index{fragment!monotone!$\ofoei$}
\index{$\ofoei$!$\monot{}{A'}$}
\begin{theorem}\label{thm:ofoeimonot}
A formula of $\ofoei(A)$ is monotone in ${A' \subseteq A}$ iff it is equivalent to a sentence given by:
\[
\varphi ::= \psi \mid a(x) \mid \exists x.\varphi \mid \forall x.\varphi \mid \varphi \land \varphi \mid \varphi \lor \varphi \mid \qu x.\varphi \mid \dqu x.\varphi
\]
where $a\in A'$ and $\psi \in \ofoei(A\setminus A')$. We call this fragment $\monot{\ofoei}{A'}(A)$.
\end{theorem}

Observe that $x \foeq y$ and $x \not\foeq y$ are included in the case $\psi \in \ofoei(A\setminus A')$. The result will follow from the following two lemmas and Remark~\ref{rem:monotprodeach}.

\begin{lemma}\label{lem:monofoeiismonot}
Every $\varphi \in \monot{\ofoei}{a}(A)$ is monotone in $a$.
\end{lemma}
\begin{proof}
The proof is basically the same as Lemma~\ref{lem:monofoismonot}.
That is, we show by induction, that any one-step formula $\varphi$ in the fragment (which may not be a sentence) satisfies, for every one-step model $(D,\val)$ and assignment ${\ass:\fovar\to D}$, 
%
\[
\text{if } (D,\val),\ass \models \varphi \text{ and } \val(a) \subseteq E \text{ then } (D,\val[a\mapsto E]),\ass \models \varphi.
\] 
%
We focus on the generalized quantifiers. Let $(D,\val),\ass \models \varphi$ and $\val(a) \subseteq E$.
%
\begin{enumerate}[\textbullet]
\item Case $\varphi = \qu x.\varphi'(x)$. By definition there exists an infinite set $I\subseteq D$ such that for all $d\in I$ we have $(D,\val),\ass[x\mapsto d] \models \varphi'(x)$. By induction hypothesis $(D,\val[a\mapsto E]),\ass[x\mapsto d] \models \varphi'(x)$ for all $d \in I$. Therefore $(D,\val[a\mapsto E]),\ass \models \qu x.\varphi'(x)$.

\item Case $\varphi = \dqu x.\varphi'(x)$. Hence there exists $I\subseteq D$ such that for all $d\in I$ we have $(D,\val),\ass[x\mapsto d] \models \varphi'(x)$ and $D\setminus I$ is \emph{finite}. By induction hypothesis $(D,\val[a\mapsto E]),\ass[x\mapsto d] \models \varphi'(x)$ for all $d \in I$. Therefore $(D,\val[a\mapsto E]),\ass \models \dqu x.\varphi'(x)$.
\end{enumerate}
%
This finishes the proof.
\end{proof}

\noindent Before going on, we introduce some notation.

\index{$\mondbnfofoei{\vlist{T}}{\Pi}{\Sigma}{A'}$}
\index{$\mondbnfinf{\Sigma}{A'}$}
\index{$\posdbnfofoei{\vlist{T}}{\Pi}{\Sigma}$}
\begin{definition}
Let $A'\subseteq A$ be a finite set of names. The $A'$-positive variant of $\dbnfofoei{\vlist{T}}{\Pi}{\Sigma}$ is given as follows:
\begin{align*}
	\mondbnfofoei{\vlist{T}}{\Pi}{\Sigma}{A'} &:= \mondbnfofoe{\vlist{T}}{\Pi \cup \Sigma}{A'} \land \mondbnfinf{\Sigma}{A'}\\
	%
	\mondbnfofoe{\vlist{T}}{\Lambda}{A'} &:= \exists \vlist{x}.\big(\arediff{\vlist{x}} \land \bigwedge_i \tau^{A'}_{T_i}(x_i) \land \forall z.(\arediff{\vlist{x},z} \lthen \bigvee_{S\in \Lambda} \tau^{A'}_S(z))\big) \\
	%
	\mondbnfinf{\Sigma}{A'} &:= \bigwedge_{S\in\Sigma} \qu y.\tau^{A'}_S(y) \land \dqu y.\bigvee_{S\in\Sigma} \tau^{A'}_S(y) .
\end{align*}
When the set $A'$ is a singleton $\{a\}$ we will write $a$ instead of $A'$. The positive variant of $\dbnfofoei{\vlist{T}}{\Pi}{\Sigma}$ is defined as $\posdbnfofoei{\vlist{T}}{\Pi}{\Sigma} := \mondbnfofoei{\vlist{T}}{\Pi}{\Sigma}{A}$.
\end{definition}

\noindent We are now ready to give the translation.

\begin{lemma}
There is a translation $(-)^\tmono:\ofoei(A) \to \monot{\ofoei}{a}(A)$ such that
a formula ${\varphi \in \ofoei(A)}$ is monotone in $a$ if and only if $\varphi\equiv \varphi^\tmono$.
\end{lemma}
\begin{proof}
We assume that $\varphi$ is in the normal form $\bigvee \dbnfofoei{\vlist{T}}{\Pi}{\Sigma}$ where 
%
\[
\dbnfofoei{\vlist{T}}{\Pi}{\Sigma} = \dbnfofoe{\vlist{T}}{\Pi \cup \Sigma} \land \dbnfinf{\Sigma} .
\]
%
for some sets of types $\Pi,\Sigma \subseteq \wp A$ and each $T_i \subseteq A$.
%
For the translation we define
\[
(\bigvee \dbnfofoei{\vlist{T}}{\Pi}{\Sigma})^\tmono:= \bigvee \mondbnfofoei{\vlist{T}}{\Pi}{\Sigma}{a}.
\]
%
From the construction it is clear that $\varphi^\tmono \in \monot{\ofoei}{a}(A)$ and therefore the right-to-left direction of the lemma is immediate by Lemma~\ref{lem:monofoeiismonot}. For the left-to-right direction assume that $\varphi$ is monotone in $a$, we have to prove that $(D,\val) \models \varphi$ if and only if $(D,\val) \models \varphi^\tmono$.

\bigskip
\noindent \fbox{$\Rightarrow$} This direction is trivial.

\bigskip
\noindent \fbox{$\Leftarrow$}
Assume $(D,\val) \models \varphi^\tmono$, and in particular that $(D,\val) \models \mondbnfofoei{\vlist{T}}{\Pi}{\Sigma}{a}$.
Observe that the elements of $D$ can be partitioned in the following way:
%
\begin{enumerate}[(a)]
	\itemsep 0 pt
	\item Distinct elements $t_i \in D$ such that each $t_i$ satisfies $\tau^{a}_{T_i}(x)$,% witnessing the $a$-positive type $T_i \in \vlist{T}$,
	\item\label{it:dpi} Disjoint sets $\compset{D_S \subseteq D \mid S \in \Sigma}$ such that each $D_S$ is \emph{infinite} and every $d \in D_S$ is a witness for the $a$-positive type $S \in \Sigma$,
	\item\label{it:ds} A \emph{finite} set $D_\Pi \subseteq D$ of witnesses of the $a$-positive types in $\Pi$.
\end{enumerate}
%
Following this partition, every element $d\in D$ is be the witness of an $a$-type in either (a)~$\vlist{T}$, (b)~$\Sigma$, or (c)~$\Pi$. We use $S_d \in \vlist{T}\cup \Pi \cup \Sigma$ to denote the $a$-type which $d$ witnesses. Now, we are talking about $a$-types, there might be a slight difference between $S_d$ and the actual type that each $d$ has (namely $\val^\natural(d)$). That is, it could be that $d\in\val(a)$ but that $a\notin S_d$. What we want to do now is to shrink $\val$ in such a way that the witnessed ($S_d$) type and the actual type coincide. We give a new valuation $\uval$ defined as $\uval^\natural(d) := S_d$.\footnote{Recall that a valuation $\uval:A\to\wp \osmoddom$ can also be represented as a marking $\uval^\natural: \osmoddom\to\wp A$ given by $\uval^\natural(d) := \{a \in A \mid d\in \val(a)\}$.} Observe that $\uval(a) \subseteq \val(a)$ and $\uval(b) = \val(b)$ for $b\in A\setminus\{a\}$.
%
\begin{claimfirst}
	$(D,\uval) \models \varphi$.
\end{claimfirst}
%
\begin{pfclaim}
	First we check that $(D,\uval) \models \dbnfofoe{\vlist{T}}{\Pi \cup \Sigma}$. It is easy to see that the elements $t_i$ work as witnesses for the \emph{full} types $T_i$. That is $(D,\uval) \models \tau_{T_i}(t_i)$ for every $i$. To prove the universal part of the formula it is enough to show that:
	%
	\begin{enumerate}
		\itemsep 0 pt
		\item Every element $d\in D_\Pi$ realizes the full type $S_d \in \Pi$,
		\item For all $S\in \Sigma$, every element of $D_S$ realizes the full type $S$.
	\end{enumerate}
	%
	Let $d$ be an element of either $D_\Pi$ or any of the $D_S$. By~\eqref{it:dpi} and~\eqref{it:ds} we know $(D,\val) \models \tau_{S_d}^a(d)$. If $a\in S_d$ we can trivially conclude $(D,\uval) \models \tau_{S_d}(d)$. If $a\notin S_d$, by definition of $\uval$ we know that $d \notin \uval(a)$ and hence we can also conclude that $(D,\uval) \models \tau_{S_d}(d)$.

	To prove that $(D,\uval) \models \bigwedge_{S\in\Sigma} \qu y.\tau_S(y) \land \dqu y.\bigvee_{S\in\Sigma} \tau_S(y)$ we only need to observe that the existential part is satisfied because each $D_S$ is infinite by~\eqref{it:ds} and the universal part is satisfied because the set $D_\Pi \cup \vlist{T}$ is finite by~\eqref{it:dpi}.
\end{pfclaim}
%
To finish the proof, note that by monotonicity of $\varphi$ we get $(D,\val) \models \varphi$.
%
% \begin{itemize}
% 	\itemsep 0 pt
% 	\item $(D,\val') \models \varphi^\tmono$,
% 	\item $\val'(a) \subseteq \val(a)$ and $\val'(b) = \val(b)$ for all $b\in A\setminus\{a\}$,
% 	\item For every $W \subset \val'(a)$ we have $(D,\val'[a\mapsto W]) \not\models \varphi^\tmono$.
% \end{itemize}
% %
% We prove that $(D,\val') \models \varphi$ and by monotonicity we get $(D,\val) \models \varphi$.
% Let $\vlist{d}$ be the sequence of pairwise distinct elements of $D$
% witnessing the types $\vlist{T}$ in $\mondbnfofoe{\vlist{T}}{\Pi \cup \Sigma}{a}$.
% %making $\arediff{\vlist{x}} \land \bigwedge_i \tau^a_{T_i}(x_i)  \land \forall z.(\arediff{\vlist{x},z} \lthen \bigvee_{S\in \Sigma \cup \Pi} \tau^a_S(z))$ true. Now, we modify $\val$ as follows. We state that $\val_1|_{D\setminus \vlist{d}}=\val|_{D\setminus \vlist{d}}$, while over $d_i \in \vlist{d}$ it is defined by
% % \[
% % \val_1(d_i)= \begin{cases}
% % \val(d_i) & \text{if } a \in T_i \\
% % \val(d_i) \setminus \{a\} & \text{else.} 
% % \end{cases}
% % \]
% % This means that 
% % \begin{equation}\label{eq:start}
% % (D,\val_1) \models \tau_{T_i}(d_i).
% %  \end{equation}
% % Now, we have that 
% %
% % \begin{claimfirst}\label{eq:1}
% % 	$(D\setminus \vlist{d},\val) \models  \forall z.\bigvee_{S\in \Sigma\cup\Pi} \tau^a_S(z) \land \bigwedge_{S\in\Sigma} \qu y.\tau^a_S(y) \land \dqu y.\bigvee_{S\in\Sigma} \tau^a_S(y)$.
% % \end{claimfirst}
% % %
% % \begin{pfclaim}
% % 	Prove.
% % \end{pfclaim}
%
% % \begin{equation}\label{eq:1}
% % (D\setminus \vlist{d},\val_1) \models  \forall z.\bigvee_{S\in \Sigma\cup\Pi} \tau^a_S(z)\land
% % \bigwedge_{S\in\Sigma} \qu y.\tau^a_S(y) \land \dqu y.\bigvee_{S\in\Sigma} \tau^a_S(y).
% % \end{equation}
%
% Let $D_\Pi= \{ d' \in D\setminus \vlist{d} \mid (D\setminus \vlist{d}, \val) \not\models \tau^a_S(d'), \forall S\in \Sigma\}$, and $D_\Sigma = D\setminus (\vlist{d} \cup D_\Pi)$. 
% By (\ref{eq:1}), $D_\Pi$ is finite and therefore it holds that
% \begin{equation}\label{eq:2}
% (D_\Sigma, \val_1)\models \forall z.\bigvee_{S\in \Sigma} \tau^a_S(z)\land
% \bigwedge_{S\in\Sigma} \qu y.\tau^a_S(y) \land \dqu y.\bigvee_{S\in\Sigma} \tau^a_S(y).
% \end{equation}
% Let $(D_S : S \in \Sigma)$ be a partition of $D_\Sigma$ such that, for each $D_S$:
% \begin{enumerate}
% \item $D_S$ is infinite, and
% \item $(D,\val) \models  \tau^a_S(d)$, for every $d \in D_S$. 
% \end{enumerate}
% We modify $\val_1$ by stating that $\val_2|_{ \vlist{d}}=\val_1|_{\vlist{d}}$, while over $d' \in D\setminus \vlist{d},$ $\val_2$ is defined as follows. If $d' \in D_\Sigma$, then there is an unique $S\in \Sigma$ such that $d' \in D_S$. Thence
% \[
% \val_2(d')= \begin{cases}
% \val(d') & \text{if } a \in S \\
% \val(d') \setminus \{a\} & \text{else.} 
% \end{cases}
% \]
% If $d' \in D_\Pi$, then let $S_1, \dots, S_\ell$ be the list of all types in $\Pi$ such that  $(D,\val) \models  \tau^a_{S_k}(d')$. Thus
% \[
% \val_2(d')= \begin{cases}
% \val(d') & \text{if } a \in S_1 \cup \dots \cup S_\ell \\
% \val(d') \setminus \{a\} & \text{else.} 
% \end{cases}
% \]
% By construction, we have that the following two facts hold:
%  \begin{itemize}
% \item $(D, \val_2) \models \bigvee_{S\in \Pi} \tau_S(d)$,
%  for each $d \in D_\Pi$, 
%  \item $(D, \val_2) \models \tau_S(d)$, for each $S \in \Sigma$ and $d \in D_S$. 
% \end{itemize}
% In particular, this means that 
% $(D, \val_2) \models \bigvee_{S\in \Sigma\cup\Pi} \tau_S(d),
% \text{ for each }d \in D_\Sigma \cup D_\Pi.
% $
% Thus, since $D_\Pi$ was chosen to be finite and each $D_S$ is infinite, for $S \in \Sigma$,
% \begin{equation}\label{eq:3}
% (D\setminus \vlist{d}, \val_2) \models \forall z.\bigvee_{S\in \Sigma\cup\Pi} \tau_S(z)\land
% \bigwedge_{S\in\Sigma} \qu y.\tau_S(y) \land \dqu y.\bigvee_{S\in\Sigma} \tau_S(y).
% \end{equation}
%  By (\ref{eq:start}) and (\ref{eq:3}) and the fact that $\val_2|_{ \vlist{d}}=\val_1|_{\vlist{d}}$, we have that $(D, \val_2) \models \varphi$.
% Now, notice that $\val_2(a) \subseteq \val(a)$ and $\val_1(a)=\val(a)$, for each $a \in A\setminus \{a\}$. By monotonicity of $\varphi$ in $a$ we can conclude that $(D, V) \models \varphi$.
\end{proof}

Putting together the above lemmas we obtain Theorem~\ref{thm:ofoeimonot}. Moreover, a careful analysis of the translation gives us the following corollary, providing normal forms for the monotone fragment of $\ofoei$.

\begin{corollary}\label{cor:ofoeipositivenf}
	Let $\varphi \in \ofoei(A)$, the following hold:
	\begin{enumerate}[(i)]
		\item The formula $\varphi$ is monotone in $A' \subseteq A$ iff it is equivalent to a formula $\bigvee \mondbnfofoei{\vlist{T}}{\Pi}{\Sigma}{A'}$ for $\Sigma\subseteq\Pi \subseteq \wp A$ and $\vlist{T} \in \wp(A)^k$ for some $k$.
		%
		\item The formula $\varphi$ is monotone in every $a\in A$ (i.e., $\varphi\in{\ofoei}^+(A)$) iff it is equivalent to a formula in the basic form $\bigvee \posdbnfofoei{\vlist{T}}{\Pi}{\Sigma}$ for types $\Sigma\subseteq\Pi \subseteq \wp A$ and $\vlist{T} \in \wp(A)^k$ for some $k$.
		% , where
		% %
		% %For the translation, let
		% %$(\bigvee \dbnfofo{\Sigma})^\tmono:= \bigvee \mondbnfofo{\Sigma}{a}$ and
		% %
		% \begin{align*}
		% 	\posdbnfofoei{\vlist{T}}{\Pi}{\Sigma} :=\ & \mondbnfofoe{\vlist{T}}{\Pi \cup \Sigma}{+} \land \posdbnfinf{\Sigma} \\
		% 	%
		% 	\posdbnfofoe{\vlist{T}}{\Lambda} :=\ & \exists \vlist{x}.\big(\arediff{\vlist{x}} \land \bigwedge_i \tau^+_{T_i}(x_i) \land \forall z.(\arediff{\vlist{x},z} \lthen \bigvee_{S\in \Lambda} \tau^+_S(z))\big) \\
		% 	%
		% 	\posdbnfinf{\Sigma} :=\ & \bigwedge_{S\in\Sigma} \qu y.\tau^+_S(y) \land \dqu y.\bigvee_{S\in\Sigma} \tau^+_S(y) .
		% \end{align*}
		% %for some types $\Pi,\Sigma \subseteq \wp A$ and $T_i \subseteq A$.
	\end{enumerate}
	%
\end{corollary}
\begin{proof}
	We only remark that to obtain $\Sigma\subseteq\Pi$ in the above normal forms it is enough to use Proposition~\ref{prop:bfofoei-sigmapi} before applying the translation.
\end{proof}

\subsection{One-step continuity}\label{subsec:one-stepcont}
% !TEX root = ../main.tex

Recall from Chapter~\ref{chap:sub} that a formula $\varphi \in \llang(A)$ is continuous in $\{\vlist{a}\} \subseteq A$ if $\varphi$ is monotone in $\vlist{a}$ and additionally, for every $(D,\val)$ and assignment $\ass:\fovar\to D$,
\[
\text{if } (D,\val),\ass \models \varphi \text{ then } \exists \vlist{U} \subseteq_\omega \val(\vlist{a}) \text{ such that } (D, \val[\vlist{a} \mapsto \vlist{U}]),\ass \models \varphi.
\]

\begin{remark}\label{rem:contprodeach}
	It was proved in Proposition~\ref{prop:contiffpcont} that continuity in the product coincides with continuity in every variable. Therefore, in the following proofs we will, in general, consider continuity in every single $a_i$ instead of in the full $\vlist{a}$. This is equivalent, and only done to avoid an even more complex notation.
\end{remark}

In this section we will characterize the continuous fragment of $\ofo$ and $\ofoei$ but we will not characterize that of $\ofoe$, since it is not used in this dissertation.
% It will be useful to give a syntactic characterization of continuity for several one-step logics.

%%
\subsubsection{Continuous fragment of $\ofo$}

\index{fragment!continuous!$\ofo$}
\index{$\ofo$!$\cont{}{A'}$}
\begin{theorem}\label{thm:ofocont}
A formula of $\ofo(A)$ is continuous in $A' \subseteq A$ iff it is equivalent to a sentence given by:
\[
\varphi ::= \psi \mid a(x) \mid \exists x.\varphi \mid \varphi \land \varphi \mid \varphi \lor \varphi
\]
where $a\in A'$ and $\psi \in \ofo(A\setminus A')$. We denote this fragment as $\cont{\ofo}{A'}(A)$.
\end{theorem}
%
The theorem will follow from the next two lemmas and Remark~\ref{rem:contprodeach}.

\begin{lemma}\label{lem:cofoiscont}
Every $\varphi \in \cont{\ofo}{a}(A)$ is continuous in $a$.
\end{lemma}
\begin{proof}
First observe that $\varphi$ is monotone in $a$ by Theorem~\ref{thm:ofomonot}.
We show, by induction, that any one-step formula $\varphi$ in the fragment (which may not be a sentence) satisfies, for every one-step model $(D,\val)$, assignment ${\ass:\fovar\to D}$,
%
\[
\text{if } (D,\val),\ass \models \varphi \text{ then } \exists U \subseteq_\omega \val(a) \text{ such that } (D,\val[a\mapsto U]),\ass \models \varphi.
\]
%
\begin{enumerate}[\textbullet]
\item If ${\varphi = \psi \in \ofo(A\setminus \{a\})}$ changes in the $a$ part of the valuation will make no difference and hence the condition is trivial. 

\item Case $\varphi = a(x)$: if $(D,\val),\ass \models a(x)$ then $\ass(x)\in \val(a)$. Clearly, $\ass(x) \in \val[a\mapsto \{\ass(x)\}](a)$ and hence $(D, \val[a\mapsto \{\ass(x)\}]),\ass \models a(x)$. %For the other direction assume $U \subseteq_\omega \val(a)$ and $(D, \val[a\mapsto U]),\ass \models a(x)$. This means that $\ass(x) \in U \subseteq \val(a)$, hence $(D, \val),\ass \models a(x)$.

\item Case $\varphi = \varphi_1 \lor \varphi_2$: assume $(D,\val),\ass \models \varphi$. Without loss of generality we can assume that $(D,\val),\ass \models \varphi_1$ and hence by induction hypothesis we have that there is $U \subseteq_\omega \val(a)$ such that $(D,\val[a\mapsto U]),\ass \models \varphi_1$ which clearly implies $(D,\val[a\mapsto U]),\ass \models \varphi$. %For the other direction let $U \subseteq_\omega \val(a)$ and assume wlog that $(D,\val[a\mapsto U]),\ass \models \varphi_1$. By induction hypothesis $(D,\val),\ass \models \varphi_1$ which entails $(D,\val),\ass \models \varphi$.

\item Case $\varphi = \varphi_1 \land \varphi_2$: assume $(D,\val),\ass \models \varphi$. By induction hypothesis we have $U_1,U_2 \subseteq_\omega \val(a)$ such that $(D,\val[a\mapsto U_1]),\ass \models \varphi_1$ and $(D,\val[a\mapsto U_2]),\ass \models \varphi_2$. By monotonicity this also holds with $\val[a\mapsto U_1 \cup U_2]$ and therefore $(D,\val[a\mapsto U_1 \cup U_2]),\ass \models \varphi$. %The other direction is very similar to the case of disjunction.

\item Case $\varphi = \exists x.\varphi'(x)$ and $(D,\val),\ass \models \varphi$. By definition there exists $d\in D$ such that $(D,\val),\ass[x\mapsto d] \models \varphi'(x)$. By induction hypothesis there exists $U \subseteq_\omega \val(a)$ such that $(D,\val[a\mapsto U]),\ass[x\mapsto d] \models \varphi'(x)$ and hence $(D,\val[a\mapsto U]),\ass \models \exists x.\varphi'(x)$.
% \item Case $\varphi = \exists x.\varphi'(x)$ and $(D,\val),\ass \models \varphi$. This occurs iff there exists $d\in D$ such that $(D,\val),\ass[x\mapsto d] \models \varphi'(x)$. By induction hypothesis this is equivalent to $\exists U \subseteq_\omega \val(a)$ such that $(D,\val[a\mapsto U]),\ass[x\mapsto d] \models \varphi'(x)$ which holds iff $(D,\val[a\mapsto U]),\ass \models \exists x.\varphi'(x)$.
\end{enumerate}
This finishes the proof.
\end{proof}

\begin{lemma}
There is a translation $(-)^\tcont:\monot{\ofo}{a}(A) \to \cont{\ofo}{a}(A)$ such that
a formula ${\varphi \in \monot{\ofo}{a}(A)}$ is continuous in $a$ if and only if $\varphi\equiv \varphi^\tcont$.
\end{lemma}
\begin{proof}
To define the translation we assume, without loss of generality, that $\varphi$ is in the basic form $\bigvee \mondbnfofo{\Sigma}{a}$.
% where
% %
% \[
% \mondbnfofo{\Sigma}{a} := \bigwedge_{S\in\Sigma} \exists x. \tau^a_S(x) \land \forall x. \bigvee_{S\in\Sigma} \tau^a_S(x).
% \]
%
For the translation, let
$(\bigvee \mondbnfofo{\Sigma}{a})^\tcont := \bigvee \mondgbnfofo{\Sigma}{\Sigma^{-}_{a}}{a}$
where
%
% \[
% \mondgbnfofo{\Sigma}{\Pi}{a} := \bigwedge_{S\in\Sigma} \exists x. \tau^a_S(x) \land \forall x. \bigvee_{S\in\Pi} \tau^a_S(x)
% \]
%
% and
$\Sigma^{-}_{a} := \{S\in \Sigma \mid a\notin S\}$.
%Intuitively, the translation says that we should be able to divide the description given by $\dbnfofo{\Sigma}$ in two parts: (1) $\nabla^{-}_p(\Gamma)$ gives a complete (existential and universal) description with respect to colours which are not $a$ and (2) an existential description where $a$ can only appear as a positive constraint.

\bigskip
From the construction it is clear that $\varphi^\tcont \in \cont{\ofo}{a}(A)$ and therefore the right-to-left direction of the lemma is immediate by Lemma~\ref{lem:cofoiscont}. For the left-to-right direction assume that $\varphi$ is continuous in $a$, we have to prove that $(D,\val) \models \varphi$ iff $(D,\val) \models \varphi^\tcont$, for every one-step model $(D,\val)$. We will take a slightly different but equivalent approach.

It is easy to prove that $(D,\val) \equiv_\fo (D\times \omega,\val_\pi)$ where $D\times\omega$ has countably many copies of each element in $D$ and $\val_\pi(a) := \{(d,k) \mid d\in \val(a), k\in\omega\}$.
%
Moreover, as $\varphi$ is continuous in $a$ there is $U \subseteq_\omega \val_\pi(a)$ such that $\val'_\pi := \val[a \mapsto U]$ satisfies $(D\times\omega,\val_\pi) \models \varphi$ iff $(D\times\omega,\val'_\pi) \models \varphi$.
%
Therefore, it is enough to prove that $(D\times\omega,\val'_\pi) \models \varphi$ iff $(D\times\omega,\val'_\pi) \models \varphi^\tcont$. % Assume again that $\varphi = \bigvee \dbnfofo{\Sigma}$.

\bigskip
\noindent \fbox{$\Rightarrow$}
Let $(D\times\omega,\val'_\pi) \models \mondbnfofo{\Sigma}{a}$, we show that $(D\times\omega,\val'_\pi) \models \mondgbnfofo{\Sigma}{\Sigma^{-}_{a}}{a}$. The existential part of $\mondgbnfofo{\Sigma}{\Sigma^{-}_{a}}{a}$ is trivially true. We have to show that every element of $(D\times\omega,\val'_\pi)$ realizes an $a$-positive type in $\Sigma^{-}_{a}$. Take $(d,k) \in D\times\omega$ and let $T$ be the (full) type of $(d,k)$. If $a\notin T$ then trivially $T\in \Sigma^{-}_{a}$ and we are done. Suppose $a\in T$. Observe that in $D\times\omega$ we have infinitely many copies of $d\in D$. However, as $\val'_\pi(a)$ is finite, there must be some $(d,k')$ with type $T' := T\setminus\{a\}$.
%\fcwarning{Write better}
For $\mondbnfofo{\Sigma}{a}$ to be true we must have $T'\in \Sigma$ and hence $T'\in \Sigma^{-}_{a}$. It is easy to see that $(d,k)$ realizes the $a$-positive type $T'$.

\bigskip
\noindent \fbox{$\Leftarrow$}
Let $(D\times\omega,\val'_\pi) \models \mondgbnfofo{\Sigma}{\Sigma^{-}_{a}}{a}$, we show that $(D\times\omega,\val'_\pi) \models \mondbnfofo{\Sigma}{a}$. The existential part is trivial. For the universal part just observe that $\Sigma^{-}_{a} \subseteq \Sigma$.
\end{proof}

Putting together the above lemmas we obtain Theorem~\ref{thm:ofocont}. Moreover, a careful analysis of the translation gives us the following corollary, providing normal forms for the continuous fragment of $\ofo$.

\begin{corollary}\label{cor:ofocontinuousnf}
	Let $\varphi \in \ofo(A)$, the following hold:
	\begin{enumerate}[(i)]
		\item The formula $\varphi$ is continuous in $a \in A$ iff it is equivalent to a formula $\bigvee \mondgbnfofo{\Sigma}{\Sigma^{-}_{a}}{a}$ for some types $\Sigma \subseteq \wp A$, where $\Sigma^{-}_{a} := \{S\in \Sigma \mid a\notin S\}$.
		%
		\item If $\varphi$ is monotone in $A$ (i.e., $\varphi\in\ofo^+(A)$) then $\varphi$ is continuous in $a \in A$ iff it is equivalent to a formula in the basic form $\bigvee \posdgbnfofo{\Sigma}{\Sigma^{-}_{a}}$ for some types $\Sigma \subseteq \wp A$, where $\Sigma^{-}_{a} := \{S\in \Sigma \mid a\notin S\}$.
	\end{enumerate}
\end{corollary}

%%
\subsubsection{Continuous fragment of $\ofoei$}

\index{fragment!continuous!$\ofoei$}
\index{$\ofoei$!$\cont{}{A'}$}
\begin{theorem}\label{thm:ofoeicont}
A formula of $\ofoei(A)$ is continuous in $ A' \subseteq A$ iff it is equivalent to a sentence given by:
\[
\varphi ::= \psi \mid a(x) \mid \exists x.\varphi \mid \varphi \land \varphi \mid \varphi \lor \varphi \mid \wqu x.(\varphi,\psi)
\]
where $a\in A'$ and $\psi \in \ofoei(A\setminus A')$. Recall from Definition~\ref{def:contmufoei} that $\wqu x.(\varphi,\psi)$ is defined as $\forall x.(\varphi(x) \lor \psi(x)) \land \dqu x.\psi(x)$. We denote this fragment as $\cont{\ofoei}{A'}(A)$.
\end{theorem}

\noindent The theorem will follow from the next two lemmas and Remark~\ref{rem:contprodeach}.

\begin{lemma}\label{lem:cofoeiiscont}
Every $\varphi \in \cont{\ofoei}{a}(A)$ is continuous in $a$.
\end{lemma}
\begin{proof}
Observe that monotonicity of $\varphi$ is guaranteed by Theorem~\ref{thm:ofoeimonot}.
We show, by induction, that any formula of the fragment (which may not be a sentence) satisfies, for every one-step model $(D,\val)$ and assignment ${\ass:\fovar\to D}$,
%
\[
\text{if } (D,\val),\ass \models \varphi \text{ then } \exists U \subseteq_\omega \val(a) \text{ such that } (D,\val[a\mapsto U]),\ass \models \varphi.
\]
%
We focus on the inductive case of the new quantifier. Let $\varphi' = \wqu x.(\varphi,\psi)$, i.e., %In other words,
%
\[\varphi' = \forall x.\underbrace{(\varphi(x) \lor \psi(x))}_{\alpha(x)} \land \underbrace{\dqu x.\psi(x)}_\beta.\]
%
Let $(D,\val),\ass \models \varphi'$. By induction hypothesis,
for every $\ass_d := \ass[x\mapsto d]$ which satisfies $(D,\val),\ass_d \models \alpha(x)$ there is $U_d \subseteq_\omega \val(a)$ such that $(D,\val[a\mapsto U_d]),\ass_d \models \alpha(x)$. The crucial observation is that because of $\beta$, %part of $\varphi'$ we know that
only finitely many elements of $d$ make $\psi(d)$ false. Let $U := \bigcup \{U_d \mid (D,\val), \ass_d \not\models \psi(x) \}$. Note that $U$ is a finite union of finite sets, hence finite.
%
\begin{claimfirst}
	Let $\val_U := \val[a\mapsto U]$; then we have $(D,\val_U),\ass \models \varphi'$.
\end{claimfirst}
%
\begin{pfclaim}
	It is clear that $(D,\val_U),\ass \models \beta$ because $\psi$ is $a$-free. To show that the first conjunct is true we have to show that $(D,\val_U),\ass_d \models \varphi(x) \lor \psi(x)$ for every $d\in D$. We consider two cases: (i) if $(D,\val),\ass_d \models \psi(x)$ we are done, again because $\psi$ is $a$-free; (ii) if the former is not the case then $U_d \subseteq U$; moreover, we knew that $(D,\val[a\mapsto U_d]),\ass_d \models \alpha(x)$ and by monotonicity of $\alpha(x)$ we can conclude that $(D,\val_U),\ass_d \models \alpha(x)$.
\end{pfclaim}
%
This finishes the proof of the lemma.
\end{proof}



\begin{lemma}\label{lem:ofoeictrans}
	There is a translation $(-)^\tcont:\monot{\ofoei}{a}(A) \to \cont{\ofoei}{a}(A)$ such that
a formula $\varphi \in \monot{\ofoei}{a}(A)$ is continuous in $a$ if and only if $\varphi\equiv \varphi^\tcont$.
\end{lemma}
\begin{proof} We assume that $\varphi$ is in basic normal form, i.e., $\varphi = \bigvee \mondbnfofoei{\vlist{T}}{\Pi}{\Sigma}{a}$.
% where
% \[
% \mondbnfofoei{\vlist{T}}{\Pi}{\Sigma}{a} = \mondbnfoe{\vlist{T}}{\Pi \cup \Sigma}{a} \land
% \bigwedge_{S\in\Sigma} \qu y.\tau^a_S(y) \land \dqu y.\bigvee_{S\in\Sigma} \tau^a_S(y) .
% \]
For the translation let $(\bigvee \mondbnfofoei{\vlist{T}}{\Pi}{\Sigma}{a})^\tcont := \bigvee \mondbnfofoei{\vlist{T}}{\Pi}{\Sigma}{a}^\tcont$ where
\[
\mondbnfofoei{\vlist{T}}{\Pi}{\Sigma}{a}^\tcont :=
\begin{cases}
	\bot &\text{ if } a\in \bigcup \Sigma\\
	\mondbnfofoei{\vlist{T}}{\Pi}{\Sigma}{a} &\text{ otherwise}.
\end{cases}
\]

First we prove the right-to-left direction of the lemma. By Lemma~\ref{lem:cofoeiiscont} it is enough to show that $\varphi^\tcont \in \cont{\ofoei}{a}(A)$. We focus on the disjuncts of $\varphi^\tcont$. The interesting case is when $a\notin \bigcup \Sigma$. If we rearrange $\mondbnfofoei{\vlist{T}}{\Pi}{\Sigma}{a}$ and define the formulas $\varphi', \psi$ as follows:
%
\begin{align*}
\mondbnfofoei{\vlist{T}}{\Pi}{\Sigma}{a} \equiv \exists \vlist{x}.\Big(& \arediff{\vlist{x}} \land \bigwedge_i \tau^a_{T_i}(x_i)\ \land \\
& \forall z.(\underbrace{\lnot\arediff{\vlist{x},z} \lor \bigvee_{S\in \Pi} \tau^a_S(z)}_{\varphi'(\vlist{x},z)} \lor \underbrace{\bigvee_{S\in \Sigma} \tau^a_S(z)}_{\psi(z)})\ \land \\
& \dqu y.\underbrace{\bigvee_{S\in\Sigma} \tau^a_S(y)}_{\psi(y)} \Big) \land \bigwedge_{S\in\Sigma} \qu y.\tau^a_S(y),
\end{align*}
%
then we get that
{\small
\[
\mondbnfofoei{\vlist{T}}{\Pi}{\Sigma}{a} \equiv \exists \vlist{x}.\Big(\arediff{\vlist{x}} \land \bigwedge_i \tau^a_{T_i}(x_i) \land \wqu z.(\varphi'(\vlist{x},z),\psi(z)) \Big) \land \bigwedge_{S\in\Sigma} \qu y.\tau^a_S(y)
\]}
%
which, because $a\notin \bigcup \Sigma$, is in the required fragment.

For the left-to-right direction of the lemma we have to prove that $\varphi \equiv \varphi^\tcont$.

\bigskip
\noindent\fbox{$\Leftarrow$} Let $(D,\val) \models \varphi^\tcont$. The only difference between $\varphi$ and $\varphi^\tcont$ is that some disjuncts may have been replaced by $\bot$. Therefore this direction is trivial.

\bigskip
\noindent\fbox{$\Rightarrow$} Let $(D,\val) \models \varphi$. Because $\varphi$ is continuous in $a$ we may assume that $\val(a)$ is finite. Let $\mondbnfofoei{\vlist{T}}{\Pi}{\Sigma}{a}$ be a disjunct of $\varphi$ such that $(D,\val) \models \mondbnfofoei{\vlist{T}}{\Pi}{\Sigma}{a}$. If $a \notin \bigcup\Sigma$ we trivially conclude that $(D,\val) \models \varphi^\tcont$ because the disjunct remains unchanged. Suppose now that $a\in \bigcup\Sigma$, then there must be some $S\in\Sigma$ with $a\in S$. Because $(D,\val) \models \mondbnfofoei{\vlist{T}}{\Pi}{\Sigma}{a}$ we have, in particular, that $(D,\val) \models \qu y.\tau^a_S(x)$ and hence $\val(a)$ must be infinite which is absurd.
\end{proof}

Putting together the above lemmas we obtain Theorem~\ref{thm:ofoeicont}. Moreover, a careful analysis of the translation gives us the following corollary, providing normal forms for the continuous fragment of $\ofoei$.

\begin{corollary}\label{cor:ofoeicontinuousnf}
Let $\varphi \in \ofoei(A)$, the following hold:
	\begin{enumerate}[(i)]
		\item The formula $\varphi$ is continuous in $a \in A$ iff it is equivalent to a formula in the basic form $\bigvee \mondbnfofoei{\vlist{T}}{\Pi}{\Sigma}{a}$ for some types $\Sigma\subseteq\Pi \subseteq \wp A$ and $T_i \subseteq A$ such that $a\notin \bigcup\Sigma$.
		%
		\item If $\varphi$ is monotone in every element of $A$ (i.e., $\varphi\in{\ofoei}^+(A)$) then $\varphi$ is continuous in $a \in A$ iff it is equivalent to a formula in the basic form $\bigvee \posdbnfofoei{\vlist{T}}{\Pi}{\Sigma}$ for some types $\Sigma\subseteq\Pi \subseteq \wp A$ and $T_i \subseteq A$ such that $a\notin \bigcup\Sigma$.
	\end{enumerate}
\end{corollary}
\begin{proof}
	We only remark that to obtain $\Sigma\subseteq\Pi$ in the above normal forms it is enough to use Proposition~\ref{prop:bfofoei-sigmapi} before applying the translation.
\end{proof}

\subsection{Dual fragments}\label{subsec:one-stepduals}
In this subsection we
%recall the notions of co-continuity and complete multiplicativity, which are dual to continuity and complete additivity, respectively. Next, we
give syntactic characterizations of the co-continuous  fragment of several one-step logics. This notion is dual to continuity.
%
% \bigskip
% Recall from Definition~\ref{def:os-continuity} that a formula $\varphi\in\llang(A)$ is \emph{continuous in $a\in A$} if $\varphi$ is monotone in $a$ and additionally, for every $(D,\val)$ and assignment $\ass:\fovar\to D$,
% \[
% \text{if } (D,\val),\ass \models \varphi \text{ then } \exists U \subseteq_\omega \val(a) \text{ such that } (D, \val[a \mapsto U]),\ass \models \varphi.
% \]
% %
% We say that $\varphi$ is \emph{co-continuous in $a\in A$} if the Boolean dual $\dual{\varphi}$ of $\varphi$ (\emph{cf.}~Definition~\ref{d:bdual1}) is continuous in $a\in A$.
%
%To define syntactic fragments for the dual notions
We first give a concrete definition of the dualisation operator of Definition~\ref{d:bdual1}.% and then show that the one-step language $\ofoei$ is closed under Boolean duals.

\index{dual!$\ofoei$}
\begin{definition}\label{def:concreteduals} 
The \emph{dual} $\varphi^{\delta} \in {\ofoei}(A)$ of $\varphi\in {\ofoei}(A)$ is given by:
\begin{align*}
 (a(x))^{\delta} & :=  a(x) 
 & (\lnot a(x))^{\delta} & :=  \lnot a(x) 
\\ (\top)^{\delta} & :=  \bot 
  & (\bot)^{\delta} & :=  \top 
\\  (x \approx y)^{\delta} & :=  x \not\approx y 
  & (x \not\approx y)^{\delta}& :=  x \approx y 
\\ (\varphi \wedge \psi)^{\delta} &:=  \varphi^{\delta} \vee \psi^{\delta} 
  &(\varphi \vee \psi)^{\delta}& :=  \varphi^{\delta} \wedge \psi^{\delta}
\\ (\exists x.\psi)^{\delta} &:=  \forall x.\psi^{\delta} 
  &(\forall x.\psi)^{\delta} &:=  \exists x.\psi^{\delta} 
\\ (\exists^{\infty} x.\psi)^{\delta} &:= \forall^{\infty} x.\psi^{\delta} 
  &(\forall^{\infty} x.\psi)^{\delta} &:=  \exists^{\infty} x.\psi^{\delta}
\end{align*}
\end{definition}

\begin{remark}
	Observe that if $\varphi \in \llang(A)$ for $\llang\in\{\ofo,\ofoe,\ofoei\}$ then $\varphi^{\delta} \in \llang(A)$. Moreover, the operator preserves positivity of the predicates, that is, if $\varphi \in \llang^+(A)$ then $\varphi^{\delta} \in \llang^+(A)$.
\end{remark}

\noindent The proof of the following proposition is a routine check.

\begin{proposition}\label{props:duals}
For every $\varphi \in \ofoei(A)$, $\varphi$ and $\varphi^{\delta}$ are Boolean duals.
\end{proposition}

We are now ready to give the syntactic definition of the dual fragments for the one-step logics into consideration.

\begin{definition}\label{def:cocontfrag}\label{def:multfrag}
The fragment $\cocont{\ofoei}{A'}(A)$ is given by the sentences generated by:
\[
\varphi ::= \psi \mid a(x) \mid \forall x.\varphi \mid \dqu x.\varphi \mid \varphi \lor \varphi \mid \varphi \land \varphi
\]
where $a\in A'$ and $\psi \in \ofoei(A\setminus A')$. Observe that the equality is included in $\psi$. The fragment $\cocont{\ofo}{A'}(A)$ is defined as $\cocont{\ofoei}{A'}(A)$ but without the clause for $\dqu$ and with $\psi\in \ofo(A\setminus A')$.
\end{definition}

The following proposition states that the above fragments are actually the duals of the fragments defined earlier in this chapter.

\begin{proposition}\label{prop:newfragsduals}
The following hold:
	%
	\begin{align*}
		\cocont{\ofoei}{A'}(A) &= \{\varphi \mid \varphi^\delta \in \cont{\ofoei}{A'}(A)\} \\
		\cocont{\ofo}{A'}(A) &= \{\varphi \mid \varphi^\delta \in \cont{\ofo}{A'}(A)\}.
	\end{align*}
	%
\end{proposition}
\begin{proof}
	Easily proved by induction.
\end{proof}

\noindent As a corollary, we get a characterisation for co-continuity.

\begin{corollary}~
	Let $\llang \in \{\ofo,\ofoei\}$. A formula $\varphi \in \llang(A)$ is co-continuous in $a\in A$ if and only if it is equivalent to some $\varphi' \in \cocont{\llang}{a}(A)$.
\end{corollary}
\begin{proof}
	Consequence of Proposition~\ref{prop:newfragsduals} and~\ref{props:duals}.
\end{proof}


\clearpage

%%%%
%%%% AUTOMATA
%%%%

\section{Parity automata}\label{sec:parityaut}
%!TEX root = ../00CFVZ_TOCL.tex

We recall the definition of a parity automaton, adapted to our setting.
Recall from Section~\ref{sec:onestep} that formulas of an arbitrary one-step language $\llang$ are interpreted over  one-step models, that is, a tuple $\osmodel = (D,\val: A \to \wp D)$. %Whenever we say `one-step model' in this section we will be referring to \emph{single-sorted} one-step models. 
Recall that the class of all one-step models is denoted by $\umods$ and that we write $\llang^+(A)$ to denote the fragment where every predicate $a\in A$ occurs only positively.
Without loss of generality, from now on we always assume that every bound variable occurring in a sentence is bound by an unique quantifier (generalised or not).




\begin{definition}[Parity Automata] \label{def:partityaut}
A \emph{parity automaton} based on the one-step language $\llang$ and 
alphabet $\wp(\props)$ is a tuple $\aut = \tup{A,\tmap,\pmap,a_I}$ such that $A$ is a
finite set of states, $a_I \in A$ is the initial state,
$\tmap: A\times \wp(\props) \to \llang^+(A)$
is the transition map, and $\pmap: A \to \nat$ is the parity map.
The class of such automata will be denoted by $\Aut(\llang)$.

Acceptance of a $\props$-transition
system $\model = \tup{\moddom,R,\tscolors,s_I}$ by $\aut$ is determined by the \emph{acceptance game}
$\agame(\aut,\model)$ of $\aut$ on $\model$. This is the parity game defined
according to the rules of the following table.
%
% \begin{table*}[ht]
%   \centering
\begin{center}
\small
\begin{tabular}{|l|c|l|c|} \hline
Position & Pl'r & Admissible moves & Parity \\
\hline
    $(a,s) \in A \times \moddom$
  & $\eloise$
  & $\{\val : A \to \wp(R[s]) \mid (R[s],\val) \models \tmap (a, \tscolors(s)) \}$
  & $\pmap(a)$ 
\\
%     $(a,s) \in A \times \moddom$
%   & $\abelard$
%   & $\{(a,s,\aact) \in A \times \moddom  \mid \aact \in \acts\}$
%   & $\pmap(a)$ 
% \\
%     $(a,s,\aact) \in A \times \moddom $
%   & $\eloise$
%   & $\{\val : A \to \wp(R_\aact[s]) \mid (R_\aact[s],\val) \models \tmap (a, \tscolors(s), \aact) \}$
%   & $\pmap(a)$ 
% \\
    $\val : A \rightarrow \wp(\moddom)$
  & $\abelard$
  & $\{(b,t) \mid t \in \val(b)\}$
  & $\max(\pmap[A])$
\\ \hline
 \end{tabular}
\end{center}
%
A transition system $\model$ is \emph{accepted} by $\aut$ if $\exists$ has
a winning strategy in $\agame(\aut,\model)@(a_I,s_I)$, and \emph{rejected}
if $(a_I,s_I)$ is a winning position for $\abelard$. We write $\trees(\aut)$ for the class of trees (also called the \emph{tree language}) recognised by $\bbA$.
\end{definition}


The automata-theoretic characterisation of $\wmso$ and $\nmso$ will use classes of parity automata constrained by two additional properties: weakness
and continuity. 

\begin{definition}[Weakness, Continuity]
\label{def:weak}
Let $\llang$ be a one-step language, and let $\bbA = \tup{A,\tmap,\pmap,a_I}$
be in $\Aut(\llang)$. Write $\ord$ for the reachability relation in $\bbA$, i.e. the reflexive-transitive closure of $\{ (a,b) \mid \exists c. b \text{ occurs in }\tmap(a,c)\}$. A \emph{strongly connected $\ord$-component} ($\ord$-SCC) is a subset $M\subseteq A$ such that, for every $a,b \in M$ we have $a \ord b$ and $b \ord c$. The SCC is called \emph{maximal} (MSCC) when $M\cup\{a\}$ ceases to be a SCC for any choice of $a \in A\setminus M$.

%Given the semantics of the one-step language $\llang$, the (semantic) notion of (co-)continuity applies to one-step formulas (see for instance section~\ref{subsec:one-stepcont}). 

We formulate two requirements on automata from $\Aut(\llang)$:
\begin{description}
\item[(weakness)] if $a \ord b$ and $b \ord a$ then $\pmap(a) = \pmap(b)$.
\item[(continuity)] let $a,b$ be states such that both $a\ord b$ and
$b \ord a$, and let $c\in C$;
    if ${\pmap(b)}=1$ then $\tmap(b,c)$ is continuous in $a$.
    If $\pmap(b)=0$, then $\tmap(b,c)$ is co-continuous in $a$.
\end{description}
A \emph{weak parity automaton} is an automaton $\aut \in \Aut(\llang)$ additionally satisfying the \emph{(weakness)}. We call it \emph{continuous-weak} if it additionally satisfies the \emph{(continuity)} condition. We let $\AutW(\llang)$ denote the class of weak and $\AutWC(\llang)$  the class of continuous-weak automata.
\end{definition}


Intuitively, weakness forbids an automaton to register non-trivial properties concerning the vertical `dimension' of input trees, whereas continuity expresses a constraint on how much of the horizontal `dimension' of an input tree the automaton is allowed to process. In terms of second-order logic, they correspond respectively to quantification over `vertically' finite (i.e. included in well-founded subtrees) and `horizontally' finite (i.e. included in finitely branching subtrees) sets. The conjunction of weakness and continuity thus corresponds to quantification over finite sets. 


\begin{remark}
Any weak parity automaton $\bbA$ is equivalent to a weak parity automaton
$\bbA'$ with $\pmap: A' \to \{0,1\}$. This is because \emph{(weakness)} prevents states of different parity to occur infinitely often in acceptance games, meaning that we may just assign $0$ to any even-parity and $1$ to any odd-parity state. From now on we assume such a restricted parity map for weak parity automata.
\end{remark}

Henceforth we call \emph{$\nmso$-automata} the members of the class $\AutW(\ofoe)$ and \emph{$\wmso$-automata} those of $\AutWC(\olque)$. The names are justified by the logic characterising these classes, see Theorem \ref{t:wmsoauto} below.

\begin{remark}
Observe that, contrarily to weakness, continuity is given above as a semantic condition. Thanks to Theorem XXX \fznote{theorem about syntactic char. for continuity}, we can work with a completely syntactic definition of $\wmso$-automata, the continuity condition above being equivalently replaced by the following. 
\begin{description}
	\itemsep 0 pt
	\item[(continuity, syntactically)] if $\pmap(a)$ is odd (resp. even) then, for each $c\in C$ we have
	   $\tmap(a,c) \in \cont{{\olque}^+}{b}(A)$ (resp. $\tmap(a,c) \in \cocont{{\olque}^+}{b}(A)$).
\end{description} 
\end{remark}

We are now able to state the main theorem of this section, which takes care of
one direction of Theorem~\ref{t:m1}. 

\begin{theorem}
\label{t:wmsoauto}
For $\sllang \in \{ \wmso, \nmso\}$, there is an effective construction transforming an $\sllang$-formula $\phi$
into an $\sllang$-automaton $\bbA_{\phi}$ that is equivalent
to $\phi$ on the class of trees.
That is, for any tree $\bbT$, $\bbA_{\phi}$ accepts $\bbT$ if and only if $\bbT \models {\phi}$
%\begin{enumerate}[(a)]
%\item There is an effective construction transforming a $\wmso$-formula $\phi$
%into a $\wmso$-automaton $\bbA_{\phi}$ that is equivalent
%to $\phi$ on the class of trees.
%That is, for any tree $\bbT$, $\bbA_{\phi}$ accepts $\bbT$ if and only if $\bbT \models {\phi}$
%\item There is an effective construction transforming a $\nmso$-formula $\phi$
%into a $\nmso$-automaton $\bbA_{\phi}$ that is equivalent
%to $\phi$ on the class of trees.
%That is, for any tree $\bbT$, $\bbA_{\phi}$ accepts $\bbT$ if and only if $\bbT \models {\phi}$
%\end{enumerate}
\end{theorem}

As usual, the proof of this theorem proceeds by induction on the complexity of
$\phi$. The inductive step requires that the class of automata is closed under the boolean operations and projection (noetherian for $\nmso$-automata, finite for $\wmso$-automata). Clearly, the latter closure property requires most of the work. The next we first
provide a simulation theorem that put $\wmso$-automata in a suitable shape
for the projection construction.

%%%



\section{Automata Characterisation of $\wmso$}\label{sec:autwmso}

%!TEX root = ../00CFVZ_TOCL.tex


In this subsection we work with the members of the class $\AutWC(\olque)$, which we henceforth call \emph{$\wmso$-automata}. Whereas continuity has been abstractly formulated as a \emph{semantic} condition, thanks to Theorem XXX \fznote{theorem about syntactic char. for continuity} we can work with a completely syntactic definition of $\wmso$-automata, \emph{(continuity)} for $\llang = \ofoe$ being equivalently to the following condition. 
\begin{description}
	\itemsep 0 pt
	\item[(continuity, syntactically)] if $\pmap(a)$ is odd (resp. even) then, for each $c\in C$ we have
	   $\tmap(a,c) \in \cont{{\olque}^+}{b}(A)$ (resp. $\tmap(a,c) \in \cocont{{\olque}^+}{b}(A)$).
\end{description} 

In the remainder of this subsection we prove the following result, yielding the direction from formulas to automata of the characterisation theorem for $\wmso$.

\begin{theorem}
\label{t:wmsoauto}
There is an effective construction transforming a $\wmso$-formula $\phi$
into a $\wmso$-automaton $\bbA_{\phi}$ that is equivalent
to $\phi$ on the class of trees.
That is, for any tree $\bbT$, $\bbA_{\phi}$ accepts $\bbT$ if and only if $\bbT \models {\phi}$.
\end{theorem}

The proof proceeds by induction on the complexity of
$\phi$. For the inductive steps, we will need to verify that the class of
$\wmso$-automata is closed under the boolean operations and finite projection.
The latter closure property requires most of the work: we devote Section \ref{sec:simulationwmso} to a simulation theorem that put $\wmso$-automata in a suitable shape
for the projection construction.
%
To this aim, it is convenient to define a closure operation on tree languages corresponding
to the semantics of $\wmso$ quantification. The inductive step of the proof of Theorem \ref{t:wmsoauto} will show that tree languages accepted by $\wmso$-automata are closed under this operation.

\begin{definition}\label{def:tree_finproj}
Fix a set $\prop$ of proposition letters, $p \not\in P$ and a language $\trees$ of $\p (\prop\cup\{p\})$-labeled
trees.
The \emph{finitary projection} of $\trees$ over $p$ is the language
${\finexists} p.\trees$ of $\p (\prop)$-labeled trees
given as follows:
%
$$
{\finexists} p.\trees = \{\bbT \mid \text{ $\exists$ a finite $p$-variant } \bbT' \text{ of } \bbT \text{ with } \bbT' \in \trees\}.
$$
%
A class $K$ of tree languages is \emph{closed under finitary projection
over $p$} if, for any language $\trees$ in $K$, also ${{\finexists} p}.\trees$ is in $K$.
\end{definition} 



\subsection{Simulation theorem for $\wmso$-automata}\label{sec:simulationwmso}

\noindent Our next goal is a \emph{projection construction} that, given
a $\wmso$-automaton $\aut$, provides one recognizing ${\finexists p}.\trees(\aut)$. For $\mso$-automata, an analogous construction crucially uses the following \emph{simulation theorem}: every
$\mso$-automaton $\aut$ is equivalent to a \emph{non-deterministic} one $\aut'$ \cite{Walukiewicz96}.
Semantically, non-determinism yields the appealing property that every node of the input model $\bbT$ is associated with at most one state of $\aut'$ during the acceptance game--- that means, $\eloise$'s strategy $f$ in $\agame(\aut',\bbT)$ is \emph{functional} (\emph{cf.} Definition \ref{def:StratfunctionalFinitary} below). This is particularly helpful because, to define a $p$-variant of $\bbT$
that is accepted by the projection construct on $\aut'$, we
can infer whether a node $s$ should be labeled with $p$ by the value $f(a,s)$, where $a$ is the unique state of $\aut'$ (by functionality) that $f$ associates with $s$. Now, in the case of $\wmso$-automata we are interested in guessing
\emph{finitary} $p$-variants, which requires $f$ to be functional only on a \emph{finite} set of nodes. Thus the idea of our simulation theorem is to turn a $\wmso$-automaton $\aut$ into an equivalent one $\aut^{\f}$ that behaves non-deterministically on a \emph{finite} portion of any accepted tree.

For $\mso$-automata, the simulation theorem is based on a powerset construction: if the starting automaton has carrier $A$, the resulting non-deterministic automaton is based on ``macro-states'' from the set $\shA := \pw (A \times A)$.\footnote{The use of carrier $\pw (A \times A)$ instead of the more obvious $\pw A$ is needed to correctly associate with a run on macro-states the corresponding bundle of runs of the original automaton $\aut$ (\emph{cf.} \cite{Walukiewicz96}).} Analogously, for $\wmso$-automata we will associate the non-deterministic behaviour with macro-states. However, as explained above, the automaton $\aut^{\f}$ that we construct has to be non-deterministic just on finitely many nodes of the input and may behave as $\aut$ (i.e. in ``alternating mode'') on the others. To this aim, $\aut^{\f}$ will be ``two-sorted'', roughly consisting of one copy of $\aut$ (with carrier $A$) and a variant of its powerset construction, based both on $A$ and $\shA$. For any accepted $\bbT$, the idea is to make any match $\pi$ of $\mc{A}(\aut^{\f},\bbT)$ consist of two parts:
\begin{description}
  \item[(\textbf{Non-deterministic mode})] for finitely many rounds $\pi$ is played on macro-states, i.e. positions are of the form $\shA \times T$. In her strategy player $\exists$ assigns macro-states (from $\shA$) only to \emph{finitely many} nodes, and states (from $A$) to the rest. Also, her strategy is functional in $\shA$, i.e. it assigns \emph{at most one macro-state} to each node.
  \item[(\textbf{Alternating mode})] At a certain round, $\pi$ abandons macro-states and turns into a match of the game $\mc{A}(\aut,\bbT)$, i.e. all next positions are from $A \times T$ (and are played according to a non-necessarily functional strategy). %of shape $(a,t) \in A \times T$.
\end{description}
Therefore successful runs of $\aut^{\noet}$ will have the property of processing only a \emph{finite} amount of the input with $\aut^{\noet}$ being in a macro-state and all the rest with $\aut^{\noet}$ behaving exactly as $\aut$. We now proceed in steps towards the construction of $\aut^{\noet}$. The following is a notion of lifting for types on states that is instrumental in defining a translation to types on macro-states. The distinction between empty and non-empty subsets of $A$ is to make sure that empty types on $A$ are lifted to empty types on $\pw A$.
\begin{definition}
Given a set $A$ and $\Sigma \subseteq \wp A$, we define the \emph{lifting} $\lift{\Sigma} \subseteq \wp \wp A$ as $\{\{S\} \mid S \in \Sigma \wedge S \neq \emptyset\} \cup
    \{\emptyset \mid \emptyset \in \Sigma \}$.
\end{definition}

Next we define a translation on the sentences associated with the
transition function of the original $\wmso$-automaton. Following the intuition given above, we want to work with sentences that can be made true by assigning macro-states (from $\shA$) to finitely many nodes in the model, and ordinary states (from $A$) to all the other nodes. Moreover, each node should be associated with \emph{at most one} macro-state, because of functionality. Corollary \ref{def:functionalsentenceofoe}.\ref{pt:ofoeifunctionalcontinuous} expresses these desiderata as \emph{functional continuity in $\shA$} and suggests the following syntactic shape in order to fulfil them.

\begin{definition}\label{DEF_finitary_lifting}
Let $\varphi \in {\olque}^+(A \times A)$ be a formula of shape $\posdbnfofoei{\vlist{T}}{\Pi}{\Sigma}$ for some $\Pi,\Sigma \subseteq \shA$ and $\vlist{T} = \{T_1,\dots,T_k\} \subseteq \shA$. Fix $\widetilde{\Sigma} \df \{\Ran(S) \mid S \in \Sigma\} \subseteq \wp A$. We define $\varphi^{\fin} \in {\olque}^+(A \cup \shA)$ as $\posdbnfofoei{\lift{\vlist{T}}}{\lift{\Pi} \cup \lift{\Sigma}}{\widetilde{\Sigma}}$, that means,
\begin{equation}\label{eq:unfoldingNablaolque}
\begin{aligned}
\varphi^{\fin} =\ &
    \exists \vlist{x}.\big(\arediff{\vlist{x}} \land \bigwedge_{0 \leq i \leq n} \tau^+_{\lift{T}_i}(x_i)
\land
    \forall z.(\arediff{\vlist{x},z} \lthen \bigvee_{S\in \lift{\Pi} \cup \lift{\Sigma} \cup \widetilde{\Sigma}} \tau^+_S(z))\big)
\land
\\ & 
    \bigwedge_{P\in\widetilde{\Sigma}} \qu y.{\tau}^{+}_P(y)
 \land
    \dqu y.\bigvee_{P\in\widetilde{\Sigma}} {\tau}^{+}_P(y)
    \end{aligned}
\end{equation}
\end{definition}

%We refer to \eqref{eq:unfoldingNablaolque} below for the unfolding of the expression $\varphi^{\fin}$. Observe that each ${\tau}^{+}_{P}$ with $P \in \widetilde{\Sigma}$ appearing in $\varphi^{\fin}$ is a (positive) $A$-type, as $P = \Ran(S) \subseteq A$ for some $S \in \Sigma$. Our desiderata on this translation concern the notions of \emph{continuity} and \emph{functionality}.

% NOT NEEDED ANYMORE AFTER ADDING LEMMA ON CONTINOUS AND FUNCTIONAL SENTENCES IN PREVIOUS SECTION
%\begin{definition}\label{def:functionalcontinuous_sentence} Given a set $A$ of unary predicates and $B \subseteq A$, we say that a sentence $\varphi \in {\olque}^+(A)$ is \emph{functionally continuous in $B$} if it is functional in $B$ (Definition \ref{def:functionalsentenceolque}) if, whenever $(D,\val \: A \to \wp(D)) \models \varphi$ then there is a restriction $\val'$ of $\val$ such that $(D,\val' \: A \to \wp(D)) \models \varphi$ and $s \in \val'(b)$ for $b \in B$ implies $s \not\in \val'(a)$ for all $a \in A\setminus\{b\}$., for every model $(D,\val \: A \to \wp(D))$,
%\begin{align*}
%\text{if } (D,\val),\ass \models \varphi \text{ then } & \exists\ \val' \: A \to \wp(D) \text{ such that } (D, \val'),\ass \models \varphi, \\
%& \val'(a)\subseteq \val(a) \text{ for all } a \in A, \tag{$\val'$ is a restriction of $\val$}\\
% & \val'(b) \text{ is finite for all }b \in B \text{ and } \tag{continuity in $B$}\\
% & \val'(b)\cap \val'(a) = \emptyset \text{ for all } a \in A\setminus\{b\} \text{ and }b \in B\tag{functionality in $B$}.
%\end{align*}
%\end{definition}
%In words, $\varphi$ is functionally continuous in $B$ if it is continuous in each $b \in B$ and, for each model $(D,\val)$ where $\varphi$ is true, there is a restriction $\val'$ of $\val$ which both witnesses continuity and does not assign any other $a \in A$ to elements marked with some $b \in B$.
%\begin{lemma}\label{LEM_cont}
%Let $\varphi \in {\olque}^+(A \times A)$ and $\varphi^{\fin}\in {\olque}^+(A\cup \shA )$ be given as in Definition~\ref{DEF_finitary_lifting}. Then $\varphi^{\fin}$ is functionally continuous in $\shA$.
% \end{lemma}
%\begin{proof}
%Because $\varphi^{\fin}$ is in basic form with all $\shA$-types either empty or singletons, it is functional in $\shA$ by Proposition \ref{lemma:functionalsentenceofoe}. Because it
%\end{proof}
%\begin{proof}
%We first unfold the definition of $\varphi^{\fin}$ as follows:
%\begin{equation}\label{eq:unfoldingNablaolque}
%\begin{aligned}
%\varphi^{\fin} =\ &
%\underbrace{
%    \exists \vlist{x}.\big(\arediff{\vlist{x}} \land \bigwedge_{0 \leq i \leq n} \tau^+_{\lift{T}_i}(x_i)
%}_{\psi_1}
%\land \underbrace{
%    \forall z.(\arediff{\vlist{x},z} \lthen \bigvee_{S\in \lift{\Pi} \cup \lift{\Sigma} \cup \widetilde{\Sigma}} \tau^+_S(z))\big)
%}_{\psi_2}
%\land
%\\ & \underbrace{
%    \bigwedge_{P\in\widetilde{\Sigma}} \qu y.{\tau}^{+}_P(y)
%}_{\psi_3} \land
% \underbrace{
%    \dqu y.\bigvee_{P\in\widetilde{\Sigma}} {\tau}^{+}_P(y)
%}_{\psi_4} .
%\end{aligned}
%\end{equation}
%Observe that $\psi_1 \land \psi_2$ is just $\mondbnfofoe{\lift{\vlist{T}}}{\lift{\Pi} \cup \lift{\Sigma} \cup \widetilde{\Sigma}}{+}$. Now suppose that $(D,\val \: (A \cup \shA ) \to \wp(D))$ is a model where $\varphi^{\fin}$ is true. This amounts to the truth of subformulas $\psi_1$, $\psi_2$, $\psi_3$ and $\psi_4$ whose syntactic shape yields information on the types of elements of $D$. In particular, we can define a partition of $D$ into subsets $D_1$, $D_2$, $D'_2$ as follows:
%\begin{itemize}
%  \item As $\psi_1$ is true, we can pick $n$ distinct elements $s_1,\dots,s_n$ of $D$ such that $s_i$ witnesses the positive type $\lift{T}_i$, %\tau^+_{\lift{T}_i}(x_i)$,
%   that is, $s_i \in \val(S)$ for each $S \in \lift{T}_i$. We define $D_1 := \{s_1,\dots,s_n\}$.
%  %
%  \item  As $\psi_2$ is true, we can cover all the elements not in $D_1$ with two disjoint sets $D_2$ and $D'_2$ given as follows. The set $D_2$ is defined to contain all the elements not in $D_1$ witnessing a type ${\tau}^{+}_P(z)$ with $P \in \widetilde{\Sigma}$. The set $D'_2$ is just the complement of $D_1 \cup D_2$: by syntactic shape of $\psi_2$, all elements of $D'_2$ witness a positive type ${\tau}^{+}_S$ with
%  $S \in \lift{\Pi} \cup \lift{\Sigma}$.
%  %
%  \item The truth of the subformula $\psi_4$ yields the information that the set $D_1 \cup D'_2$ is finite. If $\widetilde{\Sigma}$ is non-empty, the truth of $\psi_3$ implies that the set $D_2$ is infinite.
% \end{itemize}
%This partition uniquely associates with each $s \in D$ a type ${\tau}^{+}_S$ witnessed by $s$ and thus a set of unary predicates $S_s := S \subseteq A \cup \shA$. We can then define a valuation $\val'$ assigning to each element $s$ of $D$ exactly the set $S_s$.
%
%We now check the properties of $\val'$. As the partition inducing $\val'$ follows the syntactic shape of $\varphi^{\fin}$, one can observe that $\val'$ is a restriction of $\val$ and $(D,\val')$ makes $\varphi^{\fin}$ true. By definition of the partition, $\val'$ assigns unary predicates from $\shA$ only to elements in the finite set $D_1 \cup D'_2$, meaning that $\varphi^{\fin}$ is continuous in $\shA$. Furthermore, $\val'$ assigns at most one unary predicate from $\shA$ to each element of $D_1 \cup D'_2$, because $\lift{\vlist{T}} \cup \lift{\Pi} \cup \lift{\Sigma}$ is defined as the lifting of $\vlist{T} \cup \Pi \cup \Sigma$. It follows that $\varphi^{\fin}$ is also functional in $\shA$. Since the same restriction $\val'$ yields both properties, $\varphi^{\fin}$ is functionally continuous in $\shA$.
%\end{proof}
%
%\begin{remark} As $\varphi^{\fin}$ is of shape $\posdbnfofoei{\lift{\vlist{T}}}{\lift{\Pi} \cup \lift{\Sigma}}{\widetilde{\Sigma}}$ with $R \not\in \bigcup\widetilde{\Sigma}$ for each $R \in \shA$, by application of Corollary \ref{cor:olquecontinuousnf} we would immediately get that $\varphi^{\fin}$ is continuous in each $R \in \shA$. However, in proving Lemma \ref{LEM_cont} we privileged a direct proof allowing to show both continuity and functionality at once.
%\end{remark}

The next definition is standard (see e.g.  \cite{Walukiewicz96,Ven08}) as an intermediate step to define the transition function of the powerset construct for parity automata.

\begin{definition}\label{DEF_delta star} For a parity automaton $\aut = \tup{A,\tmap,\pmap,a_I}$, $a \in A$ and $c \in C$, we define $\tmap^{\star}(a,c) \df \tmap(a,c)[b \mapsto (a,b) \mid b \in A]$, that is, the sentence in ${\olque}^+(A\times A)$ obtained by replacing each occurrence of an unary predicate $b \in A$ in $\tmap(a,c)$ with the unary predicate $(a,b) \in A \times A$. \end{definition}

 Next we combine the previous definitions to characterise the transition function associated with the macro-states.

\begin{definition}\label{PROP_DeltaPowerset}
Let $\aut = \tup{A,\tmap,\pmap,a_I}$ be a $\wmso$-automaton. Let $c \in C$ be a label and $Q \in \shA$ a binary relation on $A$. By Corollary \ref{cor:olquepositivenf}, for some $\Pi,\Sigma \subseteq \shA$ and $T_i \subseteq A \times A$, there is a sentence $\Psi_{Q,c} \in {\olque}^+(A\times A)$ in the basic form $\bigvee \posdbnfofoei{\vlist{T}}{\Pi}{\Sigma}$ such that $\bigwedge_{a \in \Ran(Q)} \tmap^{\star}(a,c) \equiv \Psi_{Q,c}$. By definition $\Psi_{Q,c}$ is of the form $\bigvee_{i}\varphi_i$, with each $\phi_{i}$ of shape $\posdbnfofoei{\vlist{T}}{\Pi}{\Sigma}$. We put $\shDe(Q,c) := \bigvee_{i}\varphi_i^{\fin}$, where the translation $(-)^{\fin}$ is given as in Definition~\ref{DEF_finitary_lifting}. Observe that $\shDe(Q,c)$ is of type ${\olque}^+(A \cup \shA)$.
\end{definition}

We have now all the ingredients to define our two-sorted automaton.

\begin{definition}\label{def:finitaryconstruct}
Let $\aut = \tup{A,\tmap,\pmap,a_I}$ be a {\wmso-automaton}. We define the \emph{finitary construct over $\aut$} as the automaton $\aut^{\fin} = \tup{A^{\fin},\tmap^{\fin},\pmap^{\fin},a_I^{\fin}}$ given by
\begin{gather*}
      % \nonumber to remove numbering (before each equation)
        A^{\fin} \ \df \  A \cup \shA \quad\quad\quad a_I^{\fin} \ \df \  \{(a_I,a_I)\} \quad\quad\quad \pmap^{\fin}(a) \ \df \  \pmap(a) \quad\quad\quad \pmap^{\fin}(R) \ \df \  1 \\
        \tmap^{\fin}(a,c) \ \df \  \tmap(a,c) \qquad \qquad \qquad 
        \tmap^{\fin}(Q,c) \ \ \df \ \  \shDe(Q,c) \vee \! \! \! \! \bigwedge_{a \in \Ran(Q)} \! \! \! \tmap(a,c).
      \end{gather*}
\end{definition}

The idea behind this definition is that $\aut^{\fin}$ is enforced to process only a finite portion of any accepted tree while in the non-deterministic mode. This is encoded in game-theoretic terms through the notion of functional and finitary strategy. 

\begin{definition}\label{def:StratfunctionalFinitary}
Given a $\wmso$-automaton $\bbA = \tup{A,\tmap,\pmap,a_I}$ and transition system $\bbT$, a strategy $f$ for \eloise in $\mathcal{A}(\bbA,\bbT)$ is \emph{functional in $B \subseteq A$} (or simply functional, if $B=A$) if for each node $s$ in $\bbT$ there is at most one $b \in B$ such that $(b,s)$ is a reachable position in an $f$-guided match. Also $f$ is \emph{finitary} in $B$ if there are only finitely many nodes $s$ in $\bbT$ for which a position $(b,s)$ with $b \in B$ is reachable in an $f$-guided match.
\end{definition}



The next proposition establishes the desired properties of the finitary
construct.

\begin{theorem}[\textbf{Simulation Theorem for $\wmso$-automata}]\label{PROP_facts_finConstrwmso} Let $\aut$ be a $\wmso$-automaton and $\aut^{\fin}$ its finitary construct.
\begin{enumerate}[(i)]
  \itemsep 0 pt
  \item $\aut^{\fin}$ is a $\wmso$-automaton.\label{point:finConstrAut}
  \item For any $\bbT$, if $\eloise$ has a winning strategy in $\agame(\aut^{\fin},\bbT)$ from position $(a_I^{\fin},s_I)$ then she has one that is functional in $\shA$ and finitary in $\shA$.% (\emph{cf.} Definition \ref{def:StratfunctionalFinitary}).
  \label{point:finConstrStrategy}
  \item $\aut \equiv \aut^{\fin}$. \label{point:finConstrEquiv}
  \end{enumerate}
\end{theorem}
\begin{proof}
\begin{enumerate}[(i)]
\item Observe that any SCC
of $\aut^{\fin}$ involves states of exactly one sort, either $A$ or $\shA$. For SCCs on sort $A$, \textbf{(weakness)} and \textbf{(continuity)} of $\aut^{\fin}$ follow by the ones of $\aut$. For SCCs on sort $\shA$, \textbf{(weakness)} follows by observing that all macro-states in $\aut^{\fin}$ have the same parity value. Concerning \textbf{(continuity)}, by definition of $\tmap^{\fin}$ any macro-state can only appear inside a formula of the form $\varphi^{\fin} = \posdbnfofoei{\lift{\vlist{T}}}{\lift{\Pi} \cup \lift{\Sigma}}{\widetilde{\Sigma}}$ as in \eqref{eq:unfoldingNablaolque}. Because $\shA \cap \widetilde{\Sigma} = \emptyset$, by Corollary \ref{cor:ofoeicontinuousnf}.\ref{pt:ofoeimonotone} $\varphi^{\fin}$ is continuous in each $Q \in \shA$.
  \item  Let $f$ be a winning strategy for $\eloise$ in $\mathcal{A}(\aut^{\fin},\bbT)@(a_I^{\fin},s_I)$. We define a strategy $f'$ for $\eloise$ in the same game as follows:
      \begin{enumerate}[label=(\alph*),ref=\alph*]
        \item on basic positions of the form $(a,s) \in A\times T$, let $\val$ be the valuation suggested by $f$. We let the valuation suggested by $f'$ be the restriction $\val'$ of $\val$ to $A$. Observe that, as no predicate from $A^{\fin}\setminus A =\shA$ occurs in $\tmap^{\fin}(a,\V(s)) = \tmap(a,\V(s))$, then $\val'$ also makes that sentence true in $\R{s}$.
        \label{point:stat2point1}
        \item for basic positions of the form $(R,s) \in \shA \times T$, let $\val_{R,s}$ be the valuation suggested by $f$. As $f$ is winning, $\tmap^{\fin}(R,\V(s))$ is true in the model $\val_{R,s}$. If this is because the disjunct $\bigwedge_{a \in \Ran(R)} \tmap(a,\V(s))$ is made true, then we can let $f'$ suggest the restriction to $A$ of $\val_{R,s}$, for the same reason as in \eqref{point:stat2point1}. Otherwise, the disjunct $\shDe(R,\V(s)) = \bigvee_{i}\varphi_i^{\fin}$ is made true. This means that, for some $i$, $(R[s], \val_{R,s}) \models \varphi_i^{\fin}$.
             Now, by construction of $\varphi_i^{\fin}$ as in \eqref{eq:unfoldingNablaolque}, $\lift{T}_1,\dots,\lift{T}_k$ and each $S \in \in \lift{\Pi} \cup \lift{\Sigma}$ are either empty or singleton subsets of $\shA$ and $\widetilde{\Sigma} \cap \shA = \emptyset$. By Corollary \ref{def:functionalsentenceofoe}.\ref{pt:ofoeifunctionalcontinuous}, this implies that $\varphi_i^{\fin}$ is functionally continuous in $\shA$. Thus we have a restriction $\val_{R,s}'$ of $\val_{R,s}$ that verifies $\varphi_i^{\fin}$, assigns finitely many nodes to predicates from $\shA$ and associates with each node at most one predicate from $\shA$. We let $\val_{R,s}'$ be the suggestion of $f'$ from position $(R,s)$.
      \end{enumerate}
      The strategy $f'$ defined as above is immediately seen to be
      surviving for $\eloise$. It is also winning, because the set of
      basic positions on which $f'$ is defined is a subset of the one
      of the winning strategy $f$. By this observation it also follows that any $f'$-conform match visits basic positions of the form $(R,s) \in \shA \times C$ only finitely many times, as those have odd parity. By definition, the valuation suggested by $f'$ only assigns finitely many nodes to predicates in $\shA$ from positions of that shape, and no nodes from other positions. It follows that $f'$ is finitary in $\shA$. Functionality in $\shA$ also follows immediately by definition of $f'$.
  \item For the direction from left to right, it is immediate by definition of $\aut^{\fin}$ that a winning strategy for $\eloise$ in $\mc{G} = \mathcal{A}(\aut,\bbT)@(a_I,s_I)$ is also winning for $\eloise$ in $\mc{G}^{\fin} = \mathcal{A}(\aut^{\fin},\bbT)@(a_I^{\fin},s_I)$.

      For the direction from right to left, let $f$ be a winning strategy for $\eloise$ in $\mc{G}^{\fin}$. The idea is to define a strategy $f'$ for $\eloise$ in stages, while playing a match $\pi'$ in $\mc{G}$. In parallel to $\pi'$, a shadow match $\pi$ in $\mc{G}^{\fin}$ is maintained, where $\eloise$ plays according to the strategy $f$. For each round $z_i$, we want to keep the following relation between the two matches:
\smallskip
\begin{center}
\fbox{\parbox{12cm}{
Either
\begin{enumerate}[label=(\arabic*),ref=\arabic*]
  \item basic positions of the form $(Q,s) \in \shA \times T$ and $(a,s) \in A \times T$ occur respectively in $\pi$ and $\pi'$, with $a \in \Ran(Q)$,
\end{enumerate}
or
\begin{enumerate}[label=(\arabic*),ref=\arabic*]
  \item[(2)] the same basic position of the form $(a,s) \in A \times T$ occurs in both matches.
\end{enumerate}
}}\hspace*{0.3cm}($\ddag$)
\end{center}
\smallskip
The key observation is that, because $f$ is winning, a basic position of the form $(Q,s) \in \shA \times T$ can occur only for finitely many initial rounds $z_0,\dots,z_n$ that are played in $\pi$, whereas for all successive rounds $z_n,z_{n+1},\dots$ only basic positions of the form $(a,s) \in A \times T$ are encountered. Indeed, if this was not the case then either $\eloise$ would get stuck or the minimum parity occurring infinitely often would be odd, since states from $\shA$ have parity $1$.

It follows that enforcing a relation between the two matches as in ($\ddag$) suffices to prove that the defined strategy $f'$ is winning for $\eloise$ in $\pi'$. For this purpose, first observe that $(\ddag).1$ holds at the initial round, where the positions visited in $\pi'$ and $\pi$ are respectively $(a_I,s_I) \in A \times T$ and $(\{(a_I,a_I)\},s_I) \in A^{\fin} \times T$. Inductively, consider any round $z_i$ that is played in $\pi'$ and $\pi$, respectively with basic positions $(a,s) \in A \times T$ and $(q,s) \in A^{\fin} \times T$. To define the suggestion of $f'$ in $\pi'$, we distinguish two cases.
\begin{itemize}
  \item First suppose that $(q,s)$ is of the form $(Q,s) \in
  \shA\times T$. By ($\ddag$) we can assume that $a$ is in $\Ran(Q)$. Let $\val_{Q,s} :A^{\fin} \rightarrow \wp(\R{s})$ be the valuation suggested by $f$, verifying the sentence $\tmap^{\fin}(Q,\V(s))$. We distinguish two further cases, depending on which disjunct of $\tmap^{\fin}(Q,\V(s))$ is made true by $\val_{Q,s}$.
      \begin{enumerate}[label=(\roman*), ref=\roman*]
        \item If $(\R{s},\val_{Q,s})\models \bigwedge_{b \in \Ran(Q)} \tmap(b,\V(s))$, then we let $\eloise$ pick the restriction to $A$ of the valuation $\val_{Q,s}$. \label{point:valuation1}
        \item If $(\R{s},\val_{Q,s})\models \shDe(Q,\V(s))$, we let $\eloise$ pick a valuation $\val_{a,s}:A \rightarrow \p (\R{s})$ defined by putting, for each $b \in A$:
            \begin{align*}
            % \nonumber to remove numbering (before each equation)
               \val_{a,s}(b)\ :=\ \bigcup_{b \in \Ran(Q')} \{t \in \R{s} \mid t \in \val_{Q,s}(Q')\} 
               \cup  \{t \in \R{s} \mid t \in \val_{Q,s}(b)\} .
            \end{align*} \label{point:valuation2}
      \end{enumerate}
      It can be readily checked that the suggested move is admissible for $\eloise$ in $\pi$, i.e. it makes $\tmap(a,\V(s))$ true in $\R{s}$. For case \eqref{point:valuation2}, one has to observe how $\shDe$ is defined in terms of $\tmap$. In particular, the nodes assigned to $b$ by $\val_{Q,s}$ have to be assigned to $b$ also by $\val_{a,s}$, as they may be necessary to fulfill the condition, expressed with $\qu$ and $\dqu$, that infinitely many nodes witness (or that finitely many nodes do not witness) some type.

      We now show that $(\ddag)$ holds at round $z_{i+1}$. If \eqref{point:valuation1} is the case, any next position $(b,t)\in A \times T$ picked by player $\forall$ in $\pi'$ is also available for $\forall$ in $\pi$, and we end up in case $(\ddag .2)$. Suppose instead that \eqref{point:valuation2} is the case. Given the choice $(b,t) \in A \times T$ of $\forall$, by definition of $\val_{a,s}$ there are two possibilities. First, $(b,t)$ is also an available choice for $\forall$ in $\pi$, and we end up in case $(\ddag .2)$ as before. Otherwise, there is some $Q' \in \shA$ such that $b$ is in $\Ran(Q')$ and $\forall$ can choose $(Q',t)$ in the shadow match $\pi$. By letting $\pi$ advance at round $z_{i+1}$ with such a move, we are able to maintain $(\ddag .1)$ also in $z_{i+1}$.
  \item In the remaining case, inductively we are given the same basic position $(a,s) \in A\times T$ both in $\pi$ and in $\pi'$. The valuation $\val$ suggested by $f$ in $\pi$ verifies $\tmap^{\fin}(a,\V(s)) = \tmap(a,\V(s))$, thus we can let the restriction of $\val$ to $A$ be the valuation chosen by $\eloise$ in the match $\pi'$. It is immediate that any next move of $\forall$ in $\pi'$ can be mirrored by the same move in $\pi$, meaning that we are able to maintain the same position --whence the relation $(\ddag.1)$-- also in the next round.
\end{itemize}
In both cases, the suggestion of strategy $f'$ was a legitimate move for $\eloise$ maintaining the relation $(\ddag)$ between the two matches for any next round $z_{i+1}$. It follows that $f'$ is a winning strategy for $\eloise$ in $\mc{G}$.
\end{enumerate}
\end{proof}





\subsection{From formulas to automata}

In this subsection we conclude the proof of Theorem~\ref{t:wmsoauto}. %, showing that $\wmso$-automata are closed under the  operations corresponding to the connectives of $\mso$, that is: union, complementation and projection with respect to finite sets.We start with the latter.
We first focus on the case of projection with respect to finite sets, which exploits our simulation result, Theorem~\ref{PROP_facts_finConstrwmso}. The definition of the projection construction is formulated more generally for parity automata, as it will be later applied to classes other than $\AutWC(\olque)$. It clearly preserves the weakness and continuity conditions.
%%%%
%%%% PROJECTION
%%%%


%\subsection{Closure under Finitary Projection}

\begin{definition}\label{DEF_fin_projection}
Let $\aut = \tup{A, \tmap, \Omega, a_I}$ be a parity automaton on alphabet $\p(\prop \cup \{p\})$. We define the automaton ${{\exists} p}.\aut = \tup{A, \tmapProj, \Omega, a_I}$ on alphabet $\p\prop$ by putting
\begin{equation*}
% \nonumber to remove numbering (before each equation)
  \tmapProj(a,c) \ \df \ \tmap(a,c) \qquad \qquad
  \tmapProj(Q,c) \ \df \ \tmap(Q,c) \vee \tmap(Q,c\cup\{p\}).
\end{equation*}
The automaton ${{\exists} p}.\aut$ is called the \emph{finitary projection
construct of $\aut$ over $p$}.
\end{definition}


\begin{lemma}\label{PROP_fin_projection}
For each $\wmso$-automaton $\aut$ on alphabet $\p (\prop \cup \{p\})$,
we have that
$$\trees({{\exists} p}.\aut) \ \equiv\
{{\finexists} p}.\trees(\aut).
$$
\end{lemma}

\begin{proof}
First of all, by Theorem \ref{PROP_facts_finConstrwmso} we can assume that $\aut$ is two-sorted, i.e. $\aut = (\aut')^{\fin} = \tup{A^{\fin}, \tmap^{\fin},\pmap^{\fin},a_I^{\fin}}$ for some $\wmso$-automaton $\aut' = \tup{A,\tmap,\pmap, a_I}$. What we need to show is that for any tree $\bbT = \tup{T,R,\V \: \prop \to \pw T,s_I}$:
$${{\exists} p}.\aut \text{ accepts } \bbT \text{ iff } \text{there is a finite }p \text{ -variant }\bbT' \text{of } \bbT \text{  such that } \aut \text{  accepts } \bbT'.$$
For direction from left to right, we first observe that the properties stated by Lemma~\ref{PROP_facts_finConstr}, which hold for $\aut$ by assumption, by construction hold for ${{\exists} p}.\aut$ as well. Thus we can assume that the given winning strategy $f$ for $\eloise$ in $\mc{G_{\exists}} = \mc{A}({\finexists p}.\aut,\bbT)@(a_I^{\fin},s_I)$ is functional and finitary in $\shA$. Functionality allows us to associate with each node $s$ either none or a unique state $Q_s \in \shA$ such that $(Q_s,s)$ is winning for $\eloise$. We now want to isolate the nodes that $f$ treats ``as if they were labeled with $p$''. For this purpose, let $\val_{s}$ be the valuation suggested by $f$ from a position $(Q_s,s) \in \shA \times T$. As $f$ is winning, $\val_{s}$ makes $\tmapProj(Q,\tscolors(s))$ true in $\R{s}$. We define a $p$-variant $\bbT' = \tup{T,R,\V' \: \prop\cup\{p\} \to \pw T,s_I}$ of $\bbT$ by coloring with $p$ all nodes in the following set:
 \begin{equation}\label{eq:X_p}
% \nonumber to remove numbering (before each equation)
   X_p\ :=\ \{s \in T\mid (\R{s},\widetilde{\val}_{s}) \models \tmap^{\f}(Q_s,\tscolors(s)\cup\{p\})\}.
\end{equation}
The fact that $f$ is finitary in $\shA$ guarantees that $X_p$ is finite, whence $\bbT'$ is a finite $p$-variant. It remains to show that $\aut$ accepts $\bbT'$: we claim that $f$ itself is also winning for $\eloise$ in $\mc{G} = (\aut,\bbT')@(a_I,s_I)$. In order to see that, let us construct in stages an $f$-conform match $\pi^{2S}$ of $\mc{G}$ and an $f$-conform shadow match $\tilde{\pi}$ of $\mc{G_{\exists}}$. The inductive hypothesis we want to bring from one round to the next is that the same basic position occurs in both matches, as this suffices to prove that $f$ is winning for $\eloise$ in $\mc{G}$.

First we consider the case of a basic position $(Q,s) \in A^{\fin} \times T$ where $Q \in \shA$. By assumption $f$ provides a marking $\widetilde{m}_s$ that makes $\tmapProj(Q,\V(s))$ true in $\R{s}$. Thus $\widetilde{m}_s$ verifies either $\tmap^{\fin}(Q,\V(s))$ or $\tmap^{\fin}(Q,\V(s)\cup \{p\})$. Now, the match $\pi^{\fin}$ is played on the $p$-variant $\bbT'$, where the labeling $\V'(s)$ is decided by the membership of $s$ to $X_p$. According to \eqref{eq:X_p}, if $\widetilde{m}_s$ verifies $\tmap^{\fin}(Q,\V(s)\cup \{p\})$ then $s$ is in $X_p$, meaning that it is labeled with $p$ in $\bbT'$, i.e. $\V'(s) = \V(s)\cup \{p\}$. Therefore $\widetilde{m}_s$ also verifies $\tmap^{\fin}(Q,\V'(s))$ and it is a legitimate move for $\eloise$ in match $\pi^{\fin}$. In the remaining case, $\widetilde{m}_s$ verifies $\tmap^{\fin}(Q,\V(s))$ but falsifies $\tmap^{\fin}(Q,\V(s)\cup \{p\})$, implying by definition that $s$ is not in $X_p$. This means that $s$ is not labeled with $p$ in $\bbT'$, i.e. $\V'(s) = \V(s)$. Thus again $\widetilde{m}_s$ verifies $\tmap^{\fin}(Q,\V'(s))$ and it is a legitimate move for $\eloise$ in match $\pi^{\fin}$.

It remains to consider the case of a basic position $(a,s) \in A^{\fin} \times T$ with $a \in A$ a state. By definition $\tmapProj(a,\V(s))$ is just $\tmap^{\fin}(a,\V(s))$. As $(a,s)$ is winning, we can assume that no position $(Q,s)$ with $Q$ a macro-state is winning according to the same $f$, as making $\tmapProj$-sentences true never forces $\eloise$ to mark a node both with a state and a macro-state. Therefore, $s$ is not in $X_p$ either, meaning that it it is not labeled with $p$ in the $p$-variant $\bbT'$ and thus $\V'(s) = \V(s)$. This implies that $f$ makes $\tmap^{\fin}(a,\V'(s)) = \tmap^{\fin}(a,\V(s))$ true in $\R{s}$ and its suggestion is a legitimate move for $\eloise$ in match $\pi^{\fin}$. In order to conclude the proof, observe that for all positions that we consider the same marking is suggested to $\eloise$ in both games: this means that any next position that is picked by player $\abelard$ in $\pi^{\fin}$ is also available for $\abelard$ in the shadow match $\tilde{\pi}$.


We now show the direction from right to left of the statement. Let $\bbT'$ be a finite $p$-variant of
$\bbT$, with labeling function $\tscolors'$, and $g$ a winning strategy for $\exists$ in $\mc{G} = \mathcal{A}(\aut,\bbT')@(a_I,s_I)$. Our goal is to define a strategy $g'$ for $\exists$ in $\mc{G_{\exists}}$. As usual, $g'$ will be constructed in stages, while playing a match $\pi'$ in $\mc{G_{\exists}}$. In parallel to $\pi'$, a \emph{bundle} $\mc{B}$ of $g$-guided shadow matches in $\mc{G}$ is maintained, with the following condition enforced for each round $z_i$:
\smallskip
\begin{center}
\fbox{\parbox{13cm}{
\begin{enumerate}
  \item If the current (i.e. at round $z_i$) basic position in $\pi'$ is of the form $(Q,s) \in \shA \times T$, then for each $a \in\Ran(Q)$ there is an $g$-guided (partial) shadow match $\pi_a$ at basic position $(a,s) \in A\times T$ in the current bundle $\mc{B}_i$. Also, either $\bbT'_s$ is not $p$-free (i.e., it does contain a node $s'$ with $p \in \tscolors'(s')$) or $s$ has some sibling $t$ such that $\bbT'_t$ is not $p$-free.
  \item Otherwise, the current basic position in $\pi'$ is of the form $(a,s) \in A \times T$ and $\bbT'_s$ is $p$-free (i.e., it does not contain any node $s'$ with $p \in \tscolors'(s')$). Also, the bundle $\mc{B}_i$ only consists of a single $g$-guided match $\pi_a$ whose current basic position is also $(a,s)$.
\end{enumerate}
}}\hspace*{0.3cm}($\ddag$)
\end{center}
\smallskip
We briefly recall the idea behind condition ($\ddag$). Point ($\ddag.1$) describes the part of match $\pi'$ where it is still possible to encounter nodes which are labeled with $p$ in $\bbT'$. As $\tmapProj$ only takes the letter $p$ into account when defined on macro-states in $\shA$, we want $\pi'$ to visit only positions of the form $(Q,s) \in \shA \times T$ in that situation. Anytime we visit such a position $(Q,s)$ in $\pi'$, the role of the bundle is to provide one $g$-guided shadow match at position $(a,s)$ for each $a \in \Ran(Q)$.
Then $g'$ is defined in terms of what $g$ suggests from those positions.

 Point ($\ddag.2$) describes how we want the match $\pi'$ to be
 played on a $p$-free subtree: as any node that one might encounter has the same label in $\bbT$ and $\bbT'$,
it is safe to let ${\finexists p}.\aut$ behave as $\aut$ in such situation. Provided that the two matches visit the same basic positions, of the form $(a,s)\times A \times T$, we can let $g'$ just copy $g$.

The key observation is that, as $\bbT'$ is a \emph{finite} $p$-variant of $\bbT$, nodes labeled with $p$ are reachable only for finitely many rounds of $\pi'$. This means that, provided that ($\ddag$) hold at each round, ($\ddag.1$) will describe an initial segment of $\pi'$, whereas ($\ddag.2$) will describe the remaining part. Thus our proof that $g'$ is a winning strategy for $\exists$ in $\mc{G}_{\exists}$ is concluded by showing that ($\ddag$) holds for each stage of construction of $\pi'$ and $\mc{B}$.

\medskip

For this purpose, we initialize $\pi'$ from position $(\shai,s) \in \shA\times T$ and the bundle $\mc{B}$ as $\mc{B}_0 = \{\pi_{a_I}\}$, with $\pi_{a_I}$ the partial $g$-guided match consisting only of the position $(a_I,s)\in A\times T$. The situation described by ($\ddag .1$) holds at the initial stage of the construction.
Inductively, suppose that at round $z_i$ we are given a position $(q,s) \in A^{\f} \times T$ in $\pi^{\f}$ and a bundle $\mc{B}_i$ as in ($\ddag$). To show that ($\ddag$) can be maintained at round $z_{i+1}$, we distinguish two cases, corresponding respectively to situation ($\ddag.1$) and ($\ddag.2$) holding at round $z_i$.
\begin{enumerate}[label = (\Alph*), ref = \Alph*]
%\yvwarning{Notation `$q$' is confusing, see $\val'(q)$ below FZ: I corrected $q$ into $q'$ below}
  \item If $(q,s)$ is of the form $(Q,s) \in \shA \times T$, by inductive hypothesis we are given with $g$-guided shadow matches $\{\pi_a\}_{a \in \Ran(Q)}$ in $\mc{B}_i$. For each match $\pi_a$ in the bundle, we are provided with a valuation $\val_{a,s}: A \rightarrow \p (\R{s})$ making $\tmap(a,\tscolors'(s))$ true. Then we further distinguish the following two cases.
\begin{enumerate}[label = (\roman*), ref = \roman*]
  \item \label{point:TsNotPFree} Suppose first that $\bbT'_s$ is not $p$-free. We let the suggestion $\val' \: A^{\f} \to \p (\R{s})$ of $g'$ from position $(Q,s)$ be defined as follows:
       \begin{align*}
       % \nonumber to remove numbering (before each equation)
       %\widetilde{\val}_{Q,s}(Q') \ \df \  \bigcup_{a \in \Ran(Q),\ b \in \Ran(Q')}\{t\ \in \R{s}|\ t \in \val_{a,s}(b)\}.
       \val'(q')\ :=\ \begin{cases}
               \bigcap\limits_{\substack{(a,b) \in q',\\ a \in \Ran(Q)}}\{t\ \in \R{s} \mid t \in \val_{a,s}(b)\}               & q' \in \shA \\[2em]
               \bigcup\limits_{a \in \Ran(Q)} \{t\ \in \R{s} \mid t \in \val_{a,s}(q') \text{ and }\bbT'.t\text{ is $p$-free}\}              & q' \in A.
               %\\[1.5em]               \hspace{.6cm}\emptyset & \text{otherwise.}
           \end{cases}
       \end{align*}
       The definition of $\val'$ on $q' \in \shA$ is standard (\emph{cf.}~\cite[Prop. 2.21]{Zanasi:Thesis:2012}) and guarantees a correspondence between the states assigned by the markings $\{\val_{a,s}\}_{a \in \Ran(Q)}$ and the macro-states assigned by $\val'$. The definition of $\val'$ on $q' \in A$ aims at fulfilling the conditions, expressed via $\qu$ and $\dqu$, on the number of nodes in $\R{s}$ witnessing (or not) some $A$-types. Those conditions are the ones that $\shDe(Q,\tscolors'(s))$ --and thus also $\tmap^{\f}(Q,\tscolors'(s))$-- ``inherits'' by $\bigwedge_{a \in \Ran(R)} \tmap(a,\tscolors'(s))$, by definition of $\shDe$. Notice that we restrict $\val'(q')$ to the nodes $t \in \val_{a,s}(q')$ such that $\bbT'.t$ is $p$-free. As $\bbT'$ is a \emph{finite} $p$-variant, only \emph{finitely many} nodes in $\val_{a,s}(q')$ will not have this property. Therefore their exclusion, which is crucial for maintaining condition ($\ddag$) (\emph{cf.}~case \eqref{point:ddag2CardfromMacro} below), does not influence the fulfilling of the cardinality conditions expressed via $\qu$ and $\dqu$ in $\shDe(Q,\tscolors'(s))$.

       On the base of these observations, one can check that $\val'$ makes $\shDe(Q,\tscolors'(s))$--and thus also $\tmap^{\f}(Q,\tscolors'(s))$--true in $\R{s}$. In fact, to be a legitimate move for $\exists$ in $\pi'$, $\val'$ should make $\tmapProj(Q,\tscolors(s))$ true: this is the case, for $\tmap^{\f}(Q,\tscolors'(s))$ is either equal to $\tmap^{\f}(Q,\tscolors(s))$, if $p \not\in \tscolors'(s)$, or to $\tmap^{\f}(Q,\tscolors(s)\cup\{p\})$ otherwise. In order to check that we can maintain $(\ddag)$, let $(q',t) \in A^{\f} \times T$ be any next position picked by $\forall$ in $\pi'$ at round $z_{i+1}$. As before, we distinguish two cases:
       \begin{enumerate}[label = (\alph*), ref = \alph*]
         \item If $q'$ is in $A$, then, by definition of $\val'$, $\forall$ can choose $(q',t)$ in some shadow match $\pi_a$ in the bundle $\mc{B}_i$. We dismiss the bundle --i.e. make it a singleton-- and bring only $\pi_a$ to the next round in the same position $(q',t)$. Observe that, by definition of $\val'$, $\bbT'.t$ is $p$-free and thus ($\ddag.2$) holds at round $z_{i+1}$. \label{point:ddag2CardfromMacro}
         \item Otherwise, $q'$ is in $\shA$. The new bundle $\mc{B}_{i+1}$ is given in terms of the bundle $\mc{B}_i$: for each $\pi_a \in \mc{B}_i$ with $a\in \Ran(Q)$, we look if for some $b \in \Ran(q')$ the position $(b,t)$ is a legitimate move for $\forall$ at round $z_{i+1}$; if so, then we bring $\pi_a$ to round $z_{i+1}$ at position $(b,t)$ and put the resulting (partial) shadow match $\pi_b$ in $\mc{B}_{i+1}$. Observe that, if $\forall$ is able to pick such position $(q',t)$ in $\pi'$, then by definition of $\val'$ the new bundle $\mc{B}_{i+1}$ is non-empty and consists of an $g$-guided (partial) shadow match $\pi_b$ for each $b \in \Ran(q')$. In this way we are able to keep condition ($\ddag.1$) at round $z_{i+1}$.
       \end{enumerate}
    \item Let us now consider the case in which $\bbT'_s$ is $p$-free. We let $g'$ suggest the valuation $\val'$ that assigns to each node $t \in \R{s}$ all states in $\bigcup_{a \in \Ran(Q)}\{b \in A\ |\ t \in \val_{a,s}(b)\}$. It can be checked that $\val'$ makes $\bigwedge_{a \in \Ran(Q)} \tmap(a,\tscolors'(s))$ -- and then also $\tmap^{\f}(Q,\tscolors'(s))$ -- true in $\R{s}$. As $p \not\in \tscolors(s)=\tscolors'(s)$, it follows that $\val'$ also makes $\tmapProj(Q,\tscolors(s))$ true, whence it is a legitimate choice for $\exists$ in $\pi'$. Any next basic position picked by $\forall$ in $\pi'$ is of the form $(b,t) \in A \times T$, and thus condition ($\ddag.2$) holds at round $z_{i+1}$ as shown in (i.a). %\eqref{point:ddag2CardfromMacro}
  \end{enumerate}
  \item In the remaining case, $(q,s)$ is of the form $(a,s) \in A \times T$ and by inductive hypothesis we are given with a bundle $\mc{B}_i$ consisting of a single $f$-guided (partial) shadow match $\pi_a$ at the same position $(a,s)$. Let $\val_{a,s}$ be the suggestion of $\exists$ from position $(a,s)$ in $\pi_a$. Since by assumption $s$ is $p$-free, we have that $\tscolors'(s) = \tscolors(s)$, meaning that $\tmapProj(a,\tscolors(s))$ is just $\tmap(a,\tscolors(s)) = \tmap(a,\tscolors'(s))$. Thus the restriction $\val'$ of $\val$ to $A$ makes $\tmap(a,\tscolors'(t))$ true and we let it be the choice for $\exists$ in $\tilde{\pi}$. It follows that any next move made by $\forall$ in $\tilde{\pi}$ can be mirrored by $\forall$ in the shadow match $\pi_a$.
      \begin{comment}Version with minimality:
      It follows that $\tmapProj(a,\tscolors(t))$ is just $\tmap(a,\tscolors(t)) = \tmap(a,\tscolors'(t))$ and the same valuation suggested by $f$ in $\pi_a$ is a legitimate choice for $\exists$ in $\tilde{\pi}$. By letting $\exists$ choose such valuation, it follows that any next move made by $\forall$ in $\tilde{\pi}$ can be mirrored by $\forall$ in the shadow match $\pi_a$.
      \end{comment}
\end{enumerate}
%As explained above, since $\bbT'$ is a noetherian $p$-variant, then ($\ddag .1$) holds for finitely many stages of construction of $\tilde{\pi}$, whereas ($\ddag .2$) holds for all the remaining stages, by construction of $\tilde{f}$. It follows that this strategy is winning for $\exists$ in $\tilde{G}$.
\end{proof} 

%%%%%%
%%%%%% BOOLEANS
%%%%%%

\subsubsection{Closure under Boolean operations}

In this section we will show that the class of $\wmso$-automaton recognizable
tree languages is closed under the Boolean operations.
%
Start with closure under union, we just mention the following result, without
providing the (completely routine) proof.

\begin{proposition}
\label{t:cl-dis}
Let $\bbA_{0}$ and $\bbA_{1}$ be $\wmso$-automata. 
Then there is a $\wmso$-automaton $\bbA$ such that $\trees(\bbA)$ is the 
union of $\trees(\bbA_{0})$ and $\trees(\bbA_{1})$.
\end{proposition}

For closure under complementation we reuse the general results established in Section \ref{sec:parityaut} for parity automata.

\begin{proposition}
\label{t:cl-cmp}
Let $\bbA$ be an $\wmso$-automaton.
Then the automaton $\overline{\aut}$ defined in Definition~\ref{d:caut} is a
$\wmso$-automaton recognizing the complement of $\trees(\bbA)$.
\end{proposition}

\begin{proof} It suffices to check that Proposition \ref{prop:autcomplementation} restricts to the class $\AutWC(\olque)$ of $\wmso$-automata. First, the fact that $\olque$ is closed under Boolean duals (Def. \ref{d:bdual1}) implies that it holds for the class $\Aut(\olque)$. It then remains to check that the automata construction $\overline{\cdot}$ preserves weakness and continuity. But this is straightforward, given the self-dual nature of these properties.\end{proof}


%%%%
%%%% PROOF THEOREM
%%%%

We are now finally able to conclude the direction from formulas to automata of the characterisation theorem.

\begin{proof}[of Theorem \ref{t:wmsoauto}] The proof is by induction on $\varphi$.
\begin{itemize}
  \item For the base case $\varphi = p \inc q$, the corresponding 
  $\wmso$-automaton is provided in \cite[Ex. 2.6]{Zanasi:Thesis:2012}. 
  For the base case $\varphi = R(p,q)$, we give the corresponding 
  $\wmso$-automaton $\aut_{R(p,q)} = \tup{A,\tmap,\Omega,a_I}$ below:
\begin{eqnarray*}
        A  \  \df \  \{a_0,a_1\}  \qquad \qquad  a_I  \   \df  \  a_0   \qquad \qquad   \Omega(a_0)  \  \df \  0 \qquad \qquad
    \Omega(a_1)  \  \df \  1 \\
  \tmap(a_0,c)  \  \df \  \left\{
	\begin{array}{ll}
           \exists x. a_1(x) \wedge \forall y. a_0(y)  \  \mbox{if }p \in c 
	\\ \forall x\ (a_0(x))  \  \mbox{otherwise.}
	\end{array}
\right. \qquad \qquad
  \tmap(a_1,c)  \  \df \  \left\{
	\begin{array}{ll}
        \top  \  \mbox{if }q \in c \\
        \bot  \  \mbox{otherwise}
	\end{array}
\right.
\end{eqnarray*}
Note that the $\mso$-automaton for $R(p,q)$ provided in 
\cite[Ex. 2.5]{Zanasi:Thesis:2012} is \emph{not} a $\wmso$-automaton, as the 
continuity property does not hold.

\item
For the Boolean cases, where $\varphi = \psi_1 \vee \psi_2$ or $\phi = \neg\psi$
we refer to the closure properties of recognizable tree languages, see 
Theorem~\ref{t:cl-dis} and Theorem~\ref{t:cl-cmp}, 
respectivel.
  
\item 
The case $\varphi = \exists p. \psi$ follows by the following chain of
equivalences, where $\aut_{\psi}$ is given by the inductive hypothesis and 
${\finexists p}.\aut_{\psi}$ is constructed according to 
Definition~\ref{DEF_fin_projection}:
\begin{alignat*}{2}
{\finexists p}.\aut_{\psi} \text{ accepts }\mb{T} 
   & \text{ iff }
     \aut_{\psi} \text{ accepts } \mb{T}[p \mapsto X], 
     \text{ for some } X \sse_{\om} T
   & \quad\text{(Lemma~\ref{PROP_fin_projection})}
\\ & \text{ iff }
     \mb{T}[p \mapsto X] \models \psi,
     \text{ for some } X \sse_{\om} T
   & \quad\text{(induction hyp.)}
\\ & \text{ iff }
    \mb{T} \models \exists p. \psi
   & \quad\text{(semantics $\wmso$)}
\end{alignat*}
\end{itemize}
\end{proof}





\subsection{From automata to formulas}\label{sec:aut_to_form_wmso}


In what follows, we conclude the automata characterisation of $\wmso$ with the converse of Theorem~\ref{t:wmsoauto}, i.e. the direction from automata to formulas (see Theorem \ref{thm:wmso_autofor} below). To this aim, we first introduce a fixpoint extension of first-order logic.

\subsubsection{Fixpoint extension of first-order logic}

Let our first-order signature be composed of a set $\prop$ of monadic predicates (denoted with capital latin letters) and an unique binary predicate $R$. Analogously to the modal $\mu$-calculus, the fixpoint extension of $\lque(\prop)$ is defined by adding a fixpoint construction clause.

\begin{definition}
The fixed point logic $\mlque(\prop)$ is given by:
$$
\varphi ::= q(x) \mid R(x,y) \mid x \foeq y \mid \exists x.\varphi \mid \qu x.\varphi \mid \lnot\varphi \mid \varphi \land \varphi \mid \mu p.\varphi(p,x)
$$
where $p,q\in\prop$, $x,y\in\fovar$; moreover $p$ occurs only positively in $\varphi(p,x)$ and $x$ is the only free variable in $\varphi(p,x)$.
\end{definition}

The semantics of the fixpoint formula $\mu p. \phi(p, x)$ is the expected one. Given a model $\model$ and $s \in T$,  $\model \models \mu p. \phi(p, s)$ iff $s$ is in the least fixpoint of the  operator $F_\phi:\wp(T)\to \wp(T)$ defined as $F_\phi(S) := \{t \in T \mid \model[p \mapsto S] \models \phi(p, t) \}$.

Formulas of $\mlque$ may be also classified according to their alternation depth as it happens for the modal $\mu$-calculus.
The alternation-free fragment of $\mlque$ is thence defined as the collection of $\mlque$-formulas $\phi$
without nesting of greatest and least fixpoint operators, i.e. such that, for any two subformulas $\mu p.\psi_1(p,y)$ and $\nu q. \psi_2(q,z)$, predicates $p$ and $q$ do not occur free respectively in $\psi_2(q,z)$ and $\psi_1(p,y)$.

%
%
\begin{definition}
Given $p \in \prop$, we say that $\varphi \in \mlque(\prop)$ is
\begin{itemize}
\item \emph{monotone in the predicate $p$} if for every LTS $\model$ and assignment $\ass$, if $\model, \ass \models \varphi$ and $\tsval(p) \subseteq E$, then $\model[p \mapsto E], g\models \phi$.

\item \emph{continuous in the predicate $p$} if for every LTS $\model$ and assignment $\ass$ there exists some finite $S \subseteq_\omega \tsval(p)$ such that $\model, \ass \models \varphi$ if and only if $\model[p \mapsto S], \ass \models \varphi$.
\end{itemize}
\end{definition}

In the next definition, we provide a definition of the continuous fragment of $\mlque$, reminiscent of the one defined in Theorem~\ref{thm:olquecont}.
\begin{definition}
Let $\mathsf{Q}\subseteq \prop$ be a set of monadic predicates. The fragment $\cont{\mlque}{\mathsf{Q}}(\prop)$ is defined by the following rules:
$$
\varphi ::= \psi \mid q(x) \mid \exists x.\varphi(x) \mid \varphi \land \varphi \mid \varphi \lor \varphi \mid \wqu x.(\varphi,\psi) \mid \mu p. \phi'(p, x)
$$
where $q \in \mathsf{Q}$, $\psi \in \mlque(\prop\setminus \mathsf{Q})$, $p \in \prop \setminus \mathsf{Q}$, $\wqu x.(\varphi,\psi) := \forall x.(\varphi(x) \lor \psi(x)) \land \dqu x.\psi(x)$ and $\phi'(p,x)$ is a formula with only $x$ free such that $\phi'(p,x) \in \cont{\mlque}{\mathsf{Q} \cup\{p\}}(\prop)$.

%The $\contAFMC$-fragment of $\mlque$  is  obtained by adding to $\glque$ the following (semantic) rule for constructing fixed point formulas.
% \begin{itemize}
% \item given a monadic predicate letter $P$, a first-order variable $x$, and a formula $\phi(P, x)$  that contains only positive occurrences of $P$ and no free variable other than $x$, if $\phi(P,x)$ is a formula in the fragment that is continuous in $P$ then $\mu P. \phi(P, x)$ is also a formula of  the fragment. Dually for $\nu P. \phi(P, x)$.\fcwarning{`positivity' is syntactic but `continuity' is semantic}
% \end{itemize}
\end{definition}

\begin{lemma}\label{lem:colqueiscont_mu}
If $\varphi \in \cont{\mlque}{\mathsf{Q}}(\prop)$ then $\varphi$ is continuous in (each predicate in) $\mathsf{Q}$.
\end{lemma}
%
\begin{proof} First, notice that If $\varphi \in \cont{\mlque}{\mathsf{Q}}(\prop)$ then $\varphi$ is monotone  in (each predicate from) $\mathsf{Q}$. %This is proved as for
The proof goes then by induction on the complexity of $\varphi$. For the all the cases except the fixpoint one, the proof is the same as the one for Lemma~\ref{lem:colqueiscont}. For $\phi=\mu p. \phi'(p, x)$, with $\phi'(p,x) \in \cont{\mlque}{\mathsf{Q} \cup\{p\}}(\prop)$, the argument is the same as in~\cite[Lemma 1]{Fontaine08}.
\end{proof}

As for the modal $\mu$-calculus, we define the fragment $\clque$ of $\mlque$ as the one where the use of the least fixed point operator is restricted to the continuous fragment. %, that the one obtained by adding to $\lque$ the following rule for constructing fixed point formulas.
 % \begin{itemize}
 % \item given a monadic predicate letter $P$, a first-order variable $x$, and a formula $\phi(P, x)$  that contains only positive occurrences of $P$ and no free variable other than $x$, if $\phi(P,x)$ is a formula in the fragment that belongs to $\cont{\mlque}{\{P\}}(\prop)$, then $\mu P. \phi(P, x)$ is also a formula of  the fragment.
 % \end{itemize}

\begin{definition}
The fragment $\clque(\prop)$ of $\mlque(\prop)$ is given by the following restriction of the fixpoint operator to the contiuous fragment:
{\small%
$$
\varphi ::= q(x) \mid R(x,y) \mid x \foeq y \mid \exists x.\varphi \mid \qu x.\varphi \mid \lnot\varphi \mid \varphi \land \varphi \mid \mu p.\varphi'(p,x)
$$}%
where $p,q\in\prop$, $x,y\in\fovar$; and $\varphi'(p,x) \in \cont{\mlque}{\{p\}}(\prop) \cap \clque(\prop)$ is such that $p$ occurs only positively in $\varphi'$ and $x$ is the only free variable in $\varphi'$.
\end{definition}

 %%%%%%%%
We now recall a useful property of fixpoint and continuity. Let $\phi(p,x)$ a formula with only $x$ free.
Given a LTS $\model = \tup{T,R,\tscolors,s_I}$, for every ordinal $\alpha$, we define by induction the following sets, with $\lambda$ a limit ordinal.
\begin{equation*}
\phi^0(\emptyset):= \emptyset \qquad \phi^{\alpha+1}(\emptyset):= \{ s \in T \mid \model[p \mapsto \phi^\alpha(\emptyset)] \models \phi(p, s)\} \qquad \phi^{\lambda}(\emptyset):= \bigcup_{\alpha < \lambda} \phi^{\alpha}(\emptyset)
\end{equation*}
If $\phi$ is monotone in $p$, it is possible to show that $\phi^{\beta+1}(\emptyset)= \phi^{\beta}(\emptyset)$, for some ordinal $\beta$. Moreover, the set $\phi^{\beta}(\emptyset)$ is the least fixpoint of $F_\phi$ (see e.g. \cite{ArnoldN01}).



A formula $\phi(p, x)$ is said to be \emph{constructive} in $p$ if its least fixpoint is reached in at most $\omega$ steps, i.e., if for every model $\model$, the least fixpoint of $F_\phi$ equals to $\bigcup_{\alpha < \omega} \phi^{\alpha}(\emptyset)$. From a local perspective, this means that a formula $\phi(p, x)$ constructive in $p$ if for every model $\model$,  every node $s \in T$, whenever $\mu p. \phi(p,x)$ is true at $s$, then $s$ belongs to some finite approximant $\phi^{i+1}(\emptyset)$ of the least fixpoint of $F_\phi$.
The next proposition is easily verified:% by the fact that Scott proved in \cite{Fontaine08} for the modal $\mu$-calculus but that generalizes to $\mglque$ as well, states that continuous formulas are constructive.

\begin{proposition}\label{prop:constructivity}
Let $\phi(p,x)$ be a $\mlque$-formula with only $x$ free. If $\phi(p,x)$ is continuous in $p$, then for every LTS $\model$, and every node $s \in T$, there is $i < \omega$ such that
\[\model \models \mu p. \phi(p,s) \text{ iff } s \in \phi^{i+1}(\emptyset).\]
\end{proposition}

From the fact that sets $\phi^{i+1}(\emptyset)$ are essentially defined as finite unfoldings and the previous Proposition~\ref{prop:constructivity}, we obtain the following.\fcwarning{More intuition on this?}

\begin{proposition}\label{prop:cor_constructivity}
Let $\phi(p,x)$ be a $\mlque$-formula with only $x$ free and such that $\phi(p,x)$ is continuous in $p$. Let $\model$ be a LTS, and $s \in T$. Then
$\model \models \mu p. \phi(p,s)$ iff there is a finite set $p^\model \subseteq_\omega T$ such that $s\in p^\model$ and $\model[p\mapsto p^\model] \models \phi(p,t)$  for every $t \in p^\model$.
\end{proposition}
 \begin{proof}
 For the direction from left to right, assume that $\model \models \mu p. \phi(p,s)$. By Proposition~\ref{prop:constructivity}, we know that  there is $i< \omega$ such that $\model[p \mapsto \phi^i(\emptyset)] \models \phi(p, s)$. The set $\phi^i(\emptyset)$ need not to be finite. However,
 using this information, we are going construct a finite tree whose nodes $t$ are labelled by finite sets $X^m_j$, where $m$ is a node of $\model$ and $j \leq i$, satisfying the following condition:
 \begin{enumerate}
\item  if $t$ is the root, then $t$ is labelled by $X_i^s$,
\item  if $t$ is labelled by $X_j^m=\{s_1, \dots, s_\ell\}$ and $j>0$, then $t$ has $\ell$  children and for every $s_i \in X_j^m$ there is an unique child $t'$ of $t$ labelled by $X_{j-1}^{n_i}$ where $m$ is a node,
%\item if $s$ is labelled by $X_j^m$ and $j=-1$, then $X_j^m=\emptyset$,
\item for every node $t$ of the tree, if $t$ is labelled by $X_j^m$, then it holds that $X_j^m \subseteq \phi^{j}(\emptyset)$.
\end{enumerate}
If we verify that $\model[p\mapsto p^\model] \models \phi(p,s)$ holds by taking as $p^\model$ the union of all labels of the nodes of the constructed tree, we can conclude for the proof of this direction.

As starting point of the inductive construction, we start by the empty tree.  Recall that we know that  $\model[p \mapsto \phi^i(\emptyset)] \models \phi(p, s)$. Since $\phi(p,x)$ is continuous in $p$, there is a finite set $X^s_i \subseteq \phi^i(\emptyset)$ such that $\model[p \mapsto X^s_i] \models \phi(p, s)$. We then add a root to our tree and label it by $X^s_i$.
 Assume that at a leaf $s$ of our tree is labelled by $X^m_j$, for some $j < i$. If $X^m_j$ is empty, than we stop, else we proceed as follows. We know that $X^m_j\subseteq \phi^{j}(\emptyset)$. This means that $\model[p \mapsto \phi^{j-1}(\emptyset)] \models \phi(p, r)$, for every $r \in X_j^m$. By continuity, for each such $r$, there is a finite set $X^m_{j-1} \subseteq  \phi^{j-1}(\emptyset)$ such that $\model[p \mapsto X^m_{j-1}(\emptyset)] \models \phi(p, r)$. For each $r \in X^m_j$ we thus add a child to $m$ and label it with $X^r_{j-1}$. By definition of $\phi^{i+1}(\emptyset)$, the tree is finite. Let $X$ be the union of all labels of the constructed tree. $X$ is finite, and by monotonicity of $\phi(p,x)$ we have that for every $m \in X \cup \{s\}$, $\model[p \mapsto X \cup \{s\}] \models \phi(p,m)$.

For the other direction, it's enough to notice that the smallest finite set $p^\model \subseteq T$ such that $\model[p\mapsto p^\model] \models \phi(p,s)$ and $\model[p\mapsto p^\model] \models \phi(p,m)$ for all $m \in p^\model$ is the least fixpoint $F_\varphi$. %of the function that maps any $S \subseteq T$ into $\{t \in T \mid \model[P \mapsto S] \models \phi(P, t) \}$.
%the idea is the following. By assumption there is a finite set $P^\model \subset T$ such that $\model[P\mapsto P^\model] \models \phi(P,n)$ and $\model[P\mapsto P^\model] \models \phi(P,m)$  for every $m \in P^\model$. The winning strategy for \'Eloise  in $\mc{E}(\mu P.\varphi(P,x),\model)@(\mu P.\varphi(P,x); x \mapsto n)$
%is thus define as the composition of all winning strategies in $\mc{E}(\varphi(P,x),\model[P \mapsto P^\model]))@(\varphi(P,x); x \mapsto m)$
% for $m \in P^\model$.
 \end{proof}

%The previous
\noindent Proposition~\ref{prop:cor_constructivity} naturally suggests the following translation $\mgFOETr{-}:\mlque(\prop)\to\wmso(\prop)$,
\begin{gather*}
\mgFOETr{p(x)} \df p(x) \qquad
\mgFOETr{R(x,y)} \df R(x,y) \qquad
\mgFOETr{x\foeq y} \df  (x \foeq y) \qquad
\mgFOETr{\varphi \land \psi} \df \mgFOETr{\varphi} \land \mgFOETr{\psi} \\
\mgFOETr{\lnot \varphi} \df  \lnot \mgFOETr{\varphi} \qquad\qquad
\mgFOETr{\exists x. \varphi} \df  \exists x. \mgFOETr{\varphi} \qquad\qquad
\mgFOETr{\qu x. \varphi} \df  \forall p.\exists x. (\lnot p(x) \land \mgFOETr{\varphi}) \\
\mgFOETr{\mu p. \varphi(p,x)} \df  \exists p ( p(x) \land \forall y ( p(y) \to \mgFOETr{\varphi(p,y) }))
\end{gather*}
%Note that in $\mgFOETr(\mu P. \varphi(P,x))$, the predicate $P$ which occurs in $\mgFOETr(\varphi) $ is bounded by the outermost second order existential quantifier.
%
The following theorem %, which is the analogous of Theorem \ref{thm:contransweak} but for $\mlque$,
is then an immediate corollary of Proposition~\ref{prop:cor_constructivity}.

\begin{theorem}\label{thm:guard_wmso}
For every $\phi \in \clque$ and model $(\model,\ass)$, $(\model, \ass) \models \varphi$ iff $(\model, \ass) \models \mgFOETr{\varphi}$.
%
% \begin{enumerate}
% \itemsep 0pt
% \item $\model, \ass \models \varphi$,
% \item $\model, \ass \models \mgFOETr{\varphi}$.
% \end{enumerate}
%
\end{theorem}
\begin{proof}
The proof goes by induction on the complexity of $\varphi$, the only critical step being the least fixpoint operator one. But this follows by applying Proposition \ref{prop:cor_constructivity} and the induction hypothesis.
%
%Let therefore consider $\phi$ is of the form $\mu P. \psi(P,x)$. Without loss of generality, that bounded and free predicate variables are distincts.
%We first show that $(1)$ implies $(2)$. Since $\model , \ass \models \varphi$, \'Eloise has a winning strategy $f$ in $\mc{E}(\phi,\model)@(\varphi,s_I, \ass)$.
%Define $P^\model$ to be the set of node $n \in T$ such that there is a (partial) match $\pi'$ that
%%
%\begin{enumerate}
%\itemsep 0pt
%\item is consistent with $f$, and such that
%\item every position of $\pi'$ is of the form $(\gamma,m, \ass')$, with  $P$ active in $\gamma$, and
%\item the last position of $\pi'$ is of the form $(\varphi, n, \ass')$.
%\end{enumerate}
%%
%The first observation is that since $f$ is a winning strategy, all $f$-consistent matches are finite. Moreover for every position of $\pi'$ is of the form $(\psi(P,x),m, \ass')$, we have that $\model[P \mapsto P^{\model}], \ass' \models \psi(P,m)$. We construct inductively a finite tree labelled by pairs $(x, X)$ where $x$ is a node of $\model$ and $X$ is a finite set of nodes of $\model$ as follows. First, because $\model[P \mapsto P^{\model}] , \ass \models \psi(P,x)$, so there is a finite subset $X \subseteq P^\model$ such that $\model[P \mapsto X_1] , \ass \models \psi(P,x)$. Thus we color the root with $(n, X)$. Now, assume we are given a leaf colored by $(y,Y)$. Consider an enumeration $x_1, \dots, x_k$ of $Y$. For every $i \leq k$, we add a child to $(y,Y)$ labelled by $(x_i, X_i)$ where $X_i$ is given by the fact that since $\model[P \mapsto P^{\model.x_i}] , \ass \models \psi(P,x)$, there is a finite set $X_i$ of nodes in $P^{\model.x_i}$
%
%%the only player who picks successor in a partial match $\pi'$ defined as above is \'Eloise. As a consequence of K\"onig's Lemma, $P^\model$ is finite.
%%
%%By using the induction hypothesis, it is easy to check that $\model[x \mapsto s_I, P \mapsto P^\model] \models P(x) \land \forall y ( P(y) \to ST_y(\varphi) )$.
%%
%%For the other direction, the idea is the following. Because $\model[x \mapsto s_I] \models ST_x(\varphi)$,
%%there is a finite set $P^\model$ such that $\model[x \mapsto s_I, P \mapsto P^\model] \models P(x) \land \forall y ( P(y) \to ST_y(\varphi) )$. The winning strategy for \'Eloise  in $\mc{E}(\mu P.\varphi,\model)@(\mu P.\varphi,s_I)$
%%is thus define as the composition of all winning strategies in $\mc{E}(\varphi,\model[P \mapsto P^\model])@(\varphi,s)$ for $s \in P^\model$.
\end{proof}

%\begin{remark}
%Clearly the standard translation from modal logic into $\gfoe$ extend to the modal $\mu$-calculus and $\mglque$.
%\end{remark}

\subsubsection{Translating automata into formulas}
We are now ready to formulate the converse of Theorem~\ref{t:wmsoauto}, i.e. the direction from automata to formulas of the automata characterisation of $\wmso$.

\begin{theorem}\label{thm:wmso_autofor}
There is an effective procedure that given a $\wmso$-automaton returns an equivalent $\wmso$-formula.
\end{theorem}
\begin{proof}
The argument is essentially a refinement of the standard proof showing that any automaton in $\Aut(\ofo)$ can be translated into an equivalent $\mu$-formula
$\xi_\aut$ (cf. e.g. \cite{Ven08}).
The idea is the following. We see a $\wmso$-automaton as a system of equations expressed in terms of $\lque$-formulas: each state corresponds to a monadic predicate variable and the parity of a state corresponds to the least and greatest fixpoint that we seek for the associated variable, etc. One then solves this system of equations via the same inductive procedure used to obtain the formula of the modal $\mu$-calculus from the system associated with a $\Aut(\ofo)$-automaton (see e.g. \cite{ArnoldN01} for a description of the solution procedure). Because of the (weakness) and (continuity) conditions on the starting $\wmso$-automaton $\aut$, it is thence possible to verify that the resulting fixpoint formula $\xi_\aut$ belongs to $\clque$.
\end{proof}

\begin{remark}
As a corollary of the automata characterization on trees of $\wmso$, we obtain the equivalence on this class of structures between $\wmso$ and $\clque$. This consequence should be compared to the analogous result obtained by Walukiewicz in~\cite{Walukiewicz96} for FPL (fixpoint extension of $\foe$) and MSO on trees.
\end{remark}


\section{Automata Characterisation of $\nmso$}\label{sec:autnmso}

%!TEX root = ../00CFVZ_TOCL.tex
In this section we work with the members of the class $\AutW(\ofoe)$, which we henceforth call \emph{$\nmso$-automata}.


\begin{theorem}
\label{t:nmsoauto}
There is an effective construction transforming a $\nmso$-formula $\phi$
into a $\nmso$-automaton $\bbA_{\phi}$ that is equivalent
to $\phi$ on the class of trees.
That is, for any tree $\bbT$, $\bbA_{\phi}$ accepts $\bbT$ if and only if $\bbT \models {\phi}$
\end{theorem}

The rest of this section will be devoted to prove that $\nmso$-automata
characterise $\nmso$ on tree models, as expressed in Theorem~\ref{t:mt2}.
First, we focus on showing the direction from formulas to automata.
In subsections~\ref{sec:finitconstr} and \ref{sec:closureautomata} we provide
the automata constructions handling the challenging case, that is the
translation of an existential formula $\noetexists p.\psi$ of $\nmso$ into an
equivalent $\nmso$-automaton.
To this aim, we define a closure operation on tree languages corresponding
to the semantics of $\nmso$ quantification.

\begin{definition}\label{def:tree_finproj}
Let $\prop$ be a set of proposition letters, $p \not\in P$ be a proposition letter, and $\trees$ be a tree language of $\p (\prop\cup\{p\})$-labeled
trees.
The \emph{noetherian projection} of $\trees$ over $p$ is the language
$\noetexists p.\trees$ of $\p (\prop)$-labeled trees %$\model$ for which some
%noetherian $p$-variant of $\model$ exists in $\trees$.
given as follows:
%
$$
\noetexists p.\trees = \{\model \mid \text{ $\exists$ a noetherian $p$-variant } \model' \text{ of } \model \text{ with } \model' \in \trees\}.
$$
%
A class $K$ of tree languages is \emph{closed under noetherian projection
over $p$} if, for any language $\trees$ in $K$, also ${\noetexists p}.\trees$ is in $K$.
% A class $K$ is \emph{closed under noetherian projection
% over $p$} if $\trees\in K$ implies ${{\exists}_F p}.\trees \in K$.
\end{definition} 

\subsubsection{Simulation theorem}


Our next goal is a \emph{projection construction} that, given
a $\nmso$-automaton $\aut$, provides one recognizing ${\noetexists p}.\trees(\aut)$. For $\mso$-automata, an analogous construction crucially uses the following \emph{simulation theorem}: every
$\mso$-automaton $\aut$ is equivalent to a \emph{non-deterministic} one $\aut'$ \cite{Walukiewicz96}. Semantically, non-determinism yields the appealing property that any strategy $f$ for player $\exists$ in the acceptance game $\agame(\aut',\model)$ can be assumed to be functional in $A'$ (\emph{cf.} Definition \ref{def:partityaut}).
This is particularly helpful because, to define a $p$-variant of $\model$
that is accepted by the projection construct on $\aut'$, we
can infer whether a node $s$ should be labeled with $p$ by the value $f(a,s)$, where $a$ is the unique state of $\aut'$ (by functionality) that $f$ associates with $s$.

The simulation theorem for $\mso$-automata does not restrict to the setting of $\nmso$-automata, as it does not preserve the \emph{(weakness)} condition. In the sequel, we are going to show that a restricted version of non-determinism suffices for our purposes. Indeed, guessing a \emph{noetherian} $p$-variant of the input tree, as prescribed by $\nmso$ quantification, only requires a winning strategy $f$ to be functional for finitely many rounds of the acceptance game. Thus the idea of our simulation theorem is to turn any $\nmso$-automaton into an equivalent one that behaves non-deterministically on a \emph{well-founded} portion of any accepted tree.

For $\mso$-automata, the simulation theorem is based on a powerset construction: if the starting automaton has carrier $A$, the resulting non-deterministic automaton is based on ``macro-states'' from the set $\shA := \pw (A \times A)$.\footnote{The use of carrier $\pw (A \times A)$ instead of the more obvious $\pw A$ is needed to correctly associate with a run on macro-states the corresponding bundle of runs of the original automaton $\aut$ (\emph{cf.} \cite{Walukiewicz96}).} Analogously, for $\nmso$-automata we will associate the non-deterministic behaviour with macro-states. As explained, our desiderata are that the simulating automating $\aut^{\noet}$ is non-deterministic just on a well-founded portion of the input and may behave as $\aut$ (i.e. in ``alternating mode'') on the others. To this aim, $\aut^{\noet}$ will be ``two-sorted'', roughly consisting of one copy of $\aut$ (with carrier $A$) and a variant of its powerset construction, based both on $A$ and $\shA$. For any accepted $\model$, the idea is to make any match $\pi$ of $\agame(\aut^{\noet},\model)$ consist of two parts:
\begin{description}
  \item[(\textbf{Non-deterministic mode})] for finitely many rounds, $\pi$ only visits macro-states (from $\shA$) and $\eloise$ plays according to a functional strategy, meaning that she assigns \emph{at most one macro-state} (and no regular state, from $A$) to each node.%each visited basic position has shape $(q,s) \in \shA \times T$. The valuation $\val \colon A \cup \shA \to \pw (R[s])$ picked by player $\exists$ assigns macro-states (from $\shA$) only to a \emph{finite} subset of $\R{s}$ and states (from $A$) to the rest of $\R{s}$. Also, she assigns \emph{at most one macro-state} to each node.
  \item[(\textbf{Alternating mode})] At a certain round, $\pi$ abandons macro-states and turns into a match of the game $\mc{A}(\aut,\model)$, i.e. all next positions are from $A \times T$ (and are played according to a non-necessarily functional strategy). %of shape $(a,t) \in A \times T$.
\end{description}
Therefore successful runs of $\mb{A}^{\noet}$ will have the property of processing only a \emph{well-founded} amount of the input with $\mb{A}^{\noet}$ being in a macro-state and all the rest with $\mb{A}^{\noet}$ behaving exactly as $\aut$.

We now proceed in steps towards the construction of $\aut^{\noet}$. The following is a notion of lifting for types on states that is instrumental in defining a translation to types on macro-states. %The distinction between empty and non-empty subsets of $A$ is to make sure that empty types on $A$ are lifted to empty types on $\pw A$.
\begin{definition}
Given a set $A$ and $\Sigma \subseteq \wp A$, we define the \emph{lifting} $\lift{\Sigma} \subseteq \wp \wp A$ as $\{\{S\} \mid S \in \Sigma \wedge S \neq \emptyset\} \cup
    \{\emptyset \mid \emptyset \in \Sigma \}$.
%\begin{eqnarray*}
%\lift{\Sigma} & := & \{\{S\} \mid S \in \Sigma \wedge S \neq \emptyset\} \cup    \{\emptyset \mid \emptyset \in \Sigma \}.
%\end{eqnarray*}
\end{definition}

 The next definition is standard (see e.g.  \cite{Walukiewicz96,Ven08}) as an intermediate step to define the transition function of the powerset construct for parity automata. It simply \emph{tags} the (potential next) states occurring in $\tmap(a,c)$ with the information of the current state.

\begin{definition}\label{DEF_delta star} Let $\aut = \tup{A,\tmap,\pmap,a_I}$ be a $\nmso$-automaton. Fix $a \in A$, $c \in C$. The sentence $\tmap^{\star}(a,c) \in {\ofoe}^+(A\times A)$ is defined as $\tmap(a,c)[b \mapsto (a,b) \mid b \in A]$, where $\tmap(a,c)[b \mapsto (a,b) \mid b \in A] \in {\ofoe}^+(A\times A)$ is the sentence obtained by replacing each monadic predicate $b \in A$ in $\tmap(a,c)$ with the monadic predicate $(a,b) \in A \times A$.
\end{definition}

We now lift the one-step language of $\nmso$-automata from states (that we suppose in $A \times A$, by effect of Definition \ref{DEF_delta star}) to macro-states (in $\shA = \pw (A \times A)$).

\begin{definition}\label{DEF_finitary_lifting}
Let $\varphi \in {\ofoe}^+(A \times A)$ be of shape $\dbnfofoe{\vlist{T}}{\Pi}$ for some $\Pi \subseteq \shA$ and $\vlist{T} = \{T_1,\dots,T_k\} \subseteq \shA$. We define $\varphi^{\noet}$ as $\dbnfofoe{\lift{\vlist{T}}}{\lift{\Pi}} \in {\ofoe}^+(\shA )$.
\end{definition}

The idea of translation $(\cdot)^{\noet}$ is to encode at the one-step level the non-deterministic mode of $\aut^{\noet}$: the property to enforce is that $\varphi^{\noet}$ is \emph{functional} in $\shA$, that means, whenever $(D,\val \: A \to \wp(D)) \models \varphi^{\noet}$, then there is $\val'  \: A \to \wp(D)$ such that $(D, \val') \models \varphi^{\noet}$ and  $ \val'(q_1)\cap \val'(q_2) = \emptyset$ for all $q_1,q_2 \in \shA$. 

\begin{lemma}\label{LEM_cont}
Let $\varphi \in {\ofoe}^+(A \times A)$ and $\varphi^{\noet}\in {\ofoe}^+(\shA )$ be as in Definition~\ref{DEF_finitary_lifting}. Then $\varphi^{\noet}$ is functional in $\shA$.
 \end{lemma}

\begin{proof}
%\yvwarning{this proof could use some more detail FZ: I expanded the proof and tried to make it clearer}
We first unfold the definition of $\varphi^{\noet}$ as follows:
\begin{align*}
\varphi^{\noet} =\ &
\underbrace{
    \exists \vlist{x}.\big(\arediff{\vlist{x}} \land \bigwedge_{0 \leq i \leq n} \tau^+_{\lift{T}_i}(x_i)
}_{\psi_1}
\land \underbrace{
    \forall z.(\arediff{\vlist{x},z} \lthen \bigvee_{S\in \lift{\Pi}} \tau^+_S(z))\big)
}_{\psi_2}
\end{align*}
Now suppose that $(D,\val \: \shA \to \wp(D))$ is a model where $\varphi^{\noet}$ is true. This amounts to the truth of subformulas $\psi_1$ and $\psi_2$, whose syntactic shape yields information on the types of elements of $D$. In particular, we can define a partition of $D$ into subsets $D_1$ and $D_2$ as follows:
\begin{itemize}
  \item As $\psi_1$ is true, we can pick $n$ distinct elements $s_1,\dots,s_n$ of $D$ such that $s_i$ witnesses the positive type $\lift{T}_i$, %\tau^+_{\lift{T}_i}(x_i)$,
   that is, $s_i \in \val(S)$ for each $S \in \lift{T}_i$. We define $D_1 := \{s_1,\dots,s_n\}$.
  %
  \item  As $\psi_2$ is true, we can cover $D \setminus D_1$ with a set $D_2$ containing all the elements not in $D_1$ witnessing a type ${\tau}^{+}_S(z)$ with $S \in  \lift{\Pi}$. 
 \end{itemize}
This partition uniquely associates with each $s \in D$ a type ${\tau}^{+}_S$ witnessed by $s$ and thus a set of unary predicates $S_s := S \subseteq A \cup \shA$. We can then define a valuation $\val'$ assigning to each element $s$ of $D$ exactly the set $S_s$.

We now check the properties of $\val'$. As the partition inducing $\val'$ follows the syntactic shape of $\varphi^{\noet}$, one can observe that $\val'$ is a restriction of $\val$ and $(D,\val')$ makes $\varphi^{\noet}$ true. By definition of the partition, $\val'$ assigns at most one unary predicate from $\shA$ to each element of $D_1$, because $\lift{\vlist{T}} \cup \lift{\Pi}$ is defined as the lifting of $\vlist{T} \cup \Pi$. It follows that $\varphi^{\noet}$ is functional in $\shA$, and the property is preserved by the restriction $\val'$.
\end{proof}


 Next we combine the previous definitions to characterise the transition function associated with the macro-states.
\begin{definition}\label{PROP_DeltaPowerset}
Let $\aut = \tup{A,\tmap,\pmap,a_I}$ be a $\nmso$-automaton. Fix any $c \in C$ and $Q \in \shA$. By Theorem~\ref{thm:bnfofoe} there is a sentence $\Psi_{Q,c} \in {\ofoe}^+(A\times A)$ in the basic form $\bigvee \dbnfofoe{\vlist{T}}{\Pi}$, for some $\Pi \subseteq \shA$ and $T_i \subseteq A \times A$, such that
$$\bigwedge_{a \in \Ran(Q)} \tmap^{\star}(a,c) \equiv \Psi_{Q,c}.$$
By definition, $\Psi_{Q,c} = \bigvee_{n}\varphi_n$, with each $\phi_{k}$ of shape $\dbnfofoe{\vlist{T}}{\Pi}$.
%
We put $\shDe(Q,c) := \bigvee_{n}\varphi_n^{\noet}  \in {\ofoe}^+(\shA)$, where the translation $(\cdot)^{\noet}$ is as in Definition \ref{DEF_noetherian_lifting}.
\end{definition}

\noindent We have now all the ingredients for our two-sorted automaton.

\begin{definition}\label{def:noetherianconstruct}
Let $\aut = \tup{A,\tmap,\pmap,a_I}$ be a {\nmso-automaton}. We define the \emph{noetherian construct over $\aut$} as the automaton $\aut^{\noet} = \tup{A^{\noet},\tmap^{\noet},\pmap^{\noet},a_I^{\noet}}$ given by %taking $A^{\noet} :=A \cup \shA$, $a_I^{\noet} := \{(a_I,a_I)\}$ and
%{\small%
\begin{eqnarray*}
      % \nonumber to remove numbering (before each equation)
        A^{\noet} &:=& A \cup \shA \\
        a_I^{\noet} &:=& \{(a_I,a_I)\}\\
        \tmap^{\noet}(q,c) &:=& \left\{
	\begin{array}{ll}
        \tmap(q,c) & q\in A \\
		\shDe(q,c) \vee \bigwedge_{a \in \Ran(q)} \tmap(a,c) & q \in \shA
	\end{array}
\right.\\
        \pmap^{\noet}(q) &:=& \left\{
	\begin{array}{ll}
        \pmap(q) & \hspace{3.43cm} q\in A \\
		1 & \hspace{3.43cm} q \in \shA.
	\end{array}
\right.
\end{eqnarray*}%}
\end{definition}
The definition of $\aut^{\noet}$ enforces its behaviour to be split according to the non-deterministic and alternating mode. Indeed, for any accepted $\model$, a match $\pi$ of $\agame(\aut^{\noet},\model)$ will visit positions involving macro-states only for finitely many initial rounds, because $\pmap^{\noet}[\shA] = \{1\}$. The alternating mode will be entered when, at a certain position $(R,s)\in \shA \times T$, the winning strategy for $\exists$ makes the disjunct $\bigwedge_{a \in \Ran(R)} \tmap(a,c)$ of $\tmap^{\noet}(R,c)$ true and then all successive positions only involve states from $A$. The next proposition fixes our desiderata on $\aut^{\noet}$. 

\begin{definition}\label{def:noetherianstrategy}
We say that a strategy $f$ in an acceptance game $\agame(\aut,\model)$ is \emph{noetherian} in $B \subseteq A$ when in any $f$-guided match there can be only finitely many rounds played at a position of shape $(q,s)$ with $q \in B$.
\end{definition}

%: in particular, \ref{point:finConstrStrategy} certifies the description that we did of the non-deterministic mode of $\aut^{f}$.

\begin{proposition}[\textbf{Simulation Theorem for $\nmso$-automata}]\label{PROP_facts_finConstr} Let $\aut$ be a $\nmso$-automaton and $\aut^{\noet}$ its noetherian construct.
\begin{enumerate}[(i)]
  \itemsep 0 pt
  \item $\aut^{\noet}$ is a $\nmso$-automaton.\label{point:finConstrAut}
  \item For any $\model$, if $\exists$ has a winning strategy in $\mathcal{A}(\aut^{\noet},\model)$ from position $(a_I^{\noet},s_I)$ then she has one that is functional in $\shA$ and noetherian in $\shA$.% (\emph{cf.} Definition \ref{def:StratfunctionalFinitary}).
  \label{point:finConstrStrategy}
  \item $\aut \equiv \aut^{\noet}$. \label{point:finConstrEquiv}
  \end{enumerate}
\end{proposition}
\begin{proof}
\begin{enumerate}[(i)]
  \item We need to check that $\tmap^{\noet}$ is weak. If $q_1 \ord q_2 \ord q_1$ in $\aut^{\noet}$ then by definition of $\tmap^{\noet}$ either $q_1, q_2 \in A$ or $q_1, q_2 \in \shA$. In the first case, $\pmap^{\noet}(q_1) = \pmap(q_1) = \pmap(q_2) = \pmap^{\noet}(q_2)$ because the original automaton $\aut$ is weak. In the latter case, $\pmap^{\noet}(q_1) = 1 = \pmap^{\noet}(q_2)$ by definition of $\pmap^{\noet}$. 
\item Let $f$ be a winning strategy for $\exists$ in $\mathcal{A}(\mb{A}^{\noet},\model)@(a_I^{\noet},s_I)$. We define a strategy $f'$ for $\exists$ in the same game as follows:
      \begin{enumerate}[label=(\alph*),ref=\alph*]
        \item on basic positions of the form $(a,s) \in A\times T$, we let $f'$ suggest the restriction $\val'$ to $A$ of the valuation $\val$ suggested by $f$. As no predicate from $A^{\noet}\setminus A =\shA$ occurs in $\Delta^{\noet}(a,\V(s)) = \Delta(a,\V(s))$, then $\val'$ also makes that sentence true in $\R{s}$.
        \begin{comment} With minimality
        on basic positions of the form $(a,s) \in A\times T$, $f'$ is defined as $f$. Indeed, as no predicate from $\shA$ occurs in $\Delta^{\noet}(a,\V(s))$, we can assume that the valuation suggested by $f$ does not assign any of them to nodes in $\R{s}$.
        \end{comment}
        \label{point:stat2point1}
        \item for basic positions of the form $(R,s) \in \shA \times T$, let $\val_{R,s}$ be the valuation suggested by $f$. As $f$ is winning, $\Delta^{\noet}(R,\V(s))$ is made true by $\val_{R,s}$. If this is because the disjunct $\bigwedge_{a \in \Ran(R)} \Delta(a,\V(s))$ is made true, then we can let $f'$ suggest the restriction to $A$ of $\val_{R,s}$, for the same reason as in \eqref{point:stat2point1}. Otherwise, the disjunct $\shDe(R,\V(s)) = \bigvee_{i}\varphi_i^{\noet}$ is made true. This means that, for some $i$, $(R[s], \val_{R,s}) \models \varphi_i^{\noet}$. By Lemma \ref{LEM_cont} $\varphi_i^{\noet}$ is functional in $\shA$, meaning that we have a restriction $\val_{R,s}'$ of $\val_{R,s}$ that verifies $\varphi_i^{\noet}$, assigns finitely many nodes to predicates from $\shA$ and associates with each node at most one predicate from $\shA$. We let $\val_{R,s}'$ be the suggestion of $f'$ from position $(R,s)$.
      \end{enumerate}
      The strategy $f'$ defined as above is immediately seen to be
      surviving for $\exists$. It is also winning, because the set of
      basic positions on which $f'$ is defined is a subset of the one
      of the winning strategy $f$. By this observation it also follows that any $f'$-conform match visits basic positions of the form $(R,s) \in \shA \times C$ only finitely many times, as those have odd parity.
  \item For the direction from left to right, it is immediate by definition of $\mb{A}^{\noet}$ that a winning strategy for $\exists$ in $\mc{G} = \mathcal{A}(\aut,\model)@(a_I,s_I)$ is also winning for $\exists$ in $\mc{G}^{\noet} = \mathcal{A}(\mb{A}^{\noet},\model)@(a_I^{\noet},s_I)$.

      For the direction from right to left, let $f$ be a winning strategy for $\exists$ in $\mc{G}^{\noet}$. The idea is to define a strategy $f'$ for $\exists$ in stages, while playing a match $\pi'$ in $\mc{G}$. In parallel to $\pi'$, a shadow match $\pi$ in $\mc{G}^{\noet}$ is maintained, where $\exists$ plays according to the strategy $f$. For each round $z_i$, we want to keep the following relation between the two matches:
\smallskip
\begin{center}
\fbox{\parbox{12cm}{
Either
\begin{enumerate}[label=(\arabic*),ref=\arabic*]
  \item basic positions of the form $(Q,s) \in \shA \times T$ and $(a,s) \in A \times T$ occur respectively in $\pi$ and $\pi'$, with $a \in \Ran(Q)$,
\end{enumerate}
or
\begin{enumerate}[label=(\arabic*),ref=\arabic*]
  \item[(2)] the same basic position of the form $(a,s) \in A \times T$ occurs in both matches.
\end{enumerate}
}}\hspace*{0.3cm}($\ddag$)
\end{center}
\smallskip
The key observation is that, because $f$ is winning, a basic position of the form $(Q,s) \in \shA \times T$ can occur only for finitely many initial rounds $z_0,\dots,z_n$ that are played in $\pi$, whereas for all successive rounds $z_n,z_{n+1},\dots$ only basic positions of the form $(a,s) \in A \times T$ are encountered. Indeed, if this was not the case then either $\exists$ would get stuck or the minimum parity occurring infinitely often would be odd, since states from $\shA$ have parity $1$.

It follows that enforcing a relation between the two matches as in ($\ddag$) suffices to prove that the defined strategy $f'$ is winning for $\exists$ in $\pi'$. For this purpose, first observe that $(\ddag).1$ holds at the initial round, where the positions visited in $\pi'$ and $\pi$ are respectively $(a_I,s_I) \in A \times T$ and $(\{(a_I,a_I)\},s_I) \in A^{\noet} \times T$. Inductively, consider any round $z_i$ that is played in $\pi'$ and $\pi$, respectively with basic positions $(a,s) \in A \times T$ and $(q,s) \in A^{\noet} \times T$. In order to define the suggestion of $f'$ in $\pi'$, we distinguish two cases.
\begin{itemize}
  \item First suppose that $(q,s)$ is of the form $(Q,s) \in
  \shA\times T$. By ($\ddag$) we can assume that $a$ is in $\Ran(Q)$. Let $\val_{Q,s} :A^{\noet} \rightarrow \wp(\R{s})$ be the valuation suggested by $f$, verifying the sentence $\Delta^{\noet}(Q,\V(s))$. We distinguish two further cases, depending on which disjunct of $\Delta^{\noet}(Q,\V(s))$ is made true by $\val_{Q,s}$.
      \begin{enumerate}[label=(\roman*), ref=\roman*]
        \item If $(\R{s},\val_{Q,s})\models \bigwedge_{b \in \Ran(Q)} \Delta(b,\V(s))$, then we let $\exists$ pick the restriction to $A$ of the valuation $\val_{Q,s}$. \label{point:valuation1}
        \item If $(\R{s},\val_{Q,s})\models \shDe(Q,\V(s))$, we let $\exists$ pick a valuation $\val_{a,s}:A \rightarrow \p (\R{s})$ defined by putting, for each $b \in A$:
            \begin{align*}
            % \nonumber to remove numbering (before each equation)
               \val_{a,s}(b)\ :=\ \bigcup_{b \in \Ran(Q')} &\{t \in \R{s} \mid t \in \val_{Q,s}(Q')\} \\
               \cup\ \ \ \ \ & \{t \in \R{s} \mid t \in \val_{Q,s}(b)\} .
            \end{align*} \label{point:valuation2}
      \end{enumerate}
      It can be readily checked that the suggested move is admissible for $\exists$ in $\pi$, i.e. it makes $\Delta(a,\V(s))$ true in $\R{s}$. For case \eqref{point:valuation2}, one has to observe how $\shDe$ is defined in terms of $\Delta$. In particular, the nodes assigned to $b$ by $\val_{Q,s}$ have to be assigned to $b$ also by $\val_{a,s}$, as they may be necessary to fulfill the condition, expressed with $\qu$ and $\dqu$, that infinitely many nodes witness (or that finitely many nodes do not witness) some type.

      We now show that $(\ddag)$ holds at round $z_{i+1}$. If \eqref{point:valuation1} is the case, any next position $(b,t)\in A \times T$ picked by player $\forall$ in $\pi'$ is also available for $\forall$ in $\pi$, and we end up in case $(\ddag .2)$. Suppose instead that \eqref{point:valuation2} is the case. Given the choice $(b,t) \in A \times T$ of $\forall$, by definition of $\val_{a,s}$ there are two possibilities. First, $(b,t)$ is also an available choice for $\forall$ in $\pi$, and we end up in case $(\ddag .2)$ as before. Otherwise, there is some $Q' \in \shA$ such that $b$ is in $\Ran(Q')$ and $\forall$ can choose $(Q',t)$ in the shadow match $\pi$. By letting $\pi$ advance at round $z_{i+1}$ with such a move, we are able to maintain $(\ddag .1)$ also in $z_{i+1}$.
  \item In the remaining case, inductively we are given the same basic position $(a,s) \in A\times T$ both in $\pi$ and in $\pi'$. The valuation $\val$ suggested by $f$ in $\pi$ verifies $\Delta^{\noet}(a,\V(s)) = \Delta(a,\V(s))$, thus we can let the restriction of $\val$ to $A$ be the valuation chosen by $\exists$ in the match $\pi'$. It is immediate that any next move of $\forall$ in $\pi'$ can be mirrored by the same move in $\pi$, meaning that we are able to maintain the same position --whence the relation $(\ddag.1)$-- also in the next round.
\end{itemize}
In both cases, the suggestion of strategy $f'$ was a legitimate move for $\exists$ maintaining the relation $(\ddag)$ between the two matches for any next round $z_{i+1}$. It follows that $f'$ is a winning strategy for $\exists$ in $\mc{G}$.
%
      \begin{comment} SHORTER ALTERNATIVE VERSION OF THE PROOF
      The idea is to define a strategy $f'$ for $\exists$ in stages, while playing a match $\pi'$ in $\mathcal{A}(\aut,\model)@(a_I,s_I)$. In parallel to $\pi'$, a shadow match $\pi$ in $\mathcal{A}(\mb{A}^{\noet},\model)@(a_I^{\noet},s_I)$ is maintained, where $\exists$ plays according to the strategy $f$. Since $f$ is winning and all macro-states from $\shA$ have an odd parity, in finitely many rounds the shadow match $\pi$ reaches a stage where $\mb{A}^{\noet}$ enters a state from $A$ and ``behaves as'' $\aut$ for all successive rounds. Thus $\pi$ can be assumed to have the following structure:
       \begin{enumerate}[(I)]
         \item there is an $n$ such that, for each round $z_i$ in the initial segment $z_0,z_1,\dots,z_n$ of $\pi$, a position of the form $(R,s) \in \shA \times T$ is visited and the valuation suggested by $f$ makes the disjunct $\shDe(R,\V(s))$ of $\Delta^{\noet}(R,\V(s))$ true in $\R{s}$.
         \item At round $z_{n+1}$ a basic position of the form $(Q,t) \in \shA \times T$ is visited. The valuation suggested by $f$ makes the disjunct $\bigwedge_{a \in \Ran(R)} \Delta(a,\V(t))$ of $\Delta^{\noet}(Q,\V(t))$ true in $\R{t}$. \label{point:initialsegm}
         \item For all the next rounds $z_{n+2},z_{n+3},\dots$ only positions of the form $(a,s) \in A \times T$ are visited.
       \end{enumerate}
       In each round of the initial segment $z_0,z_1,\dots,z_n$ we can maintain the condition that, if a position $(R,s)$ is visited in $\pi$, then at the same round a position $(a,s)$ with $a \in \Ran(R)$ occurs in $\pi'$. This holds for the initial round $z_0$. For the next ones $z_1,\dots,z_n$, it can be enforced by defining $f'$ in terms of $f$ in the standard way shown, for instance, in the proof of \cite[Prop. 3.9]{Zanasi:Thesis:2012}.
       \fzwarning{More details to be provided}
       Once $\pi$ reaches round $z_n$, say with position $(Q,t)$, the valuation suggested by $f$ makes $\bigwedge_{a \in \Ran(R)} \Delta(a,\V(t))$ true in $\R{t}$ ({\it cf.} point \eqref{point:initialsegm}). By assumption, at round $z_n$, $\pi'$ visits a position $(b,t)$ with $b \in \Ran(Q)$. Then in particular the valuation suggested by $f$ makes $\Delta(b,\V(t))$ true, and we let it be the suggestion of $f'$ at that stage. By definition of $\Delta^{\noet}$, from the next round onwards we can maintain the same basic positions in $\pi$ and $\pi'$, and let $f'$ just be defined as $f$. As $\exists$ wins $\pi$, it will also win the match $\pi'$, meaning that $f'$ is a winning strategy.
       \end{comment}
\end{enumerate}
\end{proof}

\subsubsection{From formulae to automata}

We are now ready to introduce our projection construction for $\nmso$-automata and show that the class of tree languages that they recognize is closed under noetherian
projection.
\begin{definition}\label{DEF_fin_projection}
Let $\aut = \tup{A,\tmap,\pmap,a_I}$ be a $\nmso$-automaton on alphabet $\p(\prop \cup \{p\})$, with $p \not\in P$. Let $\mathbb{A}^{\noet}$
denote its noetherian construct.
We define the $\nmso$-automaton ${{\exists}_F p}.\aut := \langle A^{\noet}, a_I^{\noet},
\tmapProj, \pmap^{\noet}\rangle$ on alphabet $\p(\prop)$ by putting
\begin{eqnarray*}
\tmapProj(q,c) &:=& \left\{
	\begin{array}{ll}
        \tmap^{\noet}(q,c) & q\in A \\
		\tmap^{\noet}(q,c) \vee \tmap^{\noet}(q,c\cup\{p\}) & q \in \shA.
	\end{array}
\right.
\end{eqnarray*}
%The automaton ${{\exists}_F p}.\aut$ is called the \emph{noetherian projection construct of $\aut$ over $p$}.
\end{definition}
\begin{proposition}\label{PROP_fin_projection}
For every $\nmso$-automaton $\aut$ on alphabet ${\p (\prop \cup \{p\})}$, with $p \not\in P$, we have that $\trees({{\exists}_F p}.\aut) = {{\exists}_F p}.\trees(\aut)$.
\end{proposition}

\begin{proof} %The key is to observe
For the inclusion from left to right, first observe that ${{\exists}_F p}.\aut$ is defined in terms of $\aut^{\noet}$ and thus the properties stated in Proposition \ref{PROP_facts_finConstr} hold for ${{\exists}_F p}.\aut$ as well. In particular, given a ${\p (\prop)}$-tree $\model$, any winning strategy $f$ for $\exists$ in $\mathcal{A}({{\exists}_F p}.\aut, \model$) from position $(a_I^{\noet},s_I)$ can be assumed to be functional and noetherian in $\shA$. We can use such a strategy to guess a noetherian $p$-variant of $\model$ as follows. First, by functionality for each node $s$ there is at most one position $(Q_s,s)$, with $Q_s \in \shA$, that is reachable in any $f$-guided match. From each such position, let $\val_{Q_s,s} \colon A^{\noet} \to \pw (R[s])$ be the valuation suggested by $f$. We let $X_p$ be the set of nodes $s$ for which $\val_{Q_s,s}$ makes the disjunct $\tmap^{\noet}(Q_s,\tscolors(s)\cup\{p\})$ of $\tmapProj(Q_s,\tscolors(s))$ true: intuitively, these are the nodes on which ${{\exists}_F p}.\aut$ behaves ``as if they were labeled with $p$''. Since $f$ is noetherian in $\shA$, the $p$-variant $\model'$ of $\model$ given by labeling the nodes in $X_p$ with $p$ is noetherian. One can readily verify that $\aut^{\noet}$ (and thus $\aut$ by Proposition \ref{PROP_facts_finConstr}) accepts $\model'$ by letting $\exists$ playing the strategy $f$ in $\mathcal{A}(\aut^{\noet},\model')$.

For the inclusion from right to left, let $\model'$ be a noetherian $p$-variant of some ${\pw (\prop)}$-tree $\model$ and suppose that $\exists$ has a winning strategy $f$ for $\mathcal{A}(\aut,\model')$ from position $(a_I,s_I)$. We now sketch how $\exists$ is able to win any match $\pi$ of $\mathcal{A}({{\exists}_F p}.\aut,\model)$ from position $(a_I^{\noet},s_I)$. The idea is to enforce that, at each round of $\pi$, $\exists$ assigns macro-states only to the nodes rooting a subtree of $\model'$ where the labeling $p$ appears, and $A$-states to the others. Using the information given by $f$, $\exists$ can make this assignment so that any visited position of shape $(Q,s) \in \shA \times T$ is such that $(a,s)$ is winning for $\exists$ in $\mathcal{A}(\aut,\model')$, for each $a \in \Ran(Q)$. In particular, the assignment of $\exists$ will make true the disjunct $\tmap^{\noet}(Q,\tscolors(s)\cup\{p\})$ of $\tmapProj(Q,\tscolors(s))$ if $s$ is labeled with $p$ in $\model'$, and the disjunct $\tmap^{\noet}(Q,\tscolors(s))$ otherwise. Since $\model'$ is a \emph{noetherian} $p$-variant, player $\exists$ will be required to assign macro-states to only finitely many nodes encountered along the play, and $\pi$ will eventually arrive to a position from which no node labeled with $p$ is reachable. At that point, $\exists$ allows ${{\exists}_F p}.\aut$ to switch from the non-deterministic to the alternating mode. By construction, the match $\pi$ now moves to a position $(a,s) \in A \times T$ that is winning for $\exists$ in $\mathcal{A}(\aut,\model')$. It is also winning in $\mathcal{A}({{\exists}_F p}.\aut,\model)$, because $\model'$ agrees with $\model$ on nodes without labeling $p$ and by definition $\tmapProj(a,\tscolors(s)) = \tmap(a,\tscolors(s))$. %\emph{not} labeled with $p$,  and in this phase she maintains a bundle $\pi_{a_1},\dots,\pi_{a_n}$ of shadow matches of $\mathcal{A}(\aut,\model')$.  --- the idea is that, as long  We can define  where $\model'$ is on alphabet ${\p (\prop)}$. where $\exists$ plays according to $g$. As long as such a match visits positions of the form $(R,s) \in \shA \times T$, we are able to maintain a bundle of matches of $\mathcal{A}(\aut,\model)$, one for each $a \in \Ran(R)$, where the strategy to play is defined in terms of $g$. Since $g$ is winning, eventually it allows $\aut^{\noet}$ to enter the alternating mode, say in position $(a,s) \in A \times T$. At that point we can just keep the match in the bundle which is at position $(a,s)$ --- which exists by construction --- and keep copying the strategy $g$.
\end{proof}

We have now in position to show our characterisation result.
% \medskip

\begin{proof}[Proof of Theorem \ref{t:mt2}, direction $(\Rightarrow)$]
By induction we prove that for every $\varphi \in \nmso$ there is a $\nmso$-automaton $\aut_\varphi$ such that $\ext{\varphi} = \trees(\aut_\varphi)$. We focus on two %two non-trivial
inductive cases.
 %\begin{itemize}
   %\item

  \indent If $\varphi = \neg \psi$, let $\aut_{\psi}$ be the $\nmso$-automaton for $\psi$ given by inductive hypothesis. As $\ofoe$ is closed under Boolean duals, we can define the complementation $\overline{\aut_{\psi}}$ of $\aut_{\psi}$ as in Definition \ref{d:caut}. Notice that $\overline{\aut_{\psi}}$ is indeed a $\nmso$-automaton, satisfying the \textbf{(weakness)} and \textbf{(continuity)} conditions in virtue of their self-dual nature. Proposition \ref{PROP_complementation} yields the complementation lemma allowing to conclude that on trees $\ext{\neg \psi} = \trees(\overline{\aut_{\psi}})$.
   %\item

\indent   If $\varphi = \exists p.\psi$, let $\aut_{\psi}$ be the automaton given by inductive hypothesis. By semantics of $\nmso$, on trees $\ext{\exists p. \psi} = {{\exists}_F p}.\ext{\psi}$ and thus $\ext{\exists p.\psi} = \trees({{\exists}_F p}.\aut_{\psi})$ by Proposition \ref{PROP_fin_projection}.
 %\end{itemize}
\end{proof} 



\subsubsection{From automata to formulae}


In this section we show the other direction of Theorem \ref{t:mt2}, completing
the automata characterisation of $\wmso$ on tree models.
The argument is reminiscent of the one showing that $\MSO$-automata can be
translated into equivalent formulas of $\MSO$~\cite{Walukiewicz96}.
%It consists of two steps. First, it is shown that for every $\wmso$-automaton $\bbA$ there is an effectively constructible formula in a suitably defined $\yvF$-fragment of a fixpoint extension of $\lque$. Second, it is verified that for every formula of the aforementioned  $\yvF$-fragment  there is an effectively constructible equivalent  $\wmso$-formula.
We start by introducing a fragment of a fixpoint extension
of $\olque$ and show how it embeds into $\wmso$.

\begin{definition}
The fixed point logic $\mlque$ on $\prop$ is given by:
% {\footnotesize%
% $$
% \varphi ::= p(x) \mid x=y \mid R(x,y) \mid \exists x.\varphi \mid \qu x.\varphi \mid \lnot\varphi \mid \varphi \land \varphi \mid \mu p.\varphi(p,x)
% $$}%
\begin{align*}
\varphi ::=\ & p(x) \mid x=y \mid R(x,y) \mid \lnot\varphi \mid \varphi \land \varphi \mid\\
& \exists x.\varphi \mid \qu x.\varphi \mid \mu p.\varphi(p,x)
\end{align*}
%
where $p\in\prop$, $x,y\in\fovar$; moreover $p$ occurs only positively in $\varphi(p,x)$ and $x$ is the only free variable in $\varphi(p,x)$.
\end{definition}

% Analogously to the modal $\mu$-calculus, the fixed point logic $\mlque$
% is  obtained by adding to $\lque$ the following rule for constructing
% fixed point formulas.
% \yvwarning{Why use `$P$' instead of `$p$'?}\fcnote{My fault/choice, will explain by mail}
%  \begin{itemize}
%  \item Given $P \in \prop$, $x\in\fovar$ and $\phi(P, x)$ with only positive occurrences of $P$ and no free variable other than $x$, $\mu P. \phi(P, x)$ and $\nu P. \phi(P, x)$ are formulas of $\mlque$.
%  \end{itemize}

%The semantics of the fixpoint formulas $\mu P. \phi(P, x)$ and $\nu P. \phi(P, x)$ is the expected one.
The semantics of $\mu p. \phi(p, x)$ is the expected one: given an LTS $\model$ and $s \in T$,  $\model \models \mu p. \phi(p, s)$ iff $s$ is in the least fixpoint of the operator
%$\phi^\model_P$ that maps any $S \subseteq T$ into $\{t \in T \mid \model[P \mapsto S] \models \phi(P, t) \}$.
$\phi^\model_p(S) := \{t \in T \mid \model[p \mapsto S] \models \phi(p, t) \}$.%The semantics of $\nu P. \phi(P, x)$ is dually defined by considering the greatest instead of the least fixpoint of $F_\phi$.


%Formulas of $\mlque$ may be also classified according to their alternation depth as it happens for the modal $\mu$-calculus.
%The alternation-free fragment of $\mlque$ is thence defined as the collection of $\mlque$-formulas $\phi$
%without nesting of greatest and least fixpoint operators, i.e. such that, for any two subformulas $\mu P.\psi_1(P,y)$ and $\nu Q. \psi_2(Q,z)$, predicates $P$ and $Q$ do not occur free respectively in $\psi_2(Q,z)$ and $\psi_1(P,y)$.

%
%
\begin{definition}
Given $p \in \prop$, we say that $\varphi \in \mlque$ is
\begin{itemize}
\item \emph{monotone in $p$} iff for every LTS $\model=\tup{T,R,\tscolors, s_I}$ and assignment $\ass$, if $\model, \ass \models \varphi$ and $ \{s \in T \mid p\in \tscolors(s)\} \subseteq E$, then $\model[p \mapsto E], g\models \phi$,

\item %\emph{continuous in $P$} iff for every LTS $\model$ and assignment $\ass$ there exists some finite $S \subseteq_\omega V(P)$ such that
    \emph{continuous in $p$} iff for every LTS $\model=\tup{T,R,\tscolors, s_I}$ and assignment $\ass$ there is some finite $S \subseteq_\omega \{s \in T \mid p\in \tscolors(s)\}$ such that $\model, \ass \models \varphi$ iff $\model[p \mapsto S], \ass \models \varphi$.
\end{itemize}
\end{definition}

We provide a definition of a fragment of $\mlque$ reminiscent of the one in
Theorem \ref{thm:ofoecont}.
\begin{definition}
Given a set $Q\subseteq \prop$, the fragment $\cont{\mlque}{Q}$ is defined by the following rules:
% {\small%
% $$
% \varphi ::= \psi \mid q(x) \mid \exists x.\varphi \mid \varphi \land \varphi \mid \varphi \lor \varphi \mid \wqu x.(\varphi,\psi) \mid \mu p. \phi'(p, x)
% $$}%
\begin{align*}
\varphi ::=\ & \psi \mid q(x) \mid \varphi \lor \varphi \mid \varphi \land \varphi \mid\\
& \exists x.\varphi \mid \wqu x.(\varphi,\psi) \mid \mu p.\varphi'(p,x)
\end{align*}
%
where $q \in Q$,
%$\psi \in \mlque(\prop\setminus Q)$,
$\psi \in \mlque$ is $Q$-free,
$\wqu x.(\varphi,\psi)
:= \forall x.(\varphi(x) \lor \psi(x)) \land \dqu x.\psi(x)$ and $\phi'(p,x)$
is a formula, with only $x$ free and $p \in \prop$, which belongs to
$\cont{\mlque}{Q \cup\{p\}}$.
\end{definition}

%We then verify that formulas in $\cont{\mlque}{A}$ are (semantical) continuous in $A$. The proof is
By combining the argument for the proof of Proposition \ref{thm:ofocont} and
the one used in proving the analogous Lemma 1 in~\cite{Fontaine08}, we can thence obtain the following:
\begin{proposition}\label{lem:cofoeiscont_mu}
If $\varphi \in \cont{\mlque}{Q}$ then $\varphi$ is continuous in (each element of) $Q$.
\end{proposition}
%\begin{proof} First, notice that If $\varphi \in \cont{\mlque}{A}$ then $\varphi$ is monotonous  in (each predicate from) $A$. %This is proved as for
%The proof goes then by induction on the complexity of $\varphi$. For the all the cases except the fixpoint one, the proof is the same as the one for Lemma \ref{lem:cofoeiscont}. For $\phi=\mu Q. \phi'(Q, x)$, with $\phi'(Q,x) \in \cont{\mlque}{A \cup\{Q\}}$, the argument is the same as the one in the proof of Lemma 1 in \cite{Fontaine08}.
%\end{proof}
%As for $\MC$, we define

\begin{definition}
The fragment $\clque$ of $\mlque$ is given by restricting the fixpoint operator to the continuous fragment:
% {\footnotesize%
% $$
% \varphi ::= p(x) \mid x=y \mid R(x,y) \mid \exists x.\varphi \mid \qu x.\varphi \mid \lnot\varphi \mid \varphi \land \varphi \mid \mu p.\varphi'(p,x)
% $$}%
\begin{align*}
\varphi ::=\ & p(x) \mid x=y \mid R(x,y) \mid \lnot\varphi \mid \varphi \land \varphi \mid\\
& \exists x.\varphi \mid \qu x.\varphi \mid \mu p.\varphi'(p,x)
\end{align*}
%
where $p\in\prop$, $x,y\in\fovar$, $\varphi'(p,x) \in \cont{\mlque}{\{p\}} \cap \clque$ is positive in $p$ and $x$ is its only free variable.% in $\varphi'$.
\end{definition}

% The fragment $\clque$ of $\mlque$  is obtained
% %by restricting the use of the least fixpoint operator to the continuous fragment. It is obtained
% by adding to $\lque$ the following restricted rule for fixpoint formulas.
%  \begin{itemize}
%  \item Given $P\in\prop$, $x\in\fovar$ and $\phi(P, x)$ with only positive occurrences of $P$ and only $x$ free,
% if $\phi(P,x) \in \cont{\mlque}{\{P\}}$, then $\mu P. \phi(P, x) \in \clque$.%$\mu P. \phi(P, x)$ is also a $\clque$-formula.
%  %if $\phi(P,x)$ is a $\clque$-formula that belongs to $\cont{\mlque}{\{P\}}$, then $\mu P. \phi(P, x)$ is also a $\clque$-formula.
%  \end{itemize}

%
%
%
%
%The logic $\mglque$ can be given a semantic in terms of evaluation games extending the one given  in \cite{BerwangerG01} for $\mgfoe$ by adding rules for the generalized quantifier.
%We present it just for $\qu$, and treat the rules for $\dqu y. \phi(\overline{x},y)$ as derived from the equivalent formula $\lnot \qu y. \lnot\phi(\overline{x},y)$
%the universal being treated as the alternation-free fragment.
%As usual, we assume that any predicate is bounded by at most one fixpoint operator
%%, any if a predicate is bounded, then the fixpoint operator bounding it is unique,
%and that bounded and free predicates are pairwise distinct.%\fzwarning{In the table: why not a clause for $\neg$, $\vee$, $\wedge$? Meaning of $\eta$, ; and :?}
%                             \begin{table}[h]
%                              \centering
%                            \begin{tabular}{|l|c|l|c|}
%                             \hline
%                              % after \\: \hline or \cline{col1-col2} \cline{col3-col4} ...
%                              Position & Player & Admissible moves & Parity\\
%                               \hline % \hline
%                           %  $( ; \overline{x}: \overline{a})$ & $\forall$ & $\{B \subseteq T \mid |B| \geq \aleph_0 \}$ & $-$ \\
%                           %   $B \subseteq T $ & $\exists$ & $\{(\lnot \phi(\overline{x},y); \overline{x}: \overline{a}, y:b)\ |\ b \in B \}$ & $-$\\
%                          %     \hline
%                            $(\qu y. \phi(\overline{x},y); \overline{x} \mapsto \overline{a})$ & $\exists$ & $\{B \subseteq T \mid |B| \geq \aleph_0 \}$ & $-$ \\
%                              $B \subseteq T $ & $\forall$ & $\{(\phi(\overline{x},y); \overline{x} \mapsto \overline{a}, y \mapsto b)\ |\ b \in B \}$ & $-$\\
%                              \hline
%%                              $(\mu P. \phi(P, x); x \mapsto a)$ & $\exists$ & $\{(\phi(P, x);  x: a)\}$ & $1$ \\
%%                             % \hline
%%                              $(\nu P. \phi(P, x); x: a)$ & $\exists$ & $\{(\phi(P, x);  x: a)\}$ & $0$ \\
%%                              %\hline
%%                              $(P(y); y: a)$ & $\exists$ & $\{(\eta P.\phi(P, x); x: a)\}$ & $-$ \\
%%
%%                              \hline
%                            \end{tabular}
%                             \caption{The new rules in the evaluation game for $\mglque$.
%                          }
%                             \label{mufo_game}
%                            \end{table}
%
% By a straightforward adaption of the corresponding proof for $\mgfoe$ in \cite{BerwangerG01},
% we obtain:
%
% \begin{theorem}
% For every model $\model$, and every formula $\mglque$-formula $\phi(x)$ with one free variable, then
% $\model \models \phi(n)$ iff $\exists$ has a winning strategy in $\mc{E}(\varphi(x),\model)@(\varphi(x); x \mapsto n)$, the evaluation game for $\phi(x)$ and $\model$ when evaluating $x$ at the node $n$.\end{theorem}
%

 %%%%%%%%
 %We now recall a useful property of fixpoints and continuity. Let $\phi(P,x)$ a formula with only $x$ free.
\noindent Given an LTS $\model$ and $p\in \prop$, for every ordinal $\alpha$ we define: %by induction the following sets:
 %\fcwarning{Why not $\phi^0(\emptyset):= \emptyset$?}
 \begin{itemize}
\item $\phi^0_p(\emptyset):= \emptyset$,
%\{ s \in T \mid \model[P \mapsto \emptyset] \models \phi(P, s)\}$,
\item $\phi_p^{\alpha+1}(\emptyset):= \{ s \in T \mid \model[p \mapsto \phi_p^\alpha(\emptyset)] \models \phi(p, s)\}$,
\item $\phi_p^{\lambda}(\emptyset):= \bigcup_{\alpha < \lambda} \phi_p^{\alpha}(\emptyset)$, with $\lambda$ limit.
\end{itemize}
%We state $\phi^{-1}(\emptyset):=\emptyset$.
If $\phi$ is monotone in $p$, then $\phi_p^{\beta+1}(\emptyset)= \phi_p^{\beta}(\emptyset)$, for some ordinal $\beta$. Also, the set $\phi_p^{\beta}(\emptyset)$ is the least fixpoint of $\phi^\model_p$ (see e.g. \cite{ArnoldN01}).



%A formula $\phi(P, x)$ is said to be \emph{constructive} in $P$ if its least fixpoint is reached in at most $\omega$ steps.
%, i.e., if for every model $\model$, the least fixpoint of $F_\phi$ equals to $\bigcup_{\alpha < \omega} \phi^{\alpha}(\emptyset)$.
 %From a local perspective, this means that
 A formula $\phi(p, x)$ is \emph{constructive} in $p$ if for every model $\model$,  every node $s \in T$, if $\model \models\mu p. \phi(p,s)$, then $s\in \phi_p^{i+1}(\emptyset)$, for some $i< \omega$.
The next proposition is easily verified:% by the fact that Scott proved in \cite{Fontaine08} for the modal $\mu$-calculus but that generalizes to $\mglque$ as well, states that continuous formulas are constructive.



%\afwarning{Verify the claim and that Gaelle's argument REALLY goes thorough also here.}
%\yvwarning{Do not attribute to Gaelle, it is obvious that a Scott continuous map reaches fixpoint in $< \omega$ steps}
\begin{proposition}\label{prop:constructivity}
Let $\phi(p,x)$ be a $\mlque$-formula with only $x$ free. If $\phi(p,x)$ is continuous in $p$, then for every LTS $\model$, and every node $s \in T$, there is $i < \omega$ such that
\[\model \models \mu p. \phi(p,s) \text{ iff } s \in \phi_p^{i+1}(\emptyset).\]
\end{proposition}

By Proposition~\ref{prop:constructivity} and the fact that sets $\phi_p^{i+1}(\emptyset)$ are essentially defined as finite unfoldings, we obtain the following.%\fcwarning{More intuition on this?}

\begin{proposition}\label{prop:cor_constructivity}
Let $\phi(p,x)$ be a $\mlque$-formula continuous in $p$ with only $x$
free. Let $\model$ be an LTS, and $s \in T$. Then
$\model \models \mu p. \phi(p,s)$ iff there is a finite set $p^\model \subseteq T$ such that $s\in p^\model$ and $\model[p\mapsto p^\model] \models \phi(p,t)$  for every $t \in p^\model$.
\end{proposition}
% \begin{proof}
% For the direction from left to right, assume that $\model \models \mu P. \phi(P,s)$.  By Proposition~\ref{prop:constructivity}, we know that  there is $i< \omega$ such that $\model[P \mapsto \phi^i(\emptyset)] \models \phi(P, s)$. The set $\phi^i(\emptyset)$ need not to be finite. However,
% using this information, we are going construct a finite tree whose nodes $t$ are labelled by finite sets $X^m_j$, where $m$ is a node of $\model$ and $j \leq i$, satisfying the following condition:
% \begin{enumerate}
%\item  if $t$ is the root, then $t$ is labelled by $X_i^s$,
%\item  if $t$ is labelled by $X_j^m=\{s_1, \dots, s_\ell\}$ and $j>0$, then $t$ has $\ell$  children and for every $s_i \in X_j^m$ there is an unique child $t'$ of $t$ labelled by $X_{j-1}^{n_i}$ where $m$ is a nodes,
%%\item if $s$ is labelled by $X_j^m$ and $j=-1$, then $X_j^m=\emptyset$,
%\item for every node $t$ of the tree, if $t$ is labelled by $X_j^m$, then it holds that $X_j^m \subseteq \phi^{j}(\emptyset)$.
%\end{enumerate}
%If we verify that $\model[P\mapsto P^\model] \models \phi(P,s)$ holds by taking as $P^\model$ the union of all labels of the nodes of the constructed tree, we can conclude for the proof of this direction.
%
%As starting point of the inductive construction, we start by the empty tree.  Recall that we know that  $\model[P \mapsto \phi^i(\emptyset)] \models \phi(P, s)$. Since $\phi(P,x)$ is continuous in $P$, there is a finite set $X^s_i \subseteq \phi^i(\emptyset)$ such that $\model[P \mapsto X^s_i] \models \phi(P, s)$. We then add a root to our tree and label it by $X^s_i$.
% Assume that at a leaf $s$ of our tree is labelled by $X^m_j$, for some $j < i$. If $X^m_j$ is empty, than we stop, else we proceed as follows. We know that $X^m_j\subseteq \phi^{j}(\emptyset)$. This means that $\model[P \mapsto \phi^{j-1}(\emptyset)] \models \phi(P, r)$, for every $r \in X_j^m$. By continuity, for each such $r$, there is a finite set $X^m_{j-1} \subseteq  \phi^{j-1}(\emptyset)$ such that $\model[P \mapsto X^m_{j-1}(\emptyset)] \models \phi(P, r)$. For each $r \in X^m_j$ we thus add a child to $m$ and label it with $X^r_{j-1}$. By definition of $\phi^{i+1}(\emptyset)$, the tree is finite. Let $X$ be the union of all labels of the constructed tree. $X$ is finite, and by monotonicity of $\phi(P,x)$ we have that for every $m \in X \cup \{s\}$, $\model[P \mapsto X \cup \{s\}] \models \phi(P,m)$.
%
%For the other direction, it is enough to notice that the smallest finite set $P^\model \subseteq T$ such that $\model[P\mapsto P^\model] \models \phi(P,s)$ and $\model[P\mapsto P^\model] \models \phi(P,m)$  for every $m \in P^\model$ is the least fixpoint of the function that maps any $S \subseteq T$ into $\{t \in T \mid \model[P \mapsto S] \models \phi(P, t) \}$.
%%the idea is the following. By assumption there is a finite set $P^\model \subset T$ such that $\model[P\mapsto P^\model] \models \phi(P,n)$ and $\model[P\mapsto P^\model] \models \phi(P,m)$  for every $m \in P^\model$. The winning strategy for \'Eloise  in $\mc{E}(\mu P.\varphi(P,x),\model)@(\mu P.\varphi(P,x); x \mapsto n)$
%%is thus define as the composition of all winning strategies in $\mc{E}(\varphi(P,x),\model[P \mapsto P^\model]))@(\varphi(P,x); x \mapsto m)$
%% for $m \in P^\model$.
% \end{proof}

Proposition \ref{prop:cor_constructivity} naturally suggests the following translation $\mgFOETr{\cdot}:\mlque\to\nmso$. It is given homomorphically in predicates, Booleans and first-order quantifiers and:
\begin{itemize}
% \begin{multicols}{2}
\itemsep 0 pt
% \item $\mgFOETr{P(x)}=P(x)$,
% \item $\mgFOETr{R(x,y)}=R(x,y)$,
% \item $\mgFOETr{x\foeq y}= (x \foeq y)$,
% \item $\mgFOETr{\varphi \land \psi}=\mgFOETr{\varphi} \land \mgFOETr{\psi}$,
% \item $\mgFOETr{\lnot \varphi}= \lnot \mgFOETr{\varphi}$,
% \item $\mgFOETr{\exists x. \varphi}=\exists x. \mgFOETr{\varphi}$,
\item $\mgFOETr{ \qu x. \varphi} := \forall p.\exists x. (\lnot p(x) \land \mgFOETr{\varphi})$,
% \end{multicols}
\item $\mgFOETr{\mu p. \varphi(p,x)} := \exists p ( p(x) \land \forall y ( p(y) \to \mgFOETr{\varphi(p,y) }))$.
\end{itemize}

%Note that in $\mgFOETr(\mu P. \varphi(P,x))$, the predicate $P$ which occurs in $\mgFOETr(\varphi) $ is bounded by the outermost second order existential quantifier.

%The following theorem %, which is the analogous of Theorem \ref{thm:contransweak} but for $\mlque$,
%is then an immediate corollary of Proposition \ref{prop:cor_constructivity}.

\begin{proposition}\label{thm:guard_wmso}
Let $\phi$ be a $\clque$-formula, $\model$ an LTS and $\ass$ an assignment.
Then $\model, \ass \models \varphi$ iff $\model, \ass \models \mgFOETr{\varphi}$.
%the following two conditions are equivalent:
%\begin{enumerate}\itemsep 0pt \item $\model, \ass \models \varphi$, \item $\model, \ass \models \mgFOETr{\varphi}$. \end{enumerate}
\end{proposition}
\begin{proof}
The proof is by induction on $\varphi$. The least fixpoint case is handled by applying Proposition \ref{prop:cor_constructivity}.
%
%Let therefore consider $\phi$ is of the form $\mu P. \psi(P,x)$. Without loss of generality, that bounded and free predicate variables are distincts.
%We first show that $(1)$ implies $(2)$. Since $\model , \ass \models \varphi$, \'Eloise has a winning strategy $f$ in $\mc{E}(\phi,\model)@(\varphi,s_I, \ass)$.
%Define $P^\model$ to be the set of node $n \in T$ such that there is a (partial) match $\pi'$ that
%%
%\begin{enumerate}
%\itemsep 0pt
%\item is consistent with $f$, and such that
%\item every position of $\pi'$ is of the form $(\gamma,m, \ass')$, with  $P$ active in $\gamma$, and
%\item the last position of $\pi'$ is of the form $(\varphi, n, \ass')$.
%\end{enumerate}
%%
%The first observation is that since $f$ is a winning strategy, all $f$-consistent matches are finite. Moreover for every position of $\pi'$ is of the form $(\psi(P,x),m, \ass')$, we have that $\model[P \mapsto P^{\model}], \ass' \models \psi(P,m)$. We construct inductively a finite tree labelled by pairs $(x, X)$ where $x$ is a node of $\model$ and $X$ is a finite set of nodes of $\model$ as follows. First, because $\model[P \mapsto P^{\model}] , \ass \models \psi(P,x)$, so there is a finite subset $X \subseteq P^\model$ such that $\model[P \mapsto X_1] , \ass \models \psi(P,x)$. Thus we color the root with $(n, X)$. Now, assume we are given a leaf colored by $(y,Y)$. Consider an enumeration $x_1, \dots, x_k$ of $Y$. For every $i \leq k$, we add a child to $(y,Y)$ labelled by $(x_i, X_i)$ where $X_i$ is given by the fact that since $\model[P \mapsto P^{\model.x_i}] , \ass \models \psi(P,x)$, there is a finite set $X_i$ of nodes in $P^{\model.x_i}$
%
%%the only player who picks successor in a partial match $\pi'$ defined as above is \'Eloise. As a consequence of K\"onig's Lemma, $P^\model$ is finite.
%%
%%By using the induction hypothesis, it is easy to check that $\model[x \mapsto s_I, P \mapsto P^\model] \models P(x) \land \forall y ( P(y) \to ST_y(\varphi) )$.
%%
%%For the other direction, the idea is the following. Because $\model[x \mapsto s_I] \models ST_x(\varphi)$,
%%there is a finite set $P^\model$ such that $\model[x \mapsto s_I, P \mapsto P^\model] \models P(x) \land \forall y ( P(y) \to ST_y(\varphi) )$. The winning strategy for \'Eloise  in $\mc{E}(\mu P.\varphi,\model)@(\mu P.\varphi,s_I)$
%%is thus define as the composition of all winning strategies in $\mc{E}(\varphi,\model[P \mapsto P^\model])@(\varphi,s)$ for $s \in P^\model$.
\end{proof}

By virtue of Proposition \ref{thm:guard_wmso}, we are able to conclude the proof of Theorem \ref{t:mt2} by showing the following statement.
\begin{proposition}\label{thm:wmsoauttof}
Every $\wmso$-automaton can be effectively translated into an equivalent $\clque$-formula.
\end{proposition}
%
%
%
%The guarded fragment $\glque$ of $\lque$ is obtained by imposing that first-order and generalized quantifiers  are relativized to atomic formulas of the form $r(x,y)$. The fixpoint extension $\mglque$  of $\glque$ is thus defined by adding a fixpoint construction clause.
%Analogously to the modal $\mu$-calculus, the  $\yvF$-fragment is finally obtained by considering only $\mglque$ formulas without alternation of fixpoints and by imposing some continuity conditions on the fixpoint construction rule.
%
\begin{proof} The argument is
 essentially a refinement of the standard proof showing that any automaton in $\yvAut(\ofo)$ can be translated into an equivalent $\mu$-formula
$\xi_\aut$ (\emph{cf.} e.g. \cite{Ven08}).
The idea is the following. We see a $\wmso$-automaton as a system of equations expressed in terms of $\lque$-formulas: each state corresponds to a monadic predicate variable and the odd/even parity of a state corresponds to the least/greatest fixpoint that we seek for the associated variable, etc. One then solves this system of equations via the same inductive procedure used to obtain the formula of the modal $\mu$-calculus from the system associated with an automaton in $\yvAut(\ofo)$ (see e.g. \cite{ArnoldN01} for a description of the solution procedure). Because of the \textbf{(weakness)} and \textbf{(continuity)} conditions on the starting $\nmso$-automaton $\aut$, it is thence possible to verify that the resulting fixpoint formula $\xi_\aut$ belongs to $\clque$.
%
%To complete the proof of the Theorem, we have to provide an effective truth-preserving translation from the  considered fragment of $\mglque$ into $\wmso$. The key observation here is that continuous fixpoint are constructive (see \cite{Fontaine08}), that is a least-fixpoint formula in one free variable $\mu p. \phi(p,x)$ of the $\yvF$-fragment of $\mglque$ is true at $s$ in a model $\model$ if $s$ belongs to some finite approximant of the least fixpoint induced by $\phi(p.x)$. From this fact, it is then possible to verify that a formula $\mu p. \phi(p,x)$ of the $\yvF$-fragment of $\mglque$ is true at $s$ in a model $\model$ iff is $\phi(p,x)$ true at $s$ in $\model[p\mapsto P^\model]$, for some finite $P^\model$, and use this propriety to obtain the truth-preserving  translation into $\wmso$.
%%by taking formulas without alternation such that fixpoint operators only bound
%%$\mglque$  is  obtained by adding to $\glque$ the following (semantic) rule for constructing fixed point formulas.
%%fragment
%%
%%
%%The main idea is to encode each state of the starting automaton as a propositional variable bounded by a greatest (if the parity is even) or least (if the parity is odd) fixpoint operator. The target of the translation is a suitable extension of first-order logic with fixpoint operators, denoted with $\mglque$. Thanks to the (weakness) and (continuity) conditions on $\nmso$-automata, we are able to infer that the target can be in fact restricted to a fragment of $\mglque$. Finally, this is proven to be included in $\nmso$ completing the proof of Theorem \ref{thm:wmso_autofor}.
\end{proof}
%%\btbs
%% \item Continue with more details according to the space that we decide to devote to this section.
%% \etbs 




\clearpage

%%%%
%%%% EXPRESSIVENESS
%%%%
\section{Expressiveness modulo bisimilarity}\label{sec:expresso}

\subsection{From modal languages to second order languages}

\subsection{From second order languages to modal languages}
In this section we are going to prove Theorem \ref{t:m1}.
Our proof of the first item of the theorem crucially involves automata.
In the previous section we saw that on trees, $\wmso$ effectively corresponds
to the automata class $\AutWC(\olque)$.
 We will now relate this class to the one of
parity automata based on $\ofo$ and satisfying similar weakness and continuity conditions.


\begin{definition}
A \emph{$\mucML$-automaton} $\aut = \tup{A,\Delta,\Omega,a_I}$ is an automaton $\aut \in \AutWC(\ofo)$ such that for all states $a,b \in A$ with $a \ord b$ and $b\ord a$ the following conditions hold:
\begin{description}
	\itemsep 0 pt
	\item[(weakness)] $\pmap(a)=\pmap(b)$,
	\item[(continuity)] if $\pmap(a)$ is odd (resp. even) then, for each $c\in C$ we have
	   $\tmap(a,c) \in \cont{\ofo^+}{b}(A)$ (resp. $\tmap(a,c) \in \cocont{\ofo^+}{b}(A)$).
\end{description}
As the class of such automata coincides with $\AutWC(\ofo)$ we use the same name to denote it.
\end{definition}

As the key technical result of our paper, in subsection~\ref{pinvariant-fragment}
we will provide a construction $(-)^{\bullet}: \AutWC(\olque) \to 
\AutWC(\ofo)$, such that for all $\bbA$ and $\bbT$ we have
\begin{equation}
\label{eq:crux}
\bbA^{\bullet} \text{ accepts } \bbT \text{ iff } \bbA \text{ accepts 
} \bbT^{\om},
\end{equation}
where $\bbT^{\om}$ is the $\om$-unravelling of $\bbT$.
As we shall see, the map $(-)^{\bullet}$ is completely determined at the 
one-step level, that is, by some model-theoretic connection between 
$\olque$ and $\ofo$.

The second fact, to be discussed in
subsection \ref{aut-to-formula}, is that for each $\mucML$-automaton $\bbA$  we can effectively construct an equivalent $\mucML$-formula $\xi_{\bbA}$.

On the basis of the above observations we show that those results are enough to prove Theorem~\ref{t:m1}(i) as follows:

\begin{proofof}{Theorem~\ref{t:m1}}
\textbf{(1)} Given a \wmso-formula $\phi$, let $\phi^{\bullet} \df
\xi_{\aut_{\phi}^{\bullet}}$.
We verify that $\phi$ is bisimulation invariant iff $\phi$ and $\phi^{\bullet}$
are equivalent.
The direction from right to left is immediate by the observation that
$\phi^{\bullet}$ is a formula in $\MC$.
The opposite direction follows from the following chain of equivalences:
\begin{align*}
\model \models \phi
  & \text{ iff } \bbT^{\om} \models \phi
  & \tag{$\phi$ bisimulation invariant}
\\ & \text{ iff }  \bbA_{\phi} \text{ accepts } \bbT^{\om}
  & \tag{$ \phi \equiv \aut_{\phi}$ on trees}
\\ & \text{ iff } \bbA_{\phi}^{\bullet} \text{ accepts } \bbT
& \tag{\ref{eq:crux}}
\\ & \text{ iff }  \bbT \models \xi_{ \aut_{\phi}^{\bullet}}
& \tag{$\aut_{\phi}^{\bullet}\equiv \xi_{ \aut_{\phi}^{\bullet}}$}
\end{align*}
\textbf{(2)} For the second part of Theorem \ref{t:m1},
We %reason as follows.
%First, let's consider the following translation.
first define, for every first-order variable $x$, a translation $ST_x$ from
the $\mu$-calculus into the set of $\mlque$-formulas with only $x$ free:

\begin{itemize}
\itemsep 0 pt
\item $ST_x(p)=p(x)$%\fcwarning{maybe $p(x)$, we use small predicates},
\item $ST_x(\varphi \land \psi)=ST_x(\varphi) \land ST_x(\psi)$,
\item $ST_x(\varphi \lor \psi)=ST_x(\varphi) \lor ST_x(\psi)$,
\item $ST_x(\lnot \varphi)= \lnot ST_x(\varphi)$,
\item $ST_x(\Diamond \varphi)=\exists y (R(x,y) \land ST_y(\varphi))$,
\item $ST_x(\mu p. \varphi)= \mu p. ST_x(\varphi)$,
\end{itemize}
Clearly, every formula of the $\mucML$-fragment of the $\mu$-calclus is mapped to a logically equivalent formula of the $\mucML$-fragment of $\mlque$. Let
 $(-)_{\bullet}:\mucML\to\wmso$ defined as the composite $\mgFOETr{-} \circ ST_x$. By Theorem \ref{thm:guard_wmso}
we obtain that $\psi \equiv \psi_{\bullet}$, for all $\psi \in
\mucML$.
\end{proofof}


%In this section we are going to prove Theorem \ref{t:m1}.
%For the first point of the theorem, we reason as follows.
%From the previous section we know that on trees, for each $\wmso$-formula $\phi$ we can construct an equivalent \wmso-automata $\aut_{\phi}$. What is thence missing is an effective procedure that takes as input a \wmso-automaton $\aut$ whose corresponding regular tree language is closed under bisimulation, and gives as output an equivalent $\mucML$-formula $\xi_\aut$.
%
%Given a \wmso-formula $\phi(x)$, let $(\phi(x))^{\bullet} :=\xi_{\aut_{\phi}}$. We verify that $\phi$ is bisimulation invariant iff $\phi$ and $\phi^{\bullet}$ are equivalent. The direction from right to left being immediate, for the other direction we connect the following equivalences:
%\begin{alignat*}{2}
%		\model \models \phi(s_I)
%			& \quad\text{ iff }\quad \unravel{\model} \models \phi(s_I) & \qquad\quad \text{($\phi(x)$ is bisimulation invariant)} \\
%			& \quad\text{ iff }\quad  \unravel{\model} \in \trees( \aut_{\phi})& \qquad\quad \text{($\aut_{\phi}\equiv \phi$ on trees)} \\
%			& \quad\text{ iff }\quad  \unravel{\model} \models \xi_{ \aut_{\phi}}& \qquad\quad \text{($\aut_{\phi}\equiv \xi_{ \aut_{\phi}}$ on trees)}\\
%			& \quad\text{ iff }\quad  \model \models \xi_{ \aut_{\phi}}& \qquad\quad \text{($\xi_{ \aut_{\phi}}$ is bisimulation invariant)}
%	\end{alignat*}
%where $\unravel{\model}$ is the tree unraveling of $\model$ with respect to $s_I$.
%
%The construction of $\xi_\aut$ will be performed in two steps. For the first step we use the following results.
%
%\begin{theorem}[\cite{Venxx}]\label{t:venema}
%Let $\llang_1$ and $\llang_1'$ be fragments of $\olque$. If $\llang^+_1$ corresponds to the $P$-invariant fragment of ${\llang_1'}^+$, then
%\begin{enumerate}
%\itemsep 0 pt
%\item $\AutWC(\llang_1)$ corresponds to the bisimulation-invariant fragment of $\AutWC(\llang_1')$, and
%\item (ADD CONSTRAINTS ABOUT THE C-FRAGMENT) $\AutWC(\llang_1)$ corresponds to the bisimulation-invariant fragment of $\AutWC(\llang_1')$.\fcwarning{Define what bisim-inv means for automata?}
%\end{enumerate}
%\end{theorem}
%
%
%
%\afwarning{Is this really true?}\fcwarning{See email about remarks on the preservation of fragments and also Yde's mail}
%\begin{remark}
%To be precise, Theorem \ref{t:venema} in \cite{Venxx} is stated for $\foe_1$ and the correspondence given by point 1 only. However, the proof apply to the more general setting of the proposition above as well.
%\end{remark}
%In subsection \ref{pinvariant-fragment} we verify that the $P$-invariant fragment of ${\olque}^+$ is $\ofo^+$. By Theorem \ref{t:venema} we therefore obtain that $\AutWC(\ofo^+)$ corresponds to the bisimulation-invariant fragment of $\AutWC(\olque)$.
%
%The second step will then consists in showing that given an automaton in $\AutWC(\ofo)$ we can construct an equivalent  $\mucML$-formula. This is done in subsection \ref{aut-to-formula}.
%
%%\begin{enumerate}
%%\item we verify that at the level of one step logic,
%%\end{enumerate}
%
%%prove this already at one-step level: FO1C ? FOE�1 C/P ?? from this follows that wAut(FO1C) ? wAut(FOE�1 C)/?
%%wAut(FO1C) corresponds to fragment F
%%Rest is standard argument
%
%%The proof for this result consists in three steps
%%how the proof of (2) can be decom-
%%posed into three more or less independent parts:
%%1. a (non-trivial) result showing that both the modal ?-calculus and (on the class of tree models) monadic second-order logic can be characterized by certain automata,
%%2. a fairly simple model-theoretic characterisation result in monadic first-order logic, and
%%3. a general result on coalgebra automata.
%
%
%
%
%For the second point of Theorem \ref{t:m1}, in subsection \ref{sub:contoweak} we will  explicit the translation $(-)_{\bullet}:\mucML\to\wmso$ and verify that  $\psi \equiv \psi_{\bullet}$.
%
%



\subsubsection{One-step translations}\label{pinvariant-fragment}

In this subsection we will define a construction that transforms an arbitrary automaton
$\bbA$ in $\AutWC(\olque)$ into an automaton $\bbA^{\bullet}$ in 
$\AutWC(\ofo)$, such that $\bbA$ and $\bbA^{\bullet}$ are related as 
in~\eqref{eq:crux}.
This construction is completely determined by the following translation at the
one-step level.

\begin{definition}
Using the fact
that by Corollary~\ref{cor:olquepositivenf}, any formula in ${\olque}^+(A)$ is 
equivalent to a disjunction of formulas of the form 
$\posdbnfolque{\vlist{T}}{\Pi}{\Sigma}$, we define the translation 
$(-)^{\bullet} : {\olque}^+(A) \to \ofo^+(A)$ as follows.
We set
\[
\Big( \posdbnfolque{\vlist{T}}{\Pi}{\Sigma} \Big)^{\bullet} \df
\bigwedge_{i} \exists x_i. \tau^+_{T_i}(x_i) \land \forall x. \bigvee_{S\in\Sigma} \tau^+_S(x)
%\posdbnfofo{\vlist{T}\cup\Sigma},
\]
and for $\al = \bigvee_{i} \al_{i}$ we define $\al^{\bullet} \df \bigvee 
\al_{i}^{\bullet}$.
\end{definition}

The key property of this translation is the following.

\begin{proposition}
\label{p-1P}
For every one-step model $(D,V)$ and every $\al \in {\olque}^+(A)$ we have
\begin{equation}
\label{eq-1cr}
(D,V) \models \alpha^{\bullet} \text{ iff } (D\times \om,V_\pi) \models \alpha,
\end{equation}
where $V_{\pi}$ % =  f^{-1} \circ V$
 is the induced valuation given by 
$V_{\pi}(a) \df \{ (d,k) \mid d \in V(a), k\in\omega\}$.
\end{proposition}

\begin{proof}
Clearly it suffices to prove \eqref{eq-1cr} for formulas of the form
$\al = \posdbnfolque{\vlist{T}}{\Pi}{\Sigma}$.
\smallskip

\noindent\fbox{$\Rightarrow$} 
Assume $(D,\val) \models \alpha^{\bullet}$, we will show that 
$(D\times \omega,\val_\pi) \models \posdbnfolque{\vlist{T}}{\Pi}{\Sigma}$.
Let $d_i$ be such that $\tau_{T_i}^+(d_i)$ in $(D,\val)$. 
It is clear that the $(d_i,i)$ provide \emph{distinct} elements satisfying 
$\tau_{T_i}^+((d_i,i))$ in $(D\times\omega,\val_{\pi})$ and therefore the 
first-order existential part of $\alpha$ is satisfied. 
With a similar but easier argument it is straightforward that the existential 
generalized quantifier part of $\alpha$ is also satisfied.
%
% Turning to the universal parts of $\alpha$ we first show that every element of 
% $D\times\omega$, which does not have its type in $\vlist{T}$, 
% satisfies the positive part of some type in $\Sigma \cup \Pi$. 
% Actually, it is easy to prove a stronger claim: they satisfy the positive part 
% of a type in $\Sigma$. 
% This is direct from the universal part of $\alpha^{\bullet}$. 
% For the universal generalized quantifier we need to show that 
% $(D\times\omega,\val_{\pi}) \models \dqu y.\bigvee_{S\in\Sigma} \tau^+_S(y)$. 
% That is, only finitely many elements of $D\times\omega$ do not realize the 
% positive part of a type in $\Sigma$.
% Again from the universal part of $\posdbnfofo{\vlist{T}\cup\Sigma}$ we know 
% that \emph{every} element in $(D,\val)$ realizes the positive part of a type in 
% $\Sigma$. 
% Then the same applies to $(D\times\omega,\val_\pi)$, so that the set of elements 
% \emph{not} realizing the positive part of a type in $\Sigma$ is empty and thus,
% in particular, finite.
For the universal parts of $\posdbnfolque{\vlist{T}}{\Pi}{\Sigma}$ it is enough to observe that, because of the universal part of $\alpha^\bullet$, \emph{every} $d\in D$ realizes a positive type in $\Sigma$. By construction, the same applies to $(D\times\omega,\val_{\pi})$, 
therefore this takes care of both universal quantifiers.
\medskip
		
\noindent\fbox{$\Leftarrow$} 
Assuming that $(D\times \omega,\val_\pi) \models 
\posdbnfolque{\vlist{T}}{\Pi}{\Sigma}$,
we will show that $(D,\val) \models \alpha^\bullet$. 
The existential part of $\alpha^{\bullet}$ is trivial. 
For the universal part we have to show that every element of $D$ realizes the 
positive part of a type in $\Sigma$. 
Suppose not, and let $d\in D$ be such that $\lnot\tau_S^+(d)$ for all $S\in 
\Sigma$. 
Then we have $(D\times\omega,\val_\pi) \not\models \tau_S^+((d,k))$ for all $k$.
That is, there are infinitely many elements not realizing the positive part of 
any type in $\Sigma$. 
Hence we have $(D\times\omega,\val_\pi) \not\models \dqu y.\bigvee_{S\in\Sigma} 
\tau_S^+(y)$. 
Absurd, because that is part of $\posdbnfolque{\vlist{T}}{\Pi}{\Sigma}$.
\end{proof}
% \medskip

\noindent\fcnote[inline,nomargin]{%
Remark.
\textnormal{A little remark of why the above proof does not work for finite trees.
(1) On finite trees, any disjunct with $\Sigma\neq\emptyset$ is equivalent to $\bot$, hence we can assume $\Sigma=\emptyset.$
(2) Instead of taking $D\times\omega$ we have to take $D\times n$ for sufficiently large $n$.
(3) The $\Rightarrow$ direction fails since $\alpha^\bullet$ is always false for $\Sigma=\emptyset$.
(4) However, the same translation as for MSO probably works, that is, $\wmso /bis \equiv \muML$ over finite trees.}
\medskip
}

As a consequence of Proposition~\ref{p-1P} we obtain the following.

\begin{definition}
Given an automaton $\bbA = \tup{A,\tmap,\pmap,a_{I}}$ in $\Aut(\olque)$, define 
the automaton $\bbA^{\bullet} \df \tup{A,\tmap^{\bullet},\pmap,a_{I}}$ in 
$\Aut(\ofo)$ by putting, for each $(a,c) \in A \times C$:
\[
\tmap^{\bullet}(a,c) \df (\tmap(a,c))^{\bullet}.
\]
\end{definition}

\begin{proposition}
For any automaton $\bbA = \tup{A,\tmap,\pmap,a_{I}}$ in $\Aut(\olque)$, and any
model $\bbT$, $\bbA$ and $\bbT$ satisfy \eqref{eq:crux}.
\end{proposition}

\begin{proof}
The proof of this proposition is based on a fairly routine comparison of the 
acceptance games $\mathcal{A}(\bbA^{\bullet},\bbT)$ and 
$\mathcal{A}(\bbA,\bbT^{\om})$.
In a slightly more general setting, the details of this proof can be found 
in~\cite{Venxx}.
\end{proof}
\medskip

It remains to be checked that the construction $(-)^{\bullet}$, which has
been defined for arbitrary automata in $\Aut(\olque)$, transforms 
$\wmso$-automata into automata in the right class, viz., $\AutWC(\ofo)$.

\begin{proposition}
Let $\bbA \in \Aut(\olque)$.
If $\bbA \in \AutWC(\olque)$, then $\bbA^{\bullet} \in \AutWC(\ofo)$.
\end{proposition}

\begin{proof}
This proposition can be verified by a straightforward inspection, at the 
one-step level, that if a formula $\al \in {\olque}^+(A)$ belongs to the fragment 
$\cont{{\olque}^+}{a}(A)$, then its translation $\al^{\bullet}$ lands in 
the fragment $\cont{\ofo^+}{a}(A)$.
\end{proof}

\begin{remark}{\rm
As a corollary of the previous two propositions we find that 
\begin{itemize}
	\itemsep 0 pt
	\item $\Aut(\ofo) \equiv \Aut(\olque)/{\bis}$, and
	\item $\AutWC(\ofo) \equiv \AutWC(\olque)/{\bis}$.
\end{itemize}
In fact, we are dealing here with an instantiation of a more general phenomenon 
that is essentially coalgebraic in nature.
In~\cite{Venxx} it is proved that if $\llang$ and $\llang'$ are two one-step
languages that are connected by a translation $(-)^{\bullet}: \llang' \to 
\llang$ satisfying a condition similar to \eqref{eq-1cr}, then we find that 
$\Aut(\llang)$ corresponds to the bisimulation-invariant fragment of 
$\Aut(\llang')$: $\Aut(\llang) \equiv \Aut(\llang')/{\bis}$.
This subsection can be generalized to prove similar results relating
$\AutW(\llang)$ to $\AutW(\llang')$, and $\AutWC(\llang)$ to 
$\AutWC(\llang')$.
}\end{remark}


\subsubsection{From automata to formulae}\label{aut-to-formula}
% !TEX root = main-wmso.tex

%The aim of this subsection is to prove the following theorem:
%\begin{theorem}\label{t:wmsobis}
%%Over all models, the bisimulation invariant fragment of \wmso is included in the  fragment $\contAFMC$ of the modal $\mu$-calculus.
%$\WMSO/{\bis} \leq \contAFMC$.
%\end{theorem}
%
%For this purpose, we first observe that by Theorem \ref{t:venema} and Corollary \ref{cor:invariant} it immediately follows that:
%
%\begin{proposition}\label{prop:invariant}
%The class $\AutWC(\ofo)$ corresponds to the bisimu\-la\-tion-invariant fragment of the class of \wmso-automata.
%\end{proposition}

In this subsection we focus on the following theorem.

\begin{theorem}\label{t:autofor}
There is an effective procedure that, given an automaton $\bbA$ in
$\AutWC(\ofo)$, returns an equivalent formula $\xi_{\bbA}$ of the fragment 
$\contAFMC$ of the modal $\mu$-calculus.
\end{theorem}

%Together with Theorem \ref{t:wmsoauto}, this means that in order to prove Theorem \ref{t:wmsobis} it is enough to verify the following:
\begin{proof}
The argument  is a refinement of the standard proof showing that any automaton 
$\aut$ from $\Aut(\ofo)$ can be translated into an equivalent $\mu$-formula 
$\xi_\aut$ (cf. e.g. \cite{Ven08}), and it is essentially a special case of the argument proving Theorem \ref{thm:wmso_autofor}.

%\end{proofsketch}


%Together with Theorem \ref{t:wmsoauto}, this means that in order to prove Theorem \ref{t:wmsobis} it is enough to verify the following:
%
%
%\begin{theorem}\label{t:autofor}
%
%There is an effective procedure that given an automaton in $\AutWC(\ofo)$, returns an equivalent formula of the fragment $\contAFMC$ of the modal $\mu$-calculus.
%\end{theorem}


%In order to prove Theorem \ref{t:autofor}, we are going to slightly modify define another class of automata, called $\contAFMC$-automata, show that the obtained class is equivalent to the class $\AutWC(\ofo)$, and then follow the proof of Theorem 6.12(2) in \cite{Ven08} stating that there is an effective procedure that given a $\contAFMC$-automaton returns an equivalent formula of the modal $\mu$-calculus.
%%
%%First some definitions and notations. Given a set $X$, let $Latt(X)$ denote the set of fine disjunctions and finite conjunctions of elements of $X$.
%Given a set of propositional letters $P$, we let $\pm P$ de note the set of formulas of the form $p$ or $ \lnot p$, with $p \in P$.
%%
%%\begin{definition}
%%
%Given sets $P$ and $A$, the collection $Mterm(P,A)$ of special flat modal terms over $A$ is given by all formulas which are disjunctions of formulas of the form:
%\[  \bigwedge \Psi \land \bigvee(\bigwedge_{i \leq \ell} \Diamond a_i \land \Box( \bigvee_{j \leq n} \bigwedge_{i_j \leq k_j} \Phi_{i_j})),\]
%%\[
%%\varphi := p \mid \lnot p \mid \Diamond a \mid \Box a \mid \bigwedge \Phi \mid \bigvee \Phi
%%\]
%where $\Psi \subseteq \pm P$, $\{a_1, \dots, a_\ell\} \subseteq A$ and $\Phi \subseteq A$.
%%\end{definition}
%The semantics of special flat modal terms is defined as expected, as well as the corresponding notion of continuity (cf. Subsection \ref{subsec:mu}).
%We are now ready for the new definition of continuous weak modal automata.
%%
%\begin{definition}[(2)]\label{def:cont2}
%A \emph{$\contAFMC$-automa\-ton} is a tuple $\aut = \tup{A,\tmap,\pmap,\ord}$ such that
%\begin{enumerate}[(i)]
%\itemsep 0 pt
%\item $A$ is a finite set of states of the automaton,
%\item $\ord$ is a quasi-ordering on $A$ (i.e., reflexive and transitive),
%\item $\tmap: A \to MTerm(\pm P, A)$ %{\sf d}{\owmsoe}^+(A,A_0, A_1)
%is the transition map s.t. $\tmap(a) \in MTerm(\pm P, \{b\in A \mid a \ord b\})$,
%\item $\pmap: A \to \mathbb{N}$ is the parity map, satisfying the following two conditions:
%\begin{description}
%\item[weakness:]  if  $a \ord b$ then $\pmap(a) \leq \pmap(b)$,
%%\item[continuity:] if $a \in A_0$ then  $\pmap(a)$ is even, while if $a \in A_1$ then  $\pmap(a)$ is odd.
%\item[continuity:] let $a,b$ be states belonging to the same $\ord$-cycle. If ${\pmap(b)}$ is odd then $\tmap(b)$ is continuous in $a$. In case $\pmap(b)$ is even, then $\tmap(b)$ is co-continuous in $a$.
%\end{description}
%\end{enumerate}
%The run is defined as expected.
%\end{definition}
%%
%
%Now, it is not difficult to verify that:
%\begin{proposition}
%Definitions~\ref{def:cont1} and~\ref{def:cont2} are equivalent. That is, for every automaton from Definition~\ref{def:cont1}, one can effectively construct an equivalent automaton from Definition~\ref{def:cont2}, and vice versa.
%\end{proposition}
%\begin{proof}
%Firstly, by Lemma \ref{} and the correspondence between modal logic and first-order logic.
%\end{proof}
%
%
%In what follows, a continuos weak modal automaton will be an automaton from Definition~\ref{def:cont2}.
%%
%%The \emph{index} of a weak parity automaton is the maximal number associated to a node in a $\ord$-cycle. If there is no such cycle, we state the index is $-1$.


From now on, we always assume that a formula $\tmap(a,c)$ is in normal form. 
Following~\cite{Ven08}, we introduce another type of automata, called $(\prop,X)$-automata, which operate on $\p{(\prop \cup X)}$-trees.
They differ from automata whose one-step language is defined over predicates in $(A \cup X)$ in that\footnote{Parity automata based on $\ofo$ are thus simply $(\prop,\emptyset)$-automata.}
\begin{itemize}
\item Monadic predicate letters from $X$ can occur in the scope of a %existential
quantifier and only there, meaning that $(\prop,X)$-automata have transition $\tmap(a,c) \in \ofo^+(A\cup X)$
%
\item The transition function is uniquely determined by the restriction of the coloring to $\prop$, that is, for every $a \in A$, and $c_1, c_2 \in C$, if $c_1 \cap \prop = c_2 \cap \prop$ then $\tmap(a, c_1)= \tmap(a, c_2)$.
\end{itemize}
We also assume that
for every $x \in X$ there is a unique $a \in A$ and an unique $c \in C$ such that $x$ occurs in $\tmap(a,c)$.
The notion of acceptance is defined as expected, the only difference with being that during the acceptance game \'Eloise has to provide a valuation only for predicates in $A$ making formula given by the transition function true. %\fcnote{Maybe a bit more on the notion of acceptance?}
It is then enough to prove the following claim.
\begin{claimfirst}\label{c:1}
There is an effective procedure that, given a $(\prop,X)$-automaton $\aut$ gives an equivalent  formula $\xi_{\aut} \in \contAFMC$ in which all occurrences of variables in $X$ are positive.
\end{claimfirst}
\begin{pfclaim} %\ref{c:1}
%\fcwarning{where does the proof of claim start and end? use environment pfclaim}
Without loss of generality, we can assume that:
\begin{itemize}
\itemsep 0 pt
\item Every (maximal) strongly connected component (SCC) in the graph of $\ord$ has an unique entrance point,
\item The directed acyclic graph (DAG) of the SCCs of $\ord$ is a tree, and more specifically,
\item %Given a state $a \in A$, let $Reach(a)$ be the set of all states $b \in A$ that are not in the same SCC as $a$ but are such that $b$ occurs in $\tmap(a,c)$, for some  $c \in C$. Then, 
$\{c \in A \mid a \leadsto c, c \prec a\}  \cap \{c \in A \mid b \leadsto c, c \prec b\}  = \emptyset$ whenever $a,b$ are in the same SCC, with $a\neq b$.
\end{itemize}

Given a $(\prop,X)$-automaton $\aut$, we are now going to define a function $\delta_\aut: A \to \ML (A \cup X \cup \prop)$
%\fcwarning{one-step ML undefined} 
that assigns to each state $a$ of $\aut$ a modal formula $\delta_\aut(a)$ over  $A \cup X \cup \prop$ representing all possible transitions from $a$ in the modal language with the property that if $b \in A$ is in the same $\ord$-cycle of $a$ and $\pmap(a)=1$, then  $\delta_\aut(a)$ is continuous in $b$. Dually for $\pmap(a)=0$.

Let $c \in C$, and assume $\Delta(a,c)$ is in positive basic form $\bigvee \posdbnfofo{\Sigma}$. We define a first translation $TR_1$ taking as argument $\Delta(a,c)$ and giving as result a formula from the (guarded fragment of) first-order logic over $A \cup X \cup \prop$ as follows. %Assume $c \cap P= \{q_{c_1}, \dots, q_{c_\ell}\}$.
With every disjunct
$$
\posdbnfofo{\Sigma} = \bigwedge_{S\in\Sigma} \exists x. \tau^+_{S}(x) \land \forall z.( \bigvee_{S\in \Sigma} \tau^+_S(z)),
$$
%
%for some set of types $\Pi \subseteq \wp A$ and $T_i \subseteq A$,
%\fcerror{The basic form for FO was wrong before}
we associate the formula

$$
TR_1(\posdbnfofo{\Sigma}) := \bigwedge_{S\in\Sigma} \exists y. (R(x,y) \land \tau^+_{S}(y)) \land \forall z.( R(x,z) \to \bigvee_{S\in \Sigma} \tau^+_S(z)).
$$
\fcwarning{Why do we need to go through this?}The formula $TR_1(\posdbnfofo{\Sigma})$ is bisimulation invariant, and it is equivalent to the modal formula

$$
TR_2(\posdbnfofo{\Sigma}) := \bigwedge_{S\in\Sigma}  \Diamond(\bigwedge S) \land \Box \bigvee_{S\in \Sigma} (\bigwedge S).
$$
\fcwarning{Maybe better to define this directly}
Let $TR_3(\Delta(a,c))= \bigwedge (c \cap \prop) \land \bigwedge_{p \in \prop \setminus c} \lnot p \land \bigvee TR_2(\posdbnfofo{\Sigma})$.
 The modal formula $\delta_\aut(a)$ is then defined as

 \[
 \bigvee_{c \in C} TR_3(\Delta(a,c)).
 \]
By construction we have, for every $\model$,
%
%\begin{description}
%\item[(*)] 
    \begin{eqnarray*}
    \model[x\mapsto s_I] \models \bigvee_{c \in C} \big (\tau_{(c \cap \prop)}(y) \land \bigvee TR_1(\posdbnfofo{\Sigma})\big) & \text{iff} &\model \mmodels \delta_\aut(a).
    \end{eqnarray*}
%\end{description}
%\fcerror{$\model,s$ has no meaning in our setting}
%
A modal automaton over $\prop$ is an automaton $ \tup{A, \delta, \pmap, a_I}$ such that $\delta : A \to \ML^+ (A)$, where $\ML^+ (A)$ is the set of all modal formulas over propositions $A \cup \prop$ such that elements from $A$ appear only positively.
The acceptance game associated with such an automaton and a tree $\model$ is determined by the (symmetric) acceptance game defined according to the rules of Table~\ref{symmetric_modal_game}.
This means that we can equivalently see the automaton $\aut$ as a modal automaton $\tup{A, \delta_\aut, \pmap, a_I}$ whose transition function satisfies the weakness and continuity conditions. Thus, from now on we see $\aut$ as the equivalent modal automaton we have just described.%\fcwarning{Define $n_A$}

\begin{table}[h]
  \centering
\begin{tabular}{|l|c|l|c|}
 \hline
  % after \\: \hline or \cline{col1-col2} \cline{col3-col4} ...
  Position & Player & Admissible moves & Parity\\
   \hline
  $(a,s) \in A \times S$ & $\exists$ & $\{(\delta_\aut(a),s)\}$ & $\pmap(a)$\\
  $(\psi_1 \vee \psi_2,s)$ & $\exists$ & $\{(\psi_1,s),(\psi_2,s) \}$ & $-$ \\
  $(\psi_1 \wedge \psi_2,s)$ & $\forall$ & $\{(\psi_1,s),(\psi_2,s) \}$ & $-$ \\
  $(\Diamond\varphi,s)$ & $\exists$ & $\{(\varphi,t)\ |\ t \in R[s] \}$ & $-$ \\
  $(\Box\varphi,s)$ & $\forall$ & $\{(\varphi,t)\ |\ t \in R[s] \}$ & $-$ \\
  % $(\bot,s)$ & $\exists$ & $\emptyset$ & $n_A$ \\
  % $(\top,s)$ & $\forall$ & $\emptyset$ & $n_A$ \\
  $(\lnot p,s) \in \prop \times S$ and $p \notin \tscolors(s)$ & $\forall$ & $\emptyset$ & $-$\\
  $(\lnot p,s) \in \prop \times S$ and $p \in \tscolors(s)$ & $\exists$ & $\emptyset$ & $-$\\
  $(p,s) \in \prop \times S$ and $p \in \tscolors(s)$ & $\forall$ & $\emptyset$ & $-$\\
  $(p,s) \in \prop \times S$ and $p \notin \tscolors(s)$ & $\exists$ & $\emptyset$ & $-$\\

  \hline
\end{tabular}
 \caption{(Symmetric) acceptance game for modal automata}
 \label{symmetric_modal_game}
\end{table}





For the construction of $\xi_\aut$, we proceed by induction on the tree height of the DAG $t$ of SCC. If the height is 1, that is the DAG is a single point graph, we reason as follows.
 We have two cases to consider: either the SCC is trivial (i.e. it consists of a single non looping node), or not.
In the first case, $A=\{a_I\}$ and $\aut$ is equivalent to $\xi_\aut:=\delta_\aut(a_I)$.

For the second case, let us assume that $\pmap(a_I)=1$, the case when it is $0$ being, mutatis mutandis, the same.
Let $A=\{a_0, \dots, a_\ell\}$, and $a_I=a_\ell$.
Since $\aut$ is a weak modal automaton satisfying the continuity condition, given $a,b \in A$, if $b$ occurs in $\delta_\aut(a)$, then $b$ is only in the scope of $\Diamond$ operator. 
We can now see the automaton $\bbA$
as a system of modal equations, and by applying the standard inductive procedure \emph{solves} this system of equations and
construct the least fixpoint formula $\xi_\aut$ equivalent to $\aut$, 

%Now, since every variable bounded by a $\mu$-operator  in the obtained formula  is in the scope only of $\Diamond$ operators too. 
%The refinement that our setting requires is that if $\bbA$ is an automaton 
%in $\Aut(\ofo)$, then the solution of the associated system of equations 
%can be obtained as a formula in the fragment $\yvF \sse \MC$.
%
Here the key observation is that the weakness and continuity conditions on
strongly connected components of the automaton ensure that when we execute
a single step in solving the system of equations, we 
may work within the 
(syntactically) continuous fragment of the modal $\mu$-calculus.
From this, we can deduce that $\xi_\aut \in \contAFMC$. Clearly the procedure preserves the polarity of each $x \in X$, meaning that all variables in $X$ are positive in $\xi_\aut$.


For the induction step, assume the successors of the root of $t$ are $(n_1, \dots, n_\ell)$. For each $i \in \{1,\dots,\ell\}$, let $a_i$ be the entrance point of the SCC of $n_i$, and let $\aut_{i}$ be the automaton $\aut$ but having as an initial state $a_i$. If we do not consider the states that are not reachable by $a_i$, the DAG of the SCC of $\aut_{i}$ is $t.{n_i}$ (the subtree of $t$ starting at $n_i$).
Let $Y=\{a_1, \dots, a_\ell\}$, and $M$ be the set of states of $\aut$ that belong to the root of $t$. We assume $X \cap Y = \emptyset$. The structure
\[
\aut_M := \tup{M, \delta_\aut|_{M}, \pmap|_{M}, a_I}
\] is a $(\prop, X \cup Y)$-automaton.

The inductive hypothesis applies to automata $\aut_M, \aut_{1}, \dots, \aut_{\ell}$. Thus we obtain fixpoint formulas $\xi_M, \xi_1, \dots, \xi_\ell$, the former taking free variables in $\prop \cup X \cup Y$, all the remaining in $\prop \cup X$, equivalent to $\aut_M, \aut_1, \dots, \aut_\ell$ respectively.
Notice that:
\begin{itemize}
\item by induction hypothesis, $\xi_M, \xi_1, \dots, \xi_\ell \in \contAFMC$, every variable $x \in X \cup Y$ is positive in each of those formulas (if $x$ occurs in it),
\item by construction, if $x$ is free in $\xi_i$, then $x$ is not bounded in $\xi_{M}$.
\end{itemize}
 We can therefore deduce that
 $\xi_\aut= \xi_{M}[a_1\mapsto\xi_{1}, \dots, a_\ell\mapsto\xi_{\ell}] \in \contAFMC$, and that each variable from $X$ occurs positively in $\xi_\aut$.

 We verify that $\model \mmodels \xi_{M}[a_1\mapsto\xi_{1}, \dots, a_\ell\mapsto\xi_{\ell}]$ iff $\model \in \trees(\aut)$, for every model $\model$.
 %\fcwarning{I think the $(\model,s)$ thould be $\model$ or rather a variant of $\model$ with a changed point}
But this follows by the following two facts:
\begin{enumerate}
\item $\model\mmodels \xi_{M}[a_1\mapsto\xi_{1}, \dots, a_\ell\mapsto\xi_{\ell}]$ iff $\model[a_1\mapsto \ext{\xi_1}^{\model}, \dots, a_\ell\mapsto \ext{\xi_\ell}^{\model}  ] \mmodels \xi_M$,
\item $\model \in \trees(\aut)$  iff $\model[a_1\mapsto \ext{\aut_1}^{\model}, \dots, a_\ell\mapsto \ext{\aut_\ell}^{\model}  ] \in \trees(\aut_M)$
\end{enumerate}
where %$\ext{\xi_i\ext{_{\model}=\{ s \in \model \mid s \in \ext{\xi_i\ext{^\model \}$, and 
$\ext{\aut_i}^{\model} := \{ s \in \model \mid \model.s \in \trees(\aut_i) \}$. 
\end{pfclaim}
%Since $\aut$ was a weak automaton, we can thus use Lemma 2.4(1) of AlbeFac JSL and conclude that $\xi_{M}[a_1\mapsto\xi_{1}, \dots, a_\ell\mapsto\xi_{\ell}]$ is equivalent to $\aut$.
This finishes the proof of Theorem. 
\end{proof}


\clearpage


\appendix

\section{Proofs of Section \ref{sec:aut_to_formulas_nmso}}\label{app:nmso-aut-to-formulas}
\noindent


%%%%%%%%%%%%%%%%%%%%%%%%%%%%%%%%%%%%%%%%%%%%%%%%%%%%%%%%%%%%%%%%%%%%%%%%%%%%%%%%%%%%%%%%%%%%%%%%%%%%%%%%%%%%%%
%%%%%%%% WEAK AUTOMATA TO NDB %%%%%%%%%%%%%%%%%%%%%%%%%%%%%%%%%%%%%%%%%%%%%%%%%%%%%%%%%%%%%%%%%%%%%%%%%%%%%%%
%%%%%%%%%%%%%%%%%%%%%%%%%%%%%%%%%%%%%%%%%%%%%%%%%%%%%%%%%%%%%%%%%%%%%%%%%%%%%%%%%%%%%%%%%%%%%%%%%%%%%%%%%%%%%%


In this appendix we give a more detailed proof of Theorem \ref{thm:nmso_autofor}.

\begin{definition} Let $B = \{b_1,\dots,b_k\}$ and $P = \{p_1,\dots,p_j\}$ be two finite collection of set variables, representing respectively the states of an B\"{u}chi automaton and the propositional letters forming the labels of a $C$-labeled tree. In table \ref{fig:tableWFMSOFormulae} we fix some abbreviations for $\nmso$ formulae, expressing concepts which are easily seen to be definable in this logic. All valuations (for which we use the notation $\|\cdot\|$) are referred to a fixed tree $\model$. Observe that for each $p_i \in P$ the set $\|p_i\|$ is determined by the labelling function $\V:T\rightarrow C$ of $\model$.

\begin{table}[ht]\centering
%\begin{center}
\begin{tabular}{|p{1.8cm}|p{11.5cm}|}
\hline
\textbf{Formula} & \textbf{Meaning} \\ \hline \hline
$\mathit{Pfix}_z(p)$ & The set $\|p\|$ is a prefix of the subtree $\model.{\|z\|}$. \\
\hline
$\mathit{Root}(x)$ & The node $\|x\|$ is the root of $\model$. \\
\hline
$\mathit{Front}(p)$ & The set $\|p\|$ is a frontier of $\model$. \\
\hline
$\mathit{State}_{a,B}(x)$ & $a$ is the only set variable in $B$ such that the node $\|x\|$ is in $\|a\|$ (we say that \emph{the state $a$ marks $\|x\|$}). \\
 \hline
$\mathit{Part}_B(p)$ & Each node in the set $\|p\|$ is marked with a unique $a \in B$. \\
\hline
$\mathit{Trans}_{B,C}(p)$ & For each node $\|x\| \in \|p\|$, state $a \in B$, label $c\in C$, if $a$ marks $\|x\|$ and $\|x\|$ is labeled with $c$ then $\tmap(a,c)$ holds in $\R{\|x\|}$ (this latter condition is rendered by relativizing all quantifiers of $\tmap(a,c)$ to elements of $\R{\|x\|}$). \\
\hline
\end{tabular}
%\end{center}
\label{fig:tableWFMSOFormulae}\caption{{Abbreviations for $\nmso$ formulae}}\end{table}

We also define $\mathit{Surv}_{B,C}(p)$ as $\mathit{Part}_B(p) \wedge \mathit{Trans}_{B,C}(p)$, where $\mathit{Part}_B(p)$ and $\mathit{Trans}_{B,C}(p)$ are given as in table \ref{fig:tableWFMSOFormulae}. Intuitively, if $\mathit{Surv}_{B,C}(p)$ holds then player $\exists$ is guaranteed to have a legitimate move available from any node $\|x\|$ in $\|p\|$, assigning exactly one state to each $t \in \R{\|x\|}$.
\end{definition}


\begin{definition}\label{DEF_K_m} Let $\baut$ and $\overline{\baut}$ be B\"{u}chi automata, with $B = \{b_1,\dots,b_k\}$ and $F\subseteq B$ respectively the set of states and of accepting states of $\baut$. For each $b \in B$, we define by induction a sequence of formulae $K_i^b(x)$. Put $K_0^{b}(x) := \top$. The formula $K_{i+1}^{b}(x)$ is given as follows:

\begin{align*}
% \nonumber to remove numbering (before each equation)
  K_{i+1}^{b}(x)\ :=\ & \forall p\ \exists p^{\prime}\ \exists b_1\dots\exists b_k\ \Bigg(\mathit{Pfix}_x(p) \rightarrow
                       \bigg(p \subseteq p^{\prime} \wedge \mathit{Pfix}_x(p^{\prime}) \wedge \mathit{Surv}_{B,C}(p^{\prime}) \wedge
                       \mathit{State}_{b,B}(x)  \\
                       & \wedge \Big(\forall y\ \big(y\in \mathit{Front}(p^{\prime})\rightarrow 
                       (\bigvee_{b^{\prime} \in F} (\mathit{State}_{b^{\prime},B}(y) \wedge K_i^{b^{\prime}}(y)))\big)\Big)\bigg)\Bigg).
\end{align*}
Let $m$ be the product of the cardinalities of the carriers of $\baut$ and $\overline{\baut}$. The formula $\varphi_{\baut,\overline{\baut}} \in \nmso$ is defined by putting
\begin{equation*}
    \varphi_{\baut,\overline{\baut}}\ :=\ \exists y\ (\mathit{Root}(y) \wedge K_{m+1}^{b_I}(y)).
\end{equation*}
\end{definition}

Observe that, for any $k < \omega$, $K_k^b(x)$ is a formula of $\nmso$. We refer to section \ref{sec:weakgoeswell} for an intuitive reading of the semantics of $K_{i+1}^{b}(x)$.

Our next goal is to show the main result of section \ref{sec:weakgoeswell}.

\begin{trivlist}
\item \textbf{Theorem~\ref{thm:nmso_autofor}}. \ThWeakAutToWFMSO
\end{trivlist}

In this aim, we fix the following terminology concerning a winning strategy $f$ for $\exists$ in some acceptance game $\mc{G}$: a basic position $(a,s)$ is \emph{$f$-admissible} if there is some $f$-conform match of $\mc{G}$ where $(a,s)$ occurs. %Recall that functional strategies are the ones assigning \emph{exactly one state} to each successor of the node under consideration. Then we can observe that, if $f$ is functional, each node $s$ of $\model$ can be associated with a unique state $a_s$ of $\aut$ and a unique $f$-admissible position $(a_s,s)$.

In the next we fill in the details of the proof sketch provided in section \ref{sec:weakgoeswell}. As stated there, we can assume to work with B\"{u}chi automata $\baut$ and $\overline{\baut}$, equivalent respectively to $\aut$ and the weak $\MSO$-automaton $\overline{\aut}$ recognizing the complement $\overline{\mathcal{L}(\aut)}$ of the tree language $\mathcal{L}(\aut)$. Then theorem \ref{thm:nmso_autofor} is an immediate consequence of the following proposition.

\begin{proposition} Let $\baut= \langle B,b_I,\tmap,F\rangle$ and $\overline{\baut} = \langle \overline{B},\overline{b}_I,\overline{\tmap},\overline{F}\rangle$ be B\"{u}chi automata such that $\mathcal{L}(\overline{\baut}) = \overline{\mathcal{L}(\baut)}$. Let $\varphi_{\baut,\overline{\baut}} \in \nmso$ be given in terms of $\baut$ and $\overline{\baut}$ according to definition \ref{DEF_K_m}. Over tree languages, we have that $\mathcal{L}(  \baut) = \|\varphi_{\baut,\overline{\baut}}\|$. \end{proposition}
%%%%%%%%%%%%%%%%%%%%%%%%%%%%%%%%%%%%%%%%%%%%%%%%%%%%%%%%%%%%%%%%%%%%%%%%%%%%%%%%%%%%%%%%%%%%%%%%%%%%%%%%%%%%%%%%%%%%%%%%%%%%%%%%%%%%%%%%%%%%%%%%%%%%%%%%%%%%%%%%%%%%%%%%
%%%%%%%%%%%%%%%%%%%%%%%%%%%%%%%%%%%%%%%%%%%%%%%%%%%%%%%%%%%%%%%%%%%%%%%%%%%%%%%%%%%%%%%%%%%%%%%%%%%%%%%%%%%%%%%%%%%%%%%%%%%%%%%%%%%%%%%%%%%%%%%%%%%%%%%%%%%%%%%%%%%%%%%%
%%%%%%%%%% PROOF TO BE REFINED ? %%%%%%%%%%%%%%%%%%%%%%%%%%%%%%%%%%%%%%%%%%%%%%%%%%%%%%%%%%%%%%%%%%%%%%%%%%%%%%%%%%%%%%%%%%%%%%%%%%%%%%%%%%%%%%%%%%%%%%%%%%%%%%%%%%%%%%%%%%%%%%%
%%%%%%%%%%%%%%%%%%%%%%%%%%%%%%%%%%%%%%%%%%%%%%%%%%%%%%%%%%%%%%%%%%%%%%%%%%%%%%%%%%%%%%%%%%%%%%%%%%%%%%%%%%%%%%%%%%%%%%%%%%%%%%%%%%%%%%%%%%%%%%%%%%%%%%%%%%%%%%%%%%%%%%%%
%%%%%%%%%%%%%%%%%%%%%%%%%%%%%%%%%%%%%%%%%%%%%%%%%%%%%%%%%%%%%%%%%%%%%%%%%%%%%%%%%%%%%%%%%%%%%%%%%%%%%%%%%%%%%%%%%%%%%%%%%%%%%%%%%%%%%%%%%%%%%%%%%%%%%%%%%%%%%%%%%%%%%%%%
\begin{proof} \begin{Iff-RL} Let $\model$ be a tree and $f$ be a winning strategy for $\exists$ in $\mc{G} = \agame(\baut,\mathbb{T})@(b_I,s_I)$, which we can assume to be functional by Remark \ref{rmk:Buchifunctional}. Since $f$ is winning then we are provided with an $\omega$-accepting sequence $(E_i)_{i < \omega}$ for $f$ over $\baut$ and $\model$, according to proposition \ref{PROP_Buchi_finite_segments}. Our goal is to show that $\mathbb{T} \models \varphi_{\baut,\overline{\baut}}$. In fact, it suffices to show the following statement.

\smallskip

\begin{claim}\label{CLAIM_WeakParityInWFMSO2} For each $i < \omega$, for each $(b,s)\in B\times T$, if $(b,s)$ is a winning position for $\exists$ in $\mc{G}$, then $\mathbb{T} \models K_i^{b}(x)$, with $\|x\| = s$. \end{claim}

\begin{proof}[Proof of Claim \ref{CLAIM_WeakParityInWFMSO2}]
We proceed by induction on $i$. Since $K_0^{b}(x) = \top$, the base case is trivial. Inductively, let $(b,s)$ be a winning position for $\exists$ in $\mc{G}$. We put $\|x\| = s$ and we claim that $\mathbb{T} \models K_{i+1}^{b}(x)$. Following the syntactic shape of $K_{i+1}^{b}(x)$, we let $\|p\|$ be an arbitrary prefix $E$ of $\mathbb{T}.s$. By definition of the sequence $(E_i)_{i < \omega}$, for each $i < \omega$ we have that $\mathit{Ft}(E_i) < \mathit{Ft}(E_{i+1})$, implying that there is some prefix $E_n$ in the sequence such that $E \subseteq E_n$. We pick $E_n \cap {T.s}$ as the witness for the set-variable $p^{\prime}$ in $K_{i+1}^{b}(x)$.

We still need to provide witnesses for set-variables $b_1,\dots,b_k$ occurring in $K_{i+1}^{b}(x)$. The idea is to let them be suggested by the strategy $f$. Since $f$ is functional, any node $s \in \model$ is associated with a unique $b_s \in B$ and a unique $f$-admissible basic position $(b_s,s)$. For each $b_j$ in $\{b_1,\dots,b_k\}$, we define its valuation by putting
\begin{eqnarray}\label{EQ_b_j_eval}
% \nonumber to remove numbering (before each equation)
  \|b_j\| &:=& \{s \in (E_n \cap T.s)\ |\ b_j = b_s\}.
\end{eqnarray}
Since $E_n \cap T.s$ is well-founded then $\|b_j\|$ is noetherian, so that it is a legitimate witness for $b_j$ according to the semantics of $\nmso$.

The subformula ${Surv}_{B,C}(p^{\prime})$ of $K_{i+1}^{b}(x)$ holds because the strategy $f$ is assumed to be functional and winning for $\exists$, so in particular it is functional and surviving for $\exists$ in $E_n \cap T.s = \|p^{\prime}\|$. Concerning the subformula $\mathit{State}_{b,B}(x)$, by assumption $(b,s)$ is a winning position for $\exists$. This means that $b$ is the unique set-variable marking $s = \|x\|$ according to \eqref{EQ_b_j_eval}, so that $\mathit{State}_{b,B}(x)$ holds. It remains to show the statement
\begin{eqnarray}\label{EQ_forallinF_K_m}
% \nonumber to remove numbering (before each equation)
  \forall y\ (y\in \mathit{Front}(p^{\prime})\rightarrow (\bigvee_{b \in F} \mathit{State}_{b,B}(y) \wedge K_i^{b}(y))).
\end{eqnarray}
For this purpose, let $\|y\|$ be some node on the frontier of $E_n = \|p^{\prime}\|$. By \eqref{EQ_b_j_eval} and the fact that $f$ is functional, there is a unique set-variable $\|b\|$ marking $\|y\|$, such that $(b,\|y\|)$ is $f$-admissible. Therefore $(b,\|y\|)$ is a winning position for $\exists$ in $\mc{G}$, and $K_i^{b}(y)$ holds by inductive hypothesis. The fact that $b$ is in $F$ follows from properties of the frontier of $E_n$ as in definition \ref{DEF_accepting_sequence}.
\end{proof}

By applying claim \ref{CLAIM_WeakParityInWFMSO2} to the winning position $(b_I,s_I)$ we have that $\mathbb{T} \models K_{n}^{b_I}(x)$ for each $n<\omega$, with $x$ witnessed by $s_I$. Then in particular $\mathbb{T} \models \exists x\ (\mathit{Root}(x) \wedge K_{m+1}^{b_I}(x))$. This completes the proof of direction $(\Rightarrow)$.

\end{Iff-RL}
\begin{Iff-LR} Let $\model$ be a tree where $\varphi_{\baut,\overline{\baut}}$ is true. We need to show that $\model$ is accepted by $\baut$.

The idea of the proof is as follows. Suppose by way of contradiction that $\baut$ does not accept $\mathbb{T}$. Then the tree $\model$ is accepted by $\overline{\baut}$. Let $\overline{f}$ be a functional winning strategy for $\exists$ in $\agame(\overline{\baut},\mathbb{T})$. Suppose that we can prove from the previous assumptions the existence of an $m$-trap for $\baut$ and $\overline{\baut}$. Then by proposition \ref{PROP_Rabin_trap} we have that $L(\baut) \cap L(\overline{\baut}) \neq \emptyset$, contradicting the fact that $L(\overline{\baut}) = \overline{L(\baut)}$.

In order to complete the proof of direction $(\Leftarrow)$, it remains to verify the following claim.

\smallskip

\begin{claim}\label{CLAIM_WeakParityInWFMSO3} There exists an $m$-trap for $\baut$ and $\overline{\baut}$.\end{claim}

\begin{proof}[Proof of Claim \ref{CLAIM_WeakParityInWFMSO3}] By definition \ref{DEF_Rabin_trap}, we have to to provide the following components:
\begin{enumerate}
  \item a strictly increasing sequence $(E_i)_{i\leq m}$ of prefixes of $\mathbb{T}$, with $E_0 = \{s_I\}$;
  \item a strategy $f^{B}$ for $\exists$ in $\mc{G}=\mc{A}(\baut,\mathbb{T})@(b_I,s_I)$ which is surviving for $\exists$ in $E_m$;
  \item a strategy $f^{\overline{B}}$ for $\exists$ in $\overline{\mc{G}}=\mc{A}(\overline{\baut},\mathbb{T})@(\overline{b}_I,s_I)$ which is surviving for $\exists$ in $E_m$;
  \item an $m$-accepting sequence $(G_i^{B})_{i\leq m}$ for $f^{B}$ over $\baut$ and $\mathbb{T}$;
  \item an $m$-accepting sequence $(G_i^{\overline{B}})_{i\leq m}$ for $f^{\overline{B}}$ over $\overline{\baut}$ and $\mathbb{T}$.
\end{enumerate}
Moreover, $(E_i)_{i\leq m}$, $(G_i^{B})_{i\leq m}$ and $(G_i^{\overline{B}})_{i\leq m}$ have to present the interleaving behavior described in definition \ref{DEF_Rabin_trap}.

\smallskip

We put the strategy $\overline{f}$ as witness for $f^{\overline{B}}$. By assumption $\overline{f}$ is a winning strategy for $\exists$ in $\overline{\mc{G}}$. Then, by proposition \ref{PROP_Buchi_finite_segments}, we are also given with an $\omega$-accepting sequence $(E^{\overline{f}}_i)_{i< \omega}$ for $\overline{f}$ over $\overline{\baut}$ and $\model$.

It remains to define the other components of the $m$-trap, which is what we do next. The idea is to define the surviving strategy $f^{B}$, the sequences $(E_i)_{i\leq m}$ and $(G_i^{B})_{i\leq m}$ by using the assumption that $\mathbb{T} \models \varphi_{\baut,\overline{\baut}}$. The last component, namely the sequence $(G_i^{\overline{B}})_{i\leq m}$, will be defined from $(E^{\overline{f}}_i)_{i< \omega}$.

\smallskip

The construction of the strategy $f^{B}$ and the sequences $(E_i)_{i\leq m}$, $(G_i^{B})_{i\leq m}$ and $(G_i^{\overline{B}})_{i\leq m}$ proceeds in stages, by induction on $i \leq m$. In particular, $f^{B}$ will be defined as the last element $f^{B}_m$ in a sequence of strategies $(f^{B}_i)_{i\leq m}$.

Given $i\leq m$, the inductive hypothesis that we want to maintain along the construction can be expressed as follows.

\bigskip
\begin{center}
\fbox{\parbox{13.4cm}{
\begin{enumerate}
  \item If $i < m$ then $\mathit{Ft}(E_{i}) \leq \mathit{Ft}(G_{i}^{B}) < \mathit{Ft}(E_{i+1})$, otherwise $i = m$ and $\mathit{Ft}(E_{i}) \leq \mathit{Ft}(G_{i}^{B})$.
  \item If $i < m$ then $\mathit{Ft}(E_i) \leq \mathit{Ft}(G_{i}^{\overline{B}}) < \mathit{Ft}(E_{i+1})$, otherwise $i = m$ and $\mathit{Ft}(E_{i}) \leq \mathit{Ft}(G_{i}^{\overline{B}})$.
  \item The sets $E_{i}$, $G_{i}^{B}$ $G_i^{\overline{B}}$ are prefixes of $\model$.
  \item The function $f^{B}_{i}$ is a strategy $\exists$ in $\mc{G}$ which is functional and surviving in $G^{B}_{i}$. If $i \geq 1$, then $f^{B}_{i}$ extends $f^{B}_{i-1}$.
  \item For each $s \in \mathit{Ft}(G^{\overline{B}}_i)$, there is a unique $\overline{b}_s \in \overline{B}$ such that the position $(\overline{b}_s,s)$ is $\overline{f}$-admissible; also, $\overline{b}_s$ is in $\overline{F}$.
  \item For each $s \in \mathit{Ft}(G^{B}_i)$, there is a unique $b_s \in B$ such that the formula $K_{m-i}^{b_s}(x)$ holds for $s = \|x\|$. The position $(b_s,s)$ is $f^{B}_i$-admissible; also, $b_s$ is in $F$.
\end{enumerate}
}}\hspace*{0.2cm}($\ddag$)
\end{center}

\bigskip

Let us first show why the different components form an $m$-trap if condition ($\ddag$) can be maintained. By ($\ddag . 4$) the strategy $f^{B} = f^{B}_m$ for $\exists$ in $\mc{G}$ would be functional and surviving in $G^{B}_{m}$. By ($\ddag . 1$) we have that $\mathit{Ft}(E_{m}) \leq \mathit{Ft}(G_{m}^{B})$, meaning that $f^{B}$ is also surviving in $E_{m}$, as requested by point 1 of the definition of $m$-trap (definition \ref{DEF_Rabin_trap}). For $f^{\overline{B}} = \overline{f}$, we know by assumption that $\overline{f}$ is functional and winning for $\exists$ in $\overline{\mc{G}}$. Since $E_m$ is a subset of $T$, then $f^{\overline{B}}$ is also surviving in $E_m$, as requested by point 2 of definition \ref{DEF_Rabin_trap}.

For points 3 and 4 of definition \ref{DEF_Rabin_trap}, we have to check that $(G_i^{B})_{i\leq m}$ and $(G_i^{\overline{B}})_{i\leq m}$ are $m$-accepting sequences respectively for $f^{B}$ and $f^{\overline{B}}$. For this purpose, there are three conditions to check according to the definition of accepting sequence (definition \ref{DEF_accepting_sequence}). The first condition is that $(G_i^{B})_{i\leq m}$ and $(G_i^{\overline{B}})_{i\leq m}$ are sequences of prefixes, which is given by ($\ddag . 3$). The second condition, on the relation between frontiers of each $G_i^{B}$, $G_i^{\overline{B}}$ and $E_i$, is given by ($\ddag . 1$) and ($\ddag . 2$). Concerning the third condition of definition \ref{DEF_accepting_sequence}, for each $i\leq m$, the requirements on $\mathit{Ft}(G_i^{B})$ and $\mathit{Ft}(G_i^{\overline{B}})$ are fulfilled by ($\ddag . 5$) and ($\ddag . 6$).

The last two points of definition \ref{DEF_Rabin_trap}, concerning the interleaving of the frontiers of $(E_i)_{i\leq m}$, $(G_i^{B})_{i\leq m}$ and $(G_i^{\overline{B}})_{i\leq m}$, just correspond to ($\ddag . 1$) and ($\ddag . 2$). Therefore what we obtain is indeed an $m$-trap, provided that we are able to maintain condition ($\ddag$).

\bigskip

Now we proceed with the inductive construction. For the base case, let $E_0 := \{s_I\}$. We define the first element $G_0^{\overline{B}}$ in the sequence $(G_i^{\overline{B}})_{i\leq m}$ as the smallest prefix in the sequence $(E^{\overline{f}}_i)_{i< \omega}$ such that $E_0 \subseteq G_0^{\overline{B}}$, that is simply $E^{\overline{f}}_0$ because $(E^{\overline{f}}_i)_{i< \omega}$ is monotone.

In order to define $G_0^B$, we observe that the unique witness for $x$ in $\exists x\ (\mathit{Root}(x)\wedge K_{m+1}^{b_I}(x))$ must be $s_I$. Then, by putting $E_0$ as the witness of the variable $p$ in $K_{m+1}^{b_I}(x)$, we are provided with a prefix $G^{B}_0$ witnessing the variable $p^{\prime}$ in $K_{m+1}^{b_I}(x)$. We let such $G^{B}_0$ be the first element in the sequence $(G_i^{B})_{i\leq m}$.

In order to define the first surviving strategy $f^{B}_0$ in the sequence $(f^{B}_i)_{i\leq m}$, we fix valuations $\|p\|=E_0$ and $\|p^{\prime}\| = G^{B}_0$ in the formula $K_m^{b_I}(x)$ and we consider the witnesses for set-variables $b_1,\dots,b_k$ in $K_{m+1}^{b_I}(x)$. By definition of $K_{m+1}^{b_I}(x)$, for each node $s$ in $G^{B}_0$ there is a unique $b_s \in \{b_1,\dots,b_k\}$ such that $s \in \|b_s\|$. This yields a strategy $f^{B}_0$ for $\exists$ in $\mc{G}$ (actually, for partial matches which are played along nodes of $G^{B}_0$), which we define as follows:
 \begin{enumerate}
   \item $f^{B}_0$ is defined at the basic position $(b_I,s_I)$;
   \item given a basic position $(b_s,s) \in B \times G^{B}_0$ with $s \not\in \mathit{Ft}(G^{B}_0)$, we let $f^{B}_0$ suggest to $\exists$ a marking assigning $b_t$ to $t$, for each $t \in \R{s}$;
   \item we leave $f^{B}_0$ undefined on all other basic positions from $B \times T$.
 \end{enumerate}

Given prefixes $G_0^{\overline{B}}$ and $G_0^B$ as above, we define $E_1$ to be the smallest prefix of $\model$ such that $\mathit{Ft}(G_0^B) < \mathit{Ft}(E_1)$ and $\mathit{Ft}(G_0^{\overline{B}}) < \mathit{Ft}(E_1)$.

\smallskip

It remains to check that conditions $1-5$ in $(\ddag)$ hold for the base case. Condition $(\ddag .1)$, $(\ddag .2)$ and $(\ddag .3)$ are clear by construction of $E_0$, $G_0^{\overline{B}}$, $G_0^B$ and $E_1$. For condition $(\ddag .4)$, by assumption we have that $\mathit{State}_{b_I,B}(y)$ and $\mathit{Surv}_{B,P}(p^{\prime})$ hold, being subformulae of $K_m^{b_I}(x)$, with $\|p^{\prime}\| = G^{B}_0$ and $\|y\| = s_I$. By construction of the strategy $f^{B}_0$, this means that $f^{B}_0$ is functional and surviving for $\exists$ in $G^{B}_0$. Analogously, $(\ddag .5)$ holds because the subformula of $K_m^{b_I}(x)$ given as in \eqref{EQ_forallinF_K_m} is true, meaning that every node on the frontier of $G^{B}_0$ is associated with a unique accepting state of $\baut$ according to $f^{B}_0$. In order to fulfill condition $(\ddag .6)$, we observe that, by definition of $K_{m+1}^{b_I}(x)$, every node $s \in \mathit{Ft}(G^{B}_0)$ is associated with a basic position $(b_s,s) \in B \times T$, such that $b_s \in F$ and $K_{m}^{b_s}(x)$ holds for $s = \|x\|$.


\bigskip

Inductively, we consider the stage $j+1 \leq m$ of the construction. By inductive hypothesis, we are given with sequences $(E_i)_{i\leq {j+1}}$, $(G_i^{B})_{i\leq j}$, $(G_i^{\overline{B}})_{i\leq j}$ and a strategy $f^{B}_j$ for $\exists$ as in $(\ddag)$.

Analogously to the base case, we define $G_{j+1}^{\overline{B}}$ as smallest prefix in the sequence $(E^{\overline{f}}_i)_{i< \omega}$ which contains $E_{j+1}$. For the definition of $G_{j+1}^{B}$ and $f^{B}_{j+1}$, the key observation is that, by inductive hypothesis, for each node $s \in \mathit{Ft}(G^{B}_j)$ we can make the following assumptions:
 \begin{enumerate}
   \item the formula $K_{m-j}^{b_s}(x)$ holds, with $s = \|x\|$;
   \item the position $(b_s,s)$ is $f^{B}_j$-admissible.
 \end{enumerate}

We let $T.s \cap E_{j+1}$ be the witness for the set-variable $p$ occurring in $K_{m-j}^{b_s}(x)$. Then by definition of $K_{m-j}^{b_s}(x)$ we are provided with a prefix $G_{j+1}^{B,s}$ of $\model.s$ witnessing the variable $p^{\prime}$, such that ${T.s} \cap E_{j+1} \subseteq G_{j+1}^{B,s}$. Also we are provided with noetherian sets of nodes witnessing variables $b_1,\dots,b_k$. Analogously to the base case, this yields a strategy $f^{B,s}_{j+1}$ for $\exists$ in $\mc{G}$, which is defined as follows:
 \begin{enumerate}
   \item $f^{B}_{j+1}$ is defined at the basic position $(b_s,s)$;
   \item for each basic position $(b_t,t)\in B \times G_{j+1}^{B,s}$ with $t \not\in \mathit{Ft}(G_{j+1}^{B,s})$, we let $f^{B}_{j+1}$ suggest to $\exists$ a marking assigning $b_r$ to $r$, for each $r \in \R{t}$.
   \item we leave $f^{B}_{j+1}$ undefined on all other basic positions from $B \times T$.
 \end{enumerate}
In other words, $f^{B,s}_{j+1}$ is a strategy for $\exists$ in partial matches of $\mc{A}(\aut,\model)@(b_s,s)$, which is functional, surviving in $G_{j+1}^{B,s}$ and marks each node $t \in \mathit{Ft}(G_{j+1}^{B,s})$ with a unique state from $F$.

\smallskip

We define $G_{j+1}^{B} := G_{j}^{B} \cup \bigcup_{s \in \mathit{Ft}(G^{B}_j)} G_{j+1}^{B,s}$. Since $G_{j}^{B}$ is a prefix of $\model$ and for each $s \in \mathit{Ft}(G^{B}_j)$ the set $G_{j+1}^{B,s}$ is a prefix of $\model.s$, we have that $G_{j+1}^{B}$ is a prefix of $\model$. Next, we define $f_{j+1}^{B} := f_j^{B} \cup \bigcup_{s \in \mathit{Ft}(G^{B}_j)} f_{j+1}^{B,s}$, where the union of strategies just means the union of their graphs. In order to check that $f_{j+1}^{B}$ is indeed a function, observe that by inductive hypothesis $f_j^{B}$ is defined on basic positions in $B \times (G_{j}^{B} \setminus \mathit{Ft}(G_{j}^{B}))$. By construction, for each $s \in \mathit{Ft}(G^{B}_j)$, the strategy $f_{j+1}^{B,s}$ is defined on the union of $\{(b_s,s)\}$ and $B \times (G_{j+1}^{B,s} \setminus (G_{j}^{B} \cup \mathit{Ft}(G_{j+1}^{B,s})))$. Since $(b_s,s) \in \mathit{Ft}(G_{j}^{B})$, then the domains of $f_j^{B}$ and each $f_{j+1}^{B,s}$ are all disjoints. Therefore $f_{j+1}^{B}$ is uniquely defined on each basic position in its domain.

\begin{figure}[h]
  % Requires \usepackage{graphicx}
  \includegraphics[width=8cm]{appendix/FIG_PropBuchiToWFMSO.png}\\
  \caption{\rm Construction of $G_{j+1}^B$}\label{FIG_PropBuchiToWFMSO}
\end{figure}

%For each $s \in \mathit{Ft}(G^{B}_i)$, the well-founded subtree $G_{i+1}^{B,s}$ and the strategy $f^{B,s}_{i+1}$ are defined as above.

%We briefly check that $G_{i+1}^{B}$ and $f_{i+1}^{B}$ have the desired properties. By construction for each $s \in \mathit{Ft}(G^{B}_i)$ we know that $\model_s \cap E_i\subseteq G_{i+1}^{B,s}$. It follows that $\mathit{Ft}(\model_s \cap E_i) \leq \mathit{Ft}(G_{i+1}^{B,s})$. By inductive hypothesis we also know that $\mathit{Ft}(G^{B}_i) < \mathit{Ft}(E_{i+1})$. Then it follows that \begin{equation}\label{EQ_frontiers_inductive_case}   \mathit{Ft}(E_i) \leq \mb{Ft}(\bigcup_{s \in \mathit{Ft}(G^{B}_i)} G_{i+1}^{B,s}) = \mb{Ft}(G_{i+1}^{B} \end{equation}


%The fact that $f_{i+1}^{B}$ is full, functional and surviving for $\exists$ in $\mc{A}(\baut,E_{i+1}^{B})@(b_I,s_I)$ can be checked as follows. By inductive hypothesis $f_{i}^{B}$ is full, functional and surviving for $\exists$ in $\mc{A}(\baut,E_{i}^{B})@(b_I,s_I)$.


Given $G_{j+1}^{\overline{B}}$ and $G_{j+1}^B$ as above, if $j+1 < m$ then we define $E_{j+2}$ to be the smallest prefix of $\model$ such that $\mathit{Ft}(G_{j+1}^B) < \mathit{Ft}(E_{j+2})$ and $\mathit{Ft}(G_{j+1}^{\overline{B}}) < \mathit{Ft}(E_{j+2})$. The check that all conditions in $(\ddag)$ hold for $G_{j+1}^{B}$, $f_{j+1}^{B}$ and $E_{j+2}^{B}$ is completely analogous to the base case.

We have just defined a strategy $f^{B}$, sequences $(E_i)_{i\leq m}$, $(G_i^{B})_{i\leq m}$ and $(G_i^{\overline{B}})_{i\leq m}$, such that for each $i \leq m$ condition $(\ddag)$ is respected. It follows that $\overline{f}$ and $f^{B}$ witness a trap for $\baut$ and $\overline{B}$ according to definition \ref{DEF_Rabin_trap}. This concludes the proof of the claim.
\end{proof}

The proof of claim \ref{CLAIM_WeakParityInWFMSO3} completes the proof of direction $(\Leftarrow)$.

\end{Iff-LR}

\end{proof} 

%\clearpage
% Bibliography
\bibliographystyle{ACM-Reference-Format-Journals}
\bibliography{logic}

% History dates
\received{February 2017}{XXX}{XXX}


\end{document}


