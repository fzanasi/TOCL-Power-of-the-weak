In this Section we use the tools developed in the previous parts to prove  the main results of the paper, namely that both $M=\mucML, L=\wmso$ and  $M=\AFMC, L=\nmso$ are  solutions of the equation:

\begin{equation*}
%\label{eq:1}
%\eqno{(*)}
M \equiv L / {\bis} \text{ (over the class of all LTSs)}
\end{equation*}


%Moreover, we show that in both cases the equivalences are effective. %That is, we give effective translations in both directions.

As explained in the introduction, the structure of the proof is the same in both cases. 
 Indeed, the inclusion $M \subseteq L / {\bis} $ is  given by a more or less straightforward effective translation from $M$ into $L$ (Theorem \ref{theo:ML2SO}). 
 The other inclusion on the other hand passes thorough automata.
 As a key technical result of this part, in subsection~\ref{pinvariant-fragment}
we will provide a construction 
\[(\cdot)^{\bullet}: \begin{cases} \AutW(\ofoe) \to
\AutW(\ofo), \\
\AutWC(\olque) \to
\AutWC(\ofo), \\
\end{cases}\] 
such that for all $\bbA$ and $\bbT$: % we have
\begin{equation}
\label{eq:crux}
\bbA^{\bullet} \text{ accepts } \bbT \text{ iff } \bbA \text{ accepts
} \bbT^{\om},
\end{equation}
where $\bbT^{\om}$ is the $\om$-unravelling of $\bbT$.
 
 
 
 
%%%%% OLD

%In this section we are going to prove Theorem \ref{t:m1}.
%Our proof of the first item of the theorem crucially involves automata.
%In the previous section we saw that on trees, $\wmso$ effectively corresponds
%to the automata class $\AutWC(\olque)$.
% We will now relate this class to the one of
%parity automata based on $\ofo$ and satisfying similar weakness and continuity conditions.
%
%
%\begin{definition}
%A \emph{$\mucML$-automaton} $\aut = \tup{A,\Delta,\Omega,a_I}$ is an automaton $\aut \in \AutWC(\ofo)$ such that for all states $a,b \in A$ with $a \ord b$ and $b\ord a$ the following conditions hold:
%\begin{description}
%	\itemsep 0 pt
%	\item[(weakness)] $\pmap(a)=\pmap(b)$,
%	\item[(continuity)] if $\pmap(a)$ is odd (resp. even) then, for each $c\in C$ we have
%	   $\tmap(a,c) \in \cont{\ofo^+}{b}(A)$ (resp. $\tmap(a,c) \in \cocont{\ofo^+}{b}(A)$).
%\end{description}
%As the class of such automata coincides with $\AutWC(\ofo)$ we use the same name to denote it.
%\end{definition}
%
%As the key technical result of our paper, in subsection~\ref{pinvariant-fragment}
%we will provide a construction $(-)^{\bullet}: \AutWC(\olque) \to 
%\AutWC(\ofo)$, such that for all $\bbA$ and $\bbT$ we have
%\begin{equation}
%\label{eq:crux}
%\bbA^{\bullet} \text{ accepts } \bbT \text{ iff } \bbA \text{ accepts 
%} \bbT^{\om},
%\end{equation}
%where $\bbT^{\om}$ is the $\om$-unravelling of $\bbT$.
%As we shall see, the map $(-)^{\bullet}$ is completely determined at the 
%one-step level, that is, by some model-theoretic connection between 
%$\olque$ and $\ofo$.
%
%The second fact, to be discussed in
%subsection \ref{aut-to-formula}, is that for each $\mucML$-automaton $\bbA$  we can effectively construct an equivalent $\mucML$-formula $\xi_{\bbA}$.
%
%On the basis of the above observations we show that those results are enough to prove Theorem~\ref{t:m1}(i) as follows:
%
%\begin{proofof}{Theorem~\ref{t:m1}}
%\textbf{(1)} Given a \wmso-formula $\phi$, let $\phi^{\bullet} \df
%\xi_{\aut_{\phi}^{\bullet}}$.
%We verify that $\phi$ is bisimulation invariant iff $\phi$ and $\phi^{\bullet}$
%are equivalent.
%The direction from right to left is immediate by the observation that
%$\phi^{\bullet}$ is a formula in $\MC$.
%The opposite direction follows from the following chain of equivalences:
%\begin{align*}
%\model \models \phi
%  & \text{ iff } \bbT^{\om} \models \phi
%  & \tag{$\phi$ bisimulation invariant}
%\\ & \text{ iff }  \bbA_{\phi} \text{ accepts } \bbT^{\om}
%  & \tag{$ \phi \equiv \aut_{\phi}$ on trees}
%\\ & \text{ iff } \bbA_{\phi}^{\bullet} \text{ accepts } \bbT
%& \tag{\ref{eq:crux}}
%\\ & \text{ iff }  \bbT \models \xi_{ \aut_{\phi}^{\bullet}}
%& \tag{$\aut_{\phi}^{\bullet}\equiv \xi_{ \aut_{\phi}^{\bullet}}$}
%\end{align*}
%\textbf{(2)} For the second part of Theorem \ref{t:m1},
%We %reason as follows.
%%First, let's consider the following translation.
%first define, for every first-order variable $x$, a translation $ST_x$ from
%the $\mu$-calculus into the set of $\mlque$-formulas with only $x$ free:
%
%\begin{itemize}
%\itemsep 0 pt
%\item $ST_x(p)=p(x)$%\fcwarning{maybe $p(x)$, we use small predicates},
%\item $ST_x(\varphi \land \psi)=ST_x(\varphi) \land ST_x(\psi)$,
%\item $ST_x(\varphi \lor \psi)=ST_x(\varphi) \lor ST_x(\psi)$,
%\item $ST_x(\lnot \varphi)= \lnot ST_x(\varphi)$,
%\item $ST_x(\Diamond \varphi)=\exists y (R(x,y) \land ST_y(\varphi))$,
%\item $ST_x(\mu p. \varphi)= \mu p. ST_x(\varphi)$,
%\end{itemize}
%Clearly, every formula of the $\mucML$-fragment of the $\mu$-calclus is mapped to a logically equivalent formula of the $\mucML$-fragment of $\mlque$. Let
% $(-)_{\bullet}:\mucML\to\wmso$ defined as the composite $\mgFOETr{-} \circ ST_x$. By Theorem \ref{thm:guard_wmso}
%we obtain that $\psi \equiv \psi_{\bullet}$, for all $\psi \in
%\mucML$.
%\end{proofof}
%
%
%
