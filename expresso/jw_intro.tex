In this section we are going to prove Theorem \ref{t:m1}.
Our proof of the first item of the theorem crucially involves automata.
In the previous section we saw that on trees, $\wmso$ effectively corresponds
to the automata class $\AutWC(\olque)$.
 We will now relate this class to the one of
parity automata based on $\ofo$ and satisfying similar weakness and continuity conditions.


\begin{definition}
A \emph{$\mucML$-automaton} $\aut = \tup{A,\Delta,\Omega,a_I}$ is an automaton $\aut \in \AutWC(\ofo)$ such that for all states $a,b \in A$ with $a \ord b$ and $b\ord a$ the following conditions hold:
\begin{description}
	\itemsep 0 pt
	\item[(weakness)] $\pmap(a)=\pmap(b)$,
	\item[(continuity)] if $\pmap(a)$ is odd (resp. even) then, for each $c\in C$ we have
	   $\tmap(a,c) \in \cont{\ofo^+}{b}(A)$ (resp. $\tmap(a,c) \in \cocont{\ofo^+}{b}(A)$).
\end{description}
As the class of such automata coincides with $\AutWC(\ofo)$ we use the same name to denote it.
\end{definition}

As the key technical result of our paper, in subsection~\ref{pinvariant-fragment}
we will provide a construction $(-)^{\bullet}: \AutWC(\olque) \to 
\AutWC(\ofo)$, such that for all $\bbA$ and $\bbT$ we have
\begin{equation}
\label{eq:crux}
\bbA^{\bullet} \text{ accepts } \bbT \text{ iff } \bbA \text{ accepts 
} \bbT^{\om},
\end{equation}
where $\bbT^{\om}$ is the $\om$-unravelling of $\bbT$.
As we shall see, the map $(-)^{\bullet}$ is completely determined at the 
one-step level, that is, by some model-theoretic connection between 
$\olque$ and $\ofo$.

The second fact, to be discussed in
subsection \ref{aut-to-formula}, is that for each $\mucML$-automaton $\bbA$  we can effectively construct an equivalent $\mucML$-formula $\xi_{\bbA}$.

On the basis of the above observations we show that those results are enough to prove Theorem~\ref{t:m1}(i) as follows:

\begin{proofof}{Theorem~\ref{t:m1}}
\textbf{(1)} Given a \wmso-formula $\phi$, let $\phi^{\bullet} \df
\xi_{\aut_{\phi}^{\bullet}}$.
We verify that $\phi$ is bisimulation invariant iff $\phi$ and $\phi^{\bullet}$
are equivalent.
The direction from right to left is immediate by the observation that
$\phi^{\bullet}$ is a formula in $\MC$.
The opposite direction follows from the following chain of equivalences:
\begin{align*}
\model \models \phi
  & \text{ iff } \bbT^{\om} \models \phi
  & \tag{$\phi$ bisimulation invariant}
\\ & \text{ iff }  \bbA_{\phi} \text{ accepts } \bbT^{\om}
  & \tag{$ \phi \equiv \aut_{\phi}$ on trees}
\\ & \text{ iff } \bbA_{\phi}^{\bullet} \text{ accepts } \bbT
& \tag{\ref{eq:crux}}
\\ & \text{ iff }  \bbT \models \xi_{ \aut_{\phi}^{\bullet}}
& \tag{$\aut_{\phi}^{\bullet}\equiv \xi_{ \aut_{\phi}^{\bullet}}$}
\end{align*}
\textbf{(2)} For the second part of Theorem \ref{t:m1},
We %reason as follows.
%First, let's consider the following translation.
first define, for every first-order variable $x$, a translation $ST_x$ from
the $\mu$-calculus into the set of $\mlque$-formulas with only $x$ free:

\begin{itemize}
\itemsep 0 pt
\item $ST_x(p)=p(x)$%\fcwarning{maybe $p(x)$, we use small predicates},
\item $ST_x(\varphi \land \psi)=ST_x(\varphi) \land ST_x(\psi)$,
\item $ST_x(\varphi \lor \psi)=ST_x(\varphi) \lor ST_x(\psi)$,
\item $ST_x(\lnot \varphi)= \lnot ST_x(\varphi)$,
\item $ST_x(\Diamond \varphi)=\exists y (R(x,y) \land ST_y(\varphi))$,
\item $ST_x(\mu p. \varphi)= \mu p. ST_x(\varphi)$,
\end{itemize}
Clearly, every formula of the $\mucML$-fragment of the $\mu$-calclus is mapped to a logically equivalent formula of the $\mucML$-fragment of $\mlque$. Let
 $(-)_{\bullet}:\mucML\to\wmso$ defined as the composite $\mgFOETr{-} \circ ST_x$. By Theorem \ref{thm:guard_wmso}
we obtain that $\psi \equiv \psi_{\bullet}$, for all $\psi \in
\mucML$.
\end{proofof}


%In this section we are going to prove Theorem \ref{t:m1}.
%For the first point of the theorem, we reason as follows.
%From the previous section we know that on trees, for each $\wmso$-formula $\phi$ we can construct an equivalent \wmso-automata $\aut_{\phi}$. What is thence missing is an effective procedure that takes as input a \wmso-automaton $\aut$ whose corresponding regular tree language is closed under bisimulation, and gives as output an equivalent $\mucML$-formula $\xi_\aut$.
%
%Given a \wmso-formula $\phi(x)$, let $(\phi(x))^{\bullet} :=\xi_{\aut_{\phi}}$. We verify that $\phi$ is bisimulation invariant iff $\phi$ and $\phi^{\bullet}$ are equivalent. The direction from right to left being immediate, for the other direction we connect the following equivalences:
%\begin{alignat*}{2}
%		\model \models \phi(s_I)
%			& \quad\text{ iff }\quad \unravel{\model} \models \phi(s_I) & \qquad\quad \text{($\phi(x)$ is bisimulation invariant)} \\
%			& \quad\text{ iff }\quad  \unravel{\model} \in \trees( \aut_{\phi})& \qquad\quad \text{($\aut_{\phi}\equiv \phi$ on trees)} \\
%			& \quad\text{ iff }\quad  \unravel{\model} \models \xi_{ \aut_{\phi}}& \qquad\quad \text{($\aut_{\phi}\equiv \xi_{ \aut_{\phi}}$ on trees)}\\
%			& \quad\text{ iff }\quad  \model \models \xi_{ \aut_{\phi}}& \qquad\quad \text{($\xi_{ \aut_{\phi}}$ is bisimulation invariant)}
%	\end{alignat*}
%where $\unravel{\model}$ is the tree unraveling of $\model$ with respect to $s_I$.
%
%The construction of $\xi_\aut$ will be performed in two steps. For the first step we use the following results.
%
%\begin{theorem}[\cite{Venxx}]\label{t:venema}
%Let $\llang_1$ and $\llang_1'$ be fragments of $\olque$. If $\llang^+_1$ corresponds to the $P$-invariant fragment of ${\llang_1'}^+$, then
%\begin{enumerate}
%\itemsep 0 pt
%\item $\AutWC(\llang_1)$ corresponds to the bisimulation-invariant fragment of $\AutWC(\llang_1')$, and
%\item (ADD CONSTRAINTS ABOUT THE C-FRAGMENT) $\AutWC(\llang_1)$ corresponds to the bisimulation-invariant fragment of $\AutWC(\llang_1')$.\fcwarning{Define what bisim-inv means for automata?}
%\end{enumerate}
%\end{theorem}
%
%
%
%\afwarning{Is this really true?}\fcwarning{See email about remarks on the preservation of fragments and also Yde's mail}
%\begin{remark}
%To be precise, Theorem \ref{t:venema} in \cite{Venxx} is stated for $\foe_1$ and the correspondence given by point 1 only. However, the proof apply to the more general setting of the proposition above as well.
%\end{remark}
%In subsection \ref{pinvariant-fragment} we verify that the $P$-invariant fragment of ${\olque}^+$ is $\ofo^+$. By Theorem \ref{t:venema} we therefore obtain that $\AutWC(\ofo^+)$ corresponds to the bisimulation-invariant fragment of $\AutWC(\olque)$.
%
%The second step will then consists in showing that given an automaton in $\AutWC(\ofo)$ we can construct an equivalent  $\mucML$-formula. This is done in subsection \ref{aut-to-formula}.
%
%%\begin{enumerate}
%%\item we verify that at the level of one step logic,
%%\end{enumerate}
%
%%prove this already at one-step level: FO1C ? FOE�1 C/P ?? from this follows that wAut(FO1C) ? wAut(FOE�1 C)/?
%%wAut(FO1C) corresponds to fragment F
%%Rest is standard argument
%
%%The proof for this result consists in three steps
%%how the proof of (2) can be decom-
%%posed into three more or less independent parts:
%%1. a (non-trivial) result showing that both the modal ?-calculus and (on the class of tree models) monadic second-order logic can be characterized by certain automata,
%%2. a fairly simple model-theoretic characterisation result in monadic first-order logic, and
%%3. a general result on coalgebra automata.
%
%
%
%
%For the second point of Theorem \ref{t:m1}, in subsection \ref{sub:contoweak} we will  explicit the translation $(-)_{\bullet}:\mucML\to\wmso$ and verify that  $\psi \equiv \psi_{\bullet}$.
%
%
