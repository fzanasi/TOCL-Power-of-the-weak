In this section, we are going to prove the following:

\begin{theorem}\label{theo:ML2SO} Let either $L=\wmso$, $M=\mucML$ or  $L=\nmso$, $M=\AFMC$. Then there is an effective translation  $(-)_{\bullet}:M\to L$ such that
\[\psi \equiv \psi_{\bullet}, \text{ for every } \psi \in M.\]
\end{theorem}

\subsubsection{From $\mucML$ to $\wmso$} 
We first verify that Theorem \ref{theo:ML2SO} holds for $L=\wmso$ and $M=\mucML$.
For every first-order variable $x$, we define a translation $ST_x$ from
the $\mu$-calculus into the set of $\mlque$-formulas with only $x$ free extending in the obvious way the standard translation from modal logic into first-order logic:

\begin{itemize}
\itemsep 0 pt
\item $ST_x(p)=p(x)$%\fcwarning{maybe $p(x)$, we use small predicates},
\item $ST_x(\varphi \land \psi)=ST_x(\varphi) \land ST_x(\psi)$,
\item $ST_x(\varphi \lor \psi)=ST_x(\varphi) \lor ST_x(\psi)$,
\item $ST_x(\lnot \varphi)= \lnot ST_x(\varphi)$,
\item $ST_x(\Diamond \varphi)=\exists y (R(x,y) \land ST_y(\varphi))$,
\item $ST_x(\mu p. \varphi)= \mu p. ST_x(\varphi)$,
\end{itemize}
Clearly, every formula of the $\mucML$-fragment of the $\mu$-calculus is mapped to a logically equivalent formula of the $\mucML$-fragment of $\mlque$. Let
 $(-)_{\bullet}:\mucML\to\wmso$ defined as the composite $\mgFOETr{-} \circ ST_x$. By applying Theorem \ref{thm:guard_wmso} we conclude the proof for this case of the Theorem.



\subsubsection{From $\AFMC$ to $\nmso$}
The case of Theorem \ref{theo:ML2SO} with $L=\nmso$ and $M=\AFMC$ is, perhaps surprising, more subtle than the previous one. The key clause of the inductive translation, that is, the clause for the
least fixpoint operator, will be based on Proposition~\ref{p:afmc-rstGen} below concerning the least fixpoint fragment of the $\mu$-calculus, that is $\MC$-formulas in which only the least fixpoint operator occurs in them.
For its formulation we need the following definitions.

\begin{definition}
\label{d:rst}
Let $F: \wp(T)\to \wp(T)$ be a functional; for a given $S \subseteq T$ we define
the map $F_{\rst{S}}: \wp(T)\to \wp(T)$ by putting $F_{\rst{S}}(X) :=
FX \cap S$.
\end{definition}

\begin{definition} We let the language $\muMC$ on on $\props$ be defined by the following grammar. 
   \[
   \varphi ::= q \mid \neg q 
   \mid \varphi \lor \varphi \mid \varphi \land \varphi 
   \mid \Diamond\varphi \mid \Box\varphi 
   \mid \mu p. \varphi
   \]
   where $p,q \in \props$ and $p$ is positive in $\varphi$.
\end{definition}

We therefore can formulate the central proposition as follows.
\begin{proposition}
\label{p:afmc-rstGen}
Let $\phi = \phi(p)$ be a formula of $\muMC$
in which all occurrences of $p$ are 
% guarded and 
positive, and let $\model$ be a LTS.
Then for every $r \in T$ it holds that 
\begin{equation}
\label{eq:afmc-m}
r \in \mng{\mu p.\phi}^{\model} \text{ iff }
\text{ there is a noetherian set $X$ such that } r \in \LFP. (\phi^{\model}_{p})_{\rst{X}}.
\end{equation}
\end{proposition}

Before proving the above Proposition \ref{p:afmc-rstGen}, we show how, based on it, we can prove that Theorem \ref{theo:ML2SO} holds for $L=\nmso$ and $M=\AFMC$.

Firstly, Proposition \ref{p:afmc-rstGen} naturally suggests the following extension of the standard translation from modal logic into FO.
Let $\varphi \in \muMC$, and $x$ a first-order variable. We define the $\nmso$-formula $(\varphi)^t_x$ (with one free variable $x$) by induction on the complexity of $\varphi$ as follows:

\begin{itemize}
\item $(p)^t_x=p(x)$ and $(\lnot p)^t_x=\lnot p(x)$
%\item $(y)^t_x=x \in Y$
\item $(\varphi \lor \psi)^t_x=(\varphi)^t_x \lor (\psi)^t_x$ 
\item $(\varphi \land \psi)^t_x=(\varphi)^t_x \land (\psi)^t_x$ 
\item $(\square \varphi )^t_x= \forall z ( R(x,z) \to (\varphi)^t_z )$ 
\item $(\Diamond \varphi )^t_x= \exists z ( R(x,z) \land (\varphi)^t_z )$ 
\item $ (\mu p. \varphi)^t_x= \exists p'.(\forall  w \subseteq p'. w \in PRE((\varphi^{\model}_{p})_{\rst{p'}}) \to w(x))$,  where $w \in PRE((\varphi^{\model}_{p})_{\rst{p'}})$ expresses that $w$ is a prefixpoint of $(\phi^{\model}_{p})_{\rst{p'}}$, that is
$$w  \in PRE((\varphi^{\model}_{p})_{\rst{p'}}):= \forall y. (\varphi)^t_y \land p'(y) \to w(y),$$ with $w$ not occuring in $(\varphi)^t_y$.
\end{itemize}

Correctness of the translation is then an immediate corollary of Proposition \ref{p:afmc-rstGen}. 
\begin{proposition}\label{prop:trans} $\varphi \equiv (\varphi)^t_x$, for all $\varphi \in \muMC$.
%, for all LTS $\bbS$ and all states $s \in T$,
%\[ s \in \mng{\varphi}^{\bbS} \iff \bbS \models (\varphi)^t_x(s).\]
\end{proposition}
\begin{proof}
We verify that the translation is correct by induction on the structure of $\varphi$. The only critical case is the least fixpoint case. But this is justified by the fact that, from Proposition~\ref{p:afmc-rstGen}, the following equations hold:

\begin{align*}
s \in \mng{\mu p.\phi}^{\bbS} & \text{ iff } &
\text{ there is a noetherian set $P'$ such that } s \in \LFP. (\varphi^{\model}_{p})_{\rst{P'}} \\
& \text{ iff } &
\text{ there is a noetherian set $P'$ s.t. } s \in \bigcap \{ W \subseteq P'\mid W \in PRE((\varphi^{\model}_{p})_{\rst{P'}})\} \\
& \text{ iff } & \bbS \models \exists p'.(\forall w \subseteq p'. w \in PRE((\varphi^{\model}_{p})_{\rst{p'}}) \to w(s))
\end{align*}
This concludes the proof of the result.
\end{proof}

Secondly, the alternation free fragment enjoys the following substitution property, whose proof is immediate:
\begin{proposition}\label{p:subexmu}
Let $\mu p. \varphi \in \AFMC$, and $\nu q. \psi$ be a subformula of  $\mu p. \varphi$. Hence, if $r$ does not occurs in $\mu p. \varphi$ and $\mu p. \delta[r/ \nu q. \psi] = \mu p. \varphi$, then for every $\bbS$ it holds that
\[ \mng{\mu p.\delta}^{\bbS'} = \mng{\mu p.\varphi}^{\bbS}, \]
where $\bbS'=\bbS[r \mapsto \mng{\nu q.\psi}^{\bbS}]$.
\end{proposition}

Hence, finally, we obtain that, if for each first-order variable $x$, we extend  $(-)^t_x$ to $\AFMC$ by adding the clause $(\phi)_x^t= \lnot ( \lnot \phi)_x^t $ for $\phi=\nu q. \psi $, by Proposition \ref{p:subexmu} Proposition \ref{prop:trans} actually holds for for each $\varphi \in \AFMC$, meaning that the translation $(-)_{\bullet}:\AFMC\to\nmso$ defined as $(\phi)_{\bullet} = 
(\phi)_x^t $ satisfies the statement of  Theorem \ref{theo:ML2SO}  for $L=\nmso$ and $M=\AFMC$.



\begin{proofof}{Proposition~\ref{p:afmc-rstGen}}
We first %introduce some terminology and 
gather an easy observation concerning 
%paths and 
bundles.

%\begin{definition}
%Let $\bbS = (S,R,V)$ be a Kripke model.
%A \emph{finite path} through $\bbS$ is a non-empty sequence $(s_{i})_{i\leq n} = 
%s_{0}s_{1}\ldots s_{n}$ such that $Rs_{i}s_{i+1}$ for all $i$ with $0\leq i < n$.
%Given a path $\pi = (s_{i})_{i\leq n}$, we let $\first(\pi) = s_{0}$ and 
%$\last(\pi) = s_{n}$ denote the \emph{first} and \emph{last} state of $\pi$, 
%respectively;
%we let $\sz{\pi} = n$ denote its \emph{length}, and we let $\Ran(\pi) := 
%\{ s_{i} \mid i \leq n \}$ denote the set of states $\pi$ traverses. 
%\end{definition}
%
\begin{claimfirst}
\label{p:bundle}
Let $\model$ be a LTS. If $\mathcal{B}$ is a collection of $s$-bundles, 
then $\bigcup \mathcal{B}$ is also an $s$-bundle.
\end{claimfirst}

We now states some auxiliary results on monotone functionals.
The first of these takes care of the right-to-left direction of
\eqref{eq:afmc-m}, and it is proved by a routine transfinite induction argument:
\begin{claimfirst}
\label{p:afmc-1}
Let $F:  \wp(T)\to \wp(T)$ be monotone.
Then for every subset $S \subseteq T$ it holds that $\LFP. F_{\rst{S}}\subseteq \LFP.F$.
\end{claimfirst}

%\begin{proof}
%By a routine transfinite induction argument we may show that 
%$F_{\rst{T}}^{\al}(\nada) \sse F^{\al}(\nada) \cap T$, for all ordinals $\al$.
%For instance, in case $\al = \be+1$, we obtain 
%$F_{\rst{T}}^{\be+1}(\nada) = F_{\rst{T}}(F_{\rst{T}}^{\be}(\nada))
%= F(F_{\rst{T}}^{\be}(\nada) \cap T) \cap T \sse  F(F^{\be}(\nada))\cap T
%\sse F(F^{\be}(\nada)) = F^{\be+1}(\nada)$.
%% where the first inclusion is by the inductive hypothesis (and monotonicity of 
%% $F$).
%\end{proof}
%
For the other results we use a game-theoretic characterisation of the least 
fixpoint of such a monotone functional.

\begin{definition}
\label{d:unfgame}
Given a monotone functional $F: \wp(T)\to \wp(T)$ we define the 
\emph{unfolding game} $\UG_{F}$ as follows:
\begin{itemize}
\item at any position $s \in T$, $\eloise$ needs to pick a set $X$ such that 
$s \in FX$;
\item at any position $X \in \wp(T)$, $\abelard$ needs to pick an element of 
$X$
\item all infinite matches are won by $\abelard$.
\end{itemize}


A positional strategy $\ystrat: T \to \wp(T)$ for $\eloise$ in $\UG_{F}$ is 
\emph{descending} if 
\begin{equation}
\label{eq:dec}
s \in F^{\alpha+1}(\nada) \text{ implies } \ystrat(s) \sse F^{\alpha}(\nada).
\end{equation}
for all ordinals $\alpha$.
%% and all points $s \in F^{\al+1}(\nada)$.
\end{definition}

%
It is not the case that \emph{all} positional winning strategies for $\eloise$ in 
$\UG_{F}$ are descending, but the next result shows that there always is one.

\begin{claimfirst}
\label{p:unfg}
Let $F: \wp(T)\to \wp(T)$ be a monotone functional.
\begin{enumerate}
\item For all $s \in T$, $s \in \win_{\eloise}(\UG_{D})$ iff $s \in \LFP. F$.
\item if $s \in \LFP. F$, then \eloise has a descending winning strategy in $\UG_{D}@s$.
\end{enumerate}
\end{claimfirst}
%
\begin{pfclaim}
Point (i) corresponds to \cite[Theorem 3.14(2)]{Ven08}.
For part (ii) one can simply take the following strategy.
Given $s \in \LFP.F$, let $\alpha$ be the least ordinal such that $s \in 
F^{\alpha}(\nada)$; it is easy to see that $\alpha$ must be a successor ordinal,
say $\alpha = \beta + 1$. 
Now simply put $\ystrat(s) := F^{\beta}(\nada)$.
\end{pfclaim}

\begin{definition}
\label{d:str-tree}
Let $F: \wp(T)\to \wp(T)$ be a monotone functional, let $\ystrat$ be a 
positional winning strategy for $\eloise$ in $\UG_{F}$, and let $r \in T$. 
Define $T_{\ystrat,r} \sse T$ to be the set of states in $T$ that are 
$\ystrat$-reachable in $\UG_{F}@r$.
This set has a tree structure induced by the map $\ystrat$ itself, where the 
children of $t \in T_{\ystrat,r}$ are given by the set $\ystrat(t)$; we will
refer to $T_{\ystrat,r}$ as the \emph{strategy tree} of
$\ystrat$.
\end{definition}


Note that a strategy tree $T_{\ystrat,r}$ will have no infinite 
paths, since we define the notion only for a \emph{winning} strategy 
$\ystrat$.

\begin{claimfirst}
\label{p:afmc-2}
Let $F: \wp(T)\to \wp(T)$ be a functional, let $r \in T$, and let $\ystrat$ 
be a descending winning strategy for $\eloise$ in $\UG_{F}$.
Then
\begin{equation}
\label{eq:afmc3}
r \in \LFP. F \text{ iff } r \in \LFP. F_{\rst{T_{\ystrat,r}}}.
\end{equation}
\end{claimfirst}

\begin{pfclaim}
Let $F,r$ and $\ystrat$ be as in the formulation of the proposition.
The direction from right to left in \eqref{eq:afmc3} is immediate by
Claim~\ref{p:afmc-1}.

For the opposite direction, it clearly suffices to show that for all ordinals 
$\alpha$ we have
\begin{equation}
\label{eq:unf1}
F^{\al}(\nada) \cap T_{\ystrat,r} \sse F_{\rst{T_{\ystrat,r}}}^{\alpha}(\nada).
\end{equation}
We will prove \eqref{eq:unf1} by transfinite induction.
The base case, where $\alpha = 0$, and the inductive case where $\alpha$ is a limit
ordinal are straightforward, so we focus on the case where $\alpha$ is a 
successor ordinal, say $\alpha = \beta +1$.
Take an arbitrary state $u \in F^{\beta+1}(\nada) \cap T_{\ystrat,r}$, then we find 
$\ystrat(u) \sse F^{\beta}(\nada)$ by our assumption \eqref{eq:dec}, and 
$\ystrat(u) \sse T_{\ystrat,r}$ by definition of $T_{\ystrat,r}$.
Then the induction hypothesis yields that 
$\ystrat(u) \sse F_{\rst{T_{\ystrat,r}}}^{\beta}(\nada)$, and so we have 
$\ystrat(u) \sse F_{\rst{T_{\ystrat,r}}}^{\beta}(\nada) \cap T_{\ystrat,r}$.
But since $\ystrat$ is a positional winning strategy, and $u$ is a winning position
for $\eloise$ in $\UG_{F}$ by Claim~\ref{p:unfg}(i), $\ystrat(u)$ is a
legitimate move for $\eloise$, and so we have $u \in F(\ystrat(u))$.
Thus by monotonicity of $F$ we obtain $u \in 
F(F_{\rst{T}}^{\beta}(\nada) \cap T_{\ystrat,r})$, and since $u \in T_{\ystrat,r}$ by assumption, this
means that $u \in F_{\rst{T_{\ystrat,r}}}^{\beta+1}(\nada)$ as required.
\end{pfclaim}

We now turn to the specific case where $F$ is induced by some least fixpoint formula 
$\phi(p)$ on some LTS.


Let $\psi$ be a formula of $\muMC$, $\model$ a LTS and $\ystrat$ a winning strategy for $\eloise$ in $\egame(\psi,\model)@(\psi,s)$.  The arena of  the evaluation game $\egame(\psi,\model)@(\psi,s)$ can be seen as a tree. Hence, 
the strategy  $\ystrat$ can is a subtree of $\egame(\psi,\bbS)@(\psi,s)$ rooted in the initial position $(\psi,s)$. Let $B(\ystrat)$ be the projection on $\bbS$ of the set of all finite path of  $\ystrat$, seen as a tree, starting at the root. Finally define $T_{\ystrat,s}:=\bigcup \Ran[B(\ystrat)]$, that is as the union of all nodes traversed by some path in $B(\ystrat)$.
%
\begin{claimfirst}\label{p:strategybundledEv}
Let $\psi$ be a formula of $\muMC$, $\bbS$ a LTS and $\ystrat$ a winning strategy for $\eloise$ in $\egame(\psi,\bbS)@(\psi,s)$. Then: 
\begin{enumerate}
\item the strategy $\ystrat$, seen as a tree, is well-founded.
\item there exists an $s$-bundle $B$ such that $T_{\ystrat,s}$ is contained in $\bigcup \Ran[B]$.
\end{enumerate}
\end{claimfirst}
\begin{pfclaim}
For the first property, suppose that $\ystrat$ contains an infinite path. This yields a $\ystrat$-guided infinite match won by $\eloise$. But $\eloise$ can only win an infinite match if the maximal fixpoint variable appearing in $\phi$ is a $\nu$-variable, which is impossible because $\phi$ is in $\muMC$. %in which some fixpoint variable $x$ in $\psi$ is unfolded infinitely often. Because $\psi$ does not contain $\nu$-formulas, then $x$ must be guarded by a least fixpoint, implying that $\eloise$ loses the match. But this contradicts the assumption that $\ystrat$ is a winning strategy. 
%
For the second statement, observe that the projection $B(\ystrat)$ on $\bbS$ of $\ystrat$ itself is an $s$-bundle: if not, take the limit of the ascending chain of paths, and use K\"{o}nigs Lemma to show that there must be a $\ystrat$-guided infinite match through this path. The existence of such a match leads to a contradiction for the same reason as above.
\end{pfclaim}
%
%
%
%
%\begin{proposition}[Adequacy Theorem]\label{p:unfold=evalgame}
%Let $\phi = \phi(p)$ be a formula of $\muMC$ in which all occurrences of $x$ are positive, and let $(\bbS,s)$ be a pointed Kripke model. Then:
%\begin{equation}
%\label{eq:adeq3}
%s \in \mng{\mu p.\phi}^{\bbS} %\iff s \in \Win_{\eloi}(\UG(\phi_{x}^{\bbS}))
%\iff (\mu p.\phi,s) \in \Win_{\eloise}(\egame(\mu p.\phi,\bbS)).
%\end{equation}
%\end{proposition}
%\begin{proof} This holds more generally for any formula of $\muML$, see Yde's notes. \end{proof}
%
%
It actually turns out that:
\begin{claimfirst}\label{p:unfold=evalgame2}
Let $\phi = \phi(p)$ be a formula of $\muMC$ in which all occurrences of $x$ are positive, and let $(\bbS,s)$ be a pointed Kripke model. Then
$\exists$ has a winning strategy $\chi$ in $\UG(\phi_{x}^{\bbS})@(s, \phi(p))$ if and only she has winning strategy $\chi'$ in $\egame(\mu p.\phi,\bbS)@(s, \mu p.\phi)$. Moreover it holds that $T_{\chi,s} \subseteq T_{\chi',s}$.
\end{claimfirst}
%
We are now able to prove the following.

\begin{claimfirst}\label{p:strategybundledUnf},
Let $\phi = \phi(p)$ be a formula of $\muMC$
in which all occurrences of $p$ are 
% guarded and 
positive, and let $\bbS$ be some LTS, and write $F := 
\phi_{p}^{\bbS}$. Suppose that $s \in \mng{\mu p.\phi}^{\bbS}$. Then $\eloise$ has a descending winning strategy $\ystrat$ in $\UG(\phi_{p}^{\bbS})$ from $s$ and there exists an $s$-bundle $B$ such 
that 
% $\ystrat(s) = \Front(\ol{B}_{s})$ and 
$T_{\ystrat,s} \sse \bigcup \Ran[B]$.
\end{claimfirst}
\begin{pfclaim} 
If $s \in \mng{\mu p.\phi}^{\bbS}$ then Proposition~\ref{p:unfold=evalgame}  and Claim \ref{p:unfold=evalgame2} yield a winning strategy for $\eloise$ in $\UG(\phi_{p}^{\bbS})$ from $s$. By Claim~\ref{p:unfg}, we can assume this strategy to be descending, call it $\ystrat$. Then, another application of \ref{p:unfold=evalgame2} yields a winning strategy $\ystrat'$ for $\eloise$ in $\egame(\mu p.\phi,\bbS)$ from $(\mu p.\phi,s)$ such that $T_{\chi,s} \subseteq T_{\chi',s}$. By Claim~\ref{p:strategybundledEv}, $\ystrat'$ is such that $T_{\ystrat',s}$ is contained in $\bigcup \Ran[B]$ for some $s$-bundle $B$. Hence 
$T_{\ystrat,s} \sse \bigcup \Ran[B]$.
\end{pfclaim}


We can therefore conclude the proof of our main proposition.
Let $\phi(p)$, $\bbS$ and $F$ be as in the formulation of the proposition and suppose $r \in \mng{\mu p.\phi}^{\bbS}$. By Claim~\ref{p:strategybundledUnf}, $\eloise$ has a descending winning strategy $\ystrat$ in $\UG_{F}$ from $r$ and there is an $r$-bundle $B$ such 
that 
% $\ystrat(s) = \Front(\ol{B}_{s})$ and 
$T_{\ystrat,r} \sse \bigcup \Ran[B]$.
This suffices to show that $T_{\ystrat,r}$ is noetherian, and so the proposition 
follows by Claim~\ref{p:afmc-2}.
\end{proofof}

%%%%
%%%%
%%%%
%
%\section{Translation}
%An NMSO formula $\psi(x)$, with with one free variable $x$,  is equivalent to a $\mu$-sentence $\varphi$ if for all models $\bbS$ and all states $s \in S$,
%
%
%\[ s \in \mng{\varphi}^{\bbS} \iff \bbS \models \psi(s).\]
%

% 
% 
% 
%%%%%% OLD
%
%%In this section we are going to prove Theorem \ref{t:m1}.
%%Our proof of the first item of the theorem crucially involves automata.
%%In the previous section we saw that on trees, $\wmso$ effectively corresponds
%%to the automata class $\AutWC(\olque)$.
%% We will now relate this class to the one of
%%parity automata based on $\ofo$ and satisfying similar weakness and continuity conditions.
%%
%%
%%\begin{definition}
%%A \emph{$\mucML$-automaton} $\aut = \tup{A,\Delta,\Omega,a_I}$ is an automaton $\aut \in \AutWC(\ofo)$ such that for all states $a,b \in A$ with $a \ord b$ and $b\ord a$ the following conditions hold:
%%\begin{description}
%%	\itemsep 0 pt
%%	\item[(weakness)] $\pmap(a)=\pmap(b)$,
%%	\item[(continuity)] if $\pmap(a)$ is odd (resp. even) then, for each $c\in C$ we have
%%	   $\tmap(a,c) \in \cont{\ofo^+}{b}(A)$ (resp. $\tmap(a,c) \in \cocont{\ofo^+}{b}(A)$).
%%\end{description}
%%As the class of such automata coincides with $\AutWC(\ofo)$ we use the same name to denote it.
%%\end{definition}
%%
%%As the key technical result of our paper, in subsection~\ref{pinvariant-fragment}
%%we will provide a construction $(-)^{\bullet}: \AutWC(\olque) \to 
%%\AutWC(\ofo)$, such that for all $\bbA$ and $\bbT$ we have
%%\begin{equation}
%%\label{eq:crux}
%%\bbA^{\bullet} \text{ accepts } \bbT \text{ iff } \bbA \text{ accepts 
%%} \bbT^{\om},
%%\end{equation}
%%where $\bbT^{\om}$ is the $\om$-unravelling of $\bbT$.
%%As we shall see, the map $(-)^{\bullet}$ is completely determined at the 
%%one-step level, that is, by some model-theoretic connection between 
%%$\olque$ and $\ofo$.
%%
%%The second fact, to be discussed in
%%subsection \ref{aut-to-formula}, is that for each $\mucML$-automaton $\bbA$  we can effectively construct an equivalent $\mucML$-formula $\xi_{\bbA}$.
%%
%%On the basis of the above observations we show that those results are enough to prove Theorem~\ref{t:m1}(i) as follows:
%%
%%\begin{proofof}{Theorem~\ref{t:m1}}
%%\textbf{(1)} Given a \wmso-formula $\phi$, let $\phi^{\bullet} \df
%%\xi_{\aut_{\phi}^{\bullet}}$.
%%We verify that $\phi$ is bisimulation invariant iff $\phi$ and $\phi^{\bullet}$
%%are equivalent.
%%The direction from right to left is immediate by the observation that
%%$\phi^{\bullet}$ is a formula in $\muMC$.
%%The opposite direction follows from the following chain of equivalences:
%%\begin{align*}
%%\model \models \phi
%%  & \text{ iff } \bbT^{\om} \models \phi
%%  & \tag{$\phi$ bisimulation invariant}
%%\\ & \text{ iff }  \bbA_{\phi} \text{ accepts } \bbT^{\om}
%%  & \tag{$ \phi \equiv \aut_{\phi}$ on trees}
%%\\ & \text{ iff } \bbA_{\phi}^{\bullet} \text{ accepts } \bbT
%%& \tag{\ref{eq:crux}}
%%\\ & \text{ iff }  \bbT \models \xi_{ \aut_{\phi}^{\bullet}}
%%& \tag{$\aut_{\phi}^{\bullet}\equiv \xi_{ \aut_{\phi}^{\bullet}}$}
%%\end{align*}
%%\textbf{(2)} For the second part of Theorem \ref{t:m1},
%%We %reason as follows.
%%%First, let's consider the following translation.
%%first define, for every first-order variable $x$, a translation $ST_x$ from
%%the $\mu$-calculus into the set of $\mlque$-formulas with only $x$ free:
%%
%%\begin{itemize}
%%\itemsep 0 pt
%%\item $ST_x(p)=p(x)$%\fcwarning{maybe $p(x)$, we use small predicates},
%%\item $ST_x(\varphi \land \psi)=ST_x(\varphi) \land ST_x(\psi)$,
%%\item $ST_x(\varphi \lor \psi)=ST_x(\varphi) \lor ST_x(\psi)$,
%%\item $ST_x(\lnot \varphi)= \lnot ST_x(\varphi)$,
%%\item $ST_x(\Diamond \varphi)=\exists y (R(x,y) \land ST_y(\varphi))$,
%%\item $ST_x(\mu p. \varphi)= \mu p. ST_x(\varphi)$,
%%\end{itemize}
%%Clearly, every formula of the $\mucML$-fragment of the $\mu$-calclus is mapped to a logically equivalent formula of the $\mucML$-fragment of $\mlque$. Let
%% $(-)_{\bullet}:\mucML\to\wmso$ defined as the composite $\mgFOETr{-} \circ ST_x$. By Theorem \ref{thm:guard_wmso}
%%we obtain that $\psi \equiv \psi_{\bullet}$, for all $\psi \in
%%\mucML$.
%%\end{proofof}
%%
%%
%%
