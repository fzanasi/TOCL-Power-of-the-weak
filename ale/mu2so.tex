% \section{Fixpoint operators and second-order quantifiers}

In this section we will show how to translate some of the mu-calculi that we
encountered until now into the appropriate second-order logics.

\subsection{Translating $\mu$-calculi into second-order logics}

More specifically, our aim is to prove the following result.

\begin{theorem}
\label{t:mfl2mso}
(1) There is an effective translation $(\cdot)^{*}: \mu_{D}\ofoe \to \nmso$
such that $\phi \equiv \phi^{*}$ for every $\phi \in \mu_{D}\ofoe$; that is:
\[
\mu_{D}\ofoe \leq \nmso,
\]
and as a corollary of this we obtain that 
\[
\mu_{D}\ML \leq \nmso.
\]

(2) There is an effective translation $(\cdot)^{*}: \mu_{C}\ofoei \to \wmso$
such that $\phi \equiv \phi^{*}$ for every $\phi \in \mu_{C}\ofoei$; that is:
\[
\mu_{C}\ofoei \leq \wmso,
\]
and as a corollary of this we find that 
\[
\mu_{C}\ML \leq \wmso.
\]
\end{theorem}

Two immediate observations on this Theorem are in order.
First, note that we use the same notation $(\cdot)^{*}$ for both translations; 
this should not cause any confusion since the maps agree on formulas belonging 
to their common domain.
Consequently, in the remainder we will speak of a single translation 
$(\cdot)^{*}$.
Second, as the target language of the translation $(\cdot)^{*}$ we will take 
the \emph{two-sorted} version of second-order logic $"\mso$, as discussed in 
\ref{sec:prel-so}, and thus we will be implicitly using Fact \ref{fact:msovs2mso}.
% although in the Preliminaries we state that in this paper we will work with 
% one-sorted variants of second-order logics.
We reserve the fixed individual variable $v$ for this target language, i.e., 
every formula of the form $\phi^{*}$ will have this $v$ as its unique free 
variable; the equivalence $\phi \equiv \phi^{*}$ is to be understood accordingly.

The translation $(\cdot)^{*}$ will be defined by a straightforward induction on
the complexity of fixpoint formulas.
The two clauses of this definition that deserve some special attention are the
ones related to the fixpoint operators and the modalities.
We will discuss these cases here, in the mentioned order.

Concerning the fixpoint operators, it is important to realise that our clause
\emph{differs} from the one used in the standard inductive translation 
$(\cdot)^{s}$ of $\muML$ into ordinary $\mso$, where we would inductively
translate $(\mu p. \phi)^{*}$ as
\begin{equation}
\label{eq:st}
% \exists p\, \big(   p(x) \land \forall y\, (p(y) \to \phi^{s}_{y}) \big)
\forall p\, \big( \forall w\, (\phi^{*}[w/v] \to p(w)) \to p(v) \big).
\end{equation}
which states that $v$ belongs to any prefixpoint of $\phi$ with respect to $p$.
To understand the problem with this translation in the current context, suppose,
for instance, that we want to translate some continuous $\mu$-calculus into
$\wmso$.
Then the formula on the right hand side of \eqref{eq:st} will express that $v$ 
belongs to the intersection of all \emph{finite} prefixpoints of $\phi$, whereas
the least fixpoint is identical to the intersection of \emph{all} prefixpoints.
As a result, \eqref{eq:st} does not give the right translation for the formula 
$\mu p.\phi$ into \wmso.

To overcome this problem, we will prove that least fixpoints in restricted 
calculi like $\mu_{D}\ofoe$, $\mu_{C}\ofoei$ and many others, in fact satisfy a
rather special property.
We need the following definition to formulate this property.

\begin{definition}
\label{d:rst}
Let $F: \wp(S)\to \wp(S)$ be a functional; for a given $X \subseteq S$ we define
the map $F_{\rst{X}}: \wp(S)\to \wp(S)$ by putting $F_{\rst{S}}(X) := FX \cap S$.
\end{definition}

The observations formulated in the proposition below provide the crucial insight
underlying our embedding of various alternation-free and continuous 
$\mu$-calculus into, respectively, $\nmso$ and $\wmso$.

\begin{proposition}
\label{p:afmc-rstGen}
\label{p:keyfix}
Let $\model$ be an LTS, and let $r$ be a point in $\model$.

(1) For any formula $\phi$ with $\mu p. \phi \in \mu_{D}\ofoe$ we have
\begin{equation}
\label{eq:foe-d}
r \in \mng{\mu p.\phi}^{\model} \text{ iff there is a noetherian set $X$ such 
that } r \in \LFP. (\phi^{\model}_{p})_{\rst{X}}.
\end{equation}

(2) For any formula $\phi$ with $\mu p. \phi \in \mu_{C}\ofoei$ we have
\begin{equation}
\label{eq:foei-c}
r \in \mng{\mu p.\phi}^{\model} \text{ iff there is a finite set $X$ such 
that } r \in \LFP. (\phi^{\model}_{p})_{\rst{X}}.
\end{equation}
\end{proposition}

\begin{remark}
In fact, the statements in Proposition~\ref{p:keyfix} can be generalised to the
setting of a fixpoint logic $\mu\llang$ associated with an arbitrary one-step 
language $\llang$.
\end{remark}

The right-to-left direction of both \eqref{eq:foe-d} and \eqref{eq:foei-c} follow
from the following, more general, statement, which can be proved by a routine 
transfinite induction argument.

\begin{proposition}
% \label{p:afmc-1}
\label{p:rstfix}
Let $F:  \wp(S)\to \wp(S)$ be monotone.
Then for every subset $X \subseteq S$ it holds that $\LFP. F\rst{X}\subseteq 
\LFP.F$.
\end{proposition}

The left-to-right direction of \eqref{eq:foe-d} and \eqref{eq:foei-c} 
will be proved in the next two sections.
Since in the continuous case we can prove a slightly stronger result, applying
to \emph{arbitrary} continuous functionals, we discuss this case separately.

\newcommand{\PRE}{\mathit{PRE}}
The point of Proposition~\ref{p:keyfix} is that it naturally suggests the
following translation for the least fixpoint operator, as a subtle but important 
variation of \eqref{eq:st}:

\begin{equation}
\label{eq:trlmu}
(\mu p. \varphi)^{*} \df 
   \exists q\,\Big(\forall  p \sse q.\,
      \big(p \in \PRE((\varphi^{\model}_{p})_{\rst{q}}) \to p(v)\big)\Big),
\end{equation}
where $p \in \PRE((\varphi^{\model}_{p})_{\rst{q}})$ expresses that $p \sse q$ 
is a prefixpoint of the map $(\phi^{\model}_{p})_{\rst{q}}$, that is:
\[
p  \in \PRE((\varphi^{\model}_{p})_{\rst{q}}) \df
\forall w\, \Big(
( q(w) \land \varphi^{*}[w/v]) \to p(w)
\Big).
\]

Finally, before we can give the definition of the translation $(\cdot)^{*}$, we
briefly discuss the clause involving the modalities.
Here we need to understand the role of the \emph{one-step formulas} in the 
translation.
First an auxiliary definition.

\begin{definition}\label{def:exp}
Let $\model = \tup{T,R,\tscolors, s_I}$ be a $\wp{(\prop)}$-LTS, $A$ be a set of new variables, and $V: A \to
\wp(X)$ be a valuation on a subset $X\subseteq T$. The $\wp{(\prop\cup A)}$-LTS $\model^{V}:=\tup{T,R,\tscolors^V, s_I}$ given by defining the marking $\tscolors^V: T \to \wp{(\prop \cup A)}$ where
\[\tscolors^V(s):= 
\begin{cases} \tscolors(s) & \text{ if } s \notin X \\
\tscolors(s) \cup \{ a \in A \mid s \in V(a)\}& \text{ else,}
\end{cases}\]
is called the $V$-expansion of $\model$.
\end{definition}

The following proposition states that at the one-step level, the formulas that 
provide the semantics of the modalities of $\mu\ofoe$ and $\mu\ofoei$ can indeed 
be translated into, respectively $\nmso$ and $\wmso$.

\begin{proposition}
\label{p:1trl}
There is a translation $(\cdot)^{\dag}: \ofoei(A) \to \wmso$ such that for every 
model $\bbS$ and every valuation $V: A \to \wp(R[s_{I}])$:
\[
(R[s_{I}],V) \models \al \text{ iff } \model^{V} \models \al^{\dag}[s_{I}].
\]
Moreover, $(\cdot)^{\dag}$ restricts to first-order logic, i.e., $\al^{\dag}$ is
a first-order formula if $\al \in \ofoe$.
\end{proposition}

% \btbs
% \item
% Somehow clarify in the notation that $\al^{\dag}$ has exactly one free 
% individual variable, $v$.
% \etbs

\begin{proof}
Basically, the translation $(\cdot)^{\dag}$ \emph{restricts} all quantifiers
to the collection of successors of $v$.
In other words, $(\cdot)^{\dag}$ is the identity on basic formulas, it commutes
with the propositional connectives, and for the quantifiers $\exists$ and $\qu$
we define:
\[\begin{array}{lll}
(\exists x\, \al)^{\dag} &\df& \exists x\, (Rvx \land \al^{\dag})
\\ (\qu x\, \al)^{\dag}  &\df& \forall p \exists x\, (Rvx \land \neg p(x) 
    \land \al^{\dag})
\end{array}\]
We leave it for the reader to verify the correctness of this definition ---
observe that the clause for the infinity quantifier $\qu$ is based on the 
equivalence between $\wmso$ and $\foei$, established by 
V\"a\"an\"anen~\cite{vaananen77}.
\end{proof}

\noindent
We are now ready to define the translation used in the main result of this 
section.

\begin{definition}
By an induction on the complexity of formulas we define the following 
translation $(\cdot)^{*}$ from $\mu\foei$-formulas to formulas of monadic 
second-order logic:
\[\begin{array}{lll}
   p^{*} &\df& p(v)
\\ (\neg\phi)^{*}        &\df& \neg \phi^{*}
\\ (\phi\lor\psi)^{*}    &\df& \phi^{*} \lor \psi^{*}
\\ (\nxt{\al}(\ol{\phi}))^{*} &\df& \al^{\dag}[\phi_{i}^{*}/a_{i} \mid i \in I],
\end{array}\]
where $\al^{\dag}$ is as in Proposition~\ref{p:1trl}, and $[\phi_{i}^{*}/a_{i}
\mid i \in I]$ is the substitution that replaces every occurrence of an atomic
formula of the form $a_{i}(x)$ with the formula $\phi_{i}^{*}(x)$ (i.e. the 
formula $\phi_{i}^{*}$ with the free variable $v$ substituted by $x$).

Finally, the inductive clause for a formula of the form $\mu p.\phi$ is given
as in \eqref{eq:trlmu}.
\end{definition}

\begin{proofof}{Theorem~\ref{t:mfl2mso}}
%\btbs
%\item
%For both parts of the theorem we need to verify that 
%\\ (i) the translation $(\cdot)^{*}$ lands in the correct language, 
%\\ (ii) the translation $(\cdot)^{*}$ is truth preserving and
%\\ (iii) the statement about the modal language indeed follows from the 
%   main statement.
%\etbs
First of all, it is clear that the translation $(\cdot)^{*}$ lands in both cases in the correct language.
For both parts of the theorem, we thence prove that $(\cdot)^{*}$ is truth preserving 
by a straightforward formula induction.
E.g., for part (2) we need to show that, for an arbitrary formula $\phi\in
\mu_{C}\ofoei$ and an arbitrary model $\model$:
\begin{equation}
\label{eq:xxxx1}
\model \mmodels \phi \text{ iff } \model \models \phi^{*}[s_{I}].
\end{equation}

As discussed in the main text, the two critical cases concern the inductive 
steps for the modalities and the least fixpoint operators. Let $ \llang^{+} \in \{\ofoe,\ofoei\}$.
We start verifying the case of modalities. Hence, consider the formula $\nxt{\al}(\varphi_{1},\ldots,\varphi_{n})$ with $\al(a_{1},\ldots,a_{n}) \in \llang^{+}$. By induction hypothesis, $\phi_\ell \equiv \phi^{*}_\ell$, for $\ell=1, \dots, n$. Now, let $\model$ be a transition system. We have that
\begin{align*}
\model \mmodels \nxt{\al}(\varphi_{1},\ldots,\varphi_{n}) \text{ iff } & (R[s_{I}],V_{\overline{\varphi}}) \models \al(a_{1},\ldots,a_{n})  & \text{( by \eqref{eq:mumod})}
\\
\text{ iff } & \model^{V_{\overline{\varphi}}} \models \al^{\dag}[s_{I}] & \text{( by Prop. \ref{p:1trl})}
\\
\text{ iff } & \model \models \al^{\dag}[\phi_{i}^{*}/a_{i} \mid i \in I][s_{I}] & \text{( by \eqref{eq:valmod}, Def. \ref{def:exp} and IH)}
\end{align*}

The inductive step for the least fixpoint operator will be justified by 
Proposition~\ref{p:keyfix}.
In more detail, given a formula of the form $\mu x. \psi \in\mu_{Y} \llang^{+}$, with $Y=D$ for $\llang^{+} =\ofoe$, and $Y=C$ for $\llang^{+} =\ofoei$, 
consider the following chain of equivalences:
\begin{align*}
s_{I} \in \mng{\mu p.\psi}^{\bbS} \text{ iff } &
   \text{there is a }\begin{cases} \text{ finite} & \\ \text{ noetherian} & \end{cases} \text{ set $Q$ s.t. } 
   s_{I} \in \LFP. (\psi^{\model}_{p})_{\rst{Q}} 
   & \text{( by \eqref{eq:foe-d} and \eqref{eq:foei-c})}
\\ \text{ iff } &
\text{there is a }\begin{cases} \text{ finite} & \\ \text{ noetherian} & \end{cases} \text{ set $Q$ s.t. } 
   s_{I} \in \bigcap \Big\{ P \subseteq Q \mid P \in 
   \PRE((\psi^{\model}_{p})_{\rst{Q}})\Big\} 
\\ \text{ iff } & 
    \bbS \models \exists q.\, \Big(\forall p \subseteq q.\,
       \big(p \in \PRE((\psi^{\model}_{p})_{\rst{q}}) \to p(s_{I})\big)
       \Big)
   & (\text{IH})
\\ \text{ iff } & 
    \bbS \models (\mu p. \psi)^{*}[s_{I}].
\end{align*}
This concludes the proof of \eqref{eq:xxxx1}.

We finally verify that the statement about the modal languages follows from the main statement of the Theorem.
In order to do this, it is enough to check that for any set $Q$ of propositional variables, both $\noe{\mu\ML}{Q}\subseteq\noe{\mu\ofoei}{Q}$ and $\conoe{\mu\ML}{Q}\subseteq\conoe{\mu\ofoei}{Q}$, and $\cont{\mu\ML}{Q}\subseteq\cont{\mu\ofoei}{Q}$ and $\cocont{\mu\ML}{Q}\subseteq\cocont{\mu\ofoei}{Q}$ hold. But this follows by a straightforward induction on the structure of a formula by noticing that $\Diamond \varphi = \nxt{\al}(\varphi)$ for $\al(a):=\exists x.a(x) \in \ofoei \cap \ofoe$ and $\Box \varphi = \nxt{\al'}(\varphi)$ for $\al'(a):=\forall x.a(x) \in \ofoei \cap \ofoe$.
%This conclude the proof of the Theorem.
\end{proofof}

% \newpage
