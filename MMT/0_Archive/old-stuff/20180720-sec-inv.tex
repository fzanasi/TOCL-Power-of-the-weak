% !TEX root = ../main.tex

\section{Submodels and quotients}
\label{sec:inv}

There are various natural notions of morphism between monadic models; the one 
that we will be interested here is that  of a (strong) homomorphism.

\begin{definition}
\label{d:hom}
Let $\osmodel = (D,V)$ and $\osmodel' = (D',V')$ be two monadic models.
A map $f: D \to D'$ is a \emph{homomorphism} from $\osmodel$ to $\osmodel'$, 
notation: $f: \osmodel \to \osmodel'$, if we have $d \in V(a)$ iff $f(d) \in 
V'(a)$, for all $a \in A$ and $d \in D$.
\end{definition}

In this section we will be interested in the sentences of $\ofo, \ofoe$ and 
$\ofoei$ that are preserved under taking submodels and the ones that are 
invariant under quotients.

\begin{definition}
\label{d:inv}
Let $\osmodel = (D,V)$ and $\osmodel' = (D',V')$ be two monadic models.
We call $\osmodel$ a \emph{submodel} of $\osmodel'$ if $D \subseteq D'$ and 
the inclusion map $\iota_{DD'}: D \hookrightarrow D'$ is a homomorphism, and 
we say that $\osmodel'$ is a \emph{quotient} of $\osmodel$ if there 
is a surjective homomorphism $f: \osmodel \to \osmodel'$.

Now let $\phi$ be an $\llang$-sentence, where $\llang \in \{ \ofo, \ofoe, \ofoei
\}$.
\begin{enumerate}[(a)]
\item\label{d:inv:i} 
We say that $\phi$ is \emph{preserved under taking submodels} if 
$\osmodel \models \phi$ implies $\osmodel' \models \phi$, whenever
$\osmodel'$ is a submodel of $\osmodel$.
\item\label{d:inv:ii} Similarly, $\phi$ is \emph{invariant under taking quotients} if we have
$\osmodel \models \phi$ iff $\osmodel' \models \phi$, whenever $\osmodel'$ is
a quotient of $\osmodel$.
\end{enumerate}
\end{definition}

The first of these properties (preservation under taking submodels) is well
known from classical model theory --- it is for instance the topic of the
{\L}os-Tarski Theorem.
When it comes to quotients, in model theory one is usually more interested in
the formulas that are \emph{preserved} under surjective homomorphisms (and
the definition of homomorphism may also differ from ours): for instance, this 
is the property that is characterised by Lyndon's Theorem.
Our preference for the notion of \emph{invariance} under quotients stems from 
the fact that the property of invariance under quotients plays a key role in
characterising the \emph{bisimulation-invariant fragments} of various monadic
second-order logics, as is explained in our companion paper~\cite{companionWEAK}.


%%%%
%%%%
%%%%  LOS-TARSKI 
%%%%
%%%%

\subsection{Preservation under submodels}

In this subsection we characterise the fragments of our predicate logics 
consisting of the sentences that are preserved under taking submodels.
That is, the main result of this subsection is a {\L}os-Tarksi Theorem for
$\ofoei$.

\begin{definition}
The \emph{universal fragment} of the set $\ofoei(A)$ is the collection 
$\univ{\ofoei(A)}$ of formulas given by the following grammar:
\[
\varphi \defbnf
\top \mid \bot 
\mid a(x)
\mid \neg a(x)
\mid x \foeq y
\mid x \foneq y
% \mid \neg \varphi
\mid \varphi \lor \varphi
\mid \varphi \land \varphi
% \mid \exists x.\varphi
\mid \forall x.\varphi
% \mid \qu x.\varphi
\mid \dqu x.\varphi
\]
where $x,y\in \fovar$ and $a \in A$.
The universal fragment $\univ{\ofoe(A)}$ is obtained by deleting the clause for
$\dqu$ from this grammar, and we obtain the universal fragment $\univ{\ofo(A)}$ 
by further deleting both clauses involving the equality symbol.
\end{definition}

% \begin{theorem}
% \label{t:univ}
% There is an effective translation $(\cdot)^{\tuniv}: \ofoei(A) \to 
% \univ{\ofo(A)}$ such that, for every sentence $\phi \in \ofoe(A)$:
% 
% (1) $\phi \equiv \phi^{\tuniv}$ iff $\phi$ is preserved under taking submodels;
% 
% (2) $\phi^{\tuniv} \in \univ{\ofoe(A)}$ if $\phi \in \ofoe(A)$ and
% \todo % this may not even be true\ldots
%     $\phi^{\tuniv} \in \univ{\ofo(A)}$ if $\phi \in \ofo(A)$.
% \end{theorem}

\begin{theorem}
\label{t:univ}
Let $\phi$ be a sentence of the monadic logic $\llang(A)$, where $\llang \in 
\{ \ofo, \ofoe, \ofoei \}$.
Then $\phi$ is preserved under taking submodels if and only if there is a 
equivalent formula $\phi^{\tuniv} \in \univ{\llang(A)}$.
Furthermore, it is decidable whether a formula $\phi \in \llang(A)$ has this 
property or not.
\end{theorem}
\begin{proof}
As always, we start by verifying that the the easy side of the theorem.

\begin{claimfirst}
\label{c:univ1}
Let $\phi \in \ofoei(A)$ be a universal sentence. 
Then $\phi$ is preserved under taking submodels.
\end{claimfirst}

\begin{pfclaim} %{ \ref{c:univ1}.}
Let $(D',V')$ be a submodel of the monadic model $(D,V)$.
By induction on the complexity of a formula $\phi \in \univ{\ofoei(A)}$ we will 
show that for any assignment $g: \fovar \to D'$ we have
\[
(D,V),g' \models \phi \text{ implies } (D',V'),g \models \phi,
\]
where $g':= g \circ \iota_{D'D}$.
We will only consider the inductive step of the proof where $\phi$ is of the 
form $\dqu x. \psi$.
Define $X_{D,V} \isdef \{ d \in D \mid (D,V), g'[x \mapsto d] \models \psi \}$,
and similarly, $X_{D',V'} \isdef \{ d \in D' \mid (D',V'), g[x \mapsto d] \models 
\psi \}$.
By the inductive hypothesis we have that $X_{D,V} \cap D' \subseteq X_{D',V'}$,
implying that $D' \setminus X_{D',V'} \subseteq D \setminus X_{D,V}$.
But from this we immediately obtain that 
\[
|D \setminus X_{D,V}| < \omega
\text{ implies } |D' \setminus X_{D',V'}| < \omega,
\]
which means that $(D,V),g' \models \phi$ implies $ (D',V'),g \models \phi$, as 
required.
\end{pfclaim}

For the `hard' side of the theorem we need the following result.

\begin{claim} 
\label{c:univ2}
For any monadic logic $\llang \in \{ \ofo, \ofoe, \ofoei \}$ there is an effective 
translation $(-)^\tuniv: \llang(A) \to \univ{\llang(A)}$ such that a
formula $\phi \in \llang(A)$ is preserved under taking submodels if and only if
$\phi\equiv \phi^\tuniv$.
\end{claim}

\begin{pfclaim} %{ \ref{c:univ2}.}
We only consider the case where $\llang = \ofoei$, leaving the other cases to 
the reader.
For simple basic formulas, the translation $(-)^\tuniv$ is given as follows:
\[
(\dbnfofoei{\vlist{T}}{\Pi}{\Sigma})^{\tuniv} \isdef 
   \forall z \bigvee_{S \in \vlist{T}\cup \Pi\cup\Sigma} \tau_{S}(z) 
   \land \dqu z \bigvee_{S \in \Sigma} \tau_{S}(z).
\]
\btbs
\item comment on this translation
\etbs
To extend this definition to an arbitrary $\ofoei$-sentence $\phi$, first use Theorem~\ref{thm:bfofoei} to rewrite $\phi$ into an equivalent
basic form $\bigvee \dbnfofoei{\vlist{T}}{\Pi}{\Sigma}$. By Proposition~\ref{prop:bfofoei-sigmapi} we can assume that each
$\Sigma \subseteq \Pi \subseteq \vlist{T}$. Then define 
$\phi^{\tuniv}\isdef \bigvee \dbnfofoei{\vlist{T}}{\Pi}{\Sigma}^{\tuniv}$.

It is easy to see that $\phi^{\tuniv} \in \univ{\ofoei(A)}$, for every sentence
$\phi \in \ofoei(A)$; but then it is immediate by Claim~\ref{c:univ1} that 
$\phi$ is preserved under taking submodels if $\phi \equiv \phi^{\tuniv}$.

For the left-to-right direction, assume that $\phi$ is preserved 
under taking submodels.
It is easy to see that $\phi$ implies $\phi^{\tuniv}$, so we focus on proving
the opposite.
That is, we suppose that $(D,V) \models \phi^{\tuniv}$, and aim to show that 
$(D,V) \models \phi$.
 
By Theorem \ref{thm:bfofoei} and Proposition~\ref{prop:bfofoei-sigmapi} we may assume without loss of 
generality that $\phi$ is a disjunction of formulas of the form 
$\dbnfofoei{\vlist{T}}{\Pi}{\Sigma}$, where $\Sigma \subseteq \Pi \subseteq 
\vlist{T}$.
It follows that $(D,V)$ satisfies some  disjunct 
$   \forall z \bigvee_{S \in \vlist{T}\cup \Pi\cup\Sigma} \tau_{S}(z) 
   \land \dqu z \bigvee_{S \in \Sigma} \tau_{S}(z)$
of $\Big(\dbnfofoei{\vlist{T}}{\Pi}{\Sigma}\Big)^{\tuniv}$.
Expand $D$ with finitely many elements $\vlist{d}$, in one-one correspondence 
with $\vlist{T}$, and ensure that the type of each $d_{i}$ is $T_{i}$.
In addition, add, for each $S \in \Sigma$, infinitely many elements 
$\{ e^{S}_{n} \mid n \in \omega\}$, each of type $S$.
Call the resulting monadic model $\osmodel' = (D',V')$.

This construction is tailored to ensure that 
$(D',V') \models \dbnfofoei{\vlist{T}}{\Pi}{\Sigma}$, and so we obtain $(D',V') 
\models \phi$.
But obviously, $\osmodel$ is a submodel of $\osmodel'$, whence $(D,V) \models 
\phi$ by our assumption on $\phi$.
\end{pfclaim}

The  first part of  the theorem is thus an immediate consequence of 
Claims~\ref{c:univ1} and~\ref{c:univ2}. By applying Fact \ref{f:decido} to Claim~\ref{c:univ2} we finally obtain that for the three concerned formalisms the problem of deciding whether a formula is preserved under taking submodels is decidable.
\end{proof}

\clearpage


%%%%
%%%%
%%%%  QUOTIENTS 
%%%%
%%%%

\subsection{Invariance under quotients}

The following theorem states that monadic first-order logic \emph{without}
equality ($\ofo$) provides the quotient-invariant fragment of both monadic 
first-order logic with equality ($\ofoe$), and of infinite-monadic predicate 
logic ($\ofoei$).

\begin{theorem}
\label{t:qinv}
Let $\phi$ be a sentence of the monadic logic $\llang(A)$, where $\llang \in 
\{ \ofoe, \ofoei \}$.
Then $\phi$ is invariant under taking quotients if and only if there is a 
equivalent sentence
% $\phi^{\tmod}$ 
in $\ofo$.
Furthermore, it is decidable whether a formula $\phi \in \llang(A)$ has this 
property or not.
\end{theorem}

We first state the `easy' part of the first claim of the theorem. 
Note that in fact, we have already been using this observation in earlier parts
of the paper.

\begin{proposition}
\label{p:m-qinv}
Every sentence in $\ofo$ is invariant under taking quotients.
\end{proposition}

\begin{proof}
Let $f: D \to D'$ provide a surjective homomorphism between the models $(D,V)$ 
and $(D',V')$, and 
observe that for any assignment $g: \fovar \to D$ on $D$, the composition $f 
\circ g: \fovar \to D'$ is an assignment on $D'$.

In order to prove the proposition one may show that, for an arbitrary 
$\ofo$-formula $\phi$ and an arbitrary assignment $g: \fovar \to D$, we have
\begin{equation}
\label{eq:m-qinv1}
(D,V),g \models \phi \text{ iff } (D',V'), f \circ g \models \phi.
\end{equation}
We leave the proof of \eqref{eq:m-qinv1}, which proceeds by a straightforward 
induction on the complexity of $\phi$, as an exercise to the reader.
\end{proof}

To prove the remaining part of Theorem~\ref{t:qinv}, we start with providing 
translations from respectively $\ofoe$ and $\ofoei$ to $\ofo$.

\begin{definition}
\label{d:7-21}
For $\ofoe$-sentences in basic form we first define
\[
   \Big( \dbnfofoe{\vlist{T}}{\Pi} \Big)^{\tmoda} 
\isdef \bigwedge_{i} \exists x_i. \tau_{T_i}(x_i) \land 
\forall x. \bigvee_{S\in\Pi} \tau_S(x),
\]
and then set $(\bigvee_{i} \alpha_{i})^{\tmoda} \isdef \bigvee 
\alpha_{i}^{\tmoda}$.
This translation is extended to the collection of all $\ofoe$-sentences as expected by 
defining $\phi^{\tmoda} \isdef (\tbas{\phi})^{\tmoda}$, where $\tbas{\phi}$
is the basic normal form of $\phi$ as given by Theorem~\ref{thm:bnfofoe}.

For $\ofoei$-sentences in basic form we start with defining
\[
\Big( \dbnfofoei{\vlist{T}}{\Pi}{\Sigma} \Big)^{\tmodb} 
\isdef \bigwedge_{i} \exists x_i. \tau_{T_i}(x_i) \land 
\forall x. \bigvee_{S\in\Sigma} \tau_S(x).
\]
The definition is thus extended to the full language $\ofoei$ in the usual way: %given  by using Proposition~\ref{prop:bfofoei-sigmapi}.
%To extend this definition to 
given a $\ofoei$-sentence $\phi$, by Theorem~\ref{thm:bfofoei} and
Proposition~\ref{prop:bfofoei-sigmapi} we compute an equivalent
basic form $\bigvee \dbnfofoei{\vlist{T}}{\Pi}{\Sigma}$, with 
$\Sigma \subseteq \Pi \subseteq \vlist{T}$, and finally we set
$\phi^{\tmodb}\isdef \bigvee \dbnfofoei{\vlist{T}}{\Pi}{\Sigma}^{\tmodb}$.
\end{definition}

Note that the two translations may give \emph{different} translations for 
$\ofoe$-sentences.
Also observe that the $\Pi$ `disappears' in the translation of the formula
$\dbnfofoei{\vlist{T}}{\Pi}{\Sigma}$.

The key property of these translations is the following.

\begin{proposition}
\label{p-1P}
\begin{enumerate}
\item
For every one-step model $(D,V)$ and every $\phi \in \ofoe(A)$ we have
\begin{equation}
\label{eq-0cr}
(D,V) \models \phi^{\tmoda} \text{ iff } (D\times \omega,V_\pi) \models \phi.
\end{equation}
\item
For every one-step model $(D,V)$ and every $\phi \in \ofoei(A)$ we have
\begin{equation}
\label{eq-1cr}
(D,V) \models \phi^{\tmodb} \text{ iff } (D\times \omega,V_\pi) \models \phi.
\end{equation}
\end{enumerate}
\noindent
Here $V_{\pi}$ % =  f^{-1} \circ V$
 is the induced valuation given by 
$V_{\pi}(a) \isdef \{ (d,k) \mid d \in V(a), k\in\omega\}$.
\end{proposition}

\begin{proof}
We only prove the claim for $\ofoei$ (i.e., the second part of the proposition),
the case for $\ofoe$ being similar.
Clearly it suffices to prove \eqref{eq-1cr} for formulas of the form
$\alpha = \dbnfofoei{\vlist{T}}{\Pi}{\Sigma}$.
\smallskip

\noindent\fbox{$\Rightarrow$} 
Assume $(D, V) \models \alpha^{\tmodb}$, we will show that 
$(D\times \omega, V_\pi) \models \dbnfofoei{\vlist{T}}{\Pi}{\Sigma}$.
Let $d_i$ be such that $V^{\flat}(d_i) = T_{i}$ in $(D, V)$. 
It is clear that the $(d_i,i)$ provide \emph{distinct} elements, with each 
$(d_i,i)$ satisfying $\tau_{T_i}$ in $(D\times\omega, V_{\pi})$ and therefore 
the first-order existential part of $\alpha$ is satisfied. 
With a similar argument it is straightforward to verify that the $\qu$-part of 
$\alpha$ is also satisfied --- here we critically use the observation that
$\Sigma \subseteq \vlist{T}$, so that every type in $\Sigma$ is witnessed in 
the model $(D,V)$, and hence witnessed infinitely many times in $(D\times\omega,
V_\pi)$.

For the universal parts of $\dbnfofoei{\vlist{T}}{\Pi}{\Sigma}$ it is enough
to observe that, because of the universal part of $\alpha^\tmodb$, \emph{every}
$d\in D$ realizes a type in $\Sigma$. 
By construction, the same applies to $(D\times\omega, V_{\pi})$, 
therefore this takes care of both universal quantifiers.
\medskip
		
\noindent\fbox{$\Leftarrow$} 
Assuming that $(D\times \omega, V_\pi) \models 
\dbnfofoei{\vlist{T}}{\Pi}{\Sigma}$,
we will show that $(D, V) \models \alpha^\tmodb$. 
The existential part of $\alpha^{\tmodb}$ is trivial. 
For the universal part we have to show that every element of $D$ realizes a 
type in $\Sigma$. 
Suppose not, and let $d\in D$ be such that $\lnot\tau_S(d)$ for all $S\in 
\Sigma$. 
Then we have $(D\times\omega, V_\pi) \not\models \tau_S(d,k)$ for all $k$.
That is, there are infinitely many elements not realising any type in $\Sigma$. 
Hence we have $(D\times\omega, V_\pi) \not\models \dqu y.\bigvee_{S\in\Sigma} 
\tau_S(y)$. 
Absurd, because this formula is a conjunct of 
$\dbnfofoei{\vlist{T}}{\Pi}{\Sigma}$.
\end{proof}

\noindent
We will now show how the theorem follows from this.

\begin{proofof}{Theorem~\ref{t:qinv}}
Let $\phi$ be a sentence of $\ofoei$ (we only cover the case of $\llang = 
\ofoei$, the case for $\llang = \ofoe$ is similar).
We will show that 
\begin{equation}
\label{eq:m-qinv2}
\phi \equiv \phi^{\tmodb} \text{ iff } \text{$\phi$ is invariant under 
taking quotients}.
\end{equation}
The direction from right to left is immediate by Proposition~\ref{p:m-qinv}.
For the other direction it suffices to observe that any model $(D,V)$ is a 
quotient of its `$\omega$-product' $(D\times \omega, V_\pi)$, and to 
reason as follows:
\begin{align*}
(D,V) \models \phi 
   & \text{ iff } (D\times \omega, V_\pi) \models \phi
   & \text{(assumption on $\phi$)}
\\ & \text{ iff } (D,V) \models \phi^{\tmodb}
   & \text{(Proposition~\ref{p-1P})}
\end{align*}

Finally, it is immediate by the effectiveness of translation  $(\cdot)^{\tmodb}$, decidability of $\ofoei$ (Fact \ref{f:decido}) and \eqref{eq:m-qinv2}
that it is decidable whether a given $\ofoei$-sentence $\phi$ is invariant under 
taking quotients or not.
\end{proofof}


In our companion paper \cite{companionWEAK} on automata, we need versions of these results for 
the monotone and the continuous fragment.
For this purpose we define some slight modifications of the translations 
$(\cdot)^{\tmoda}$ and $(\cdot)^{\tmodb}$ which map positive and syntactically 
continuous formulas to respectively positive and syntactically continuous 
formulas.

\begin{theorem}
\label{t:inv1}
There are effective translations $(\cdot)^{\tmoda}: \ofoe^{+}(A) \to 
\ofo^{+}(A)$ and $(\cdot)^{\tmodb}: {\ofoei}^{+}(A) \to \ofo^{+}(A)$ such that 
$\phi \equiv \phi^{\tmoda}$ (respectively, $\phi \equiv \phi^{\tmodb}$) iff 
$\phi$ is invariant under quotients.
Moreover, we may assume that 
% $(\cdot)^{\tmodb}$ restricts to a map
$(\cdot)^{\tmodb}: \cont{{\ofoei}^{+}(A)}{B} \to \cont{\ofo^{+}(A)}{B}$,
for any $B \subseteq A$.
\end{theorem}
\begin{proof}
%\begin{definition}
We define translations $(\cdot)^{\tmoda}: {\ofoe}^{+}(A) \to {\ofo}^{+}(A)$ and 
$(\cdot)^{\tmodb}: {\ofoei}^{+}(A) \to {\ofo}^{+}(A)$ as follows. 
For $\ofoe^{+},{\ofoei}^{+}$-sentences in simple basic form we define
\[\begin{array}{lll}
     \Big( \posdbnfofoe{\vlist{T}}{\Pi} \Big)^{\tmoda} 
   & \isdef 
   & \bigwedge_{i} \exists x_i. \tau^+_{T_i}(x_i) 
     \land \forall x. \bigvee_{S\in\Pi} \tau^+_S(x),
\\   \Big( \posdbnfofoei{\vlist{T}}{\Pi}{\Sigma} \Big)^{\tmodb} 
   & \isdef 
   & \bigwedge_{i} \exists x_i. \tau^+_{T_i}(x_i) 
     \land \forall x. \bigvee_{S\in\Sigma} \tau^+_S(x),
\end{array}\]
and then we use, respectively, the Corollaries~\ref{cor:ofoepositivenf} and
\ref{cor:ofoeipositivenf} to extend these translations to the full positive
fragments $\ofoe^{+}$ and ${\ofoei}^{+}$, as we did in Definition~\ref{d:7-21}
for the full language.
%\end{definition}

%\begin{proofof}{Theorem~\ref{t:inv1}}
We leave it as an exercise for the reader to prove the analogue of 
Proposition~\ref{p-1P} for these translations, and to show how the first
statements of the theorem
%Theorem~\ref{t:inv1} 
follows from this.

Finally, to see why we may assume that $(\cdot)^{\tmodb}$ restricts to a map 
from the syntactically $B$-continuous fragment of $\ofoei(A)$ to the 
syntactically $B$-continuous fragment of $\ofo(A)$, assume that $\phi 
\in \ofoei(A)$ is continuous in $B \subseteq A$.
By Corollary~\ref{cor:ofoeicontinuousnf} we may assume that $\phi$ is a
disjunction of formulas of the form $\posdbnfofoei{\vlist{T}}{\Pi}{\Sigma}$,
where $B \cap \bigcup \Sigma = \nada$.
This implies that in the formula $\phi^{\tmodb}$ no predicate symbol $b \in
B$ occurs in the scope of a universal quantifier, and so $\phi^{\tmodb}$
is syntactically continuous in $B$ indeed.
%\end{proofof}
\end{proof}