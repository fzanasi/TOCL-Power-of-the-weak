% !TEX root = ../main.tex

We recall the definition of a parity automaton, adapted to our setting.
As we will be running parity automata over transition systems with many
relations, we will need to use multi-sorted one-step models. Intuitively,
each sort corresponds to one of the relations of the transition system.
Since we will be comparing parity automata defined in terms of various
one-step languages, it makes sense to make the following abstraction.


\begin{definition}
Given a finite set $A$ and sorts $\sorts = \{\asort_1,\dots,\asort_n\}$,
we define a \emph{one-step model} to be a tuple $\osmodel = (D_{\asort_1},\dots,D_{\asort_n},\val)$
consisting of sets $D_{\asort_1},\dots,D_{\asort_n}$ 
and a valuation $\val: A \to \wp (\bigcup_\asort D_\asort)$.
We use $D$ to denote the set $\bigcup_\asort D_\asort$ which we call the domain of $\osmodel$.
A one-step model is called \emph{strict} when the sets $D_{\asort\in\sorts}$ are pairwise disjoint, that is, when $D_{\asort_1},\dots,D_{\asort_n}$ provide a partition of $D$.
Depending on context, elements of $A$ will be called \emph{monadic predicates}, \emph{names}
or \emph{propositional variables}. When the sets $D_{\asort\in\sorts}$ are not relevant we will just write the one-step model as $(D,\val)$. The class of all one-step models will be denoted by $\umods$ and the class of all strict one-step models will be denoted by $\sumods$.


A \emph{(multi-sorted) one-step language} is a map $\llang$ assigning to each finite set $A$ and sorts $\sorts$, a set $\llang(A,\sorts)$ of objects called \emph{one-step formulas} over $A$ (on sorts $\sorts$). When the sorts are understood from context (or fixed) we simply write $\llang(A)$ instead of $\llang(A,\sorts)$.
We require that $\llang(\bigcap_{i} A_{i},\sorts) = \bigcap_{i} \llang(A_{i},\sorts)$,
so that for each $\varphi \in \llang(A,\sorts)$ there is a smallest $A_{\varphi} \subseteq A$ such
that $\varphi \in \llang(A_{\varphi},\sorts)$; this $A_{\varphi}$ is the set of names that \emph{occur} in $\varphi$.

We assume that one-step languages come with a \emph{truth}
relation: given a one-step model $\osmodel$, a formula $\varphi \in \llang$
is either \emph{true} or \emph{false} in $\osmodel$, denoted by,
respectively, $\osmodel \models \varphi$ and $\osmodel \not\models \varphi$.
%

We also assume that $\llang$ has a \emph{positive fragment} $\llang^+$
characterizing monotonicity. We say that a formula $\varphi \in \llang(A,\sorts)$ is
monotone in $a\in A$ iff $(\osmoddom,\val) \models \varphi$ implies $(\osmoddom,\val[a\mapsto E]) \models \varphi$ whenever $\val(a) \subseteq E$. Hence, we require that $\varphi \in \llang(A,\sorts)$ is
monotone in all $a\in A$ iff it is equivalent to a formula $\varphi' \in \llang^+(A,\sorts)$.
\end{definition}


Observe that every valuation $\val: A \to \wp (D)$ can equivalently be seen as a marking (or coloring) $\val^\natural:D \to \wp(A)$ given by $\val^\natural(d) := \{a \in A \mid d \in \val(a)\}$ and as a relation $Z_\val := \{ (a,d) \mid d\in \val(a)\}$.
We will use these perspectives interchangeably.


A \emph{parity automaton} based on the one-step language $\llang$, actions $\acts$ and 
alphabet $\wp(\props)$ is a tuple $\aut = \tup{A,\tmap,\pmap,a_I}$ such that $A$ is a
finite set of states of the automaton, $a_I \in A$ is the initial state,
$\tmap: A\times \wp(\props) \to \llang^+(A,\acts)$
is the transition map, and $\pmap: A \to \nat$ is the parity map.
The collection of such automata will be denoted by $\Aut(\llang,\props,\acts)$.
For the rest of the manuscript we fix the set of actions $\acts$ and omit it in our notation;
we also omit the set $\props$ when clear from context or irrelevant.
\end{definition}

Acceptance and rejection of a transition system by an automaton is defined
in terms of the following parity game.


\begin{definition}\label{def:agames}
Given an automaton $\aut = \tup{A,\tmap,\pmap,a_I}$ in $\Aut(\llang,\props)$ and a $\props$-transition
system $\model = \tup{\moddom,R_{\aact\in\acts},\tscolors,s_I}$, the \emph{acceptance game}
$\agame(\aut,\model)$ of $\aut$ on $\model$ is the parity game defined
according to the rules of the following table.
%
% \begin{table*}[ht]
%   \centering
\begin{center}
\small
\begin{tabular}{|l|c|l|c|} \hline
Position & Pl'r & Admissible moves & Parity \\
\hline
    $(a,s) \in A \times \moddom$
  & $\eloise$
  & $\{\val : A \to \wp(R[s]) \mid (R[s],\val) \models \tmap (a, \tscolors(s)) \}$
  & $\pmap(a)$ 
\\
%     $(a,s) \in A \times \moddom$
%   & $\abelard$
%   & $\{(a,s,\aact) \in A \times \moddom  \mid \aact \in \acts\}$
%   & $\pmap(a)$ 
% \\
%     $(a,s,\aact) \in A \times \moddom $
%   & $\eloise$
%   & $\{\val : A \to \wp(R_\aact[s]) \mid (R_\aact[s],\val) \models \tmap (a, \tscolors(s), \aact) \}$
%   & $\pmap(a)$ 
% \\
    $\val : A \rightarrow \wp(\moddom)$
  & $\abelard$
  & $\{(b,t) \mid t \in \val(b)\}$
  & $\max(\pmap[A])$
\\ \hline
 \end{tabular}
\end{center}
%
In this case $(R[s],\val)$ denotes $(R_{\aact_1}[s],\dots,R_{\aact_n}[s],\val)$. 
A transition system $\model$ is \emph{accepted} by $\aut$ if $\exists$ has
a winning strategy in $\agame(\aut,\model)@(a_I,s_I)$, and \emph{rejected}
if $(a_I,s_I)$ is a winning position for $\abelard$.
\end{definition}


Given an automaton $\aut$ and a transition system $\model$ we write $\model \mmodels \aut$ and $\model \models \aut$ when $\aut$ accepts $\model$. The former notation is used when the one-step language of the automaton is modal, and the latter notation is used when the one-step language is first-order (more on this will be discussed in Chapter~\ref{chap:sub}). Given a state $a \in A$ we use ``$\aut,a$'' to denote the automaton which is like $\aut$ but where the initial state is now $a$.


\begin{definition}\label{def:autgraph}
Observe that given a parity automaton $\aut$ we can induce a graph on $A$ by setting a transition from $a$ to $b$ (notation: $a \leadsto b$) if $b$ occurs in $\tmap(a,c)$ for some $c \in \wp(\props)$.
We let the \emph{reachability} relation $\ordeq$ denote the reflexive-transitive
closure of the relation $\leadsto$.

\index{component!strongly connected}
\index{component!maximal strongly connected}
A \emph{strongly connected component} (SCC)
of an automaton $\aut$ is a subset $C\subseteq A$ such that for every $b,c \in C$ we
have $b \ordeq c$ and $c \ordeq b$. 
The SCC is called \emph{maximal} (MSCC) when no proper extension of $C$
is an SCC.
\end{definition}

%%%
\myparagraphns{Closure under complementation.}
Many properties of parity automata can already be determined at the one-step level.
An important example concerns the notion of complementation.


\begin{definition}
\label{d:bdual1}
Two one-step formulas $\varphi$ and $\psi$ are each other's \emph{Boolean dual}
if for every structure $(D,\val)$ we have:
\[
(D,\val) \models \varphi \quad\text{iff}\quad (D,\val^{c}) \not\models \psi,
\]
where $\val^{c}$ is the valuation given by $\val^{c}(a) \mathrel{:=} D
\setminus \val(a)$, for all $a$.
%
A one-step language $\llang$ is \emph{closed under Boolean duals} if for every
set $A$, each formula $\varphi \in \llang(A)$ has a Boolean dual $\dual{\varphi}
\in \llang(A)$.
\end{definition}

Following ideas from~\cite{Muller1987,DBLP:conf/calco/KissigV09}, we can use Boolean duals, together with a
\emph{role switch} between $\abelard$ and $\eloise$, in order to define a
negation or complementation operation on automata.

\begin{definition}
\label{d:caut}
Assume that, for some one-step language $\llang$, the map $\dual{(-)}$
provides, for each set $A$, a Boolean dual $\dual{\varphi} \in \llang(A)$ for each
$\varphi \in \llang(A)$.
Given $\aut = \tup{A,\tmap,\pmap,a_I}$ in $\Aut(\llang)$ we define its
\emph{complement} $\dual{\aut}$ as the automaton
$\tup{A,\dual{\tmap},\dual{\pmap},a_I}$
where $\dual{\tmap}(a,c) := \dual{(\tmap(a,c))}$, and $\dual{\pmap}(a)
:= 1 + \pmap(a)$, for all $a \in A$ and $c \in \wp(\props)$.
\end{definition}

\begin{proposition}
\label{prop:autcomplementation}
Let $\llang$ and $\dual{(-)}$ be as in the previous definition.
For each automaton $\aut \in \Aut(\llang)$ and each transition system
$\model$ we have that
\[
\dual{\aut} \text{ accepts } \model
\quad\text{iff}\quad
\aut \text{ rejects } \model.
\]
\end{proposition}

The proof of Proposition~\ref{prop:autcomplementation} is based on the fact
that the power of $\eloise$ in $\agame(\dual{\aut},\model)$ is the same
as that of $\abelard$ in $\agame(\aut,\model)$, as defined in~\cite{DBLP:conf/calco/KissigV09}.

As an immediate consequence of this proposition, one may show that if the
one-step language $\llang$ is closed under Boolean duals, then the class
$\Aut(\llang)$ is closed under taking complementation.
Further on in Chapter~\ref{chap:sub} we will use Proposition~\ref{prop:autcomplementation} to show that
the same may apply to some subclasses of $\Aut(\llang)$.
