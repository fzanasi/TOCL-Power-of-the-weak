% \section{Expressiveness modulo bisimilarity}\label{sec:expresso}

In this section we will prove the main results of the paper
In this Section we use the tools developed in the previous parts to prove  the
main results of the paper, namely that both $M=\mucML, L=\wmso$ and  $M=\AFMC,
L=\nmso$ are  solutions of the equation $M \equiv L / {\bis}$:

\begin{eqnarray}
% \[\begin{array}{rcl}
%\label{eq:1}
%\eqno{(*)}
   \AFMC &\equiv& \nmso / {\bis} 
   \label{eq:z1}
\\[1mm] \mucML &\equiv& \wmso / {\bis} 
   \label{eq:z2}
% \end{array}\]
\end{eqnarray}

As explained in the introduction, the structure of the proof is the same in both 
cases. 

\begin{proofof}{\eqref{eq:z1} and \eqref{eq:z2}}
The inclusions from left to right are given by effective translations,
cf.~Theorem \ref{t:mfl2mso}.

The translation in the other direction essentially passes through automata.
As the key technical result of this part, in subsection~\ref{ss:mso2mfl} below
we will provide a construction 
\[(\cdot)^{\bullet}: \begin{cases} \AutW(\ofoe) \to
\AutW(\ofo), \\
\AutWC(\olque) \to
\AutWC(\ofo), \\
\end{cases}\] 
such that for all $\bbA$ and $\model$: % we have
\begin{equation}
\label{eq:crux}
\bbA^{\bullet} \text{ accepts } \model \text{ iff } \bbA \text{ accepts
} \model^{\om},
\end{equation}
where $\model^{\om}$ is the $\om$-unravelling of $\model$.
 
Based on this automata-theoretic construction and result, we proceed as follows,
in the case of $\wmso$ and $\mucML$ (the case of $\nmso$ and $\AFMC$ is 
completely similar).
Given a \wmso-formula $\phi$, define 
\[
\phi^{\bullet} \df \xi_{\aut_{\phi}^{\bullet}},
\]
\btbs
\item
where $\aut_{\phi}$ is \ldots\ and $\xi_{\aut_{\phi}^{\bullet}}$ is \ldots
\etbs

To prove the right-to-left direction of \eqref{eq:z2}, clearly it suffices to
show that
\begin{equation}
\label{eq:qq1}
\phi\in\wmso \text{ is bisimulation invariant iff } \phi \equiv \phi^{\bullet}.
\end{equation}

The direction from right to left of \eqref{eq:qq1} is immediate by the 
observation that $\phi^{\bullet}$ is a formula in $\MC$.
The opposite direction follows from the following chain of equivalences:
\begin{align*}
\model \models \phi
  & \text{ iff } \model^{\om} \models \phi
  & \tag{$\phi$ bisimulation invariant}
\\ & \text{ iff }  \bbA_{\phi} \text{ accepts } \model^{\om}
  & \tag{$ \phi \equiv \aut_{\phi}$ on trees}
\\ & \text{ iff } \bbA_{\phi}^{\bullet} \text{ accepts } \model
& \tag{\ref{eq:crux}}
\\ & \text{ iff }  \model \models \xi_{ \aut_{\phi}^{\bullet}}
& \tag{$\aut_{\phi}^{\bullet}\equiv \xi_{ \aut_{\phi}^{\bullet}}$}
\end{align*}
This finishes the proof of \eqref{eq:z2}, and as mentioned, the proof of
\eqref{eq:z1} is similar.
\end{proofof}

\subsection{From second order languages to modal languages}
\label{ss:mso2mfl}

In this subsection we will define a construction that, for $\llang \in
\{{\ofoe},{\olque}\}$, transforms an arbitrary $\llang$-automaton $\bbA$ into
an $\ofo$-automaton $\bbA^{\bullet}$ such that $\bbA$ and $\bbA^{\bullet}$ 
are related as in~\eqref{eq:crux}; in addition, we will make sure that 
this transformation preserves both the weakness and the continuity condition.
This operation $(\cdot)^{\bullet}$ is completely determined by the following 
translation at the one-step level.

\begin{definition}
Recall that by Corollary~\ref{cor:ofoepositivenf}(ii), any formula in 
${\ofoe}^+(A)$ is equivalent to a disjunction of formulas of the form 
$\posdbnfofoe{\vlist{T}}{\Sigma}$, whereas
by Corollary~\ref{cor:ofoeicontinuousnf}(ii), any formula in ${\olque}^+(A)$ is 
equivalent to a disjunction of formulas of the form 
$\posdbnfolque{\vlist{T}}{\Pi}{\Sigma}$. 
Based on those results, for both one step languages $\llang={\ofoe}$ and 
$\llang={\olque}$, we define the translation 
$(-)^{\bullet} : {\llang}^+(A) \to \ofo^+(A)$ by setting
% \[
% \Big( \posdbnfofoe{\vlist{T}}{\Sigma} \Big)^{\bullet} = 
% \Big( \posdbnfolque{\vlist{T}}{\Pi}{\Sigma} \Big)^{\bullet} 
% \df
% \bigwedge_{i} \exists x_i. \tau^+_{T_i}(x_i) \land 
% \forall x. \bigvee_{S\in\Sigma} \tau^+_S(x),
% \]
\[
\left.\begin{array}{l}
   \Big( \posdbnfofoe{\vlist{T}}{\Sigma} \Big)^{\bullet} 
\\ \Big( \posdbnfolque{\vlist{T}}{\Pi}{\Sigma} \Big)^{\bullet} 
\end{array}\right\}
\df \bigwedge_{i} \exists x_i. \tau^+_{T_i}(x_i) \land 
\forall x. \bigvee_{S\in\Sigma} \tau^+_S(x),
\]
and for $\al = \bigvee_{i} \al_{i}$ we define $\al^{\bullet} \df \bigvee 
\al_{i}^{\bullet}$.
\end{definition}

The key property of this translation is the following.

\begin{proposition}
\label{p-1P}
Let $\llang\in \{{\ofoe},{\olque}\}$.
For every one-step model $(D,V)$ and every $\al \in \llang^+(A)$ we have
\begin{equation}
\label{eq-1cr}
(D,V) \models \alpha^{\bullet} \text{ iff } (D\times \om,V_\pi) \models \alpha,
\end{equation}
where $V_{\pi}$ % =  f^{-1} \circ V$
 is the induced valuation given by 
$V_{\pi}(a) \df \{ (d,k) \mid d \in V(a), k\in\omega\}$.
\end{proposition}

\begin{proof}
We prove the claim for $\llang={\olque}$, the other case being similar.
Clearly it suffices to prove \eqref{eq-1cr} for formulas of the form
$\al = \posdbnfolque{\vlist{T}}{\Pi}{\Sigma}$.
\smallskip

\noindent\fbox{$\Rightarrow$} 
Assume $(D,\val) \models \alpha^{\bullet}$, we will show that 
$(D\times \omega,\val_\pi) \models \posdbnfolque{\vlist{T}}{\Pi}{\Sigma}$.
Let $d_i$ be such that $\tau_{T_i}^+(d_i)$ in $(D,\val)$. 
It is clear that the $(d_i,i)$ provide \emph{distinct} elements satisfying 
$\tau_{T_i}^+((d_i,i))$ in $(D\times\omega,\val_{\pi})$ and therefore the 
first-order existential part of $\alpha$ is satisfied. 
With a similar but easier argument it is straightforward that the existential 
generalized quantifier part of $\alpha$ is also satisfied.
For the universal parts of $\posdbnfolque{\vlist{T}}{\Pi}{\Sigma}$ it is enough
to observe that, because of the universal part of $\alpha^\bullet$, \emph{every}
$d\in D$ realizes a positive type in $\Sigma$. 
By construction, the same applies to $(D\times\omega,\val_{\pi})$, 
therefore this takes care of both universal quantifiers.
\medskip
		
\noindent\fbox{$\Leftarrow$} 
Assuming that $(D\times \omega,\val_\pi) \models 
\posdbnfolque{\vlist{T}}{\Pi}{\Sigma}$,
we will show that $(D,\val) \models \alpha^\bullet$. 
The existential part of $\alpha^{\bullet}$ is trivial. 
For the universal part we have to show that every element of $D$ realizes the 
positive part of a type in $\Sigma$. 
Suppose not, and let $d\in D$ be such that $\lnot\tau_S^+(d)$ for all $S\in 
\Sigma$. 
Then we have $(D\times\omega,\val_\pi) \not\models \tau_S^+((d,k))$ for all $k$.
That is, there are infinitely many elements not realizing the positive part of 
any type in $\Sigma$. 
Hence we have $(D\times\omega,\val_\pi) \not\models \dqu y.\bigvee_{S\in\Sigma} 
\tau_S^+(y)$. 
Absurd, because that is part of $\posdbnfolque{\vlist{T}}{\Pi}{\Sigma}$.
\end{proof}

As a consequence of Proposition~\ref{p-1P} we obtain the following.

\begin{definition}
Let $\llang\in \{{\ofoe},{\olque}\}$.
Given an automaton $\bbA = \tup{A,\tmap,\pmap,a_{I}}$ in $\Aut(\llang)$, define 
the automaton $\bbA^{\bullet} \df \tup{A,\tmap^{\bullet},\pmap,a_{I}}$ in 
$\Aut(\ofo)$ by putting, for each $(a,c) \in A \times C$:
\[
\tmap^{\bullet}(a,c) \df (\tmap(a,c))^{\bullet}.
\]
\end{definition}

\begin{proposition}
For any automaton $\bbA = \tup{A,\tmap,\pmap,a_{I}}$ in $\Aut(\llang)$, and any
model $\model$, $\bbA$ and $\model$ satisfy \eqref{eq:crux}.
\end{proposition}

\begin{proof} Let $\llang\in \{{\ofoe},{\olque}\}$.
The proof of this proposition is based on a fairly routine comparison of the 
acceptance games $\mathcal{A}(\bbA^{\bullet},\model)$ and 
$\mathcal{A}(\bbA,\model^{\om})$.
In a slightly more general setting, the details of this proof can be found 
in~\cite{Venxx}.
\end{proof}
\medskip

It remains to be checked that the construction $(-)^{\bullet}$, which has
been defined for arbitrary automata in $\Aut(\llang)$, transforms 
both $\wmso$-automata and $\nmso$-automata into automata in the right class.
This is the content of the next proposition, which can be verified by a 
straightforward inspection at the one-step level.
\begin{proposition}
Let $\bbA \in \Aut(\ofoe)\cup\Aut(\olque)$. 
\begin{itemize}
\item If $\bbA \in \AutW(\ofoe)$, then $\bbA^{\bullet} \in \AutW(\ofo)$, and
\item if $\bbA \in \AutWC(\olque)$, then $\bbA^{\bullet} \in \AutWC(\ofo)$.
\end{itemize}
\end{proposition}

%\begin{proof}
%This proposition can be verified by a straightforward inspection, at the 
%one-step level, that if a formula $\al \in {\olque}^+(A)$ belongs to the fragment 
%$\cont{{\olque}^+}{a}(A)$, then its translation $\al^{\bullet}$ lands in 
%the fragment $\cont{\ofo^+}{a}(A)$.
%\end{proof}

\begin{remark}{\rm
As a corollary of the previous two propositions we find that 
\begin{itemize}
	\itemsep 0 pt
	\item $\AutW(\ofo) \equiv \AutW(\ofoe)/{\bis}$, and
	\item $\AutWC(\ofo) \equiv \AutWC(\olque)/{\bis}$.
\end{itemize}
In fact, we are dealing here with an instantiation of a more general phenomenon 
that is essentially coalgebraic in nature.
In~\cite{Venxx} it is proved that if $\llang$ and $\llang'$ are two one-step
languages that are connected by a translation $(-)^{\bullet}: \llang' \to 
\llang$ satisfying a condition similar to \eqref{eq-1cr}, then we find that 
$\Aut(\llang)$ corresponds to the bisimulation-invariant fragment of 
$\Aut(\llang')$: $\Aut(\llang) \equiv \Aut(\llang')/{\bis}$.
This subsection can be generalized to prove similar results relating
$\AutW(\llang)$ to $\AutW(\llang')$, and $\AutWC(\llang)$ to 
$\AutWC(\llang')$.
}\end{remark}
