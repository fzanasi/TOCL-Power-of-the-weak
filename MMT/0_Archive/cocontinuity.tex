\newpage


\subsection{Dual fragments}

\btbs
\item
Shouldn't this part move to the other paper?
\etbs

In this subsection we give syntactic characterizations of the co-continuous
fragment of several one-step logics. This notion is dual to continuity.

We are now ready to give the syntactic definition of the dual fragments for the 
one-step logics into consideration.

\begin{definition}\label{def:cocontfrag}\label{def:multfrag}
The fragment $\cocont{\ofoei(A)}{B}$ is given by the sentences generated by:
\marginpar{\fbox{YV: incorrect}}
\[
\phi \defbnf \psi 
   \mid b(x) 
   \mid \forall x.\phi \mid \dqu x.\phi 
   \mid \phi \lor \phi \mid \phi \land \phi
\]
where $b\in B$ and $\psi \in \ofoei(A\setminus B)$. 
Observe that the equality is included in $\psi$. 
The fragment $\cocont{\ofo(A)}{B}$ is defined as $\cocont{\ofoei(A)}{B}$ but without the clause 
for $\dqu$ and with $\psi\in \ofo(A\setminus B)$.
\end{definition}

The following proposition states that the above fragments are actually the duals of the fragments defined earlier in this chapter.

\begin{proposition}\label{prop:newfragsduals}
The following hold:
	%
\begin{align*}
\cocont{\ofoei(A)}{B} &= \{\phi \mid 
   \phi^\delta \in \cont{\ofoei(A)}{B}\} \\
\cocont{\ofo(A)}{B} &= \{\phi \mid 
   \phi^\delta \in \cont{\ofo(A)}{B}\}.
	\end{align*}
	%
\end{proposition}
\begin{proof}
	Easily proved by induction.
\end{proof}

\noindent As a corollary, we get a characterisation for co-continuity.

\begin{corollary}~ \label{cor:cocontinuity}
	Let $\llang \in \{\ofo,\ofoei\}$. A formula $\phi \in \llang(A)$ is co-continuous in $a\in A$ if and only if it is equivalent to some $\phi' \in \cocont{\llang}{a}(A)$.
\end{corollary}
\begin{proof}	
This is a onsequence of Proposition~\ref{prop:newfragsduals} 
and~\ref{props:duals}.
\end{proof}
