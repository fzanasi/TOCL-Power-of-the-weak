% !TEX root = ../main.tex

\section{Basics}

In this section we provide the basic definitions of the monadic predicate 
liftings that we study in this paper.
Throughout this paper we fix a finite set $A$ of objects that we shall refer to 
as \emph{(monadic) predicate symbols} or \emph{names}.
We shall also assume an infinite set $\fovar$ of \emph{individual variables}.

\begin{definition}
Given a finite set $A$ we define a \emph{(monadic) model} to be a pair
$\osmodel = (D,V )$ consisting of a set $D$,  which we call the \emph{domain} of 
$\osmodel$, and an interpretation or \emph{valuation} $V : A \to \wp (D)$.
The class of all models will be denoted by $\umods$.
\end{definition}

\begin{remark}~
Note that we make the somewhat unusual choice of allowing the domain of a
monadic model to be \emph{empty}.
In view of the applications of our results to automata theory (see 
Section \ref{sec:intro}) this choice is very natural, even if it means that 
some of our proofs here become more laborious in requiring an extra check.
Observe that there is exactly one monadic model based on the empty domain;
we shall denote this model as $\emodel \isdef (\nada, \nada)$. 
\end{remark}

\begin{definition}
Observe that a valuation $V: A \to \wp (D)$ can equivalently be presented via 
its associated \emph{colouring} $V^{\flat}:D \to \wp(A)$ given by
\[
V^{\flat}(d) \isdef \{a \in A \mid d \in V(a)\}.
\]
We will use these perspectives interchangeably, calling the set $V^{\flat}(d)
\subseteq A$ the \emph{colour} or \emph{type} of $d$.
In case $D = \nada$, $V^{\flat}$ is simply the empty map.
\end{definition}

In this paper we study three languages of monadic predicate logic: the languages 
$\ofoe$ and $\ofo$ of monadic first-order logic with and without equality, 
respectively, and the extension $\ofoei$ of $\ofoe$ with the 
generalised quantifiers $\qu$ and $\dqu$.
Probably the most concise definition of the full language of monadic predicate 
logic would be given by the following grammar:
\[
\varphi \defbnf a(x)
\mid x \foeq y
\mid \neg \varphi
\mid \varphi \lor \varphi
\mid \exists x.\varphi
\mid \qu x.\varphi,
\]
where $a \in A$ and $x$ and $y$ belong to the set $\fovar$ of individual 
variables.
In this set-up we would need to introduce the quantifiers $\forall$ and $\dqu$ 
as abbreviations of $\neg \exists \neg$ and $\neg \qu \neg$, respectively.
However, for our purposes it will be more convenient to work with a variant of 
this language where all formulas are in negation normal form; that is, we only 
permit the occurrence of the negation symbol $\neg$ in front of an atomic 
formula.
In addition, for technical reasons we will add $\bot$ and $\top$ as constants,
and we will write $\neg (x \foeq y)$ as $x \foneq y$.
Thus we arrive at the following definition of our syntax.

\begin{definition}
The set $\ofoei(A)$ of \emph{monadic formulas} is given by the following grammar:
{\small \[ \varphi  \defbnf 
\top \mid \bot 
\mid a(x)
\mid \neg a(x)
\mid x \foeq y
\mid x \foneq y
\mid \varphi \lor \varphi
\mid \varphi \land \varphi
\mid \exists x.\varphi
\mid \forall x.\varphi
\mid \qu x.\varphi
\mid \dqu x.\varphi
\]
}
%\[\begin{array}{lcl}
%\varphi & \defbnf &
%\top \mid \bot 
%\mid a(x)
%\mid \neg a(x)
%\mid x \foeq y
%\mid x \foneq y
%\mid \varphi \lor \varphi
% \\
%& &
%\mid \varphi \land \varphi
%\mid \exists x.\varphi
%\mid \forall x.\varphi
%\mid \qu x.\varphi
%\mid \dqu x.\varphi
%\end{array}\]
where $a \in A$ and $x,y\in \fovar$. 
The language $\ofoe(A)$ of \emph{first-order logic with equality} is defined as
the fragment of $\ofoei(A)$ where occurrences of the generalised quantifiers 
$\qu$ and $\dqu$ are not allowed:
\[
\varphi \defbnf
\top \mid \bot 
\mid a(x)
\mid \neg a(x)
\mid x \foeq y
\mid x \foneq y
\mid \varphi \lor \varphi
\mid \varphi \land \varphi
\mid \exists x.\varphi
\mid \forall x.\varphi
\]
Finally, the language $\ofo(A)$ of \emph{first-order logic} is the equality-free
fragment of $\ofoe(A)$; that is, atomic formulas of the form $x \foeq y$ and 
$x \foneq y$ are not permitted either:
\[
\varphi \defbnf
\top \mid \bot 
\mid a(x)
\mid \neg a(x)
\mid \varphi \lor \varphi
\mid \varphi \land \varphi
\mid \exists x.\varphi
\mid \forall x.\varphi
\]

In all three languages we use the standard definition of free and bound
variables, and we call a formula a \emph{sentence} if it has no free variables.
In the sequel we will often use the symbol $\llang$ to denote either of the 
languages $\ofo$, $\ofoe$ or $\ofoei$.

For each of the languages $\llang \in \{ \ofo, \ofoe, \ofoei \}$, we define the 
\emph{positive fragment} $\llang^{+}$ of $\llang$ as the language obtained by 
almost the same grammar as for $\llang$, but with the difference that we do not
allow negative formulas of the form $\neg a(x)$.
\end{definition}

\noindent

The semantics of these languages is given as follows. 

\begin{definition}
The semantics of the languages $\ofo, \ofoe$ and $\ofoei$ is given in the form
of a truth relation $\models$ between models and sentences of the language.
To define the truth relation on a model $\osmodel = (D ,V)$, we distinguish 
cases.


\begin{description}
\item[Case $D=\nada$:] 
We define the \emph{truth relation} $\models$ on the empty model $\emodel$ for
all formulas that are Boolean combinations of sentences of the form $Qx. \phi$,
where $Q \in \{ \exists, \qu, \forall, \dqu \}$ is a quantifier.
The definition is by induction on the complexity of such sentences; the 
``atomic'' clauses, where the sentence is of the form $Qx. \phi$, is as follows:
\[\begin{array}{lll}
 \emodel \not\models Qx. \phi 
   & \text{if}\quad Q \in \{ \exists,  \qu \},&
\\   
\emodel \models Qx. \phi 
   & \text{if}\quad Q \in \{ \forall,  \dqu \}.&
\end{array}\]
The clauses for the Boolean connectives are as expected.

\item[Case $D \neq \nada$:] 
In the (standard) case of a non-empty model $\osmodel$, we extend the truth 
relation to arbitrary formulas, involving assignments of individual variables 
to elements of the domain.
That is, given a model $\osmodel = (D,V)$, an assignment $g :\fovar\to D$ and 
a formula $\phi \in \ofoei(A)$ we define the \emph{truth} relation $\models$ 
by a straightforward induction on the complexity of $\phi$.
Below we explicitly provide the clauses of the quantifiers:
%
\[\begin{array}{lll}
   \osmodel,g \models \exists x.\varphi 
   & \text{iff}\quad \osmodel,g [x\mapsto d] \models \varphi 
     \text{ for some $d\in D$},
\\   \osmodel,g \models \forall x.\varphi 
   & \text{iff}\quad \osmodel,g [x\mapsto d] \models \varphi 
     \text{ for all $d\in D$},
\\   \osmodel,g \models \qu x.\varphi 
   & \text{iff}\quad \osmodel,g [x\mapsto d] \models \varphi 
     \text{ for infinitely many $d\in D$},
\\   \osmodel,g \models \dqu x.\varphi 
   & \text{iff}\quad \osmodel,g [x\mapsto d] \models \varphi 
     \text{ for all but at most finitely many $d\in D$}.
\end{array}\]
The clauses for the atomic formulas and for the Boolean connectives are standard.
\end{description}
In what follows, when discussing the truth of $\phi$ on the empty model, we
always implicitly assume that $\phi$ is a sentence.
\end{definition}

As mentioned in the introduction, general quantifiers such as $\qu$ and $\dqu$
were introduced by Mostowski~\cite{Mostowski1957}, who proved the decidability 
for the language obtained by  extending $\ofo$ with such quantifiers. 
The decidability of the full language $\ofoei$ was then proved by Slomson
in \cite{slomson1968monadic}.\footnote{%
   The argument in \cite{slomson1968monadic} is given in terms of the so called
   \emph{Chang quantifier} but is easily seen to work also for $\qu$ and $\dqu$.
   Both Mostowski's and Slomson's decidability results can be extended to the 
   case of the empty domain.}
The case for $\ofo$ and $\ofoe$ goes back already to
\cite{Behmann1922,Loewenheim1915}.

\begin{fact}\label{f:decido}
For each logic $\llang \in \{ \ofo, \ofoe, \ofoei \}$, the problem of whether a
given $\llang$-sentence $\phi$ is satisfiable, is decidable.
\end{fact}

\noindent
In the remainder of the section we fix some further definitions and notations,
starting with some useful syntactic abbreviations.

\begin{definition}
Given a list $\vlist{y} = y_1\cdots y_n$ of individual variables, we use the
formula
\[
\arediff{\vlist{y}} \isdef
\bigwedge_{1\leq m < m^{\prime} \leq n} (y_m \foneq y_{m^{\prime}})
\]
to state that the elements $\vlist{y}$ are all distinct.
%
An \emph{$A$-type} is a formula of the form 
\[
\tau_{S}(x) \isdef
\bigwedge_{a\in S}a(x) \land \bigwedge_{a\in A\setminus S}\lnot a(x).
\]
where $S \subseteq A$.
Here and elsewhere we use the convention that $\bigwedge \nada = \top$ (and 
$\bigvee \nada = \bot$).
The \emph{positive} $A$-type $\tau_{S}^+(x)$ only bears positive information, 
and is defined as 
\[
\tau_{S}^+(x) \isdef \bigwedge_{a\in S}a(x).
\]
Given a one-step model $\osmodel = (D,V)$ we define
\[
\sz{S}_\osmodel \isdef |\{d\in D \mid \osmodel \models \tau_S(d) \}|
\]
as the number of elements of $\osmodel$ that realise the type $\tau_{S}$.
\end{definition}

We often blur the distinction between the formula $\tau_{S}(x)$ and the subset 
$S \subseteq A$, calling $S$ an $A$-type as well.
Note that we have $\osmodel \models \tau_S(d)$ iff $V^{\flat}(d) = S$, so that
we may refer to $V^{\flat}(d)$ as the \emph{type of} $d \in D$ indeed.

\begin{definition}
The quantifier rank $\qr(\varphi)$ of a formula $\varphi \in \ofoei$ (hence also 
for $\ofo$ and $\ofoe$) is defined as follows:
%
\[\begin{array}{llll}
\qr(\varphi) &\isdef & 0 
  & \text{if $\varphi$ is atomic},
\\ \qr(\neg\psi) & \isdef & \qr(\psi)
\\ \qr(\psi_{1}\mathrel{\hs}\psi_{2}) & \isdef & \max\{\qr(\psi_1),\qr(\psi_2)\}
  & \text{where } \hs \in \{ \land,\lor\}
\\ \qr(Qx.\psi) & \isdef & 1+\qr(\psi),
  & \text{where } Q \in \{\exists,\forall,\qu,\dqu\}
\end{array}\]
%
Given a monadic logic $\llang$ we write $\osmodel \equiv_k^{\llang} \osmodel'$
to indicate that the models $\osmodel$ and $\osmodel'$ satisfy exactly the same
sentences $\varphi \in \llang$ with $\qr(\varphi) \leq k$. 
We write $\osmodel \equiv^{\llang} \osmodel'$ if $\osmodel \equiv_k^{\llang}
\osmodel'$ for all $k$.
When clear from context, we may omit explicit reference to $\llang$.
\end{definition}

\begin{definition}
A \emph{partial isomorphism} between two models $(D,V )$ and $(D',V ')$ is a 
partial function $f: D \pto D'$ which is injective and satisfies that 
$d \in V (a) \Leftrightarrow f(d) \in V '(a)$ for all $a\in A$ and $ d\in 
\Dom(f)$.
Given two sequences $\vlist{d} \in D^k$ and $\vlist{d'} \in {D'}^k$ we use 
$f:\vlist{d} \mapsto \vlist{d'}$ to denote the partial function $f:D\pto D'$ 
defined as $f(d_i) \isdef d'_i$. 
We will take care to avoid cases where there exist $d_i,d_j$ such that $d_i = 
d_j$ but $d'_i \neq d'_j$.
\end{definition}

Finally, for future reference we briefly discuss the notion of \emph{Boolean 
duals}.
We first give a concrete definition of a dualisation operator on the set of 
monadic formulas.

\begin{definition}\label{def:concreteduals} 
The \emph{(Boolean) dual} $\phi^{\delta} \in {\ofoei}(A)$ of $\phi\in 
{\ofoei}(A)$ is the formula given by:
\begin{align*}
 (a(x))^{\delta} & \isdef  a(x) 
 & (\lnot a(x))^{\delta} & \isdef  \lnot a(x) 
\\ (\top)^{\delta} & \isdef  \bot 
  & (\bot)^{\delta} & \isdef  \top 
\\  (x \approx y)^{\delta} & \isdef  x \not\approx y 
  & (x \not\approx y)^{\delta}& \isdef  x \approx y 
\\ (\phi \wedge \psi)^{\delta} &\isdef  \phi^{\delta} \vee \psi^{\delta} 
  &(\phi \vee \psi)^{\delta}& \isdef  \phi^{\delta} \wedge \psi^{\delta}
\\ (\exists x.\psi)^{\delta} &\isdef  \forall x.\psi^{\delta} 
  &(\forall x.\psi)^{\delta} &\isdef  \exists x.\psi^{\delta} 
\\ (\exists^{\infty} x.\psi)^{\delta} &\isdef \forall^{\infty} x.\psi^{\delta} 
  &(\forall^{\infty} x.\psi)^{\delta} &\isdef  \exists^{\infty} x.\psi^{\delta}
\end{align*}
\end{definition}

\begin{remark}
Where $\llang\in\{\ofo,\ofoe,\ofoei\}$, observe that if $\phi \in \llang(A)$ 
then $\phi^{\delta} \in \llang(A)$. 
Moreover, the operator preserves positivity of the predicates, that is, if $\phi
\in \llang^+(A)$ then $\phi^{\delta} \in \llang^+(A)$.
\end{remark}

\noindent 
The following proposition states that the formulas $\phi$ and $\phi^{\delta}$ 
are Boolean duals.
We omit its proof, which is a routine check.

\begin{proposition}\label{props:duals}
Let $\phi \in \ofoei(A)$ be a monadic formula.
Then $\phi$ and $\phi^{\delta}$ are indeed Boolean duals, in the sense that for
every monadic model $(D,V)$ we have that
\[
(D,V) \models \phi \text{ iff } (D,V^{c}) \not\models \phi^{\delta},
\]
where $V^{c}: A \to \wp (D)$ is the valuation given by $V^{c}(a) \isdef 
D \setminus V(a)$.
\end{proposition}

