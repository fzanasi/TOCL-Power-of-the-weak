% !TEX root = ../main.tex

\section{Introduction}\label{sec:intro}

Model theory investigates the relationship between formal languages and 
semantics. 
From this perspective, among the most important results are the so called 
\emph{preservation theorems}. 
Such results typically characterise a certain language as the fragment of 
another, richer language satisfying a certain model-theoretic property.
In doing so, they therefore  link the syntactic shape of a formula with the 
semantic properties of the class of models  it defines. 
In the case of classical first-order logic, notable examples are the 
{\L}o\'s-Tarski theorem,  stating that a first-order  formula is equivalent to a universal one if and only if the class of its models is closed under taking submodels, and Lyndon's theorem, stating that a first-order formula is equivalent to  one for which each occurrence of a relation symbol $R$  is positive if and only if it is monotone with respect to the interpretation of $R$ (see e.g. \cite{Hodges1993}).

The aim of this paper is to show that similar results also hold when considering the predicate logic $\ofoei$ that allows only monadic predicate symbols and no function symbols, but that goes beyond standard first-order logic with equality in that it features the generalised quantifier `there are infinitely many'. 
 
Generalised quantifiers were introduced by Mostowski in \cite{Mostowski1957}, 
and in a more general sense by Lindstr\"{o}m in \cite{perlindstrom1966first},
the main motivation being the observation that standard first-order quantifiers 
`there are some' and `for all' are not sufficient for expressing some basic 
mathematical concepts. 
Since then, they have attracted a lot of interests, insomuch that their study
constitutes nowadays a well-established field of logic with important 
ramifications in disciplines such as linguistics and computer 
science.\footnote{%
   For an overview see e.g. 
   \cite{van1995directions,vaananen1997generalized,sep-generalized-quantifiers}.
   For an introduction to the model theory of generalised quantifiers, the 
   interested reader can consult for 
   instance~\cite[Chapter~10]{vaananen2011models}.
   }. 


Despite the fact that the absence of polyadic predicates clearly restricts its expressing power, monadic first-order logic (with identity) displays nice properties, both from a computational and a model-theoretic point of view. Indeed, the  
satisfiability problem becomes decidable  \cite{Behmann1922,Loewenheim1915}, and, in addition of an immediate application of {\L}o\'s-Tarski  and Lyndon's theorems, one can also obtain a Lindstr\"om like characterisation result  \cite{tharp1973characterization}.
Moreover, adding the possibility of quantifying over predicates does not increase the expressiveness of the language \cite{ackermann1954solvable}, meaning that when restricted to monadic predicates, monadic second order logic collapses into first-order logic. 

For what concerns monadic first-order logic extended with an infinity quantifier,
in \cite{Mostowski1957} Mostowski, already proved its decidability, whereas from 
work of V\"a\"an\"anen  \cite{vaananen77} we know that its expressive power 
coincides with that of weak monadic second-order logic restricted to monadic 
predicates, that is monadic first-order logic extended with a second order 
quantifier ranging over finite sets\footnote{%
   Extensions of monadic first-order logic with other generalised quantifiers
   have also been studied (see 
   e.g.~\cite{slomson1968monadic,caicedo1981extensions}).
   }.


\subsection*{Preservation results and proof outline.}

A preservation result involves some fragment $\llang_{\mathfrak{P}}$ of a given
yardstick logic $\llang$, related to a certain semantic property $\mathfrak{P}$. 
It is usually formulated as
\begin{equation}\label{eq:intro-0}
\phi  \in \llang \text{ has the property } \mathfrak{P} \text{ iff } 
\phi \text{ is equivalent to some } \phi' \in \llang_{\mathfrak{P}}.
\end{equation}

In this work, our main yardstick logic will be $\ofoei$. 
Table \ref{tab:0} summarises the semantic properties
($\mathfrak{P}$) we are going to consider,  
the corresponding expressively complete fragment ($\llang_{\mathfrak{P}}$) and preservation theorem.
{\small
\begin{table}[h!]
\begin{center}
\begin{tabular}{|c|c|c|}
\hline
$\mathfrak{P}$				
   & $\llang_{\mathfrak{P}}$ 		
   & Preservation Theorem 
\\ \hline \hline
     Monotonicity     
   & Positive fragment         
   & Theorem \ref{t:mono}                        
\\ (Definition \ref{def:mono})		
   & $\monot{\ofoei}{}$
   &
\\ \hline	
     Continuity     
   & Continuous fragment  
   & Theorem \ref{t:cont}            
\\ (Definition \ref{def:cont})       	
   & $\cont{\ofoei}{}$        
   &                              
\\\hline
     Preservation under submodels
   & Universal fragment    	 
   &          Theorem \ref{t:univ}                    
\\ (Definition \ref{d:inv}(\ref{d:inv}))    
   &          $\univ{\ofoei}{}$          
   &                              
\\ \hline
     Invariance under quotients & Monadic first-order logic
   & Theorem \ref{t:qinv}                 
\\ (Definition \ref{d:inv}(\ref{d:inv}))   	
   & $\ofo$             	
   &                              
\\\hline
\end{tabular}
\caption{A summary of our preservation theorems}
\label{tab:0}
\end{center}
\end{table}
}
The proof of each  preservation theorem is composed of two parts. 
The first, simpler one concerns 
the claim that each sentence in the fragment satisfies the concerned property. 
It is usually proved  by induction on the structure of the sentence.  
The other direction is the  \emph{expressive completeness statement}, stating 
that within the considered logic, the fragment is expressively complete for the 
property. 
Its verification generally requires more effort. 
In this paper, we will actually verify a stronger expressive completeness 
statement. 
Namely, for each semantic property $\mathfrak{P}$ and corresponding fragment
$\llang_{\mathfrak{P}}$ from Table \ref{tab:0}, we are going to provide an 
effective translation operation $(\cdot)^{\sf p}: \ofoei \to 
\llang_{\mathfrak{P}}$ such that \begin{equation}\label{eq:intro-i}
\text{if }\phi \in \ofoei\text{ has the property } \mathfrak{P} \text{ then }
\phi \text{ is equivalent to } \phi^{\sf p}.
\end{equation}
Since the satisfiability problem for $\ofoei$ is decidable and the translation
$(\cdot)^{\sf p}$ is effectively computable, we obtain, as an immediate 
corollary of (\ref{eq:intro-i}), that for each property $\mathfrak{P}$ listed
in Table \ref{tab:0}
\begin{equation}\label{eq:intro-ii}
\text{the problem if
%whether 
a $\ofoei$-sentence satisfies property $\mathfrak{P}$
or not is decidable.}
\end{equation}


The proof of each instance of (\ref{eq:intro-i})  will follow an uniform 
pattern, analogous to the one employed in the aim of obtaining similar results
in the context of the modal $\mu$-calculus \cite{Jan96,d2000logical,FV12}.
The crux of the adopted proof method is that, extending known results on monadic
first-order logic, for each sentence $\phi$ in $ \ofoei$ it is possible to compute
a logically equivalent sentence in \emph{basic normal norm}. 
Such normal forms will take the shape of a disjunction $\bigvee \dbnf_{\ofoei}$, 
where each disjunct $\dbnf_{\ofoei}$ characterises a class of models of $\phi$
satisfying the same set of ${\ofoei}$-sentences of equal quantifier rank as 
$\phi$. 
Based on this, it will therefore be enough to define an effective translation
$(\cdot)^{\sf p}$ for sentences in normal form, point-wise in each disjunct 
$\dbnf_{\ofoei}$, and  then verify that it indeed satisfies (\ref{eq:intro-i}). 

As a corollary of the employed proof method, we thus obtain effective normal
forms for sentences satisfying the considered property.

In addition to $\ofoei$, in this paper we are also going to consider monadic 
first-order logic with and without equality, denoted respectively by $\ofoe$ 
and $\ofo$.
Table \ref{tab:1} shows a summary of the expressive completeness and normal 
form results presented in this paper. 
\begin{table}[h!]
\begin{center}
\begin{tabular}{cc|c|c|c|c|}
\cline{3-5}
& & \multicolumn{3}{ c| }{Language} \\ \cline{3-5}
& 									& $\ofo$ 					& $\ofoe$ 				& $\ofoei$  			\\ \cline{2-5}
& \multicolumn{1}{ |c| }{ Normal forms}		& Fact \ref{fact:ofonormalform} &  Thm. \ref{thm:bnfofoe} 	& Thm. \ref{thm:bfofoei} \\ \hline
\multicolumn{1}{ |c  }{\multirow{2}{*}{Monotonicity} }&
\multicolumn{1}{ |c| }{Completeness}   & Prop. \ref{p:fomon} & Prop. \ref{p:monofoeismonot} & Prop. \ref{p:mono-ofoei} \\ \cline{2-5}		
\multicolumn{1}{ |c  }{}                        &
\multicolumn{1}{ |c| }{Normal forms} 	& Cor. \ref{cor:ofopositivenf}	 		& Cor. \ref{cor:ofoepositivenf}		& Cor. \ref{cor:ofoeipositivenf}  \\ \cline{1-5}		
\multicolumn{1}{ |c  }{\multirow{2}{*}{Continuity} }&
\multicolumn{1}{ |c| }{Completeness}   & Prop. \ref{prop:ofocont}  & Fact \ref{fact:vb}  & Prop. \ref{lem:ofoeictrans} \\ \cline{2-5}		
\multicolumn{1}{ |c  }{}                        &
\multicolumn{1}{ |c| }{Normal forms} 	& Cor. \ref{cor:ofocontinuousnf}	 		& --		& Cor. \ref{cor:ofoeicontinuousnf}  \\ \cline{1-5}			
\multicolumn{1}{ |c  }{Preservation }&
\multicolumn{1}{ |c| }{Completeness}   & \multicolumn{3}{ c| }{Prop. \ref{p:univ2} }  \\ \cline{2-5}		
\multicolumn{1}{ |c  }{under submodels}                        &
\multicolumn{1}{ |c| }{Normal forms} 	& Cor. \ref{cor:univ}(\ref{cor:ofo})	 		& Cor. \ref{cor:univ}(\ref{cor:ofoe})		& Cor. \ref{cor:univ}(\ref{cor:ofoei})  \\ \cline{1-5}	
\multicolumn{1}{ |c  }{Invariance} &
\multicolumn{1}{ |c| }{Completeness}   & Prop. \ref{p:m-qinv} &\multicolumn{2}{ c| }{Prop. \ref{p:invq} }  \\ \cline{2-5}		
\multicolumn{1}{ |c  }{under quotients}                        &
\multicolumn{1}{ |c| }{Normal forms} 	& Fact \ref{fact:ofonormalform} & \multicolumn{2}{ c| }{Cor. \ref{cor:qinv} }  \\ \cline{1-5}	
\end{tabular}
\caption{An overview of our expressive completeness and normal form results.}
\label{tab:1}
\end{center}
\end{table}
%{\small
%\begin{table}[h!]
%\begin{center}
%\begin{tabular}{cc|c|c|c|c|}
%\cline{3-5}
%& & \multicolumn{3}{ c| }{Language} \\ \cline{3-5}
%& 									& $\ofo$ 					& $\ofoe$ 				& $\ofoei$  			\\ \cline{2-5}
%& \multicolumn{1}{ |c| }{ Normal forms}		& Fact \ref{fact:ofonormalform} &  Thm. \ref{thm:bnfofoe} 	& Thm. \ref{thm:bfofoei} \\ \hline
%\multicolumn{1}{ |c  }{\multirow{2}{*}{Monotonicity} }&
%\multicolumn{1}{ |c| }{Completeness}   & Prop. \ref{p:fomon} & Prop. \ref{p:monofoeismonot} & Prop. \ref{p:mono-ofoei} \\ \cline{2-5}		
%\multicolumn{1}{ |c  }{}                        &
%\multicolumn{1}{ |c| }{Normal forms} 	& Cor. \ref{cor:ofopositivenf}	 		& Cor. \ref{cor:ofoepositivenf}		& Cor. \ref{cor:ofoeipositivenf}  \\ \cline{1-5}		
%\multicolumn{1}{ |c  }{\multirow{2}{*}{Continuity} }&
%\multicolumn{1}{ |c| }{Completeness}   & Prop. \ref{prop:ofocont}  & Fact \ref{fact:vb}  & Prop. \ref{lem:ofoeictrans} \\ \cline{2-5}		
%\multicolumn{1}{ |c  }{}                        &
%\multicolumn{1}{ |c| }{Normal forms} 	& Cor. \ref{cor:ofocontinuousnf}	 		& --		& Cor. \ref{cor:ofoeicontinuousnf}  \\ \cline{1-5}			
%\multicolumn{1}{ |c  }{\multirow{2}{*}{Preservation 
%under submodels} }&
%\multicolumn{1}{ |c| }{Completeness}   & \multicolumn{3}{ c| }{Prop. \ref{p:univ2} }  \\ \cline{2-5}		
%\multicolumn{1}{ |c  }{}                        &
%\multicolumn{1}{ |c| }{Normal forms} 	& Cor. \ref{cor:univ}(\ref{cor:ofo})	 		& Cor. \ref{cor:univ}(\ref{cor:ofoe})		& Cor. \ref{cor:univ}(\ref{cor:ofoei})  \\ \cline{1-5}	
%\multicolumn{1}{ |c  }{\multirow{2}{*}{Invariance 
%under quotients } }&
%\multicolumn{1}{ |c| }{Completeness}   & Prop. \ref{p:m-qinv} &\multicolumn{2}{ c| }{Prop. \ref{p:invq} }  \\ \cline{2-5}		
%\multicolumn{1}{ |c  }{}                        &
%\multicolumn{1}{ |c| }{Normal forms} 	& Fact \ref{fact:ofonormalform} & \multicolumn{2}{ c| }{Cor. \ref{cor:qinv} }  \\ \cline{1-5}	
%\end{tabular}
%\caption{An overview of our expressive completeness and normal form results.}
%\label{tab:1}
%\end{center}
%\end{table}
%}
\subsection*{Application of obtained results: the companion paper}
\emph{Parity automata} are finite-state systems playing a crucial role in 
obtaining decidability and expressiveness results in fixpoint logic (see e.g. 
\cite{vardi2008automata}).
They are specified by a finite set of states $A$, a distinguished, initial state
$a \in A$, a function $\Omega$ assigning to each states a priority (a natural 
number), and a transition function $\Delta$ whose co-domain is usually given by
a monadic logic in which  the set of (monadic) predicates  coincides with $A$.
Hence, each monadic logic $\llang$ induces its own class of automata 
$\aut{\llang}$.

A landmark result in this area is Janin and Walukiewicz's theorem stating that
the bisimulation-invariant fragment of monadic second order logic coincides 
with the modal $\mu$-calculus \cite{Jan96}, and the proof of this result is an 
interesting mix of the theory of parity automata and the model theory of
monadic predicate logic.
First, preservation and normal forms results are used to verify that (on tree 
models) $\aut{\monot{\ofoe}{}}$ is the class of automata characterising the
expressive power of monadic second order logic \cite{Walukiewicz96}, whereas 
$\aut{\monot{\ofo}{}}$ corresponds to the modal $\mu$-calculus \cite{JaninW95},
where $\monot{\llang}{}$ denote the positive fragment of the monadic logic 
$\llang$. 
Then, Janin-Walukieiwcz's expressiveness theorem is a consequence of these 
automata characterisations and the fact that positive monadic first-order 
logic without equality provides the quotient-invariant fragment of positive
monadic first-order logic with equality (see Theorem \ref{t:inv1}).

In our companion paper \cite{companionWEAK}, we provide a Janin-Walukiewicz 
type characterisation result for \emph{weak} monadic second order logic.
Our proof, analogously to the case of full monadic second order logic discussed 
previously,  crucially employs preservation and normal form results for $\ofoei$
listed in  Tables \ref{tab:0} and \ref{tab:1}. 



\subsection*{Other versions}
Results in this paper first appeared in the first author's PhD thesis
(\cite[Chapter 5]{carreiro2015fragments}); this journal version largely expands
material first published as part of the conference 
papers \cite{DBLP:conf/lics/FacchiniVZ13,carreiro2014weak}. 
In particular, the whole of Section \ref{sec:inv} below contains new results.
