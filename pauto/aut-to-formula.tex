


In this subsection we focus on the following theorem.

\begin{theorem}\label{t:autofor}
There is an effective procedure that, given an automaton $\bbA$ in
$\yvcwAut(\ofo)$, returns an equivalent formula $\xi_{\bbA}$ of the fragment 
$\contAFMC$ of the modal $\mu$-calculus.
\end{theorem}

%Together with Theorem \ref{t:wmsoauto}, this means that in order to prove Theorem \ref{t:wmsobis} it is enough to verify the following:
\begin{proof}
The argument  is a refinement of the standard proof showing that any automaton 
$\aut$ from $\yvAut(\ofo)$ can be translated into an equivalent $\mu$-formula 
$\xi_\aut$ (cf. e.g. \cite{Ven08}), and it is essentially a special case of the argument proving Theorem \ref{thm:wmso_autofor}.



From now on, we always assume that a formula $\tmap(a,c)$ is in normal form. 
Following~\cite{Ven08}, we introduce another type of automata, called $(\prop,X)$-automata, which operate on $\p{(\prop \cup X)}$-trees.
They differ from automata whose one-step language is defined over predicates in $(A \cup X)$ in that\footnote{Parity automata based on $\ofo$ are thus simply $(\prop,\emptyset)$-automata.}
\begin{itemize}
\item Monadic predicate letters from $X$ can occur in the scope of a %existential
quantifier and only there, meaning that $(\prop,X)$-automata have transition $\tmap(a,c) \in \ofo^+(A\cup X)$
%
\item The transition function is uniquely determined by the restriction of the coloring to $\prop$, that is, for every $a \in A$, and $c_1, c_2 \in C$, if $c_1 \cap \prop = c_2 \cap \prop$ then $\tmap(a, c_1)= \tmap(a, c_2)$.
\end{itemize}
We also assume that
for every $x \in X$ there is a unique $a \in A$ and an unique $c \in C$ such that $x$ occurs in $\tmap(a,c)$.
The notion of acceptance is defined as expected, the only difference with being that during the acceptance game \'Eloise has to provide a valuation only for predicates in $A$ making formula given by the transition function true. %\fcnote{Maybe a bit more on the notion of acceptance?}
It is then enough to prove the following claim.
\begin{claimfirst}\label{c:1}
There is an effective procedure that, given a $(\prop,X)$-automaton $\aut$ gives an equivalent  formula $\xi_{\aut} \in \contAFMC$ in which all occurrences of variables in $X$ are positive.
\end{claimfirst}
\begin{pfclaim} %\ref{c:1}
%\fcwarning{where does the proof of claim start and end? use environment pfclaim}
Without loss of generality, we can assume that:
\begin{itemize}
\itemsep 0 pt
\item Every (maximal) strongly connected component (SCC) in the graph of $\ord$ has an unique entrance point,
\item The directed acyclic graph (DAG) of the SCCs of $\ord$ is a tree, and more specifically,
\item %Given a state $a \in A$, let $Reach(a)$ be the set of all states $b \in A$ that are not in the same SCC as $a$ but are such that $b$ occurs in $\tmap(a,c)$, for some  $c \in C$. Then, 
$\{c \in A \mid a \leadsto c, c \prec a\}  \cap \{c \in A \mid b \leadsto c, c \prec b\}  = \emptyset$ whenever $a,b$ are in the same SCC, with $a\neq b$.
\end{itemize}

Given a $(\prop,X)$-automaton $\aut$, we are now going to define a function $\delta_\aut: A \to \ML (A \cup X \cup \prop)$
%\fcwarning{one-step ML undefined} 
that assigns to each state $a$ of $\aut$ a modal formula $\delta_\aut(a)$ over  $A \cup X \cup \prop$ representing all possible transitions from $a$ in the modal language with the property that if $b \in A$ is in the same $\ord$-cycle of $a$ and $\pmap(a)=1$, then  $\delta_\aut(a)$ is continuous in $b$. Dually for $\pmap(a)=0$.

Let $c \in C$, and assume $\Delta(a,c)$ is in positive basic form $\bigvee \posdbnfofo{\Sigma}$. We define a first translation $TR_1$ taking as argument $\Delta(a,c)$ and giving as result a formula from the (guarded fragment of) first-order logic over $A \cup X \cup \prop$ as follows. %Assume $c \cap P= \{q_{c_1}, \dots, q_{c_\ell}\}$.
With every disjunct
$$
\posdbnfofo{\Sigma} = \bigwedge_{S\in\Sigma} \exists x. \tau^+_{S}(x) \land \forall z.( \bigvee_{S\in \Sigma} \tau^+_S(z)),
$$
%
%for some set of types $\Pi \subseteq \wp A$ and $T_i \subseteq A$,
%\fcerror{The basic form for FO was wrong before}
we associate the formula

$$
TR_1(\posdbnfofo{\Sigma}) := \bigwedge_{S\in\Sigma} \exists y. (R(x,y) \land \tau^+_{S}(y)) \land \forall z.( R(x,z) \to \bigvee_{S\in \Sigma} \tau^+_S(z)).
$$
\fcwarning{Why do we need to go through this?}The formula $TR_1(\posdbnfofo{\Sigma})$ is bisimulation invariant, and it is equivalent to the modal formula

$$
TR_2(\posdbnfofo{\Sigma}) := \bigwedge_{S\in\Sigma}  \Diamond(\bigwedge S) \land \Box \bigvee_{S\in \Sigma} (\bigwedge S).
$$
\fcwarning{Maybe better to define this directly}
Let $TR_3(\Delta(a,c))= \bigwedge (c \cap \prop) \land \bigwedge_{p \in \prop \setminus c} \lnot p \land \bigvee TR_2(\posdbnfofo{\Sigma})$.
 The modal formula $\delta_\aut(a)$ is then defined as

 \[
 \bigvee_{c \in C} TR_3(\Delta(a,c)).
 \]
By construction we have, for every $\model$,
%
%\begin{description}
%\item[(*)] 
    \begin{eqnarray*}
    \model[x\mapsto s_I] \models \bigvee_{c \in C} \big (\tau_{(c \cap \prop)}(y) \land \bigvee TR_1(\posdbnfofo{\Sigma})\big) & \text{iff} &\model \mmodels \delta_\aut(a).
    \end{eqnarray*}
%\end{description}
%\fcerror{$\model,s$ has no meaning in our setting}
%
A modal automaton over $\prop$ is an automaton $ \tup{A, \delta, \pmap, a_I}$ such that $\delta : A \to \ML^+ (A)$, where $\ML^+ (A)$ is the set of all modal formulas over propositions $A \cup \prop$ such that elements from $A$ appear only positively.
The acceptance game associated with such an automaton and a tree $\model$ is determined by the (symmetric) acceptance game defined according to the rules of Table~\ref{symmetric_modal_game}.
This means that we can equivalently see the automaton $\aut$ as a modal automaton $\tup{A, \delta_\aut, \pmap, a_I}$ whose transition function satisfies the weakness and continuity conditions. Thus, from now on we see $\aut$ as the equivalent modal automaton we have just described.%\fcwarning{Define $n_A$}

\begin{table}[h]
  \centering
\begin{tabular}{|l|c|l|c|}
 \hline
  % after \\: \hline or \cline{col1-col2} \cline{col3-col4} ...
  Position & Player & Admissible moves & Parity\\
   \hline
  $(a,s) \in A \times S$ & $\exists$ & $\{(\delta_\aut(a),s)\}$ & $\pmap(a)$\\
  $(\psi_1 \vee \psi_2,s)$ & $\exists$ & $\{(\psi_1,s),(\psi_2,s) \}$ & $-$ \\
  $(\psi_1 \wedge \psi_2,s)$ & $\forall$ & $\{(\psi_1,s),(\psi_2,s) \}$ & $-$ \\
  $(\Diamond\varphi,s)$ & $\exists$ & $\{(\varphi,t)\ |\ t \in R[s] \}$ & $-$ \\
  $(\Box\varphi,s)$ & $\forall$ & $\{(\varphi,t)\ |\ t \in R[s] \}$ & $-$ \\
  % $(\bot,s)$ & $\exists$ & $\emptyset$ & $n_A$ \\
  % $(\top,s)$ & $\forall$ & $\emptyset$ & $n_A$ \\
  $(\lnot p,s) \in \prop \times S$ and $p \notin \tscolors(s)$ & $\forall$ & $\emptyset$ & $-$\\
  $(\lnot p,s) \in \prop \times S$ and $p \in \tscolors(s)$ & $\exists$ & $\emptyset$ & $-$\\
  $(p,s) \in \prop \times S$ and $p \in \tscolors(s)$ & $\forall$ & $\emptyset$ & $-$\\
  $(p,s) \in \prop \times S$ and $p \notin \tscolors(s)$ & $\exists$ & $\emptyset$ & $-$\\

  \hline
\end{tabular}
 \caption{(Symmetric) acceptance game for modal automata}
 \label{symmetric_modal_game}
\end{table}





For the construction of $\xi_\aut$, we proceed by induction on the tree height of the DAG $t$ of SCC. If the height is 1, that is the DAG is a single point graph, we reason as follows.
 We have two cases to consider: either the SCC is trivial (i.e. it consists of a single non looping node), or not.
In the first case, $A=\{a_I\}$ and $\aut$ is equivalent to $\xi_\aut:=\delta_\aut(a_I)$.

For the second case, let us assume that $\pmap(a_I)=1$, the case when it is $0$ being, mutatis mutandis, the same.
Let $A=\{a_0, \dots, a_\ell\}$, and $a_I=a_\ell$.
Since $\aut$ is a weak modal automaton satisfying the continuity condition, given $a,b \in A$, if $b$ occurs in $\delta_\aut(a)$, then $b$ is only in the scope of $\Diamond$ operator. 
We can now see the automaton $\bbA$
as a system of modal equations, and by applying the standard inductive procedure \emph{solves} this system of equations and
construct the least fixpoint formula $\xi_\aut$ equivalent to $\aut$, 

%Now, since every variable bounded by a $\mu$-operator  in the obtained formula  is in the scope only of $\Diamond$ operators too. 
%The refinement that our setting requires is that if $\bbA$ is an automaton 
%in $\yvAut(\ofo)$, then the solution of the associated system of equations 
%can be obtained as a formula in the fragment $\yvF \sse \MC$.
%
Here the key observation is that the weakness and continuity conditions on
strongly connected components of the automaton ensure that when we execute
a single step in solving the system of equations, we 
may work within the 
(syntactically) continuous fragment of the modal $\mu$-calculus.
From this, we can deduce that $\xi_\aut \in \contAFMC$. Clearly the procedure preserves the polarity of each $x \in X$, meaning that all variables in $X$ are positive in $\xi_\aut$.


For the induction step, assume the successors of the root of $t$ are $(n_1, \dots, n_\ell)$. For each $i \in \{1,\dots,\ell\}$, let $a_i$ be the entrance point of the SCC of $n_i$, and let $\aut_{i}$ be the automaton $\aut$ but having as an initial state $a_i$. If we do not consider the states that are not reachable by $a_i$, the DAG of the SCC of $\aut_{i}$ is $t.{n_i}$ (the subtree of $t$ starting at $n_i$).
Let $Y=\{a_1, \dots, a_\ell\}$, and $M$ be the set of states of $\aut$ that belong to the root of $t$. We assume $X \cap Y = \emptyset$. The structure
\[
\aut_M := \tup{M, \delta_\aut|_{M}, \pmap|_{M}, a_I}
\] is a $(\prop, X \cup Y)$-automaton.

The inductive hypothesis applies to automata $\aut_M, \aut_{1}, \dots, \aut_{\ell}$. Thus we obtain fixpoint formulas $\xi_M, \xi_1, \dots, \xi_\ell$, the former taking free variables in $\prop \cup X \cup Y$, all the remaining in $\prop \cup X$, equivalent to $\aut_M, \aut_1, \dots, \aut_\ell$ respectively.
Notice that:
\begin{itemize}
\item by induction hypothesis, $\xi_M, \xi_1, \dots, \xi_\ell \in \contAFMC$, every variable $x \in X \cup Y$ is positive in each of those formulas (if $x$ occurs in it),
\item by construction, if $x$ is free in $\xi_i$, then $x$ is not bounded in $\xi_{M}$.
\end{itemize}
 We can therefore deduce that
 $\xi_\aut= \xi_{M}[a_1\mapsto\xi_{1}, \dots, a_\ell\mapsto\xi_{\ell}] \in \contAFMC$, and that each variable from $X$ occurs positively in $\xi_\aut$.

 We verify that $\model \mmodels \xi_{M}[a_1\mapsto\xi_{1}, \dots, a_\ell\mapsto\xi_{\ell}]$ iff $\model \in \trees(\aut)$, for every model $\model$.
 %\fcwarning{I think the $(\model,s)$ thould be $\model$ or rather a variant of $\model$ with a changed point}
But this follows by the following two facts:
\begin{enumerate}
\item $\model\mmodels \xi_{M}[a_1\mapsto\xi_{1}, \dots, a_\ell\mapsto\xi_{\ell}]$ iff $\model[a_1\mapsto \ext{\xi_1}^{\model}, \dots, a_\ell\mapsto \ext{\xi_\ell}^{\model}  ] \mmodels \xi_M$,
\item $\model \in \trees(\aut)$  iff $\model[a_1\mapsto \ext{\aut_1}^{\model}, \dots, a_\ell\mapsto \ext{\aut_\ell}^{\model}  ] \in \trees(\aut_M)$
\end{enumerate}
where %$\ext{\xi_i\ext{_{\model}=\{ s \in \model \mid s \in \ext{\xi_i\ext{^\model \}$, and 
$\ext{\aut_i}^{\model} := \{ s \in \model \mid \model.s \in \trees(\aut_i) \}$. 
\end{pfclaim}
%Since $\aut$ was a weak automaton, we can thus use Lemma 2.4(1) of AlbeFac JSL and conclude that $\xi_{M}[a_1\mapsto\xi_{1}, \dots, a_\ell\mapsto\xi_{\ell}]$ is equivalent to $\aut$.
This finishes the proof of Theorem. 
\end{proof}
