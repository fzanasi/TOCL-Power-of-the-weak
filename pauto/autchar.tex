In this subsection we prove that $\wmso$-automata are closed under the 
operations corresponding to the connectives of $\mso$, that is: union, 
complementation and projection with respect to finite sets.
We start with the latter.

%%%%
%%%% PROJECTION
%%%%



\subsubsection{Closure under Finitary Projection}

\begin{definition}\label{DEF_fin_projection}
Let $\aut = \tup{A, \Delta, \Omega, a_I}$ be a $\wmso$-automaton on alphabet $\p(\prop \cup \{p\})$. Let $\aut^{\f}$
denote its finitary construct.
We define the automaton ${{\exists}_F p}.\mb{A} = \langle A^{\f}, a_I^{\f},
\DeltaProj, \Omega^{\f}\rangle$ on alphabet $\p\prop$ by putting
\begin{eqnarray*}
% \nonumber to remove numbering (before each equation)
  \DeltaProj(a,c) &:=& \Delta^{\f}(a,c)\\
  \DeltaProj(R,c) &:=& \Delta^{\f}(R,c) \vee \Delta^{\f}(R,c\cup\{p\}).
\end{eqnarray*}
The automaton ${{\exists}_F p}.\mb{A}$ is called the \emph{finitary projection
construct of $\mb{A}$ over $p$}.
\end{definition}

Our projection construction corresponds to a suitable closure operation on tree languages, modeling the semantics of $\wmso$ existential quantification.

\begin{definition}\label{def:tree_finproj} Let $p$ be a propositional letter and $L$ a tree language of $\p (\prop\cup\{p\})$-labeled trees. The \emph{finite projection} of $L$ over $p$ is the language ${\exists}_F p.L$ of $C$-labeled trees defined as
\begin{equation*}
    {\exists}_F p.L = \{\model \mid \text{there is a $p$-variant } \model[p\mapsto S] \text{ of } \model \text{ such that } \model[p\mapsto S] \in L \text{ and } S \text{ is finite} \}.
\end{equation*}\hfill
\end{definition}

\begin{lemma}\label{PROP_fin_projection}
For each $\wmso$-automaton $\aut$ on alphabet $\p (\prop \cup \{p\})$,
we have that
$$\trees({{\exists}_F p}.\mb{A}) \ \equiv\
{{\exists}_F p}.\trees(\mb{A}).
$$
\end{lemma}

\begin{proof}
\fcwarning{Showing this for $\aut^F$ and using Lemma~\ref{PROP_facts_finConstr}(3) leads to a way simpler proof.}What we need to show is that for any tree $\model$:
\begin{eqnarray*}
  {{\exists}_F p}.\mb{A} \text{ accepts } \mathbb{T} & \text{ iff }& \text{there is a finite $p$-variant }\model' \\
   & & \text{of }\mathbb{T}\text{  such that }\aut\text{  accepts }\model'.
\end{eqnarray*}
For direction from left to right, we first observe that the properties stated by Lemma~\ref{PROP_facts_finConstr} hold for ${{\exists}_F p}.\mb{A}$ as well, since the latter is defined in terms of $\mb{A}^{\f}$. Then we can assume that the given winning strategy $f$ for $\exists$ in $\mc{G_{\exists}} = \mc{A}({{\exists}_F p}.\mb{A},\model)@(a_I^{\f},s_I)$ is functional and finitary in $\shA$. Functionality allows us to associate with each node $s$ either none or a unique state $Q_s \in \shA$ (\emph{cf.} \cite[Prop. 3.12]{Zanasi:Thesis:2012}). We now want to isolate  the nodes that $f$ treats ``as if they were labeled with $p$''. For this purpose, let $\val_{s}$ be the valuation suggested by $f$ from a position $(Q_s,s) \in \shA \times T$. As $f$ is winning, $\val_{s}$ makes $\DeltaProj(Q,\tscolors(s))$ true in $\R{s}$. We define a $p$-variant $\model'$ of $\model$ by \fcwarning{Why the tilde?}coloring with $p$ all nodes in the following set:
 \begin{equation}\label{eq:X_p}
% \nonumber to remove numbering (before each equation)
   X_p\ :=\ \{s \in T\mid (\R{s},\widetilde{\val}_{s}) \models \Delta^{\f}(Q_s,\tscolors(s)\cup\{p\})\}.
\end{equation}
The fact that the strategy of $\exists$ is finitary in $\shA$ guarantees that $X_p$ is finite, whence $\model'$ is a finite $p$-variant. The argument showing that $\mb{A}^{\f}$ (and thus also $\mb{A}$, by Lemma~\ref{PROP_facts_finConstr}(1))\fcwarning{isn't this (3)?} accepts $\model'$ is a routine adaptation of the analogous proof for the noetherian projection of weak $\mso$-automata, for which we refer to \cite[Prop. 3.12]{Zanasi:Thesis:2012}.
\medskip

For the direction from right to left, let $\model'$ be a finite $p$-variant of
$\model$, with labeling function $\tscolors'$, and $g$ a winning strategy for $\exists$ in $\mc{G} = \mathcal{A}(\aut,\model')@(a_I,s_I)$. Our goal is to define a strategy $g'$ for $\exists$ in $\mc{G_{\exists}}$. As usual, $g'$ will be constructed in stages, while playing a match $\pi'$ in $\mc{G_{\exists}}$. In parallel to $\pi'$, a \emph{bundle} $\mc{B}$ of $g$-guided shadow matches in $\mc{G}$ is maintained, with the following condition enforced for each round $z_i$ (\emph{cf.} \cite[Prop.~ 3.12]{Zanasi:Thesis:2012}) :
\smallskip
\begin{center}
\fbox{\parbox{13cm}{
\begin{enumerate}
  \item If the current (i.e. at round $z_i$) basic position in $\pi'$ is of the form $(Q,s) \in \shA \times T$, then for each $a \in\Ran(Q)$ there is an $g$-guided (partial) shadow match $\pi_a$ at basic position $(a,s) \in A\times T$ in the current bundle $\mc{B}_i$. Also, either $\model'_s$ is not $p$-free (i.e., it does contain a node $s'$ with $p \in \tscolors'(s')$) or $s$ has some sibling $t$ such that $\model'_t$ is not $p$-free.
  \item Otherwise, the current basic position in $\pi'$ is of the form $(a,s) \in A \times T$ and $\model'_s$ is $p$-free (i.e., it does not contain any node $s'$ with $p \in \tscolors'(s')$). Also, the bundle $\mc{B}_i$ only consists of a single $g$-guided match $\pi_a$ whose current basic position is also $(a,s)$.
\end{enumerate}
}}\hspace*{0.3cm}($\ddag$)
\end{center}
\smallskip
We briefly recall the idea behind condition ($\ddag$). Point ($\ddag.1$) describes the part of match $\pi'$ where it is still possible to encounter nodes which are labeled with $p$ in $\model'$. As $\DeltaProj$ only takes the letter $p$ into account when defined on macro-states in $\shA$, we want $\pi'$ to visit only positions of the form $(R,s) \in \shA \times T$ in that situation. Anytime we visit such a position $(R,s)$ in $\pi'$, the role of the bundle is to provide one $g$-guided shadow match at position $(a,s)$ for each $a \in \Ran(R)$.
Then $g'$ is defined in terms of what $g$ suggests from those positions.

 Point ($\ddag.2$) describes how we want the match $\pi'$ to be
 played on a $p$-free subtree: as any node that one might encounter has the same label in $\model$ and $\model'$,
it is safe to let ${{\exists}_F p}.\mb{A}$ behave as $\aut$ in such situation. Provided that the two matches visit the same basic positions, of the form $(a,s)\times A \times T$, we can let $g'$ just copy $g$.

The key observation is that, as $\model'$ is a \emph{finite} $p$-variant of $\model$, nodes labeled with $p$ are reachable only for finitely many rounds of $\pi'$. This means that, provided that ($\ddag$) hold at each round, ($\ddag.1$) will describe an initial segment of $\pi'$, whereas ($\ddag.2$) will describe the remaining part. Thus our proof that $g'$ is a winning strategy for $\exists$ in $\mc{G}_{\exists}$ is concluded by showing that ($\ddag$) holds for each stage of construction of $\pi'$ and $\mc{B}$.

\medskip

For this purpose, we initialize $\pi'$ from position $(\shai,s) \in \shA\times T$ and the bundle $\mc{B}$ as $\mc{B}_0 = \{\pi_{a_I}\}$, with $\pi_{a_I}$ the partial $g$-guided match consisting only of the position $(a_I,s)\in A\times T$. The situation described by ($\ddag .1$) holds at the initial stage of the construction.
Inductively, suppose that at round $z_i$ we are given a position $(q,s) \in A^{\f} \times T$ in $\pi^{\f}$ and a bundle $\mc{B}_i$ as in ($\ddag$). To show that ($\ddag$) can be maintained at round $z_{i+1}$, we distinguish two cases, corresponding respectively to situation ($\ddag.1$) and ($\ddag.2$) holding at round $z_i$.
\begin{enumerate}[label = (\Alph*), ref = \Alph*]
%\yvwarning{Notation `$q$' is confusing, see $\val'(q)$ below FZ: I corrected $q$ into $q'$ below}
  \item If $(q,s)$ is of the form $(Q,s) \in \shA \times T$, by inductive hypothesis we are given with $g$-guided shadow matches $\{\pi_a\}_{a \in \Ran(Q)}$ in $\mc{B}_i$. For each match $\pi_a$ in the bundle, we are provided with a valuation $\val_{a,s}: A \rightarrow \p (\R{s})$ making $\Delta(a,\tscolors'(s))$ true. Then we further distinguish the following two cases.
\begin{enumerate}[label = (\roman*), ref = \roman*]
  \item \label{point:TsNotPFree} Suppose first that $\model'_s$ is not $p$-free. We let the suggestion $\val' \: A^{\f} \to \p (\R{s})$ of $g'$ from position $(Q,s)$ be defined as follows:
       \begin{align*}
       % \nonumber to remove numbering (before each equation)
       %\widetilde{\val}_{Q,s}(Q') &:=& \bigcup_{a \in \Ran(Q),\ b \in \Ran(Q')}\{t\ \in \R{s}|\ t \in \val_{a,s}(b)\}.
       \val'(q')\ :=\ \begin{cases}
               \bigcap\limits_{\substack{(a,b) \in q',\\ a \in \Ran(Q)}}\{t\ \in \R{s} \mid t \in \val_{a,s}(b)\}               & q' \in \shA \\[2em]
               \bigcup\limits_{a \in \Ran(Q)} \{t\ \in \R{s} \mid t \in \val_{a,s}(q') \text{ and }\model'.t\text{ is $p$-free}\}              & q' \in A.
               %\\[1.5em]               \hspace{.6cm}\emptyset & \text{otherwise.}
           \end{cases}
       \end{align*}
       The definition of $\val'$ on $q' \in \shA$ is standard (\emph{cf.}~\cite[Prop. 2.21]{Zanasi:Thesis:2012}) and guarantees a correspondence between the states assigned by the markings $\{\val_{a,s}\}_{a \in \Ran(Q)}$ and the macro-states assigned by $\val'$. The definition of $\val'$ on $q' \in A$ aims at fulfilling the conditions, expressed via $\qu$ and $\dqu$, on the number of nodes in $\R{s}$ witnessing (or not) some $A$-types. Those conditions are the ones that $\shDe(Q,\tscolors'(s))$ --and thus also $\Delta^{\f}(Q,\tscolors'(s))$-- ``inherits'' by $\bigwedge_{a \in \Ran(R)} \Delta(a,\tscolors'(s))$, by definition of $\shDe$. Notice that we restrict $\val'(q')$ to the nodes $t \in \val_{a,s}(q')$ such that $\model'.t$ is $p$-free. As $\model'$ is a \emph{finite} $p$-variant, only \emph{finitely many} nodes in $\val_{a,s}(q')$ will not have this property. Therefore their exclusion, which is crucial for maintaining condition ($\ddag$) (\emph{cf.}~case \eqref{point:ddag2CardfromMacro} below), does not influence the fulfilling of the cardinality conditions expressed via $\qu$ and $\dqu$ in $\shDe(Q,\tscolors'(s))$.

       On the base of these observations, one can check that $\val'$ makes $\shDe(Q,\tscolors'(s))$--and thus also $\Delta^{\f}(Q,\tscolors'(s))$--true in $\R{s}$. In fact, to be a legitimate move for $\exists$ in $\pi'$, $\val'$ should make $\DeltaProj(Q,\tscolors(s))$ true: this is the case, for $\Delta^{\f}(Q,\tscolors'(s))$ is either equal to $\Delta^{\f}(Q,\tscolors(s))$, if $p \not\in \tscolors'(s)$, or to $\Delta^{\f}(Q,\tscolors(s)\cup\{p\})$ otherwise. In order to check that we can maintain $(\ddag)$, let $(q',t) \in A^{\f} \times T$ be any next position picked by $\forall$ in $\pi'$ at round $z_{i+1}$. As before, we distinguish two cases:
       \begin{enumerate}[label = (\alph*), ref = \alph*]
         \item If $q'$ is in $A$, then, by definition of $\val'$, $\forall$ can choose $(q',t)$ in some shadow match $\pi_a$ in the bundle $\mc{B}_i$. We dismiss the bundle --i.e. make it a singleton-- and bring only $\pi_a$ to the next round in the same position $(q',t)$. Observe that, by definition of $\val'$, $\model'.t$ is $p$-free and thus ($\ddag.2$) holds at round $z_{i+1}$. \label{point:ddag2CardfromMacro}
         \item Otherwise, $q'$ is in $\shA$. The new bundle $\mc{B}_{i+1}$ is given in terms of the bundle $\mc{B}_i$: for each $\pi_a \in \mc{B}_i$ with $a\in \Ran(Q)$, we look if for some $b \in \Ran(q')$ the position $(b,t)$ is a legitimate move for $\forall$ at round $z_{i+1}$; if so, then we bring $\pi_a$ to round $z_{i+1}$ at position $(b,t)$ and put the resulting (partial) shadow match $\pi_b$ in $\mc{B}_{i+1}$. Observe that, if $\forall$ is able to pick such position $(q',t)$ in $\pi'$, then by definition of $\val'$ the new bundle $\mc{B}_{i+1}$ is non-empty and consists of an $g$-guided (partial) shadow match $\pi_b$ for each $b \in \Ran(q')$. In this way we are able to keep condition ($\ddag.1$) at round $z_{i+1}$.
       \end{enumerate}
    \item Let us now consider the case in which $\model'_s$ is $p$-free. We let $g'$ suggest the valuation $\val'$ that assigns to each node $t \in \R{s}$ all states in $\bigcup_{a \in \Ran(Q)}\{b \in A\ |\ t \in \val_{a,s}(b)\}$. It can be checked that $\val'$ makes $\bigwedge_{a \in \Ran(Q)} \Delta(a,\tscolors'(s))$ -- and then also $\Delta^{\f}(Q,\tscolors'(s))$ -- true in $\R{s}$. As $p \not\in \tscolors(s)=\tscolors'(s)$, it follows that $\val'$ also makes $\DeltaProj(Q,\tscolors(s))$ true, whence it is a legitimate choice for $\exists$ in $\pi'$. Any next basic position picked by $\forall$ in $\pi'$ is of the form $(b,t) \in A \times T$, and thus condition ($\ddag.2$) holds at round $z_{i+1}$ as shown in (i.a). %\eqref{point:ddag2CardfromMacro}
  \end{enumerate}
  \item In the remaining case, $(q,s)$ is of the form $(a,s) \in A \times T$ and by inductive hypothesis we are given with a bundle $\mc{B}_i$ consisting of a single $f$-guided (partial) shadow match $\pi_a$ at the same position $(a,s)$. Let $\val_{a,s}$ be the suggestion of $\exists$ from position $(a,s)$ in $\pi_a$. Since by assumption $s$ is $p$-free, we have that $\tscolors'(s) = \tscolors(s)$, meaning that $\DeltaProj(a,\tscolors(s))$ is just $\Delta(a,\tscolors(s)) = \Delta(a,\tscolors'(s))$. Thus the restriction $\val'$ of $\val$ to $A$ makes $\Delta(a,\tscolors'(t))$ true and we let it be the choice for $\exists$ in $\tilde{\pi}$. It follows that any next move made by $\forall$ in $\tilde{\pi}$ can be mirrored by $\forall$ in the shadow match $\pi_a$.
      \begin{comment}Version with minimality:
      It follows that $\DeltaProj(a,\tscolors(t))$ is just $\Delta(a,\tscolors(t)) = \Delta(a,\tscolors'(t))$ and the same valuation suggested by $f$ in $\pi_a$ is a legitimate choice for $\exists$ in $\tilde{\pi}$. By letting $\exists$ choose such valuation, it follows that any next move made by $\forall$ in $\tilde{\pi}$ can be mirrored by $\forall$ in the shadow match $\pi_a$.
      \end{comment}
\end{enumerate}

%As explained above, since $\model'$ is a noetherian $p$-variant, then ($\ddag .1$) holds for finitely many stages of construction of $\tilde{\pi}$, whereas ($\ddag .2$) holds for all the remaining stages, by construction of $\tilde{f}$. It follows that this strategy is winning for $\exists$ in $\tilde{G}$.

\begin{comment} OLD (22 October), SHORTER VERSION
For direction from right to left, let $g$ be a winning strategy for $\exists$ in $\mc{G} = \mathcal{A}(\aut,\model)@(a_I,s_I)$. Our goal is to define a strategy $g'$ for $\exists$ in $\mc{G_{\exists}}$. As usual, it will be constructed in stages, while playing a match $\pi'$ in $\mc{G_{\exists}}$. In parallel to $\pi'$, a \emph{bundle} of shadow matches in $\mc{G}$ is maintained, where in each of them $\exists$ plays according to $g$. The main idea is the same of \cite[Prop. 3.12]{Zanasi:Thesis:2012}: we force ${{\exists}_F p}.\mb{A}$ to stay in macro-states from $\shA$ until a subtree of $\model'$ without nodes labeled with $p$ (which we call \emph{$p$-free}) is reached. From that stage onwards, we can safely allow ${{\exists}_F p}.\mb{A}$ to behave as $\mb{A}$.
To this aim, it suffices to show that, at each round $z_i$ in $\pi'$ where a basic position of the form $(R,s) \in \shA \times T$ occurs, we can maintain the following condition:
\begin{enumerate}
  \item  for each $a \in\Ran(Q)$ there is an $f$-guided (partial) shadow match $\pi_a$ in the bundle with current (i.e. at round $z_i$) basic position $(a,s) \in A\times T$.
\end{enumerate}
This holds for round $z_0$, where we initialize $\pi$ at position $(a_I^{\f},s_I)$ and the bundle as consisting of a single shadow match $\pi_{a_I}$ at position $(a_I,s_I)$. Inductively, suppose that we are given with a basic position $(R,s)$ in $\pi'$ at some round $z_i$, and a bundle $\{\pi_a\}_{a \in \Ran(R)}$ as above. As each match in the bundle is $f$-guided, we are provided with markings $\{m_a\}_{a \in \Ran(R)}$ making the corresponding sentences true in $\sigma_R(s)$.

XXXX To be continued XXXX
\end{comment}
\begin{comment} OLDER (Summer) version
%For direction from right to left, by Claim \ref{PROP_facts_finConstr}(1) we can assume that $\mb{A}^{\f}$ accepts a finite $p$-variant of $\model$. The key observation is the following: the winning strategy for $\exists$ in the associated game can be easily shown to be winning also in $\mc{A}({{\exists}_F p}.\mb{A},\model)@(a_I^{\f},s_I)$, provided that $\mb{A}^{\f}$ visits nodes labeled with $p$ only in macro-states from $\shA$. This assumption can be proven by exploiting the fact that the $p$-variant we are considering is finite. Further details on this will be worked out at a later stage of this draft.
\end{comment}

\end{proof} 

%%%%%%
%%%%%% BOOLEANS
%%%%%%

\subsubsection{Closure under Boolean operations}

In this section we will show that the class of $\wmso$-automaton recognizable
tree languages is closed under the Boolean operations.
%
Start with closure under union, we just mention the following result, without
providing the (completely routine) proof.

\begin{theorem}
\label{t:cl-dis}
Let $\bbA_{0}$ and $\bbA_{1}$ be $\wmso$-automata. 
Then there is a $\wmso$-automaton $\bbA$ such that $\trees(\bbA)$ is the 
union of $\trees(\bbA_{0})$ and $\trees(\bbA_{1})$.
\end{theorem}

In order to prove closure under complementation, we crucially use that the 
one-step language $\olque$ is closed under Boolean duals.

\myparagraphns{Closure under complementation.}
Many properties of parity automata can already be determined at the one-step level.
An important example concerns the notion of complementation.


\begin{definition}
\label{d:bdual1}
Two one-step formulas $\varphi$ and $\psi$ are each other's \emph{Boolean dual}
if for every structure $(D,\val)$ we have:
\[
(D,\val) \models \varphi \quad\text{iff}\quad (D,\val^{c}) \not\models \psi,
\]
where $\val^{c}$ is the valuation given by $\val^{c}(a) \mathrel{:=} D
\setminus \val(a)$, for all $a$.
%
A one-step language $\llang$ is \emph{closed under Boolean duals} if for every
set $A$, each formula $\varphi \in \llang(A)$ has a Boolean dual $\dual{\varphi}
\in \llang(A)$.
\end{definition}

Following ideas from~\cite{Muller1987,DBLP:conf/calco/KissigV09}, we can use Boolean duals, together with a
\emph{role switch} between $\abelard$ and $\eloise$, in order to define a
negation or complementation operation on automata.

\begin{definition}
\label{d:caut}
Assume that, for some one-step language $\llang$, the map $\dual{(-)}$
provides, for each set $A$, a Boolean dual $\dual{\varphi} \in \llang(A)$ for each
$\varphi \in \llang(A)$.
Given $\aut = \tup{A,\tmap,\pmap,a_I}$ in $\Aut(\llang)$ we define its
\emph{complement} $\dual{\aut}$ as the automaton
$\tup{A,\dual{\tmap},\dual{\pmap},a_I}$
where $\dual{\tmap}(a,c) := \dual{(\tmap(a,c))}$, and $\dual{\pmap}(a)
:= 1 + \pmap(a)$, for all $a \in A$ and $c \in \wp(\props)$.
\end{definition}

\begin{proposition}
\label{prop:autcomplementation}
Let $\llang$ and $\dual{(-)}$ be as in the previous definition.
For each automaton $\aut \in \Aut(\llang)$ and each transition system
$\model$ we have that
\[
\dual{\aut} \text{ accepts } \model
\quad\text{iff}\quad
\aut \text{ rejects } \model.
\]
\end{proposition}

The proof of Proposition~\ref{prop:autcomplementation} is based on the fact
that the power of $\eloise$ in $\agame(\dual{\aut},\model)$ is the same
as that of $\abelard$ in $\agame(\aut,\model)$, as defined in~\cite{DBLP:conf/calco/KissigV09}.

As an immediate consequence of this proposition, one may show that if the
one-step language $\llang$ is closed under Boolean duals, then the class
$\Aut(\llang)$ is closed under taking complementation.
Further on we will use Proposition~\ref{prop:autcomplementation} to show that
the same may apply to some subclasses of $\Aut(\llang)$.


\begin{theorem}
\label{t:cl-cmp}
Let $\bbA$ be an $\wmso$-automaton.
Then the automaton $\overline{\aut}$ defined in Definition~\ref{d:caut} is a
$\wmso$-automaton recognizing the complement of $\trees(\bbA)$.
\end{theorem}

\begin{proof}
Since we already know that $\overline{\bbA}$ accepts exactly the transition
systems that are rejected by $\bbA$, we only need to check that 
$\overline{\bbA}$ is indeed a $\wmso$-automaton.
But this is straightforward: for instance, continuity can be checked by 
observing the self-dual nature of this property.
\end{proof}


%%%%
%%%% PROOF THEOREM
%%%%

\subsection{Proof of Theorem \ref{t:wmsoauto}}

\begin{proof} The proof is by induction on $\varphi$.
\begin{itemize}
  \item For the base case $\varphi = p \inc q$, the corresponding 
  $\wmso$-automaton is provided in \cite[Ex. 2.6]{Zanasi:Thesis:2012}. 
  For the base case $\varphi = R(p,q)$, we give the corresponding 
  $\wmso$-automaton $\aut_{R(p,q)} = \tup{A,\Delta,\Omega,a_I}$ below:
\begin{eqnarray*}
        A &:=& \{a_0,a_1\}\\
        a_I &:=& a_0\\
  \Delta(a_0,c) &:=& \left\{
	\begin{array}{ll}
           \exists x. a_1(x) \wedge \forall y. a_0(y) & \mbox{if }p \in c 
	\\ \forall x\ (a_0(x)) & \mbox{otherwise}
	\end{array}
\right. \\
  \Delta(a_1,c) &:=& \left\{
	\begin{array}{ll}
        \top & \mbox{if }q \in c \\
        \bot & \mbox{otherwise}
	\end{array}
\right. \\
    \Omega(a_0) &:=& 0\\
    \Omega(a_1) &:=& 1.
\end{eqnarray*}
Note that the $\mso$-automaton for $R(p,q)$ provided in 
\cite[Ex. 2.5]{Zanasi:Thesis:2012} is \emph{not} a $\wmso$-automaton, as the 
continuity property does not hold.

\item
For the Boolean cases, where $\varphi = \psi_1 \vee \psi_2$ or $\phi = \neg\psi$
we refer to the closure properties of recognizable tree languages, see 
Theorem~\ref{t:cl-dis} and Theorem~\ref{t:cl-cmp}, 
respectivel.
  
\item 
For the case $\varphi = \exists p. \psi$, consider the following chain of
equivalences, where $\aut_{\psi}$ is given by the inductive hypothesis and 
${{\exists}_F p}.\aut_{\psi}$ is constructed according to 
Definition \ref{DEF_fin_projection}:
\begin{alignat*}{2}
{{\exists}_F p}.\aut_{\psi} \text{ accepts }\mb{T} 
   & \text{ iff }
     \aut_{\psi} \text{ accepts } \mb{T}[p \mapsto X], 
     \text{ for some } X \sse_{\om} T
   & \quad\text{(Lemma~\ref{PROP_fin_projection})}
\\ & \text{ iff }
     \mb{T}[p \mapsto X] \models \psi,
     \text{ for some } X \sse_{\om} T
   & \quad\text{(induction hyp.)}
\\ & \text{ iff }
    \mb{T} \models \exists p. \psi
   & \quad\text{(semantics $\wmso$)}
\end{alignat*}
\end{itemize}
\end{proof}

