%!TEX root = ../00CFVZ_TOCL.tex
In this section we work with the members of the class $\AutW(\ofoe)$, which we henceforth call \emph{$\nmso$-automata}.


\begin{theorem}
\label{t:nmsoauto}
There is an effective construction transforming a $\nmso$-formula $\phi$
into a $\nmso$-automaton $\bbA_{\phi}$ that is equivalent
to $\phi$ on the class of trees.
That is, for any tree $\bbT$, $\bbA_{\phi}$ accepts $\bbT$ if and only if $\bbT \models {\phi}$
\end{theorem}

The rest of this section will be devoted to prove that $\nmso$-automata
characterise $\nmso$ on tree models, as expressed in Theorem~\ref{t:mt2}.
First, we focus on showing the direction from formulas to automata.
In subsections~\ref{sec:finitconstr} and \ref{sec:closureautomata} we provide
the automata constructions handling the challenging case, that is the
translation of an existential formula $\noetexists p.\psi$ of $\nmso$ into an
equivalent $\nmso$-automaton.
To this aim, we define a closure operation on tree languages corresponding
to the semantics of $\nmso$ quantification.

\begin{definition}\label{def:tree_finproj}
Let $\prop$ be a set of proposition letters, $p \not\in P$ be a proposition letter, and $\trees$ be a tree language of $\p (\prop\cup\{p\})$-labeled
trees.
The \emph{noetherian projection} of $\trees$ over $p$ is the language
$\noetexists p.\trees$ of $\p (\prop)$-labeled trees %$\model$ for which some
%noetherian $p$-variant of $\model$ exists in $\trees$.
given as follows:
%
$$
\noetexists p.\trees = \{\model \mid \text{ $\exists$ a noetherian $p$-variant } \model' \text{ of } \model \text{ with } \model' \in \trees\}.
$$
%
A class $K$ of tree languages is \emph{closed under noetherian projection
over $p$} if, for any language $\trees$ in $K$, also ${\noetexists p}.\trees$ is in $K$.
% A class $K$ is \emph{closed under noetherian projection
% over $p$} if $\trees\in K$ implies ${{\exists}_F p}.\trees \in K$.
\end{definition} 

\subsubsection{Simulation theorem}


Our next goal is a \emph{projection construction} that, given
a $\nmso$-automaton $\aut$, provides one recognizing ${\noetexists p}.\trees(\aut)$. For $\mso$-automata, an analogous construction crucially uses the following \emph{simulation theorem}: every
$\mso$-automaton $\aut$ is equivalent to a \emph{non-deterministic} one $\aut'$ \cite{Walukiewicz96}. Semantically, non-determinism yields the appealing property that any strategy $f$ for player $\exists$ in the acceptance game $\agame(\aut',\model)$ can be assumed to be functional in $A'$ (\emph{cf.} Definition \ref{def:partityaut}).
This is particularly helpful because, to define a $p$-variant of $\model$
that is accepted by the projection construct on $\aut'$, we
can infer whether a node $s$ should be labeled with $p$ by the value $f(a,s)$, where $a$ is the unique state of $\aut'$ (by functionality) that $f$ associates with $s$.

The simulation theorem for $\mso$-automata does not restrict to the setting of $\nmso$-automata, as it does not preserve the \emph{(weakness)} condition. In the sequel, we are going to show that a restricted version of non-determinism suffices for our purposes. Indeed, guessing a \emph{noetherian} $p$-variant of the input tree, as prescribed by $\nmso$ quantification, only requires a winning strategy $f$ to be functional for finitely many rounds of the acceptance game. Thus the idea of our simulation theorem is to turn any $\nmso$-automaton into an equivalent one that behaves non-deterministically on a \emph{well-founded} portion of any accepted tree.

For $\mso$-automata, the simulation theorem is based on a powerset construction: if the starting automaton has carrier $A$, the resulting non-deterministic automaton is based on ``macro-states'' from the set $\shA := \pw (A \times A)$.\footnote{The use of carrier $\pw (A \times A)$ instead of the more obvious $\pw A$ is needed to correctly associate with a run on macro-states the corresponding bundle of runs of the original automaton $\aut$ (\emph{cf.} \cite{Walukiewicz96}).} Analogously, for $\nmso$-automata we will associate the non-deterministic behaviour with macro-states. As explained, our desiderata are that the simulating automating $\aut^{\noet}$ is non-deterministic just on a well-founded portion of the input and may behave as $\aut$ (i.e. in ``alternating mode'') on the others. To this aim, $\aut^{\noet}$ will be ``two-sorted'', roughly consisting of one copy of $\aut$ (with carrier $A$) and a variant of its powerset construction, based both on $A$ and $\shA$. For any accepted $\model$, the idea is to make any match $\pi$ of $\agame(\aut^{\noet},\model)$ consist of two parts:
\begin{description}
  \item[(\textbf{Non-deterministic mode})] for finitely many rounds, $\pi$ only visits macro-states (from $\shA$) and $\eloise$ plays according to a functional strategy, meaning that she assigns \emph{at most one macro-state} (and no regular state, from $A$) to each node.%each visited basic position has shape $(q,s) \in \shA \times T$. The valuation $\val \colon A \cup \shA \to \pw (R[s])$ picked by player $\exists$ assigns macro-states (from $\shA$) only to a \emph{finite} subset of $\R{s}$ and states (from $A$) to the rest of $\R{s}$. Also, she assigns \emph{at most one macro-state} to each node.
  \item[(\textbf{Alternating mode})] At a certain round, $\pi$ abandons macro-states and turns into a match of the game $\mc{A}(\aut,\model)$, i.e. all next positions are from $A \times T$ (and are played according to a non-necessarily functional strategy). %of shape $(a,t) \in A \times T$.
\end{description}
Therefore successful runs of $\mb{A}^{\noet}$ will have the property of processing only a \emph{well-founded} amount of the input with $\mb{A}^{\noet}$ being in a macro-state and all the rest with $\mb{A}^{\noet}$ behaving exactly as $\aut$.

We now proceed in steps towards the construction of $\aut^{\noet}$. The following is a notion of lifting for types on states that is instrumental in defining a translation to types on macro-states. %The distinction between empty and non-empty subsets of $A$ is to make sure that empty types on $A$ are lifted to empty types on $\pw A$.
\begin{definition}
Given a set $A$ and $\Sigma \subseteq \wp A$, we define the \emph{lifting} $\lift{\Sigma} \subseteq \wp \wp A$ as $\{\{S\} \mid S \in \Sigma \wedge S \neq \emptyset\} \cup
    \{\emptyset \mid \emptyset \in \Sigma \}$.
%\begin{eqnarray*}
%\lift{\Sigma} & := & \{\{S\} \mid S \in \Sigma \wedge S \neq \emptyset\} \cup    \{\emptyset \mid \emptyset \in \Sigma \}.
%\end{eqnarray*}
\end{definition}

 The next definition is standard (see e.g.  \cite{Walukiewicz96,Ven08}) as an intermediate step to define the transition function of the powerset construct for parity automata. It simply \emph{tags} the (potential next) states occurring in $\tmap(a,c)$ with the information of the current state.

\begin{definition}\label{DEF_delta star} Let $\aut = \tup{A,\tmap,\pmap,a_I}$ be a $\nmso$-automaton. Fix $a \in A$, $c \in C$. The sentence $\tmap^{\star}(a,c) \in {\ofoe}^+(A\times A)$ is defined as $\tmap(a,c)[b \mapsto (a,b) \mid b \in A]$, where $\tmap(a,c)[b \mapsto (a,b) \mid b \in A] \in {\ofoe}^+(A\times A)$ is the sentence obtained by replacing each monadic predicate $b \in A$ in $\tmap(a,c)$ with the monadic predicate $(a,b) \in A \times A$.
\end{definition}

We now lift the one-step language of $\nmso$-automata from states (that we suppose in $A \times A$, by effect of Definition \ref{DEF_delta star}) to macro-states (in $\shA = \pw (A \times A)$).

\begin{definition}\label{DEF_finitary_lifting}
Let $\varphi \in {\ofoe}^+(A \times A)$ be of shape $\dbnfofoe{\vlist{T}}{\Pi}$ for some $\Pi \subseteq \shA$ and $\vlist{T} = \{T_1,\dots,T_k\} \subseteq \shA$. We define $\varphi^{\noet}$ as $\dbnfofoe{\lift{\vlist{T}}}{\lift{\Pi}} \in {\ofoe}^+(\shA )$.
\end{definition}

The idea of translation $(\cdot)^{\noet}$ is to encode at the one-step level the non-deterministic mode of $\aut^{\noet}$: the property to enforce is that $\varphi^{\noet}$ is \emph{functional} in $\shA$, that means, whenever $(D,\val \: A \to \wp(D)) \models \varphi^{\noet}$, then there is $\val'  \: A \to \wp(D)$ such that $(D, \val') \models \varphi^{\noet}$ and  $ \val'(q_1)\cap \val'(q_2) = \emptyset$ for all $q_1,q_2 \in \shA$. 

\begin{lemma}\label{LEM_cont}
Let $\varphi \in {\ofoe}^+(A \times A)$ and $\varphi^{\noet}\in {\ofoe}^+(\shA )$ be as in Definition~\ref{DEF_finitary_lifting}. Then $\varphi^{\noet}$ is functional in $\shA$.
 \end{lemma}

\begin{proof}
%\yvwarning{this proof could use some more detail FZ: I expanded the proof and tried to make it clearer}
We first unfold the definition of $\varphi^{\noet}$ as follows:
\begin{align*}
\varphi^{\noet} =\ &
\underbrace{
    \exists \vlist{x}.\big(\arediff{\vlist{x}} \land \bigwedge_{0 \leq i \leq n} \tau^+_{\lift{T}_i}(x_i)
}_{\psi_1}
\land \underbrace{
    \forall z.(\arediff{\vlist{x},z} \lthen \bigvee_{S\in \lift{\Pi}} \tau^+_S(z))\big)
}_{\psi_2}
\end{align*}
Now suppose that $(D,\val \: \shA \to \wp(D))$ is a model where $\varphi^{\noet}$ is true. This amounts to the truth of subformulas $\psi_1$ and $\psi_2$, whose syntactic shape yields information on the types of elements of $D$. In particular, we can define a partition of $D$ into subsets $D_1$ and $D_2$ as follows:
\begin{itemize}
  \item As $\psi_1$ is true, we can pick $n$ distinct elements $s_1,\dots,s_n$ of $D$ such that $s_i$ witnesses the positive type $\lift{T}_i$, %\tau^+_{\lift{T}_i}(x_i)$,
   that is, $s_i \in \val(S)$ for each $S \in \lift{T}_i$. We define $D_1 := \{s_1,\dots,s_n\}$.
  %
  \item  As $\psi_2$ is true, we can cover $D \setminus D_1$ with a set $D_2$ containing all the elements not in $D_1$ witnessing a type ${\tau}^{+}_S(z)$ with $S \in  \lift{\Pi}$. 
 \end{itemize}
This partition uniquely associates with each $s \in D$ a type ${\tau}^{+}_S$ witnessed by $s$ and thus a set of unary predicates $S_s := S \subseteq A \cup \shA$. We can then define a valuation $\val'$ assigning to each element $s$ of $D$ exactly the set $S_s$.

We now check the properties of $\val'$. As the partition inducing $\val'$ follows the syntactic shape of $\varphi^{\noet}$, one can observe that $\val'$ is a restriction of $\val$ and $(D,\val')$ makes $\varphi^{\noet}$ true. By definition of the partition, $\val'$ assigns at most one unary predicate from $\shA$ to each element of $D_1$, because $\lift{\vlist{T}} \cup \lift{\Pi}$ is defined as the lifting of $\vlist{T} \cup \Pi$. It follows that $\varphi^{\noet}$ is functional in $\shA$, and the property is preserved by the restriction $\val'$.
\end{proof}


 Next we combine the previous definitions to characterise the transition function associated with the macro-states.
\begin{definition}\label{PROP_DeltaPowerset}
Let $\aut = \tup{A,\tmap,\pmap,a_I}$ be a $\nmso$-automaton. Fix any $c \in C$ and $Q \in \shA$. By Theorem~\ref{thm:bnfofoe} there is a sentence $\Psi_{Q,c} \in {\ofoe}^+(A\times A)$ in the basic form $\bigvee \dbnfofoe{\vlist{T}}{\Pi}$, for some $\Pi \subseteq \shA$ and $T_i \subseteq A \times A$, such that
$$\bigwedge_{a \in \Ran(Q)} \tmap^{\star}(a,c) \equiv \Psi_{Q,c}.$$
By definition, $\Psi_{Q,c} = \bigvee_{n}\varphi_n$, with each $\phi_{k}$ of shape $\dbnfofoe{\vlist{T}}{\Pi}$.
%
We put $\shDe(Q,c) := \bigvee_{n}\varphi_n^{\noet}  \in {\ofoe}^+(\shA)$, where the translation $(\cdot)^{\noet}$ is as in Definition \ref{DEF_noetherian_lifting}.
\end{definition}

\noindent We have now all the ingredients for our two-sorted automaton.

\begin{definition}\label{def:noetherianconstruct}
Let $\aut = \tup{A,\tmap,\pmap,a_I}$ be a {\nmso-automaton}. We define the \emph{noetherian construct over $\aut$} as the automaton $\aut^{\noet} = \tup{A^{\noet},\tmap^{\noet},\pmap^{\noet},a_I^{\noet}}$ given by %taking $A^{\noet} :=A \cup \shA$, $a_I^{\noet} := \{(a_I,a_I)\}$ and
%{\small%
\begin{eqnarray*}
      % \nonumber to remove numbering (before each equation)
        A^{\noet} &:=& A \cup \shA \\
        a_I^{\noet} &:=& \{(a_I,a_I)\}\\
        \tmap^{\noet}(q,c) &:=& \left\{
	\begin{array}{ll}
        \tmap(q,c) & q\in A \\
		\shDe(q,c) \vee \bigwedge_{a \in \Ran(q)} \tmap(a,c) & q \in \shA
	\end{array}
\right.\\
        \pmap^{\noet}(q) &:=& \left\{
	\begin{array}{ll}
        \pmap(q) & \hspace{3.43cm} q\in A \\
		1 & \hspace{3.43cm} q \in \shA.
	\end{array}
\right.
\end{eqnarray*}%}
\end{definition}
The definition of $\aut^{\noet}$ enforces its behaviour to be split according to the non-deterministic and alternating mode. Indeed, for any accepted $\model$, a match $\pi$ of $\agame(\aut^{\noet},\model)$ will visit positions involving macro-states only for finitely many initial rounds, because $\pmap^{\noet}[\shA] = \{1\}$. The alternating mode will be entered when, at a certain position $(R,s)\in \shA \times T$, the winning strategy for $\exists$ makes the disjunct $\bigwedge_{a \in \Ran(R)} \tmap(a,c)$ of $\tmap^{\noet}(R,c)$ true and then all successive positions only involve states from $A$. The next proposition fixes our desiderata on $\aut^{\noet}$. 

\begin{definition}\label{def:noetherianstrategy}
We say that a strategy $f$ in an acceptance game $\agame(\aut,\model)$ is \emph{noetherian} in $B \subseteq A$ when in any $f$-guided match there can be only finitely many rounds played at a position of shape $(q,s)$ with $q \in B$.
\end{definition}

%: in particular, \ref{point:finConstrStrategy} certifies the description that we did of the non-deterministic mode of $\aut^{f}$.

\begin{proposition}[\textbf{Simulation Theorem for $\nmso$-automata}]\label{PROP_facts_finConstr} Let $\aut$ be a $\nmso$-automaton and $\aut^{\noet}$ its noetherian construct.
\begin{enumerate}[(i)]
  \itemsep 0 pt
  \item $\aut^{\noet}$ is a $\nmso$-automaton.\label{point:finConstrAut}
  \item For any $\model$, if $\exists$ has a winning strategy in $\mathcal{A}(\aut^{\noet},\model)$ from position $(a_I^{\noet},s_I)$ then she has one that is functional in $\shA$ and noetherian in $\shA$.% (\emph{cf.} Definition \ref{def:StratfunctionalFinitary}).
  \label{point:finConstrStrategy}
  \item $\aut \equiv \aut^{\noet}$. \label{point:finConstrEquiv}
  \end{enumerate}
\end{proposition}
\begin{proof}
\begin{enumerate}[(i)]
  \item We need to check that $\tmap^{\noet}$ is weak. If $q_1 \ord q_2 \ord q_1$ in $\aut^{\noet}$ then by definition of $\tmap^{\noet}$ either $q_1, q_2 \in A$ or $q_1, q_2 \in \shA$. In the first case, $\pmap^{\noet}(q_1) = \pmap(q_1) = \pmap(q_2) = \pmap^{\noet}(q_2)$ because the original automaton $\aut$ is weak. In the latter case, $\pmap^{\noet}(q_1) = 1 = \pmap^{\noet}(q_2)$ by definition of $\pmap^{\noet}$. 
\item Let $f$ be a winning strategy for $\exists$ in $\mathcal{A}(\mb{A}^{\noet},\model)@(a_I^{\noet},s_I)$. We define a strategy $f'$ for $\exists$ in the same game as follows:
      \begin{enumerate}[label=(\alph*),ref=\alph*]
        \item on basic positions of the form $(a,s) \in A\times T$, we let $f'$ suggest the restriction $\val'$ to $A$ of the valuation $\val$ suggested by $f$. As no predicate from $A^{\noet}\setminus A =\shA$ occurs in $\Delta^{\noet}(a,\V(s)) = \Delta(a,\V(s))$, then $\val'$ also makes that sentence true in $\R{s}$.
        \begin{comment} With minimality
        on basic positions of the form $(a,s) \in A\times T$, $f'$ is defined as $f$. Indeed, as no predicate from $\shA$ occurs in $\Delta^{\noet}(a,\V(s))$, we can assume that the valuation suggested by $f$ does not assign any of them to nodes in $\R{s}$.
        \end{comment}
        \label{point:stat2point1}
        \item for basic positions of the form $(R,s) \in \shA \times T$, let $\val_{R,s}$ be the valuation suggested by $f$. As $f$ is winning, $\Delta^{\noet}(R,\V(s))$ is made true by $\val_{R,s}$. If this is because the disjunct $\bigwedge_{a \in \Ran(R)} \Delta(a,\V(s))$ is made true, then we can let $f'$ suggest the restriction to $A$ of $\val_{R,s}$, for the same reason as in \eqref{point:stat2point1}. Otherwise, the disjunct $\shDe(R,\V(s)) = \bigvee_{i}\varphi_i^{\noet}$ is made true. This means that, for some $i$, $(R[s], \val_{R,s}) \models \varphi_i^{\noet}$. By Lemma \ref{LEM_cont} $\varphi_i^{\noet}$ is functional in $\shA$, meaning that we have a restriction $\val_{R,s}'$ of $\val_{R,s}$ that verifies $\varphi_i^{\noet}$, assigns finitely many nodes to predicates from $\shA$ and associates with each node at most one predicate from $\shA$. We let $\val_{R,s}'$ be the suggestion of $f'$ from position $(R,s)$.
      \end{enumerate}
      The strategy $f'$ defined as above is immediately seen to be
      surviving for $\exists$. It is also winning, because the set of
      basic positions on which $f'$ is defined is a subset of the one
      of the winning strategy $f$. By this observation it also follows that any $f'$-conform match visits basic positions of the form $(R,s) \in \shA \times C$ only finitely many times, as those have odd parity.
  \item For the direction from left to right, it is immediate by definition of $\mb{A}^{\noet}$ that a winning strategy for $\exists$ in $\mc{G} = \mathcal{A}(\aut,\model)@(a_I,s_I)$ is also winning for $\exists$ in $\mc{G}^{\noet} = \mathcal{A}(\mb{A}^{\noet},\model)@(a_I^{\noet},s_I)$.

      For the direction from right to left, let $f$ be a winning strategy for $\exists$ in $\mc{G}^{\noet}$. The idea is to define a strategy $f'$ for $\exists$ in stages, while playing a match $\pi'$ in $\mc{G}$. In parallel to $\pi'$, a shadow match $\pi$ in $\mc{G}^{\noet}$ is maintained, where $\exists$ plays according to the strategy $f$. For each round $z_i$, we want to keep the following relation between the two matches:
\smallskip
\begin{center}
\fbox{\parbox{12cm}{
Either
\begin{enumerate}[label=(\arabic*),ref=\arabic*]
  \item basic positions of the form $(Q,s) \in \shA \times T$ and $(a,s) \in A \times T$ occur respectively in $\pi$ and $\pi'$, with $a \in \Ran(Q)$,
\end{enumerate}
or
\begin{enumerate}[label=(\arabic*),ref=\arabic*]
  \item[(2)] the same basic position of the form $(a,s) \in A \times T$ occurs in both matches.
\end{enumerate}
}}\hspace*{0.3cm}($\ddag$)
\end{center}
\smallskip
The key observation is that, because $f$ is winning, a basic position of the form $(Q,s) \in \shA \times T$ can occur only for finitely many initial rounds $z_0,\dots,z_n$ that are played in $\pi$, whereas for all successive rounds $z_n,z_{n+1},\dots$ only basic positions of the form $(a,s) \in A \times T$ are encountered. Indeed, if this was not the case then either $\exists$ would get stuck or the minimum parity occurring infinitely often would be odd, since states from $\shA$ have parity $1$.

It follows that enforcing a relation between the two matches as in ($\ddag$) suffices to prove that the defined strategy $f'$ is winning for $\exists$ in $\pi'$. For this purpose, first observe that $(\ddag).1$ holds at the initial round, where the positions visited in $\pi'$ and $\pi$ are respectively $(a_I,s_I) \in A \times T$ and $(\{(a_I,a_I)\},s_I) \in A^{\noet} \times T$. Inductively, consider any round $z_i$ that is played in $\pi'$ and $\pi$, respectively with basic positions $(a,s) \in A \times T$ and $(q,s) \in A^{\noet} \times T$. In order to define the suggestion of $f'$ in $\pi'$, we distinguish two cases.
\begin{itemize}
  \item First suppose that $(q,s)$ is of the form $(Q,s) \in
  \shA\times T$. By ($\ddag$) we can assume that $a$ is in $\Ran(Q)$. Let $\val_{Q,s} :A^{\noet} \rightarrow \wp(\R{s})$ be the valuation suggested by $f$, verifying the sentence $\Delta^{\noet}(Q,\V(s))$. We distinguish two further cases, depending on which disjunct of $\Delta^{\noet}(Q,\V(s))$ is made true by $\val_{Q,s}$.
      \begin{enumerate}[label=(\roman*), ref=\roman*]
        \item If $(\R{s},\val_{Q,s})\models \bigwedge_{b \in \Ran(Q)} \Delta(b,\V(s))$, then we let $\exists$ pick the restriction to $A$ of the valuation $\val_{Q,s}$. \label{point:valuation1}
        \item If $(\R{s},\val_{Q,s})\models \shDe(Q,\V(s))$, we let $\exists$ pick a valuation $\val_{a,s}:A \rightarrow \p (\R{s})$ defined by putting, for each $b \in A$:
            \begin{align*}
            % \nonumber to remove numbering (before each equation)
               \val_{a,s}(b)\ :=\ \bigcup_{b \in \Ran(Q')} &\{t \in \R{s} \mid t \in \val_{Q,s}(Q')\} \\
               \cup\ \ \ \ \ & \{t \in \R{s} \mid t \in \val_{Q,s}(b)\} .
            \end{align*} \label{point:valuation2}
      \end{enumerate}
      It can be readily checked that the suggested move is admissible for $\exists$ in $\pi$, i.e. it makes $\Delta(a,\V(s))$ true in $\R{s}$. For case \eqref{point:valuation2}, one has to observe how $\shDe$ is defined in terms of $\Delta$. In particular, the nodes assigned to $b$ by $\val_{Q,s}$ have to be assigned to $b$ also by $\val_{a,s}$, as they may be necessary to fulfill the condition, expressed with $\qu$ and $\dqu$, that infinitely many nodes witness (or that finitely many nodes do not witness) some type.

      We now show that $(\ddag)$ holds at round $z_{i+1}$. If \eqref{point:valuation1} is the case, any next position $(b,t)\in A \times T$ picked by player $\forall$ in $\pi'$ is also available for $\forall$ in $\pi$, and we end up in case $(\ddag .2)$. Suppose instead that \eqref{point:valuation2} is the case. Given the choice $(b,t) \in A \times T$ of $\forall$, by definition of $\val_{a,s}$ there are two possibilities. First, $(b,t)$ is also an available choice for $\forall$ in $\pi$, and we end up in case $(\ddag .2)$ as before. Otherwise, there is some $Q' \in \shA$ such that $b$ is in $\Ran(Q')$ and $\forall$ can choose $(Q',t)$ in the shadow match $\pi$. By letting $\pi$ advance at round $z_{i+1}$ with such a move, we are able to maintain $(\ddag .1)$ also in $z_{i+1}$.
  \item In the remaining case, inductively we are given the same basic position $(a,s) \in A\times T$ both in $\pi$ and in $\pi'$. The valuation $\val$ suggested by $f$ in $\pi$ verifies $\Delta^{\noet}(a,\V(s)) = \Delta(a,\V(s))$, thus we can let the restriction of $\val$ to $A$ be the valuation chosen by $\exists$ in the match $\pi'$. It is immediate that any next move of $\forall$ in $\pi'$ can be mirrored by the same move in $\pi$, meaning that we are able to maintain the same position --whence the relation $(\ddag.1)$-- also in the next round.
\end{itemize}
In both cases, the suggestion of strategy $f'$ was a legitimate move for $\exists$ maintaining the relation $(\ddag)$ between the two matches for any next round $z_{i+1}$. It follows that $f'$ is a winning strategy for $\exists$ in $\mc{G}$.
%
      \begin{comment} SHORTER ALTERNATIVE VERSION OF THE PROOF
      The idea is to define a strategy $f'$ for $\exists$ in stages, while playing a match $\pi'$ in $\mathcal{A}(\aut,\model)@(a_I,s_I)$. In parallel to $\pi'$, a shadow match $\pi$ in $\mathcal{A}(\mb{A}^{\noet},\model)@(a_I^{\noet},s_I)$ is maintained, where $\exists$ plays according to the strategy $f$. Since $f$ is winning and all macro-states from $\shA$ have an odd parity, in finitely many rounds the shadow match $\pi$ reaches a stage where $\mb{A}^{\noet}$ enters a state from $A$ and ``behaves as'' $\aut$ for all successive rounds. Thus $\pi$ can be assumed to have the following structure:
       \begin{enumerate}[(I)]
         \item there is an $n$ such that, for each round $z_i$ in the initial segment $z_0,z_1,\dots,z_n$ of $\pi$, a position of the form $(R,s) \in \shA \times T$ is visited and the valuation suggested by $f$ makes the disjunct $\shDe(R,\V(s))$ of $\Delta^{\noet}(R,\V(s))$ true in $\R{s}$.
         \item At round $z_{n+1}$ a basic position of the form $(Q,t) \in \shA \times T$ is visited. The valuation suggested by $f$ makes the disjunct $\bigwedge_{a \in \Ran(R)} \Delta(a,\V(t))$ of $\Delta^{\noet}(Q,\V(t))$ true in $\R{t}$. \label{point:initialsegm}
         \item For all the next rounds $z_{n+2},z_{n+3},\dots$ only positions of the form $(a,s) \in A \times T$ are visited.
       \end{enumerate}
       In each round of the initial segment $z_0,z_1,\dots,z_n$ we can maintain the condition that, if a position $(R,s)$ is visited in $\pi$, then at the same round a position $(a,s)$ with $a \in \Ran(R)$ occurs in $\pi'$. This holds for the initial round $z_0$. For the next ones $z_1,\dots,z_n$, it can be enforced by defining $f'$ in terms of $f$ in the standard way shown, for instance, in the proof of \cite[Prop. 3.9]{Zanasi:Thesis:2012}.
       \fzwarning{More details to be provided}
       Once $\pi$ reaches round $z_n$, say with position $(Q,t)$, the valuation suggested by $f$ makes $\bigwedge_{a \in \Ran(R)} \Delta(a,\V(t))$ true in $\R{t}$ ({\it cf.} point \eqref{point:initialsegm}). By assumption, at round $z_n$, $\pi'$ visits a position $(b,t)$ with $b \in \Ran(Q)$. Then in particular the valuation suggested by $f$ makes $\Delta(b,\V(t))$ true, and we let it be the suggestion of $f'$ at that stage. By definition of $\Delta^{\noet}$, from the next round onwards we can maintain the same basic positions in $\pi$ and $\pi'$, and let $f'$ just be defined as $f$. As $\exists$ wins $\pi$, it will also win the match $\pi'$, meaning that $f'$ is a winning strategy.
       \end{comment}
\end{enumerate}
\end{proof}

\subsubsection{From formulae to automata}

We are now ready to introduce our projection construction for $\nmso$-automata and show that the class of tree languages that they recognize is closed under noetherian
projection.
\begin{definition}\label{DEF_fin_projection}
Let $\aut = \tup{A,\tmap,\pmap,a_I}$ be a $\nmso$-automaton on alphabet $\p(\prop \cup \{p\})$, with $p \not\in P$. Let $\mathbb{A}^{\noet}$
denote its noetherian construct.
We define the $\nmso$-automaton ${{\exists}_F p}.\aut := \langle A^{\noet}, a_I^{\noet},
\tmapProj, \pmap^{\noet}\rangle$ on alphabet $\p(\prop)$ by putting
\begin{eqnarray*}
\tmapProj(q,c) &:=& \left\{
	\begin{array}{ll}
        \tmap^{\noet}(q,c) & q\in A \\
		\tmap^{\noet}(q,c) \vee \tmap^{\noet}(q,c\cup\{p\}) & q \in \shA.
	\end{array}
\right.
\end{eqnarray*}
%The automaton ${{\exists}_F p}.\aut$ is called the \emph{noetherian projection construct of $\aut$ over $p$}.
\end{definition}
\begin{proposition}\label{PROP_fin_projection}
For every $\nmso$-automaton $\aut$ on alphabet ${\p (\prop \cup \{p\})}$, with $p \not\in P$, we have that $\trees({{\exists}_F p}.\aut) = {{\exists}_F p}.\trees(\aut)$.
\end{proposition}

\begin{proof} %The key is to observe
For the inclusion from left to right, first observe that ${{\exists}_F p}.\aut$ is defined in terms of $\aut^{\noet}$ and thus the properties stated in Proposition \ref{PROP_facts_finConstr} hold for ${{\exists}_F p}.\aut$ as well. In particular, given a ${\p (\prop)}$-tree $\model$, any winning strategy $f$ for $\exists$ in $\mathcal{A}({{\exists}_F p}.\aut, \model$) from position $(a_I^{\noet},s_I)$ can be assumed to be functional and noetherian in $\shA$. We can use such a strategy to guess a noetherian $p$-variant of $\model$ as follows. First, by functionality for each node $s$ there is at most one position $(Q_s,s)$, with $Q_s \in \shA$, that is reachable in any $f$-guided match. From each such position, let $\val_{Q_s,s} \colon A^{\noet} \to \pw (R[s])$ be the valuation suggested by $f$. We let $X_p$ be the set of nodes $s$ for which $\val_{Q_s,s}$ makes the disjunct $\tmap^{\noet}(Q_s,\tscolors(s)\cup\{p\})$ of $\tmapProj(Q_s,\tscolors(s))$ true: intuitively, these are the nodes on which ${{\exists}_F p}.\aut$ behaves ``as if they were labeled with $p$''. Since $f$ is noetherian in $\shA$, the $p$-variant $\model'$ of $\model$ given by labeling the nodes in $X_p$ with $p$ is noetherian. One can readily verify that $\aut^{\noet}$ (and thus $\aut$ by Proposition \ref{PROP_facts_finConstr}) accepts $\model'$ by letting $\exists$ playing the strategy $f$ in $\mathcal{A}(\aut^{\noet},\model')$.

For the inclusion from right to left, let $\model'$ be a noetherian $p$-variant of some ${\pw (\prop)}$-tree $\model$ and suppose that $\exists$ has a winning strategy $f$ for $\mathcal{A}(\aut,\model')$ from position $(a_I,s_I)$. We now sketch how $\exists$ is able to win any match $\pi$ of $\mathcal{A}({{\exists}_F p}.\aut,\model)$ from position $(a_I^{\noet},s_I)$. The idea is to enforce that, at each round of $\pi$, $\exists$ assigns macro-states only to the nodes rooting a subtree of $\model'$ where the labeling $p$ appears, and $A$-states to the others. Using the information given by $f$, $\exists$ can make this assignment so that any visited position of shape $(Q,s) \in \shA \times T$ is such that $(a,s)$ is winning for $\exists$ in $\mathcal{A}(\aut,\model')$, for each $a \in \Ran(Q)$. In particular, the assignment of $\exists$ will make true the disjunct $\tmap^{\noet}(Q,\tscolors(s)\cup\{p\})$ of $\tmapProj(Q,\tscolors(s))$ if $s$ is labeled with $p$ in $\model'$, and the disjunct $\tmap^{\noet}(Q,\tscolors(s))$ otherwise. Since $\model'$ is a \emph{noetherian} $p$-variant, player $\exists$ will be required to assign macro-states to only finitely many nodes encountered along the play, and $\pi$ will eventually arrive to a position from which no node labeled with $p$ is reachable. At that point, $\exists$ allows ${{\exists}_F p}.\aut$ to switch from the non-deterministic to the alternating mode. By construction, the match $\pi$ now moves to a position $(a,s) \in A \times T$ that is winning for $\exists$ in $\mathcal{A}(\aut,\model')$. It is also winning in $\mathcal{A}({{\exists}_F p}.\aut,\model)$, because $\model'$ agrees with $\model$ on nodes without labeling $p$ and by definition $\tmapProj(a,\tscolors(s)) = \tmap(a,\tscolors(s))$. %\emph{not} labeled with $p$,  and in this phase she maintains a bundle $\pi_{a_1},\dots,\pi_{a_n}$ of shadow matches of $\mathcal{A}(\aut,\model')$.  --- the idea is that, as long  We can define  where $\model'$ is on alphabet ${\p (\prop)}$. where $\exists$ plays according to $g$. As long as such a match visits positions of the form $(R,s) \in \shA \times T$, we are able to maintain a bundle of matches of $\mathcal{A}(\aut,\model)$, one for each $a \in \Ran(R)$, where the strategy to play is defined in terms of $g$. Since $g$ is winning, eventually it allows $\aut^{\noet}$ to enter the alternating mode, say in position $(a,s) \in A \times T$. At that point we can just keep the match in the bundle which is at position $(a,s)$ --- which exists by construction --- and keep copying the strategy $g$.
\end{proof}

We have now in position to show our characterisation result.
% \medskip

\begin{proof}[Proof of Theorem \ref{t:mt2}, direction $(\Rightarrow)$]
By induction we prove that for every $\varphi \in \nmso$ there is a $\nmso$-automaton $\aut_\varphi$ such that $\ext{\varphi} = \trees(\aut_\varphi)$. We focus on two %two non-trivial
inductive cases.
 %\begin{itemize}
   %\item

  \indent If $\varphi = \neg \psi$, let $\aut_{\psi}$ be the $\nmso$-automaton for $\psi$ given by inductive hypothesis. As $\ofoe$ is closed under Boolean duals, we can define the complementation $\overline{\aut_{\psi}}$ of $\aut_{\psi}$ as in Definition \ref{d:caut}. Notice that $\overline{\aut_{\psi}}$ is indeed a $\nmso$-automaton, satisfying the \textbf{(weakness)} and \textbf{(continuity)} conditions in virtue of their self-dual nature. Proposition \ref{PROP_complementation} yields the complementation lemma allowing to conclude that on trees $\ext{\neg \psi} = \trees(\overline{\aut_{\psi}})$.
   %\item

\indent   If $\varphi = \exists p.\psi$, let $\aut_{\psi}$ be the automaton given by inductive hypothesis. By semantics of $\nmso$, on trees $\ext{\exists p. \psi} = {{\exists}_F p}.\ext{\psi}$ and thus $\ext{\exists p.\psi} = \trees({{\exists}_F p}.\aut_{\psi})$ by Proposition \ref{PROP_fin_projection}.
 %\end{itemize}
\end{proof} 



\subsubsection{From automata to formulae}


In this section we show the other direction of Theorem \ref{t:mt2}, completing
the automata characterisation of $\wmso$ on tree models.
The argument is reminiscent of the one showing that $\MSO$-automata can be
translated into equivalent formulas of $\MSO$~\cite{Walukiewicz96}.
%It consists of two steps. First, it is shown that for every $\wmso$-automaton $\bbA$ there is an effectively constructible formula in a suitably defined $\yvF$-fragment of a fixpoint extension of $\lque$. Second, it is verified that for every formula of the aforementioned  $\yvF$-fragment  there is an effectively constructible equivalent  $\wmso$-formula.
We start by introducing a fragment of a fixpoint extension
of $\olque$ and show how it embeds into $\wmso$.

\begin{definition}
The fixed point logic $\mlque$ on $\prop$ is given by:
% {\footnotesize%
% $$
% \varphi ::= p(x) \mid x=y \mid R(x,y) \mid \exists x.\varphi \mid \qu x.\varphi \mid \lnot\varphi \mid \varphi \land \varphi \mid \mu p.\varphi(p,x)
% $$}%
\begin{align*}
\varphi ::=\ & p(x) \mid x=y \mid R(x,y) \mid \lnot\varphi \mid \varphi \land \varphi \mid\\
& \exists x.\varphi \mid \qu x.\varphi \mid \mu p.\varphi(p,x)
\end{align*}
%
where $p\in\prop$, $x,y\in\fovar$; moreover $p$ occurs only positively in $\varphi(p,x)$ and $x$ is the only free variable in $\varphi(p,x)$.
\end{definition}

% Analogously to the modal $\mu$-calculus, the fixed point logic $\mlque$
% is  obtained by adding to $\lque$ the following rule for constructing
% fixed point formulas.
% \yvwarning{Why use `$P$' instead of `$p$'?}\fcnote{My fault/choice, will explain by mail}
%  \begin{itemize}
%  \item Given $P \in \prop$, $x\in\fovar$ and $\phi(P, x)$ with only positive occurrences of $P$ and no free variable other than $x$, $\mu P. \phi(P, x)$ and $\nu P. \phi(P, x)$ are formulas of $\mlque$.
%  \end{itemize}

%The semantics of the fixpoint formulas $\mu P. \phi(P, x)$ and $\nu P. \phi(P, x)$ is the expected one.
The semantics of $\mu p. \phi(p, x)$ is the expected one: given an LTS $\model$ and $s \in T$,  $\model \models \mu p. \phi(p, s)$ iff $s$ is in the least fixpoint of the operator
%$\phi^\model_P$ that maps any $S \subseteq T$ into $\{t \in T \mid \model[P \mapsto S] \models \phi(P, t) \}$.
$\phi^\model_p(S) := \{t \in T \mid \model[p \mapsto S] \models \phi(p, t) \}$.%The semantics of $\nu P. \phi(P, x)$ is dually defined by considering the greatest instead of the least fixpoint of $F_\phi$.


%Formulas of $\mlque$ may be also classified according to their alternation depth as it happens for the modal $\mu$-calculus.
%The alternation-free fragment of $\mlque$ is thence defined as the collection of $\mlque$-formulas $\phi$
%without nesting of greatest and least fixpoint operators, i.e. such that, for any two subformulas $\mu P.\psi_1(P,y)$ and $\nu Q. \psi_2(Q,z)$, predicates $P$ and $Q$ do not occur free respectively in $\psi_2(Q,z)$ and $\psi_1(P,y)$.

%
%
\begin{definition}
Given $p \in \prop$, we say that $\varphi \in \mlque$ is
\begin{itemize}
\item \emph{monotone in $p$} iff for every LTS $\model=\tup{T,R,\tscolors, s_I}$ and assignment $\ass$, if $\model, \ass \models \varphi$ and $ \{s \in T \mid p\in \tscolors(s)\} \subseteq E$, then $\model[p \mapsto E], g\models \phi$,

\item %\emph{continuous in $P$} iff for every LTS $\model$ and assignment $\ass$ there exists some finite $S \subseteq_\omega V(P)$ such that
    \emph{continuous in $p$} iff for every LTS $\model=\tup{T,R,\tscolors, s_I}$ and assignment $\ass$ there is some finite $S \subseteq_\omega \{s \in T \mid p\in \tscolors(s)\}$ such that $\model, \ass \models \varphi$ iff $\model[p \mapsto S], \ass \models \varphi$.
\end{itemize}
\end{definition}

We provide a definition of a fragment of $\mlque$ reminiscent of the one in
Theorem \ref{thm:ofoecont}.
\begin{definition}
Given a set $Q\subseteq \prop$, the fragment $\cont{\mlque}{Q}$ is defined by the following rules:
% {\small%
% $$
% \varphi ::= \psi \mid q(x) \mid \exists x.\varphi \mid \varphi \land \varphi \mid \varphi \lor \varphi \mid \wqu x.(\varphi,\psi) \mid \mu p. \phi'(p, x)
% $$}%
\begin{align*}
\varphi ::=\ & \psi \mid q(x) \mid \varphi \lor \varphi \mid \varphi \land \varphi \mid\\
& \exists x.\varphi \mid \wqu x.(\varphi,\psi) \mid \mu p.\varphi'(p,x)
\end{align*}
%
where $q \in Q$,
%$\psi \in \mlque(\prop\setminus Q)$,
$\psi \in \mlque$ is $Q$-free,
$\wqu x.(\varphi,\psi)
:= \forall x.(\varphi(x) \lor \psi(x)) \land \dqu x.\psi(x)$ and $\phi'(p,x)$
is a formula, with only $x$ free and $p \in \prop$, which belongs to
$\cont{\mlque}{Q \cup\{p\}}$.
\end{definition}

%We then verify that formulas in $\cont{\mlque}{A}$ are (semantical) continuous in $A$. The proof is
By combining the argument for the proof of Proposition \ref{thm:ofocont} and
the one used in proving the analogous Lemma 1 in~\cite{Fontaine08}, we can thence obtain the following:
\begin{proposition}\label{lem:cofoeiscont_mu}
If $\varphi \in \cont{\mlque}{Q}$ then $\varphi$ is continuous in (each element of) $Q$.
\end{proposition}
%\begin{proof} First, notice that If $\varphi \in \cont{\mlque}{A}$ then $\varphi$ is monotonous  in (each predicate from) $A$. %This is proved as for
%The proof goes then by induction on the complexity of $\varphi$. For the all the cases except the fixpoint one, the proof is the same as the one for Lemma \ref{lem:cofoeiscont}. For $\phi=\mu Q. \phi'(Q, x)$, with $\phi'(Q,x) \in \cont{\mlque}{A \cup\{Q\}}$, the argument is the same as the one in the proof of Lemma 1 in \cite{Fontaine08}.
%\end{proof}
%As for $\MC$, we define

\begin{definition}
The fragment $\clque$ of $\mlque$ is given by restricting the fixpoint operator to the continuous fragment:
% {\footnotesize%
% $$
% \varphi ::= p(x) \mid x=y \mid R(x,y) \mid \exists x.\varphi \mid \qu x.\varphi \mid \lnot\varphi \mid \varphi \land \varphi \mid \mu p.\varphi'(p,x)
% $$}%
\begin{align*}
\varphi ::=\ & p(x) \mid x=y \mid R(x,y) \mid \lnot\varphi \mid \varphi \land \varphi \mid\\
& \exists x.\varphi \mid \qu x.\varphi \mid \mu p.\varphi'(p,x)
\end{align*}
%
where $p\in\prop$, $x,y\in\fovar$, $\varphi'(p,x) \in \cont{\mlque}{\{p\}} \cap \clque$ is positive in $p$ and $x$ is its only free variable.% in $\varphi'$.
\end{definition}

% The fragment $\clque$ of $\mlque$  is obtained
% %by restricting the use of the least fixpoint operator to the continuous fragment. It is obtained
% by adding to $\lque$ the following restricted rule for fixpoint formulas.
%  \begin{itemize}
%  \item Given $P\in\prop$, $x\in\fovar$ and $\phi(P, x)$ with only positive occurrences of $P$ and only $x$ free,
% if $\phi(P,x) \in \cont{\mlque}{\{P\}}$, then $\mu P. \phi(P, x) \in \clque$.%$\mu P. \phi(P, x)$ is also a $\clque$-formula.
%  %if $\phi(P,x)$ is a $\clque$-formula that belongs to $\cont{\mlque}{\{P\}}$, then $\mu P. \phi(P, x)$ is also a $\clque$-formula.
%  \end{itemize}

%
%
%
%
%The logic $\mglque$ can be given a semantic in terms of evaluation games extending the one given  in \cite{BerwangerG01} for $\mgfoe$ by adding rules for the generalized quantifier.
%We present it just for $\qu$, and treat the rules for $\dqu y. \phi(\overline{x},y)$ as derived from the equivalent formula $\lnot \qu y. \lnot\phi(\overline{x},y)$
%the universal being treated as the alternation-free fragment.
%As usual, we assume that any predicate is bounded by at most one fixpoint operator
%%, any if a predicate is bounded, then the fixpoint operator bounding it is unique,
%and that bounded and free predicates are pairwise distinct.%\fzwarning{In the table: why not a clause for $\neg$, $\vee$, $\wedge$? Meaning of $\eta$, ; and :?}
%                             \begin{table}[h]
%                              \centering
%                            \begin{tabular}{|l|c|l|c|}
%                             \hline
%                              % after \\: \hline or \cline{col1-col2} \cline{col3-col4} ...
%                              Position & Player & Admissible moves & Parity\\
%                               \hline % \hline
%                           %  $( ; \overline{x}: \overline{a})$ & $\forall$ & $\{B \subseteq T \mid |B| \geq \aleph_0 \}$ & $-$ \\
%                           %   $B \subseteq T $ & $\exists$ & $\{(\lnot \phi(\overline{x},y); \overline{x}: \overline{a}, y:b)\ |\ b \in B \}$ & $-$\\
%                          %     \hline
%                            $(\qu y. \phi(\overline{x},y); \overline{x} \mapsto \overline{a})$ & $\exists$ & $\{B \subseteq T \mid |B| \geq \aleph_0 \}$ & $-$ \\
%                              $B \subseteq T $ & $\forall$ & $\{(\phi(\overline{x},y); \overline{x} \mapsto \overline{a}, y \mapsto b)\ |\ b \in B \}$ & $-$\\
%                              \hline
%%                              $(\mu P. \phi(P, x); x \mapsto a)$ & $\exists$ & $\{(\phi(P, x);  x: a)\}$ & $1$ \\
%%                             % \hline
%%                              $(\nu P. \phi(P, x); x: a)$ & $\exists$ & $\{(\phi(P, x);  x: a)\}$ & $0$ \\
%%                              %\hline
%%                              $(P(y); y: a)$ & $\exists$ & $\{(\eta P.\phi(P, x); x: a)\}$ & $-$ \\
%%
%%                              \hline
%                            \end{tabular}
%                             \caption{The new rules in the evaluation game for $\mglque$.
%                          }
%                             \label{mufo_game}
%                            \end{table}
%
% By a straightforward adaption of the corresponding proof for $\mgfoe$ in \cite{BerwangerG01},
% we obtain:
%
% \begin{theorem}
% For every model $\model$, and every formula $\mglque$-formula $\phi(x)$ with one free variable, then
% $\model \models \phi(n)$ iff $\exists$ has a winning strategy in $\mc{E}(\varphi(x),\model)@(\varphi(x); x \mapsto n)$, the evaluation game for $\phi(x)$ and $\model$ when evaluating $x$ at the node $n$.\end{theorem}
%

 %%%%%%%%
 %We now recall a useful property of fixpoints and continuity. Let $\phi(P,x)$ a formula with only $x$ free.
\noindent Given an LTS $\model$ and $p\in \prop$, for every ordinal $\alpha$ we define: %by induction the following sets:
 %\fcwarning{Why not $\phi^0(\emptyset):= \emptyset$?}
 \begin{itemize}
\item $\phi^0_p(\emptyset):= \emptyset$,
%\{ s \in T \mid \model[P \mapsto \emptyset] \models \phi(P, s)\}$,
\item $\phi_p^{\alpha+1}(\emptyset):= \{ s \in T \mid \model[p \mapsto \phi_p^\alpha(\emptyset)] \models \phi(p, s)\}$,
\item $\phi_p^{\lambda}(\emptyset):= \bigcup_{\alpha < \lambda} \phi_p^{\alpha}(\emptyset)$, with $\lambda$ limit.
\end{itemize}
%We state $\phi^{-1}(\emptyset):=\emptyset$.
If $\phi$ is monotone in $p$, then $\phi_p^{\beta+1}(\emptyset)= \phi_p^{\beta}(\emptyset)$, for some ordinal $\beta$. Also, the set $\phi_p^{\beta}(\emptyset)$ is the least fixpoint of $\phi^\model_p$ (see e.g. \cite{ArnoldN01}).



%A formula $\phi(P, x)$ is said to be \emph{constructive} in $P$ if its least fixpoint is reached in at most $\omega$ steps.
%, i.e., if for every model $\model$, the least fixpoint of $F_\phi$ equals to $\bigcup_{\alpha < \omega} \phi^{\alpha}(\emptyset)$.
 %From a local perspective, this means that
 A formula $\phi(p, x)$ is \emph{constructive} in $p$ if for every model $\model$,  every node $s \in T$, if $\model \models\mu p. \phi(p,s)$, then $s\in \phi_p^{i+1}(\emptyset)$, for some $i< \omega$.
The next proposition is easily verified:% by the fact that Scott proved in \cite{Fontaine08} for the modal $\mu$-calculus but that generalizes to $\mglque$ as well, states that continuous formulas are constructive.



%\afwarning{Verify the claim and that Gaelle's argument REALLY goes thorough also here.}
%\yvwarning{Do not attribute to Gaelle, it is obvious that a Scott continuous map reaches fixpoint in $< \omega$ steps}
\begin{proposition}\label{prop:constructivity}
Let $\phi(p,x)$ be a $\mlque$-formula with only $x$ free. If $\phi(p,x)$ is continuous in $p$, then for every LTS $\model$, and every node $s \in T$, there is $i < \omega$ such that
\[\model \models \mu p. \phi(p,s) \text{ iff } s \in \phi_p^{i+1}(\emptyset).\]
\end{proposition}

By Proposition~\ref{prop:constructivity} and the fact that sets $\phi_p^{i+1}(\emptyset)$ are essentially defined as finite unfoldings, we obtain the following.%\fcwarning{More intuition on this?}

\begin{proposition}\label{prop:cor_constructivity}
Let $\phi(p,x)$ be a $\mlque$-formula continuous in $p$ with only $x$
free. Let $\model$ be an LTS, and $s \in T$. Then
$\model \models \mu p. \phi(p,s)$ iff there is a finite set $p^\model \subseteq T$ such that $s\in p^\model$ and $\model[p\mapsto p^\model] \models \phi(p,t)$  for every $t \in p^\model$.
\end{proposition}
% \begin{proof}
% For the direction from left to right, assume that $\model \models \mu P. \phi(P,s)$.  By Proposition~\ref{prop:constructivity}, we know that  there is $i< \omega$ such that $\model[P \mapsto \phi^i(\emptyset)] \models \phi(P, s)$. The set $\phi^i(\emptyset)$ need not to be finite. However,
% using this information, we are going construct a finite tree whose nodes $t$ are labelled by finite sets $X^m_j$, where $m$ is a node of $\model$ and $j \leq i$, satisfying the following condition:
% \begin{enumerate}
%\item  if $t$ is the root, then $t$ is labelled by $X_i^s$,
%\item  if $t$ is labelled by $X_j^m=\{s_1, \dots, s_\ell\}$ and $j>0$, then $t$ has $\ell$  children and for every $s_i \in X_j^m$ there is an unique child $t'$ of $t$ labelled by $X_{j-1}^{n_i}$ where $m$ is a nodes,
%%\item if $s$ is labelled by $X_j^m$ and $j=-1$, then $X_j^m=\emptyset$,
%\item for every node $t$ of the tree, if $t$ is labelled by $X_j^m$, then it holds that $X_j^m \subseteq \phi^{j}(\emptyset)$.
%\end{enumerate}
%If we verify that $\model[P\mapsto P^\model] \models \phi(P,s)$ holds by taking as $P^\model$ the union of all labels of the nodes of the constructed tree, we can conclude for the proof of this direction.
%
%As starting point of the inductive construction, we start by the empty tree.  Recall that we know that  $\model[P \mapsto \phi^i(\emptyset)] \models \phi(P, s)$. Since $\phi(P,x)$ is continuous in $P$, there is a finite set $X^s_i \subseteq \phi^i(\emptyset)$ such that $\model[P \mapsto X^s_i] \models \phi(P, s)$. We then add a root to our tree and label it by $X^s_i$.
% Assume that at a leaf $s$ of our tree is labelled by $X^m_j$, for some $j < i$. If $X^m_j$ is empty, than we stop, else we proceed as follows. We know that $X^m_j\subseteq \phi^{j}(\emptyset)$. This means that $\model[P \mapsto \phi^{j-1}(\emptyset)] \models \phi(P, r)$, for every $r \in X_j^m$. By continuity, for each such $r$, there is a finite set $X^m_{j-1} \subseteq  \phi^{j-1}(\emptyset)$ such that $\model[P \mapsto X^m_{j-1}(\emptyset)] \models \phi(P, r)$. For each $r \in X^m_j$ we thus add a child to $m$ and label it with $X^r_{j-1}$. By definition of $\phi^{i+1}(\emptyset)$, the tree is finite. Let $X$ be the union of all labels of the constructed tree. $X$ is finite, and by monotonicity of $\phi(P,x)$ we have that for every $m \in X \cup \{s\}$, $\model[P \mapsto X \cup \{s\}] \models \phi(P,m)$.
%
%For the other direction, it is enough to notice that the smallest finite set $P^\model \subseteq T$ such that $\model[P\mapsto P^\model] \models \phi(P,s)$ and $\model[P\mapsto P^\model] \models \phi(P,m)$  for every $m \in P^\model$ is the least fixpoint of the function that maps any $S \subseteq T$ into $\{t \in T \mid \model[P \mapsto S] \models \phi(P, t) \}$.
%%the idea is the following. By assumption there is a finite set $P^\model \subset T$ such that $\model[P\mapsto P^\model] \models \phi(P,n)$ and $\model[P\mapsto P^\model] \models \phi(P,m)$  for every $m \in P^\model$. The winning strategy for \'Eloise  in $\mc{E}(\mu P.\varphi(P,x),\model)@(\mu P.\varphi(P,x); x \mapsto n)$
%%is thus define as the composition of all winning strategies in $\mc{E}(\varphi(P,x),\model[P \mapsto P^\model]))@(\varphi(P,x); x \mapsto m)$
%% for $m \in P^\model$.
% \end{proof}

Proposition \ref{prop:cor_constructivity} naturally suggests the following translation $\mgFOETr{\cdot}:\mlque\to\nmso$. It is given homomorphically in predicates, Booleans and first-order quantifiers and:
\begin{itemize}
% \begin{multicols}{2}
\itemsep 0 pt
% \item $\mgFOETr{P(x)}=P(x)$,
% \item $\mgFOETr{R(x,y)}=R(x,y)$,
% \item $\mgFOETr{x\foeq y}= (x \foeq y)$,
% \item $\mgFOETr{\varphi \land \psi}=\mgFOETr{\varphi} \land \mgFOETr{\psi}$,
% \item $\mgFOETr{\lnot \varphi}= \lnot \mgFOETr{\varphi}$,
% \item $\mgFOETr{\exists x. \varphi}=\exists x. \mgFOETr{\varphi}$,
\item $\mgFOETr{ \qu x. \varphi} := \forall p.\exists x. (\lnot p(x) \land \mgFOETr{\varphi})$,
% \end{multicols}
\item $\mgFOETr{\mu p. \varphi(p,x)} := \exists p ( p(x) \land \forall y ( p(y) \to \mgFOETr{\varphi(p,y) }))$.
\end{itemize}

%Note that in $\mgFOETr(\mu P. \varphi(P,x))$, the predicate $P$ which occurs in $\mgFOETr(\varphi) $ is bounded by the outermost second order existential quantifier.

%The following theorem %, which is the analogous of Theorem \ref{thm:contransweak} but for $\mlque$,
%is then an immediate corollary of Proposition \ref{prop:cor_constructivity}.

\begin{proposition}\label{thm:guard_wmso}
Let $\phi$ be a $\clque$-formula, $\model$ an LTS and $\ass$ an assignment.
Then $\model, \ass \models \varphi$ iff $\model, \ass \models \mgFOETr{\varphi}$.
%the following two conditions are equivalent:
%\begin{enumerate}\itemsep 0pt \item $\model, \ass \models \varphi$, \item $\model, \ass \models \mgFOETr{\varphi}$. \end{enumerate}
\end{proposition}
\begin{proof}
The proof is by induction on $\varphi$. The least fixpoint case is handled by applying Proposition \ref{prop:cor_constructivity}.
%
%Let therefore consider $\phi$ is of the form $\mu P. \psi(P,x)$. Without loss of generality, that bounded and free predicate variables are distincts.
%We first show that $(1)$ implies $(2)$. Since $\model , \ass \models \varphi$, \'Eloise has a winning strategy $f$ in $\mc{E}(\phi,\model)@(\varphi,s_I, \ass)$.
%Define $P^\model$ to be the set of node $n \in T$ such that there is a (partial) match $\pi'$ that
%%
%\begin{enumerate}
%\itemsep 0pt
%\item is consistent with $f$, and such that
%\item every position of $\pi'$ is of the form $(\gamma,m, \ass')$, with  $P$ active in $\gamma$, and
%\item the last position of $\pi'$ is of the form $(\varphi, n, \ass')$.
%\end{enumerate}
%%
%The first observation is that since $f$ is a winning strategy, all $f$-consistent matches are finite. Moreover for every position of $\pi'$ is of the form $(\psi(P,x),m, \ass')$, we have that $\model[P \mapsto P^{\model}], \ass' \models \psi(P,m)$. We construct inductively a finite tree labelled by pairs $(x, X)$ where $x$ is a node of $\model$ and $X$ is a finite set of nodes of $\model$ as follows. First, because $\model[P \mapsto P^{\model}] , \ass \models \psi(P,x)$, so there is a finite subset $X \subseteq P^\model$ such that $\model[P \mapsto X_1] , \ass \models \psi(P,x)$. Thus we color the root with $(n, X)$. Now, assume we are given a leaf colored by $(y,Y)$. Consider an enumeration $x_1, \dots, x_k$ of $Y$. For every $i \leq k$, we add a child to $(y,Y)$ labelled by $(x_i, X_i)$ where $X_i$ is given by the fact that since $\model[P \mapsto P^{\model.x_i}] , \ass \models \psi(P,x)$, there is a finite set $X_i$ of nodes in $P^{\model.x_i}$
%
%%the only player who picks successor in a partial match $\pi'$ defined as above is \'Eloise. As a consequence of K\"onig's Lemma, $P^\model$ is finite.
%%
%%By using the induction hypothesis, it is easy to check that $\model[x \mapsto s_I, P \mapsto P^\model] \models P(x) \land \forall y ( P(y) \to ST_y(\varphi) )$.
%%
%%For the other direction, the idea is the following. Because $\model[x \mapsto s_I] \models ST_x(\varphi)$,
%%there is a finite set $P^\model$ such that $\model[x \mapsto s_I, P \mapsto P^\model] \models P(x) \land \forall y ( P(y) \to ST_y(\varphi) )$. The winning strategy for \'Eloise  in $\mc{E}(\mu P.\varphi,\model)@(\mu P.\varphi,s_I)$
%%is thus define as the composition of all winning strategies in $\mc{E}(\varphi,\model[P \mapsto P^\model])@(\varphi,s)$ for $s \in P^\model$.
\end{proof}

By virtue of Proposition \ref{thm:guard_wmso}, we are able to conclude the proof of Theorem \ref{t:mt2} by showing the following statement.
\begin{proposition}\label{thm:wmsoauttof}
Every $\wmso$-automaton can be effectively translated into an equivalent $\clque$-formula.
\end{proposition}
%
%
%
%The guarded fragment $\glque$ of $\lque$ is obtained by imposing that first-order and generalized quantifiers  are relativized to atomic formulas of the form $r(x,y)$. The fixpoint extension $\mglque$  of $\glque$ is thus defined by adding a fixpoint construction clause.
%Analogously to the modal $\mu$-calculus, the  $\yvF$-fragment is finally obtained by considering only $\mglque$ formulas without alternation of fixpoints and by imposing some continuity conditions on the fixpoint construction rule.
%
\begin{proof} The argument is
 essentially a refinement of the standard proof showing that any automaton in $\yvAut(\ofo)$ can be translated into an equivalent $\mu$-formula
$\xi_\aut$ (\emph{cf.} e.g. \cite{Ven08}).
The idea is the following. We see a $\wmso$-automaton as a system of equations expressed in terms of $\lque$-formulas: each state corresponds to a monadic predicate variable and the odd/even parity of a state corresponds to the least/greatest fixpoint that we seek for the associated variable, etc. One then solves this system of equations via the same inductive procedure used to obtain the formula of the modal $\mu$-calculus from the system associated with an automaton in $\yvAut(\ofo)$ (see e.g. \cite{ArnoldN01} for a description of the solution procedure). Because of the \textbf{(weakness)} and \textbf{(continuity)} conditions on the starting $\nmso$-automaton $\aut$, it is thence possible to verify that the resulting fixpoint formula $\xi_\aut$ belongs to $\clque$.
%
%To complete the proof of the Theorem, we have to provide an effective truth-preserving translation from the  considered fragment of $\mglque$ into $\wmso$. The key observation here is that continuous fixpoint are constructive (see \cite{Fontaine08}), that is a least-fixpoint formula in one free variable $\mu p. \phi(p,x)$ of the $\yvF$-fragment of $\mglque$ is true at $s$ in a model $\model$ if $s$ belongs to some finite approximant of the least fixpoint induced by $\phi(p.x)$. From this fact, it is then possible to verify that a formula $\mu p. \phi(p,x)$ of the $\yvF$-fragment of $\mglque$ is true at $s$ in a model $\model$ iff is $\phi(p,x)$ true at $s$ in $\model[p\mapsto P^\model]$, for some finite $P^\model$, and use this propriety to obtain the truth-preserving  translation into $\wmso$.
%%by taking formulas without alternation such that fixpoint operators only bound
%%$\mglque$  is  obtained by adding to $\glque$ the following (semantic) rule for constructing fixed point formulas.
%%fragment
%%
%%
%%The main idea is to encode each state of the starting automaton as a propositional variable bounded by a greatest (if the parity is even) or least (if the parity is odd) fixpoint operator. The target of the translation is a suitable extension of first-order logic with fixpoint operators, denoted with $\mglque$. Thanks to the (weakness) and (continuity) conditions on $\nmso$-automata, we are able to infer that the target can be in fact restricted to a fragment of $\mglque$. Finally, this is proven to be included in $\nmso$ completing the proof of Theorem \ref{thm:wmso_autofor}.
\end{proof}
%%\btbs
%% \item Continue with more details according to the space that we decide to devote to this section.
%% \etbs 
