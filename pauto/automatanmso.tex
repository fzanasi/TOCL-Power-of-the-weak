%!TEX root = ../00CFVZ_TOCL.tex


The rest of this section will be devoted to prove that $\wmso$-automata
characterize $\wmso$ on tree models, as expressed in Theorem~\ref{t:mt2}.
First, we focus on showing the direction from formulas to automata.
In subsections~\ref{sec:finitconstr} and \ref{sec:closureautomata} we provide
the automata constructions handling the challenging case, that is the
translation of an existential formula $\exists p.\psi$ of $\wmso$ into an
equivalent $\wmso$-automaton.
To this aim, we define a closure operation on tree languages corresponding
to the semantics of $\wmso$ quantification.

\begin{definition}\label{def:tree_finproj}
Let $\prop$ be a set of proposition letters, $p \not\in P$ be a proposition letter, and $\trees$ be a tree language of $\p (\prop\cup\{p\})$-labeled
trees.
The \emph{finitary projection} of $\trees$ over $p$ is the language
${\exists}_F p.\trees$ of $\p (\prop)$-labeled trees %$\model$ for which some
%finitary $p$-variant of $\model$ exists in $\trees$.
given as follows:
%
$$
{\exists}_F p.\trees = \{\model \mid \text{ $\exists$ a finite $p$-variant } \model' \text{ of } \model \text{ with } \model' \in \trees\}.
$$
%
A class $K$ of tree languages is \emph{closed under finitary projection
over $p$} if, for any language $\trees$ in $K$, also ${{\exists}_F p}.\trees$ is in $K$.
% A class $K$ is \emph{closed under finitary projection
% over $p$} if $\trees\in K$ implies ${{\exists}_F p}.\trees \in K$.
\end{definition} 

\subsubsection{Simulation theorem}


\noindent Our next goal is a \emph{projection construction} that, given
a $\wmso$-automaton $\mb{A}$, provides one recognizing ${{\exists}_F p}.\trees(\mb{A})$. For $\mso$-automata, an analogous construction crucially uses the following \emph{simulation theorem}: every
$\mso$-automaton $\aut$ is equivalent to a \emph{non-deterministic} one $\aut'$ \cite{Walukiewicz96}.
Semantically, non-determinism yields the appealing property that any strategy $f$ for player $\exists$ in the acceptance game $\mc{A}(\aut',\model)$ can be assumed to be functional in $A'$ (\emph{cf.} Definition \ref{def:StratfunctionalFinitary}).
This is particularly helpful because, to define a $p$-variant of $\model$
that is accepted by the projection construct on $\aut'$, we
can infer whether a node $s$ should be labeled with $p$ by the value $f(a,s)$, where $a$ is the unique state of $\aut'$ (by functionality) that $f$ associates with $s$. Now, in the case of $\wmso$-automata we are interested in guessing
\emph{finitary} $p$-variants, which requires $f$ to be functional only on a \emph{finite} set of nodes. Thus the idea of our simulation theorem is to turn a $\wmso$-automaton $\aut$ into an equivalent one $\aut^{\f}$ that behaves non-deterministically on a \emph{finite} portion of any accepted tree.

For $\mso$-automata, the simulation theorem is based on a powerset construction: if the starting automaton has carrier $A$, the resulting non-deterministic automaton is based on ``macro-states'' from the set $\shA := \pw (A \times A)$.\footnote{The use of carrier $\pw (A \times A)$ instead of the more obvious $\pw A$ is needed to correctly associate with a run on macro-states the corresponding bundle of runs of the original automaton $\aut$ (\emph{cf.} \cite{Walukiewicz96}).} Analogously, for $\wmso$-automata we will associate the non-deterministic behavior with macro-states. However, as explained above, the automaton $\aut^{\f}$ that we construct has to be non-deterministic just on finitely many nodes of the input and may behave as $\aut$ (i.e. in ``alternating mode'') on the others. To this aim, $\aut^{\f}$ will be ``two-sorted'', roughly consisting of one copy of $\aut$ (with carrier $A$) and a variant of its powerset construction, based both on $A$ and $\shA$. For any accepted $\model$, the idea is to make any match $\pi$ of $\mc{A}(\aut^{\f},\model)$ consist of two parts:
\begin{description}
  \item[(\textbf{Non-deterministic mode})] for finitely many rounds of $\pi$, each visited basic position has shape $(q,s) \in \shA \times T$. The valuation $\val \colon A \cup \shA \to \pw (R[s])$ picked by player $\exists$ assigns macro-states (from $\shA$) only to a \emph{finite} subset of $\R{s}$ and states (from $A$) to the rest of $\R{s}$. Also, she assigns \emph{at most one macro-state} to each node.
  \item[(\textbf{Alternating mode})] At a certain round, $\pi$ visits the last position with a macro-state and turns into a match of the game $\mc{A}(\aut,\model)$, i.e. all next positions are from $A \times T$. %of shape $(a,t) \in A \times T$.
\end{description}
Therefore successful runs of $\mb{A}^{\f}$ will have the property of processing only a \emph{finite} amount of the input with $\mb{A}^{\f}$ being in a macro-state and all the rest with $\mb{A}^{\f}$ behaving exactly as $\aut$. %The leading intuition is that the distinction between the two sorts of states allows us to isolate a finite portion of the tree (the one associated with macro-states) where $\mb{A}^{\f}$ behaves as a non-deterministic automaton.

We now proceed in steps towards the construction of $\mb{A}^{\f}$. The following is a notion of lifting for types on states that is instrumental in defining a translation to types on macro-states. %The distinction between empty and non-empty subsets of $A$ is to make sure that empty types on $A$ are lifted to empty types on $\pw A$.
\begin{definition}
Given a set $A$ and $\Sigma \subseteq \wp A$, we define the \emph{lifting} $\lift{\Sigma} \subseteq \wp \wp A$ as $\{\{S\} \mid S \in \Sigma \wedge S \neq \emptyset\} \cup
    \{\emptyset \mid \emptyset \in \Sigma \}$.
%\begin{eqnarray*}
%\lift{\Sigma} & := & \{\{S\} \mid S \in \Sigma \wedge S \neq \emptyset\} \cup    \{\emptyset \mid \emptyset \in \Sigma \}.
%\end{eqnarray*}
\end{definition}

 The next definition is standard (see e.g.  \cite{Walukiewicz96,Ven08}) as an intermediate step to define the transition function of the powerset construct for parity automata. It simply \emph{tags} the (potential next) states occurring in $\Delta(a,c)$ with the information of the current state.

\begin{definition}\label{DEF_delta star} Let $\mb{A} = \tup{A,\Delta,\Omega,a_I}$ be a $\wmso$-automaton. Fix $a \in A$, $c \in C$. We define
%The sentence $\Delta^{\star}(a,c) \in {\olque}^+(A\times A)$ is defined as $\Delta(a,c)[b \mapsto (a,b) \mid b \in A]$.
\begin{eqnarray*}\Delta^{\star}(a,c) &:=& \Delta(a,c)[b \mapsto (a,b) \mid b \in A]     \end{eqnarray*}
where $\Delta(a,c)[b \mapsto (a,b) \mid b \in A] \in {\olque}^+(A\times A)$ is the sentence obtained by replacing each monadic predicate $b \in A$ in $\Delta(a,c)$ with the monadic predicate $(a,b) \in A \times A$.
\end{definition}

We now define a translation for the one-step language of $\wmso$-automata, which can be thought as based on carrier $A \times A$ by effect of the transformation of Definition \ref{DEF_delta star}.

\begin{definition}\label{DEF_finitary_lifting}
Let $\varphi \in {\olque}^+(A \times A)$ be of the shape
%
$$
\varphi = \mondbnfofoe{\vlist{T}}{\Pi \cup \Sigma}{+} \land
\mondbnfinf{\Sigma} %\bigwedge_{S\in\Sigma} \qu y.{\tau}^{+}_S(y) \land \dqu y.\bigvee_{S\in\Sigma} {\tau}^{+}_S(y)
$$
%
where $\Pi,\Sigma \subseteq \shA$ and each $T_i \subseteq A \times A$
(provided by Theorem~\ref{cor:olquepositivenf})%
%, that is, $ \varphi =  \posdbnfolque{\vlist{T}}{\Pi}{\Sigma}$
. Let $\widetilde{\Sigma}\subseteq \wp A$ be $\widetilde{\Sigma} := \{\Ran(S) \mid S \in \Sigma\}$. Define the translation $\varphi^{\f} \in {\olque}^+(A \cup \shA )$ as:
$$
\varphi^\f := \mondbnfofoe{\lift{\vlist{T}}}{\lift{\Pi} \cup \lift{\Sigma} \cup \widetilde{\Sigma}}{+} \land
\mondbnfinf{\widetilde{\Sigma}} .
$$
%Observe that each ${\tau}^{+}_{P}$ is a (positive) $A$-type, as $P = \Ran(S) \subseteq A$ for some $S \in \Sigma$.
%Observe that each $P\in \widetilde{\Sigma}$ we have $P = \Ran(S) \subseteq A$ for some $S \in \Sigma$.
\end{definition}

The idea of translation $(\cdot)^{\f}$ is to encode at the one-step level the non-deterministic mode of $\aut^{\f}$. As no macro-state occurs in $\mondbnfinf{\widetilde{\Sigma}}$,
%$(\bigcup \widetilde{\Sigma}) \cap \shA = \emptyset$,
by Corollary \ref{cor:olquecontinuousnf} the sentence $\varphi^{\f}$ is continuous in each $R\in \shA$, i.e. it can be made true in a domain $D$ by assigning macro-states to \emph{finitely many} elements of $D$. Moreover, macro-states occur in $\varphi^{\f}$ only inside lifted types in $\lift{\vlist{T}},\lift{\Pi}$ or $\lift{\Sigma}$: then, by definition of $\lift{(\cdot)}$, $\varphi^{\f}$ can be made true in $D$ by assigning \emph{at most one} macro-state to any element.

 Next we combine the previous definitions to characterize the transition function associated with the macro-states.
\begin{definition}\label{PROP_DeltaPowerset}
Let $\mb{A} = \tup{A,\Delta,\Omega,a_I}$ be a $\wmso$-automaton. Fix any $c \in C$ and $Q \in \shA$. By Theorem~\ref{cor:olquepositivenf} there is a sentence $\Psi_{Q,c} \in {\olque}^+(A\times A)$ in the basic form $\bigvee \posdbnfolque{\vlist{T}}{\Pi}{\Sigma}$, for some $\Pi,\Sigma \subseteq \shA$ and $T_i \subseteq A \times A$, such that
%\begin{eqnarray*}
% \nonumber to remove numbering (before each equation)
%  \bigwedge_{a \in \Ran(Q)} \Delta^{\star}(a,c) &\equiv& \Psi_{Q,c}.
$$\bigwedge_{a \in \Ran(Q)} \Delta^{\star}(a,c) \equiv \Psi_{Q,c}.$$
%\end{eqnarray*}
By definition, $\Psi_{Q,c} = \bigvee_{n}\varphi_n$, with each $\phi_{k}$ of shape
%
$$\posdbnfolque{\vlist{T}}{\Pi}{\Sigma} = \mondbnfofoe{\vlist{T}}{\Pi \cup \Sigma}{+} \land
\mondbnfinf{\Sigma}.$$
%
We put $\shDe(Q,c) := \bigvee_{n}\varphi_n^{\f}$, where the translation $(\cdot)^{\f}$ is as in Definition \ref{DEF_finitary_lifting}. Observe that $\shDe(Q,c) \in {\olque}^+(A \cup \shA)$.
\end{definition}

\noindent We have now all the ingredients for our two-sorted automaton.

\begin{definition}\label{def:finitaryconstruct}
Let $\mb{A} = \tup{A,\Delta,\Omega,a_I}$ be a {\wmso-automaton}. We define the \emph{finitary construct over $\mb{A}$} as the automaton $\mb{A}^{\f} = \tup{A^{\f},\Delta^{\f},\Omega^{\f},a_I^{\f}}$ given by %taking $A^{\f} :=A \cup \shA$, $a_I^{\f} := \{(a_I,a_I)\}$ and
%{\small%
\begin{eqnarray*}
      % \nonumber to remove numbering (before each equation)
        A^{\f} &:=& A \cup \shA \\
        a_I^{\f} &:=& \{(a_I,a_I)\}\\
        \Delta^{\f}(q,c) &:=& \left\{
	\begin{array}{ll}
        \Delta(q,c) & q\in A \\
		\shDe(q,c) \vee \bigwedge_{a \in \Ran(q)} \Delta(a,c) & q \in \shA
	\end{array}
\right.\\
        \Omega^{\f}(q) &:=& \left\{
	\begin{array}{ll}
        \Omega(q) & \hspace{3.43cm} q\in A \\
		1 & \hspace{3.43cm} q \in \shA.
	\end{array}
\right.
\end{eqnarray*}%}
\end{definition}
The definition of $\aut^{\f}$ enforces its behavior to be split according to the non-deterministic and alternating mode. Indeed, for any accepted $\model$, a match $\pi$ of $\mc{A}(\aut^{\f},\model)$ will visit positions involving macro-states only for finitely many initial rounds, because $\Omega^{\f}[\shA] = \{1\}$. The alternating mode will be entered when, at a certain position $(R,s)\in \shA \times T$, the winning strategy for $\exists$ makes the disjunct $\bigwedge_{a \in \Ran(R)} \Delta(a,c)$ of $\Delta^{\f}(R,c)$ true and then all successive positions only involve states from $A$. The next proposition fixes our desiderata on $\aut^{\f}$.
%: in particular, \ref{point:finConstrStrategy} certifies the description that we did of the non-deterministic mode of $\mb{A}^{f}$.

\begin{proposition}[\textbf{Simulation Theorem for $\wmso$-automata}]\label{PROP_facts_finConstr} Let $\mb{A}$ be a $\wmso$-automaton and $\mb{A}^{\f}$ its finitary construct.
\begin{enumerate}[(i)]
  \itemsep 0 pt
  \item $\mb{A}^{\f}$ is a $\wmso$-automaton.\label{point:finConstrAut}
  \item For any $\model$, if $\exists$ has a winning strategy in $\mathcal{A}(\aut^{\f},\model)$ from position $(a_I^{\f},s_I)$ then she has one that is functional in $\shA$ and finitary in $\shA$ (\emph{cf.} Definition \ref{def:StratfunctionalFinitary}).
  \label{point:finConstrStrategy}
  \item $\mb{A} \equiv \mb{A}^{\f}$. \label{point:finConstrEquiv}
  \end{enumerate}
\end{proposition}
\begin{proof}
For~\ref{point:finConstrAut}, observe that any SCC
of $\mb{A}^{\f}$ involves states of exactly one sort, either $A$ or $\shA$. For SCCs on sort $A$, \textbf{(weakness)} and \textbf{(continuity)} of $\mb{A}^{\f}$ follow by the ones of $\mb{A}$. For SCCs on sort $\shA$, \textbf{(weakness)} follows by observing that all macro-states in $\mb{A}^{\f}$ have the same parity value. Concerning \textbf{(continuity)}, by definition of $\Delta^{\f}$ any macro-state can only appear inside a formula of the form $\varphi^{\f}$, which as observed above is continuous in each $Q \in \shA$. Statement~\ref{point:finConstrStrategy} again follows by properties of the translation $(\cdot)^{\f}$ and the observation that a winning strategy will eventually let $\mb{A}^{\f}$ enter the alternating mode. The argument for~\ref{point:finConstrEquiv} goes as follows. For the direction from left to right, if player $\exists$ has a winning strategy $f$ for $\mathcal{A}(\aut,\model)$ from position $(a_I,s_I)$, then $f$ will also be winning in $\mathcal{A}(\aut^{\f},\model)$ from position $(a_I^{\f},s_I)$ --- intuitively, playing $f$ amounts to immediately let $\aut^{\f}$ enter the alternating mode by making the disjunct $\bigwedge_{a \in \Ran(a_I^{\f})}\Delta(a,\tscolors(s_I)) = \Delta(a_I,\tscolors(s_I))$ of $\Delta^{\f}(a_I^{\f},\tscolors(s_I))$ true. Conversely, suppose that $\aut^{\f}$ accepts $\model$ and let $g$ be the corresponding winning strategy for $\exists$. We can make $\exists$ win any match $\pi$ of $\mathcal{A}(\aut,\model)$ played from position $(a_I,s_I)$ as follows. While playing $\pi$, we maintain a ``shadow match'' $\pi'$ of $\mathcal{A}(\aut^{\f},\model)$ from position $(a_I^{\f},s_I)$ where $\exists$ plays according to $g$. By suitably defining the strategy of $\exists$ in $\pi$ in terms of $g$, we can enforce the following relation between $\pi$ and $\pi'$ at each round: either (I) the same position of the form $(a,s) \in A \times T$ occurs in both matches or (II) a position $(a,s) \in A \times T$ in $\pi$ corresponds to a position $(Q,s)  \in \shA \times T$ in $\pi'$, with $a \in \Ran(Q)$. Since $g$ is winning, eventually $\aut^{\f}$ will enter the alternating mode and thus situation (I) will be the case for all but finitely many initial rounds, implying that $\exists$ wins the match $\pi$.


\subsubsection{From formulae to automata}

We are now ready to introduce our projection construction for $\wmso$-automata and show that the class of tree languages that they recognize is closed under finitary
projection.
\begin{definition}\label{DEF_fin_projection}
Let $\aut = \tup{A,\Delta,\Omega,a_I}$ be a $\wmso$-automaton on alphabet $\p(\prop \cup \{p\})$, with $p \not\in P$. Let $\mathbb{A}^{\f}$
denote its finitary construct.
We define the $\wmso$-automaton ${{\exists}_F p}.\mb{A} := \langle A^{\f}, a_I^{\f},
\DeltaProj, \Omega^{\f}\rangle$ on alphabet $\p(\prop)$ by putting
\begin{eqnarray*}
\DeltaProj(q,c) &:=& \left\{
	\begin{array}{ll}
        \Delta^{\f}(q,c) & q\in A \\
		\Delta^{\f}(q,c) \vee \Delta^{\f}(q,c\cup\{p\}) & q \in \shA.
	\end{array}
\right.
\end{eqnarray*}
%The automaton ${{\exists}_F p}.\mb{A}$ is called the \emph{finitary projection construct of $\mb{A}$ over $p$}.
\end{definition}
\begin{proposition}\label{PROP_fin_projection}
For every $\wmso$-automaton $\aut$ on alphabet ${\p (\prop \cup \{p\})}$, with $p \not\in P$, we have that $\trees({{\exists}_F p}.\mb{A}) = {{\exists}_F p}.\trees(\mb{A})$.
\end{proposition}

\begin{proof} %The key is to observe
For the inclusion from left to right, first observe that ${{\exists}_F p}.\mb{A}$ is defined in terms of $\aut^{\f}$ and thus the properties stated in Proposition \ref{PROP_facts_finConstr} hold for ${{\exists}_F p}.\mb{A}$ as well. In particular, given a ${\p (\prop)}$-tree $\model$, any winning strategy $f$ for $\exists$ in $\mathcal{A}({{\exists}_F p}.\mb{A}, \model$) from position $(a_I^{\f},s_I)$ can be assumed to be functional and finitary in $\shA$. We can use such a strategy to guess a finitary $p$-variant of $\model$ as follows. First, by functionality for each node $s$ there is at most one position $(Q_s,s)$, with $Q_s \in \shA$, that is reachable in any $f$-guided match. From each such position, let $\val_{Q_s,s} \colon A^{\f} \to \pw (R[s])$ be the valuation suggested by $f$. We let $X_p$ be the set of nodes $s$ for which $\val_{Q_s,s}$ makes the disjunct $\Delta^{\f}(Q_s,\tscolors(s)\cup\{p\})$ of $\DeltaProj(Q_s,\tscolors(s))$ true: intuitively, these are the nodes on which ${{\exists}_F p}.\mb{A}$ behaves ``as if they were labeled with $p$''. Since $f$ is finitary in $\shA$, the $p$-variant $\model'$ of $\model$ given by labeling the nodes in $X_p$ with $p$ is finitary. One can readily verify that $\aut^{\f}$ (and thus $\aut$ by Proposition \ref{PROP_facts_finConstr}) accepts $\model'$ by letting $\exists$ playing the strategy $f$ in $\mathcal{A}(\aut^{\f},\model')$.

For the inclusion from right to left, let $\model'$ be a finitary $p$-variant of some ${\pw (\prop)}$-tree $\model$ and suppose that $\exists$ has a winning strategy $f$ for $\mathcal{A}(\aut,\model')$ from position $(a_I,s_I)$. We now sketch how $\exists$ is able to win any match $\pi$ of $\mathcal{A}({{\exists}_F p}.\mb{A},\model)$ from position $(a_I^{\f},s_I)$. The idea is to enforce that, at each round of $\pi$, $\exists$ assigns macro-states only to the nodes rooting a subtree of $\model'$ where the labeling $p$ appears, and $A$-states to the others. Using the information given by $f$, $\exists$ can make this assignment so that any visited position of shape $(Q,s) \in \shA \times T$ is such that $(a,s)$ is winning for $\exists$ in $\mathcal{A}(\aut,\model')$, for each $a \in \Ran(Q)$. In particular, the assignment of $\exists$ will make true the disjunct $\Delta^{\f}(Q,\tscolors(s)\cup\{p\})$ of $\DeltaProj(Q,\tscolors(s))$ if $s$ is labeled with $p$ in $\model'$, and the disjunct $\Delta^{\f}(Q,\tscolors(s))$ otherwise. Since $\model'$ is a \emph{finitary} $p$-variant, player $\exists$ will be required to assign macro-states to only finitely many nodes encountered along the play, and $\pi$ will eventually arrive to a position from which no node labeled with $p$ is reachable. At that point, $\exists$ allows ${{\exists}_F p}.\mb{A}$ to switch from the non-deterministic to the alternating mode. By construction, the match $\pi$ now moves to a position $(a,s) \in A \times T$ that is winning for $\exists$ in $\mathcal{A}(\aut,\model')$. It is also winning in $\mathcal{A}({{\exists}_F p}.\mb{A},\model)$, because $\model'$ agrees with $\model$ on nodes without labeling $p$ and by definition $\DeltaProj(a,\tscolors(s)) = \Delta(a,\tscolors(s))$. %\emph{not} labeled with $p$,  and in this phase she maintains a bundle $\pi_{a_1},\dots,\pi_{a_n}$ of shadow matches of $\mathcal{A}(\aut,\model')$.  --- the idea is that, as long  We can define  where $\model'$ is on alphabet ${\p (\prop)}$. where $\exists$ plays according to $g$. As long as such a match visits positions of the form $(R,s) \in \shA \times T$, we are able to maintain a bundle of matches of $\mathcal{A}(\aut,\model)$, one for each $a \in \Ran(R)$, where the strategy to play is defined in terms of $g$. Since $g$ is winning, eventually it allows $\mb{A}^{\f}$ to enter the alternating mode, say in position $(a,s) \in A \times T$. At that point we can just keep the match in the bundle which is at position $(a,s)$ --- which exists by construction --- and keep copying the strategy $g$.
\end{proof}

We have now in position to show our characterization result.
% \medskip

\begin{proof}[Proof of Theorem \ref{t:mt2}, direction $(\Rightarrow)$]
By induction we prove that for every $\varphi \in \wmso$ there is a $\wmso$-automaton $\aut_\varphi$ such that $\ext{\varphi} = \trees(\aut_\varphi)$. We focus on two %two non-trivial
inductive cases.
 %\begin{itemize}
   %\item

  \indent If $\varphi = \neg \psi$, let $\aut_{\psi}$ be the $\wmso$-automaton for $\psi$ given by inductive hypothesis. As $\olque$ is closed under Boolean duals, we can define the complementation $\overline{\aut_{\psi}}$ of $\aut_{\psi}$ as in Definition \ref{d:caut}. Notice that $\overline{\aut_{\psi}}$ is indeed a $\wmso$-automaton, satisfying the \textbf{(weakness)} and \textbf{(continuity)} conditions in virtue of their self-dual nature. Proposition \ref{PROP_complementation} yields the complementation lemma allowing to conclude that on trees $\ext{\neg \psi} = \trees(\overline{\aut_{\psi}})$.
   %\item

\indent   If $\varphi = \exists p.\psi$, let $\aut_{\psi}$ be the automaton given by inductive hypothesis. By semantics of $\wmso$, on trees $\ext{\exists p. \psi} = {{\exists}_F p}.\ext{\psi}$ and thus $\ext{\exists p.\psi} = \trees({{\exists}_F p}.\aut_{\psi})$ by Proposition \ref{PROP_fin_projection}.
 %\end{itemize}
\end{proof} 



\subsubsection{From automata to formulae}


In this section we show the other direction of Theorem \ref{t:mt2}, completing
the automata characterization of $\yvWMSO$ on tree models.
The argument is reminiscent of the one showing that $\MSO$-automata can be
translated into equivalent formulas of $\MSO$~\cite{Walukiewicz96}.
%It consists of two steps. First, it is shown that for every $\yvWMSO$-automaton $\bbA$ there is an effectively constructible formula in a suitably defined $\yvF$-fragment of a fixpoint extension of $\lque$. Second, it is verified that for every formula of the aforementioned  $\yvF$-fragment  there is an effectively constructible equivalent  $\yvWMSO$-formula.
We start by introducing a fragment of a fixpoint extension
of $\lque$ and show how it embeds into $\yvWMSO$.

\begin{definition}
The fixed point logic $\mlque$ on $\prop$ is given by:
% {\footnotesize%
% $$
% \varphi ::= p(x) \mid x=y \mid R(x,y) \mid \exists x.\varphi \mid \qu x.\varphi \mid \lnot\varphi \mid \varphi \land \varphi \mid \mu p.\varphi(p,x)
% $$}%
\begin{align*}
\varphi ::=\ & p(x) \mid x=y \mid R(x,y) \mid \lnot\varphi \mid \varphi \land \varphi \mid\\
& \exists x.\varphi \mid \qu x.\varphi \mid \mu p.\varphi(p,x)
\end{align*}
%
where $p\in\prop$, $x,y\in\fovar$; moreover $p$ occurs only positively in $\varphi(p,x)$ and $x$ is the only free variable in $\varphi(p,x)$.
\end{definition}

% Analogously to the modal $\mu$-calculus, the fixed point logic $\mlque$
% is  obtained by adding to $\lque$ the following rule for constructing
% fixed point formulas.
% \yvwarning{Why use `$P$' instead of `$p$'?}\fcnote{My fault/choice, will explain by mail}
%  \begin{itemize}
%  \item Given $P \in \prop$, $x\in\fovar$ and $\phi(P, x)$ with only positive occurrences of $P$ and no free variable other than $x$, $\mu P. \phi(P, x)$ and $\nu P. \phi(P, x)$ are formulas of $\mlque$.
%  \end{itemize}

%The semantics of the fixpoint formulas $\mu P. \phi(P, x)$ and $\nu P. \phi(P, x)$ is the expected one.
The semantics of $\mu p. \phi(p, x)$ is the expected one: given an LTS $\model$ and $s \in T$,  $\model \models \mu p. \phi(p, s)$ iff $s$ is in the least fixpoint of the operator
%$\phi^\model_P$ that maps any $S \subseteq T$ into $\{t \in T \mid \model[P \mapsto S] \models \phi(P, t) \}$.
$\phi^\model_p(S) := \{t \in T \mid \model[p \mapsto S] \models \phi(p, t) \}$.%The semantics of $\nu P. \phi(P, x)$ is dually defined by considering the greatest instead of the least fixpoint of $F_\phi$.


%Formulas of $\mlque$ may be also classified according to their alternation depth as it happens for the modal $\mu$-calculus.
%The alternation-free fragment of $\mlque$ is thence defined as the collection of $\mlque$-formulas $\phi$
%without nesting of greatest and least fixpoint operators, i.e. such that, for any two subformulas $\mu P.\psi_1(P,y)$ and $\nu Q. \psi_2(Q,z)$, predicates $P$ and $Q$ do not occur free respectively in $\psi_2(Q,z)$ and $\psi_1(P,y)$.

%
%
\begin{definition}
Given $p \in \prop$, we say that $\varphi \in \mlque$ is
\begin{itemize}
\item \emph{monotone in $p$} iff for every LTS $\model=\tup{T,R,\tscolors, s_I}$ and assignment $\ass$, if $\model, \ass \models \varphi$ and $ \{s \in T \mid p\in \tscolors(s)\} \subseteq E$, then $\model[p \mapsto E], g\models \phi$,

\item %\emph{continuous in $P$} iff for every LTS $\model$ and assignment $\ass$ there exists some finite $S \subseteq_\omega V(P)$ such that
    \emph{continuous in $p$} iff for every LTS $\model=\tup{T,R,\tscolors, s_I}$ and assignment $\ass$ there is some finite $S \subseteq_\omega \{s \in T \mid p\in \tscolors(s)\}$ such that $\model, \ass \models \varphi$ iff $\model[p \mapsto S], \ass \models \varphi$.
\end{itemize}
\end{definition}

We provide a definition of a fragment of $\mlque$ reminiscent of the one in
Theorem \ref{thm:olquecont}.
\begin{definition}
Given a set $Q\subseteq \prop$, the fragment $\cont{\mlque}{Q}$ is defined by the following rules:
% {\small%
% $$
% \varphi ::= \psi \mid q(x) \mid \exists x.\varphi \mid \varphi \land \varphi \mid \varphi \lor \varphi \mid \wqu x.(\varphi,\psi) \mid \mu p. \phi'(p, x)
% $$}%
\begin{align*}
\varphi ::=\ & \psi \mid q(x) \mid \varphi \lor \varphi \mid \varphi \land \varphi \mid\\
& \exists x.\varphi \mid \wqu x.(\varphi,\psi) \mid \mu p.\varphi'(p,x)
\end{align*}
%
where $q \in Q$,
%$\psi \in \mlque(\prop\setminus Q)$,
$\psi \in \mlque$ is $Q$-free,
$\wqu x.(\varphi,\psi)
:= \forall x.(\varphi(x) \lor \psi(x)) \land \dqu x.\psi(x)$ and $\phi'(p,x)$
is a formula, with only $x$ free and $p \in \prop$, which belongs to
$\cont{\mlque}{Q \cup\{p\}}$.
\end{definition}

%We then verify that formulas in $\cont{\mlque}{A}$ are (semantical) continuous in $A$. The proof is
By combining the argument for the proof of Proposition \ref{thm:ofocont} and
the one used in proving the analogous Lemma 1 in~\cite{Fontaine08}, we can thence obtain the following:
\begin{proposition}\label{lem:colqueiscont_mu}
If $\varphi \in \cont{\mlque}{Q}$ then $\varphi$ is continuous in (each element of) $Q$.
\end{proposition}
%\begin{proof} First, notice that If $\varphi \in \cont{\mlque}{A}$ then $\varphi$ is monotonous  in (each predicate from) $A$. %This is proved as for
%The proof goes then by induction on the complexity of $\varphi$. For the all the cases except the fixpoint one, the proof is the same as the one for Lemma \ref{lem:colqueiscont}. For $\phi=\mu Q. \phi'(Q, x)$, with $\phi'(Q,x) \in \cont{\mlque}{A \cup\{Q\}}$, the argument is the same as the one in the proof of Lemma 1 in \cite{Fontaine08}.
%\end{proof}
%As for $\MC$, we define

\begin{definition}
The fragment $\clque$ of $\mlque$ is given by restricting the fixpoint operator to the continuous fragment:
% {\footnotesize%
% $$
% \varphi ::= p(x) \mid x=y \mid R(x,y) \mid \exists x.\varphi \mid \qu x.\varphi \mid \lnot\varphi \mid \varphi \land \varphi \mid \mu p.\varphi'(p,x)
% $$}%
\begin{align*}
\varphi ::=\ & p(x) \mid x=y \mid R(x,y) \mid \lnot\varphi \mid \varphi \land \varphi \mid\\
& \exists x.\varphi \mid \qu x.\varphi \mid \mu p.\varphi'(p,x)
\end{align*}
%
where $p\in\prop$, $x,y\in\fovar$, $\varphi'(p,x) \in \cont{\mlque}{\{p\}} \cap \clque$ is positive in $p$ and $x$ is its only free variable.% in $\varphi'$.
\end{definition}

% The fragment $\clque$ of $\mlque$  is obtained
% %by restricting the use of the least fixpoint operator to the continuous fragment. It is obtained
% by adding to $\lque$ the following restricted rule for fixpoint formulas.
%  \begin{itemize}
%  \item Given $P\in\prop$, $x\in\fovar$ and $\phi(P, x)$ with only positive occurrences of $P$ and only $x$ free,
% if $\phi(P,x) \in \cont{\mlque}{\{P\}}$, then $\mu P. \phi(P, x) \in \clque$.%$\mu P. \phi(P, x)$ is also a $\clque$-formula.
%  %if $\phi(P,x)$ is a $\clque$-formula that belongs to $\cont{\mlque}{\{P\}}$, then $\mu P. \phi(P, x)$ is also a $\clque$-formula.
%  \end{itemize}

%
%
%
%
%The logic $\mglque$ can be given a semantic in terms of evaluation games extending the one given  in \cite{BerwangerG01} for $\mgfoe$ by adding rules for the generalized quantifier.
%We present it just for $\qu$, and treat the rules for $\dqu y. \phi(\overline{x},y)$ as derived from the equivalent formula $\lnot \qu y. \lnot\phi(\overline{x},y)$
%the universal being treated as the alternation-free fragment.
%As usual, we assume that any predicate is bounded by at most one fixpoint operator
%%, any if a predicate is bounded, then the fixpoint operator bounding it is unique,
%and that bounded and free predicates are pairwise distinct.%\fzwarning{In the table: why not a clause for $\neg$, $\vee$, $\wedge$? Meaning of $\eta$, ; and :?}
%                             \begin{table}[h]
%                              \centering
%                            \begin{tabular}{|l|c|l|c|}
%                             \hline
%                              % after \\: \hline or \cline{col1-col2} \cline{col3-col4} ...
%                              Position & Player & Admissible moves & Parity\\
%                               \hline % \hline
%                           %  $( ; \overline{x}: \overline{a})$ & $\forall$ & $\{B \subseteq T \mid |B| \geq \aleph_0 \}$ & $-$ \\
%                           %   $B \subseteq T $ & $\exists$ & $\{(\lnot \phi(\overline{x},y); \overline{x}: \overline{a}, y:b)\ |\ b \in B \}$ & $-$\\
%                          %     \hline
%                            $(\qu y. \phi(\overline{x},y); \overline{x} \mapsto \overline{a})$ & $\exists$ & $\{B \subseteq T \mid |B| \geq \aleph_0 \}$ & $-$ \\
%                              $B \subseteq T $ & $\forall$ & $\{(\phi(\overline{x},y); \overline{x} \mapsto \overline{a}, y \mapsto b)\ |\ b \in B \}$ & $-$\\
%                              \hline
%%                              $(\mu P. \phi(P, x); x \mapsto a)$ & $\exists$ & $\{(\phi(P, x);  x: a)\}$ & $1$ \\
%%                             % \hline
%%                              $(\nu P. \phi(P, x); x: a)$ & $\exists$ & $\{(\phi(P, x);  x: a)\}$ & $0$ \\
%%                              %\hline
%%                              $(P(y); y: a)$ & $\exists$ & $\{(\eta P.\phi(P, x); x: a)\}$ & $-$ \\
%%
%%                              \hline
%                            \end{tabular}
%                             \caption{The new rules in the evaluation game for $\mglque$.
%                          }
%                             \label{mufo_game}
%                            \end{table}
%
% By a straightforward adaption of the corresponding proof for $\mgfoe$ in \cite{BerwangerG01},
% we obtain:
%
% \begin{theorem}
% For every model $\model$, and every formula $\mglque$-formula $\phi(x)$ with one free variable, then
% $\model \models \phi(n)$ iff $\exists$ has a winning strategy in $\mc{E}(\varphi(x),\model)@(\varphi(x); x \mapsto n)$, the evaluation game for $\phi(x)$ and $\model$ when evaluating $x$ at the node $n$.\end{theorem}
%

 %%%%%%%%
 %We now recall a useful property of fixpoints and continuity. Let $\phi(P,x)$ a formula with only $x$ free.
\noindent Given an LTS $\model$ and $p\in \prop$, for every ordinal $\alpha$ we define: %by induction the following sets:
 %\fcwarning{Why not $\phi^0(\emptyset):= \emptyset$?}
 \begin{itemize}
\item $\phi^0_p(\emptyset):= \emptyset$,
%\{ s \in T \mid \model[P \mapsto \emptyset] \models \phi(P, s)\}$,
\item $\phi_p^{\alpha+1}(\emptyset):= \{ s \in T \mid \model[p \mapsto \phi_p^\alpha(\emptyset)] \models \phi(p, s)\}$,
\item $\phi_p^{\lambda}(\emptyset):= \bigcup_{\alpha < \lambda} \phi_p^{\alpha}(\emptyset)$, with $\lambda$ limit.
\end{itemize}
%We state $\phi^{-1}(\emptyset):=\emptyset$.
If $\phi$ is monotone in $p$, then $\phi_p^{\beta+1}(\emptyset)= \phi_p^{\beta}(\emptyset)$, for some ordinal $\beta$. Also, the set $\phi_p^{\beta}(\emptyset)$ is the least fixpoint of $\phi^\model_p$ (see e.g. \cite{ArnoldN01}).



%A formula $\phi(P, x)$ is said to be \emph{constructive} in $P$ if its least fixpoint is reached in at most $\omega$ steps.
%, i.e., if for every model $\model$, the least fixpoint of $F_\phi$ equals to $\bigcup_{\alpha < \omega} \phi^{\alpha}(\emptyset)$.
 %From a local perspective, this means that
 A formula $\phi(p, x)$ is \emph{constructive} in $p$ if for every model $\model$,  every node $s \in T$, if $\model \models\mu p. \phi(p,s)$, then $s\in \phi_p^{i+1}(\emptyset)$, for some $i< \omega$.
The next proposition is easily verified:% by the fact that Scott proved in \cite{Fontaine08} for the modal $\mu$-calculus but that generalizes to $\mglque$ as well, states that continuous formulas are constructive.



%\afwarning{Verify the claim and that Gaelle's argument REALLY goes thorough also here.}
%\yvwarning{Do not attribute to Gaelle, it is obvious that a Scott continuous map reaches fixpoint in $< \omega$ steps}
\begin{proposition}\label{prop:constructivity}
Let $\phi(p,x)$ be a $\mlque$-formula with only $x$ free. If $\phi(p,x)$ is continuous in $p$, then for every LTS $\model$, and every node $s \in T$, there is $i < \omega$ such that
\[\model \models \mu p. \phi(p,s) \text{ iff } s \in \phi_p^{i+1}(\emptyset).\]
\end{proposition}

By Proposition~\ref{prop:constructivity} and the fact that sets $\phi_p^{i+1}(\emptyset)$ are essentially defined as finite unfoldings, we obtain the following.%\fcwarning{More intuition on this?}

\begin{proposition}\label{prop:cor_constructivity}
Let $\phi(p,x)$ be a $\mlque$-formula continuous in $p$ with only $x$
free. Let $\model$ be an LTS, and $s \in T$. Then
$\model \models \mu p. \phi(p,s)$ iff there is a finite set $p^\model \subseteq T$ such that $s\in p^\model$ and $\model[p\mapsto p^\model] \models \phi(p,t)$  for every $t \in p^\model$.
\end{proposition}
% \begin{proof}
% For the direction from left to right, assume that $\model \models \mu P. \phi(P,s)$.  By Proposition~\ref{prop:constructivity}, we know that  there is $i< \omega$ such that $\model[P \mapsto \phi^i(\emptyset)] \models \phi(P, s)$. The set $\phi^i(\emptyset)$ need not to be finite. However,
% using this information, we are going construct a finite tree whose nodes $t$ are labelled by finite sets $X^m_j$, where $m$ is a node of $\model$ and $j \leq i$, satisfying the following condition:
% \begin{enumerate}
%\item  if $t$ is the root, then $t$ is labelled by $X_i^s$,
%\item  if $t$ is labelled by $X_j^m=\{s_1, \dots, s_\ell\}$ and $j>0$, then $t$ has $\ell$  children and for every $s_i \in X_j^m$ there is an unique child $t'$ of $t$ labelled by $X_{j-1}^{n_i}$ where $m$ is a nodes,
%%\item if $s$ is labelled by $X_j^m$ and $j=-1$, then $X_j^m=\emptyset$,
%\item for every node $t$ of the tree, if $t$ is labelled by $X_j^m$, then it holds that $X_j^m \subseteq \phi^{j}(\emptyset)$.
%\end{enumerate}
%If we verify that $\model[P\mapsto P^\model] \models \phi(P,s)$ holds by taking as $P^\model$ the union of all labels of the nodes of the constructed tree, we can conclude for the proof of this direction.
%
%As starting point of the inductive construction, we start by the empty tree.  Recall that we know that  $\model[P \mapsto \phi^i(\emptyset)] \models \phi(P, s)$. Since $\phi(P,x)$ is continuous in $P$, there is a finite set $X^s_i \subseteq \phi^i(\emptyset)$ such that $\model[P \mapsto X^s_i] \models \phi(P, s)$. We then add a root to our tree and label it by $X^s_i$.
% Assume that at a leaf $s$ of our tree is labelled by $X^m_j$, for some $j < i$. If $X^m_j$ is empty, than we stop, else we proceed as follows. We know that $X^m_j\subseteq \phi^{j}(\emptyset)$. This means that $\model[P \mapsto \phi^{j-1}(\emptyset)] \models \phi(P, r)$, for every $r \in X_j^m$. By continuity, for each such $r$, there is a finite set $X^m_{j-1} \subseteq  \phi^{j-1}(\emptyset)$ such that $\model[P \mapsto X^m_{j-1}(\emptyset)] \models \phi(P, r)$. For each $r \in X^m_j$ we thus add a child to $m$ and label it with $X^r_{j-1}$. By definition of $\phi^{i+1}(\emptyset)$, the tree is finite. Let $X$ be the union of all labels of the constructed tree. $X$ is finite, and by monotonicity of $\phi(P,x)$ we have that for every $m \in X \cup \{s\}$, $\model[P \mapsto X \cup \{s\}] \models \phi(P,m)$.
%
%For the other direction, it is enough to notice that the smallest finite set $P^\model \subseteq T$ such that $\model[P\mapsto P^\model] \models \phi(P,s)$ and $\model[P\mapsto P^\model] \models \phi(P,m)$  for every $m \in P^\model$ is the least fixpoint of the function that maps any $S \subseteq T$ into $\{t \in T \mid \model[P \mapsto S] \models \phi(P, t) \}$.
%%the idea is the following. By assumption there is a finite set $P^\model \subset T$ such that $\model[P\mapsto P^\model] \models \phi(P,n)$ and $\model[P\mapsto P^\model] \models \phi(P,m)$  for every $m \in P^\model$. The winning strategy for \'Eloise  in $\mc{E}(\mu P.\varphi(P,x),\model)@(\mu P.\varphi(P,x); x \mapsto n)$
%%is thus define as the composition of all winning strategies in $\mc{E}(\varphi(P,x),\model[P \mapsto P^\model]))@(\varphi(P,x); x \mapsto m)$
%% for $m \in P^\model$.
% \end{proof}

Proposition \ref{prop:cor_constructivity} naturally suggests the following translation $\mgFOETr{\cdot}:\mlque\to\wmso$. It is given homomorphically in predicates, Booleans and first-order quantifiers and:
\begin{itemize}
% \begin{multicols}{2}
\itemsep 0 pt
% \item $\mgFOETr{P(x)}=P(x)$,
% \item $\mgFOETr{R(x,y)}=R(x,y)$,
% \item $\mgFOETr{x\foeq y}= (x \foeq y)$,
% \item $\mgFOETr{\varphi \land \psi}=\mgFOETr{\varphi} \land \mgFOETr{\psi}$,
% \item $\mgFOETr{\lnot \varphi}= \lnot \mgFOETr{\varphi}$,
% \item $\mgFOETr{\exists x. \varphi}=\exists x. \mgFOETr{\varphi}$,
\item $\mgFOETr{ \qu x. \varphi} := \forall p.\exists x. (\lnot p(x) \land \mgFOETr{\varphi})$,
% \end{multicols}
\item $\mgFOETr{\mu p. \varphi(p,x)} := \exists p ( p(x) \land \forall y ( p(y) \to \mgFOETr{\varphi(p,y) }))$.
\end{itemize}

%Note that in $\mgFOETr(\mu P. \varphi(P,x))$, the predicate $P$ which occurs in $\mgFOETr(\varphi) $ is bounded by the outermost second order existential quantifier.

%The following theorem %, which is the analogous of Theorem \ref{thm:contransweak} but for $\mlque$,
%is then an immediate corollary of Proposition \ref{prop:cor_constructivity}.

\begin{proposition}\label{thm:guard_wmso}
Let $\phi$ be a $\clque$-formula, $\model$ an LTS and $\ass$ an assignment.
Then $\model, \ass \models \varphi$ iff $\model, \ass \models \mgFOETr{\varphi}$.
%the following two conditions are equivalent:
%\begin{enumerate}\itemsep 0pt \item $\model, \ass \models \varphi$, \item $\model, \ass \models \mgFOETr{\varphi}$. \end{enumerate}
\end{proposition}
\begin{proof}
The proof is by induction on $\varphi$. The least fixpoint case is handled by applying Proposition \ref{prop:cor_constructivity}.
%
%Let therefore consider $\phi$ is of the form $\mu P. \psi(P,x)$. Without loss of generality, that bounded and free predicate variables are distincts.
%We first show that $(1)$ implies $(2)$. Since $\model , \ass \models \varphi$, \'Eloise has a winning strategy $f$ in $\mc{E}(\phi,\model)@(\varphi,s_I, \ass)$.
%Define $P^\model$ to be the set of node $n \in T$ such that there is a (partial) match $\pi'$ that
%%
%\begin{enumerate}
%\itemsep 0pt
%\item is consistent with $f$, and such that
%\item every position of $\pi'$ is of the form $(\gamma,m, \ass')$, with  $P$ active in $\gamma$, and
%\item the last position of $\pi'$ is of the form $(\varphi, n, \ass')$.
%\end{enumerate}
%%
%The first observation is that since $f$ is a winning strategy, all $f$-consistent matches are finite. Moreover for every position of $\pi'$ is of the form $(\psi(P,x),m, \ass')$, we have that $\model[P \mapsto P^{\model}], \ass' \models \psi(P,m)$. We construct inductively a finite tree labelled by pairs $(x, X)$ where $x$ is a node of $\model$ and $X$ is a finite set of nodes of $\model$ as follows. First, because $\model[P \mapsto P^{\model}] , \ass \models \psi(P,x)$, so there is a finite subset $X \subseteq P^\model$ such that $\model[P \mapsto X_1] , \ass \models \psi(P,x)$. Thus we color the root with $(n, X)$. Now, assume we are given a leaf colored by $(y,Y)$. Consider an enumeration $x_1, \dots, x_k$ of $Y$. For every $i \leq k$, we add a child to $(y,Y)$ labelled by $(x_i, X_i)$ where $X_i$ is given by the fact that since $\model[P \mapsto P^{\model.x_i}] , \ass \models \psi(P,x)$, there is a finite set $X_i$ of nodes in $P^{\model.x_i}$
%
%%the only player who picks successor in a partial match $\pi'$ defined as above is \'Eloise. As a consequence of K\"onig's Lemma, $P^\model$ is finite.
%%
%%By using the induction hypothesis, it is easy to check that $\model[x \mapsto s_I, P \mapsto P^\model] \models P(x) \land \forall y ( P(y) \to ST_y(\varphi) )$.
%%
%%For the other direction, the idea is the following. Because $\model[x \mapsto s_I] \models ST_x(\varphi)$,
%%there is a finite set $P^\model$ such that $\model[x \mapsto s_I, P \mapsto P^\model] \models P(x) \land \forall y ( P(y) \to ST_y(\varphi) )$. The winning strategy for \'Eloise  in $\mc{E}(\mu P.\varphi,\model)@(\mu P.\varphi,s_I)$
%%is thus define as the composition of all winning strategies in $\mc{E}(\varphi,\model[P \mapsto P^\model])@(\varphi,s)$ for $s \in P^\model$.
\end{proof}

By virtue of Proposition \ref{thm:guard_wmso}, we are able to conclude the proof of Theorem \ref{t:mt2} by showing the following statement.
\begin{proposition}\label{thm:wmsoauttof}
Every $\yvWMSO$-automaton can be effectively translated into an equivalent $\clque$-formula.
\end{proposition}
%
%
%
%The guarded fragment $\glque$ of $\lque$ is obtained by imposing that first-order and generalized quantifiers  are relativized to atomic formulas of the form $r(x,y)$. The fixpoint extension $\mglque$  of $\glque$ is thus defined by adding a fixpoint construction clause.
%Analogously to the modal $\mu$-calculus, the  $\yvF$-fragment is finally obtained by considering only $\mglque$ formulas without alternation of fixpoints and by imposing some continuity conditions on the fixpoint construction rule.
%
\begin{proof} The argument is
 essentially a refinement of the standard proof showing that any automaton in $\yvAut(\ofo)$ can be translated into an equivalent $\mu$-formula
$\xi_\aut$ (\emph{cf.} e.g. \cite{Ven08}).
The idea is the following. We see a $\yvWMSO$-automaton as a system of equations expressed in terms of $\lque$-formulas: each state corresponds to a monadic predicate variable and the odd/even parity of a state corresponds to the least/greatest fixpoint that we seek for the associated variable, etc. One then solves this system of equations via the same inductive procedure used to obtain the formula of the modal $\mu$-calculus from the system associated with an automaton in $\yvAut(\ofo)$ (see e.g. \cite{ArnoldN01} for a description of the solution procedure). Because of the \textbf{(weakness)} and \textbf{(continuity)} conditions on the starting $\wmso$-automaton $\aut$, it is thence possible to verify that the resulting fixpoint formula $\xi_\aut$ belongs to $\clque$.
%
%To complete the proof of the Theorem, we have to provide an effective truth-preserving translation from the  considered fragment of $\mglque$ into $\yvWMSO$. The key observation here is that continuous fixpoint are constructive (see \cite{Fontaine08}), that is a least-fixpoint formula in one free variable $\mu p. \phi(p,x)$ of the $\yvF$-fragment of $\mglque$ is true at $s$ in a model $\model$ if $s$ belongs to some finite approximant of the least fixpoint induced by $\phi(p.x)$. From this fact, it is then possible to verify that a formula $\mu p. \phi(p,x)$ of the $\yvF$-fragment of $\mglque$ is true at $s$ in a model $\model$ iff is $\phi(p,x)$ true at $s$ in $\model[p\mapsto P^\model]$, for some finite $P^\model$, and use this propriety to obtain the truth-preserving  translation into $\yvWMSO$.
%%by taking formulas without alternation such that fixpoint operators only bound
%%$\mglque$  is  obtained by adding to $\glque$ the following (semantic) rule for constructing fixed point formulas.
%%fragment
%%
%%
%%The main idea is to encode each state of the starting automaton as a propositional variable bounded by a greatest (if the parity is even) or least (if the parity is odd) fixpoint operator. The target of the translation is a suitable extension of first-order logic with fixpoint operators, denoted with $\mglque$. Thanks to the (weakness) and (continuity) conditions on $\wmso$-automata, we are able to infer that the target can be in fact restricted to a fragment of $\mglque$. Finally, this is proven to be included in $\wmso$ completing the proof of Theorem \ref{thm:wmso_autofor}.
\end{proof}
%%\btbs
%% \item Continue with more details according to the space that we decide to devote to this section.
%% \etbs 
