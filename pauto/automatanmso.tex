%!TEX root = ../00CFVZ_TOCL.tex
This section establishes an automata characterisation for $\nmso$ in terms of weak parity automata from the class $\AutW(\ofoe)$, which we henceforth call \emph{$\nmso$-automata}. We first focus on the direction from formulas to automata.

\begin{theorem}
\label{t:nmsoauto}
There is an effective construction transforming a $\nmso$-formula $\phi$
into a $\nmso$-automaton $\bbA_{\phi}$ that is equivalent
to $\phi$ on the class of trees.
That is, for any tree $\bbT$, $\bbA_{\phi}$ accepts $\bbT$ if and only if $\bbT \models {\phi}$
\end{theorem}

The proof for Theorem \ref{t:nmsoauto} will follow closely the steps for proving the analogous results for $\wmso$ (Theorem \ref{t:wmsoauto}). Again, the crux of the matter is to show that the tree languages recognised by $\nmso$-automata are closed under the relevant notion of projection. If for $\wmso$ this was finitary projection (Def. \ref{def:tree_finproj}), the notion mimicking $\nmso$-quantification is \emph{noetherian} projection.

\begin{definition}\label{def:tree_finproj}
Fix a set $\prop$ of proposition letters, $p \not\in P$ and a language $\trees$ of $\p (\prop\cup\{p\})$-labeled
trees.
The \emph{noetherian projection} of $\trees$ over $p$ is the language of $\p (\prop)$-labeled trees given as $\noetexists p.\trees \df \{\bbT \mid \text{ $\exists$ a noetherian $p$-variant } \bbT' \text{ of } \bbT \text{ with } \bbT' \in \trees\}$. A class $K$ of tree languages is \emph{closed under noetherian projection over $p$} if, for any language $\trees$ in $K$, also ${\noetexists p}.\trees$ is in $K$.
\end{definition} 

\subsection{Simulation theorem for $\nmso$-automata}\label{sec:simulation_nmso}

Just as for $\wmso$-automata, also for $\nmso$-automata the projection construction will rely on a simulation theorem, constructing a two-sorted automaton $\aut^{\noet}$ consisting of a copy of the original automaton, based on states $A$, and a variation of its powerset construction, based on macro-states $\shA = \pw(A \times A)$. For any accepted $\bbT$, we want any match $\pi$ of $\mc{A}(\aut^{\noet},\bbT)$ to split in two parts:
\begin{description}
  \item[(\textbf{Non-deterministic mode})] for finitely many rounds $\pi$ is played on macro-states, i.e. positions are of the form $\shA \times T$. The strategy of player $\exists$ is functional in $\shA$, i.e. it assigns \emph{at most one macro-state} to each node.
  \item[(\textbf{Alternating mode})] At a certain round, $\pi$ abandons macro-states and turns into a match of the game $\mc{A}(\aut,\bbT)$, i.e. all next positions are from $A \times T$ (and are played according to a non-necessarily functional strategy). %of shape $(a,t) \in A \times T$.
\end{description}
The only difference with the two-sorted construction for $\wmso$-automata is that, in the non-deterministic mode, the cardinality of nodes to which $\exists$'s strategy assigns macro-states is irrelevant. Indeed, $\nmso$'s finiteness is only on the vertical dimension: assigning an odd parity value to macro-states will suffice to guarantee that the non-deterministic mode processes just a well-founded portion of any accepted tree.

We now proceed in steps towards the construction of $\aut^{\noet}$. First, the following lifting from states to macro-states parallels Definition \ref{DEF_finitary_lifting}, but for the one-step language $\ofoe$ proper of $\nmso$-automata.

\begin{definition}\label{DEF_noetherian_lifting}
Let $\varphi \in {\ofoe}^+(A \times A)$ be of shape $\dbnfofoe{\vlist{T}}{\Pi}$ for some $\Pi \subseteq \shA$ and $\vlist{T} = \{T_1,\dots,T_k\} \subseteq \shA$. We define $\varphi^{\noet}$ as $\dbnfofoe{\lift{\vlist{T}}}{\lift{\Pi}} \in {\ofoe}^+(\shA )$, that means,
\begin{equation}\label{eq:unfoldingNablaofoe}
\varphi^{\noet} \ \df \ 
    \exists \vlist{x}.\big(\arediff{\vlist{x}} \land \bigwedge_{0 \leq i \leq n} \tau^+_{\lift{T}_i}(x_i)
\land 
    \forall z.(\arediff{\vlist{x},z} \lthen \bigvee_{S\in \lift{\Pi} } \tau^+_S(z))\big)
\end{equation}
\end{definition}

It is instructive to compare \eqref{eq:unfoldingNablaofoe} with its $\wmso$-counterpart \eqref{eq:unfoldingNablaolque}: the difference is that, because the quantifies $\qu$ and $\dqu$ are missing, the sentence does not impose any cardinality requirement, but only enforces functionality (Definition \ref{def:functionalsentenceofoe}).

%The idea of translation $(\cdot)^{\noet}$ is to encode at the one-step level the non-deterministic mode of $\aut^{\noet}$: the property to enforce is that $\varphi^{\noet}$ is \emph{functional} in $\shA$, that means, whenever $(D,\val \: A \to \wp(D)) \models \varphi^{\noet}$, then there is $\val'  \: A \to \wp(D)$ such that $(D, \val') \models \varphi^{\noet}$ and  $ \val'(q_1)\cap \val'(q_2) = \emptyset$ for all $q_1,q_2 \in \shA$. 

\begin{lemma}\label{lemma:automatafunctionalsentence}
Let $\varphi \in {\ofoe}^+(A \times A)$ and $\varphi^{\noet}\in {\ofoe}^+(\shA )$ be as in Definition~\ref{DEF_noetherian_lifting}. Then $\varphi^{\noet}$ is functional in $\shA$.
 \end{lemma}
\begin{proof}
Each element of $\lift{\vlist{T}}$ and $\lift{\Pi}$ is by definition either the empty set or a singleton $\{Q\}$ for some $Q \in \shA$. Then the statement follows from Proposition \ref{lemma:functionalsentenceofoe}.\end{proof}

 We are now ready to define the transition function for macro-states. The following adapts Definition \ref{PROP_DeltaPowerset} to the one step-language $\ofoe$ of $\nmso$-automata, and its normal form result, Theorem~\ref{thm:bnfofoe}. Also, we use the translation $(\cdot)^{\star}$ from Definition \ref{DEF_delta star}.
\begin{definition}\label{PROP_DeltaPowerset_noet}
Let $\aut = \tup{A,\tmap,\pmap,a_I}$ be a $\nmso$-automaton. Fix any $c \in C$ and $Q \in \shA$. By Theorem~\ref{thm:bnfofoe} there is a sentence $\Psi_{Q,c} \in {\ofoe}^+(A\times A)$ in the basic form $\bigvee \dbnfofoe{\vlist{T}}{\Pi}$, for some $\Pi \subseteq \shA$ and $T_i \subseteq A \times A$, such that $\bigwedge_{a \in \Ran(Q)} \tmap^{\star}(a,c) \equiv \Psi_{Q,c}$.
By definition, $\Psi_{Q,c} = \bigvee_{n}\varphi_n$, with each $\phi_{k}$ of shape $\dbnfofoe{\vlist{T}}{\Pi}$.
%
We put $\bmDe(Q,c) := \bigvee_{n}\varphi_n^{\noet}  \in {\ofoe}^+(\shA)$, where the translation $(\cdot)^{\noet}$ is as in Definition \ref{DEF_noetherian_lifting}.
\end{definition}

\noindent We have now all the ingredients for the two-sorted construction over $\nmso$-automata.

\begin{definition}\label{def:noetherianconstruct}
Let $\aut = \tup{A,\tmap,\pmap,a_I}$ be a {\nmso-automaton}. We define the \emph{noetherian construct over $\aut$} as the automaton $\aut^{\noet} = \tup{A^{\noet},\tmap^{\noet},\pmap^{\noet},a_I^{\noet}}$ given by %taking $A^{\noet} :=A \cup \shA$, $a_I^{\noet} := \{(a_I,a_I)\}$ and
\begin{gather*}
      % \nonumber to remove numbering (before each equation)
        A^{\noet} \ \df \  A \cup \shA \quad\quad\quad a_I^{\noet} \ \df \  \{(a_I,a_I)\} \quad\quad\quad \pmap^{\noet}(a) \ \df \  \pmap(a) \quad\quad\quad \pmap^{\noet}(R) \ \df \  1 \\
        \tmap^{\noet}(a,c) \ \df \  \tmap(a,c) \qquad \qquad \qquad 
        \tmap^{\noet}(Q,c) \ \ \df \ \  \bmDe(Q,c) \vee \! \! \! \! \bigwedge_{a \in \Ran(Q)} \! \! \! \tmap(a,c).
      \end{gather*}
\end{definition}
The construction is the same as the one for $\wmso$-automata (Definition \ref{def:finitaryconstruct}) but for the definition of the transition function for macro-states, which is now free of any cardinality requirement. %The definition of $\aut^{\noet}$ enforces its behaviour to be split according to the non-deterministic and alternating mode. Indeed, for any accepted $\bbT$, a match $\pi$ of $\agame(\aut^{\noet},\bbT)$ will visit positions involving macro-states only for finitely many initial rounds, because $\pmap^{\noet}[\shA] = \{1\}$. The alternating mode will be entered when, at a certain position $(R,s)\in \shA \times T$, the winning strategy for $\exists$ makes the disjunct $\bigwedge_{a \in \Ran(R)} \tmap(a,c)$ of $\tmap^{\noet}(R,c)$ true and then all successive positions only involve states from $A$. The next proposition fixes our desiderata on $\aut^{\noet}$. 

\begin{definition}\label{def:noetherianstrategy}
We say that a strategy $f$ in an acceptance game $\agame(\aut,\bbT)$ is \emph{noetherian} in $B \subseteq A$ when in any $f$-guided match there can be only finitely many rounds played at a position of shape $(q,s)$ with $q \in B$.
\end{definition}

%: in particular, \ref{point:finConstrStrategy} certifies the description that we did of the non-deterministic mode of $\aut^{f}$.

\begin{theorem}[\textbf{Simulation Theorem for $\nmso$-automata}]\label{PROP_facts_noetConstr} Let $\aut$ be a $\nmso$-automaton and $\aut^{\noet}$ its noetherian construct.
\begin{enumerate}[(i)]
  \itemsep 0 pt
  \item $\aut^{\noet}$ is a $\nmso$-automaton.\label{point:finConstrAut}
  \item For any $\bbT$, if $\eloise$ has a winning strategy in $\mathcal{A}(\aut^{\noet},\bbT)$ from position $(a_I^{\noet},s_I)$ then she has one that is functional in $\shA$ and noetherian in $\shA$.% (\emph{cf.} Definition \ref{def:StratfunctionalFinitary}).
  \label{point:finConstrStrategy}
  \item $\aut \equiv \aut^{\noet}$. \label{point:finConstrEquiv}
  \end{enumerate}
\end{theorem}
\begin{proof}
The proof follows the same steps as the one of Proposition \ref{PROP_facts_finConstrwmso}, minus all the concerns about continuity of the constructed automaton and any associated winning strategy $f$ being finitary. One still has to show that $f$ is noetherian in $\shA$ (``vertically finitary''), but this is enforced by macro-states having an odd parity: visiting one of them infinitely often would mean $\exists$'s loss.
\end{proof}

\begin{remark}
As mentioned, the class $\Aut(\ofoe)$ of automata characterising $\MSO$ \cite{Jan96} also enjoys a simulation theorem \cite{Walukiewicz96}, turning any automaton into a non-deterministic one. Given that the class $\AutW(\ofoe)$ only differs for the weakness constraint, one may wonder if the simulation result for $\Aut(\ofoe)$ could not actually be restricted to $\AutW(\ofoe)$, making our two-sorted construction redundant. This is actually not the case: not only Walukiewicz's simulation theorem \cite{Walukiewicz96} does not preserve the weakness constraint, but even in that case it would not serve our purposes. A fully non-deterministic automaton is instrumental in guessing a $p$-variant of any accepted tree, but it does not guarantee that the $p$-variant is also noetherian, as the two-sorted construct does.
\end{remark}

\subsection{From formulas to automata}

We can now conclude one direction of the automata characterisation of $\nmso$.

\begin{lemma}\label{PROP_noet_projection}
For each $\nmso$-automaton $\aut$ on alphabet $\p (\prop \cup \{p\})$,
we have that
$$\trees({{\exists} p}.\aut) \ \equiv\
{{\noetexists} p}.\trees(\aut).
$$
\end{lemma}
\begin{proof} The argument is the same as for $\wmso$-automata (Lemma \ref{PROP_fin_projection}). As in that proof, the inclusion from left to right relies on the simulation result (Theorem \ref{PROP_facts_noetConstr}): because $\aut$ can be assumed to be two-sorted, ${{\exists} p}.\aut$ is two-sorted too: its non-deterministic mode can be used to guess a noetherian $p$-variant of any accepted tree. \end{proof}

\begin{proof}[of Theorem \ref{t:nmsoauto}] As for its $\wmso$-counterpart Theorem \ref{t:wmsoauto}, the proof is by induction on $\varphi \in \nmso$. The boolean inductive cases are handled by the $\nmso$-versions of Lemma \ref{t:cl-dis} and \ref{t:cl-cmp}. The projection case follows from Lemma~\ref{PROP_noet_projection}.
\end{proof} 

\subsection{From automata to formulas} \label{sec:aut_to_formulas_nmso}

In this section we show the converse of Theorem
\ref{t:nmsoauto}, from automata to formulas. The analogous result for $\wmso$ (Section \ref{sec:aut_to_form_wmso}) passed through a first-order fixpoint logic. For $\nmso$, it is enlightening to go a different route, which gives another automata characterisation to $\nmso$-expressivity, in terms of {B\"{u}chi automata} \cite{Rab70}. 
%, and . One can think of a  $\baut$ as a parity automaton on the one-step language $\ofoe$ for which every winning strategy is functional (thus it is always in the ``non-deterministic mode'', \emph{cf.} Section \ref{sec:simulation_nmso}) and the parity map $\pmap$ can only take values in $\{0,1\}$ --- equivalently, there is a set $F$ of \emph{accepting states} (those with parity $0$) and $\baut$ accepts the input if one state from $F$ occurs infinitely often.

Rabin showed that (what we call) $\nmso$-automata are equivalent to B\"{u}chi automata on binary trees \cite{Rab70}. In the sequel, we generalise the result to arbitrary transition systems (Proposition \ref{PROP_Weak parity automata in buchinondet}). Next we will show how to encode the B\"{u}chi acceptance game as a $\nmso$-formula: the key insight is that winning conditions for this game are inherently simpler than for a generic parity game and require to check only well-founded portions of the model under consideration. Putting together these two results we achieve a translation from $\nmso$-automata to $\nmso$-formulas (Theorem \ref{thm:nmso_autofor}).

\subsubsection{From $\nmso$-automata to B\"{u}chi Automata.}


\begin{definition} A (non-deterministic) \emph{B\"{u}chi automaton} is a parity automaton $\baut = \tup{A,\tmap,\pmap,a_I}$ on the one-step language $\ofoe$ such that (i) for each $a \in A$, $\pmap(a) \in \{0,1\}$, and (ii) for each $a \in A$  and $c \in \pw P$, $\tmap(a,c)$ is in the basic form $\bigvee \posdbnfofoe{\vlist{T}}{\Pi}$ (Definition~\ref{def:monbasicformofoe}) with each $T_1,\dots,T_k$ and $S \in \Pi$ either empty or a singleton. 

In perhaps more familiar terms, the condition on $\pmap$ is equivalent to say that there is a set $F$ of  of \emph{accepting states} (those with parity $0$) and $\baut$ accepts an input $\model$ if one state from $F$ occurs infinitely often. Thus $\baut$ can be also presented as a tuple $\tup{A,\tmap,F,a_I}$.
\end{definition}

To emphasise their diversity, henceforth we shall present the acceptance conditions for B\"{u}chi automata in terms of $F$ rather than $\pmap$. 

\begin{remark} \label{rmk:Buchifunctional}
Non-determinism in the above definition is expressed by the requirement on the transition map. Semantically, it ensures that each $\tmap(a,c)$ is functional in $A$ (by Proposition \ref{lemma:functionalsentenceofoe}) and thus any winning strategy for $\eloise$ in the acceptance game $\agame(\baut,\model)$ is also functional (Definition \ref{def:StratfunctionalFinitary}).\end{remark}

Next we show that $\nmso$-automata can be simulated by B\"{u}chi Automata.

\begin{definition}
\label{DEF_BuchiPowersetConstruction}
Let $\aut = \tup{A,\tmap,\pmap,a_I}$ be a $\nmso$-automaton. We define the \emph{B\"{u}chi powerset construct} over
$\aut$ as the B\"{u}chi automaton $\aut^{\buc} = \tup{\shA, \bmDe ,F_{\pmap},a_I^{\noet}}$, where $\shA$, $\bmDe$ and $a_I^{\noet}$ are as in Definition \ref{def:noetherianconstruct} and $F_{\pmap} = \{Q \in \shA\ |\ \pmap(a)=0 \text{ for all } a \in \mathit{Ran}(Q)\}$.
\end{definition}

Observe that $\aut^{\buc}$ is indeed a B\"{u}chi automaton, because sentences $\bmDe(Q,c)$ are by construction of the required shape.

\begin{proposition}
\label{p:4:2}
\label{PROP_Weak parity automata in buchinondet}
Let $\aut$ be a $\wmso$-automaton and $\aut^{\buc}$ the
B\"{u}chi powerset construct over $\aut$.
We have that
    $   \autlang( \aut) =\autlang(\aut^{\buc})$.
\end{proposition}
\begin{proof} The proof is similar to the ones of Proposition \ref{PROP_facts_finConstrwmso}.\ref{point:finConstrEquiv} and Proposition \ref{PROP_fin_projection} in that we have to maintain a relationship between an $\aut^{\buc}$-match played in macro-states and (possibly more than one) $\aut$-match played in states. Thanks to Remark \ref{rmk:weak01}, we shall assume that the parity map $\pmap$ of $\aut$ ranges over $\{0,1\}$. 

For the inclusion from left to right, given $\bbT = \{ T,R,\V,s_I\}$ and a winning strategy $f$ for $\eloise$ in $\mc{G} = \agame(\aut,\bbT)@(a_I,s_I)$, we construct a winning strategy $f^{\buc}$ for $\eloise$ in $\mc{G}^{\buc} = \agame(\aut^{\buc},\bbT)(a_I^{\noet},s_I)$ in stages, while playing a match $\pi^{\buc}$ in $\mc{G}^{\buc}$. In parallel to $\pi^{\buc}$, we mantain a \emph{bundle} $\mc{B}$ of $f$-guided shadow matches in $\mc{G}$, under the induction hypothesis that, when playing at position $(Q,s)$ in $\pi^{\buc}$, the bundle consists of a match at position $(a,s)$ for each $a \in \Ran(Q)$. Now, this condition verifies when initialising $\pi^{\buc}$ at position $(a_I^{\fin},s_I)$ and the bundle as the single match at position $(a_I,s_I)$. Inductively, given a position $(Q,s)$ in $\pi^{\buc}$, by inductive hypothesis for all $a \in \Ran(Q)$ the position $(a,s)$ is winning, thus the strategy $f$ suggests a valuation $\val_{a,s}: A \rightarrow \p (\R{s})$ making $\tmap(a,\tscolors(s))$ true. We then define the suggestion $\val: \shA \rightarrow \p (\R{s})$ of $f^{\buc}$ from position $(Q,s)$ to be the assignment to a macro-state $R \in \shA$ of the set of nodes $\{\bigcap\limits_{\substack{(a,b) \in R,\\ a \in \Ran(Q)}}\{t\ \in \R{s} \mid t \in \val_{a,s}(b)\}$. By construction, $\val$ makes $\bmDe(Q,c)$ true and thus it is a legitimate move for $\eloise$. Moreover, it guarantees that any next position $(R,t)$ picked by $\abelard$ will maintain the inductive hypothesis with respect to the bundle of shadow matches, which can be advanced to positions $(b,t)$ for each $b \in \Ran(R)$.

With this definition $Q$ appears infinitely often in an $f^{\buc}$-guided match if and only if each $a \in \Ran(Q)$ appears in the shadow matches infinitely often. As shadow matches are winning and $\pmap$ ranges over $\{0,1\}$, it must be that $\pmap(a) = 0$, thus $Q$ is in $F_{\pmap}$.

For the direction from right to left, given $\bbT = \{ T,R,\V,s_I\}$ and a winning strategy $f^{\buc}$ for $\eloise$ in $\mc{G}^{\buc} = \agame(\aut^{\buc},\bbT)@(a_I^{\noet},s_I)$, we construct a winning strategy $f$ for $\eloise$ in $\mc{G} = \agame(\aut,\bbT)(a_I,s_I)$ in stages, while playing a match $\pi$ in $\mc{G}$. In parallel to $\pi$ we maintain an $f^{\buc}$-guided shadow match $\pi^{\buc}$ in $\mc{G}^{\buc}$, under the induction hypothesis that, when playing at position $(a,s)$ in $\pi$, the shadow match is in position $(Q,s)$ with $a \in \Ran(Q)$. This is true for initial positions $(a_I^{\fin},s_I)$ in $\pi^{\buc}$ and $(a_I,s_I)$ in $\pi$. Inductively, given a position $(a,s)$ in $\pi$ and $(Q,s)$ in $\pi^{\buc}$ with $a \in \Ran(Q)$, let $\val^{\buc} \: \shA \rightarrow \p (\R{s})$ be the suggestion of $f^{\buc}$ from $(Q,s)$. We let the suggestion $\val \: A \rightarrow \p (\R{s})$ of $f$ be the assignment of $b$ to the set of nodes $\bigcup_{b \in \Ran(R)} \{t \in \R{s} \mid t \in \val_{Q,s}(R)\} \cup  \{t \in \R{s} \mid t \in \val_{Q,s}(b)\}$. This makes $\tmap(a,s)$ true and moreover any next position $(b,t)$ picked by $\abelard$ can be mirrored by a position $(R,t)$ in the shadow match with $b \in \Ran(R)$, thus maintaining the inductive hypothesis. To see that such $f$ is winning, observe that if $b \in A$ is seen infinitely often during an $f$-guided match then, for some $R$ such that $b \in \Ran(R)$, $R$ is seen infinitely often during an $f^{\buc}$-guided match. Thus $R \in F_{\pmap}$, meaning that $\pmap(b) = 0$ and the match is won by $\eloise$.
\end{proof}

\subsubsection{Properties of B\"{u}chi automata.}\label{sec:propsBuchi}

In order to express acceptance for B\"{u}chi
automata in terms of $\nmso$-formulas, we need to formalise two key intuitions.
\begin{enumerate}
\item checking whether a B\"{u}chi automaton $\baut$
  accepts a tree $\bbT$ reduces to verifying a condition on prefixes of
  $\bbT$ (Proposition \ref{PROP_Buchi_finite_segments});
\item checking whether the intersection of the languages of two
   B\"{u}chi automata is non-empty can proceed via the
  construction of a finite sequence of well-founded trees with certain
  properties (Proposition \ref{PROP_Rabin_trap}).
\end{enumerate}

The idea is that a run of a B\"{u}chi automaton on a tree $\bbT$ can be split into several tasks concerning well-founded subtrees (and prefixes, which are just a particular kind of well-founded subtrees) of $\bbT$, and there is never the need to consider $\bbT$ as a whole. %We informally refer to this as the \emph{bounded information property} of non-deterministic B\"{u}chi automata.

\begin{definition}\label{DEF_accepting_sequence}
Let $\baut = \langle B,b_I,\tmap,F\rangle$ be a B\"{u}chi automaton and $\bbT$ a tree. Let $f$ be a surviving strategy for $\exists$ in $\mathcal{A}(\baut,\bbT)@(b_I,s_I)$. Let $\gamma \leq \omega$ be an ordinal. \emph{A $\gamma$-accepting sequence for $f$ over $\baut$ and $\bbT$} is a sequence $(E_i)_{i < \gamma}$ such that, for all $i < \gamma$:
\begin{enumerate}
\item $E_i$ is a prefix of $\bbT$;
\item $\mathit{Ft}(E_i) < \mathit{Ft}(E_{i+1})$;
\item for each $s$ in the frontier of $E_i$, there is a unique $a\in A$ such
   that $f$ is defined on position $(a,s)$; in addition, $a$ is in $F$.
\end{enumerate}
\end{definition}

Intuitively, for $k <\omega$, a $k$-accepting sequence for a surviving strategy
$f$ witnesses the fact that $f$ `behaves as' a winning strategy for $\exists$
in the prefix $E_k$ of $\bbT$. For each prefix $E_i$ in the sequence, the
condition that each $s \in \mathit{Ft}(E_i)$ is associated with a \emph{unique}
accepting state is because $f$ can be assumed to be
functional (Remark \ref{rmk:Buchifunctional}).

\begin{proposition}\label{PROP_Buchi_finite_segments}
Let $\baut = \langle B,b_I,\tmap,F\rangle$ be a B\"{u}chi automaton and $\bbT$ a tree. The following are equivalent.
\begin{itemize}
  \item Player $\exists$ has a winning strategy in $\mathcal{A}(\baut,\bbT)@(b_I,s_I)$.
  \item Player $\exists$ has a surviving strategy $f$ in $\mathcal{A}(\baut,\bbT)@(b_I,s_I)$ and there is an $\omega$-accepting sequence for $f$ over $\baut$ and $\bbT$.
\end{itemize}
\end{proposition}

For B\"{u}chi automata $\baut_1$ and $\baut_2$ and a tree
$\bbT \in L(\baut_1)\cap L(\baut_2)$, let $(G^1_i)_{i<\omega}$ and
$(G^2_i)_{i<\omega}$ be $\omega$-accepting sequences respectively for $\baut_1$
and $\baut_2$ on $\bbT$. We introduce the notion of \emph{$k$-trap} for
$\baut_1$ and $\baut_2$. The idea is that a $k$-trap is a finite sequence
$(E_i)_{i\leq k}$ witnessing some kind of interleaving of the sequences
$(G^1_i)_{i<\omega}$ and $(G^2_i)_{i<\omega}$ up to level $k$.

In this aim, we first introduce the following auxiliary notion. Let $\baut$ be a B\"{u}chi automaton and $\bbT$ a tree. Given a set of nodes $N \subset T$, we say that a strategy $f$ for player $\exists$ in $\mathcal{A}(\baut,\bbT)@(b_I,s_I)$ is \emph{surviving in $N$} if, for each basic position $(b,s) \in B\times N$ that is reached in some $f$-conform match, the marking $m$ suggested by $f$ makes $\tmap(b,\V(s))$ true in $\R{s}$.

\begin{definition}\label{DEF_Rabin_trap}
Let $\baut_1 = \langle B_1,b_I^1,\tmap_1,F_1\rangle$ and
$\baut_2= \langle B_2,b_I^2,\tmap_2,F_2\rangle$ be B\"{u}chi automata and
let $\bbT$ be a tree.
Given some fixed $k < \omega$, let $(E_i)_{i\leq k}$ be a sequence of prefixes
of $\bbT$ such that $E_0 = \{s_I\}$ and $E_i \varsubsetneq E_{i+1}$ for
all $i \leq k$.

We say that $\bbT$ and $(E_i)_{i\leq k}$ constitute a \emph{k-trap for
$\baut_1$ and $\baut_2$} if there exist
\begin{enumerate}
  \item a strategy $f_1$ for $\exists$ in $\mathcal{A}(\baut_1,\bbT)@(b_I^1,s_I)$
  which is surviving in $E_k$,
  \item a strategy $f_2$ for $\exists$ in $\mathcal{A}(\baut_2,\bbT)@(b_I^2,s_I)$ which is surviving in $E_k$,
  \item a $k$-accepting sequence $(G_i^1)_{i\leq k}$ for $f_1$ over $\baut^1$ and $\bbT$,
  \item a $k$-accepting sequence $(G_i^2)_{i\leq k}$ for $f_2$ over $\baut^2$ and $\bbT$,
\end{enumerate}
such that, for all $i < k$, the following conditions hold:
\begin{itemize}
  \item $\mathit{Ft}(E_i) \leq \mathit{Ft}(G^1_i) < \mathit{Ft}(E_{i+1})$;
  \item $\mathit{Ft}(E_i) \leq \mathit{Ft}(G^2_i) < \mathit{Ft}(E_{i+1})$.
\end{itemize}
We say that the strategies $f_1$ and $f_2$ \emph{witness} the k-trap for $\baut_1$ and $\baut_2$.
\end{definition}

%\begin{figure}[h]
%  % Requires \usepackage{graphicx}
%  \begin{center}
%  \includegraphics{FIG_trap.png}\\
%  \caption{initial part of a $k$-trap}\label{FIG_trap}
%  \end{center}
%\end{figure}

\begin{proposition}[\mrg \cite{Rab70}]\label{PROP_Rabin_trap}
Let $\baut_1$ and $\baut_2$ be B\"{u}chi automata and let $m$ be the
product of the cardinalities of their carriers.
If there exists an $m$-trap for $\baut_1$ and $\baut_2$ then
$\trees(\baut_1) \cap \trees(\baut_2) \neq \emptyset$.
\end{proposition}


\subsubsection{From B\"{u}chi automata to formulas} \label{sec:fromBuchiToformulas}
We are now ready to prove the converse of Theorem~\ref{t:nmsoauto}, i.e. the direction from automata to formulas.

\newcommand{\ThWeakAutToWFMSO}{
There is an effective procedure that given a $\nmso$-automaton returns an equivalent $\nmso$-formula.
}
\begin{theorem}\label{thm:nmso_autofor}
\ThWeakAutToWFMSO
\end{theorem}
\begin{proof}
Let $\aut$ be a $\nmso$-automaton and $\baut$ an
B\"{u}chi automaton which is equivalent to $\aut$, as in
Proposition \ref{p:4:2}.
Clearly then it suffices to come up with a formula $\phi$ in $\nmso$ that
holds in a tree $\bbT$ if and only if $\baut$ accepts $\bbT$.
Since $\nmso$-automata are closed under complementation, we are also
provided with a $\nmso$-automaton $\overline{\aut}$ recognizing
the complement of $\trees(\aut)$, and consequently with an
B\"{u}chi automaton $\overline{\baut}$ which is equivalent to
$\overline{\aut}$.
Our formula $\varphi = \varphi_{\baut,\overline{\baut}}$ depends on
both $\baut$ and $\overline{\baut}$.


More concretely, let $m$ be the product of the cardinalities of $B$ and
$\overline{B}$.
The formula $\varphi_{\baut,\overline{\baut}} \in
\nmso$ will express the existence of a strategy $f$ for $\exists$ and an
$m+1$-accepting sequence $(E_i)_{i \leq {m+1}}$ such that $f$ is functional and
surviving in $E_{m+1}$.
The key observation is that the encoding of $(E_i)_{i \leq {m+1}}$ and the associated surviving strategy into a formula only
requires variables for noetherian sets of nodes.
For this, the expressive power of $\nmso$ will suffice.

Proposition \ref{PROP_Buchi_finite_segments} will help showing one direction of
the equivalence, namely that, given a tree $\bbT$ and a winning strategy $f$
for $\exists$ in $\mathcal{A}(\baut,\bbT)$, the formula
$\varphi_{\baut,\overline{\baut}}$ is true in $\bbT$.
%
For the converse direction, we use the automaton $\overline{\baut}$
accepting the complement of the language of $\baut$. The idea is to
suppose by way of contradiction that $\overline{\baut}$ accepts a tree
$\bbT$ where $\varphi_{\baut,\overline{\baut}}$ is true.
Then by Proposition \ref{PROP_Buchi_finite_segments} there is an
$\omega$-accepting sequence $(E^{\delta}_i)_{i < \omega}$ witnessing the fact
that $\bbT$ is in $\trees(\overline{\baut})$.
The $\omega$-accepting sequence $(E^{\delta}_i)_{i < \omega}$ contains an
$m+1$-accepting sequence $(E^{\delta}_i)_{i \leq {m+1}}$. By the fact that
$\varphi_{\baut,\overline{\baut}}$ is true, we also have an $m+1$-accepting
sequence $(E_i)_{i \leq {m+1}}$. Then we can show that the two sequences witness
a trap for $\baut$ and $\overline{\baut}$ as in Definition
\ref{DEF_Rabin_trap}. But by Proposition \ref{PROP_Rabin_trap} this means
that the intersection of $\trees(\baut)$ and
$\trees(\overline{\baut})$ is non-empty, contradicting the fact
that $\overline{\baut}$ accepts the complement of $\trees(\baut)$.


The definition of $\varphi_{\baut,\overline{\baut}}$ essentially follows the
same line of reasoning as in \cite{Rab70}.
Given any state $b \in B$, we define by induction on $i < \omega$ a formula
$K_{i}^{b}(x)\in \nmso$, where no variable different from $x$ occurs free.
For the base case, we put $K_0^b(x) := \top$. Inductively, $K_{i+1}^{b}(x)$ is
given as a formula expressing the following situation (relative to a tree
$\bbT$):
\begin{itemize}
\item
Given a node $s$ on which $x$ is being evaluated, for each prefix $E$ of
$\bbT.s$, there is a prefix $E^{\prime}$ of $\bbT.s$ including $E$, and a
function $m_p : B\rightarrow \p (E)$, such that $\exists$ has a functional
strategy $f$ in $\mc{A}(\baut,\bbT.{s})@(s,b)$, which is surviving in
$E^{\prime}$ and has the following properties:
  \begin{itemize}
    \item from each basic position $(b',t)$ on which it is defined, the
    strategy $f$ suggests to $\exists$ the restriction of $m_p$ to a marking
    $m_{p,t}:B \rightarrow \p (\R{t})$;
    \item for each node $t$ on the frontier of $E^{\prime}$, let $b_t \in B$
    be the unique state in $B$ such that $(b_t,t)$ is a reachable position in
    an $f$-conform match. Then $b_t$ is an accepting state in $F$, and the
    formula $K_i^{b_t}(y)$ is true in $\bbT$ for $y$ evaluated on $t$.
       \end{itemize}
\end{itemize}

Given a formula $\mathit{Root}(y)$ stating that $y$ is the root of the tree, we
define $\varphi_{\baut,\overline{\baut}}$ as $\exists y\ (\mathit{Root}(y) \wedge
K_{m+1}^{b_I}(y))$. The proof is then concluded by showing $\trees(  \baut) = \|\varphi_{\baut,\overline{\baut}}\|$, see Appendix \ref{app:nmso-aut-to-formulas} for the details.
\end{proof}
\medskip

As an immediate corollary we obtain the following characterization of
$\nmso$, which generalizes Rabin's automata-theoretic characterization of
$\wmso$ on binary trees \cite{Rab70} (note that $\nmso = \wmso$ on this class of models).

\begin{corollary}\label{PROP_WFMSO_nondetbuchi}
A tree language $\trees$ is $\nmso$-definable if and only if there are
 B\"{u}chi automata $\baut$ and $\overline{\baut}$
such that $\trees = \trees(\baut)$ and $\overline{\trees}
= \trees(\overline{\baut})$.
\end{corollary}