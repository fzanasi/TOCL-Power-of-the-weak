In this section we provide an automata-theoretic characterization of $\wmso$.
As argued in the introduction, $\wmso$-automata will be defined as the
automata in $\Aut(\olque)$ that satisfy two additional properties: weakness
and continuity.
We now briefly discuss these properties in a slightly more general setting,
starting with weakness.

\begin{definition}
\label{def:weak}
Let $\llang$ be a one-step language, and let $\bbA = \tup{A,\tmap,\pmap,a_I}$
be in $\Aut(\llang)$.
Given two states $a,b$ of $\bbA$,
we say that there is a transition from $a$ to $b$, notation: $a \leadsto b$,
if $b$ occurs in $\tmap(a,c)$ for some $c \in C$.
We let the \emph{reachability} relation $\ord$ denote the reflexive-transitive
closure of the relation $\leadsto$.

%\begin{definition}
%Given a weak automaton $\aut$,
A \emph{strongly connected $\ord$-component} ($\ord$-SCC) is a subset $M\subseteq A$ such that for every $a,b \in M$ we have $a \ord b$ and $b \ord c$. The SCC is called \emph{maximal} (MSCC) when $M\cup\{a\}$ ceases to be a SCC for any choice of $a \in A\setminus M$.
%\end{definition}

We say that $\pmap$ is a \emph{weak} parity condition, and $\bbA$ is a
\emph{weak} parity automaton if we have
\begin{description}
\item[(weakness)] if $a \ord b$ and $b \ord a$ then $\pmap(a) = \pmap(b)$.
%\fcwarning{Can you explain to me how this is not stronger than requiring $a \ord b$ AND $b \ord a$? FZ: they can be shown to be equivalent, but with the formulation that you suggest we have also remark 4.2, which we do not have that easily asking just for the condition $a\leq b \Rightarrow \pmapega(a)\leq \pmapega(b)$.}
\end{description}
\end{definition}
\begin{comment}
\btbs
\item
story: weakness corresponds to noetherian projection
\item
Ex: \eqref{eq-wfmso} in introduction states that over the class of all
trees, the weak $\ofoe$-automata correspond to the $\mso$-variant
$\yvnmso$, where quantification is restricted to noetherian subsets.
\etbs
\end{comment}

\begin{remark}
Any weak parity automaton $\bbA$ is equivalent to a weak parity automaton
%\yvwarning{Is this correct?}\afnote{Assign to every state in a even scc parity 0, and 1 for every state in a odd scc.}
$\bbA'$ with $\pmap: A' \to \{0,1\}$. From now on we such a parity map for weak parity automata.
\end{remark}

As explained in the introduction, the leading intuition is that weak parity automata are those unable to register non-trivial properties concerning the vertical `dimension' of input trees. Indeed on trees they characterize $\nmso$, that is, the fragment of $\MSO$ where the quantification is restricted to \emph{well-founded} subsets of trees (corresponding to the notion of \emph{noetherian} subset if we consider arbitrary transition systems). We refer to the literature on weak automata \cite{MullerSaoudiSchupp92} and on $\nmso$ \cite{DBLP:conf/lics/FacchiniVZ13,Zanasi:Thesis:2012} for more details.

We now turn to the second condition that we will be interested in,
viz., continuity. Intuitively, this property expresses a constraint on how much of the horizontal `dimension' of an input tree the automaton is allowed to process. It will be instrumental in showing that the class of automata that we are going to shape characterizes $\wmso$. The idea is that, as $\wmso$-quantifiers range over finite sets, the weakeness condition corresponds to those sets being `vertically' finite (i.e. included in well-founded subtrees), whereas the continuity condition corresponds to them being `horizontally' finite (i.e. included in finitely branching subtrees).

\begin{comment}
We now turn to the second property that we require of $\mso$-automata,
viz., continuity.
Here the story is that requiring the transition map of our automata to meet
some continuity condition with respect to some states of the automaton, gives
us a grip on the `horizontal' dimension of quantification in $\wmso$,
i.e., the dimension corresponding to the fact that finite subsets of a
tree $\bbT$ are subsets of a \emph{finitely} branching subtree of $\bbT$.

\btbs
\item
Can we provide more insight here, or is it too technical?
\etbs
\end{comment}

First we formulate our continuity condition abstractly in the setting of $\Aut(\llang)$.
Given the semantics of the one-step language $\llang$, the (semantic) notion
of (co-)continuity applies to one-step formulas (see for instance
section~\ref{subsec:one-stepcont}). We can then formulate the following requirement on automata from $\Aut(\llang)$:

\begin{description}
\item[(continuity)] let $a,b$ be states such that both $a\ord b$ and
$b \ord a$, and let $c\in C$;
%\begin{itemize}
    %\item
    if ${\pmap(b)}=1$ then $\tmap(b,c)$ is continuous in $a$.
    %\item
    If $\pmap(b)=0$, then $\tmap(b,c)$ is co-continuous in $a$.
%\end{itemize}
\end{description}

For the automata used in this article we need to combine the constraints for the horizontal and vertical dimensions, yielding automata with both the weakness and continuity constraints.

\begin{definition}
A \emph{continuous-weak parity automaton} is an automaton $\aut \in \Aut(\llang)$ addittionally satisfying both the \textbf{(weakness)} and \textbf{(continuity)} conditions.
We let $\AutWC(\llang_1)$ denote the class of such automata.%\fcnote{We need this for Section 5}
\end{definition}

Observe that, so far, the continuity condition is given semantically.
However, given that the one-step languages that we are interested in have
a \emph{syntactic characterization} of continuity (see for example Theorem~\ref{thm:olquecont})
we will give concrete definitions of these automata that take advantage of the mentioned characterizations.

%Concretely, \WMSO-automata are then defined as follows:
\begin{definition}
A \emph{$\wmso$-automaton} $\aut = \tup{A,\tmap,\pmap,a_I}$ is an automaton $\aut \in \Aut(\olque)$ such that for all states $a,b \in A$ with $a \ord b$ and $b\ord a$ the following conditions hold:
\begin{description}
	\itemsep 0 pt
	\item[(weakness)] $\pmap(a)=\pmap(b)$,
	\item[(continuity)] if $\pmap(a)$ is odd (resp. even) then, for each $c\in C$ we have
	   $\tmap(a,c) \in \cont{{\olque}^+}{b}(A)$ (resp. $\tmap(a,c) \in \cocont{{\olque}^+}{b}(A)$).
\end{description}
As the class of such automata coincides with $\AutWC(\olque)$ we use the same notation for it.
\end{definition}


We have now arrived at the main theorem of this section, which takes care of
one direction of Theorem~\ref{t:m1}. The main theorem of this section states that

\begin{theorem}
\label{t:wmsoauto}
There is an effective construction transforming a $\wmso$-formula $\phi$
into a $\wmso$-automaton $\bbA_{\phi}$ that is equivalent
to $\phi$ on the class of trees.
That is, for any tree $\bbT$,
\begin{equation}
\bbA_{\phi} \text{ accepts } \bbT \text{ iff } \bbT \models {\phi}.
\end{equation}
\end{theorem}

As usual, the proof of this theorem proceeds by induction on the complexity of
$\phi$.
For the inductive steps of this proof, we need to verify that the class of
\wmso-automata is closed under the boolean operations and finite projection.
Clearly, the latter closure property requires most of the work; we first
provide a simulation theorem that put $\wmso$-automata in a suitable shape
for the projection construction.
The proofs in the next two sections are (nontrivial) modifications of the
analogous results proved in~\cite{DBLP:conf/lics/FacchiniVZ13}.
%
It will be convenient to use the following terminology.

\begin{definition}
The \emph{tree language $\trees(\aut)$ recognized} by an automaton $\bbA$ is
defined as the collection of trees that are accepted by $\bbA$.
A class of trees is \emph{$\wmso$-automaton recognizable} (or
\emph{recognizable}, if clear from context), if it is of the form
$\trees(\aut)$ for a $\wmso$-automaton $\bbA$.
\end{definition}