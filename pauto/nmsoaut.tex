

In this section we provide an automata-theoretic characterization of $\nmso$.
As argued in the introduction, $\nmso$-automata will be defined as the
automata in $\Aut(\ofoe)$ that satisfy weakness as an additional property

We now briefly discuss this property in a slightly more general setting. Intuitively, a parity automaton is weak if the  reachability relation on its states induced by the transition function `respects' the parity map. This intuition is maked precise in what follows.

\begin{definition}
\label{def:weak}
Let $\llang$ be a one-step language, and let $\bbA = \tup{A,\tmap,\pmap,a_I}$
be in $\Aut(\llang)$.
Given two states $a,b$ of $\bbA$,
we say that there is a transition from $a$ to $b$, notation: $a \leadsto b$,
if $b$ occurs in $\tmap(a,c)$ for some $c \in C$.
We let the \emph{reachability} relation $\ord$ denote the reflexive-transitive
closure of the relation $\leadsto$.

%\begin{definition}
%Given a weak automaton $\aut$,
A \emph{strongly connected $\ord$-component} ($\ord$-SCC) is a subset $M\subseteq A$ such that for every $a,b \in M$ we have $a \ord b$ and $b \ord c$. The SCC is called \emph{maximal} (MSCC) when $M\cup\{a\}$ ceases to be a SCC for any choice of $a \in A\setminus M$.
%\end{definition}

We say that $\pmap$ is a \emph{weak} parity condition, and $\bbA$ is a
\emph{weak} parity automaton if we have
\begin{description}
\item[(weakness)] if $a \ord b$ and $b \ord a$ then $\pmap(a) = \pmap(b)$.
%\fcwarning{Can you explain to me how this is not stronger than requiring $a \ord b$ AND $b \ord a$? FZ: they can be shown to be equivalent, but with the formulation that you suggest we have also remark 4.2, which we do not have that easily asking just for the condition $a\leq b \Rightarrow \pmapega(a)\leq \pmapega(b)$.}
\end{description}
\end{definition}
\begin{comment}
\btbs
\item
story: weakness corresponds to noetherian projection
\item
Ex: \eqref{eq-wfmso} in introduction states that over the class of all
trees, the weak $\ofoe$-automata correspond to the $\mso$-variant
$\nmso$, where quantification is restricted to noetherian subsets.
\etbs
\end{comment}



\begin{remark}
Any weak parity automaton $\bbA$ is equivalent to a weak parity automaton
%\yvwarning{Is this correct?}\afnote{Assign to every state in a even scc parity 0, and 1 for every state in a odd scc.}
$\bbA'$ with $\pmap: A' \to \{0,1\}$. From now on we such a parity map for weak parity automata.
\end{remark}


%%%%%%%% HERE BELOW LICS 13


%{\bf HERE below from LICS 13}
%
%
%%%%
%%%%
%%%%
%
%\subsection{Automata over trees.}
%\noindent In this section we work with a restricted class of $\mso$-automata, called \emph{weak} $\mso$-automata. Intuitively, an $\mso$-automaton is weak if the  reachability relation on its states induced by the transition function `respects' the parity map.
%
%First, we present a first-order logic on a signature given by a set of unary
%predicates $A$ that will be used to define the transition function of automata.
%We define $\m{For}^{+}(A)$ as the set of monadic first-order formulae with
%identity ($\approx$), where negation can only occur in front of atomic formulae
%of the kind $x \approx y$.
% Given a subset $S$ of $A$, we introduce the notation
%\begin{eqnarray*}
%% \nonumber to remove numbering (before each equation)
%  \tau_S^+(x) &:=& \bigwedge_{a\in S}a(x).
%\end{eqnarray*}
%The formula $\tau_S^+(x)$ is called a \emph{(positive) $A$-type}. We use the convention that, if $S$ is the empty set, then $\tau_S^+(x)$ is $\top$ and we call it an \emph{empty} $A$-type.
%Given a set $Y \subseteq \m{For}^{+}(A)$ of formulae, $\SLatt(Y) = \{\bigvee X\ |\ X\subseteq_{\omega} Y\}$ is the collection of all finite disjunctions of formulae in $Y$.
%We indicate with $\m{FO}^{+}(A)$ the set of \emph{sentences} from $\m{For}^{+}(A)$.
%%Let $B_1\dots B_k$ and $C_1\dots C_j$ be sequences of subsets of $A$, possibly empty if $k=0$ or $j=0$.
%A sentence $\varphi \in \m{FO}^+(A)$ is in \emph{basic form} if it is of shape
%\begin{eqnarray*}
%% \nonumber to remove numbering (before each equation)
%  \varphi &=& \exists x_1\dots \exists x_k\ \Big(\m{diff}(\bar{x}) \wedge
%  \  \bigwedge_{1\leq i\leq k} \tau^{+}_{B_i}(x_i)
%  \\ && \hspace*{5mm}\wedge \; \forall z\ (\m{diff}(\bar{x},z) \rightarrow  \bigvee_{1\leq l\leq j}
%    \tau^{+}_{C_l} (z))\Big),
%\label{EQ_positive_basic_form}
%\end{eqnarray*}
%where each $\tau_{B_i}^+(x_i)$ and $\tau_{C_l}^+ (z)$ is an $A$-type,  $\m{diff}(y_1,\dots,y_n) := \bigwedge_{1\leq m < m^{\prime} <n} (y_m \not\approx y_{m^{\prime}})$, and the conditional subformula is defined as expected.
%We denote with $\m{BF}^+(A)$ the set of all sentences from $\m{FO}^+(A)$ that are in basic form.
%A sentence $\varphi \in \m{BF}^+(A)$ is in \emph{functional basic form} if, for each non-empty $A$-type $\tau_{S}^+(x)$ occurring in it, $S$ is a singleton. If $\varphi$ is in functional basic form and no empty type occurs in it then we say that $\varphi$ is in \emph{special basic form}. We denote with $\m{FBF}^+(A)$ and $\m{SBF}^+(A)$ respectively the set of all sentences in $\m{BF}^+(A)$ which are in functional basic form and in special basic form. %Similarly, we denote with $\m{QBF}^+(A)$ the set of sentences $\varphi \in \m{BF}^+(A)$ in \emph{quasi-special basic form}, that is, the ones in which $B$ is either empty  each $A$-type $\tau_{B}^+(x)$ the set $B$ is either empty or a singleton.
%%ALE: I call functional what fabio called special. should we change to special?
%
%Turning to the semantics, given a set $X$, a function $m:A \rightarrow \p (X)$
%and a valuation $v:\m{Var} \rightarrow X$, we define the notion of a formula
%$\varphi \in \m{For}^+(A)$ being \emph{true} in $(X,m,v)$ in the obvious way.
%In this setting, we call the function $m$ a \emph{marking}.
%%, and the triple $(X,m,v)$ an \emph{$A$-structure}.
%
%\begin{definition} \label{def_WMSOparityautomata}
%An \emph{$\mso$-automaton} on alphabet $C$ is a tuple $\mathbb{A}\ =\ \langle A, a_I, \tmaplta, \pmapega\rangle$ where:
% \begin{itemize}
%    \item $A$ is a finite set of states, $a_I \in A$ is the initial state,
%   \item $\tmaplta: A \times C \rightarrow \SLatt(FO^+(A))$,
%   \item $\pmapega: A \to \omega$ is a parity function,
% \end{itemize}
%\end{definition}
%Let $\mathbb{A}$ be an $\mso$-automaton.
%Given two states $a,b \in A$, we say that $b$ is reachable from $a$ if there is
%a sequence $a_0, \dots, a_n$ of states in $A$ such that $a_0=a$, $a_n=b$ and
%for every $i<n$, $a_{i+1}$ occurs in $\tmaplta(a_i,c)$,  for some $c \in C$.
%An  $\mso$-automaton is called \emph{weak} if  for every $a, b \in A$, $a$
%is reachable from $b$ and $b$ is reachable from $a$, then $\pmapega(a) =
%\pmapega(b)$.
%It is called \emph{non-deterministic} if $\tmaplta$ has type $A \times C
%\rightarrow \SLatt(\m{FBF}^{+}(A))$.
%
%Given a tree $\mb{T}$, the \emph{acceptance game}
%$\mathcal{A}(\mathbb{A},\mathbb{T})$ of $\mb{A}$ on $\mb{T}$  is the parity game
%defined according to the rules of table \ref{exists_asymmetric_game}.
%Finite matches of $\mathcal{A}(\mathbb{A},\mathbb{T})$ are lost by the player
%who gets stuck.
%An infinite match of $\mathcal{A}(\mathbb{A},\mathbb{T})$ is won by $\exists$ if
%and only if the \emph{minimum} parity occurring infinitely often is even.
%\begin{table*}[ht]
%  \centering
%\begin{tabular}{|l|c|l|c|}
% \hline
%  % after \\: \hline or \cline{col1-col2} \cline{col3-col4} ...
%  Position & Player & Admissible moves & Parity\\
%  \hline
%  $(a,s) \in A \times S$ & $\exists$ & $\{m : A \rightarrow \wp(\R{s})\ |\ (\R{s},m) \vDash \tmaplta (a, \V(s)) \}$ & $\pmapega(a)$\\
%  $m : A \rightarrow \wp(\R{s})$ & $\forall$ & $\{(b,t)\ |\ t \in m(b)\}$ & $\m{Max}(\pmapega[A])$\\
%  \hline
% \end{tabular}
% \caption{\rm Acceptance game for $\mso$-automata}
%  \label{exists_asymmetric_game}
%\end{table*}
%The tree $\mb{T}$ is \emph{accepted} by $\mb{A}$ if and only if $\exists$ has
%a winning strategy in $\mathcal{A}(\mathbb{A},\mathbb{T})@(a_I,s_I)$.
%The tree language accepted by $\mb{A}$ is denoted by
%$\mathcal{L}(\mathbb{A})$.
%\begin{remark}\label{rem:weak}{\rm
%It is easy to see that every weak $\mso$-automaton can be seen as having a
%parity function ranging only over priorities $\{0,1\}$. Intuitively, states
%with priority $0$ are the accepting states, whereas states with priority $1$
%are the rejecting state. This is because a weak $\mso$-automaton accepts a
%tree iff in the corresponding acceptance game, \'Eloise can always force a play
%to finally stay in an even (i.e. accepting) strongly connected component of the
%automaton.
%}\end{remark}
%
%\begin{fact}[\mrg \cite{Walukiewicz96}]
%\label{PROP_MSO_to_MSOAutomata}
%For every $\varphi \in \mso$, there is an effectively constructible
%$\mso$-automaton $\mathbb{A}_{\varphi}$ such that on tree languages
%$\|\varphi\|=\mathcal{L}(\mathbb{A}_{\varphi})$.
%%\begin{eqnarray*}
%%% \nonumber to remove numbering (before each equation)
%%  \text{for any tree $\mb{T}$,  }\mb{T}\models \varphi &\m{iff}& \mb{A}_{\varphi} \text{ accepts }\mb{T}.
%%\end{eqnarray*}
%\end{fact}
%
%
%In what follows, we show that the analogon of the previous theorem also holds
%for $\WFMSO$ and weak $\mso$-automata.
%The argument proceeds by induction on $\varphi$.  We focus on the inductive
%case of $\WFMSO$ existential quantification, which is the only non-trivial
%part of the proof. For this purpose, we define a closure operation on tree
%languages corresponding to the semantics of $\WFMSO$ existential
%quantification.
%
%\begin{definition}\label{DEF_wellclosedpvariant} Let $\mb{T}$ be a tree and $p$ a propositional letter (not necessarily in $P$).
%Let $\mc{L}$ be a tree language. The \emph{noetherian projection} of
%$\mc{L}$ over $p$ is the language ${{\exists}_W p}.\mc{L}$ defined as
%the class of trees $\mathbb{T}$ such that there is a noetherian $p$-variant
%$ \mathbb{T}' \text{ of } \mathbb{T}$, with $\mathbb{T}' \in \mc{L}$.
%
%A class $\mc{C}$ of tree languages is \emph{closed under noetherian projection
%over $p$} if, for any language $\mc{L}$ in $\mc{C}$, also the language ${{\exists}_W p}.\mc{L}$ is in $\mc{C}$.
%\end{definition}
%
%
%\subsection{The Two-Sorted Construction.}
%\noindent Our goal is to provide a \emph{projection construction} that, given
%a weak $\mso$-automaton $\mb{A}$, provides a weak $\mso$-automaton
%${{\exists}_W p}.\mb{A}$ recognizing ${{\exists}_W p}.\mc{L}(\mb{A})$.
%
%The idea is to proceed by analogy with the construction showing that the tree
%languages recognized by $\mso$-automata are closed under projection.
%In the case of $\mso$-automata, the proof crucially uses the fact that every
%$\mso$-automaton can be simulated by a \emph{non-deterministic}
%$\mso$-automaton.
%
%\begin{fact}[Simulation Theorem \cite{Walukiewicz96,Safra:1988}]
%\label{PROP_SimulationTheorem}
%For every $\mso$-automaton $\mathbb{A}$, there is an effectively
%constructible non-deterministic $\mso$-automaton $\mb{A}^{n}$ which is
%equivalent to $\mb{A}$.
%\end{fact}
%
%Unfortunately, the proof of this result does not transfer to the setting of
%\emph{weak} $\mso$-automata, in the sense that starting with a weak
%automaton $\mb{A}$ one does not necessarily end up with an automaton
%$\mb{A}^{n}$ which is also weak.
%This means that we cannot use the full power of non-determinism in the
%projection construction for weak $\mso$-automata. This notwithstanding,
%in the sequel we show how a restricted version of non-determinism suffices
%for our purposes.
%
%Let $\mb{A}$ be a weak $\mso$-automaton, $\mb{T}$ a tree and $f$ a winning
%strategy for $\exists$ in $\mc{G}_A = \mc{A}(\mb{A},\mb{T})@(a_I,s_I)$.
%It is not difficult to verify that non-determinism corresponds to the property
%that any marking suggested by $f$ assigns \emph{at most one} state to the
%successors of the node under consideration.
%If this is the case, we say that $f$ is \emph{functional}.
%The nice thing about this property is that it propagates, in the sense that
%if $\exists$ plays a functional strategy $f$ in
%$\mc{A}(\mb{A},\mb{T})@(a_I,s_I)$, then for any node $s$ in $\mb{T}$ there is
%at most one state $a$ of the automaton
%such that the position $(a,s)$ can be
%reached, in any match that is
%consistent with $f$.
%This is particularly helpful when, in order to define a $p$-variant of $\mb{T}$
%that is accepted by the projection construction over $\mb{A}$, we
%need to decide whether such a node $s$ should be labeled with $p$ or not.
%
%Now, in the case of weak $\mso$-automata we are interested only in
%\emph{noetherian} $p$-variants: the main idea is that guessing a noetherian
%$p$-variant only requires $f$ to be functional in a finite initial segment
%(i.e. a partial match) $\pi_F$ of each $f$-conform match $\pi$ of $\mc{G}_A$.
%This amounts to say that $\mb{A}$ behaves as a non-deterministic automaton as
%far as the match is played along $\pi_F$.
%We call this behavior the \emph{non-deterministic mode} of $\mb{A}$.
%During the remaining part of the match, in which $f$ is no longer required to
%be functional, we say that $\mb{A}$ has entered the \emph{alternating mode}.
%This distinction induces a well-founded subtree $\mb{W}$ of $\mb{T}$,
%consisting of the nodes from which $f$ is functionally defined.
%A noetherian $p$-variant of $\mb{T}$ is built by allowing only nodes in
%$\mb{W}$ to be labeled with $p$.
%
%To formalize this argument, which goes back to \cite{MullerSaoudiSchupp92},
%we first show that every weak $\mso$-automaton $\mathbb{A}$ can be turned
%into an equivalent weak $\mso$-automaton $\mathbb{A}^{2S}$, which we
%call \emph{two-sorted} since its carrier is split into an initial
%non-deterministic and a subsequent alternating part.
%For the precise definition of the non-deterministic part of $\mb{A}^{2S}$ we
%need the following proposition.
%
%\begin{proposition}\label{PROP:A=ref_pow_constr_of_A}
%For every $\mso$-automaton $\mathbb{A} =
%\langle A, a_I, \tmaplta, \pmapega\rangle$,
%there is an effectively constructible automaton
%$\mbshA = \langle \shA,\shai,\shDe,\NBT \rangle$ which is non-deterministic, i.e. $\shDe$ has type $\shA \times C
%\rightarrow \SLatt(\m{FBF}^{+}(\shA))$, is based on the set
%$\shA = \p(A \times A)$ of binary relations over $A$,
%takes the singleton
%set $\shai = \{ (a_{I},a_{I})\}$ as its starting state,
%and has the property that for any binary relation $Q \subseteq A \times A$,
%and any tree $\mb{T}$: %we have
%\[
%\mbshA_{Q} \text{ accepts } \mb{T} \text{ iff }
%\mb{A}_{a} \text{ accepts } \mb{T}, \text{ for all } a \in \m{Ran}(Q),
%\]
%where $\mb{A}_{a} = \langle A, a, \tmaplta, \pmapega\rangle$ denotes the variant
%of the automaton $\mb{A}$ that takes $a$ as its starting state, and similarly
%for $\mbshA_{Q}$.
%
%In particular, the automaton $\mbshA$ itself is equivalent to $\mb{A}$.
%\end{proposition}
%
%We call $\mbshA$ the \emph{refined powerset
%construct} over $\mathbb{A}$.
%Note that $\shA$ is \emph{almost} a non-deterministic $\mso$-automaton, the only difference being that the acceptance condition is not given by a parity
%condition.
%
%We now turn to the definition of the automaton $\mathbb{A}^{2S}$, which we
%call \emph{two-sorted}, because it roughly consists of a copy of $\mbshA$
%`followed by' a copy of $\mb{A}$.
%As we observed, $\mbshA$ is a non-deterministic automaton, whereas
%$\mb{A}$ generally is not.
%Thus, given a tree $\mb{T}$, the idea is to make any match $\pi$ of
%$\mc{A}(\mathbb{A}^{2S},\mb{T})$ consist of two parts:
%\begin{itemize}
%  \item \textbf{(Non-deterministic mode)} During finitely many steps,
%      $\pi$ can be seen as a match of the acceptance game of $\shA$ on
%      $\mb{T}$, where any winning strategy for $\exists$ can be assumed to
%      be functional;
%  \item \textbf{(Alternating mode)} At a certain stage, $\pi$ abandons the
%      non-deterministic part of $\mb{A}^{2S}$ and turns into a match of the
%      acceptance game of $\mb{A}$ on $\mb{T}$.
%\end{itemize}
%The definition of $\mathbb{A}^{2S}$ will guarantee the correctness of this
%construction, making $\mathbb{A}^{2S}$ equivalent to the original automaton
%$\mb{A}$.
%
%\begin{definition}\label{DEF_two-sorted automaton}
%Let $\mathbb{A}\ =\ \langle A, a_I, \tmaplta, \pmapega\rangle$ be a weak
%$\mso$-automaton and
%$\shA = \langle \shA,a_I^{\sharp},\tmaplta^{\sharp},\NBT\rangle$ its refined
%powerset construct.
%The weak $\mso$-automaton $\mathbb{A}^{2S}\ =\ \langle A^{2S}, a_I^{2S},
%\tmaplta^{2S}, \pmapega^{2S}\rangle$ is defined as follows.
%\begin{eqnarray*}
%      % \nonumber to remove numbering (before each equation)
%        A^{2S} &:=& A \cup \shA \\
%        %\leq^{2S} &:=& \leq\ \cup\ (\shA \times A)\ \cup\ (\shA \times \shA)\\
%        a_I^{2S} &:=& \shai\\
%        \tmaplta^{2S}(a,c) &:=& \tmaplta(a,c)\\
%        \tmaplta^{2S}(R,c) &:=& \shDe(R,c) \vee \bigwedge_{a \in \m{Ran}(R)} \tmaplta(a,c)\\
%        \pmapega^{2S}(a) &:=& \pmapega(a)\\
%        \pmapega^{2S}(R) &:=& 1
%      \end{eqnarray*}
%Here $a$ and $R$ denote arbitrary states in $A$ and $\shA$, respectively.
%The automaton $\mathbb{A}^{2S}$ is called the \emph{two-sorted construct over
%$\mathbb{A}$}.
%\end{definition}
%
%Then we can prove a version of a simulation theorem that will suffice
%for our purposes.
%
%\newcommand{\propTwoSortConstr}{
%Let $\mb{A} = \langle A,a_I,\tmaplta,\pmapega\rangle$ be a weak $\mso$-automaton
%and $\mb{A}^{2S}$ the two-sorted construct on $\mb{A}$.
%Then $\mathcal{L}(\mathbb{A}^{2S})=\mathcal{L}(\mathbb{A})$.
%}
%
%\begin{proposition}\label{PROP_A=A2S}
%\propTwoSortConstr
%\end{proposition}
%
%\subsection{Closure under noetherian projection.}
%\noindent
%We are now ready to show the main result of this section: the class of tree
%languages recognized by weak $\mso$-automata is closed under noetherian
%projection.
%The argument is analogous to the one showing that $\mso$-automata are
%closed under projection, but we use the two-sorted construction instead of
%the refined powerset construction. The $p$-variant induced by the projection
%automaton will be guaranteed to be noetherian because all nodes labeled with
%$p$ are visited when the automaton is in non-deterministic mode.
%
%\begin{definition}\label{DEF_bounded_projection}
%Let $\mathbb{A}\ =\ \langle A,  a_I, \tmaplta, \pmapega\rangle$ be a weak
%$\mso$-automaton on alphabet $\p(P \cup \{p\})$. Let $\mathbb{A}^{2S}$
%denote its two-sorted construct.
%We define the automaton ${{\exists}_W p}.\mb{A} = \langle A^{2S}, a_I^{2S},
%\tmapltaProj, \pmapega^{2S}\rangle$ on alphabet $C$ by putting
%\begin{eqnarray*}
%% \nonumber to remove numbering (before each equation)
%  \tmapltaProj(a,c) &:=& \tmaplta^{2S}(a,c)\\
%  \tmapltaProj(R,c) &:=& \tmaplta^{2S}(R,c) \vee \tmaplta^{2S}(R,c\cup\{p\}).
%\end{eqnarray*}
%The automaton ${{\exists}_W p}.\mb{A}$ is called the \emph{two-sorted projection
%construct of $\mb{A}$ over $p$}.
%\end{definition}
%
%\newcommand{\propBoundedProjection}{
%For each weak $\mso$-automaton $\mathbb{A}$ on alphabet $\p (P \cup \{p\})$,
%%let ${{\exists}_W p}.\mb{A}$ be the two-sorted projection construct of $\mb{A}$
%%over $p$, on alphabet $C$.
%%Then
%we have that $\mathcal{L}({{\exists}_W p}.\mb{A}) \ =\
%{{\exists}_W p}.\mathcal{L}(\mb{A})$.
%}
%\begin{proposition}\label{PROP_bounded_projection}
%\propBoundedProjection
%\end{proposition}
%
%As mentioned, the above proposition takes care of the only non-trivial
%induction case in the inductive proof of the following analogon of
%Fact \ref{PROP_MSO_to_MSOAutomata}:
%
%\newcommand{\ThWFMSOToWeakAut}{
%For every $\varphi \in \WFMSO$, there is an effectively constructible
%weak $\mso$-automaton $\mathbb{A}_{\varphi}$ such that on tree languages $\|\varphi\|=
%\mathcal{L}(\mb{A}_\varphi)$.
%    %\begin{eqnarray*}
%    % \nonumber to remove numbering (before each equation)
%    %  \text{for any tree $\mb{T}$,  }\mb{T}\models \varphi &\m{iff}& \mb{A}_{\varphi} \text{ accepts }\mb{T}.
%    %\end{eqnarray*}
%}
%
%\begin{theorem}\label{W_VMSO_in_weak_automata}
%\ThWFMSOToWeakAut
%\end{theorem}
%
%
%
%\begin{remark}\label{RMK:FBFtoSB}{\rm
%Given any non-deterministic $\mso$-automaton
%$\mb{A}= \langle A,a_I,\tmaplta,\pmapega \rangle$ where $\tmaplta: A \times C \to
%\SLatt(\m{FBF}^+(A))$ we can construct an equivalent non-deterministic
%$\mso$-automaton $\mb{A}^{\prime} =
%\langle A^{\prime},a_I,\tmaplta^{\prime},\pmapega^{\prime} \rangle$ with
%$\tmaplta^{\prime}$ of type
%$A^{\prime} \times C \rightarrow \SLatt(\m{SBF}^+(A^{\prime}))$.
%That is, we may replace an arbitrary `$\m{FBF}$-automaton' $\mb{A}$ with an
%equivalent `$\m{SBF}$-automaton' $\mb{A}'$.
%This automaton $\mb{A}^{\prime}$ is based on carrier $A \cup \{a^{\top}\}$,
%where $a^{\top} \not\in A$ acts as a `bin state' always leading to the
%acceptance of the input tree.
%For each $a \in A$ and $c \in C$, we can replace the empty $A$-types
%$\tau_S^+(x) = \top$ occurring in $\tmaplta(a,c)$ with $a^{\top}(x)$.
%This leads to the definition of a transition function $\tmaplta^{\prime}$
%associated only with sentences in \emph{special} basic form.
%It is readily seen that any winning strategy for $\exists$ in the acceptance
%game for $\mb{A}^{\prime}$ and some input tree $\mb{T}$ can be assumed to mark
%each node of $\mb{T}$ with \emph{exactly one} state of $\mb{A}^{\prime}$.
%This strengthening of the functionality condition conveniently simplifies
%the constructions presented in the next section.
%}\end{remark}


