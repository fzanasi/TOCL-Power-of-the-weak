%!TEX root = ../00CFVZ_TOCL.tex


In this subsection we work with the members of the class $\AutWC(\olque)$, which we henceforth call \emph{$\wmso$-automata}. Whereas continuity has been abstractly formulated as a \emph{semantic} condition, thanks to Theorem XXX \fznote{theorem about syntactic char. for continuity} we can work with a completely syntactic definition of $\wmso$-automata, \emph{(continuity)} for $\llang = \ofoe$ being equivalently to the following condition. 
\begin{description}
	\itemsep 0 pt
	\item[(continuity, syntactically)] if $\pmap(a)$ is odd (resp. even) then, for each $c\in C$ we have
	   $\tmap(a,c) \in \cont{{\olque}^+}{b}(A)$ (resp. $\tmap(a,c) \in \cocont{{\olque}^+}{b}(A)$).
\end{description} 

In the remainder of this subsection we prove the following result, yielding the direction from formulas to automata of the characterisation theorem for $\wmso$.

\begin{theorem}
\label{t:wmsoauto}
There is an effective construction transforming a $\wmso$-formula $\phi$
into a $\wmso$-automaton $\bbA_{\phi}$ that is equivalent
to $\phi$ on the class of trees.
That is, for any tree $\bbT$, $\bbA_{\phi}$ accepts $\bbT$ if and only if $\bbT \models {\phi}$.
\end{theorem}

The proof proceeds by induction on the complexity of
$\phi$. For the inductive steps, we will need to verify that the class of
$\wmso$-automata is closed under the boolean operations and finite projection.
The latter closure property requires most of the work: we devote Section \ref{sec:simulationwmso} to a simulation theorem that put $\wmso$-automata in a suitable shape
for the projection construction.
%
To this aim, it is convenient to define a closure operation on tree languages corresponding
to the semantics of $\wmso$ quantification. The inductive step of the proof of Theorem \ref{t:wmsoauto} will show that tree languages accepted by $\wmso$-automata are closed under this operation.

\begin{definition}\label{def:tree_finproj}
Fix a set $\prop$ of proposition letters, $p \not\in P$ and a language $\trees$ of $\p (\prop\cup\{p\})$-labeled
trees.
The \emph{finitary projection} of $\trees$ over $p$ is the language
${\finexists}_F p.\trees$ of $\p (\prop)$-labeled trees
given as follows:
%
$$
{\finexists}_F p.\trees = \{\model \mid \text{ $\exists$ a finite $p$-variant } \model' \text{ of } \model \text{ with } \model' \in \trees\}.
$$
%
A class $K$ of tree languages is \emph{closed under finitary projection
over $p$} if, for any language $\trees$ in $K$, also ${{\finexists}_F p}.\trees$ is in $K$.
\end{definition} 



\subsubsection{Simulation theorem for $\wmso$-automata}\label{sec:simulationwmso}

\noindent Our next goal is a \emph{projection construction} that, given
a $\wmso$-automaton $\mb{A}$, provides one recognizing ${{\exists}_F p}.\trees(\mb{A})$. For $\mso$-automata, an analogous construction crucially uses the following \emph{simulation theorem}: every
$\mso$-automaton $\aut$ is equivalent to a \emph{non-deterministic} one $\aut'$ \cite{Walukiewicz96}.
Semantically, non-determinism yields the appealing property that every node of the input model $\model$ is associated with at most one state of $\aut'$ during the acceptance game--- that means, $\eloise$'s strategy $f$ in $\agame(\aut',\model)$ is \emph{functional} (\emph{cf.} Definition \ref{def:StratfunctionalFinitary} below). This is particularly helpful because, to define a $p$-variant of $\model$
that is accepted by the projection construct on $\aut'$, we
can infer whether a node $s$ should be labeled with $p$ by the value $f(a,s)$, where $a$ is the unique state of $\aut'$ (by functionality) that $f$ associates with $s$. Now, in the case of $\wmso$-automata we are interested in guessing
\emph{finitary} $p$-variants, which requires $f$ to be functional only on a \emph{finite} set of nodes. Thus the idea of our simulation theorem is to turn a $\wmso$-automaton $\aut$ into an equivalent one $\aut^{\f}$ that behaves non-deterministically on a \emph{finite} portion of any accepted tree.

For $\mso$-automata, the simulation theorem is based on a powerset construction: if the starting automaton has carrier $A$, the resulting non-deterministic automaton is based on ``macro-states'' from the set $\shA := \pw (A \times A)$.\footnote{The use of carrier $\pw (A \times A)$ instead of the more obvious $\pw A$ is needed to correctly associate with a run on macro-states the corresponding bundle of runs of the original automaton $\aut$ (\emph{cf.} \cite{Walukiewicz96}).} Analogously, for $\wmso$-automata we will associate the non-deterministic behavior with macro-states. However, as explained above, the automaton $\aut^{\f}$ that we construct has to be non-deterministic just on finitely many nodes of the input and may behave as $\aut$ (i.e. in ``alternating mode'') on the others. To this aim, $\aut^{\f}$ will be ``two-sorted'', roughly consisting of one copy of $\aut$ (with carrier $A$) and a variant of its powerset construction, based both on $A$ and $\shA$. For any accepted $\model$, the idea is to make any match $\pi$ of $\mc{A}(\aut^{\f},\model)$ consist of two parts:
\begin{description}
  \item[(\textbf{Non-deterministic mode})] for finitely many rounds $\pi$ is played on macro-states, i.e. positions are of the form $\shA \times T$. In her strategy player $\exists$ assigns macro-states (from $\shA$) only to \emph{finitely many} nodes, and states (from $A$) to the rest. Also, her strategy is functional in $\shA$, i.e. it assigns \emph{at most one macro-state} to each node.
  \item[(\textbf{Alternating mode})] At a certain round, $\pi$ abandons macro-states and turns into a match of the game $\mc{A}(\aut,\model)$, i.e. all next positions are from $A \times T$ (and are played according to a non-necessarily functional strategy). %of shape $(a,t) \in A \times T$.
\end{description}
Therefore successful runs of $\aut^{\noet}$ will have the property of processing only a \emph{finite} amount of the input with $\aut^{\noet}$ being in a macro-state and all the rest with $\aut^{\noet}$ behaving exactly as $\aut$. We now proceed in steps towards the construction of $\aut^{\noet}$. The following is a notion of lifting for types on states that is instrumental in defining a translation to types on macro-states. The distinction between empty and non-empty subsets of $A$ is to make sure that empty types on $A$ are lifted to empty types on $\pw A$.
\begin{definition}
Given a set $A$ and $\Sigma \subseteq \wp A$, we define the \emph{lifting} $\lift{\Sigma} \subseteq \wp \wp A$ as $\{\{S\} \mid S \in \Sigma \wedge S \neq \emptyset\} \cup
    \{\emptyset \mid \emptyset \in \Sigma \}$.
\end{definition}

Next we define a translation on the sentences associated with the
transition function of the original $\wmso$-automaton. Following the intuition given above, we want to work with sentences that can be made true by assigning macro-states (from $\shA$) to finitely many nodes in the model, and ordinary states (from $A$) to all the other nodes. 

\begin{definition}\label{DEF_finitary_lifting}
Let $\varphi \in {\olque}^+(A \times A)$ be a formula of shape $\posdbnfofoei{\vlist{T}}{\Pi}{\Sigma}$ for some $\Pi,\Sigma \subseteq \shA$ and $\vlist{T} = \{T_1,\dots,T_k\} \subseteq \shA$. Fix $\widetilde{\Sigma} \df \{\Ran(S) \mid S \in \Sigma\} \subseteq \wp A$. We define $\varphi^{\fin} \in {\olque}^+(A \cup \shA)$ as $\posdbnfofoei{\lift{\vlist{T}}}{\lift{\Pi} \cup \lift{\Sigma}}{\widetilde{\Sigma}}$\end{definition}

We refer to \eqref{eq:unfoldingNablaolque} below for the unfolding of the expression $\varphi^{\fin}$. Observe that each ${\tau}^{+}_{P}$ with $P \in \widetilde{\Sigma}$ appearing in $\varphi^{\fin}$ is a (positive) $A$-type, as $P = \Ran(S) \subseteq A$ for some $S \in \Sigma$. Our desiderata on this translation concern the notions of \emph{continuity} and \emph{functionality}.
%The idea of translation $(\cdot)^{\noet}$ is to encode at the one-step level the non-deterministic mode of $\aut^{\noet}$: the property to enforce is that $\varphi^{\noet}$ is \emph{functional} in $\shA$, that means, whenever $(D,\val \: A \to \wp(D)) \models \varphi^{\noet}$, then there is $\val'  \: A \to \wp(D)$ such that $(D, \val') \models \varphi^{\noet}$ and  $ \val'(q_1)\cap \val'(q_2) = \emptyset$ for all $q_1,q_2 \in \shA$. 

\begin{definition} Given a set $A$ of unary predicates and $B \subseteq A$, we say that a sentence $\varphi \in {\olque}^+(A)$ is \emph{functionally continuous in $B$} if, for every model $(D,\val \: A \to \wp(D))$,
\begin{align*}
\text{if } (D,\val),\ass \models \varphi \text{ then } & \exists\ \val' \: A \to \wp(D) \text{ such that } (D, \val'),\ass \models \varphi, \\
& \val'(a)\subseteq \val(a) \text{ for all } a \in A, \tag{$\val'$ is a restriction of $\val$}\\
 & \val'(b) \text{ is finite for all }b \in B \text{ and } \tag{continuity in $B$}\\
 & \val'(b)\cap \val'(a) = \emptyset \text{ for all } a \in A\setminus\{b\} \text{ and }b \in B\tag{functionality in $B$}.
\end{align*}
%Moreover, $\varphi$ is \emph{functionally continuous in $B \subseteq A$} if it is so for each $b \in B$.
\end{definition}
In words, $\varphi$ is functionally continuous in $B$ if it is continuous in each $b \in B$ and, for each model $(D,\val)$ where $\varphi$ is true, there is a restriction $\val'$ of $\val$ which both witnesses continuity and does not assign any other $a \in A$ to elements marked with some $b \in B$.
\begin{lemma}\label{LEM_cont}
Let $\varphi \in {\olque}^+(A \times A)$ and $\varphi^{\fin}\in {\olque}^+(A\cup \shA )$ be given as in Definition~\ref{DEF_finitary_lifting}. Then $\varphi^{\fin}$ is functionally continuous in $\shA$.
 \end{lemma}

\begin{proof}
%\yvwarning{this proof could use some more detail FZ: I expanded the proof and tried to make it clearer}
We first unfold the definition of $\varphi^{\fin}$ as follows:
\begin{equation}\label{eq:unfoldingNablaolque}
\begin{aligned}
\varphi^{\fin} =\ &
\underbrace{
    \exists \vlist{x}.\big(\arediff{\vlist{x}} \land \bigwedge_{0 \leq i \leq n} \tau^+_{\lift{T}_i}(x_i)
}_{\psi_1}
\land \underbrace{
    \forall z.(\arediff{\vlist{x},z} \lthen \bigvee_{S\in \lift{\Pi} \cup \lift{\Sigma} \cup \widetilde{\Sigma}} \tau^+_S(z))\big)
}_{\psi_2}
\land
\\ & \underbrace{
    \bigwedge_{P\in\widetilde{\Sigma}} \qu y.{\tau}^{+}_P(y)
}_{\psi_3} \land
 \underbrace{
    \dqu y.\bigvee_{P\in\widetilde{\Sigma}} {\tau}^{+}_P(y)
}_{\psi_4} .
\end{aligned}
\end{equation}
Observe that $\psi_1 \land \psi_2$ is just $\mondbnfofoe{\lift{\vlist{T}}}{\lift{\Pi} \cup \lift{\Sigma} \cup \widetilde{\Sigma}}{+}$. Now suppose that $(D,\val \: (A \cup \shA ) \to \wp(D))$ is a model where $\varphi^{\fin}$ is true. This amounts to the truth of subformulas $\psi_1$, $\psi_2$, $\psi_3$ and $\psi_4$ whose syntactic shape yields information on the types of elements of $D$. In particular, we can define a partition of $D$ into subsets $D_1$, $D_2$, $D'_2$ as follows:
\begin{itemize}
  \item As $\psi_1$ is true, we can pick $n$ distinct elements $s_1,\dots,s_n$ of $D$ such that $s_i$ witnesses the positive type $\lift{T}_i$, %\tau^+_{\lift{T}_i}(x_i)$,
   that is, $s_i \in \val(S)$ for each $S \in \lift{T}_i$. We define $D_1 := \{s_1,\dots,s_n\}$.
  %
  \item  As $\psi_2$ is true, we can cover all the elements not in $D_1$ with two disjoint sets $D_2$ and $D'_2$ given as follows. The set $D_2$ is defined to contain all the elements not in $D_1$ witnessing a type ${\tau}^{+}_P(z)$ with $P \in \widetilde{\Sigma}$. The set $D'_2$ is just the complement of $D_1 \cup D_2$: by syntactic shape of $\psi_2$, all elements of $D'_2$ witness a positive type ${\tau}^{+}_S$ with
  $S \in \lift{\Pi} \cup \lift{\Sigma}$.
  %
  \item The truth of the subformula $\psi_4$ yields the information that the set $D_1 \cup D'_2$ is finite. If $\widetilde{\Sigma}$ is non-empty, the truth of $\psi_3$ implies that the set $D_2$ is infinite.
 \end{itemize}
This partition uniquely associates with each $s \in D$ a type ${\tau}^{+}_S$ witnessed by $s$ and thus a set of unary predicates $S_s := S \subseteq A \cup \shA$. We can then define a valuation $\val'$ assigning to each element $s$ of $D$ exactly the set $S_s$.

We now check the properties of $\val'$. As the partition inducing $\val'$ follows the syntactic shape of $\varphi^{\fin}$, one can observe that $\val'$ is a restriction of $\val$ and $(D,\val')$ makes $\varphi^{\fin}$ true. By definition of the partition, $\val'$ assigns unary predicates from $\shA$ only to elements in the finite set $D_1 \cup D'_2$, meaning that $\varphi^{\fin}$ is continuous in $\shA$. Furthermore, $\val'$ assigns at most one unary predicate from $\shA$ to each element of $D_1 \cup D'_2$, because $\lift{\vlist{T}} \cup \lift{\Pi} \cup \lift{\Sigma}$ is defined as the lifting of $\vlist{T} \cup \Pi \cup \Sigma$. It follows that $\varphi^{\fin}$ is also functional in $\shA$. Since the same restriction $\val'$ yields both properties, $\varphi^{\fin}$ is functionally continuous in $\shA$.
\end{proof}

\begin{remark} As $\varphi^{\fin}$ is of shape $\posdbnfofoei{\lift{\vlist{T}}}{\lift{\Pi} \cup \lift{\Sigma}}{\widetilde{\Sigma}}$ with $R \not\in \bigcup\widetilde{\Sigma}$ for each $R \in \shA$, by application of Corollary \ref{cor:olquecontinuousnf} we would immediately get that $\varphi^{\fin}$ is continuous in each $R \in \shA$. However, in proving Lemma \ref{LEM_cont} we privileged a direct proof allowing to show both continuity and functionality at once.
\end{remark}

The next definition is standard (see e.g.  \cite{Walukiewicz96,Ven08}) as an intermediate step to define the transition function of the powerset construct for parity automata.

\begin{definition}\label{DEF_delta star} Let $\aut = \tup{A,\tmap,\pmap,a_I}$ be a $\wmso$-automaton. Fix $a \in A$ and $c \in C$. We define $\tmap^{\star}(a,c) \df \tmap(a,c)[b \mapsto (a,b) \mid b \in A]$, that is, the sentence in ${\olque}^+(A\times A)$ obtained by replacing each occurrence of an unary predicate $b \in A$ in $\tmap(a,c)$ with the unary predicate $(a,b) \in A \times A$. \end{definition}

 Next we combine the previous definitions to characterise the transition function associated with the macro-states.

\begin{definition}\label{PROP_DeltaPowerset}
Let $\aut = \tup{A,\tmap,\pmap,a_I}$ be a $\wmso$-automaton. Let $c \in C$ be a label and $Q \in \shA$ a binary relation on $A$. By Corollary \ref{cor:olquepositivenf}, for some $\Pi,\Sigma \subseteq \shA$ and $T_i \subseteq A \times A$, there is a sentence $\Psi_{Q,c} \in {\olque}^+(A\times A)$ in the basic form $\bigvee \posdbnfofoei{\vlist{T}}{\Pi}{\Sigma}$ such that
\begin{eqnarray*}
% \nonumber to remove numbering (before each equation)
  \bigwedge_{a \in \Ran(Q)} \tmap^{\star}(a,c) &\equiv& \Psi_{Q,c}.
\end{eqnarray*}
By definition $\Psi_{Q,c}$ is of the form $\bigvee_{i}\varphi_i$, with each $\phi_{i}$ of shape $\posdbnfofoei{\vlist{T}}{\Pi}{\Sigma}$. We put $\shDe(Q,c) := \bigvee_{i}\varphi_i^{\fin}$, where the translation $(-)^{\fin}$ is given as in Definition~\ref{DEF_finitary_lifting}. Observe that $\shDe(Q,c)$ is of type ${\olque}^+(A \cup \shA)$.
\end{definition}

We have now all the ingredients to define our two-sorted automaton.

\begin{definition}\label{def:finitaryconstruct}
Let $\aut = \tup{A,\tmap,\pmap,a_I}$ be a {\wmso-automaton}. We define the \emph{finitary construct over $\aut$} as the automaton $\aut^{\fin} = \tup{A^{\fin},\tmap^{\fin},\pmap^{\fin},a_I^{\fin}}$ given by
\begin{eqnarray*}
      % \nonumber to remove numbering (before each equation)
        A^{\fin} &:=& A \cup \shA \\
        %\leq^{2S} &:=& \leq\ \cup\ (\shA \times A)\ \cup\ (\shA \times \shA)\\
        a_I^{\fin} &:=& \{(a_I,a_I)\}\\
        \tmap^{\fin}(a,c) &:=& \tmap(a,c)\\
        \tmap^{\fin}(R,c) &:=& \shDe(R,c) \vee \bigwedge_{a \in \Ran(R)} \tmap(a,c)\\
        \pmap^{\fin}(a) &:=& \pmap(a)\\
        \pmap^{\fin}(R) &:=& 1.
      \end{eqnarray*}
\end{definition}

The idea behind this definition is that $\aut^{\fin}$ is enforced to process only a finite portion of any accepted tree while in the non-deterministic mode. This is encoded in game-theoretic terms through the notion of functional and finitary strategy. 

\begin{definition}\label{def:StratfunctionalFinitary}
Given a $\wmso$-automaton $\bbA = \tup{A,\tmap,\pmap,a_I}$ and transition system $\bbT$, a strategy $f$ for \eloise in $\mathcal{A}(\bbA,\model)$ is \emph{functional in $B \subseteq A$} (or simply functional, if $B=A$) if for each node $s$ in $\bbT$ there is at most one $b \in B$ such that $(b,s)$ is a reachable position in an $f$-guided match. Also $f$ is \emph{finitary} in $B$ if there are only finitely many nodes $s$ in $\bbT$ for which a position $(b,s)$ with $b \in B$ is reachable in an $f$-guided match.
\end{definition}



The next proposition establishes the desired properties of the finitary
construct.

\begin{proposition}[\textbf{Simulation Theorem for $\wmso$-automata}]\label{PROP_facts_finConstrwmso} Let $\aut$ be a $\wmso$-automaton and $\aut^{\fin}$ its finitary construct.
\begin{enumerate}[(i)]
  \itemsep 0 pt
  \item $\aut^{\fin}$ is a $\wmso$-automaton.\label{point:finConstrAut}
  \item For any $\model$, if $\exists$ has a winning strategy in $\agame(\aut^{\fin},\model)$ from position $(a_I^{\fin},s_I)$ then she has one that is functional in $\shA$ and finitary in $\shA$.% (\emph{cf.} Definition \ref{def:StratfunctionalFinitary}).
  \label{point:finConstrStrategy}
  \item $\aut \equiv \aut^{\fin}$. \label{point:finConstrEquiv}
  \end{enumerate}
\end{proposition}
\begin{proof}
\begin{enumerate}[(i)]
\item Observe that any SCC
of $\aut^{\fin}$ involves states of exactly one sort, either $A$ or $\shA$. For SCCs on sort $A$, \textbf{(weakness)} and \textbf{(continuity)} of $\aut^{\fin}$ follow by the ones of $\aut$. For SCCs on sort $\shA$, \textbf{(weakness)} follows by observing that all macro-states in $\aut^{\fin}$ have the same parity value. Concerning \textbf{(continuity)}, by definition of $\tmap^{\fin}$ any macro-state can only appear inside a formula of the form $\varphi^{\fin}$, which is by Lemma \ref{LEM_cont} is continuous in each $Q \in \shA$.
  \item  Let $f$ be a winning strategy for $\exists$ in $\mathcal{A}(\aut^{\fin},\model)@(a_I^{\fin},s_I)$. We define a strategy $f'$ for $\exists$ in the same game as follows:
      \begin{enumerate}[label=(\alph*),ref=\alph*]
        \item on basic positions of the form $(a,s) \in A\times T$, let $\val$ be the valuation suggested by $f$. We let the valuation suggested by $f'$ be the restriction $\val'$ of $\val$ to $A$. Observe that, as no predicate from $A^{\fin}\setminus A =\shA$ occurs in $\tmap^{\fin}(a,\V(s)) = \tmap(a,\V(s))$, then $\val'$ also makes that sentence true in $\R{s}$.
        \begin{comment} With minimality
        on basic positions of the form $(a,s) \in A\times T$, $f'$ is defined as $f$. Indeed, as no predicate from $\shA$ occurs in $\tmap^{\fin}(a,\V(s))$, we can assume that the valuation suggested by $f$ does not assign any of them to nodes in $\R{s}$.
        \end{comment}
        \label{point:stat2point1}
        \item for basic positions of the form $(R,s) \in \shA \times T$, let $\val_{R,s}$ be the valuation suggested by $f$. As $f$ is winning, $\tmap^{\fin}(R,\V(s))$ is true in the model $\val_{R,s}$. If this is because the disjunct $\bigwedge_{a \in \Ran(R)} \tmap(a,\V(s))$ is made true, then we can let $f'$ suggest the restriction to $A$ of $\val_{R,s}$, for the same reason as in \eqref{point:stat2point1}. Otherwise, the disjunct $\shDe(R,\V(s)) = \bigvee_{i}\varphi_i^{\fin}$ is made true. This means that, for some $i$,
             $$(R[s], \val_{R,s}) \models \varphi_i^{\fin}.$$
             By Lemma \ref{LEM_cont} $\varphi_i^{\fin}$ is functionally continuous in $\shA$, meaning that we have a restriction $\val_{R,s}'$ of $\val_{R,s}$ that verifies $\varphi_i^{\fin}$, assigns finitely many nodes to predicates from $\shA$ and associates with each node at most one predicate from $\shA$. We let $\val_{R,s}'$ be the suggestion of $f'$ from position $(R,s)$.
      \end{enumerate}
      The strategy $f'$ defined as above is immediately seen to be
      surviving for $\exists$. It is also winning, because the set of
      basic positions on which $f'$ is defined is a subset of the one
      of the winning strategy $f$. By this observation it also follows that any $f'$-conform match visits basic positions of the form $(R,s) \in \shA \times C$ only finitely many times, as those have odd parity. By definition, the valuation suggested by $f'$ only assigns finitely many nodes to predicates in $\shA$ from positions of that shape, and no nodes from other positions. It follows that $f'$ is finitary in $\shA$. Functionality in $\shA$ also follows immediately by definition of $f'$.
  \item For the direction from left to right, it is immediate by definition of $\aut^{\fin}$ that a winning strategy for $\exists$ in $\mc{G} = \mathcal{A}(\aut,\model)@(a_I,s_I)$ is also winning for $\exists$ in $\mc{G}^{\fin} = \mathcal{A}(\aut^{\fin},\model)@(a_I^{\fin},s_I)$.

      For the direction from right to left, let $f$ be a winning strategy for $\exists$ in $\mc{G}^{\fin}$. The idea is to define a strategy $f'$ for $\exists$ in stages, while playing a match $\pi'$ in $\mc{G}$. In parallel to $\pi'$, a shadow match $\pi$ in $\mc{G}^{\fin}$ is maintained, where $\exists$ plays according to the strategy $f$. For each round $z_i$, we want to keep the following relation between the two matches:
\smallskip
\begin{center}
\fbox{\parbox{12cm}{
Either
\begin{enumerate}[label=(\arabic*),ref=\arabic*]
  \item basic positions of the form $(Q,s) \in \shA \times T$ and $(a,s) \in A \times T$ occur respectively in $\pi$ and $\pi'$, with $a \in \Ran(Q)$,
\end{enumerate}
or
\begin{enumerate}[label=(\arabic*),ref=\arabic*]
  \item[(2)] the same basic position of the form $(a,s) \in A \times T$ occurs in both matches.
\end{enumerate}
}}\hspace*{0.3cm}($\ddag$)
\end{center}
\smallskip
The key observation is that, because $f$ is winning, a basic position of the form $(Q,s) \in \shA \times T$ can occur only for finitely many initial rounds $z_0,\dots,z_n$ that are played in $\pi$, whereas for all successive rounds $z_n,z_{n+1},\dots$ only basic positions of the form $(a,s) \in A \times T$ are encountered. Indeed, if this was not the case then either $\exists$ would get stuck or the minimum parity occurring infinitely often would be odd, since states from $\shA$ have parity $1$.

It follows that enforcing a relation between the two matches as in ($\ddag$) suffices to prove that the defined strategy $f'$ is winning for $\exists$ in $\pi'$. For this purpose, first observe that $(\ddag).1$ holds at the initial round, where the positions visited in $\pi'$ and $\pi$ are respectively $(a_I,s_I) \in A \times T$ and $(\{(a_I,a_I)\},s_I) \in A^{\fin} \times T$. Inductively, consider any round $z_i$ that is played in $\pi'$ and $\pi$, respectively with basic positions $(a,s) \in A \times T$ and $(q,s) \in A^{\fin} \times T$. In order to define the suggestion of $f'$ in $\pi'$, we distinguish two cases.
\begin{itemize}
  \item First suppose that $(q,s)$ is of the form $(Q,s) \in
  \shA\times T$. By ($\ddag$) we can assume that $a$ is in $\Ran(Q)$. Let $\val_{Q,s} :A^{\fin} \rightarrow \wp(\R{s})$ be the valuation suggested by $f$, verifying the sentence $\tmap^{\fin}(Q,\V(s))$. We distinguish two further cases, depending on which disjunct of $\tmap^{\fin}(Q,\V(s))$ is made true by $\val_{Q,s}$.
      \begin{enumerate}[label=(\roman*), ref=\roman*]
        \item If $(\R{s},\val_{Q,s})\models \bigwedge_{b \in \Ran(Q)} \tmap(b,\V(s))$, then we let $\exists$ pick the restriction to $A$ of the valuation $\val_{Q,s}$. \label{point:valuation1}
        \item If $(\R{s},\val_{Q,s})\models \shDe(Q,\V(s))$, we let $\exists$ pick a valuation $\val_{a,s}:A \rightarrow \p (\R{s})$ defined by putting, for each $b \in A$:
            \begin{align*}
            % \nonumber to remove numbering (before each equation)
               \val_{a,s}(b)\ :=\ \bigcup_{b \in \Ran(Q')} \{t \in \R{s} \mid t \in \val_{Q,s}(Q')\} 
               \cup  \{t \in \R{s} \mid t \in \val_{Q,s}(b)\} .
            \end{align*} \label{point:valuation2}
      \end{enumerate}
      It can be readily checked that the suggested move is admissible for $\exists$ in $\pi$, i.e. it makes $\tmap(a,\V(s))$ true in $\R{s}$. For case \eqref{point:valuation2}, one has to observe how $\shDe$ is defined in terms of $\tmap$. In particular, the nodes assigned to $b$ by $\val_{Q,s}$ have to be assigned to $b$ also by $\val_{a,s}$, as they may be necessary to fulfill the condition, expressed with $\qu$ and $\dqu$, that infinitely many nodes witness (or that finitely many nodes do not witness) some type.

      We now show that $(\ddag)$ holds at round $z_{i+1}$. If \eqref{point:valuation1} is the case, any next position $(b,t)\in A \times T$ picked by player $\forall$ in $\pi'$ is also available for $\forall$ in $\pi$, and we end up in case $(\ddag .2)$. Suppose instead that \eqref{point:valuation2} is the case. Given the choice $(b,t) \in A \times T$ of $\forall$, by definition of $\val_{a,s}$ there are two possibilities. First, $(b,t)$ is also an available choice for $\forall$ in $\pi$, and we end up in case $(\ddag .2)$ as before. Otherwise, there is some $Q' \in \shA$ such that $b$ is in $\Ran(Q')$ and $\forall$ can choose $(Q',t)$ in the shadow match $\pi$. By letting $\pi$ advance at round $z_{i+1}$ with such a move, we are able to maintain $(\ddag .1)$ also in $z_{i+1}$.
  \item In the remaining case, inductively we are given the same basic position $(a,s) \in A\times T$ both in $\pi$ and in $\pi'$. The valuation $\val$ suggested by $f$ in $\pi$ verifies $\tmap^{\fin}(a,\V(s)) = \tmap(a,\V(s))$, thus we can let the restriction of $\val$ to $A$ be the valuation chosen by $\exists$ in the match $\pi'$. It is immediate that any next move of $\forall$ in $\pi'$ can be mirrored by the same move in $\pi$, meaning that we are able to maintain the same position --whence the relation $(\ddag.1)$-- also in the next round.
\end{itemize}
In both cases, the suggestion of strategy $f'$ was a legitimate move for $\exists$ maintaining the relation $(\ddag)$ between the two matches for any next round $z_{i+1}$. It follows that $f'$ is a winning strategy for $\exists$ in $\mc{G}$.
\end{enumerate}
\end{proof}





\subsubsection{From formulae to automata}
In this subsection we conclude the proof of Theorem \ref{t:wmsoauto}. %, showing that $\wmso$-automata are closed under the  operations corresponding to the connectives of $\mso$, that is: union, complementation and projection with respect to finite sets.We start with the latter.

We first focus on the case of projection with respect to finite sets, which exploits our simulation result, Theorem \ref{PROP_facts_finConstrwmso}.
%%%%
%%%% PROJECTION
%%%%



\subsubsection{Closure under Finitary Projection}

\begin{definition}\label{DEF_fin_projection}
Let $\aut = \tup{A, \Delta, \Omega, a_I}$ be a $\wmso$-automaton on alphabet $\p(\prop \cup \{p\})$. Let $\aut^{\f}$
denote its finitary construct.
We define the automaton ${{\exists}_F p}.\mb{A} = \langle A^{\f}, a_I^{\f},
\DeltaProj, \Omega^{\f}\rangle$ on alphabet $\p\prop$ by putting
\begin{eqnarray*}
% \nonumber to remove numbering (before each equation)
  \DeltaProj(a,c) &:=& \Delta^{\f}(a,c) \qquad \qquad�
  \DeltaProj(R,c) &:=& \Delta^{\f}(R,c) \vee \Delta^{\f}(R,c\cup\{p\}).
\end{eqnarray*}
The automaton ${{\exists}_F p}.\mb{A}$ is called the \emph{finitary projection
construct of $\mb{A}$ over $p$}.
\end{definition}

Our projection construction corresponds to a suitable closure operation on tree languages, modeling the semantics of $\wmso$ existential quantification.

\begin{definition}\label{def:tree_finproj} Let $p$ be a propositional letter and $L$ a tree language of $\p (\prop\cup\{p\})$-labeled trees. The \emph{finite projection} of $L$ over $p$ is the language ${\exists}_F p.L$ of $C$-labeled trees defined as
\begin{equation*}
    {\exists}_F p.L = \{\model \mid \text{there is a $p$-variant } \model[p\mapsto S] \text{ of } \model \text{ such that } \model[p\mapsto S] \in L \text{ and } S \text{ is finite} \}.
\end{equation*}\hfill
\end{definition}

\begin{lemma}\label{PROP_fin_projection}
For each $\wmso$-automaton $\aut$ on alphabet $\p (\prop \cup \{p\})$,
we have that
$$\trees({{\exists}_F p}.\mb{A}) \ \equiv\
{{\exists}_F p}.\trees(\mb{A}).
$$
\end{lemma}

\begin{proof}
\fcwarning{Showing this for $\aut^F$ and using Lemma~\ref{PROP_facts_finConstr}(3) leads to a way simpler proof.}What we need to show is that for any tree $\model$:
\begin{eqnarray*}
  {{\exists}_F p}.\mb{A} \text{ accepts } \mathbb{T} & \text{ iff }& \text{there is a finite $p$-variant }\model' \\
   & & \text{of }\mathbb{T}\text{  such that }\aut\text{  accepts }\model'.
\end{eqnarray*}
For direction from left to right, we first observe that the properties stated by Lemma~\ref{PROP_facts_finConstr} hold for ${{\exists}_F p}.\mb{A}$ as well, since the latter is defined in terms of $\mb{A}^{\f}$. Then we can assume that the given winning strategy $f$ for $\exists$ in $\mc{G_{\exists}} = \mc{A}({{\exists}_F p}.\mb{A},\model)@(a_I^{\f},s_I)$ is functional and finitary in $\shA$. Functionality allows us to associate with each node $s$ either none or a unique state $Q_s \in \shA$ (\emph{cf.} \cite[Prop. 3.12]{Zanasi:Thesis:2012}). We now want to isolate  the nodes that $f$ treats ``as if they were labeled with $p$''. For this purpose, let $\val_{s}$ be the valuation suggested by $f$ from a position $(Q_s,s) \in \shA \times T$. As $f$ is winning, $\val_{s}$ makes $\DeltaProj(Q,\tscolors(s))$ true in $\R{s}$. We define a $p$-variant $\model'$ of $\model$ by \fcwarning{Why the tilde?}coloring with $p$ all nodes in the following set:
 \begin{equation}\label{eq:X_p}
% \nonumber to remove numbering (before each equation)
   X_p\ :=\ \{s \in T\mid (\R{s},\widetilde{\val}_{s}) \models \Delta^{\f}(Q_s,\tscolors(s)\cup\{p\})\}.
\end{equation}
The fact that the strategy of $\exists$ is finitary in $\shA$ guarantees that $X_p$ is finite, whence $\model'$ is a finite $p$-variant. The argument showing that $\mb{A}^{\f}$ (and thus also $\mb{A}$, by Lemma~\ref{PROP_facts_finConstr}(1))\fcwarning{isn't this (3)?} accepts $\model'$ is a routine adaptation of the analogous proof for the noetherian projection of weak $\mso$-automata, for which we refer to \cite[Prop. 3.12]{Zanasi:Thesis:2012}.
\medskip

For the direction from right to left, let $\model'$ be a finite $p$-variant of
$\model$, with labeling function $\tscolors'$, and $g$ a winning strategy for $\exists$ in $\mc{G} = \mathcal{A}(\aut,\model')@(a_I,s_I)$. Our goal is to define a strategy $g'$ for $\exists$ in $\mc{G_{\exists}}$. As usual, $g'$ will be constructed in stages, while playing a match $\pi'$ in $\mc{G_{\exists}}$. In parallel to $\pi'$, a \emph{bundle} $\mc{B}$ of $g$-guided shadow matches in $\mc{G}$ is maintained, with the following condition enforced for each round $z_i$ (\emph{cf.} \cite[Prop.~ 3.12]{Zanasi:Thesis:2012}) :
\smallskip
\begin{center}
\fbox{\parbox{13cm}{
\begin{enumerate}
  \item If the current (i.e. at round $z_i$) basic position in $\pi'$ is of the form $(Q,s) \in \shA \times T$, then for each $a \in\Ran(Q)$ there is an $g$-guided (partial) shadow match $\pi_a$ at basic position $(a,s) \in A\times T$ in the current bundle $\mc{B}_i$. Also, either $\model'_s$ is not $p$-free (i.e., it does contain a node $s'$ with $p \in \tscolors'(s')$) or $s$ has some sibling $t$ such that $\model'_t$ is not $p$-free.
  \item Otherwise, the current basic position in $\pi'$ is of the form $(a,s) \in A \times T$ and $\model'_s$ is $p$-free (i.e., it does not contain any node $s'$ with $p \in \tscolors'(s')$). Also, the bundle $\mc{B}_i$ only consists of a single $g$-guided match $\pi_a$ whose current basic position is also $(a,s)$.
\end{enumerate}
}}\hspace*{0.3cm}($\ddag$)
\end{center}
\smallskip
We briefly recall the idea behind condition ($\ddag$). Point ($\ddag.1$) describes the part of match $\pi'$ where it is still possible to encounter nodes which are labeled with $p$ in $\model'$. As $\DeltaProj$ only takes the letter $p$ into account when defined on macro-states in $\shA$, we want $\pi'$ to visit only positions of the form $(R,s) \in \shA \times T$ in that situation. Anytime we visit such a position $(R,s)$ in $\pi'$, the role of the bundle is to provide one $g$-guided shadow match at position $(a,s)$ for each $a \in \Ran(R)$.
Then $g'$ is defined in terms of what $g$ suggests from those positions.

 Point ($\ddag.2$) describes how we want the match $\pi'$ to be
 played on a $p$-free subtree: as any node that one might encounter has the same label in $\model$ and $\model'$,
it is safe to let ${{\exists}_F p}.\mb{A}$ behave as $\aut$ in such situation. Provided that the two matches visit the same basic positions, of the form $(a,s)\times A \times T$, we can let $g'$ just copy $g$.

The key observation is that, as $\model'$ is a \emph{finite} $p$-variant of $\model$, nodes labeled with $p$ are reachable only for finitely many rounds of $\pi'$. This means that, provided that ($\ddag$) hold at each round, ($\ddag.1$) will describe an initial segment of $\pi'$, whereas ($\ddag.2$) will describe the remaining part. Thus our proof that $g'$ is a winning strategy for $\exists$ in $\mc{G}_{\exists}$ is concluded by showing that ($\ddag$) holds for each stage of construction of $\pi'$ and $\mc{B}$.

\medskip

For this purpose, we initialize $\pi'$ from position $(\shai,s) \in \shA\times T$ and the bundle $\mc{B}$ as $\mc{B}_0 = \{\pi_{a_I}\}$, with $\pi_{a_I}$ the partial $g$-guided match consisting only of the position $(a_I,s)\in A\times T$. The situation described by ($\ddag .1$) holds at the initial stage of the construction.
Inductively, suppose that at round $z_i$ we are given a position $(q,s) \in A^{\f} \times T$ in $\pi^{\f}$ and a bundle $\mc{B}_i$ as in ($\ddag$). To show that ($\ddag$) can be maintained at round $z_{i+1}$, we distinguish two cases, corresponding respectively to situation ($\ddag.1$) and ($\ddag.2$) holding at round $z_i$.
\begin{enumerate}[label = (\Alph*), ref = \Alph*]
%\yvwarning{Notation `$q$' is confusing, see $\val'(q)$ below FZ: I corrected $q$ into $q'$ below}
  \item If $(q,s)$ is of the form $(Q,s) \in \shA \times T$, by inductive hypothesis we are given with $g$-guided shadow matches $\{\pi_a\}_{a \in \Ran(Q)}$ in $\mc{B}_i$. For each match $\pi_a$ in the bundle, we are provided with a valuation $\val_{a,s}: A \rightarrow \p (\R{s})$ making $\Delta(a,\tscolors'(s))$ true. Then we further distinguish the following two cases.
\begin{enumerate}[label = (\roman*), ref = \roman*]
  \item \label{point:TsNotPFree} Suppose first that $\model'_s$ is not $p$-free. We let the suggestion $\val' \: A^{\f} \to \p (\R{s})$ of $g'$ from position $(Q,s)$ be defined as follows:
       \begin{align*}
       % \nonumber to remove numbering (before each equation)
       %\widetilde{\val}_{Q,s}(Q') &:=& \bigcup_{a \in \Ran(Q),\ b \in \Ran(Q')}\{t\ \in \R{s}|\ t \in \val_{a,s}(b)\}.
       \val'(q')\ :=\ \begin{cases}
               \bigcap\limits_{\substack{(a,b) \in q',\\ a \in \Ran(Q)}}\{t\ \in \R{s} \mid t \in \val_{a,s}(b)\}               & q' \in \shA \\[2em]
               \bigcup\limits_{a \in \Ran(Q)} \{t\ \in \R{s} \mid t \in \val_{a,s}(q') \text{ and }\model'.t\text{ is $p$-free}\}              & q' \in A.
               %\\[1.5em]               \hspace{.6cm}\emptyset & \text{otherwise.}
           \end{cases}
       \end{align*}
       The definition of $\val'$ on $q' \in \shA$ is standard (\emph{cf.}~\cite[Prop. 2.21]{Zanasi:Thesis:2012}) and guarantees a correspondence between the states assigned by the markings $\{\val_{a,s}\}_{a \in \Ran(Q)}$ and the macro-states assigned by $\val'$. The definition of $\val'$ on $q' \in A$ aims at fulfilling the conditions, expressed via $\qu$ and $\dqu$, on the number of nodes in $\R{s}$ witnessing (or not) some $A$-types. Those conditions are the ones that $\shDe(Q,\tscolors'(s))$ --and thus also $\Delta^{\f}(Q,\tscolors'(s))$-- ``inherits'' by $\bigwedge_{a \in \Ran(R)} \Delta(a,\tscolors'(s))$, by definition of $\shDe$. Notice that we restrict $\val'(q')$ to the nodes $t \in \val_{a,s}(q')$ such that $\model'.t$ is $p$-free. As $\model'$ is a \emph{finite} $p$-variant, only \emph{finitely many} nodes in $\val_{a,s}(q')$ will not have this property. Therefore their exclusion, which is crucial for maintaining condition ($\ddag$) (\emph{cf.}~case \eqref{point:ddag2CardfromMacro} below), does not influence the fulfilling of the cardinality conditions expressed via $\qu$ and $\dqu$ in $\shDe(Q,\tscolors'(s))$.

       On the base of these observations, one can check that $\val'$ makes $\shDe(Q,\tscolors'(s))$--and thus also $\Delta^{\f}(Q,\tscolors'(s))$--true in $\R{s}$. In fact, to be a legitimate move for $\exists$ in $\pi'$, $\val'$ should make $\DeltaProj(Q,\tscolors(s))$ true: this is the case, for $\Delta^{\f}(Q,\tscolors'(s))$ is either equal to $\Delta^{\f}(Q,\tscolors(s))$, if $p \not\in \tscolors'(s)$, or to $\Delta^{\f}(Q,\tscolors(s)\cup\{p\})$ otherwise. In order to check that we can maintain $(\ddag)$, let $(q',t) \in A^{\f} \times T$ be any next position picked by $\forall$ in $\pi'$ at round $z_{i+1}$. As before, we distinguish two cases:
       \begin{enumerate}[label = (\alph*), ref = \alph*]
         \item If $q'$ is in $A$, then, by definition of $\val'$, $\forall$ can choose $(q',t)$ in some shadow match $\pi_a$ in the bundle $\mc{B}_i$. We dismiss the bundle --i.e. make it a singleton-- and bring only $\pi_a$ to the next round in the same position $(q',t)$. Observe that, by definition of $\val'$, $\model'.t$ is $p$-free and thus ($\ddag.2$) holds at round $z_{i+1}$. \label{point:ddag2CardfromMacro}
         \item Otherwise, $q'$ is in $\shA$. The new bundle $\mc{B}_{i+1}$ is given in terms of the bundle $\mc{B}_i$: for each $\pi_a \in \mc{B}_i$ with $a\in \Ran(Q)$, we look if for some $b \in \Ran(q')$ the position $(b,t)$ is a legitimate move for $\forall$ at round $z_{i+1}$; if so, then we bring $\pi_a$ to round $z_{i+1}$ at position $(b,t)$ and put the resulting (partial) shadow match $\pi_b$ in $\mc{B}_{i+1}$. Observe that, if $\forall$ is able to pick such position $(q',t)$ in $\pi'$, then by definition of $\val'$ the new bundle $\mc{B}_{i+1}$ is non-empty and consists of an $g$-guided (partial) shadow match $\pi_b$ for each $b \in \Ran(q')$. In this way we are able to keep condition ($\ddag.1$) at round $z_{i+1}$.
       \end{enumerate}
    \item Let us now consider the case in which $\model'_s$ is $p$-free. We let $g'$ suggest the valuation $\val'$ that assigns to each node $t \in \R{s}$ all states in $\bigcup_{a \in \Ran(Q)}\{b \in A\ |\ t \in \val_{a,s}(b)\}$. It can be checked that $\val'$ makes $\bigwedge_{a \in \Ran(Q)} \Delta(a,\tscolors'(s))$ -- and then also $\Delta^{\f}(Q,\tscolors'(s))$ -- true in $\R{s}$. As $p \not\in \tscolors(s)=\tscolors'(s)$, it follows that $\val'$ also makes $\DeltaProj(Q,\tscolors(s))$ true, whence it is a legitimate choice for $\exists$ in $\pi'$. Any next basic position picked by $\forall$ in $\pi'$ is of the form $(b,t) \in A \times T$, and thus condition ($\ddag.2$) holds at round $z_{i+1}$ as shown in (i.a). %\eqref{point:ddag2CardfromMacro}
  \end{enumerate}
  \item In the remaining case, $(q,s)$ is of the form $(a,s) \in A \times T$ and by inductive hypothesis we are given with a bundle $\mc{B}_i$ consisting of a single $f$-guided (partial) shadow match $\pi_a$ at the same position $(a,s)$. Let $\val_{a,s}$ be the suggestion of $\exists$ from position $(a,s)$ in $\pi_a$. Since by assumption $s$ is $p$-free, we have that $\tscolors'(s) = \tscolors(s)$, meaning that $\DeltaProj(a,\tscolors(s))$ is just $\Delta(a,\tscolors(s)) = \Delta(a,\tscolors'(s))$. Thus the restriction $\val'$ of $\val$ to $A$ makes $\Delta(a,\tscolors'(t))$ true and we let it be the choice for $\exists$ in $\tilde{\pi}$. It follows that any next move made by $\forall$ in $\tilde{\pi}$ can be mirrored by $\forall$ in the shadow match $\pi_a$.
      \begin{comment}Version with minimality:
      It follows that $\DeltaProj(a,\tscolors(t))$ is just $\Delta(a,\tscolors(t)) = \Delta(a,\tscolors'(t))$ and the same valuation suggested by $f$ in $\pi_a$ is a legitimate choice for $\exists$ in $\tilde{\pi}$. By letting $\exists$ choose such valuation, it follows that any next move made by $\forall$ in $\tilde{\pi}$ can be mirrored by $\forall$ in the shadow match $\pi_a$.
      \end{comment}
\end{enumerate}

%As explained above, since $\model'$ is a noetherian $p$-variant, then ($\ddag .1$) holds for finitely many stages of construction of $\tilde{\pi}$, whereas ($\ddag .2$) holds for all the remaining stages, by construction of $\tilde{f}$. It follows that this strategy is winning for $\exists$ in $\tilde{G}$.

\end{proof} 

%%%%%%
%%%%%% BOOLEANS
%%%%%%

\subsubsection{Closure under Boolean operations}

In this section we will show that the class of $\wmso$-automaton recognizable
tree languages is closed under the Boolean operations.
%
Start with closure under union, we just mention the following result, without
providing the (completely routine) proof.

\begin{theorem}
\label{t:cl-dis}
Let $\bbA_{0}$ and $\bbA_{1}$ be $\wmso$-automata. 
Then there is a $\wmso$-automaton $\bbA$ such that $\trees(\bbA)$ is the 
union of $\trees(\bbA_{0})$ and $\trees(\bbA_{1})$.
\end{theorem}

In order to prove closure under complementation, we crucially use that the 
one-step language $\olque$ is closed under Boolean duals.

\myparagraphns{Closure under complementation.}
Many properties of parity automata can already be determined at the one-step level.
An important example concerns the notion of complementation.


\begin{definition}
\label{d:bdual1}
Two one-step formulas $\varphi$ and $\psi$ are each other's \emph{Boolean dual}
if for every structure $(D,\val)$ we have:
\[
(D,\val) \models \varphi \quad\text{iff}\quad (D,\val^{c}) \not\models \psi,
\]
where $\val^{c}$ is the valuation given by $\val^{c}(a) \mathrel{:=} D
\setminus \val(a)$, for all $a$.
%
A one-step language $\llang$ is \emph{closed under Boolean duals} if for every
set $A$, each formula $\varphi \in \llang(A)$ has a Boolean dual $\dual{\varphi}
\in \llang(A)$.
\end{definition}

Following ideas from~\cite{Muller1987,DBLP:conf/calco/KissigV09}, we can use Boolean duals, together with a
\emph{role switch} between $\abelard$ and $\eloise$, in order to define a
negation or complementation operation on automata.

\begin{definition}
\label{d:caut}
Assume that, for some one-step language $\llang$, the map $\dual{(-)}$
provides, for each set $A$, a Boolean dual $\dual{\varphi} \in \llang(A)$ for each
$\varphi \in \llang(A)$.
Given $\aut = \tup{A,\tmap,\pmap,a_I}$ in $\Aut(\llang)$ we define its
\emph{complement} $\dual{\aut}$ as the automaton
$\tup{A,\dual{\tmap},\dual{\pmap},a_I}$
where $\dual{\tmap}(a,c) := \dual{(\tmap(a,c))}$, and $\dual{\pmap}(a)
:= 1 + \pmap(a)$, for all $a \in A$ and $c \in \wp(\props)$.
\end{definition}

\begin{proposition}
\label{prop:autcomplementation}
Let $\llang$ and $\dual{(-)}$ be as in the previous definition.
For each automaton $\aut \in \Aut(\llang)$ and each transition system
$\model$ we have that
\[
\dual{\aut} \text{ accepts } \model
\quad\text{iff}\quad
\aut \text{ rejects } \model.
\]
\end{proposition}

The proof of Proposition~\ref{prop:autcomplementation} is based on the fact
that the power of $\eloise$ in $\agame(\dual{\aut},\model)$ is the same
as that of $\abelard$ in $\agame(\aut,\model)$, as defined in~\cite{DBLP:conf/calco/KissigV09}.

As an immediate consequence of this proposition, one may show that if the
one-step language $\llang$ is closed under Boolean duals, then the class
$\Aut(\llang)$ is closed under taking complementation.
Further on we will use Proposition~\ref{prop:autcomplementation} to show that
the same may apply to some subclasses of $\Aut(\llang)$.


\begin{theorem}
\label{t:cl-cmp}
Let $\bbA$ be an $\wmso$-automaton.
Then the automaton $\overline{\aut}$ defined in Definition~\ref{d:caut} is a
$\wmso$-automaton recognizing the complement of $\trees(\bbA)$.
\end{theorem}

\begin{proof}
Since we already know that $\overline{\bbA}$ accepts exactly the transition
systems that are rejected by $\bbA$, we only need to check that 
$\overline{\bbA}$ is indeed a $\wmso$-automaton.
But this is straightforward: for instance, continuity can be checked by 
observing the self-dual nature of this property.
\end{proof}


%%%%
%%%% PROOF THEOREM
%%%%

\subsection{Proof of Theorem \ref{t:wmsoauto}}

\begin{proof} The proof is by induction on $\varphi$.
\begin{itemize}
  \item For the base case $\varphi = p \inc q$, the corresponding 
  $\wmso$-automaton is provided in \cite[Ex. 2.6]{Zanasi:Thesis:2012}. 
  For the base case $\varphi = R(p,q)$, we give the corresponding 
  $\wmso$-automaton $\aut_{R(p,q)} = \tup{A,\Delta,\Omega,a_I}$ below:
\begin{eqnarray*}
        A &:=& \{a_0,a_1\}\\
        a_I &:=& a_0\\
  \Delta(a_0,c) &:=& \left\{
	\begin{array}{ll}
           \exists x. a_1(x) \wedge \forall y. a_0(y) & \mbox{if }p \in c 
	\\ \forall x\ (a_0(x)) & \mbox{otherwise}
	\end{array}
\right. \\
  \Delta(a_1,c) &:=& \left\{
	\begin{array}{ll}
        \top & \mbox{if }q \in c \\
        \bot & \mbox{otherwise}
	\end{array}
\right. \\
    \Omega(a_0) &:=& 0\\
    \Omega(a_1) &:=& 1.
\end{eqnarray*}
Note that the $\mso$-automaton for $R(p,q)$ provided in 
\cite[Ex. 2.5]{Zanasi:Thesis:2012} is \emph{not} a $\wmso$-automaton, as the 
continuity property does not hold.

\item
For the Boolean cases, where $\varphi = \psi_1 \vee \psi_2$ or $\phi = \neg\psi$
we refer to the closure properties of recognizable tree languages, see 
Theorem~\ref{t:cl-dis} and Theorem~\ref{t:cl-cmp}, 
respectivel.
  
\item 
For the case $\varphi = \exists p. \psi$, consider the following chain of
equivalences, where $\aut_{\psi}$ is given by the inductive hypothesis and 
${{\exists}_F p}.\aut_{\psi}$ is constructed according to 
Definition \ref{DEF_fin_projection}:
\begin{alignat*}{2}
{{\exists}_F p}.\aut_{\psi} \text{ accepts }\mb{T} 
   & \text{ iff }
     \aut_{\psi} \text{ accepts } \mb{T}[p \mapsto X], 
     \text{ for some } X \sse_{\om} T
   & \quad\text{(Lemma~\ref{PROP_fin_projection})}
\\ & \text{ iff }
     \mb{T}[p \mapsto X] \models \psi,
     \text{ for some } X \sse_{\om} T
   & \quad\text{(induction hyp.)}
\\ & \text{ iff }
    \mb{T} \models \exists p. \psi
   & \quad\text{(semantics $\wmso$)}
\end{alignat*}
\end{itemize}
\end{proof}



\subsubsection{From automata to formulae}
\input{pauto/autotowmso.tex}