%!TEX root = ../00CFVZ_TOCL.tex


In this subsection we work with the members of the class $\AutWC(\olque)$, which we henceforth call \emph{$\wmso$-automata}. Whereas continuity has been abstractly formulated as a \emph{semantic} condition, we can work with a completely syntactic definition of $\wmso$-automata. Indeed, thanks to Theorem \ref{thm:ofoeicont} and Corollary \ref{cor:cocontinuity}, \emph{(continuity)} for $\llang = \olque$ is equivalent to the following condition. 
\begin{description}
	\itemsep 0 pt
	\item[(continuity, syntactically)] if $\pmap(a)$ is odd (resp. even) then, for each $c\in C$ we have
	   $\tmap(a,c) \in \cont{{\olque}^+}{b}(A)$ (resp. $\tmap(a,c) \in \cocont{{\olque}^+}{b}(A)$).
\end{description} 

In the next two sections we focus on the following result, yielding the direction from formulas to automata of the characterisation theorem for $\wmso$.

\begin{theorem}
\label{t:wmsoauto}
There is an effective construction transforming a $\wmso$-formula $\phi$
into a $\wmso$-automaton $\bbA_{\phi}$ that is equivalent
to $\phi$ on the class of trees.
That is, for any tree $\bbT$, $\bbA_{\phi}$ accepts $\bbT$ if and only if $\bbT \models {\phi}$.
\end{theorem}

The proof proceeds by induction on the complexity of
$\phi$. For the inductive steps, we will need to verify that the class of
$\wmso$-automata is closed under the boolean operations and finite projection.
The latter closure property requires most of the work: we devote Section \ref{sec:simulationwmso} to a simulation theorem that put $\wmso$-automata in a suitable shape
for the projection construction.
%
To this aim, it is convenient to define a closure operation on tree languages corresponding
to the semantics of $\wmso$ quantification. The inductive step of the proof of Theorem \ref{t:wmsoauto} will show that tree languages accepted by $\wmso$-automata are closed under this operation.

\begin{definition}\label{def:tree_finproj}
Fix a set $\prop$ of proposition letters, $p \not\in P$ and a language $\trees$ of $\p (\prop\cup\{p\})$-labeled
trees.
The \emph{finitary projection} of $\trees$ over $p$ is the language of $\p (\prop)$-labeled trees defined as ${\finexists} p.\trees \df \{\bbT \mid \text{ $\exists$ a finite $p$-variant } \bbT' \text{ of } \bbT \text{ with } \bbT' \in \trees\}$.
%
A class $K$ of tree languages is \emph{closed under finitary projection
over $p$} if, for any language $\trees$ in $K$, also ${{\finexists} p}.\trees$ is in $K$.
\end{definition} 



\subsection{Simulation theorem for $\wmso$-automata}\label{sec:simulationwmso}

\noindent Our next goal is a \emph{projection construction} that, given
a $\wmso$-automaton $\aut$, provides one recognizing ${\finexists p}.\trees(\aut)$. For $\mso$-automata, an analogous construction crucially uses the following \emph{simulation theorem}: every
$\mso$-automaton $\aut$ is equivalent to a \emph{non-deterministic} one $\aut'$ \cite{Walukiewicz96}.
Semantically, non-determinism yields the appealing property that every node of the input model $\bbT$ is associated with at most one state of $\aut'$ during the acceptance game--- that means, $\eloise$'s strategy $f$ in $\agame(\aut',\bbT)$ is \emph{functional} (\emph{cf.} Definition \ref{def:StratfunctionalFinitary} below). This is particularly helpful because, to define a $p$-variant of $\bbT$
that is accepted by the projection construct on $\aut'$, we
can infer whether a node $s$ should be labeled with $p$ by the value $f(a,s)$, where $a$ is the unique state of $\aut'$ (by functionality) that $f$ associates with $s$. Now, in the case of $\wmso$-automata we are interested in guessing
\emph{finitary} $p$-variants, which requires $f$ to be functional only on a \emph{finite} set of nodes. Thus the idea of our simulation theorem is to turn a $\wmso$-automaton $\aut$ into an equivalent one $\aut^{\f}$ that behaves non-deterministically on a \emph{finite} portion of any accepted tree.

For $\mso$-automata, the simulation theorem is based on a powerset construction: if the starting automaton has carrier $A$, the resulting non-deterministic automaton is based on ``macro-states'' from the set $\shA := \pw (A \times A)$.\footnote{The use of carrier $\pw (A \times A)$ instead of the more obvious $\pw A$ is needed to correctly associate with a run on macro-states the corresponding bundle of runs of the original automaton $\aut$ (\emph{cf.} \cite{Walukiewicz96}).} Analogously, for $\wmso$-automata we will associate the non-deterministic behaviour with macro-states. However, as explained above, the automaton $\aut^{\f}$ that we construct has to be non-deterministic just on finitely many nodes of the input and may behave as $\aut$ (i.e. in ``alternating mode'') on the others. To this aim, $\aut^{\f}$ will be ``two-sorted'', roughly consisting of one copy of $\aut$ (with carrier $A$) and a variant of its powerset construction, based both on $A$ and $\shA$. For any accepted $\bbT$, the idea is to make any match $\pi$ of $\mc{A}(\aut^{\f},\bbT)$ consist of two parts:
\begin{description}
  \item[(\textbf{Non-deterministic mode})] for finitely many rounds $\pi$ is played on macro-states, i.e. positions are of the form $\shA \times T$. In her strategy player $\exists$ assigns macro-states (from $\shA$) only to \emph{finitely many} nodes, and states (from $A$) to the rest. Also, her strategy is functional in $\shA$, i.e. it assigns \emph{at most one macro-state} to each node.
  \item[(\textbf{Alternating mode})] At a certain round, $\pi$ abandons macro-states and turns into a match of the game $\mc{A}(\aut,\bbT)$, i.e. all next positions are from $A \times T$ (and are played according to a non-necessarily functional strategy). %of shape $(a,t) \in A \times T$.
\end{description}
Therefore successful runs of $\aut^{\noet}$ will have the property of processing only a \emph{finite} amount of the input with $\aut^{\noet}$ being in a macro-state and all the rest with $\aut^{\noet}$ behaving exactly as $\aut$. We now proceed in steps towards the construction of $\aut^{\noet}$. The following is a notion of lifting for types on states that is instrumental in defining a translation to types on macro-states. The distinction between empty and non-empty subsets of $A$ is to make sure that empty types on $A$ are lifted to empty types on $\pw A$.
\begin{definition}
Given a set $A$ and $\Sigma \subseteq \wp A$, we define the \emph{lifting} $\lift{\Sigma} \subseteq \wp \wp A$ as $\{\{S\} \mid S \in \Sigma \wedge S \neq \emptyset\} \cup
    \{\emptyset \mid \emptyset \in \Sigma \}$.
\end{definition}

Next we define a translation on the sentences associated with the
transition function of the original $\wmso$-automaton. Following the intuition given above, we want to work with sentences that can be made true by assigning macro-states (from $\shA$) to finitely many nodes in the model, and ordinary states (from $A$) to all the other nodes. Moreover, each node should be associated with \emph{at most one} macro-state, because of functionality. Corollary \ref{def:functionalsentenceofoe}.\ref{pt:ofoeifunctionalcontinuous} expresses these desiderata as \emph{functional continuity in $\shA$} and suggests the following syntactic shape in order to fulfil them.

\begin{definition}\label{DEF_finitary_lifting}
Let $\varphi \in {\olque}^+(A \times A)$ be a formula of shape $\posdbnfofoei{\vlist{T}}{\Pi}{\Sigma}$ for some $\Pi,\Sigma \subseteq \shA$ and $\vlist{T} = \{T_1,\dots,T_k\} \subseteq \shA$. Fix $\widetilde{\Sigma} \df \{\Ran(S) \mid S \in \Sigma\} \subseteq \wp A$. We define $\varphi^{\fin} \in {\olque}^+(A \cup \shA)$ as $\posdbnfofoei{\lift{\vlist{T}}}{\lift{\Pi} \cup \lift{\Sigma}}{\widetilde{\Sigma}}$, that means,
\begin{equation}\label{eq:unfoldingNablaolque}
\begin{aligned}
\varphi^{\fin} \df\ &
    \exists \vlist{x}.\big(\arediff{\vlist{x}} \land \bigwedge_{0 \leq i \leq n} \tau^+_{\lift{T}_i}(x_i)
\land
    \forall z.(\arediff{\vlist{x},z} \lthen \bigvee_{S\in \lift{\Pi} \cup \lift{\Sigma} \cup \widetilde{\Sigma}} \tau^+_S(z))\big)
\land
\\ & 
    \bigwedge_{P\in\widetilde{\Sigma}} \qu y.{\tau}^{+}_P(y)
 \land
    \dqu y.\bigvee_{P\in\widetilde{\Sigma}} {\tau}^{+}_P(y)
    \end{aligned}
\end{equation}
\end{definition}

%We refer to \eqref{eq:unfoldingNablaolque} below for the unfolding of the expression $\varphi^{\fin}$. Observe that each ${\tau}^{+}_{P}$ with $P \in \widetilde{\Sigma}$ appearing in $\varphi^{\fin}$ is a (positive) $A$-type, as $P = \Ran(S) \subseteq A$ for some $S \in \Sigma$. Our desiderata on this translation concern the notions of \emph{continuity} and \emph{functionality}.

% NOT NEEDED ANYMORE AFTER ADDING LEMMA ON CONTINOUS AND FUNCTIONAL SENTENCES IN PREVIOUS SECTION
%\begin{definition}\label{def:functionalcontinuous_sentence} Given a set $A$ of unary predicates and $B \subseteq A$, we say that a sentence $\varphi \in {\olque}^+(A)$ is \emph{functionally continuous in $B$} if it is functional in $B$ (Definition \ref{def:functionalsentenceolque}) if, whenever $(D,\val \: A \to \wp(D)) \models \varphi$ then there is a restriction $\val'$ of $\val$ such that $(D,\val' \: A \to \wp(D)) \models \varphi$ and $s \in \val'(b)$ for $b \in B$ implies $s \not\in \val'(a)$ for all $a \in A\setminus\{b\}$., for every model $(D,\val \: A \to \wp(D))$,
%\begin{align*}
%\text{if } (D,\val),\ass \models \varphi \text{ then } & \exists\ \val' \: A \to \wp(D) \text{ such that } (D, \val'),\ass \models \varphi, \\
%& \val'(a)\subseteq \val(a) \text{ for all } a \in A, \tag{$\val'$ is a restriction of $\val$}\\
% & \val'(b) \text{ is finite for all }b \in B \text{ and } \tag{continuity in $B$}\\
% & \val'(b)\cap \val'(a) = \emptyset \text{ for all } a \in A\setminus\{b\} \text{ and }b \in B\tag{functionality in $B$}.
%\end{align*}
%\end{definition}
%In words, $\varphi$ is functionally continuous in $B$ if it is continuous in each $b \in B$ and, for each model $(D,\val)$ where $\varphi$ is true, there is a restriction $\val'$ of $\val$ which both witnesses continuity and does not assign any other $a \in A$ to elements marked with some $b \in B$.
%\begin{lemma}\label{LEM_cont}
%Let $\varphi \in {\olque}^+(A \times A)$ and $\varphi^{\fin}\in {\olque}^+(A\cup \shA )$ be given as in Definition~\ref{DEF_finitary_lifting}. Then $\varphi^{\fin}$ is functionally continuous in $\shA$.
% \end{lemma}
%\begin{proof}
%Because $\varphi^{\fin}$ is in basic form with all $\shA$-types either empty or singletons, it is functional in $\shA$ by Proposition \ref{lemma:functionalsentenceofoe}. Because it
%\end{proof}
%\begin{proof}
%We first unfold the definition of $\varphi^{\fin}$ as follows:
%\begin{equation}\label{eq:unfoldingNablaolque}
%\begin{aligned}
%\varphi^{\fin} =\ &
%\underbrace{
%    \exists \vlist{x}.\big(\arediff{\vlist{x}} \land \bigwedge_{0 \leq i \leq n} \tau^+_{\lift{T}_i}(x_i)
%}_{\psi_1}
%\land \underbrace{
%    \forall z.(\arediff{\vlist{x},z} \lthen \bigvee_{S\in \lift{\Pi} \cup \lift{\Sigma} \cup \widetilde{\Sigma}} \tau^+_S(z))\big)
%}_{\psi_2}
%\land
%\\ & \underbrace{
%    \bigwedge_{P\in\widetilde{\Sigma}} \qu y.{\tau}^{+}_P(y)
%}_{\psi_3} \land
% \underbrace{
%    \dqu y.\bigvee_{P\in\widetilde{\Sigma}} {\tau}^{+}_P(y)
%}_{\psi_4} .
%\end{aligned}
%\end{equation}
%Observe that $\psi_1 \land \psi_2$ is just $\mondbnfofoe{\lift{\vlist{T}}}{\lift{\Pi} \cup \lift{\Sigma} \cup \widetilde{\Sigma}}{+}$. Now suppose that $(D,\val \: (A \cup \shA ) \to \wp(D))$ is a model where $\varphi^{\fin}$ is true. This amounts to the truth of subformulas $\psi_1$, $\psi_2$, $\psi_3$ and $\psi_4$ whose syntactic shape yields information on the types of elements of $D$. In particular, we can define a partition of $D$ into subsets $D_1$, $D_2$, $D'_2$ as follows:
%\begin{itemize}
%  \item As $\psi_1$ is true, we can pick $n$ distinct elements $s_1,\dots,s_n$ of $D$ such that $s_i$ witnesses the positive type $\lift{T}_i$, %\tau^+_{\lift{T}_i}(x_i)$,
%   that is, $s_i \in \val(S)$ for each $S \in \lift{T}_i$. We define $D_1 := \{s_1,\dots,s_n\}$.
%  %
%  \item  As $\psi_2$ is true, we can cover all the elements not in $D_1$ with two disjoint sets $D_2$ and $D'_2$ given as follows. The set $D_2$ is defined to contain all the elements not in $D_1$ witnessing a type ${\tau}^{+}_P(z)$ with $P \in \widetilde{\Sigma}$. The set $D'_2$ is just the complement of $D_1 \cup D_2$: by syntactic shape of $\psi_2$, all elements of $D'_2$ witness a positive type ${\tau}^{+}_S$ with
%  $S \in \lift{\Pi} \cup \lift{\Sigma}$.
%  %
%  \item The truth of the subformula $\psi_4$ yields the information that the set $D_1 \cup D'_2$ is finite. If $\widetilde{\Sigma}$ is non-empty, the truth of $\psi_3$ implies that the set $D_2$ is infinite.
% \end{itemize}
%This partition uniquely associates with each $s \in D$ a type ${\tau}^{+}_S$ witnessed by $s$ and thus a set of unary predicates $S_s := S \subseteq A \cup \shA$. We can then define a valuation $\val'$ assigning to each element $s$ of $D$ exactly the set $S_s$.
%
%We now check the properties of $\val'$. As the partition inducing $\val'$ follows the syntactic shape of $\varphi^{\fin}$, one can observe that $\val'$ is a restriction of $\val$ and $(D,\val')$ makes $\varphi^{\fin}$ true. By definition of the partition, $\val'$ assigns unary predicates from $\shA$ only to elements in the finite set $D_1 \cup D'_2$, meaning that $\varphi^{\fin}$ is continuous in $\shA$. Furthermore, $\val'$ assigns at most one unary predicate from $\shA$ to each element of $D_1 \cup D'_2$, because $\lift{\vlist{T}} \cup \lift{\Pi} \cup \lift{\Sigma}$ is defined as the lifting of $\vlist{T} \cup \Pi \cup \Sigma$. It follows that $\varphi^{\fin}$ is also functional in $\shA$. Since the same restriction $\val'$ yields both properties, $\varphi^{\fin}$ is functionally continuous in $\shA$.
%\end{proof}
%
%\begin{remark} As $\varphi^{\fin}$ is of shape $\posdbnfofoei{\lift{\vlist{T}}}{\lift{\Pi} \cup \lift{\Sigma}}{\widetilde{\Sigma}}$ with $R \not\in \bigcup\widetilde{\Sigma}$ for each $R \in \shA$, by application of Corollary \ref{cor:olquecontinuousnf} we would immediately get that $\varphi^{\fin}$ is continuous in each $R \in \shA$. However, in proving Lemma \ref{LEM_cont} we privileged a direct proof allowing to show both continuity and functionality at once.
%\end{remark}

The next definition is standard (see e.g.  \cite{Walukiewicz96,Ven08}) as an intermediate step to define the transition function of the powerset automaton.

\begin{definition}\label{DEF_delta star} For a parity automaton $\aut = \tup{A,\tmap,\pmap,a_I}$, $a \in A$ and $c \in C$, we define $\tmap^{\star}(a,c) \df \tmap(a,c)[b \mapsto (a,b) \mid b \in A]$, that is, the sentence in ${\olque}^+(A\times A)$ obtained by replacing each occurrence of an unary predicate $b \in A$ in $\tmap(a,c)$ with the unary predicate $(a,b) \in A \times A$. \end{definition}

We combine the previous definitions to form the transition function for macro-states.

\begin{definition}\label{PROP_DeltaPowerset}
Let $\aut = \tup{A,\tmap,\pmap,a_I}$ be a $\wmso$-automaton. Fix $c \in C$ and $Q \in \shA$. By Corollary \ref{cor:ofoeipositivenf}, for some $\Pi,\Sigma \subseteq \shA$ and $T_i \subseteq A \times A$, there is a sentence $\Psi_{Q,c} \in {\olque}^+(A\times A)$ in the basic form $\bigvee \posdbnfofoei{\vlist{T}}{\Pi}{\Sigma}$ such that $\bigwedge_{a \in \Ran(Q)} \tmap^{\star}(a,c) \equiv \Psi_{Q,c}$. By definition $\Psi_{Q,c}$ is of the form $\bigvee_{i}\varphi_i$, with each $\phi_{i}$ of shape $\posdbnfofoei{\vlist{T}}{\Pi}{\Sigma}$. We put $\shDe(Q,c) := \bigvee_{i}\varphi_i^{\fin}$, where the translation $(-)^{\fin}$ is given as in Definition~\ref{DEF_finitary_lifting}. Observe that $\shDe(Q,c)$ is of type ${\olque}^+(A \cup \shA)$.
\end{definition}

We have now all the ingredients to define our two-sorted automaton.

\begin{definition}\label{def:finitaryconstruct}
Let $\aut = \tup{A,\tmap,\pmap,a_I}$ be a {\wmso-automaton}. We define the \emph{finitary construct over $\aut$} as the automaton $\aut^{\fin} = \tup{A^{\fin},\tmap^{\fin},\pmap^{\fin},a_I^{\fin}}$ given by
\begin{gather*}
      % \nonumber to remove numbering (before each equation)
        A^{\fin} \ \df \  A \cup \shA \quad\quad\quad a_I^{\fin} \ \df \  \{(a_I,a_I)\} \quad\quad\quad \pmap^{\fin}(a) \ \df \  \pmap(a) \quad\quad\quad \pmap^{\fin}(R) \ \df \  1 \\
        \tmap^{\fin}(a,c) \ \df \  \tmap(a,c) \qquad \qquad \qquad 
        \tmap^{\fin}(Q,c) \ \ \df \ \  \shDe(Q,c) \vee \! \! \! \! \bigwedge_{a \in \Ran(Q)} \! \! \! \tmap(a,c).
      \end{gather*}
\end{definition}

The idea behind this definition is that $\aut^{\fin}$ is enforced to process only a finite portion of any accepted tree while in the non-deterministic mode. This is encoded in game-theoretic terms through the notion of functional and finitary strategy. 

\begin{definition}\label{def:StratfunctionalFinitary}
Given a $\wmso$-automaton $\bbA = \tup{A,\tmap,\pmap,a_I}$ and transition system $\bbT$, a strategy $f$ for \eloise in $\mathcal{A}(\bbA,\bbT)$ is \emph{functional in $B \subseteq A$} (or simply functional, if $B=A$) if for each node $s$ in $\bbT$ there is at most one $b \in B$ such that $(b,s)$ is a reachable position in an $f$-guided match. Also $f$ is \emph{finitary} in $B$ if there are only finitely many nodes $s$ in $\bbT$ for which a position $(b,s)$ with $b \in B$ is reachable in an $f$-guided match.
\end{definition}



The next proposition establishes the desired properties of the finitary
construct.

\begin{theorem}[\textbf{Simulation Theorem for $\wmso$-automata}]\label{PROP_facts_finConstrwmso} Let $\aut$ be a $\wmso$-automaton and $\aut^{\fin}$ its finitary construct.
\begin{enumerate}[(i)]
  \itemsep 0 pt
  \item $\aut^{\fin}$ is a $\wmso$-automaton.\label{point:finConstrAut}
  \item For any $\bbT$, if $\eloise$ has a winning strategy in $\agame(\aut^{\fin},\bbT)$ from position $(a_I^{\fin},s_I)$ then she has one that is functional in $\shA$ and finitary in $\shA$.% (\emph{cf.} Definition \ref{def:StratfunctionalFinitary}).
  \label{point:finConstrStrategy}
  \item $\aut \equiv \aut^{\fin}$. \label{point:finConstrEquiv}
  \end{enumerate}
\end{theorem}
\begin{proof}
\begin{enumerate}[(i)]
\item Observe that any SCC
of $\aut^{\fin}$ involves states of exactly one sort, either $A$ or $\shA$. For SCCs on sort $A$, weakness and continuity of $\aut^{\fin}$ follow by the ones of $\aut$. For SCCs on sort $\shA$, wekaness follows by observing that all macro-states in $\aut^{\fin}$ have the same parity value. Concerning continuity, by definition of $\tmap^{\fin}$ any macro-state can only appear inside a formula of the form $\varphi^{\fin} = \posdbnfofoei{\lift{\vlist{T}}}{\lift{\Pi} \cup \lift{\Sigma}}{\widetilde{\Sigma}}$ as in \eqref{eq:unfoldingNablaolque}. Because $\shA \cap \widetilde{\Sigma} = \emptyset$, by Corollary \ref{cor:ofoeicontinuousnf}.\ref{pt:ofoeimonotone} $\varphi^{\fin}$ is continuous in each $Q \in \shA$.
  \item  Let $f$ be a winning strategy for $\eloise$ in $\mathcal{A}(\aut^{\fin},\bbT)@(a_I^{\fin},s_I)$. We define a strategy $f'$ for $\eloise$ in the same game as follows:
      \begin{enumerate}[label=(\alph*),ref=\alph*]
        \item on basic positions of the form $(a,s) \in A\times T$, let $\val$ be the valuation suggested by $f$. We let the valuation suggested by $f'$ be the restriction $\val'$ of $\val$ to $A$. Observe that, as no predicate from $A^{\fin}\setminus A =\shA$ occurs in $\tmap^{\fin}(a,\V(s)) = \tmap(a,\V(s))$, then $\val'$ also makes that sentence true in $\R{s}$.
        \label{point:stat2point1}
        \item for basic positions of the form $(R,s) \in \shA \times T$, let $\val_{R,s}$ be the valuation suggested by $f$. As $f$ is winning, $\tmap^{\fin}(R,\V(s))$ is true in the model $\val_{R,s}$. If this is because the disjunct $\bigwedge_{a \in \Ran(R)} \tmap(a,\V(s))$ is made true, then we can let $f'$ suggest the restriction to $A$ of $\val_{R,s}$, for the same reason as in \eqref{point:stat2point1}. Otherwise, the disjunct $\shDe(R,\V(s)) = \bigvee_{i}\varphi_i^{\fin}$ is made true. This means that, for some $i$, $(R[s], \val_{R,s}) \models \varphi_i^{\fin}$.
             Now, by construction of $\varphi_i^{\fin}$ as in \eqref{eq:unfoldingNablaolque}, $\lift{T}_1,\dots,\lift{T}_k$ and each $S \in \in \lift{\Pi} \cup \lift{\Sigma}$ are either empty or singleton subsets of $\shA$ and $\widetilde{\Sigma} \cap \shA = \emptyset$. By Corollary \ref{def:functionalsentenceofoe}.\ref{pt:ofoeifunctionalcontinuous}, this implies that $\varphi_i^{\fin}$ is functionally continuous in $\shA$. Thus we have a restriction $\val_{R,s}'$ of $\val_{R,s}$ that verifies $\varphi_i^{\fin}$, assigns finitely many nodes to predicates from $\shA$ and associates with each node at most one predicate from $\shA$. We let $\val_{R,s}'$ be the suggestion of $f'$ from position $(R,s)$.
      \end{enumerate}
      The strategy $f'$ defined as above is immediately seen to be
      surviving for $\eloise$. It is also winning, because the set of
      basic positions on which $f'$ is defined is a subset of the one
      of the winning strategy $f$. By this observation it also follows that any $f'$-conform match visits basic positions of the form $(R,s) \in \shA \times C$ only finitely many times, as those have odd parity. By definition, the valuation suggested by $f'$ only assigns finitely many nodes to predicates in $\shA$ from positions of that shape, and no nodes from other positions. It follows that $f'$ is finitary in $\shA$. Functionality in $\shA$ also follows immediately by definition of $f'$.
  \item For the direction from left to right, it is immediate by definition of $\aut^{\fin}$ that a winning strategy for $\eloise$ in $\mc{G} = \mathcal{A}(\aut,\bbT)@(a_I,s_I)$ is also winning for $\eloise$ in $\mc{G}^{\fin} = \mathcal{A}(\aut^{\fin},\bbT)@(a_I^{\fin},s_I)$.

      For the direction from right to left, let $f$ be a winning strategy for $\eloise$ in $\mc{G}^{\fin}$. The idea is to define a strategy $f'$ for $\eloise$ in stages, while playing a match $\pi'$ in $\mc{G}$. In parallel to $\pi'$, a shadow match $\pi$ in $\mc{G}^{\fin}$ is maintained, where $\eloise$ plays according to the strategy $f$. For each round $z_i$, we want to keep the following relation between the two matches:
\smallskip
\begin{center}
\fbox{\parbox{12cm}{
Either
\begin{enumerate}[label=(\arabic*),ref=\arabic*]
  \item positions of the form $(Q,s) \in \shA \times T$ and $(a,s) \in A \times T$ occur respectively in $\pi$ and $\pi'$, with $a \in \Ran(Q)$,
\end{enumerate}
or
\begin{enumerate}[label=(\arabic*),ref=\arabic*]
  \item[(2)] the same position of the form $(a,s) \in A \times T$ occurs in both matches.
\end{enumerate}
}}\hspace*{0.3cm}($\ddag$)
\end{center}
\smallskip
The key observation is that, because $f$ is winning, a basic position of the form $(Q,s) \in \shA \times T$ can occur only for finitely many initial rounds $z_0,\dots,z_n$ that are played in $\pi$, whereas for all successive rounds $z_n,z_{n+1},\dots$ only basic positions of the form $(a,s) \in A \times T$ are encountered. Indeed, if this was not the case then either $\eloise$ would get stuck or the minimum parity occurring infinitely often would be odd, since states from $\shA$ have parity $1$.

It follows that enforcing a relation between the two matches as in ($\ddag$) suffices to prove that the defined strategy $f'$ is winning for $\eloise$ in $\pi'$. For this purpose, first observe that $(\ddag).1$ holds at the initial round, where the positions visited in $\pi'$ and $\pi$ are respectively $(a_I,s_I) \in A \times T$ and $(\{(a_I,a_I)\},s_I) \in A^{\fin} \times T$. Inductively, consider any round $z_i$ that is played in $\pi'$ and $\pi$, respectively with basic positions $(a,s) \in A \times T$ and $(q,s) \in A^{\fin} \times T$. To define the suggestion of $f'$ in $\pi'$, we distinguish two cases.
\begin{itemize}
  \item First suppose that $(q,s)$ is of the form $(Q,s) \in
  \shA\times T$. By ($\ddag$) we can assume that $a$ is in $\Ran(Q)$. Let $\val_{Q,s} :A^{\fin} \rightarrow \wp(\R{s})$ be the valuation suggested by $f$, verifying the sentence $\tmap^{\fin}(Q,\V(s))$. We distinguish two further cases, depending on which disjunct of $\tmap^{\fin}(Q,\V(s))$ is made true by $\val_{Q,s}$.
      \begin{enumerate}[label=(\roman*), ref=\roman*]
        \item If $(\R{s},\val_{Q,s})\models \bigwedge_{b \in \Ran(Q)} \tmap(b,\V(s))$, then we let $\eloise$ pick the restriction to $A$ of the valuation $\val_{Q,s}$. \label{point:valuation1}
        \item If $(\R{s},\val_{Q,s})\models \shDe(Q,\V(s))$, we let $\eloise$ pick a valuation $\val_{a,s}:A \rightarrow \p (\R{s})$ defined by putting, for each $b \in A$:
            \begin{align*}
            % \nonumber to remove numbering (before each equation)
               \val_{a,s}(b)\ :=\ \bigcup_{b \in \Ran(Q')} \{t \in \R{s} \mid t \in \val_{Q,s}(Q')\} 
               \cup  \{t \in \R{s} \mid t \in \val_{Q,s}(b)\} .
            \end{align*} \label{point:valuation2}
      \end{enumerate}
      It can be readily checked that the suggested move is admissible for $\eloise$ in $\pi$, i.e. it makes $\tmap(a,\V(s))$ true in $\R{s}$. For case \eqref{point:valuation2}, observe that the nodes assigned to $b$ by $\val_{Q,s}$ have to be assigned to $b$ also by $\val_{a,s}$, as they may be necessary to fulfill the condition, expressed with $\qu$ and $\dqu$ in $\shDe$, that infinitely many nodes witness (or that finitely many nodes do not witness) some type.

      We now show that $(\ddag)$ holds at round $z_{i+1}$. If \eqref{point:valuation1} is the case, any next position $(b,t)\in A \times T$ picked by player $\forall$ in $\pi'$ is also available for $\forall$ in $\pi$, and we end up in case $(\ddag .2)$. Suppose instead that \eqref{point:valuation2} is the case. Given the choice $(b,t) \in A \times T$ of $\forall$, by definition of $\val_{a,s}$ there are two possibilities. First, $(b,t)$ is also an available choice for $\forall$ in $\pi$, and we end up in case $(\ddag .2)$ as before. Otherwise, there is some $Q' \in \shA$ such that $b$ is in $\Ran(Q')$ and $\forall$ can choose $(Q',t)$ in the shadow match $\pi$. By letting $\pi$ advance at round $z_{i+1}$ with such a move, we are able to maintain $(\ddag .1)$ also in $z_{i+1}$.
  \item In the remaining case, inductively we are given the same basic position $(a,s) \in A\times T$ both in $\pi$ and in $\pi'$. The valuation $\val$ suggested by $f$ in $\pi$ verifies $\tmap^{\fin}(a,\V(s)) = \tmap(a,\V(s))$, thus we can let the restriction of $\val$ to $A$ be the valuation chosen by $\eloise$ in the match $\pi'$. It is immediate that any next move of $\forall$ in $\pi'$ can be mirrored by the same move in $\pi$, meaning that we are able to maintain the same position --whence the relation $(\ddag.1)$-- also in the next round.
\end{itemize}
In both cases, the suggestion of strategy $f'$ was a legitimate move for $\eloise$ maintaining the relation $(\ddag)$ between the two matches for any next round $z_{i+1}$. It follows that $f'$ is a winning strategy for $\eloise$ in $\mc{G}$.
\end{enumerate}
\end{proof}





\subsection{From formulas to automata}

In this subsection we conclude the proof of Theorem~\ref{t:wmsoauto}. %, showing that $\wmso$-automata are closed under the  operations corresponding to the connectives of $\mso$, that is: union, complementation and projection with respect to finite sets.We start with the latter.
We first focus on the case of projection with respect to finite sets, which exploits our simulation result, Theorem~\ref{PROP_facts_finConstrwmso}. The definition of the projection construction is formulated more generally for parity automata, as it will be later applied to classes other than $\AutWC(\olque)$. It clearly preserves the weakness and continuity conditions.
%%%%
%%%% PROJECTION
%%%%


%\subsection{Closure under Finitary Projection}

\begin{definition}\label{DEF_fin_projection}
Let $\aut = \tup{A, \tmap, \Omega, a_I}$ be a parity automaton on alphabet $\p(\prop \cup \{p\})$. We define the automaton ${{\exists} p}.\aut = \tup{A, \tmapProj, \Omega, a_I}$ on alphabet $\p\prop$ by putting
\begin{equation*}
% \nonumber to remove numbering (before each equation)
  \tmapProj(a,c) \ \df \ \tmap(a,c) \qquad \qquad
  \tmapProj(Q,c) \ \df \ \tmap(Q,c) \vee \tmap(Q,c\cup\{p\}).
\end{equation*}
The automaton ${{\exists} p}.\aut$ is called the \emph{finitary projection
construct of $\aut$ over $p$}.
\end{definition}


\begin{lemma}\label{PROP_fin_projection} For each $\wmso$-automaton $\aut$ on alphabet $\p (\prop \cup \{p\})$, $$\trees({{\exists} p}.\aut) \equiv
{{\finexists} p}.\trees(\aut).$$
\end{lemma}

\begin{proof}
First of all, by Theorem \ref{PROP_facts_finConstrwmso} we can assume that $\aut$ is two-sorted, i.e. $\aut = (\aut')^{\fin} = \tup{A^{\fin}, \tmap^{\fin},\pmap^{\fin},a_I^{\fin}}$ for some $\wmso$-automaton $\aut' = \tup{A,\tmap,\pmap, a_I}$. What we need to show is that for any tree $\bbT = \tup{T,R,\V \: \prop \to \pw T,s_I}$:
$${{\exists} p}.\aut \text{ accepts } \bbT \text{ iff } \text{there is a finite }p \text{ -variant }\bbT' \text{of } \bbT \text{  such that } \aut \text{  accepts } \bbT'.$$
For direction from left to right, we first observe that the properties stated by Theorem~\ref{PROP_facts_finConstrwmso}, which hold for $\aut$ by assumption, by construction hold for ${{\exists} p}.\aut$ as well. Thus we can assume that the given winning strategy $f$ for $\eloise$ in $\mc{G_{\exists}} = \mc{A}({\finexists p}.\aut,\bbT)@(a_I^{\fin},s_I)$ is functional and finitary in $\shA$. Functionality allows us to associate with each node $s$ either none or a unique state $Q_s \in \shA$ such that $(Q_s,s)$ is winning for $\eloise$. We now want to isolate the nodes that $f$ treats ``as if they were labeled with $p$''. For this purpose, let $\val_{s}$ be the valuation suggested by $f$ from a position $(Q_s,s) \in \shA \times T$. As $f$ is winning, $\val_{s}$ makes $\tmapProj(Q,\tscolors(s))$ true in $\R{s}$. We define a $p$-variant $\bbT' = \tup{T,R,\V' \: \prop\cup\{p\} \to \pw T,s_I}$ of $\bbT$ by coloring with $p$ all nodes in the following set:
 \begin{equation}\label{eq:X_p}
% \nonumber to remove numbering (before each equation)
   X_p\ :=\ \{s \in T\mid (\R{s},\widetilde{\val}_{s}) \models \tmap^{\f}(Q_s,\tscolors(s)\cup\{p\})\}.
\end{equation}
The fact that $f$ is finitary in $\shA$ guarantees that $X_p$ is finite, whence $\bbT'$ is a finite $p$-variant. It remains to show that $\aut$ accepts $\bbT'$: we claim that $f$ itself is also winning for $\eloise$ in $\mc{G} = (\aut,\bbT')@(a_I,s_I)$. In order to see that, let us construct in stages an $f$-conform match $\pi^{2S}$ of $\mc{G}$ and an $f$-conform shadow match $\tilde{\pi}$ of $\mc{G_{\exists}}$. The inductive hypothesis we want to bring from one round to the next is that the same basic position occurs in both matches, as this suffices to prove that $f$ is winning for $\eloise$ in $\mc{G}$.

First we consider the case of a basic position $(Q,s) \in A^{\fin} \times T$ where $Q \in \shA$. By assumption $f$ provides a marking $\widetilde{m}_s$ that makes $\tmapProj(Q,\V(s))$ true in $\R{s}$. Thus $\widetilde{m}_s$ verifies either $\tmap^{\fin}(Q,\V(s))$ or $\tmap^{\fin}(Q,\V(s)\cup \{p\})$. Now, the match $\pi^{\fin}$ is played on the $p$-variant $\bbT'$, where the labeling $\V'(s)$ is decided by the membership of $s$ to $X_p$. According to \eqref{eq:X_p}, if $\widetilde{m}_s$ verifies $\tmap^{\fin}(Q,\V(s)\cup \{p\})$ then $s$ is in $X_p$, meaning that it is labeled with $p$ in $\bbT'$, i.e. $\V'(s) = \V(s)\cup \{p\}$. Therefore $\widetilde{m}_s$ also verifies $\tmap^{\fin}(Q,\V'(s))$ and it is a legitimate move for $\eloise$ in match $\pi^{\fin}$. In the remaining case, $\widetilde{m}_s$ verifies $\tmap^{\fin}(Q,\V(s))$ but falsifies $\tmap^{\fin}(Q,\V(s)\cup \{p\})$, implying by definition that $s$ is not in $X_p$. This means that $s$ is not labeled with $p$ in $\bbT'$, i.e. $\V'(s) = \V(s)$. Thus again $\widetilde{m}_s$ verifies $\tmap^{\fin}(Q,\V'(s))$ and it is a legitimate move for $\eloise$ in match $\pi^{\fin}$.

It remains to consider the case of a basic position $(a,s) \in A^{\fin} \times T$ with $a \in A$ a state. By definition $\tmapProj(a,\V(s))$ is just $\tmap^{\fin}(a,\V(s))$. As $(a,s)$ is winning, we can assume that no position $(Q,s)$ with $Q$ a macro-state is winning according to the same $f$, as making $\tmapProj$-sentences true never forces $\eloise$ to mark a node both with a state and a macro-state. Therefore, $s$ is not in $X_p$ either, meaning that it it is not labeled with $p$ in the $p$-variant $\bbT'$ and thus $\V'(s) = \V(s)$. This implies that $f$ makes $\tmap^{\fin}(a,\V'(s)) = \tmap^{\fin}(a,\V(s))$ true in $\R{s}$ and its suggestion is a legitimate move for $\eloise$ in match $\pi^{\fin}$. In order to conclude the proof, observe that for all positions that we consider the same marking is suggested to $\eloise$ in both games: this means that any next position that is picked by player $\abelard$ in $\pi^{\fin}$ is also available for $\abelard$ in the shadow match $\tilde{\pi}$.


We now show the direction from right to left of the statement. Let $\bbT'$ be a finite $p$-variant of
$\bbT$, with labeling function $\tscolors'$, and $g$ a winning strategy for $\exists$ in $\mc{G} = \mathcal{A}(\aut,\bbT')@(a_I,s_I)$. Our goal is to define a strategy $g'$ for $\exists$ in $\mc{G_{\exists}}$. As usual, $g'$ will be constructed in stages, while playing a match $\pi'$ in $\mc{G_{\exists}}$. In parallel to $\pi'$, a \emph{bundle} $\mc{B}$ of $g$-guided shadow matches in $\mc{G}$ is maintained, with the following condition enforced for each round $z_i$:
\smallskip
\begin{center}
\fbox{\parbox{13.5cm}{
\begin{enumerate}
  \item If the current basic position in $\pi'$ is of the form $(Q,s) \in \shA \times T$, then for each $a \in\Ran(Q)$ there is an $g$-guided (partial) shadow match $\pi_a$ at basic position $(a,s) \in A\times T$ in the current bundle $\mc{B}_i$. Also, either $\bbT'_s$ is not $p$-free (i.e., it does contain a node $s'$ with $p \in \tscolors'(s')$) or $s$ has some sibling $t$ such that $\bbT'_t$ is not $p$-free.
  \item Otherwise, the current basic position in $\pi'$ is of the form $(a,s) \in A \times T$ and $\bbT'_s$ is $p$-free. Also, the bundle $\mc{B}_i$ only consists of a single $g$-guided match $\pi_a$ whose current basic position is also $(a,s)$.
\end{enumerate}
}}\hspace*{0.3cm}($\ddag$)
\end{center}
\smallskip
We recall the idea behind ($\ddag$). Point ($\ddag.1$) describes the part of match $\pi'$ where it is still possible to encounter nodes which are labeled with $p$ in $\bbT'$. As $\tmapProj$ only takes the letter $p$ into account when defined on macro-states in $\shA$, we want $\pi'$ to visit only positions of the form $(Q,s) \in \shA \times T$ in that situation. Anytime we visit such a position $(Q,s)$ in $\pi'$, the role of the bundle is to provide one $g$-guided shadow match at position $(a,s)$ for each $a \in \Ran(Q)$.
Then $g'$ is defined in terms of what $g$ suggests from those positions.

 Point ($\ddag.2$) describes how we want the match $\pi'$ to be
 played on a $p$-free subtree: as any node that one might encounter has the same label in $\bbT$ and $\bbT'$,
it is safe to let ${\finexists p}.\aut$ behave as $\aut$ in such situation. Provided that the two matches visit the same basic positions, of the form $(a,s)\times A \times T$, we can let $g'$ just copy $g$.

The key observation is that, as $\bbT'$ is a \emph{finite} $p$-variant of $\bbT$, nodes labeled with $p$ are reachable only for finitely many rounds of $\pi'$. This means that, provided that ($\ddag$) hold at each round, ($\ddag.1$) will describe an initial segment of $\pi'$, whereas ($\ddag.2$) will describe the remaining part. Thus our proof that $g'$ is a winning strategy for $\exists$ in $\mc{G}_{\exists}$ is concluded by showing that ($\ddag$) holds for each stage of construction of $\pi'$ and $\mc{B}$.

For this purpose, we initialize $\pi'$ from position $(\shai,s) \in \shA\times T$ and the bundle $\mc{B}$ as $\mc{B}_0 = \{\pi_{a_I}\}$, with $\pi_{a_I}$ the partial $g$-guided match consisting only of the position $(a_I,s)\in A\times T$. The situation described by ($\ddag .1$) holds at the initial stage of the construction.
Inductively, suppose that at round $z_i$ we are given a position $(q,s) \in A^{\f} \times T$ in $\pi^{\f}$ and a bundle $\mc{B}_i$ as in ($\ddag$). To show that ($\ddag$) can be maintained at round $z_{i+1}$, we distinguish two cases, corresponding respectively to situation ($\ddag.1$) and ($\ddag.2$) holding at round $z_i$.
\begin{enumerate}[label = (\Alph*), ref = \Alph*]
%\yvwarning{Notation `$q$' is confusing, see $\val'(q)$ below FZ: I corrected $q$ into $q'$ below}
  \item If $(q,s)$ is of the form $(Q,s) \in \shA \times T$, by inductive hypothesis we are given with $g$-guided shadow matches $\{\pi_a\}_{a \in \Ran(Q)}$ in $\mc{B}_i$. For each match $\pi_a$ in the bundle, we are provided with a valuation $\val_{a,s}: A \rightarrow \p (\R{s})$ making $\tmap(a,\tscolors'(s))$ true. Then we further distinguish the following two cases.
\begin{enumerate}[label = (\roman*), ref = \roman*]
  \item \label{point:TsNotPFree} Suppose first that $\bbT'_s$ is not $p$-free. We let the suggestion $\val' \: A^{\f} \to \p (\R{s})$ of $g'$ from position $(Q,s)$ be defined as follows:
       \begin{align*}
       % \nonumber to remove numbering (before each equation)
       %\widetilde{\val}_{Q,s}(Q') \ \df \  \bigcup_{a \in \Ran(Q),\ b \in \Ran(Q')}\{t\ \in \R{s}|\ t \in \val_{a,s}(b)\}.
       \val'(q')\ :=\ \begin{cases}
               \bigcap\limits_{\substack{(a,b) \in q',\\ a \in \Ran(Q)}}\{t\ \in \R{s} \mid t \in \val_{a,s}(b)\}               & q' \in \shA \\[2em]
               \bigcup\limits_{a \in \Ran(Q)} \{t\ \in \R{s} \mid t \in \val_{a,s}(q') \text{ and }\bbT'.t\text{ is $p$-free}\}              & q' \in A.
               %\\[1.5em]               \hspace{.6cm}\emptyset & \text{otherwise.}
           \end{cases}
       \end{align*}
       The definition of $\val'$ on $q' \in \shA$ is standard (\emph{cf.}~\cite[Prop. 2.21]{Zanasi:Thesis:2012}) and guarantees a correspondence between the states assigned by the markings $\{\val_{a,s}\}_{a \in \Ran(Q)}$ and the macro-states assigned by $\val'$. The definition of $\val'$ on $q' \in A$ aims at fulfilling the conditions, expressed via $\qu$ and $\dqu$, on the number of nodes in $\R{s}$ witnessing (or not) some $A$-types. Those conditions are the ones that $\shDe(Q,\tscolors'(s))$ --and thus also $\tmap^{\f}(Q,\tscolors'(s))$-- ``inherits'' by $\bigwedge_{a \in \Ran(R)} \tmap(a,\tscolors'(s))$, by definition of $\shDe$. Notice that we restrict $\val'(q')$ to the nodes $t \in \val_{a,s}(q')$ such that $\bbT'.t$ is $p$-free. As $\bbT'$ is a \emph{finite} $p$-variant, only \emph{finitely many} nodes in $\val_{a,s}(q')$ will not have this property. Therefore their exclusion, which is crucial for maintaining condition ($\ddag$) (\emph{cf.}~case \eqref{point:ddag2CardfromMacro} below), does not influence the fulfilling of the cardinality conditions expressed via $\qu$ and $\dqu$ in $\shDe(Q,\tscolors'(s))$.

       On the base of these observations, one can check that $\val'$ makes $\shDe(Q,\tscolors'(s))$--and thus also $\tmap^{\f}(Q,\tscolors'(s))$--true in $\R{s}$. In fact, to be a legitimate move for $\exists$ in $\pi'$, $\val'$ should make $\tmapProj(Q,\tscolors(s))$ true: this is the case, for $\tmap^{\f}(Q,\tscolors'(s))$ is either equal to $\tmap^{\f}(Q,\tscolors(s))$, if $p \not\in \tscolors'(s)$, or to $\tmap^{\f}(Q,\tscolors(s)\cup\{p\})$ otherwise. In order to check that we can maintain $(\ddag)$, let $(q',t) \in A^{\f} \times T$ be any next position picked by $\forall$ in $\pi'$ at round $z_{i+1}$. As before, we distinguish two cases:
       \begin{enumerate}[label = (\alph*), ref = \alph*]
         \item If $q'$ is in $A$, then, by definition of $\val'$, $\forall$ can choose $(q',t)$ in some shadow match $\pi_a$ in the bundle $\mc{B}_i$. We dismiss the bundle --i.e. make it a singleton-- and bring only $\pi_a$ to the next round in the same position $(q',t)$. Observe that, by definition of $\val'$, $\bbT'.t$ is $p$-free and thus ($\ddag.2$) holds at round $z_{i+1}$. \label{point:ddag2CardfromMacro}
         \item Otherwise, $q'$ is in $\shA$. The new bundle $\mc{B}_{i+1}$ is given in terms of the bundle $\mc{B}_i$: for each $\pi_a \in \mc{B}_i$ with $a\in \Ran(Q)$, we look if for some $b \in \Ran(q')$ the position $(b,t)$ is a legitimate move for $\forall$ at round $z_{i+1}$; if so, then we bring $\pi_a$ to round $z_{i+1}$ at position $(b,t)$ and put the resulting (partial) shadow match $\pi_b$ in $\mc{B}_{i+1}$. Observe that, if $\forall$ is able to pick such position $(q',t)$ in $\pi'$, then by definition of $\val'$ the new bundle $\mc{B}_{i+1}$ is non-empty and consists of an $g$-guided (partial) shadow match $\pi_b$ for each $b \in \Ran(q')$. In this way we are able to keep condition ($\ddag.1$) at round $z_{i+1}$.
       \end{enumerate}
    \item Let us now consider the case in which $\bbT'_s$ is $p$-free. We let $g'$ suggest the valuation $\val'$ that assigns to each node $t \in \R{s}$ all states in $\bigcup_{a \in \Ran(Q)}\{b \in A\ |\ t \in \val_{a,s}(b)\}$. It can be checked that $\val'$ makes $\bigwedge_{a \in \Ran(Q)} \tmap(a,\tscolors'(s))$ -- and then also $\tmap^{\f}(Q,\tscolors'(s))$ -- true in $\R{s}$. As $p \not\in \tscolors(s)=\tscolors'(s)$, it follows that $\val'$ also makes $\tmapProj(Q,\tscolors(s))$ true, whence it is a legitimate choice for $\exists$ in $\pi'$. Any next basic position picked by $\forall$ in $\pi'$ is of the form $(b,t) \in A \times T$, and thus condition ($\ddag.2$) holds at round $z_{i+1}$ as shown in (i.a). %\eqref{point:ddag2CardfromMacro}
  \end{enumerate}
  \item In the remaining case, $(q,s)$ is of the form $(a,s) \in A \times T$ and by inductive hypothesis we are given with a bundle $\mc{B}_i$ consisting of a single $f$-guided (partial) shadow match $\pi_a$ at the same position $(a,s)$. Let $\val_{a,s}$ be the suggestion of $\exists$ from position $(a,s)$ in $\pi_a$. Since by assumption $s$ is $p$-free, we have that $\tscolors'(s) = \tscolors(s)$, meaning that $\tmapProj(a,\tscolors(s))$ is just $\tmap(a,\tscolors(s)) = \tmap(a,\tscolors'(s))$. Thus the restriction $\val'$ of $\val$ to $A$ makes $\tmap(a,\tscolors'(t))$ true and we let it be the choice for $\exists$ in $\tilde{\pi}$. It follows that any next move made by $\forall$ in $\tilde{\pi}$ can be mirrored by $\forall$ in the shadow match $\pi_a$.
      \begin{comment}Version with minimality:
      It follows that $\tmapProj(a,\tscolors(t))$ is just $\tmap(a,\tscolors(t)) = \tmap(a,\tscolors'(t))$ and the same valuation suggested by $f$ in $\pi_a$ is a legitimate choice for $\exists$ in $\tilde{\pi}$. By letting $\exists$ choose such valuation, it follows that any next move made by $\forall$ in $\tilde{\pi}$ can be mirrored by $\forall$ in the shadow match $\pi_a$.
      \end{comment}
\end{enumerate}
%As explained above, since $\bbT'$ is a noetherian $p$-variant, then ($\ddag .1$) holds for finitely many stages of construction of $\tilde{\pi}$, whereas ($\ddag .2$) holds for all the remaining stages, by construction of $\tilde{f}$. It follows that this strategy is winning for $\exists$ in $\tilde{G}$.
\end{proof} 

%%%%%%
%%%%%% BOOLEANS
%%%%%%

\subsubsection{Closure under Boolean operations}

In this section we show that the class of tree languages recognised by $\wmso$-automata is closed under the Boolean operations.
%
For union, we use the following result, without
providing the straightforward proof.

\begin{lemma}
\label{t:cl-dis}
Let $\bbA_{0}$ and $\bbA_{1}$ be $\wmso$-automata. 
Then there is a $\wmso$-automaton $\bbA$ such that $\trees(\bbA)$ is the 
union of $\trees(\bbA_{0})$ and $\trees(\bbA_{1})$.
\end{lemma}

For closure under complementation we reuse the general results established in Section \ref{sec:parityaut} for parity automata.

\begin{lemma}
\label{t:cl-cmp}
Let $\bbA$ be an $\wmso$-automaton.
Then the automaton $\overline{\aut}$ defined in Definition~\ref{d:caut} is a
$\wmso$-automaton recognizing the complement of $\trees(\bbA)$.
\end{lemma}

\begin{proof} It suffices to check that Proposition \ref{prop:autcomplementation} restricts to the class $\AutWC(\olque)$ of $\wmso$-automata. First, the fact that $\olque$ is closed under Boolean duals (Def. \ref{d:bdual1}) implies that it holds for the class $\Aut(\olque)$. It then remains to check that the dual automata construction $\overline{(\cdot)}$ preserves weakness and continuity. But this is straightforward, given the self-dual nature of these properties.\end{proof}


%%%%
%%%% PROOF THEOREM
%%%%

We are now finally able to conclude the direction from formulas to automata of the characterisation theorem.

\begin{proof}[of Theorem \ref{t:wmsoauto}] The proof is by induction on $\varphi$.
\begin{itemize}
  \item For the base case $\varphi = p \inc q$, the corresponding 
  $\wmso$-automaton is provided in \cite[Ex. 2.6]{Zanasi:Thesis:2012}. 
  For the base case $\varphi = R(p,q)$, we give the corresponding 
  $\wmso$-automaton $\aut_{R(p,q)} = \tup{A,\tmap,\Omega,a_I}$ below:
\begin{eqnarray*}
        A  \  \df \  \{a_0,a_1\}  \qquad \qquad  a_I  \   \df  \  a_0   \qquad \qquad   \Omega(a_0)  \  \df \  0 \qquad \qquad
    \Omega(a_1)  \  \df \  1 \\
  \tmap(a_0,c)  \  \df \  \left\{
	\begin{array}{ll}
           \exists x. a_1(x) \wedge \forall y. a_0(y)  \  \mbox{if }p \in c 
	\\ \forall x\ (a_0(x))  \  \mbox{otherwise.}
	\end{array}
\right. \qquad \qquad
  \tmap(a_1,c)  \  \df \  \left\{
	\begin{array}{ll}
        \top  \  \mbox{if }q \in c \\
        \bot  \  \mbox{otherwise}
	\end{array}
\right.
\end{eqnarray*}
Note that the $\mso$-automaton for $R(p,q)$ provided in 
\cite[Ex. 2.5]{Zanasi:Thesis:2012} is \emph{not} a $\wmso$-automaton, as the 
continuity property does not hold.

\item
For the Boolean cases, where $\varphi = \psi_1 \vee \psi_2$ or $\phi = \neg\psi$
we refer to the closure properties of recognizable tree languages, see 
Lemma~\ref{t:cl-dis} and~\ref{t:cl-cmp}, 
respectively.
  
\item 
The case $\varphi = \exists p. \psi$ follows by the following chain of
equivalences, where $\aut_{\psi}$ is given by the inductive hypothesis and 
${\finexists p}.\aut_{\psi}$ is constructed according to 
Definition~\ref{DEF_fin_projection}:
\begin{alignat*}{2}
{\finexists p}.\aut_{\psi} \text{ accepts }\mb{T} 
   & \text{ iff }
     \aut_{\psi} \text{ accepts } \mb{T}[p \mapsto X], 
     \text{ for some } X \sse_{\om} T
   & \quad\text{(Lemma~\ref{PROP_fin_projection})}
\\ & \text{ iff }
     \mb{T}[p \mapsto X] \models \psi,
     \text{ for some } X \sse_{\om} T
   & \quad\text{(induction hyp.)}
\\ & \text{ iff }
    \mb{T} \models \exists p. \psi
   & \quad\text{(semantics $\wmso$)}
\end{alignat*}
\end{itemize}
\end{proof}





\subsection{From automata to formulas}\label{sec:aut_to_form_wmso}


In what follows, we conclude the automata characterisation of $\wmso$ with the converse of Theorem~\ref{t:wmsoauto}, i.e. the direction from automata to formulas (see Theorem \ref{thm:wmso_autofor} below). To this aim, we first introduce a fixpoint extension of first-order logic.

\subsubsection{Fixpoint extension of first-order logic}

Let our first-order signature be composed of a set $\prop$ of monadic predicates (denoted with capital latin letters) and an unique binary predicate $R$. Analogously to the modal $\mu$-calculus, the fixpoint extension of $\lque(\prop)$ is defined by adding a fixpoint construction clause.

\begin{definition}
The fixed point logic $\mlque(\prop)$ is given by:
$$
\varphi ::= q(x) \mid R(x,y) \mid x \foeq y \mid \exists x.\varphi \mid \qu x.\varphi \mid \lnot\varphi \mid \varphi \land \varphi \mid \mu p.\varphi(p,x)
$$
where $p,q\in\prop$, $x,y\in\fovar$; moreover $p$ occurs only positively in $\varphi(p,x)$ and $x$ is the only free variable in $\varphi(p,x)$.
\end{definition}

The semantics of the fixpoint formula $\mu p. \phi(p, x)$ is the expected one. Given a model $\model$ and $s \in T$,  $\model \models \mu p. \phi(p, s)$ iff $s$ is in the least fixpoint of the  operator $F_\phi:\wp(T)\to \wp(T)$ defined as $F_\phi(S) := \{t \in T \mid \model[p \mapsto S] \models \phi(p, t) \}$.

Formulas of $\mlque$ may be also classified according to their alternation depth as it happens for the modal $\mu$-calculus.
The alternation-free fragment of $\mlque$ is thence defined as the collection of $\mlque$-formulas $\phi$
without nesting of greatest and least fixpoint operators, i.e. such that, for any two subformulas $\mu p.\psi_1(p,y)$ and $\nu q. \psi_2(q,z)$, predicates $p$ and $q$ do not occur free respectively in $\psi_2(q,z)$ and $\psi_1(p,y)$.

%
%
\begin{definition}
Given $p \in \prop$, we say that $\varphi \in \mlque(\prop)$ is
\begin{itemize}
\item \emph{monotone in the predicate $p$} if for every LTS $\model$ and assignment $\ass$, if $\model, \ass \models \varphi$ and $\tsval(p) \subseteq E$, then $\model[p \mapsto E], g\models \phi$.

\item \emph{continuous in the predicate $p$} if for every LTS $\model$ and assignment $\ass$ there exists some finite $S \subseteq_\omega \tsval(p)$ such that $\model, \ass \models \varphi$ if and only if $\model[p \mapsto S], \ass \models \varphi$.
\end{itemize}
\end{definition}

In the next definition, we provide a definition of the continuous fragment of $\mlque$, reminiscent of the one defined in Theorem~\ref{thm:ofoeicont}.
\begin{definition}
Let $\mathsf{Q}\subseteq \prop$ be a set of monadic predicates. The fragment $\cont{\mlque}{\mathsf{Q}}(\prop)$ is defined by the following rules:
$$
\varphi ::= \psi \mid q(x) \mid \exists x.\varphi(x) \mid \varphi \land \varphi \mid \varphi \lor \varphi \mid \wqu x.(\varphi,\psi) \mid \mu p. \phi'(p, x)
$$
where $q \in \mathsf{Q}$, $\psi \in \mlque(\prop\setminus \mathsf{Q})$, $p \in \prop \setminus \mathsf{Q}$, $\wqu x.(\varphi,\psi) := \forall x.(\varphi(x) \lor \psi(x)) \land \dqu x.\psi(x)$ and $\phi'(p,x)$ is a formula with only $x$ free such that $\phi'(p,x) \in \cont{\mlque}{\mathsf{Q} \cup\{p\}}(\prop)$.

%The $\contAFMC$-fragment of $\mlque$  is  obtained by adding to $\glque$ the following (semantic) rule for constructing fixed point formulas.
% \begin{itemize}
% \item given a monadic predicate letter $P$, a first-order variable $x$, and a formula $\phi(P, x)$  that contains only positive occurrences of $P$ and no free variable other than $x$, if $\phi(P,x)$ is a formula in the fragment that is continuous in $P$ then $\mu P. \phi(P, x)$ is also a formula of  the fragment. Dually for $\nu P. \phi(P, x)$.\fcwarning{`positivity' is syntactic but `continuity' is semantic}
% \end{itemize}
\end{definition}

\begin{lemma}\label{lem:colqueiscont_mu}
If $\varphi \in \cont{\mlque}{\mathsf{Q}}(\prop)$ then $\varphi$ is continuous in (each predicate in) $\mathsf{Q}$.
\end{lemma}
%
\begin{proof} First, notice that If $\varphi \in \cont{\mlque}{\mathsf{Q}}(\prop)$ then $\varphi$ is monotone  in (each predicate from) $\mathsf{Q}$. %This is proved as for
The proof goes then by induction on the complexity of $\varphi$. For the all the cases except the fixpoint one, the proof is the same as the one for Lemma~\ref{lem:cofoeiiscont}. For $\phi=\mu p. \phi'(p, x)$, with $\phi'(p,x) \in \cont{\mlque}{\mathsf{Q} \cup\{p\}}(\prop)$, the argument is the same as in~\cite[Lemma 1]{Fontaine08}.
\end{proof}

As for the modal $\mu$-calculus, we define the fragment $\clque$ of $\mlque$ as the one where the use of the least fixed point operator is restricted to the continuous fragment. %, that the one obtained by adding to $\lque$ the following rule for constructing fixed point formulas.
 % \begin{itemize}
 % \item given a monadic predicate letter $P$, a first-order variable $x$, and a formula $\phi(P, x)$  that contains only positive occurrences of $P$ and no free variable other than $x$, if $\phi(P,x)$ is a formula in the fragment that belongs to $\cont{\mlque}{\{P\}}(\prop)$, then $\mu P. \phi(P, x)$ is also a formula of  the fragment.
 % \end{itemize}

\begin{definition}
The fragment $\clque(\prop)$ of $\mlque(\prop)$ is given by the following restriction of the fixpoint operator to the contiuous fragment:
{\small%
$$
\varphi ::= q(x) \mid R(x,y) \mid x \foeq y \mid \exists x.\varphi \mid \qu x.\varphi \mid \lnot\varphi \mid \varphi \land \varphi \mid \mu p.\varphi'(p,x)
$$}%
where $p,q\in\prop$, $x,y\in\fovar$; and $\varphi'(p,x) \in \cont{\mlque}{\{p\}}(\prop) \cap \clque(\prop)$ is such that $p$ occurs only positively in $\varphi'$ and $x$ is the only free variable in $\varphi'$.
\end{definition}

 %%%%%%%%
We now recall a useful property of fixpoint and continuity. Let $\phi(p,x)$ a formula with only $x$ free.
Given a LTS $\model = \tup{T,R,\tscolors,s_I}$, for every ordinal $\alpha$, we define by induction the following sets, with $\lambda$ a limit ordinal.
\begin{equation*}
\phi^0(\emptyset):= \emptyset \qquad \phi^{\alpha+1}(\emptyset):= \{ s \in T \mid \model[p \mapsto \phi^\alpha(\emptyset)] \models \phi(p, s)\} \qquad \phi^{\lambda}(\emptyset):= \bigcup_{\alpha < \lambda} \phi^{\alpha}(\emptyset)
\end{equation*}
If $\phi$ is monotone in $p$, it is possible to show that $\phi^{\beta+1}(\emptyset)= \phi^{\beta}(\emptyset)$, for some ordinal $\beta$. Moreover, the set $\phi^{\beta}(\emptyset)$ is the least fixpoint of $F_\phi$ (see e.g. \cite{ArnoldN01}).



A formula $\phi(p, x)$ is said to be \emph{constructive} in $p$ if its least fixpoint is reached in at most $\omega$ steps, i.e., if for every model $\model$, the least fixpoint of $F_\phi$ equals to $\bigcup_{\alpha < \omega} \phi^{\alpha}(\emptyset)$. From a local perspective, this means that a formula $\phi(p, x)$ constructive in $p$ if for every model $\model$,  every node $s \in T$, whenever $\mu p. \phi(p,x)$ is true at $s$, then $s$ belongs to some finite approximant $\phi^{i+1}(\emptyset)$ of the least fixpoint of $F_\phi$.
The next proposition is easily verified:% by the fact that Scott proved in \cite{Fontaine08} for the modal $\mu$-calculus but that generalizes to $\mglque$ as well, states that continuous formulas are constructive.

\begin{proposition}\label{prop:constructivity}
Let $\phi(p,x)$ be a $\mlque$-formula with only $x$ free. If $\phi(p,x)$ is continuous in $p$, then for every LTS $\model$, and every node $s \in T$, there is $i < \omega$ such that
\[\model \models \mu p. \phi(p,s) \text{ iff } s \in \phi^{i+1}(\emptyset).\]
\end{proposition}

From the fact that sets $\phi^{i+1}(\emptyset)$ are essentially defined as finite unfoldings and the previous Proposition~\ref{prop:constructivity}, we obtain the following.\fcwarning{More intuition on this?}

\begin{proposition}\label{prop:cor_constructivity}
Let $\phi(p,x)$ be a $\mlque$-formula with only $x$ free and such that $\phi(p,x)$ is continuous in $p$. Let $\model$ be a LTS, and $s \in T$. Then
$\model \models \mu p. \phi(p,s)$ iff there is a finite set $p^\model \subseteq_\omega T$ such that $s\in p^\model$ and $\model[p\mapsto p^\model] \models \phi(p,t)$  for every $t \in p^\model$.
\end{proposition}
 \begin{proof}
 For the direction from left to right, assume that $\model \models \mu p. \phi(p,s)$. By Proposition~\ref{prop:constructivity}, we know that  there is $i< \omega$ such that $\model[p \mapsto \phi^i(\emptyset)] \models \phi(p, s)$. The set $\phi^i(\emptyset)$ need not to be finite. However,
 using this information, we are going construct a finite tree whose nodes $t$ are labelled by finite sets $X^m_j$, where $m$ is a node of $\model$ and $j \leq i$, satisfying the following condition:
 \begin{enumerate}
\item  if $t$ is the root, then $t$ is labelled by $X_i^s$,
\item  if $t$ is labelled by $X_j^m=\{s_1, \dots, s_\ell\}$ and $j>0$, then $t$ has $\ell$  children and for every $s_i \in X_j^m$ there is an unique child $t'$ of $t$ labelled by $X_{j-1}^{n_i}$ where $m$ is a node,
%\item if $s$ is labelled by $X_j^m$ and $j=-1$, then $X_j^m=\emptyset$,
\item for every node $t$ of the tree, if $t$ is labelled by $X_j^m$, then it holds that $X_j^m \subseteq \phi^{j}(\emptyset)$.
\end{enumerate}
If we verify that $\model[p\mapsto p^\model] \models \phi(p,s)$ holds by taking as $p^\model$ the union of all labels of the nodes of the constructed tree, we can conclude for the proof of this direction.

As starting point of the inductive construction, we start by the empty tree.  Recall that we know that  $\model[p \mapsto \phi^i(\emptyset)] \models \phi(p, s)$. Since $\phi(p,x)$ is continuous in $p$, there is a finite set $X^s_i \subseteq \phi^i(\emptyset)$ such that $\model[p \mapsto X^s_i] \models \phi(p, s)$. We then add a root to our tree and label it by $X^s_i$.
 Assume that at a leaf $s$ of our tree is labelled by $X^m_j$, for some $j < i$. If $X^m_j$ is empty, than we stop, else we proceed as follows. We know that $X^m_j\subseteq \phi^{j}(\emptyset)$. This means that $\model[p \mapsto \phi^{j-1}(\emptyset)] \models \phi(p, r)$, for every $r \in X_j^m$. By continuity, for each such $r$, there is a finite set $X^m_{j-1} \subseteq  \phi^{j-1}(\emptyset)$ such that $\model[p \mapsto X^m_{j-1}(\emptyset)] \models \phi(p, r)$. For each $r \in X^m_j$ we thus add a child to $m$ and label it with $X^r_{j-1}$. By definition of $\phi^{i+1}(\emptyset)$, the tree is finite. Let $X$ be the union of all labels of the constructed tree. $X$ is finite, and by monotonicity of $\phi(p,x)$ we have that for every $m \in X \cup \{s\}$, $\model[p \mapsto X \cup \{s\}] \models \phi(p,m)$.

For the other direction, just notice that the smallest finite set $p^\model \subseteq T$ such that $\model[p\mapsto p^\model] \models \phi(p,s)$ and $\model[p\mapsto p^\model] \models \phi(p,m)$ for all $m \in p^\model$ is the least fixpoint $F_\varphi$. %of the function that maps any $S \subseteq T$ into $\{t \in T \mid \model[P \mapsto S] \models \phi(P, t) \}$.
%the idea is the following. By assumption there is a finite set $P^\model \subset T$ such that $\model[P\mapsto P^\model] \models \phi(P,n)$ and $\model[P\mapsto P^\model] \models \phi(P,m)$  for every $m \in P^\model$. The winning strategy for \'Eloise  in $\mc{E}(\mu P.\varphi(P,x),\model)@(\mu P.\varphi(P,x); x \mapsto n)$
%is thus define as the composition of all winning strategies in $\mc{E}(\varphi(P,x),\model[P \mapsto P^\model]))@(\varphi(P,x); x \mapsto m)$
% for $m \in P^\model$.
 \end{proof}

%The previous
\noindent Proposition~\ref{prop:cor_constructivity} naturally suggests the following translation $\mgFOETr{-}:\mlque(\prop)\to\wmso(\prop)$,
\begin{gather*}
\mgFOETr{p(x)} \df p(x) \qquad
\mgFOETr{R(x,y)} \df R(x,y) \qquad
\mgFOETr{x\foeq y} \df  (x \foeq y) \qquad
\mgFOETr{\varphi \land \psi} \df \mgFOETr{\varphi} \land \mgFOETr{\psi} \\
\mgFOETr{\lnot \varphi} \df  \lnot \mgFOETr{\varphi} \qquad\qquad
\mgFOETr{\exists x. \varphi} \df  \exists x. \mgFOETr{\varphi} \qquad\qquad
\mgFOETr{\qu x. \varphi} \df  \forall p.\exists x. (\lnot p(x) \land \mgFOETr{\varphi}) \\
\mgFOETr{\mu p. \varphi(p,x)} \df  \exists p ( p(x) \land \forall y ( p(y) \to \mgFOETr{\varphi(p,y) }))
\end{gather*}
%Note that in $\mgFOETr(\mu P. \varphi(P,x))$, the predicate $P$ which occurs in $\mgFOETr(\varphi) $ is bounded by the outermost second order existential quantifier.
%
The following theorem %, which is the analogous of Theorem \ref{thm:contransweak} but for $\mlque$,
is then an immediate corollary of Proposition~\ref{prop:cor_constructivity}.

\begin{theorem}\label{thm:guard_wmso}
For every $\phi \in \clque$ and model $(\model,\ass)$, $(\model, \ass) \models \varphi$ iff $(\model, \ass) \models \mgFOETr{\varphi}$.
%
% \begin{enumerate}
% \itemsep 0pt
% \item $\model, \ass \models \varphi$,
% \item $\model, \ass \models \mgFOETr{\varphi}$.
% \end{enumerate}
%
\end{theorem}
\begin{proof}
The proof goes by induction on the complexity of $\varphi$, the only critical step being the least fixpoint operator one. But this follows by applying Proposition \ref{prop:cor_constructivity} and the induction hypothesis.
%
%Let therefore consider $\phi$ is of the form $\mu P. \psi(P,x)$. Without loss of generality, that bounded and free predicate variables are distincts.
%We first show that $(1)$ implies $(2)$. Since $\model , \ass \models \varphi$, \'Eloise has a winning strategy $f$ in $\mc{E}(\phi,\model)@(\varphi,s_I, \ass)$.
%Define $P^\model$ to be the set of node $n \in T$ such that there is a (partial) match $\pi'$ that
%%
%\begin{enumerate}
%\itemsep 0pt
%\item is consistent with $f$, and such that
%\item every position of $\pi'$ is of the form $(\gamma,m, \ass')$, with  $P$ active in $\gamma$, and
%\item the last position of $\pi'$ is of the form $(\varphi, n, \ass')$.
%\end{enumerate}
%%
%The first observation is that since $f$ is a winning strategy, all $f$-consistent matches are finite. Moreover for every position of $\pi'$ is of the form $(\psi(P,x),m, \ass')$, we have that $\model[P \mapsto P^{\model}], \ass' \models \psi(P,m)$. We construct inductively a finite tree labelled by pairs $(x, X)$ where $x$ is a node of $\model$ and $X$ is a finite set of nodes of $\model$ as follows. First, because $\model[P \mapsto P^{\model}] , \ass \models \psi(P,x)$, so there is a finite subset $X \subseteq P^\model$ such that $\model[P \mapsto X_1] , \ass \models \psi(P,x)$. Thus we color the root with $(n, X)$. Now, assume we are given a leaf colored by $(y,Y)$. Consider an enumeration $x_1, \dots, x_k$ of $Y$. For every $i \leq k$, we add a child to $(y,Y)$ labelled by $(x_i, X_i)$ where $X_i$ is given by the fact that since $\model[P \mapsto P^{\model.x_i}] , \ass \models \psi(P,x)$, there is a finite set $X_i$ of nodes in $P^{\model.x_i}$
%
%%the only player who picks successor in a partial match $\pi'$ defined as above is \'Eloise. As a consequence of K\"onig's Lemma, $P^\model$ is finite.
%%
%%By using the induction hypothesis, it is easy to check that $\model[x \mapsto s_I, P \mapsto P^\model] \models P(x) \land \forall y ( P(y) \to ST_y(\varphi) )$.
%%
%%For the other direction, the idea is the following. Because $\model[x \mapsto s_I] \models ST_x(\varphi)$,
%%there is a finite set $P^\model$ such that $\model[x \mapsto s_I, P \mapsto P^\model] \models P(x) \land \forall y ( P(y) \to ST_y(\varphi) )$. The winning strategy for \'Eloise  in $\mc{E}(\mu P.\varphi,\model)@(\mu P.\varphi,s_I)$
%%is thus define as the composition of all winning strategies in $\mc{E}(\varphi,\model[P \mapsto P^\model])@(\varphi,s)$ for $s \in P^\model$.
\end{proof}

%\begin{remark}
%Clearly the standard translation from modal logic into $\gfoe$ extend to the modal $\mu$-calculus and $\mglque$.
%\end{remark}

\subsubsection{Translating automata into formulas}
We are now ready to formulate the converse of Theorem~\ref{t:wmsoauto}, i.e. the direction from automata to formulas of the automata characterisation of $\wmso$.

\begin{theorem}\label{thm:wmso_autofor}
There is an effective procedure that given a $\wmso$-automaton returns an equivalent $\wmso$-formula.
\end{theorem}
\begin{proof}
The argument is essentially a refinement of the standard proof showing that any automaton in $\Aut(\ofo)$ can be translated into an equivalent $\mu$-formula
$\xi_\aut$ (cf. e.g. \cite{Ven08}).
The idea is the following. We see a $\wmso$-automaton as a system of equations expressed in terms of $\lque$-formulas: each state corresponds to a monadic predicate variable and the parity of a state corresponds to the least and greatest fixpoint that we seek for the associated variable, etc. One then solves this system of equations via the same inductive procedure used to obtain the formula of the modal $\mu$-calculus from the system associated with a $\Aut(\ofo)$-automaton (see e.g. \cite{ArnoldN01} for a description of the solution procedure). Because of the (weakness) and (continuity) conditions on the starting $\wmso$-automaton $\aut$, it is thence possible to verify that the resulting fixpoint formula $\xi_\aut$ belongs to $\clque$.
\end{proof}

\begin{remark}
As a corollary of the automata characterization on trees of $\wmso$, we obtain the equivalence on this class of structures between $\wmso$ and $\clque$. This consequence should be compared to the analogous result obtained by Walukiewicz in~\cite{Walukiewicz96} for FPL (fixpoint extension of $\foe$) and MSO on trees.
\end{remark}
