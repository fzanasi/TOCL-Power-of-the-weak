Using the syntactic characterization of (co-)continuity given in the previous section for ${\olque}^+(A)$, we are now able to state a proper definition of the automata for $\wmso$ outlined in the introduction.
\begin{definition}
\label{d:wmso-aut}
A \emph{$\wmso$-automaton} $\tup{A,\Delta,\Omega,a_I}$ is an automaton in
$\yvAut(\olque)$ such that, for all states $a,b \in A$ with $a \ord b$ and
$b\ord a$, and for all $c \in C$, the following holds:
\begin{description}
\item[(weakness)] $\pmap(a)=\pmap(b)$;
\item[(continuity)]
  if $\pmap(a)$ is odd then $\tmap(a,c)$ is in $\cont{{\olque}^+}{b}(A)$;
  if $\pmap(a)$ is even then $\tmap(a,c)$ is in $\cocont{{\olque}^+}{b}(A)$.
\end{description}
We use $\yvcwAut(\olque)$ to denote the class of such automata.
\end{definition}



\begin{remark}{\rm
Albeit similar to the finitary construction, the \emph{two-sorted construction}
(\emph{cf.}~\cite[Def.~3.7]{Zanasi:Thesis:2012}, \cite{DBLP:conf/lics/FacchiniVZ13}) used for weak $\mso$-automata would have not been suitable for our purposes, as it fails to preserve the \textbf{(continuity)} condition when applied to $\wmso$-automata. Similarly, the powerset construction used in the simulation theorem for $\mso$-automata preserves neither the \textbf{(weakness)} nor the \textbf{(continuity)} condition.
}\end{remark} 



\subsubsection{Simulation theorem}
%!TEX root = ../00CFVZ_TOCL.tex

In this section we show that any $\wmso$-automaton $\aut$ can be simulated by a ``two-sorted'' $\wmso$-automaton $\mb{A}^{\f}$. The leading intuition is that $\mb{A}^{\f}$ will consist of one copy of $\aut$ (based on a set of states $A$) and a variant of its powerset construction, which will be based both on states from $A$ and ``macro-states'' from $\pw (A \times A)$.\footnote{It is customary for powerset constructions on parity automata to encode macro-states as binary relations between states (from $\pw (A \times A)$) instead of plain sets (from $\pw A$). Such additional structure is needed to correctly associate with a run on macro-states the corresponding bundle of runs of the original automaton $\aut$. We refer to the standard literature on parity automata (e.g. \cite{Walukiewicz96,ALG02}) for further details.} Successful runs of $\mb{A}^{\f}$ will have the property of processing only a \emph{finite} amount of the input with $\mb{A}^{\f}$ being in a macro-state and all the rest with $\mb{A}^{\f}$ behaving exactly as $\aut$.
%\yvwarning{some more intuitions are required, eg on macro-states FZ: I put some more intuition on the finitary construction. I emphasized the novelties, while in my opinion the observations on macro-states being relations is standard for tree automata and should not be recalled in details (I recalled in a footnote).}

\fcwarning{consider figure, see ``soaut-'' files from PDL paper}


\begin{figure}[h]
\centering
\begin{tikzpicture}[node distance=1.3cm,>=stealth',bend angle=45,auto]
  \tikzstyle{node}=[circle,fill=black,minimum size=5pt,inner sep=1pt]
  \tikzstyle{place}=[circle,thick,draw=blue!75,fill=blue!20,minimum size=6mm]
  \tikzstyle{iplace}=[place,draw=red!75,fill=red!20]
  \tikzstyle{inter}=[dashed,draw=black!50,->]

  \begin{scope}
    % First net
    \node [node] (w1)                                    {};
    \node [node] (c1) [below of=w1]                      {};
    \node [node] (s)  [below of=c1] {};
    \node [node] (c2) [below of=s]                       {};
    \node [node] (w2) [below of=c2]                      {};

    \node [node] (e1) [left of=c1] {}
      edge [pre,bend left]                  (w1)
      edge [post,bend right]                (s)
      edge [post]                           (c1);

    \node [node] (e2) [left of=c2] {}
      edge [pre,bend right]                 (w2)
      edge [post,bend left]                 (s)
      edge [post]                           (c2);

    \node [node] (l1) [right of=c1] {}
      edge [pre]                            (c1)
      edge [pre,bend left]                  (s)
      edge [post,bend right] (w1);

    \node [node] (l2) [right of=c2] {}
      edge [pre]                            (c2)
      edge [pre,bend right]                 (s)
      edge [post,bend left]         (w2);
  \end{scope}

  \begin{scope}[xshift=-6cm]
    % Second net
    \node [iplace,tokens=1] (w1')   {}
      edge [inter,bend angle=20, bend left] (e1);

    \node [place,tokens=2] (c1') [below of=w1'] {}
    	edge [inter,bend angle=30, bend right] (l1);

    \node [place,tokens=2] (s1') [below of=c1',xshift=-5mm]      {};
    \node [place,tokens=3]
                      (s2') [below of=c1',xshift=5mm] {};
    \node [place,tokens=4]     (c2') [below of=s1',xshift=5mm]                      {}
    	edge [inter,bend angle=20, bend right] (l2);

    \node [place,tokens=3]
                      (w2') [below of=c2']                                 {};

    \node [place,tokens=3] (e1') [left of=c1'] {}
      edge [pre,bend left]                  (w1')
      edge [post]                           (s1')
      edge [pre]                            (s2')
      edge [post]                           (c1')
      edge [inter,bend angle=20, bend right] (c1);

    \node [place,tokens=3] (e2') [left of=c2'] {}
      edge [pre,bend right]                 (w2')
      edge [post]                           (s1')
      edge [pre]                            (s2')
      edge [post]                           (c2')
      edge [inter,bend angle=20, bend right] (c2);

    \node [place,tokens=3] (l1') [right of=c1'] {}
      edge [pre]                            (c1')
      edge [pre]                            (s1')
      edge [post]                           (s2')
      edge [post,bend right]  (w1');

    \node [place,tokens=3] (l2') [right of=c2'] {}
      edge [pre]                            (c2')
      edge [pre]                            (s1')
      edge [post]                           (s2')
      edge [post,bend left]        (w2')
      edge [inter,bend angle=20, bend right] (w2);
  \end{scope}

  \begin{pgfonlayer}{background}
    \filldraw [line width=4mm,join=round,black!10]
      % (w1.north  -| l1.east)  rectangle (w2.south  -| e1.west)
      (w1'.north  -| l1.east) rectangle (w2'.south  -| e1.west)
      (w1'.north -| l1'.east) rectangle (w2'.south -| e1'.west);
  \end{pgfonlayer}
\end{tikzpicture}
\caption{Two-part construction, initial state in red (illustrative)}
\label{fig:twopart}
\end{figure}


\begin{comment} %Extended alternative version...
For the purpose of defining $\mb{A}^{\f}$, we first need some preliminaries to shape our variant of the powerset construction. As standard in automata theory, the idea is that the powerset construct $\mb{A}^{\sharp}$ on $\mb{A}$ mimics multiple runs of $\mb{A}$ acting in parallel on the same input. Thus states of $\mb{A}^{\sharp}$ will be in fact ``macro-states'', each representing a set of states of $\mb{A}$. However, in order to define a correct notion of acceptance for $\mb{A}^{\sharp}$, one needs to keep track of the structure of each run of $\mb{A}$ that is simulated in parallel. For this reason, the macro-states of $\mb{A}^{\sharp}$ are not given as plain sets of states from $\aut$ but instead as \emph{binary relations}: the idea is that a pair $(a,b)$ in a macro state encodes the information that one of the simulated runs of $\aut$ is in state $b$ at the current stage and was at state $a$ at the previous stage. We refer to the standard literature on tree automata (e.g. \cite{Walukiewicz96,ALG02}) for further details.

Next we introduce the ingredients to define our variant of the powerset construction. The following is a notion of lifting for types on states that is instrumental in defining a translation to types on macro-states.
\end{comment}


To achieve this result, we first need some preliminary definitions. The following is a notion of lifting for types on states that is instrumental in defining a translation to types on macro-states. The distinction between empty and non-empty subsets of $A$ is to make sure that empty types on $A$ are lifted to empty types on $\pw A$.

\begin{definition}\label{def:typelifting}
Given a set $A$ of unary predicates and $\Sigma \subseteq \wp A$, we define the \emph{lifting} $\lift{\Sigma} \subseteq \wp \wp A$ as $\{\{S\} \mid S \in \Sigma \wedge S \neq \emptyset\} \cup
    \{\emptyset \mid \emptyset \in \Sigma \}$.
\end{definition}

The next step is to define a translation on the sentences associated with the
transition function of the original $\wmso$-automaton, say with set of states $A$. Following the intuition given above, the idea is that we want to work with sentences that can be made true by assigning macro-states (from $\wp(A \times A)$) to finitely many nodes in the model, and ordinary states (from $A$) to all the other nodes. Henceforth, we use the notation $\shA$ for the set $\wp(A \times A)$.

\begin{definition}\label{DEF_finitary_lifting}
Let $\varphi \in {\olque}^+(A \times A)$ be a formula of shape $\posdbnfofoei{\vlist{T}}{\Pi}{\Sigma}$ for some $\Pi,\Sigma \subseteq \shA$ and $\vlist{T} = \{T_1,\dots,T_k\} \subseteq \shA$. Let $\widetilde{\Sigma}\subseteq \wp A$ be $\widetilde{\Sigma} := \{\Ran(S) \mid S \in \Sigma\}$. We define $\varphi^{\f} \in {\olque}^+(A \cup \shA )$ as follows:
$$(\posdbnfofoei{\vlist{T}}{\Pi}{\Sigma})^{\f} := \posdbnfofoei{\lift{\vlist{T}}}{\lift{\Pi} \cup \lift{\Sigma}}{\widetilde{\Sigma}}. $$
Observe that each ${\tau}^{+}_{P}$ with $P \in \widetilde{\Sigma}$ appearing in $\varphi^{\f}$ is a (positive) $A$-type, as $P = \Ran(S) \subseteq A$ for some $S \in \Sigma$.
%Following the notation of Definition \ref{def:basicform_folque}, $\varphi$ and $\varphi^{\f}$ can be represented respectively as $\dbnfolque{\vlist{T}}{\Pi}{\Sigma}$ and $\dbnfolque{\lift{\vlist{T}}}{\lift{\Pi} \cup \lift{\Sigma}}{\Sigma}$.
\end{definition}

\noindent Our desiderata on the translation $(-)^{\f}$ concern the notions of \emph{continuity} and \emph{functionality}.

\begin{definition} Given a set $A$ of unary predicates and $B \subseteq A$, we say that a sentence $\varphi \in {\olque}^+(A)$ is \emph{functionally continuous in $B$} if, for every model $(D,\val \: A \to \wp(D))$,
\begin{align*}
\text{if } (D,\val),\ass \models \varphi \text{ then } & \exists\ \val' \: A \to \wp(D) \text{ such that } (D, \val'),\ass \models \varphi, \\
& \val'(a)\subseteq \val(a) \text{ for all } a \in A, \tag{$\val'$ is a restriction of $\val$}\\
 & \val'(b) \text{ is finite for all }b \in B \text{ and } \tag{continuity in $B$}\\
 & \val'(b)\cap \val'(a) = \emptyset \text{ for all } a \in A\setminus\{b\} \text{ and }b \in B\tag{functionality in $B$}.
\end{align*}
%Moreover, $\varphi$ is \emph{functionally continuous in $B \subseteq A$} if it is so for each $b \in B$.
\end{definition}
In words, $\varphi$ is functionally continuous in $B$ if it is continuous in each $b \in B$ and, for each model $(D,\val)$ where $\varphi$ is true, there is a restriction $\val'$ of $\val$ which both witnesses continuity and does not assign any other $a \in A$ to the elements marked with some $b \in B$.

\begin{lemma}\label{LEM_cont}
Let $\varphi \in {\olque}^+(A \times A)$ and $\varphi^{\f}\in {\olque}^+(A\cup \shA )$ be given as in Definition~\ref{DEF_finitary_lifting}. Then $\varphi^{\f}$ is functionally continuous in $\shA$.
 \end{lemma}

\begin{proof}
%\yvwarning{this proof could use some more detail FZ: I expanded the proof and tried to make it clearer}
We first unfold the definition of $\varphi^{\f}$ as follows:
\begin{align*}
\varphi^{\f} =\ &
\underbrace{
    \exists \vlist{x}.\big(\arediff{\vlist{x}} \land \bigwedge_{0 \leq i \leq n} \tau^+_{\lift{T}_i}(x_i)
}_{\psi_1}
\land \underbrace{
    \forall z.(\arediff{\vlist{x},z} \lthen \bigvee_{S\in \lift{\Pi} \cup \lift{\Sigma} \cup \widetilde{\Sigma}} \tau^+_S(z))\big)
}_{\psi_2}
\land
\\ & \underbrace{
    \bigwedge_{P\in\widetilde{\Sigma}} \qu y.{\tau}^{+}_P(y)
}_{\psi_3} \land
 \underbrace{
    \dqu y.\bigvee_{P\in\widetilde{\Sigma}} {\tau}^{+}_P(y)
}_{\psi_4} .
\end{align*}
Observe that $\psi_1 \land \psi_2$ is just $\mondbnfofoe{\lift{\vlist{T}}}{\lift{\Pi} \cup \lift{\Sigma} \cup \widetilde{\Sigma}}{+}$. Now suppose that $(D,\val \: (A \cup \shA ) \to \wp(D))$ is a model where $\varphi^{\f}$ is true. This amounts to the truth of subformulas $\psi_1$, $\psi_2$, $\psi_3$ and $\psi_4$ whose syntactic shape yields information on the types of elements of $D$. In particular, we can define a partition of $D$ into subsets $D_1$, $D_2$, $D'_2$ as follows:
\begin{itemize}
  \item As $\psi_1$ is true, we can pick $n$ distinct elements $s_1,\dots,s_n$ of $D$ such that $s_i$ witnesses the positive type $\lift{T}_i$, %\tau^+_{\lift{T}_i}(x_i)$,
   that is, $s_i \in \val(S)$ for each $S \in \lift{T}_i$. We define $D_1 := \{s_1,\dots,s_n\}$.
  %
  \item  As $\psi_2$ is true, we can cover all the elements not in $D_1$ with two disjoint sets $D_2$ and $D'_2$ given as follows. The set $D_2$ is defined to contain all the elements not in $D_1$ witnessing a type ${\tau}^{+}_P(z)$ with $P \in \widetilde{\Sigma}$. The set $D'_2$ is just the complement of $D_1 \cup D_2$: by syntactic shape of $\psi_2$, all elements of $D'_2$ witness a positive type ${\tau}^{+}_S$ with
  $S \in \lift{\Pi} \cup \lift{\Sigma}$.
  %
  \item The truth of the subformula $\psi_4$ yields the information that the set $D_1 \cup D'_2$ is finite. If $\widetilde{\Sigma}$ is non-empty, the truth of $\psi_3$ implies that the set $D_2$ is infinite.
 \end{itemize}
This partition uniquely associates with each $s \in D$ a type ${\tau}^{+}_S$ witnessed by $s$ and thus a set of unary predicates $S_s := S \subseteq A \cup \shA$. We can then define a valuation $\val'$ assigning to each element $s$ of $D$ exactly the set $S_s$.

We now check the properties of $\val'$. As the partition inducing $\val'$ follows the syntactic shape of $\varphi^{\f}$, one can observe that $\val'$ is a restriction of $\val$ and $(D,\val')$ makes $\varphi^{\f}$ true. By definition of the partition, $\val'$ assigns unary predicates from $\shA$ only to elements in the finite set $D_1 \cup D'_2$, meaning that $\varphi^{\f}$ is continuous in $\shA$. Furthermore, $\val'$ assigns at most one unary predicate from $\shA$ to each element of $D_1 \cup D'_2$, because $\lift{\vlist{T}} \cup \lift{\Pi} \cup \lift{\Sigma}$ is defined as the lifting of $\vlist{T} \cup \Pi \cup \Sigma$. It follows that $\varphi^{\f}$ is also functional in $\shA$. Since the same restriction $\val'$ yields both properties, $\varphi^{\f}$ is functionally continuous in $\shA$.
\end{proof}

\begin{remark} As $\varphi^{\f}$ is of shape $\posdbnfofoei{\lift{\vlist{T}}}{\lift{\Pi} \cup \lift{\Sigma}}{\widetilde{\Sigma}}$ with $R \not\in \bigcup\widetilde{\Sigma}$ for each $R \in \shA$, by application of Corollary \ref{cor:olquecontinuousnf} we would immediately get that $\varphi^{\f}$ is continuous in each $R \in \shA$. However, we do not use this observation in proving Lemma \ref{LEM_cont} and propose instead a more direct argument, allowing to show both continuity and functionality at once.
\end{remark}

The next definition is standard (see e.g.  \cite{Walukiewicz96,Ven08}) as an intermediate step to define the transition function of the powerset construct for parity automata.

\begin{definition}\label{DEF_delta star} Let $\mb{A} = \tup{A,\Delta,\Omega,a_I}$ be a $\wmso$-automaton. Fix $a \in A$ and $c \in C$. The sentence $\Delta^{\star}(a,c)$ is defined as
\begin{eqnarray*}
      % \nonumber to remove numbering (before each equation)
        \Delta^{\star}(a,c) &:=& \Delta(a,c)[b \mapsto (a,b) \mid b \in A].
      \end{eqnarray*}
%where $\Delta(a,c)[(a,b)\setminus b\mid b \in A]$ denotes the sentence in ${\olque}^+(A\times A)$ obtained by replacing each occurrence of an unary predicate $b \in A$ in $\Delta(a,c)$ with the unary predicate $(a,b) \in A \times A$.
\end{definition}

 Next we combine the previous definitions to characterize the transition function associated with the macro-states.

\begin{definition}\label{PROP_DeltaPowerset}
Let $\aut = \tup{A,\Delta,\Omega,a_I}$ be a $\wmso$-automaton. Let $c \in C$ be a label and $Q \in \shA$ a binary relation on $A$. By Corollary \ref{cor:olquepositivenf}, for some $\Pi,\Sigma \subseteq \shA$ and $T_i \subseteq A \times A$, there is a sentence $\Psi_{Q,c} \in {\olque}^+(A\times A)$ in the basic form $\bigvee \posdbnfofoei{\vlist{T}}{\Pi}{\Sigma}$ such that
\begin{eqnarray*}
% \nonumber to remove numbering (before each equation)
  \bigwedge_{a \in \Ran(Q)} \Delta^{\star}(a,c) &\equiv& \Psi_{Q,c}.
\end{eqnarray*}
By definition $\Psi_{Q,c}$ is of the form $\bigvee_{i}\varphi_i$, with each $\phi_{i}$ of shape $\posdbnfofoei{\vlist{T}}{\Pi}{\Sigma}$. We put $\shDe(Q,c) := \bigvee_{i}\varphi_i^{\f}$, where the translation $(-)^{\f}$ is given as in Definition~\ref{DEF_finitary_lifting}. Observe that $\shDe(Q,c)$ is of type ${\olque}^+(A \cup \shA)$.
\end{definition}

We have now all the ingredients to define our two-sorted automaton.

\begin{definition}\label{def:finitaryconstruct}
Let $\aut = \tup{A,\Delta,\Omega,a_I}$ be a {\wmso-automaton}. We define the \emph{finitary construct over $\mb{A}$} as the automaton $\aut^{\f} = \tup{A^{\f},\Delta^{\f},\Omega^{\f},a_I^{\f}}$ given by
\begin{eqnarray*}
      % \nonumber to remove numbering (before each equation)
        A^{\f} &:=& A \cup \shA \\
        %\leq^{2S} &:=& \leq\ \cup\ (\shA \times A)\ \cup\ (\shA \times \shA)\\
        a_I^{\f} &:=& \{(a_I,a_I)\}\\
        \Delta^{\f}(a,c) &:=& \Delta(a,c)\\
        \Delta^{\f}(R,c) &:=& \shDe(R,c) \vee \bigwedge_{a \in \Ran(R)} \Delta(a,c)\\
        \Omega^{\f}(a) &:=& \Omega(a)\\
        \Omega^{\f}(R) &:=& 1.
      \end{eqnarray*}
\end{definition}

The underlying idea of Definition \ref{def:finitaryconstruct} is the same of the \emph{two-sorted construction} (\emph{cf.} \cite[Def.~3.7]{Zanasi:Thesis:2012}, \cite{DBLP:conf/lics/FacchiniVZ13}) for weak $\mso$-automata. In both cases we want that macro-states process just a \emph{well-founded} portion of any accepted tree: this is guaranteed by associating all macro-states with the odd parity value $1$. However, for the finitary construction we aim at the stronger condition that such a portion is \emph{finite}. To achieve this, the key difference with the two-sorted construction is in the use of the translation $(-)^{\f}$ to define $\shDe$: as $\Delta$ may be specified using quantifiers $\qu$ and $\dqu$, it serves the purpose of tracking the cardinality constraints of the original $\wmso$-automaton and ensure that they are not lifted to constraints on macro-states.

\begin{comment} Remark on minimality
\begin{remark} Let $\mb{A}^{\f}$ be the finitary construct of some $\wmso$-automaton $\mb{A}$ and $\model$ an input model. While playing the acceptance game $\mathcal{A}(\mb{A}^{\f},\model)$, it is in $\exists$'s interest to make the fewest number of moves available for $\forall$. Thus at any position $(q,s) \in A^{\f} \times T$, a rational choice for her would be to assign to each node in $\sigma_R(s)$ only the ``strictly necessary'' amount of states that makes $\Delta^{\f}(q,\V(s))$ true. Following this intuition, we can always assume the following on $\exists$'s strategy $f$ in $\mathcal{A}(\mb{A}^{\f},\model)$:
\begin{itemize}
  \item from any position of the form $(a,s) \in A \times T$, the valuation suggested by $f$ only assigns predicates from $A$ to nodes in $\R{s}$;
  \item from any position of the form $(R,s) \in \shA \times T$, the valuation suggested by $f$ assigns either only predicates from $A$ or only predicates from $\shA$ to nodes in $\R{s}$.
\end{itemize}
The first assumption can be made as only predicates from $A$ appear in $\Delta^{\f}(a,\V(s)) = \Delta(a,\V(s)$, whereas the second depends upon the fact that $\Delta^{\f}(R,\V(s))$ consists of two disjuncts
By assuming that $\exists$ plays according to this idea of rationality, we can rule out redundant valuations such as $\val^{\prime}$. Following these intuitions, we introduce the notion of \emph{minimal strategy}.
\end{remark}
\end{comment}

The next proposition establishes the desired properties of the finitary
construct. To this aim, we first introduce the notions of functional and finitary strategy.

\begin{definition}\label{def:StratfunctionalFinitary}
Given a $\wmso$-automaton $\bbA = \tup{A,\tmap,\pmap,a_I}$ and transition system $\bbT$, a strategy $f$ for \eloise in $\mathcal{A}(\bbA,\model)$ is \emph{functional in $B \subseteq A$} (or simply functional, if $B=A$) if for each node $s$ in $\bbT$ there is at most one $b \in B$ such that $(b,s)$ is a reachable position in an $f$-guided match. Also $f$ is \emph{finitary} in $B$ if there are only finitely many nodes $s$ in $\bbT$ for which a position $(b,s)$ with $b \in B$ is reachable in an $f$-guided match.
\end{definition}

% we say that a strategy for $\exists$ in an acceptance game $\mathcal{A}(\aut,\model)$
% is \emph{functional in a set $B$} of states if it assigns at most one state
% from $B$ to each node in $\mb{T}$, and it is \emph{finitary in $B$} if it
% assigns states from $B$ only to finitely many nodes of $\mb{T}$.

\begin{lemma}\label{PROP_facts_finConstr} Let $\mb{A}$ be a $\wmso$-automaton and $\mb{A}^{\f}$ its finitary construct over $\mb{A}$. The following holds:
\begin{enumerate}
  \itemsep 0 pt
  \item $\mb{A}^{\f}$ is a $\wmso$-automaton. \label{point:finConstrAut}
  \item For any $\mb{T}$, if $\exists$ has a winning strategy in
  the game $\mathcal{A}(\aut^{\f},\model)@(a_I^{\f},s_I)$, then she has a winning strategy in the same game which is both functional and finitary in $\shA$. \label{point:finConstrStrategy}
  \item $\mb{A} \equiv \mb{A}^{\f}$. \label{point:finConstrEquiv}
  \end{enumerate}
\end{lemma}
\begin{proof} %As observed above, the finitary construction resembles the two-sorted construction.
We address each point separately.
\begin{enumerate}
  \item We need to show that $\mb{A}^{\f}$ is weak and respects the continuity condition. For this purpose, we fix the following observation:
      \begin{itemize}
      \item[($\star$)] by definition of $\Delta^{\f}$, for any macro-state $R \in \shA$ and state $a \in A$, it is never the case that $a \preceq R$.
      \end{itemize}
      This means that, when considering a strongly connected component of $\mb{A}^{\f}$, we may assume that all states involved are either from $A$ or from $\shA$.

      In order to prove our claim, let $q_1,q_2 \in A^{\f}$ be two states of $\mb{A}^{\f}$ such that $q_1 \preceq q_2$ and $q_2 \preceq q_1$. By observation ($\star$), we can distinguish the following two cases:
      \begin{enumerate}[(\roman*)]
        \item if $q_1$ and $q_2$ are states from $A$, then for $i \in \{1,2\}$ the value of $\Omega^{\f}(q_i)$ and $\Delta^{\f}(q_i,c)$ is defined respectively as $\Omega(q_i)$ and $\Delta(q_i,c)$ like in the $\wmso$-automaton $\mb{A}$. It follows that they satisfy both the continuity and weakness condition.
        \item Otherwise, $q_1$ and $q_2$ are macro-states in $\shA$. For the weakness condition, observe that all macro-states in $\aut^{\f}$ have the same parity value. For the continuity condition, suppose that $q_1$ occurs in $\Delta^{\f}(q_2,c)$ for some $c \in C$. By definition $q_1$ can only appear in the disjunct $\shDe(q_2,c) = \bigvee_{i}\varphi_i^{\f}$ of $\Delta^{\f}(q_2,c)$. By Lemma \ref{LEM_cont}, we know that each $\varphi_i^{\f}$ is continuous in $\shA$. Then in particular $\Delta^{\f}(q_2,c)$ is continuous in $q_1$. By definition $\Omega^{\f}(q_1) =1$ is odd, meaning that the continuity condition holds. The case in which $q_2$ appears in $\Delta^{\f}(q_1,c)$ is just symmetric.
      \end{enumerate}
  \item  Let $f$ be a winning strategy for $\exists$ in $\mathcal{A}(\mb{A}^{\f},\model)@(a_I^{\f},s_I)$. We define a strategy $f'$ for $\exists$ in the same game as follows:
      \begin{enumerate}[label=(\alph*),ref=\alph*]
        \item on basic positions of the form $(a,s) \in A\times T$, let $\val$ be the valuation suggested by $f$. We let the valuation suggested by $f'$ be the restriction $\val'$ of $\val$ to $A$. Observe that, as no predicate from $A^{\f}\setminus A =\shA$ occurs in $\Delta^{\f}(a,\V(s)) = \Delta(a,\V(s))$, then $\val'$ also makes that sentence true in $\R{s}$.
        \begin{comment} With minimality
        on basic positions of the form $(a,s) \in A\times T$, $f'$ is defined as $f$. Indeed, as no predicate from $\shA$ occurs in $\Delta^{\f}(a,\V(s))$, we can assume that the valuation suggested by $f$ does not assign any of them to nodes in $\R{s}$.
        \end{comment}
        \label{point:stat2point1}
        \item for basic positions of the form $(R,s) \in \shA \times T$, let $\val_{R,s}$ be the valuation suggested by $f$. As $f$ is winning, $\Delta^{\f}(R,\V(s))$ is true in the model $\val_{R,s}$. If this is because the disjunct $\bigwedge_{a \in \Ran(R)} \Delta(a,\V(s))$ is made true, then we can let $f'$ suggest the restriction to $A$ of $\val_{R,s}$, for the same reason as in \eqref{point:stat2point1}. Otherwise, the disjunct $\shDe(R,\V(s)) = \bigvee_{i}\varphi_i^{\f}$ is made true. This means that, for some $i$,
             $$(R[s], \val_{R,s}) \models \varphi_i^{\f}.$$
             By Lemma \ref{LEM_cont} $\varphi_i^{\f}$ is functionally continuous in $\shA$, meaning that we have a restriction $\val_{R,s}'$ of $\val_{R,s}$ that verifies $\varphi_i^{\f}$, assigns finitely many nodes to predicates from $\shA$ and associates with each node at most one predicate from $\shA$. We let $\val_{R,s}'$ be the suggestion of $f'$ from position $(R,s)$.
      \end{enumerate}
      The strategy $f'$ defined as above is immediately seen to be
      surviving for $\exists$. It is also winning, because the set of
      basic positions on which $f'$ is defined is a subset of the one
      of the winning strategy $f$. By this observation it also follows that any $f'$-conform match visits basic positions of the form $(R,s) \in \shA \times C$ only finitely many times, as those have odd parity. By definition, the valuation suggested by $f'$ only assigns finitely many nodes to predicates in $\shA$ from positions of that shape, and no nodes from other positions. It follows that $f'$ is finitary in $\shA$. Functionality in $\shA$ also follows immediately by definition of $f'$.
  \item The proof is entirely analogous to the one presented in \cite[Prop. 3.9]{Zanasi:Thesis:2012} for the two-sorted construction. For the direction from left to right, it is immediate by definition of $\mb{A}^{\f}$ that a winning strategy for $\exists$ in $\mc{G} = \mathcal{A}(\aut,\model)@(a_I,s_I)$ is also winning for $\exists$ in $\mc{G}^{\f} = \mathcal{A}(\mb{A}^{\f},\model)@(a_I^{\f},s_I)$.

      For the direction from right to left, let $f$ be a winning strategy for $\exists$ in $\mc{G}^{\f}$. The idea is to define a strategy $f'$ for $\exists$ in stages, while playing a match $\pi'$ in $\mc{G}$. In parallel to $\pi'$, a shadow match $\pi$ in $\mc{G}^{\f}$ is maintained, where $\exists$ plays according to the strategy $f$. For each round $z_i$, we want to keep the following relation between the two matches:
\smallskip
\begin{center}
\fbox{\parbox{12cm}{
Either
\begin{enumerate}[label=(\arabic*),ref=\arabic*]
  \item basic positions of the form $(Q,s) \in \shA \times T$ and $(a,s) \in A \times T$ occur respectively in $\pi$ and $\pi'$, with $a \in \Ran(Q)$,
\end{enumerate}
or
\begin{enumerate}[label=(\arabic*),ref=\arabic*]
  \item[(2)] the same basic position of the form $(a,s) \in A \times T$ occurs in both matches.
\end{enumerate}
}}\hspace*{0.3cm}($\ddag$)
\end{center}
\smallskip
The key observation is that, because $f$ is winning, a basic position of the form $(Q,s) \in \shA \times T$ can occur only for finitely many initial rounds $z_0,\dots,z_n$ that are played in $\pi$, whereas for all successive rounds $z_n,z_{n+1},\dots$ only basic positions of the form $(a,s) \in A \times T$ are encountered. Indeed, if this was not the case then either $\exists$ would get stuck or the minimum parity occurring infinitely often would be odd, since states from $\shA$ have parity $1$.

It follows that enforcing a relation between the two matches as in ($\ddag$) suffices to prove that the defined strategy $f'$ is winning for $\exists$ in $\pi'$. For this purpose, first observe that $(\ddag).1$ holds at the initial round, where the positions visited in $\pi'$ and $\pi$ are respectively $(a_I,s_I) \in A \times T$ and $(\{(a_I,a_I)\},s_I) \in A^{\f} \times T$. Inductively, consider any round $z_i$ that is played in $\pi'$ and $\pi$, respectively with basic positions $(a,s) \in A \times T$ and $(q,s) \in A^{\f} \times T$. In order to define the suggestion of $f'$ in $\pi'$, we distinguish two cases.
\begin{itemize}
  \item First suppose that $(q,s)$ is of the form $(Q,s) \in
  \shA\times T$. By ($\ddag$) we can assume that $a$ is in $\Ran(Q)$. Let $\val_{Q,s} :A^{\f} \rightarrow \wp(\R{s})$ be the valuation suggested by $f$, verifying the sentence $\Delta^{\f}(Q,\V(s))$. We distinguish two further cases, depending on which disjunct of $\Delta^{\f}(Q,\V(s))$ is made true by $\val_{Q,s}$.
      \begin{enumerate}[label=(\roman*), ref=\roman*]
        \item If $(\R{s},\val_{Q,s})\models \bigwedge_{b \in \Ran(Q)} \Delta(b,\V(s))$, then we let $\exists$ pick the restriction to $A$ of the valuation $\val_{Q,s}$. \label{point:valuation1}
        \item If $(\R{s},\val_{Q,s})\models \shDe(Q,\V(s))$, we let $\exists$ pick a valuation $\val_{a,s}:A \rightarrow \p (\R{s})$ defined by putting, for each $b \in A$:
            \begin{align*}
            % \nonumber to remove numbering (before each equation)
               \val_{a,s}(b)\ :=\ \bigcup_{b \in \Ran(Q')} &\{t \in \R{s} \mid t \in \val_{Q,s}(Q')\} \\
               \cup\ \ \ \ \ & \{t \in \R{s} \mid t \in \val_{Q,s}(b)\} .
            \end{align*} \label{point:valuation2}
      \end{enumerate}
      It can be readily checked that the suggested move is admissible for $\exists$ in $\pi$, i.e. it makes $\Delta(a,\V(s))$ true in $\R{s}$. For case \eqref{point:valuation2}, one has to observe how $\shDe$ is defined in terms of $\Delta$. In particular, the nodes assigned to $b$ by $\val_{Q,s}$ have to be assigned to $b$ also by $\val_{a,s}$, as they may be necessary to fulfill the condition, expressed with $\qu$ and $\dqu$, that infinitely many nodes witness (or that finitely many nodes do not witness) some type.

      We now show that $(\ddag)$ holds at round $z_{i+1}$. If \eqref{point:valuation1} is the case, any next position $(b,t)\in A \times T$ picked by player $\forall$ in $\pi'$ is also available for $\forall$ in $\pi$, and we end up in case $(\ddag .2)$. Suppose instead that \eqref{point:valuation2} is the case. Given the choice $(b,t) \in A \times T$ of $\forall$, by definition of $\val_{a,s}$ there are two possibilities. First, $(b,t)$ is also an available choice for $\forall$ in $\pi$, and we end up in case $(\ddag .2)$ as before. Otherwise, there is some $Q' \in \shA$ such that $b$ is in $\Ran(Q')$ and $\forall$ can choose $(Q',t)$ in the shadow match $\pi$. By letting $\pi$ advance at round $z_{i+1}$ with such a move, we are able to maintain $(\ddag .1)$ also in $z_{i+1}$.
  \item In the remaining case, inductively we are given the same basic position $(a,s) \in A\times T$ both in $\pi$ and in $\pi'$. The valuation $\val$ suggested by $f$ in $\pi$ verifies $\Delta^{\f}(a,\V(s)) = \Delta(a,\V(s))$, thus we can let the restriction of $\val$ to $A$ be the valuation chosen by $\exists$ in the match $\pi'$. It is immediate that any next move of $\forall$ in $\pi'$ can be mirrored by the same move in $\pi$, meaning that we are able to maintain the same position --whence the relation $(\ddag.1)$-- also in the next round.
\end{itemize}
In both cases, the suggestion of strategy $f'$ was a legitimate move for $\exists$ maintaining the relation $(\ddag)$ between the two matches for any next round $z_{i+1}$. It follows that $f'$ is a winning strategy for $\exists$ in $\mc{G}$.
%
      \begin{comment} SHORTER ALTERNATIVE VERSION OF THE PROOF
      The idea is to define a strategy $f'$ for $\exists$ in stages, while playing a match $\pi'$ in $\mathcal{A}(\aut,\model)@(a_I,s_I)$. In parallel to $\pi'$, a shadow match $\pi$ in $\mathcal{A}(\mb{A}^{\f},\model)@(a_I^{\f},s_I)$ is maintained, where $\exists$ plays according to the strategy $f$. Since $f$ is winning and all macro-states from $\shA$ have an odd parity, in finitely many rounds the shadow match $\pi$ reaches a stage where $\mb{A}^{\f}$ enters a state from $A$ and ``behaves as'' $\aut$ for all successive rounds. Thus $\pi$ can be assumed to have the following structure:
       \begin{enumerate}[(I)]
         \item there is an $n$ such that, for each round $z_i$ in the initial segment $z_0,z_1,\dots,z_n$ of $\pi$, a position of the form $(R,s) \in \shA \times T$ is visited and the valuation suggested by $f$ makes the disjunct $\shDe(R,\V(s))$ of $\Delta^{\f}(R,\V(s))$ true in $\R{s}$.
         \item At round $z_{n+1}$ a basic position of the form $(Q,t) \in \shA \times T$ is visited. The valuation suggested by $f$ makes the disjunct $\bigwedge_{a \in \Ran(R)} \Delta(a,\V(t))$ of $\Delta^{\f}(Q,\V(t))$ true in $\R{t}$. \label{point:initialsegm}
         \item For all the next rounds $z_{n+2},z_{n+3},\dots$ only positions of the form $(a,s) \in A \times T$ are visited.
       \end{enumerate}
       In each round of the initial segment $z_0,z_1,\dots,z_n$ we can maintain the condition that, if a position $(R,s)$ is visited in $\pi$, then at the same round a position $(a,s)$ with $a \in \Ran(R)$ occurs in $\pi'$. This holds for the initial round $z_0$. For the next ones $z_1,\dots,z_n$, it can be enforced by defining $f'$ in terms of $f$ in the standard way shown, for instance, in the proof of \cite[Prop. 3.9]{Zanasi:Thesis:2012}.
       \fzwarning{More details to be provided}
       Once $\pi$ reaches round $z_n$, say with position $(Q,t)$, the valuation suggested by $f$ makes $\bigwedge_{a \in \Ran(R)} \Delta(a,\V(t))$ true in $\R{t}$ ({\it cf.} point \eqref{point:initialsegm}). By assumption, at round $z_n$, $\pi'$ visits a position $(b,t)$ with $b \in \Ran(Q)$. Then in particular the valuation suggested by $f$ makes $\Delta(b,\V(t))$ true, and we let it be the suggestion of $f'$ at that stage. By definition of $\Delta^{\f}$, from the next round onwards we can maintain the same basic positions in $\pi$ and $\pi'$, and let $f'$ just be defined as $f$. As $\exists$ wins $\pi$, it will also win the match $\pi'$, meaning that $f'$ is a winning strategy.
       \end{comment}
\end{enumerate}
\end{proof}

%Details on the proof have to be provided. The idea is that the first statement is true because $\mb{A}^{\f}$ is a weak automaton and its continuity property follows by the continuity of $\mb{A}$ and Lemma \ref{LEM_cont}. The argument showing the second statement should be analogous to the one relating weak $\\val{MSO}$-automata and their two-sorted construct, see \cite[Prop. 3.9]{Zanasi:Thesis:2012}. The third statement should again follow by Lemma \ref{LEM_cont}.

\begin{remark}
While the finitary construction is a variant of the two-sorted one given
in~\cite{DBLP:conf/lics/FacchiniVZ13}, it is
worth noticing that the latter would have not been suitable for our purposes.
Indeed, suppose to define the two-sorted construct $\mb{A}^{2S}$ over a
$\wmso$-automaton $\mb{A}$, analogously to the case of weak $\MSO$-automata.
Then $\mb{A}^{2S}$ will generally \emph{not} be a $\wmso$-automaton.
The problem lies in the \textbf{(continuity)} condition: since all macro-states in
$\mb{A}^{2S}$ have parity $1$, whenever two of them, say $R$ and $Q$, are such that $R \preceq Q$ and $Q \preceq R$, then the sentence $\Delta^{2S}(R,c)$ should be continuous in $Q$. But this is not necessarily the case, since the truth of $\Delta^{2S}(R,c)$ may depend upon the truth of a subformula of the form $\exists^{\infty}x.Q(x)$, requiring $Q$ to be interpreted over infinitely many nodes. (This problem is overcome in the finitary construction by using the translation $(-)^{\f}$ to define $\Delta^{\f}$.)

As a consequence, we cannot use the two-sorted construction to show that
$\wmso$-automata are closed under noetherian projection
(\cite[Def. 3]{DBLP:conf/lics/FacchiniVZ13}).
This observation is coherent with the fact that $\nmso$ is \emph{not} a fragment of $\wmso$. Similarly, the simulation theorem for $\mso$-automata \cite{Walukiewicz96} preserves neither the \textbf{(weakness)} nor the \textbf{(continuity)} condition and thus it cannot show
closure under (arbitrary) projection for $\wmso$-automata.\end{remark} 

\subsubsection{From formulae to automata}
In this subsection we conclude the proof of Theorem \ref{t:wmsoauto}. %, showing that $\wmso$-automata are closed under the  operations corresponding to the connectives of $\mso$, that is: union, complementation and projection with respect to finite sets.We start with the latter.

We first focus on the case of projection with respect to finite sets, which exploits our simulation result, Theorem \ref{PROP_facts_finConstrwmso}.
%%%%
%%%% PROJECTION
%%%%



\subsubsection{Closure under Finitary Projection}

\begin{definition}\label{DEF_fin_projection}
Let $\aut = \tup{A, \Delta, \Omega, a_I}$ be a $\wmso$-automaton on alphabet $\p(\prop \cup \{p\})$. Let $\aut^{\f}$
denote its finitary construct.
We define the automaton ${{\exists}_F p}.\mb{A} = \langle A^{\f}, a_I^{\f},
\DeltaProj, \Omega^{\f}\rangle$ on alphabet $\p\prop$ by putting
\begin{eqnarray*}
% \nonumber to remove numbering (before each equation)
  \DeltaProj(a,c) &:=& \Delta^{\f}(a,c) \qquad \qquad�
  \DeltaProj(R,c) &:=& \Delta^{\f}(R,c) \vee \Delta^{\f}(R,c\cup\{p\}).
\end{eqnarray*}
The automaton ${{\exists}_F p}.\mb{A}$ is called the \emph{finitary projection
construct of $\mb{A}$ over $p$}.
\end{definition}

Our projection construction corresponds to a suitable closure operation on tree languages, modeling the semantics of $\wmso$ existential quantification.

\begin{definition}\label{def:tree_finproj} Let $p$ be a propositional letter and $L$ a tree language of $\p (\prop\cup\{p\})$-labeled trees. The \emph{finite projection} of $L$ over $p$ is the language ${\exists}_F p.L$ of $C$-labeled trees defined as
\begin{equation*}
    {\exists}_F p.L = \{\model \mid \text{there is a $p$-variant } \model[p\mapsto S] \text{ of } \model \text{ such that } \model[p\mapsto S] \in L \text{ and } S \text{ is finite} \}.
\end{equation*}\hfill
\end{definition}

\begin{lemma}\label{PROP_fin_projection}
For each $\wmso$-automaton $\aut$ on alphabet $\p (\prop \cup \{p\})$,
we have that
$$\trees({{\exists}_F p}.\mb{A}) \ \equiv\
{{\exists}_F p}.\trees(\mb{A}).
$$
\end{lemma}

\begin{proof}
\fcwarning{Showing this for $\aut^F$ and using Lemma~\ref{PROP_facts_finConstr}(3) leads to a way simpler proof.}What we need to show is that for any tree $\model$:
\begin{eqnarray*}
  {{\exists}_F p}.\mb{A} \text{ accepts } \mathbb{T} & \text{ iff }& \text{there is a finite $p$-variant }\model' \\
   & & \text{of }\mathbb{T}\text{  such that }\aut\text{  accepts }\model'.
\end{eqnarray*}
For direction from left to right, we first observe that the properties stated by Lemma~\ref{PROP_facts_finConstr} hold for ${{\exists}_F p}.\mb{A}$ as well, since the latter is defined in terms of $\mb{A}^{\f}$. Then we can assume that the given winning strategy $f$ for $\exists$ in $\mc{G_{\exists}} = \mc{A}({{\exists}_F p}.\mb{A},\model)@(a_I^{\f},s_I)$ is functional and finitary in $\shA$. Functionality allows us to associate with each node $s$ either none or a unique state $Q_s \in \shA$ (\emph{cf.} \cite[Prop. 3.12]{Zanasi:Thesis:2012}). We now want to isolate  the nodes that $f$ treats ``as if they were labeled with $p$''. For this purpose, let $\val_{s}$ be the valuation suggested by $f$ from a position $(Q_s,s) \in \shA \times T$. As $f$ is winning, $\val_{s}$ makes $\DeltaProj(Q,\tscolors(s))$ true in $\R{s}$. We define a $p$-variant $\model'$ of $\model$ by \fcwarning{Why the tilde?}coloring with $p$ all nodes in the following set:
 \begin{equation}\label{eq:X_p}
% \nonumber to remove numbering (before each equation)
   X_p\ :=\ \{s \in T\mid (\R{s},\widetilde{\val}_{s}) \models \Delta^{\f}(Q_s,\tscolors(s)\cup\{p\})\}.
\end{equation}
The fact that the strategy of $\exists$ is finitary in $\shA$ guarantees that $X_p$ is finite, whence $\model'$ is a finite $p$-variant. The argument showing that $\mb{A}^{\f}$ (and thus also $\mb{A}$, by Lemma~\ref{PROP_facts_finConstr}(1))\fcwarning{isn't this (3)?} accepts $\model'$ is a routine adaptation of the analogous proof for the noetherian projection of weak $\mso$-automata, for which we refer to \cite[Prop. 3.12]{Zanasi:Thesis:2012}.
\medskip

For the direction from right to left, let $\model'$ be a finite $p$-variant of
$\model$, with labeling function $\tscolors'$, and $g$ a winning strategy for $\exists$ in $\mc{G} = \mathcal{A}(\aut,\model')@(a_I,s_I)$. Our goal is to define a strategy $g'$ for $\exists$ in $\mc{G_{\exists}}$. As usual, $g'$ will be constructed in stages, while playing a match $\pi'$ in $\mc{G_{\exists}}$. In parallel to $\pi'$, a \emph{bundle} $\mc{B}$ of $g$-guided shadow matches in $\mc{G}$ is maintained, with the following condition enforced for each round $z_i$ (\emph{cf.} \cite[Prop.~ 3.12]{Zanasi:Thesis:2012}) :
\smallskip
\begin{center}
\fbox{\parbox{13cm}{
\begin{enumerate}
  \item If the current (i.e. at round $z_i$) basic position in $\pi'$ is of the form $(Q,s) \in \shA \times T$, then for each $a \in\Ran(Q)$ there is an $g$-guided (partial) shadow match $\pi_a$ at basic position $(a,s) \in A\times T$ in the current bundle $\mc{B}_i$. Also, either $\model'_s$ is not $p$-free (i.e., it does contain a node $s'$ with $p \in \tscolors'(s')$) or $s$ has some sibling $t$ such that $\model'_t$ is not $p$-free.
  \item Otherwise, the current basic position in $\pi'$ is of the form $(a,s) \in A \times T$ and $\model'_s$ is $p$-free (i.e., it does not contain any node $s'$ with $p \in \tscolors'(s')$). Also, the bundle $\mc{B}_i$ only consists of a single $g$-guided match $\pi_a$ whose current basic position is also $(a,s)$.
\end{enumerate}
}}\hspace*{0.3cm}($\ddag$)
\end{center}
\smallskip
We briefly recall the idea behind condition ($\ddag$). Point ($\ddag.1$) describes the part of match $\pi'$ where it is still possible to encounter nodes which are labeled with $p$ in $\model'$. As $\DeltaProj$ only takes the letter $p$ into account when defined on macro-states in $\shA$, we want $\pi'$ to visit only positions of the form $(R,s) \in \shA \times T$ in that situation. Anytime we visit such a position $(R,s)$ in $\pi'$, the role of the bundle is to provide one $g$-guided shadow match at position $(a,s)$ for each $a \in \Ran(R)$.
Then $g'$ is defined in terms of what $g$ suggests from those positions.

 Point ($\ddag.2$) describes how we want the match $\pi'$ to be
 played on a $p$-free subtree: as any node that one might encounter has the same label in $\model$ and $\model'$,
it is safe to let ${{\exists}_F p}.\mb{A}$ behave as $\aut$ in such situation. Provided that the two matches visit the same basic positions, of the form $(a,s)\times A \times T$, we can let $g'$ just copy $g$.

The key observation is that, as $\model'$ is a \emph{finite} $p$-variant of $\model$, nodes labeled with $p$ are reachable only for finitely many rounds of $\pi'$. This means that, provided that ($\ddag$) hold at each round, ($\ddag.1$) will describe an initial segment of $\pi'$, whereas ($\ddag.2$) will describe the remaining part. Thus our proof that $g'$ is a winning strategy for $\exists$ in $\mc{G}_{\exists}$ is concluded by showing that ($\ddag$) holds for each stage of construction of $\pi'$ and $\mc{B}$.

\medskip

For this purpose, we initialize $\pi'$ from position $(\shai,s) \in \shA\times T$ and the bundle $\mc{B}$ as $\mc{B}_0 = \{\pi_{a_I}\}$, with $\pi_{a_I}$ the partial $g$-guided match consisting only of the position $(a_I,s)\in A\times T$. The situation described by ($\ddag .1$) holds at the initial stage of the construction.
Inductively, suppose that at round $z_i$ we are given a position $(q,s) \in A^{\f} \times T$ in $\pi^{\f}$ and a bundle $\mc{B}_i$ as in ($\ddag$). To show that ($\ddag$) can be maintained at round $z_{i+1}$, we distinguish two cases, corresponding respectively to situation ($\ddag.1$) and ($\ddag.2$) holding at round $z_i$.
\begin{enumerate}[label = (\Alph*), ref = \Alph*]
%\yvwarning{Notation `$q$' is confusing, see $\val'(q)$ below FZ: I corrected $q$ into $q'$ below}
  \item If $(q,s)$ is of the form $(Q,s) \in \shA \times T$, by inductive hypothesis we are given with $g$-guided shadow matches $\{\pi_a\}_{a \in \Ran(Q)}$ in $\mc{B}_i$. For each match $\pi_a$ in the bundle, we are provided with a valuation $\val_{a,s}: A \rightarrow \p (\R{s})$ making $\Delta(a,\tscolors'(s))$ true. Then we further distinguish the following two cases.
\begin{enumerate}[label = (\roman*), ref = \roman*]
  \item \label{point:TsNotPFree} Suppose first that $\model'_s$ is not $p$-free. We let the suggestion $\val' \: A^{\f} \to \p (\R{s})$ of $g'$ from position $(Q,s)$ be defined as follows:
       \begin{align*}
       % \nonumber to remove numbering (before each equation)
       %\widetilde{\val}_{Q,s}(Q') &:=& \bigcup_{a \in \Ran(Q),\ b \in \Ran(Q')}\{t\ \in \R{s}|\ t \in \val_{a,s}(b)\}.
       \val'(q')\ :=\ \begin{cases}
               \bigcap\limits_{\substack{(a,b) \in q',\\ a \in \Ran(Q)}}\{t\ \in \R{s} \mid t \in \val_{a,s}(b)\}               & q' \in \shA \\[2em]
               \bigcup\limits_{a \in \Ran(Q)} \{t\ \in \R{s} \mid t \in \val_{a,s}(q') \text{ and }\model'.t\text{ is $p$-free}\}              & q' \in A.
               %\\[1.5em]               \hspace{.6cm}\emptyset & \text{otherwise.}
           \end{cases}
       \end{align*}
       The definition of $\val'$ on $q' \in \shA$ is standard (\emph{cf.}~\cite[Prop. 2.21]{Zanasi:Thesis:2012}) and guarantees a correspondence between the states assigned by the markings $\{\val_{a,s}\}_{a \in \Ran(Q)}$ and the macro-states assigned by $\val'$. The definition of $\val'$ on $q' \in A$ aims at fulfilling the conditions, expressed via $\qu$ and $\dqu$, on the number of nodes in $\R{s}$ witnessing (or not) some $A$-types. Those conditions are the ones that $\shDe(Q,\tscolors'(s))$ --and thus also $\Delta^{\f}(Q,\tscolors'(s))$-- ``inherits'' by $\bigwedge_{a \in \Ran(R)} \Delta(a,\tscolors'(s))$, by definition of $\shDe$. Notice that we restrict $\val'(q')$ to the nodes $t \in \val_{a,s}(q')$ such that $\model'.t$ is $p$-free. As $\model'$ is a \emph{finite} $p$-variant, only \emph{finitely many} nodes in $\val_{a,s}(q')$ will not have this property. Therefore their exclusion, which is crucial for maintaining condition ($\ddag$) (\emph{cf.}~case \eqref{point:ddag2CardfromMacro} below), does not influence the fulfilling of the cardinality conditions expressed via $\qu$ and $\dqu$ in $\shDe(Q,\tscolors'(s))$.

       On the base of these observations, one can check that $\val'$ makes $\shDe(Q,\tscolors'(s))$--and thus also $\Delta^{\f}(Q,\tscolors'(s))$--true in $\R{s}$. In fact, to be a legitimate move for $\exists$ in $\pi'$, $\val'$ should make $\DeltaProj(Q,\tscolors(s))$ true: this is the case, for $\Delta^{\f}(Q,\tscolors'(s))$ is either equal to $\Delta^{\f}(Q,\tscolors(s))$, if $p \not\in \tscolors'(s)$, or to $\Delta^{\f}(Q,\tscolors(s)\cup\{p\})$ otherwise. In order to check that we can maintain $(\ddag)$, let $(q',t) \in A^{\f} \times T$ be any next position picked by $\forall$ in $\pi'$ at round $z_{i+1}$. As before, we distinguish two cases:
       \begin{enumerate}[label = (\alph*), ref = \alph*]
         \item If $q'$ is in $A$, then, by definition of $\val'$, $\forall$ can choose $(q',t)$ in some shadow match $\pi_a$ in the bundle $\mc{B}_i$. We dismiss the bundle --i.e. make it a singleton-- and bring only $\pi_a$ to the next round in the same position $(q',t)$. Observe that, by definition of $\val'$, $\model'.t$ is $p$-free and thus ($\ddag.2$) holds at round $z_{i+1}$. \label{point:ddag2CardfromMacro}
         \item Otherwise, $q'$ is in $\shA$. The new bundle $\mc{B}_{i+1}$ is given in terms of the bundle $\mc{B}_i$: for each $\pi_a \in \mc{B}_i$ with $a\in \Ran(Q)$, we look if for some $b \in \Ran(q')$ the position $(b,t)$ is a legitimate move for $\forall$ at round $z_{i+1}$; if so, then we bring $\pi_a$ to round $z_{i+1}$ at position $(b,t)$ and put the resulting (partial) shadow match $\pi_b$ in $\mc{B}_{i+1}$. Observe that, if $\forall$ is able to pick such position $(q',t)$ in $\pi'$, then by definition of $\val'$ the new bundle $\mc{B}_{i+1}$ is non-empty and consists of an $g$-guided (partial) shadow match $\pi_b$ for each $b \in \Ran(q')$. In this way we are able to keep condition ($\ddag.1$) at round $z_{i+1}$.
       \end{enumerate}
    \item Let us now consider the case in which $\model'_s$ is $p$-free. We let $g'$ suggest the valuation $\val'$ that assigns to each node $t \in \R{s}$ all states in $\bigcup_{a \in \Ran(Q)}\{b \in A\ |\ t \in \val_{a,s}(b)\}$. It can be checked that $\val'$ makes $\bigwedge_{a \in \Ran(Q)} \Delta(a,\tscolors'(s))$ -- and then also $\Delta^{\f}(Q,\tscolors'(s))$ -- true in $\R{s}$. As $p \not\in \tscolors(s)=\tscolors'(s)$, it follows that $\val'$ also makes $\DeltaProj(Q,\tscolors(s))$ true, whence it is a legitimate choice for $\exists$ in $\pi'$. Any next basic position picked by $\forall$ in $\pi'$ is of the form $(b,t) \in A \times T$, and thus condition ($\ddag.2$) holds at round $z_{i+1}$ as shown in (i.a). %\eqref{point:ddag2CardfromMacro}
  \end{enumerate}
  \item In the remaining case, $(q,s)$ is of the form $(a,s) \in A \times T$ and by inductive hypothesis we are given with a bundle $\mc{B}_i$ consisting of a single $f$-guided (partial) shadow match $\pi_a$ at the same position $(a,s)$. Let $\val_{a,s}$ be the suggestion of $\exists$ from position $(a,s)$ in $\pi_a$. Since by assumption $s$ is $p$-free, we have that $\tscolors'(s) = \tscolors(s)$, meaning that $\DeltaProj(a,\tscolors(s))$ is just $\Delta(a,\tscolors(s)) = \Delta(a,\tscolors'(s))$. Thus the restriction $\val'$ of $\val$ to $A$ makes $\Delta(a,\tscolors'(t))$ true and we let it be the choice for $\exists$ in $\tilde{\pi}$. It follows that any next move made by $\forall$ in $\tilde{\pi}$ can be mirrored by $\forall$ in the shadow match $\pi_a$.
      \begin{comment}Version with minimality:
      It follows that $\DeltaProj(a,\tscolors(t))$ is just $\Delta(a,\tscolors(t)) = \Delta(a,\tscolors'(t))$ and the same valuation suggested by $f$ in $\pi_a$ is a legitimate choice for $\exists$ in $\tilde{\pi}$. By letting $\exists$ choose such valuation, it follows that any next move made by $\forall$ in $\tilde{\pi}$ can be mirrored by $\forall$ in the shadow match $\pi_a$.
      \end{comment}
\end{enumerate}

%As explained above, since $\model'$ is a noetherian $p$-variant, then ($\ddag .1$) holds for finitely many stages of construction of $\tilde{\pi}$, whereas ($\ddag .2$) holds for all the remaining stages, by construction of $\tilde{f}$. It follows that this strategy is winning for $\exists$ in $\tilde{G}$.

\end{proof} 

%%%%%%
%%%%%% BOOLEANS
%%%%%%

\subsubsection{Closure under Boolean operations}

In this section we will show that the class of $\wmso$-automaton recognizable
tree languages is closed under the Boolean operations.
%
Start with closure under union, we just mention the following result, without
providing the (completely routine) proof.

\begin{theorem}
\label{t:cl-dis}
Let $\bbA_{0}$ and $\bbA_{1}$ be $\wmso$-automata. 
Then there is a $\wmso$-automaton $\bbA$ such that $\trees(\bbA)$ is the 
union of $\trees(\bbA_{0})$ and $\trees(\bbA_{1})$.
\end{theorem}

In order to prove closure under complementation, we crucially use that the 
one-step language $\olque$ is closed under Boolean duals.

\myparagraphns{Closure under complementation.}
Many properties of parity automata can already be determined at the one-step level.
An important example concerns the notion of complementation.


\begin{definition}
\label{d:bdual1}
Two one-step formulas $\varphi$ and $\psi$ are each other's \emph{Boolean dual}
if for every structure $(D,\val)$ we have:
\[
(D,\val) \models \varphi \quad\text{iff}\quad (D,\val^{c}) \not\models \psi,
\]
where $\val^{c}$ is the valuation given by $\val^{c}(a) \mathrel{:=} D
\setminus \val(a)$, for all $a$.
%
A one-step language $\llang$ is \emph{closed under Boolean duals} if for every
set $A$, each formula $\varphi \in \llang(A)$ has a Boolean dual $\dual{\varphi}
\in \llang(A)$.
\end{definition}

Following ideas from~\cite{Muller1987,DBLP:conf/calco/KissigV09}, we can use Boolean duals, together with a
\emph{role switch} between $\abelard$ and $\eloise$, in order to define a
negation or complementation operation on automata.

\begin{definition}
\label{d:caut}
Assume that, for some one-step language $\llang$, the map $\dual{(-)}$
provides, for each set $A$, a Boolean dual $\dual{\varphi} \in \llang(A)$ for each
$\varphi \in \llang(A)$.
Given $\aut = \tup{A,\tmap,\pmap,a_I}$ in $\Aut(\llang)$ we define its
\emph{complement} $\dual{\aut}$ as the automaton
$\tup{A,\dual{\tmap},\dual{\pmap},a_I}$
where $\dual{\tmap}(a,c) := \dual{(\tmap(a,c))}$, and $\dual{\pmap}(a)
:= 1 + \pmap(a)$, for all $a \in A$ and $c \in \wp(\props)$.
\end{definition}

\begin{proposition}
\label{prop:autcomplementation}
Let $\llang$ and $\dual{(-)}$ be as in the previous definition.
For each automaton $\aut \in \Aut(\llang)$ and each transition system
$\model$ we have that
\[
\dual{\aut} \text{ accepts } \model
\quad\text{iff}\quad
\aut \text{ rejects } \model.
\]
\end{proposition}

The proof of Proposition~\ref{prop:autcomplementation} is based on the fact
that the power of $\eloise$ in $\agame(\dual{\aut},\model)$ is the same
as that of $\abelard$ in $\agame(\aut,\model)$, as defined in~\cite{DBLP:conf/calco/KissigV09}.

As an immediate consequence of this proposition, one may show that if the
one-step language $\llang$ is closed under Boolean duals, then the class
$\Aut(\llang)$ is closed under taking complementation.
Further on we will use Proposition~\ref{prop:autcomplementation} to show that
the same may apply to some subclasses of $\Aut(\llang)$.


\begin{theorem}
\label{t:cl-cmp}
Let $\bbA$ be an $\wmso$-automaton.
Then the automaton $\overline{\aut}$ defined in Definition~\ref{d:caut} is a
$\wmso$-automaton recognizing the complement of $\trees(\bbA)$.
\end{theorem}

\begin{proof}
Since we already know that $\overline{\bbA}$ accepts exactly the transition
systems that are rejected by $\bbA$, we only need to check that 
$\overline{\bbA}$ is indeed a $\wmso$-automaton.
But this is straightforward: for instance, continuity can be checked by 
observing the self-dual nature of this property.
\end{proof}


%%%%
%%%% PROOF THEOREM
%%%%

\subsection{Proof of Theorem \ref{t:wmsoauto}}

\begin{proof} The proof is by induction on $\varphi$.
\begin{itemize}
  \item For the base case $\varphi = p \inc q$, the corresponding 
  $\wmso$-automaton is provided in \cite[Ex. 2.6]{Zanasi:Thesis:2012}. 
  For the base case $\varphi = R(p,q)$, we give the corresponding 
  $\wmso$-automaton $\aut_{R(p,q)} = \tup{A,\Delta,\Omega,a_I}$ below:
\begin{eqnarray*}
        A &:=& \{a_0,a_1\}\\
        a_I &:=& a_0\\
  \Delta(a_0,c) &:=& \left\{
	\begin{array}{ll}
           \exists x. a_1(x) \wedge \forall y. a_0(y) & \mbox{if }p \in c 
	\\ \forall x\ (a_0(x)) & \mbox{otherwise}
	\end{array}
\right. \\
  \Delta(a_1,c) &:=& \left\{
	\begin{array}{ll}
        \top & \mbox{if }q \in c \\
        \bot & \mbox{otherwise}
	\end{array}
\right. \\
    \Omega(a_0) &:=& 0\\
    \Omega(a_1) &:=& 1.
\end{eqnarray*}
Note that the $\mso$-automaton for $R(p,q)$ provided in 
\cite[Ex. 2.5]{Zanasi:Thesis:2012} is \emph{not} a $\wmso$-automaton, as the 
continuity property does not hold.

\item
For the Boolean cases, where $\varphi = \psi_1 \vee \psi_2$ or $\phi = \neg\psi$
we refer to the closure properties of recognizable tree languages, see 
Theorem~\ref{t:cl-dis} and Theorem~\ref{t:cl-cmp}, 
respectivel.
  
\item 
For the case $\varphi = \exists p. \psi$, consider the following chain of
equivalences, where $\aut_{\psi}$ is given by the inductive hypothesis and 
${{\exists}_F p}.\aut_{\psi}$ is constructed according to 
Definition \ref{DEF_fin_projection}:
\begin{alignat*}{2}
{{\exists}_F p}.\aut_{\psi} \text{ accepts }\mb{T} 
   & \text{ iff }
     \aut_{\psi} \text{ accepts } \mb{T}[p \mapsto X], 
     \text{ for some } X \sse_{\om} T
   & \quad\text{(Lemma~\ref{PROP_fin_projection})}
\\ & \text{ iff }
     \mb{T}[p \mapsto X] \models \psi,
     \text{ for some } X \sse_{\om} T
   & \quad\text{(induction hyp.)}
\\ & \text{ iff }
    \mb{T} \models \exists p. \psi
   & \quad\text{(semantics $\wmso$)}
\end{alignat*}
\end{itemize}
\end{proof}



\subsubsection{From automata to formulae}

In what follows, we verify that WMSO-automata capture exactly the expressive power of WMSO on the class of tree models. Since we already proved the direction from formulas into automata (Theorem~\ref{t:wmsoauto}), we just have to verify that there is a sound translation going in the other direction.
%This is done by adapting and combining the proof of Theorem~\ref{t:autofor} with the proof of Theorem~\ref{thm:contransweak} of the previous section. 
For this purpose, we first introduce a fixpoint extension of first-order logic.

\subsubsection{Fixpoint extension of  first-order logic}
%Let us start in formally defining the loosely guarded fragment of $\lque$ (see \cite{GradelW99}). %We are just interested to the behavior of this logic over trees.
%Recall that our language consists therefore only of a set

Let our first-order signature be composed of a set $\prop$ of monadic predicates (denoted with capital latin letters) and an unique binary predicate $R$.
%
%\begin{definition}
%The loosely guarded fragment $\glque(\prop)$ of $\lque(\prop)$ is the smallest collection of formulas containing all atomic formulas, closed under Boolean connectives and such that:
%\begin{itemize}
%\item if $\vlist{x}=(x_1, \dots, x_m)$ and $\vlist{y}=(y_1, \dots, y_n)$ are variables, $\phi(\vlist{x},\vlist{y})$ is a $\glque(\prop)$-formula whose free variables are among $\{x_1, \dots, x_m,y_1, \dots, y_n\}$, and $\alpha_1(\vlist{x},\vlist{y})$, \dots, $\alpha_k(\vlist{x},\vlist{y})$ are atomic formulas, then the formulas
%\begin{itemize}
%\item $\forall \vlist{y} (\bigwedge_{1\leq \ell \leq k}\alpha_\ell(\vlist{x},\vlist{y}) \to \phi(\vlist{x},\vlist{y}))$,
%\item $\dqu \vlist{y} (\bigwedge_{1\leq \ell \leq k}\alpha_\ell(\vlist{x},\vlist{y}) \to \phi(\vlist{x},\vlist{y}))$,
%\item $\exists \vlist{y} (\bigwedge_{1\leq \ell \leq k}\alpha_\ell(\vlist{x},\vlist{y}) \land \phi(\vlist{x},\vlist{y}))$ and
%\item $\qu \vlist{y} (\bigwedge_{1\leq \ell \leq k}\alpha_\ell(\vlist{x},\vlist{y}) \land \phi(\vlist{x},\vlist{y}))$
%\end{itemize}
%are in $\glque(\prop)$.
%\end{itemize}
%%with $\vlist{x}=(x_1, \dots, x_m)$, $\vlist{y}=(y_1, \dots, y_n)$ to simplify notation.
%\end{definition}
%
%
%
Analogously to the modal $\mu$-calculus, the fixpoint extension of $\lque(\prop)$ is defined by adding a fixpoint construction clause.

\begin{definition}
The fixed point logic $\mlque(\prop)$ is given by:
$$
\varphi ::= q(x) \mid R(x,y) \mid x \foeq y \mid \exists x.\varphi \mid \qu x.\varphi \mid \lnot\varphi \mid \varphi \land \varphi \mid \mu p.\varphi(p,x)
$$
where $p,q\in\prop$, $x,y\in\fovar$; moreover $p$ occurs only positively in $\varphi(p,x)$ and $x$ is the only free variable in $\varphi(p,x)$.
\end{definition}

% \begin{definition}
% The fixed point logic $\mlque(\prop)$ is obtained as follows:
%  \begin{enumerate}
%  \item For all $\varphi \in \lque(\prop)$ we have $\varphi \in \mlque(\prop)$,
%  \item Let $\phi(P, x) \in \mlque(\prop)$ such that $P\in\prop$ occurs only positively and $x$ is the only free variable, then $\mu P. \phi(P, x)$ and $\nu P. \phi(P, x)$ are formulas of $\mlque(\prop)$.
%  \end{enumerate}
% \end{definition}

% The semantics of  the fixpoint formulas $\mu P. \phi(P, x)$ and $\nu P. \phi(P, x)$ is the expected one. Given a model $\model$ and $s \in T$,  $\model \models \mu P. \phi(P, s)$ iff $s$ is in the least fixpoint of the  operator $F_\phi:\wp(T)\to \wp(T)$ defined as $F_\phi(S) := \{t \in T \mid \model[P \mapsto S] \models \phi(P, t) \}$. The semantics of $\nu P. \phi(P, x)$ is dually defined by considering the greatest instead of the least fixpoint of $F_\phi$.

The semantics of the fixpoint formula $\mu p. \phi(p, x)$ is the expected one. Given a model $\model$ and $s \in T$,  $\model \models \mu p. \phi(p, s)$ iff $s$ is in the least fixpoint of the  operator $F_\phi:\wp(T)\to \wp(T)$ defined as $F_\phi(S) := \{t \in T \mid \model[p \mapsto S] \models \phi(p, t) \}$.

Formulas of $\mlque$ may be also classified according to their alternation depth as it happens for the modal $\mu$-calculus.
The alternation-free fragment of $\mlque$ is thence defined as the collection of $\mlque$-formulas $\phi$
without nesting of greatest and least fixpoint operators, i.e. such that, for any two subformulas $\mu p.\psi_1(p,y)$ and $\nu q. \psi_2(q,z)$, predicates $p$ and $q$ do not occur free respectively in $\psi_2(q,z)$ and $\psi_1(p,y)$.

%
%
\begin{definition}
Given $p \in \prop$, we say that $\varphi \in \mlque(\prop)$ is
\begin{itemize}
\item \emph{monotone in the predicate $p$} iff for every LTS $\model$ and assignment $\ass$, \[ \text{if }\model, \ass \models \varphi \text{ and $\tsval(p) \subseteq E$, then }\model[p \mapsto E], g\models \phi\]

\item \emph{continuous in the predicate $p$} iff for every LTS $\model$ and assignment $\ass$ there exists some finite $S \subseteq_\omega \tsval(p)$ such that
$$
\model, \ass \models \varphi \quad\text{iff}\quad \model[p \mapsto S], \ass \models \varphi .
$$
\end{itemize}
\end{definition}

In the next definition, we provide a definition of the continuous fragment of $\mlque$, reminiscent of the one defined in Theorem~\ref{thm:olquecont}.
\begin{definition}
Let $\mathsf{Q}\subseteq \prop$ be a set of monadic predicates. The fragment $\cont{\mlque}{\mathsf{Q}}(\prop)$ is defined by the following rules:
$$
\varphi ::= \psi \mid q(x) \mid \exists x.\varphi(x) \mid \varphi \land \varphi \mid \varphi \lor \varphi \mid \wqu x.(\varphi,\psi) \mid \mu p. \phi'(p, x)
$$
where $q \in \mathsf{Q}$, $\psi \in \mlque(\prop\setminus \mathsf{Q})$, $p \in \prop \setminus \mathsf{Q}$, $\wqu x.(\varphi,\psi) := \forall x.(\varphi(x) \lor \psi(x)) \land \dqu x.\psi(x)$ and $\phi'(p,x)$ is a formula with only $x$ free such that $\phi'(p,x) \in \cont{\mlque}{\mathsf{Q} \cup\{p\}}(\prop)$.

%The $\contAFMC$-fragment of $\mlque$  is  obtained by adding to $\glque$ the following (semantic) rule for constructing fixed point formulas.
% \begin{itemize}
% \item given a monadic predicate letter $P$, a first-order variable $x$, and a formula $\phi(P, x)$  that contains only positive occurrences of $P$ and no free variable other than $x$, if $\phi(P,x)$ is a formula in the fragment that is continuous in $P$ then $\mu P. \phi(P, x)$ is also a formula of  the fragment. Dually for $\nu P. \phi(P, x)$.\fcwarning{`positivity' is syntactic but `continuity' is semantic}
% \end{itemize}
\end{definition}

%We then verify that formulas in $\cont{\mlque}{A}(\prop)$ are (semantical) continuous in $A$. The proof is

\begin{lemma}\label{lem:colqueiscont_mu}
If $\varphi \in \cont{\mlque}{\mathsf{Q}}(\prop)$ then $\varphi$ is continuous in (each predicate from) $\mathsf{Q}$.
\end{lemma}
%
\begin{proof} First, notice that If $\varphi \in \cont{\mlque}{\mathsf{Q}}(\prop)$ then $\varphi$ is monotone  in (each predicate from) $\mathsf{Q}$. %This is proved as for
The proof goes then by induction on the complexity of $\varphi$. For the all the cases except the fixpoint one, the proof is the same as the one for Lemma~\ref{lem:colqueiscont}. For $\phi=\mu p. \phi'(p, x)$, with $\phi'(p,x) \in \cont{\mlque}{\mathsf{Q} \cup\{p\}}(\prop)$, the argument is the same as in~\cite[Lemma 1]{Fontaine08}.
\end{proof}

% \begin{definition}
% The guarded fragment $\glque$ of $\lque$ is the smallest collection of formulas containing all atomic formulas, closed under Boolean connectives and such that:
% \begin{itemize}
% \item if $x$ and $y$ are variables and $\phi(x,y)$ is a $\glque$-formula whose free variables are among $\{x,y\}$, then the formulas 
% \begin{itemize}
% \item $\forall y (r(x,y) \to \phi(x,y))$ and $\dqu y (r(x,y) \to \phi(x,y))$,
% \item $\exists y (r(x,y) \land \phi(x,y))$ and $\qu y (r(x,y) \land \phi(x,y))$
% \end{itemize}
% are in $\glque$.
% \end{itemize}
% \end{definition}

As for the modal $\mu$-calculus, we define the fragment $\clque$ of $\mlque$ as the one where the use of the least fixed point operator is restricted to the continuous fragment. %, that the one obtained by adding to $\lque$ the following rule for constructing fixed point formulas.
 % \begin{itemize}
 % \item given a monadic predicate letter $P$, a first-order variable $x$, and a formula $\phi(P, x)$  that contains only positive occurrences of $P$ and no free variable other than $x$, if $\phi(P,x)$ is a formula in the fragment that belongs to $\cont{\mlque}{\{P\}}(\prop)$, then $\mu P. \phi(P, x)$ is also a formula of  the fragment.
 % \end{itemize}

\begin{definition}
The fragment $\clque(\prop)$ of $\mlque(\prop)$ is given by the following restriction of the fixpoint operator to the contiuous fragment:
{\small%
$$
\varphi ::= q(x) \mid R(x,y) \mid x \foeq y \mid \exists x.\varphi \mid \qu x.\varphi \mid \lnot\varphi \mid \varphi \land \varphi \mid \mu p.\varphi'(p,x)
$$}%
where $p,q\in\prop$, $x,y\in\fovar$; and $\varphi'(p,x) \in \cont{\mlque}{\{p\}}(\prop) \cap \clque(\prop)$ is such that $p$ occurs only positively in $\varphi'$ and $x$ is the only free variable in $\varphi'$.
\end{definition}

%
%
%
%
%The logic $\mglque$ can be given a semantic in terms of evaluation games extending the one given  in \cite{BerwangerG01} for $\mgfoe$ by adding rules for the generalized quantifier.
%We present it just for $\qu$, and treat the rules for $\dqu y. \phi(\vlist{x},y)$ as derived from the equivalent formula $\lnot \qu y. \lnot\phi(\vlist{x},y)$
%the universal being treated as the alternation-free fragment.
%As usual, we assume that any predicate is bounded by at most one fixpoint operator
%%, any if a predicate is bounded, then the fixpoint operator bounding it is unique,
%and that bounded and free predicates are pairwise distinct.%\fzwarning{In the table: why not a clause for $\neg$, $\vee$, $\wedge$? Meaning of $\eta$, ; and :?}
%                             \begin{table}[h]
%                              \centering
%                            \begin{tabular}{|l|c|l|c|}
%                             \hline
%                              % after \\: \hline or \cline{col1-col2} \cline{col3-col4} ...
%                              Position & Player & Admissible moves & Parity\\
%                               \hline % \hline
%                           %  $( ; \vlist{x}: \vlist{a})$ & $\forall$ & $\{B \subseteq T \mid |B| \geq \aleph_0 \}$ & $-$ \\
%                           %   $B \subseteq T $ & $\exists$ & $\{(\lnot \phi(\vlist{x},y); \vlist{x}: \vlist{a}, y:b)\ |\ b \in B \}$ & $-$\\
%                          %     \hline
%                            $(\qu y. \phi(\vlist{x},y); \vlist{x} \mapsto \vlist{a})$ & $\exists$ & $\{B \subseteq T \mid |B| \geq \aleph_0 \}$ & $-$ \\
%                              $B \subseteq T $ & $\forall$ & $\{(\phi(\vlist{x},y); \vlist{x} \mapsto \vlist{a}, y \mapsto b)\ |\ b \in B \}$ & $-$\\
%                              \hline
%%                              $(\mu P. \phi(P, x); x \mapsto a)$ & $\exists$ & $\{(\phi(P, x);  x: a)\}$ & $1$ \\
%%                             % \hline
%%                              $(\nu P. \phi(P, x); x: a)$ & $\exists$ & $\{(\phi(P, x);  x: a)\}$ & $0$ \\
%%                              %\hline
%%                              $(P(y); y: a)$ & $\exists$ & $\{(\eta P.\phi(P, x); x: a)\}$ & $-$ \\
%%
%%                              \hline
%                            \end{tabular}
%                             \caption{The new rules in the evaluation game for $\mglque$.
%                          }
%                             \label{mufo_game}
%                            \end{table}
%
% By a straightforward adaption of the corresponding proof for $\mgfoe$ in \cite{BerwangerG01},
% we obtain:
%
% \begin{theorem}
% For every model $\model$, and every formula $\mglque$-formula $\phi(x)$ with one free variable, then
% $\model \models \phi(n)$ iff $\exists$ has a winning strategy in $\mc{E}(\varphi(x),\model)@(\varphi(x); x \mapsto n)$, the evaluation game for $\phi(x)$ and $\model$ when evaluating $x$ at the node $n$.\end{theorem}
%

 %%%%%%%%
We now recall a useful property of fixpoint and continuity. Let $\phi(p,x)$ a formula with only $x$ free.
Given a LTS $\model$, for every ordinal $\alpha$, we define by induction the following sets:
%\fcwarning{Why not $\phi^0(\emptyset):= \emptyset$?}
\begin{itemize}
	\itemsep 0 pt
	\item $\phi^0(\emptyset):= \emptyset$,
	%\{ s \in T \mid \model[P \mapsto \emptyset] \models \phi(P, s)\}$,
	\item $\phi^{\alpha+1}(\emptyset):= \{ s \in T \mid \model[p \mapsto \phi^\alpha(\emptyset)] \models \phi(p, s)\}$,
	\item $\phi^{\lambda}(\emptyset):= \bigcup_{\alpha < \lambda} \phi^{\alpha}(\emptyset)$, with $\lambda$ limit.
\end{itemize}
%We state $\phi^{-1}(\emptyset):=\emptyset$.
If $\phi$ is monotone in $p$, it is possible to show that $\phi^{\beta+1}(\emptyset)= \phi^{\beta}(\emptyset)$, for some ordinal $\beta$. Moreover, the set $\phi^{\beta}(\emptyset)$ is the least fixpoint of $F_\phi$ (cf. for instance \cite{ArnoldN01}).



A formula $\phi(p, x)$ is said to be \emph{constructive} in $p$ if its least fixpoint is reached in at most $\omega$ steps, i.e., if for every model $\model$, the least fixpoint of $F_\phi$ equals to $\bigcup_{\alpha < \omega} \phi^{\alpha}(\emptyset)$. From a local perspective, this means that a formula $\phi(p, x)$ constructive in $p$ if for every model $\model$,  every node $s \in T$, whenever $\mu p. \phi(p,x)$ is true at $s$, then $s$ belongs to some finite approximant $\phi^{i+1}(\emptyset)$ of the least fixpoint of $F_\phi$.
The next proposition is easily verified:% by the fact that Scott proved in \cite{Fontaine08} for the modal $\mu$-calculus but that generalizes to $\mglque$ as well, states that continuous formulas are constructive.



%\afwarning{Verify the claim and that Gaelle's argument REALLY goes thorough also here.}
%\yvwarning{Do not attribute to Gaelle, it is obvious that a Scott continuous map reaches fixpoint in $< \omega$ steps}
\begin{proposition}\label{prop:constructivity}
Let $\phi(p,x)$ be a $\mlque$-formula with only $x$ free. If $\phi(p,x)$ is continuous in $p$, then for every LTS $\model$, and every node $s \in T$, there is $i < \omega$ such that
\[\model \models \mu p. \phi(p,s) \text{ iff } s \in \phi^{i+1}(\emptyset).\]
\end{proposition}

From the fact that sets $\phi^{i+1}(\emptyset)$ are essentially defined as finite unfoldings and the previous Proposition~\ref{prop:constructivity}, we obtain the following.\fcwarning{More intuition on this?}

\begin{proposition}\label{prop:cor_constructivity}
Let $\phi(p,x)$ be a $\mlque$-formula with only $x$ free and such that $\phi(p,x)$ is continuous in $p$. Let $\model$ be a LTS, and $s \in T$. Then
$\model \models \mu p. \phi(p,s)$ iff there is a finite set $p^\model \subseteq_\omega T$ such that $s\in p^\model$ and $\model[p\mapsto p^\model] \models \phi(p,t)$  for every $t \in p^\model$.
\end{proposition}
 \begin{proof}
 For the direction from left to right, assume that $\model \models \mu p. \phi(p,s)$. By Proposition~\ref{prop:constructivity}, we know that  there is $i< \omega$ such that $\model[p \mapsto \phi^i(\emptyset)] \models \phi(p, s)$. The set $\phi^i(\emptyset)$ need not to be finite. However,
 using this information, we are going construct a finite tree whose nodes $t$ are labelled by finite sets $X^m_j$, where $m$ is a node of $\model$ and $j \leq i$, satisfying the following condition:
 \begin{enumerate}
\item  if $t$ is the root, then $t$ is labelled by $X_i^s$,
\item  if $t$ is labelled by $X_j^m=\{s_1, \dots, s_\ell\}$ and $j>0$, then $t$ has $\ell$  children and for every $s_i \in X_j^m$ there is an unique child $t'$ of $t$ labelled by $X_{j-1}^{n_i}$ where $m$ is a node,
%\item if $s$ is labelled by $X_j^m$ and $j=-1$, then $X_j^m=\emptyset$,
\item for every node $t$ of the tree, if $t$ is labelled by $X_j^m$, then it holds that $X_j^m \subseteq \phi^{j}(\emptyset)$.
\end{enumerate}
If we verify that $\model[p\mapsto p^\model] \models \phi(p,s)$ holds by taking as $p^\model$ the union of all labels of the nodes of the constructed tree, we can conclude for the proof of this direction.

As starting point of the inductive construction, we start by the empty tree.  Recall that we know that  $\model[p \mapsto \phi^i(\emptyset)] \models \phi(p, s)$. Since $\phi(p,x)$ is continuous in $p$, there is a finite set $X^s_i \subseteq \phi^i(\emptyset)$ such that $\model[p \mapsto X^s_i] \models \phi(p, s)$. We then add a root to our tree and label it by $X^s_i$.
 Assume that at a leaf $s$ of our tree is labelled by $X^m_j$, for some $j < i$. If $X^m_j$ is empty, than we stop, else we proceed as follows. We know that $X^m_j\subseteq \phi^{j}(\emptyset)$. This means that $\model[p \mapsto \phi^{j-1}(\emptyset)] \models \phi(p, r)$, for every $r \in X_j^m$. By continuity, for each such $r$, there is a finite set $X^m_{j-1} \subseteq  \phi^{j-1}(\emptyset)$ such that $\model[p \mapsto X^m_{j-1}(\emptyset)] \models \phi(p, r)$. For each $r \in X^m_j$ we thus add a child to $m$ and label it with $X^r_{j-1}$. By definition of $\phi^{i+1}(\emptyset)$, the tree is finite. Let $X$ be the union of all labels of the constructed tree. $X$ is finite, and by monotonicity of $\phi(p,x)$ we have that for every $m \in X \cup \{s\}$, $\model[p \mapsto X \cup \{s\}] \models \phi(p,m)$.

For the other direction, it's enough to notice that the smallest finite set $p^\model \subseteq T$ such that $\model[p\mapsto p^\model] \models \phi(p,s)$ and $\model[p\mapsto p^\model] \models \phi(p,m)$ for all $m \in p^\model$ is the least fixpoint $F_\varphi$. %of the function that maps any $S \subseteq T$ into $\{t \in T \mid \model[P \mapsto S] \models \phi(P, t) \}$.
%the idea is the following. By assumption there is a finite set $P^\model \subset T$ such that $\model[P\mapsto P^\model] \models \phi(P,n)$ and $\model[P\mapsto P^\model] \models \phi(P,m)$  for every $m \in P^\model$. The winning strategy for \'Eloise  in $\mc{E}(\mu P.\varphi(P,x),\model)@(\mu P.\varphi(P,x); x \mapsto n)$
%is thus define as the composition of all winning strategies in $\mc{E}(\varphi(P,x),\model[P \mapsto P^\model]))@(\varphi(P,x); x \mapsto m)$
% for $m \in P^\model$.
 \end{proof}

%The previous
\noindent Proposition~\ref{prop:cor_constructivity} naturally suggests the following translation $\mgFOETr{-}:\mlque(\prop)\to\wmso(\prop)$,

\begin{itemize}
	\itemsep 0 pt
	\item $\mgFOETr{p(x)}=p(x)$,
	\item $\mgFOETr{R(x,y)}=R(x,y)$
	\item $\mgFOETr{x\foeq y}= (x \foeq y)$
	\item $\mgFOETr{\varphi \land \psi}=\mgFOETr{\varphi} \land \mgFOETr{\psi}$,
	%\item $ST_x(\varphi \lor \psi)=ST_x(\varphi) \lor ST_x(\psi)$,
	\item $\mgFOETr{\lnot \varphi}= \lnot \mgFOETr{\varphi}$,
	\item $\mgFOETr{\exists x. \varphi}= \exists x. \mgFOETr{\varphi}$,
	\item $\mgFOETr{\qu x. \varphi}= \forall p.\exists x. (\lnot p(x) \land \mgFOETr{\varphi})$,
	\item $\mgFOETr{\mu p. \varphi(p,x)}= \exists p ( p(x) \land \forall y ( p(y) \to \mgFOETr{\varphi(p,y) }))$.
\end{itemize}
%Note that in $\mgFOETr(\mu P. \varphi(P,x))$, the predicate $P$ which occurs in $\mgFOETr(\varphi) $ is bounded by the outermost second order existential quantifier.
%
The following theorem %, which is the analogous of Theorem \ref{thm:contransweak} but for $\mlque$,
is then an immediate corollary of Proposition~\ref{prop:cor_constructivity}.

\begin{theorem}\label{thm:guard_wmso}
For every formula $\phi$ in $\clque$, every LTS $\model$, and assignment $\ass$, we have $\model, \ass \models \varphi$ iff $\model, \ass \models \mgFOETr{\varphi}$.
%
% \begin{enumerate}
% \itemsep 0pt
% \item $\model, \ass \models \varphi$,
% \item $\model, \ass \models \mgFOETr{\varphi}$.
% \end{enumerate}
%
\end{theorem}
\begin{proof}
The proof goes by induction on the complexity of $\varphi$, the only critical step being the least fixpoint operator one. But this follows by applying Proposition \ref{prop:cor_constructivity} and the induction hypothesis.
%
%Let therefore consider $\phi$ is of the form $\mu P. \psi(P,x)$. Without loss of generality, that bounded and free predicate variables are distincts.
%We first show that $(1)$ implies $(2)$. Since $\model , \ass \models \varphi$, \'Eloise has a winning strategy $f$ in $\mc{E}(\phi,\model)@(\varphi,s_I, \ass)$.
%Define $P^\model$ to be the set of node $n \in T$ such that there is a (partial) match $\pi'$ that
%%
%\begin{enumerate}
%\itemsep 0pt
%\item is consistent with $f$, and such that
%\item every position of $\pi'$ is of the form $(\gamma,m, \ass')$, with  $P$ active in $\gamma$, and
%\item the last position of $\pi'$ is of the form $(\varphi, n, \ass')$.
%\end{enumerate}
%%
%The first observation is that since $f$ is a winning strategy, all $f$-consistent matches are finite. Moreover for every position of $\pi'$ is of the form $(\psi(P,x),m, \ass')$, we have that $\model[P \mapsto P^{\model}], \ass' \models \psi(P,m)$. We construct inductively a finite tree labelled by pairs $(x, X)$ where $x$ is a node of $\model$ and $X$ is a finite set of nodes of $\model$ as follows. First, because $\model[P \mapsto P^{\model}] , \ass \models \psi(P,x)$, so there is a finite subset $X \subseteq P^\model$ such that $\model[P \mapsto X_1] , \ass \models \psi(P,x)$. Thus we color the root with $(n, X)$. Now, assume we are given a leaf colored by $(y,Y)$. Consider an enumeration $x_1, \dots, x_k$ of $Y$. For every $i \leq k$, we add a child to $(y,Y)$ labelled by $(x_i, X_i)$ where $X_i$ is given by the fact that since $\model[P \mapsto P^{\model.x_i}] , \ass \models \psi(P,x)$, there is a finite set $X_i$ of nodes in $P^{\model.x_i}$
%
%%the only player who picks successor in a partial match $\pi'$ defined as above is \'Eloise. As a consequence of K\"onig's Lemma, $P^\model$ is finite.
%%
%%By using the induction hypothesis, it is easy to check that $\model[x \mapsto s_I, P \mapsto P^\model] \models P(x) \land \forall y ( P(y) \to ST_y(\varphi) )$.
%%
%%For the other direction, the idea is the following. Because $\model[x \mapsto s_I] \models ST_x(\varphi)$,
%%there is a finite set $P^\model$ such that $\model[x \mapsto s_I, P \mapsto P^\model] \models P(x) \land \forall y ( P(y) \to ST_y(\varphi) )$. The winning strategy for \'Eloise  in $\mc{E}(\mu P.\varphi,\model)@(\mu P.\varphi,s_I)$
%%is thus define as the composition of all winning strategies in $\mc{E}(\varphi,\model[P \mapsto P^\model])@(\varphi,s)$ for $s \in P^\model$.
\end{proof}

%\begin{remark}
%Clearly the standard translation from modal logic into $\gfoe$ extend to the modal $\mu$-calculus and $\mglque$.
%\end{remark}

\subsubsection{Translating automata into formulas}
We are now ready to prove the second main result of the paper.

\begin{theorem}\label{thm:wmso_autofor}
There is an effective procedure that given an automaton in $\yvcwAut({\olque})$, returns an equivalent WMSO-formula.
\end{theorem}
\begin{proofsketch}
The argument   is
 essentially a refinement of the standard proof showing that any automaton in $\yvAut(\ofo)$ can be translated into an equivalent $\mu$-formula
$\xi_\aut$ (cf. e.g. \cite{Ven08}).
The idea is the following. We see a $\yvWMSO$-automaton as a system of equations expressed in terms of $\lque$-formulas: each state corresponds to a monadic predicate variable and the parity of a state corresponds to the least and greatest fixpoint that we seek for the associated variable, etc. One then solves this system of equations via the same inductive procedure used to obtain the formula of the modal $\mu$-calculus from the system associated with a  $\yvAut(\ofo)$-automaton (see e.g. \cite{ArnoldN01} for a description of the solution procedure). Because of the (weakness) and (continuity) conditions on the starting $\wmso$-automaton $\aut$, it is thence possible to verify that the resulting fixpoint formula $\xi_\aut$ belongs to $\clque$.
\end{proofsketch}

\begin{remark}
As a corollary of the automata characterization on trees of \wmso, we obtain the equivalence on this class of structures between \wmso and $\clque$. This consequence should be compared to the analogous result obtained by Walukiewicz in~\cite{Walukiewicz96} for FPL (fixpoint extension of $\foe$) and MSO on trees.
\end{remark}

