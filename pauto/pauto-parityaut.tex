%!TEX root = ../00CFVZ_TOCL.tex

We recall the definition of a parity automaton, adapted to our setting.
Recall from Section~\ref{sec:onestep} that formulas of an arbitrary one-step language $\llang$ are interpreted over  one-step models, that is, a tuple $\osmodel = (D,\val: A \to \wp D)$. %Whenever we say `one-step model' in this section we will be referring to \emph{single-sorted} one-step models. 
Recall that the class of all one-step models is denoted by $\umods$ and that we write $\llang^+(A)$ to denote the fragment where every predicate $a\in A$ occurs only positively.
Without loss of generality, from now on we always assume that every bound variable occurring in a sentence is bound by an unique quantifier (generalised or not).




\begin{definition}[Parity Automata]
A \emph{parity automaton} based on the one-step language $\llang$ and 
alphabet $\wp(\props)$ is a tuple $\aut = \tup{A,\tmap,\pmap,a_I}$ such that $A$ is a
finite set of states, $a_I \in A$ is the initial state,
$\tmap: A\times \wp(\props) \to \llang^+(A)$
is the transition map, and $\pmap: A \to \nat$ is the parity map.
The class of such automata will be denoted by $\Aut(\llang)$.

Acceptance of a $\props$-transition
system $\model = \tup{\moddom,R,\tscolors,s_I}$ by $\aut$ is determined by the \emph{acceptance game}
$\agame(\aut,\model)$ of $\aut$ on $\model$. This is the parity game defined
according to the rules of the following table.
%
% \begin{table*}[ht]
%   \centering
\begin{center}
\small
\begin{tabular}{|l|c|l|c|} \hline
Position & Pl'r & Admissible moves & Parity \\
\hline
    $(a,s) \in A \times \moddom$
  & $\eloise$
  & $\{\val : A \to \wp(R[s]) \mid (R[s],\val) \models \tmap (a, \tscolors(s)) \}$
  & $\pmap(a)$ 
\\
%     $(a,s) \in A \times \moddom$
%   & $\abelard$
%   & $\{(a,s,\aact) \in A \times \moddom  \mid \aact \in \acts\}$
%   & $\pmap(a)$ 
% \\
%     $(a,s,\aact) \in A \times \moddom $
%   & $\eloise$
%   & $\{\val : A \to \wp(R_\aact[s]) \mid (R_\aact[s],\val) \models \tmap (a, \tscolors(s), \aact) \}$
%   & $\pmap(a)$ 
% \\
    $\val : A \rightarrow \wp(\moddom)$
  & $\abelard$
  & $\{(b,t) \mid t \in \val(b)\}$
  & $\max(\pmap[A])$
\\ \hline
 \end{tabular}
\end{center}
%
A transition system $\model$ is \emph{accepted} by $\aut$ if $\exists$ has
a winning strategy in $\agame(\aut,\model)@(a_I,s_I)$, and \emph{rejected}
if $(a_I,s_I)$ is a winning position for $\abelard$. We write $\trees(\aut)$ for the class of trees (also called the \emph{tree language}) recognised by $\bbA$.
\end{definition}

The automata-theoretic characterisation of $\wmso$ and $\nmso$ will use classes of parity automata constrained by two additional properties: weakness
and continuity. 

\begin{definition}[Weakness, Continuity]
\label{def:weak}
Let $\llang$ be a one-step language, and let $\bbA = \tup{A,\tmap,\pmap,a_I}$
be in $\Aut(\llang)$. Write $\ord$ for the reachability relation in $\bbA$, i.e. the reflexive-transitive closure of $\{ (a,b) \mid \exists c. b \text{ occurs in }\tmap(a,c)\}$. A \emph{strongly connected $\ord$-component} ($\ord$-SCC) is a subset $M\subseteq A$ such that, for every $a,b \in M$ we have $a \ord b$ and $b \ord c$. The SCC is called \emph{maximal} (MSCC) when $M\cup\{a\}$ ceases to be a SCC for any choice of $a \in A\setminus M$.

%Given the semantics of the one-step language $\llang$, the (semantic) notion of (co-)continuity applies to one-step formulas (see for instance section~\ref{subsec:one-stepcont}). 

We formulate two requirements on automata from $\Aut(\llang)$:
\begin{description}
\item[(weakness)] if $a \ord b$ and $b \ord a$ then $\pmap(a) = \pmap(b)$.
\item[(continuity)] let $a,b$ be states such that both $a\ord b$ and
$b \ord a$, and let $c\in C$;
    if ${\pmap(b)}=1$ then $\tmap(b,c)$ is continuous in $a$.
    If $\pmap(b)=0$, then $\tmap(b,c)$ is co-continuous in $a$.
\end{description}
A \emph{weak parity automaton} is an automaton $\aut \in \Aut(\llang)$ additionally satisfying the \emph{(weakness)}. We call it \emph{continuous-weak} if it additionally satisfies the \emph{(continuity)} condition. We let $\AutW(\llang)$ denote the class of weak and $\AutWC(\llang)$  the class of continuous-weak automata.
\end{definition}

\begin{remark}
Any weak parity automaton $\bbA$ is equivalent to a weak parity automaton
$\bbA'$ with $\pmap: A' \to \{0,1\}$. From now on we such a parity map for weak parity automata.
\end{remark}

Intuitively, weakness forbids an automaton to register non-trivial properties concerning the vertical `dimension' of input trees, whereas continuity expresses a constraint on how much of the horizontal `dimension' of an input tree the automaton is allowed to process. In terms of second-order logic, they correspond respectively to quantification over `vertically' finite (i.e. included in well-founded subtrees) and `horizontally' finite (i.e. included in finitely branching subtrees) sets. The conjunction of weakness and continuity thus corresponds to quantification over finite sets. 

Henceforth we call \emph{$\nmso$-automata} the members of the class $\AutW(\ofoe)$ and \emph{$\wmso$-automata} those of $\AutWC(\olque)$. The names are justified by the logic characterising these classes, see Theorem \ref{t:wmsoauto} below.

\begin{remark}
Observe that, contrarily to weakness, continuity is given above as a semantic condition. Thanks to Theorem XXX \fznote{theorem about syntactic char. for continuity}, we can work with a completely syntactic definition of $\wmso$-automata, the continuity condition above being equivalently replaced by the following. 
\begin{description}
	\itemsep 0 pt
	\item[(continuity, syntactically)] if $\pmap(a)$ is odd (resp. even) then, for each $c\in C$ we have
	   $\tmap(a,c) \in \cont{{\olque}^+}{b}(A)$ (resp. $\tmap(a,c) \in \cocont{{\olque}^+}{b}(A)$).
\end{description} 
\end{remark}

We are now able to state the main theorem of this section, which takes care of
one direction of Theorem~\ref{t:m1}. 

\begin{theorem}
\label{t:wmsoauto}
For $\sllang \in \{ \wmso, \nmso\}$, there is an effective construction transforming an $\sllang$-formula $\phi$
into an $\sllang$-automaton $\bbA_{\phi}$ that is equivalent
to $\phi$ on the class of trees.
That is, for any tree $\bbT$, $\bbA_{\phi}$ accepts $\bbT$ if and only if $\bbT \models {\phi}$
%\begin{enumerate}[(a)]
%\item There is an effective construction transforming a $\wmso$-formula $\phi$
%into a $\wmso$-automaton $\bbA_{\phi}$ that is equivalent
%to $\phi$ on the class of trees.
%That is, for any tree $\bbT$, $\bbA_{\phi}$ accepts $\bbT$ if and only if $\bbT \models {\phi}$
%\item There is an effective construction transforming a $\nmso$-formula $\phi$
%into a $\nmso$-automaton $\bbA_{\phi}$ that is equivalent
%to $\phi$ on the class of trees.
%That is, for any tree $\bbT$, $\bbA_{\phi}$ accepts $\bbT$ if and only if $\bbT \models {\phi}$
%\end{enumerate}
\end{theorem}

As usual, the proof of this theorem proceeds by induction on the complexity of
$\phi$. The inductive step requires that the class of automata is closed under the boolean operations and projection (noetherian for $\nmso$-automata, finite for $\wmso$-automata). Clearly, the latter closure property requires most of the work. The next we first
provide a simulation theorem that put $\wmso$-automata in a suitable shape
for the projection construction.

%%%
