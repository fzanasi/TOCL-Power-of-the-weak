

In what follows, we verify that WMSO-automata capture exactly the expressive power of WMSO on the class of tree models. Since we already proved the direction from formulas into automata (Theorem~\ref{t:wmsoauto}), we just have to verify that there is a sound translation going in the other direction.
%This is done by adapting and combining the proof of Theorem~\ref{t:autofor} with the proof of Theorem~\ref{thm:contransweak} of the previous section. 
For this purpose, we first introduce a fixpoint extension of first-order logic.

\subsection{Fixpoint extension of  first-order logic}
%Let us start in formally defining the loosely guarded fragment of $\lque$ (see \cite{GradelW99}). %We are just interested to the behavior of this logic over trees.
%Recall that our language consists therefore only of a set

Let our first-order signature be composed of a set $\prop$ of monadic predicates (denoted with capital latin letters) and an unique binary predicate $R$.
%
%\begin{definition}
%The loosely guarded fragment $\glque(\prop)$ of $\lque(\prop)$ is the smallest collection of formulas containing all atomic formulas, closed under Boolean connectives and such that:
%\begin{itemize}
%\item if $\vlist{x}=(x_1, \dots, x_m)$ and $\vlist{y}=(y_1, \dots, y_n)$ are variables, $\phi(\vlist{x},\vlist{y})$ is a $\glque(\prop)$-formula whose free variables are among $\{x_1, \dots, x_m,y_1, \dots, y_n\}$, and $\alpha_1(\vlist{x},\vlist{y})$, \dots, $\alpha_k(\vlist{x},\vlist{y})$ are atomic formulas, then the formulas
%\begin{itemize}
%\item $\forall \vlist{y} (\bigwedge_{1\leq \ell \leq k}\alpha_\ell(\vlist{x},\vlist{y}) \to \phi(\vlist{x},\vlist{y}))$,
%\item $\dqu \vlist{y} (\bigwedge_{1\leq \ell \leq k}\alpha_\ell(\vlist{x},\vlist{y}) \to \phi(\vlist{x},\vlist{y}))$,
%\item $\exists \vlist{y} (\bigwedge_{1\leq \ell \leq k}\alpha_\ell(\vlist{x},\vlist{y}) \land \phi(\vlist{x},\vlist{y}))$ and
%\item $\qu \vlist{y} (\bigwedge_{1\leq \ell \leq k}\alpha_\ell(\vlist{x},\vlist{y}) \land \phi(\vlist{x},\vlist{y}))$
%\end{itemize}
%are in $\glque(\prop)$.
%\end{itemize}
%%with $\vlist{x}=(x_1, \dots, x_m)$, $\vlist{y}=(y_1, \dots, y_n)$ to simplify notation.
%\end{definition}
%
%
%
Analogously to the modal $\mu$-calculus, the fixpoint extension of $\lque(\prop)$ is defined by adding a fixpoint construction clause.

\begin{definition}
The fixed point logic $\mlque(\prop)$ is given by:
$$
\varphi ::= q(x) \mid R(x,y) \mid x \foeq y \mid \exists x.\varphi \mid \qu x.\varphi \mid \lnot\varphi \mid \varphi \land \varphi \mid \mu p.\varphi(p,x)
$$
where $p,q\in\prop$, $x,y\in\fovar$; moreover $p$ occurs only positively in $\varphi(p,x)$ and $x$ is the only free variable in $\varphi(p,x)$.
\end{definition}

% \begin{definition}
% The fixed point logic $\mlque(\prop)$ is obtained as follows:
%  \begin{enumerate}
%  \item For all $\varphi \in \lque(\prop)$ we have $\varphi \in \mlque(\prop)$,
%  \item Let $\phi(P, x) \in \mlque(\prop)$ such that $P\in\prop$ occurs only positively and $x$ is the only free variable, then $\mu P. \phi(P, x)$ and $\nu P. \phi(P, x)$ are formulas of $\mlque(\prop)$.
%  \end{enumerate}
% \end{definition}

% The semantics of  the fixpoint formulas $\mu P. \phi(P, x)$ and $\nu P. \phi(P, x)$ is the expected one. Given a model $\model$ and $s \in T$,  $\model \models \mu P. \phi(P, s)$ iff $s$ is in the least fixpoint of the  operator $F_\phi:\wp(T)\to \wp(T)$ defined as $F_\phi(S) := \{t \in T \mid \model[P \mapsto S] \models \phi(P, t) \}$. The semantics of $\nu P. \phi(P, x)$ is dually defined by considering the greatest instead of the least fixpoint of $F_\phi$.

The semantics of the fixpoint formula $\mu p. \phi(p, x)$ is the expected one. Given a model $\model$ and $s \in T$,  $\model \models \mu p. \phi(p, s)$ iff $s$ is in the least fixpoint of the  operator $F_\phi:\wp(T)\to \wp(T)$ defined as $F_\phi(S) := \{t \in T \mid \model[p \mapsto S] \models \phi(p, t) \}$.

Formulas of $\mlque$ may be also classified according to their alternation depth as it happens for the modal $\mu$-calculus.
The alternation-free fragment of $\mlque$ is thence defined as the collection of $\mlque$-formulas $\phi$
without nesting of greatest and least fixpoint operators, i.e. such that, for any two subformulas $\mu p.\psi_1(p,y)$ and $\nu q. \psi_2(q,z)$, predicates $p$ and $q$ do not occur free respectively in $\psi_2(q,z)$ and $\psi_1(p,y)$.

%
%
\begin{definition}
Given $p \in \prop$, we say that $\varphi \in \mlque(\prop)$ is
\begin{itemize}
\item \emph{monotone in the predicate $p$} iff for every LTS $\model$ and assignment $\ass$, \[ \text{if }\model, \ass \models \varphi \text{ and $\tsval(p) \subseteq E$, then }\model[p \mapsto E], g\models \phi\]

\item \emph{continuous in the predicate $p$} iff for every LTS $\model$ and assignment $\ass$ there exists some finite $S \subseteq_\omega \tsval(p)$ such that
$$
\model, \ass \models \varphi \quad\text{iff}\quad \model[p \mapsto S], \ass \models \varphi .
$$
\end{itemize}
\end{definition}

In the next definition, we provide a definition of the continuous fragment of $\mlque$, reminiscent of the one defined in Theorem~\ref{thm:olquecont}.
\begin{definition}
Let $\mathsf{Q}\subseteq \prop$ be a set of monadic predicates. The fragment $\cont{\mlque}{\mathsf{Q}}(\prop)$ is defined by the following rules:
$$
\varphi ::= \psi \mid q(x) \mid \exists x.\varphi(x) \mid \varphi \land \varphi \mid \varphi \lor \varphi \mid \wqu x.(\varphi,\psi) \mid \mu p. \phi'(p, x)
$$
where $q \in \mathsf{Q}$, $\psi \in \mlque(\prop\setminus \mathsf{Q})$, $p \in \prop \setminus \mathsf{Q}$, $\wqu x.(\varphi,\psi) := \forall x.(\varphi(x) \lor \psi(x)) \land \dqu x.\psi(x)$ and $\phi'(p,x)$ is a formula with only $x$ free such that $\phi'(p,x) \in \cont{\mlque}{\mathsf{Q} \cup\{p\}}(\prop)$.

%The $\contAFMC$-fragment of $\mlque$  is  obtained by adding to $\glque$ the following (semantic) rule for constructing fixed point formulas.
% \begin{itemize}
% \item given a monadic predicate letter $P$, a first-order variable $x$, and a formula $\phi(P, x)$  that contains only positive occurrences of $P$ and no free variable other than $x$, if $\phi(P,x)$ is a formula in the fragment that is continuous in $P$ then $\mu P. \phi(P, x)$ is also a formula of  the fragment. Dually for $\nu P. \phi(P, x)$.\fcwarning{`positivity' is syntactic but `continuity' is semantic}
% \end{itemize}
\end{definition}

%We then verify that formulas in $\cont{\mlque}{A}(\prop)$ are (semantical) continuous in $A$. The proof is

\begin{lemma}\label{lem:colqueiscont_mu}
If $\varphi \in \cont{\mlque}{\mathsf{Q}}(\prop)$ then $\varphi$ is continuous in (each predicate from) $\mathsf{Q}$.
\end{lemma}
%
\begin{proof} First, notice that If $\varphi \in \cont{\mlque}{\mathsf{Q}}(\prop)$ then $\varphi$ is monotone  in (each predicate from) $\mathsf{Q}$. %This is proved as for
The proof goes then by induction on the complexity of $\varphi$. For the all the cases except the fixpoint one, the proof is the same as the one for Lemma~\ref{lem:colqueiscont}. For $\phi=\mu p. \phi'(p, x)$, with $\phi'(p,x) \in \cont{\mlque}{\mathsf{Q} \cup\{p\}}(\prop)$, the argument is the same as in~\cite[Lemma 1]{Fontaine08}.
\end{proof}

% \begin{definition}
% The guarded fragment $\glque$ of $\lque$ is the smallest collection of formulas containing all atomic formulas, closed under Boolean connectives and such that:
% \begin{itemize}
% \item if $x$ and $y$ are variables and $\phi(x,y)$ is a $\glque$-formula whose free variables are among $\{x,y\}$, then the formulas 
% \begin{itemize}
% \item $\forall y (r(x,y) \to \phi(x,y))$ and $\dqu y (r(x,y) \to \phi(x,y))$,
% \item $\exists y (r(x,y) \land \phi(x,y))$ and $\qu y (r(x,y) \land \phi(x,y))$
% \end{itemize}
% are in $\glque$.
% \end{itemize}
% \end{definition}

As for the modal $\mu$-calculus, we define the fragment $\clque$ of $\mlque$ as the one where the use of the least fixed point operator is restricted to the continuous fragment. %, that the one obtained by adding to $\lque$ the following rule for constructing fixed point formulas.
 % \begin{itemize}
 % \item given a monadic predicate letter $P$, a first-order variable $x$, and a formula $\phi(P, x)$  that contains only positive occurrences of $P$ and no free variable other than $x$, if $\phi(P,x)$ is a formula in the fragment that belongs to $\cont{\mlque}{\{P\}}(\prop)$, then $\mu P. \phi(P, x)$ is also a formula of  the fragment.
 % \end{itemize}

\begin{definition}
The fragment $\clque(\prop)$ of $\mlque(\prop)$ is given by the following restriction of the fixpoint operator to the contiuous fragment:
{\small%
$$
\varphi ::= q(x) \mid R(x,y) \mid x \foeq y \mid \exists x.\varphi \mid \qu x.\varphi \mid \lnot\varphi \mid \varphi \land \varphi \mid \mu p.\varphi'(p,x)
$$}%
where $p,q\in\prop$, $x,y\in\fovar$; and $\varphi'(p,x) \in \cont{\mlque}{\{p\}}(\prop) \cap \clque(\prop)$ is such that $p$ occurs only positively in $\varphi'$ and $x$ is the only free variable in $\varphi'$.
\end{definition}

%
%
%
%
%The logic $\mglque$ can be given a semantic in terms of evaluation games extending the one given  in \cite{BerwangerG01} for $\mgfoe$ by adding rules for the generalized quantifier.
%We present it just for $\qu$, and treat the rules for $\dqu y. \phi(\vlist{x},y)$ as derived from the equivalent formula $\lnot \qu y. \lnot\phi(\vlist{x},y)$
%the universal being treated as the alternation-free fragment.
%As usual, we assume that any predicate is bounded by at most one fixpoint operator
%%, any if a predicate is bounded, then the fixpoint operator bounding it is unique,
%and that bounded and free predicates are pairwise distinct.%\fzwarning{In the table: why not a clause for $\neg$, $\vee$, $\wedge$? Meaning of $\eta$, ; and :?}
%                             \begin{table}[h]
%                              \centering
%                            \begin{tabular}{|l|c|l|c|}
%                             \hline
%                              % after \\: \hline or \cline{col1-col2} \cline{col3-col4} ...
%                              Position & Player & Admissible moves & Parity\\
%                               \hline % \hline
%                           %  $( ; \vlist{x}: \vlist{a})$ & $\forall$ & $\{B \subseteq T \mid |B| \geq \aleph_0 \}$ & $-$ \\
%                           %   $B \subseteq T $ & $\exists$ & $\{(\lnot \phi(\vlist{x},y); \vlist{x}: \vlist{a}, y:b)\ |\ b \in B \}$ & $-$\\
%                          %     \hline
%                            $(\qu y. \phi(\vlist{x},y); \vlist{x} \mapsto \vlist{a})$ & $\exists$ & $\{B \subseteq T \mid |B| \geq \aleph_0 \}$ & $-$ \\
%                              $B \subseteq T $ & $\forall$ & $\{(\phi(\vlist{x},y); \vlist{x} \mapsto \vlist{a}, y \mapsto b)\ |\ b \in B \}$ & $-$\\
%                              \hline
%%                              $(\mu P. \phi(P, x); x \mapsto a)$ & $\exists$ & $\{(\phi(P, x);  x: a)\}$ & $1$ \\
%%                             % \hline
%%                              $(\nu P. \phi(P, x); x: a)$ & $\exists$ & $\{(\phi(P, x);  x: a)\}$ & $0$ \\
%%                              %\hline
%%                              $(P(y); y: a)$ & $\exists$ & $\{(\eta P.\phi(P, x); x: a)\}$ & $-$ \\
%%
%%                              \hline
%                            \end{tabular}
%                             \caption{The new rules in the evaluation game for $\mglque$.
%                          }
%                             \label{mufo_game}
%                            \end{table}
%
% By a straightforward adaption of the corresponding proof for $\mgfoe$ in \cite{BerwangerG01},
% we obtain:
%
% \begin{theorem}
% For every model $\model$, and every formula $\mglque$-formula $\phi(x)$ with one free variable, then
% $\model \models \phi(n)$ iff $\exists$ has a winning strategy in $\mc{E}(\varphi(x),\model)@(\varphi(x); x \mapsto n)$, the evaluation game for $\phi(x)$ and $\model$ when evaluating $x$ at the node $n$.\end{theorem}
%

 %%%%%%%%
We now recall a useful property of fixpoint and continuity. Let $\phi(p,x)$ a formula with only $x$ free.
Given a LTS $\model$, for every ordinal $\alpha$, we define by induction the following sets:
%\fcwarning{Why not $\phi^0(\emptyset):= \emptyset$?}
\begin{itemize}
	\itemsep 0 pt
	\item $\phi^0(\emptyset):= \emptyset$,
	%\{ s \in T \mid \model[P \mapsto \emptyset] \models \phi(P, s)\}$,
	\item $\phi^{\alpha+1}(\emptyset):= \{ s \in T \mid \model[p \mapsto \phi^\alpha(\emptyset)] \models \phi(p, s)\}$,
	\item $\phi^{\lambda}(\emptyset):= \bigcup_{\alpha < \lambda} \phi^{\alpha}(\emptyset)$, with $\lambda$ limit.
\end{itemize}
%We state $\phi^{-1}(\emptyset):=\emptyset$.
If $\phi$ is monotone in $p$, it is possible to show that $\phi^{\beta+1}(\emptyset)= \phi^{\beta}(\emptyset)$, for some ordinal $\beta$. Moreover, the set $\phi^{\beta}(\emptyset)$ is the least fixpoint of $F_\phi$ (cf. for instance \cite{ArnoldN01}).



A formula $\phi(p, x)$ is said to be \emph{constructive} in $p$ if its least fixpoint is reached in at most $\omega$ steps, i.e., if for every model $\model$, the least fixpoint of $F_\phi$ equals to $\bigcup_{\alpha < \omega} \phi^{\alpha}(\emptyset)$. From a local perspective, this means that a formula $\phi(p, x)$ constructive in $p$ if for every model $\model$,  every node $s \in T$, whenever $\mu p. \phi(p,x)$ is true at $s$, then $s$ belongs to some finite approximant $\phi^{i+1}(\emptyset)$ of the least fixpoint of $F_\phi$.
The next proposition is easily verified:% by the fact that Scott proved in \cite{Fontaine08} for the modal $\mu$-calculus but that generalizes to $\mglque$ as well, states that continuous formulas are constructive.



%\afwarning{Verify the claim and that Gaelle's argument REALLY goes thorough also here.}
%\yvwarning{Do not attribute to Gaelle, it is obvious that a Scott continuous map reaches fixpoint in $< \omega$ steps}
\begin{proposition}\label{prop:constructivity}
Let $\phi(p,x)$ be a $\mlque$-formula with only $x$ free. If $\phi(p,x)$ is continuous in $p$, then for every LTS $\model$, and every node $s \in T$, there is $i < \omega$ such that
\[\model \models \mu p. \phi(p,s) \text{ iff } s \in \phi^{i+1}(\emptyset).\]
\end{proposition}

From the fact that sets $\phi^{i+1}(\emptyset)$ are essentially defined as finite unfoldings and the previous Proposition~\ref{prop:constructivity}, we obtain the following.\fcwarning{More intuition on this?}

\begin{proposition}\label{prop:cor_constructivity}
Let $\phi(p,x)$ be a $\mlque$-formula with only $x$ free and such that $\phi(p,x)$ is continuous in $p$. Let $\model$ be a LTS, and $s \in T$. Then
$\model \models \mu p. \phi(p,s)$ iff there is a finite set $p^\model \subseteq_\omega T$ such that $s\in p^\model$ and $\model[p\mapsto p^\model] \models \phi(p,t)$  for every $t \in p^\model$.
\end{proposition}
 \begin{proof}
 For the direction from left to right, assume that $\model \models \mu p. \phi(p,s)$. By Proposition~\ref{prop:constructivity}, we know that  there is $i< \omega$ such that $\model[p \mapsto \phi^i(\emptyset)] \models \phi(p, s)$. The set $\phi^i(\emptyset)$ need not to be finite. However,
 using this information, we are going construct a finite tree whose nodes $t$ are labelled by finite sets $X^m_j$, where $m$ is a node of $\model$ and $j \leq i$, satisfying the following condition:
 \begin{enumerate}
\item  if $t$ is the root, then $t$ is labelled by $X_i^s$,
\item  if $t$ is labelled by $X_j^m=\{s_1, \dots, s_\ell\}$ and $j>0$, then $t$ has $\ell$  children and for every $s_i \in X_j^m$ there is an unique child $t'$ of $t$ labelled by $X_{j-1}^{n_i}$ where $m$ is a node,
%\item if $s$ is labelled by $X_j^m$ and $j=-1$, then $X_j^m=\emptyset$,
\item for every node $t$ of the tree, if $t$ is labelled by $X_j^m$, then it holds that $X_j^m \subseteq \phi^{j}(\emptyset)$.
\end{enumerate}
If we verify that $\model[p\mapsto p^\model] \models \phi(p,s)$ holds by taking as $p^\model$ the union of all labels of the nodes of the constructed tree, we can conclude for the proof of this direction.

As starting point of the inductive construction, we start by the empty tree.  Recall that we know that  $\model[p \mapsto \phi^i(\emptyset)] \models \phi(p, s)$. Since $\phi(p,x)$ is continuous in $p$, there is a finite set $X^s_i \subseteq \phi^i(\emptyset)$ such that $\model[p \mapsto X^s_i] \models \phi(p, s)$. We then add a root to our tree and label it by $X^s_i$.
 Assume that at a leaf $s$ of our tree is labelled by $X^m_j$, for some $j < i$. If $X^m_j$ is empty, than we stop, else we proceed as follows. We know that $X^m_j\subseteq \phi^{j}(\emptyset)$. This means that $\model[p \mapsto \phi^{j-1}(\emptyset)] \models \phi(p, r)$, for every $r \in X_j^m$. By continuity, for each such $r$, there is a finite set $X^m_{j-1} \subseteq  \phi^{j-1}(\emptyset)$ such that $\model[p \mapsto X^m_{j-1}(\emptyset)] \models \phi(p, r)$. For each $r \in X^m_j$ we thus add a child to $m$ and label it with $X^r_{j-1}$. By definition of $\phi^{i+1}(\emptyset)$, the tree is finite. Let $X$ be the union of all labels of the constructed tree. $X$ is finite, and by monotonicity of $\phi(p,x)$ we have that for every $m \in X \cup \{s\}$, $\model[p \mapsto X \cup \{s\}] \models \phi(p,m)$.

For the other direction, it's enough to notice that the smallest finite set $p^\model \subseteq T$ such that $\model[p\mapsto p^\model] \models \phi(p,s)$ and $\model[p\mapsto p^\model] \models \phi(p,m)$ for all $m \in p^\model$ is the least fixpoint $F_\varphi$. %of the function that maps any $S \subseteq T$ into $\{t \in T \mid \model[P \mapsto S] \models \phi(P, t) \}$.
%the idea is the following. By assumption there is a finite set $P^\model \subset T$ such that $\model[P\mapsto P^\model] \models \phi(P,n)$ and $\model[P\mapsto P^\model] \models \phi(P,m)$  for every $m \in P^\model$. The winning strategy for \'Eloise  in $\mc{E}(\mu P.\varphi(P,x),\model)@(\mu P.\varphi(P,x); x \mapsto n)$
%is thus define as the composition of all winning strategies in $\mc{E}(\varphi(P,x),\model[P \mapsto P^\model]))@(\varphi(P,x); x \mapsto m)$
% for $m \in P^\model$.
 \end{proof}

%The previous
\noindent Proposition~\ref{prop:cor_constructivity} naturally suggests the following translation $\mgFOETr{-}:\mlque(\prop)\to\wmso(\prop)$,

\begin{itemize}
	\itemsep 0 pt
	\item $\mgFOETr{p(x)}=p(x)$,
	\item $\mgFOETr{R(x,y)}=R(x,y)$
	\item $\mgFOETr{x\foeq y}= (x \foeq y)$
	\item $\mgFOETr{\varphi \land \psi}=\mgFOETr{\varphi} \land \mgFOETr{\psi}$,
	%\item $ST_x(\varphi \lor \psi)=ST_x(\varphi) \lor ST_x(\psi)$,
	\item $\mgFOETr{\lnot \varphi}= \lnot \mgFOETr{\varphi}$,
	\item $\mgFOETr{\exists x. \varphi}= \exists x. \mgFOETr{\varphi}$,
	\item $\mgFOETr{\qu x. \varphi}= \forall p.\exists x. (\lnot p(x) \land \mgFOETr{\varphi})$,
	\item $\mgFOETr{\mu p. \varphi(p,x)}= \exists p ( p(x) \land \forall y ( p(y) \to \mgFOETr{\varphi(p,y) }))$.
\end{itemize}
%Note that in $\mgFOETr(\mu P. \varphi(P,x))$, the predicate $P$ which occurs in $\mgFOETr(\varphi) $ is bounded by the outermost second order existential quantifier.
%
The following theorem %, which is the analogous of Theorem \ref{thm:contransweak} but for $\mlque$,
is then an immediate corollary of Proposition~\ref{prop:cor_constructivity}.

\begin{theorem}\label{thm:guard_wmso}
For every formula $\phi$ in $\clque$, every LTS $\model$, and assignment $\ass$, we have $\model, \ass \models \varphi$ iff $\model, \ass \models \mgFOETr{\varphi}$.
%
% \begin{enumerate}
% \itemsep 0pt
% \item $\model, \ass \models \varphi$,
% \item $\model, \ass \models \mgFOETr{\varphi}$.
% \end{enumerate}
%
\end{theorem}
\begin{proof}
The proof goes by induction on the complexity of $\varphi$, the only critical step being the least fixpoint operator one. But this follows by applying Proposition \ref{prop:cor_constructivity} and the induction hypothesis.
%
%Let therefore consider $\phi$ is of the form $\mu P. \psi(P,x)$. Without loss of generality, that bounded and free predicate variables are distincts.
%We first show that $(1)$ implies $(2)$. Since $\model , \ass \models \varphi$, \'Eloise has a winning strategy $f$ in $\mc{E}(\phi,\model)@(\varphi,s_I, \ass)$.
%Define $P^\model$ to be the set of node $n \in T$ such that there is a (partial) match $\pi'$ that
%%
%\begin{enumerate}
%\itemsep 0pt
%\item is consistent with $f$, and such that
%\item every position of $\pi'$ is of the form $(\gamma,m, \ass')$, with  $P$ active in $\gamma$, and
%\item the last position of $\pi'$ is of the form $(\varphi, n, \ass')$.
%\end{enumerate}
%%
%The first observation is that since $f$ is a winning strategy, all $f$-consistent matches are finite. Moreover for every position of $\pi'$ is of the form $(\psi(P,x),m, \ass')$, we have that $\model[P \mapsto P^{\model}], \ass' \models \psi(P,m)$. We construct inductively a finite tree labelled by pairs $(x, X)$ where $x$ is a node of $\model$ and $X$ is a finite set of nodes of $\model$ as follows. First, because $\model[P \mapsto P^{\model}] , \ass \models \psi(P,x)$, so there is a finite subset $X \subseteq P^\model$ such that $\model[P \mapsto X_1] , \ass \models \psi(P,x)$. Thus we color the root with $(n, X)$. Now, assume we are given a leaf colored by $(y,Y)$. Consider an enumeration $x_1, \dots, x_k$ of $Y$. For every $i \leq k$, we add a child to $(y,Y)$ labelled by $(x_i, X_i)$ where $X_i$ is given by the fact that since $\model[P \mapsto P^{\model.x_i}] , \ass \models \psi(P,x)$, there is a finite set $X_i$ of nodes in $P^{\model.x_i}$
%
%%the only player who picks successor in a partial match $\pi'$ defined as above is \'Eloise. As a consequence of K\"onig's Lemma, $P^\model$ is finite.
%%
%%By using the induction hypothesis, it is easy to check that $\model[x \mapsto s_I, P \mapsto P^\model] \models P(x) \land \forall y ( P(y) \to ST_y(\varphi) )$.
%%
%%For the other direction, the idea is the following. Because $\model[x \mapsto s_I] \models ST_x(\varphi)$,
%%there is a finite set $P^\model$ such that $\model[x \mapsto s_I, P \mapsto P^\model] \models P(x) \land \forall y ( P(y) \to ST_y(\varphi) )$. The winning strategy for \'Eloise  in $\mc{E}(\mu P.\varphi,\model)@(\mu P.\varphi,s_I)$
%%is thus define as the composition of all winning strategies in $\mc{E}(\varphi,\model[P \mapsto P^\model])@(\varphi,s)$ for $s \in P^\model$.
\end{proof}

%\begin{remark}
%Clearly the standard translation from modal logic into $\gfoe$ extend to the modal $\mu$-calculus and $\mglque$.
%\end{remark}

\subsection{Translating automata into formulas}
We are now ready to prove the second main result of the paper.

\begin{theorem}\label{thm:wmso_autofor}
There is an effective procedure that given an automaton in $\yvcwAut({\olque})$, returns an equivalent WMSO-formula.
\end{theorem}
\begin{proofsketch}
The argument   is
 essentially a refinement of the standard proof showing that any automaton in $\yvAut(\ofo)$ can be translated into an equivalent $\mu$-formula
$\xi_\aut$ (cf. e.g. \cite{Ven08}).
The idea is the following. We see a $\yvWMSO$-automaton as a system of equations expressed in terms of $\lque$-formulas: each state corresponds to a monadic predicate variable and the parity of a state corresponds to the least and greatest fixpoint that we seek for the associated variable, etc. One then solves this system of equations via the same inductive procedure used to obtain the formula of the modal $\mu$-calculus from the system associated with a  $\yvAut(\ofo)$-automaton (see e.g. \cite{ArnoldN01} for a description of the solution procedure). Because of the (weakness) and (continuity) conditions on the starting $\wmso$-automaton $\aut$, it is thence possible to verify that the resulting fixpoint formula $\xi_\aut$ belongs to $\clque$.
\end{proofsketch}

\begin{remark}
As a corollary of the automata characterization on trees of \wmso, we obtain the equivalence on this class of structures between \wmso and $\clque$. This consequence should be compared to the analogous result obtained by Walukiewicz in~\cite{Walukiewicz96} for FPL (fixpoint extension of $\foe$) and MSO on trees.
\end{remark}

