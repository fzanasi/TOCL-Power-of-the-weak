%!TEX root = ../00CFVZ_TOCL.tex

Given a one step logic $\llang(A)$, we say that a formula $\varphi \in \llang(A)$ is continuous in $\{\vlist{a}\} \subseteq A$ if $\varphi$ is monotone in $\vlist{a}$ and additionally, for every $(D,\val)$ and assignment $\ass:\fovar\to D$,
\[
\text{if } (D,\val),\ass \models \varphi \text{ then } \exists \vlist{U} \subseteq_\omega \val(\vlist{a}) \text{ such that } (D, \val[\vlist{a} \mapsto \vlist{U}]),\ass \models \varphi.
\]

\begin{remark}\label{rem:contprodeach}
	As for monotonicity, it is not difficult to verify that continuity in the product coincides with continuity in every variable. Therefore, in the following proofs we will, in general, consider continuity in every single $a_i$ instead of in the full $\vlist{a}$. This is equivalent, and only done to avoid an even more complex notation.
\end{remark}

In this section we will characterise the continuous fragment of $\ofo$ and $\ofoei$ but we will not characterise that of $\ofoe$, since it is not used in this dissertation.
% It will be useful to give a syntactic characterization of continuity for several one-step logics.

%%
\subsubsection{Continuous fragment of $\ofo$}

\index{fragment!continuous!$\ofo$}
\index{$\ofo$!$\cont{}{A'}$}
\begin{theorem}\label{thm:ofocont}
A formula of $\ofo(A)$ is continuous in $A' \subseteq A$ iff it is equivalent to a sentence given by:
\[
\varphi ::= \psi \mid a(x) \mid \exists x.\varphi \mid \varphi \land \varphi \mid \varphi \lor \varphi
\]
where $a\in A'$ and $\psi \in \ofo(A\setminus A')$. We denote this fragment as $\cont{\ofo}{A'}(A)$.
\end{theorem}
%
The theorem will follow from the next two lemmas and Remark~\ref{rem:contprodeach}.

\begin{lemma}\label{lem:cofoiscont}
Every $\varphi \in \cont{\ofo}{a}(A)$ is continuous in $a$.
\end{lemma}
\begin{proof}
First observe that $\varphi$ is monotone in $a$ by Theorem~\ref{thm:ofomonot}.
We show, by induction, that any one-step formula $\varphi$ in the fragment (which may not be a sentence) satisfies, for every one-step model $(D,\val)$, assignment ${\ass:\fovar\to D}$,
%
\[
\text{if } (D,\val),\ass \models \varphi \text{ then } \exists U \subseteq_\omega \val(a) \text{ such that } (D,\val[a\mapsto U]),\ass \models \varphi.
\]
%
\begin{enumerate}[\textbullet]
\item If ${\varphi = \psi \in \ofo(A\setminus \{a\})}$ changes in the $a$ part of the valuation will make no difference and hence the condition is trivial. 

\item Case $\varphi = a(x)$: if $(D,\val),\ass \models a(x)$ then $\ass(x)\in \val(a)$. Clearly, $\ass(x) \in \val[a\mapsto \{\ass(x)\}](a)$ and hence $(D, \val[a\mapsto \{\ass(x)\}]),\ass \models a(x)$. %For the other direction assume $U \subseteq_\omega \val(a)$ and $(D, \val[a\mapsto U]),\ass \models a(x)$. This means that $\ass(x) \in U \subseteq \val(a)$, hence $(D, \val),\ass \models a(x)$.

\item Case $\varphi = \varphi_1 \lor \varphi_2$: assume $(D,\val),\ass \models \varphi$. Without loss of generality we can assume that $(D,\val),\ass \models \varphi_1$ and hence by induction hypothesis we have that there is $U \subseteq_\omega \val(a)$ such that $(D,\val[a\mapsto U]),\ass \models \varphi_1$ which clearly implies $(D,\val[a\mapsto U]),\ass \models \varphi$. %For the other direction let $U \subseteq_\omega \val(a)$ and assume wlog that $(D,\val[a\mapsto U]),\ass \models \varphi_1$. By induction hypothesis $(D,\val),\ass \models \varphi_1$ which entails $(D,\val),\ass \models \varphi$.

\item Case $\varphi = \varphi_1 \land \varphi_2$: assume $(D,\val),\ass \models \varphi$. By induction hypothesis we have $U_1,U_2 \subseteq_\omega \val(a)$ such that $(D,\val[a\mapsto U_1]),\ass \models \varphi_1$ and $(D,\val[a\mapsto U_2]),\ass \models \varphi_2$. By monotonicity this also holds with $\val[a\mapsto U_1 \cup U_2]$ and therefore $(D,\val[a\mapsto U_1 \cup U_2]),\ass \models \varphi$. %The other direction is very similar to the case of disjunction.

\item Case $\varphi = \exists x.\varphi'(x)$ and $(D,\val),\ass \models \varphi$. By definition there exists $d\in D$ such that $(D,\val),\ass[x\mapsto d] \models \varphi'(x)$. By induction hypothesis there exists $U \subseteq_\omega \val(a)$ such that $(D,\val[a\mapsto U]),\ass[x\mapsto d] \models \varphi'(x)$ and hence $(D,\val[a\mapsto U]),\ass \models \exists x.\varphi'(x)$.
% \item Case $\varphi = \exists x.\varphi'(x)$ and $(D,\val),\ass \models \varphi$. This occurs iff there exists $d\in D$ such that $(D,\val),\ass[x\mapsto d] \models \varphi'(x)$. By induction hypothesis this is equivalent to $\exists U \subseteq_\omega \val(a)$ such that $(D,\val[a\mapsto U]),\ass[x\mapsto d] \models \varphi'(x)$ which holds iff $(D,\val[a\mapsto U]),\ass \models \exists x.\varphi'(x)$.
\end{enumerate}
This finishes the proof.
\end{proof}

\begin{lemma}
There is a translation $(-)^\tcont:\monot{\ofo}{a}(A) \to \cont{\ofo}{a}(A)$ such that
a formula ${\varphi \in \monot{\ofo}{a}(A)}$ is continuous in $a$ if and only if $\varphi\equiv \varphi^\tcont$.
\end{lemma}
\begin{proof}
To define the translation we assume, without loss of generality, that $\varphi$ is in the basic form $\bigvee \mondbnfofo{\Sigma}{a}$.
% where
% %
% \[
% \mondbnfofo{\Sigma}{a} := \bigwedge_{S\in\Sigma} \exists x. \tau^a_S(x) \land \forall x. \bigvee_{S\in\Sigma} \tau^a_S(x).
% \]
%
For the translation, let
$(\bigvee \mondbnfofo{\Sigma}{a})^\tcont := \bigvee \mondgbnfofo{\Sigma}{\Sigma^{-}_{a}}{a}$
where
%
% \[
% \mondgbnfofo{\Sigma}{\Pi}{a} := \bigwedge_{S\in\Sigma} \exists x. \tau^a_S(x) \land \forall x. \bigvee_{S\in\Pi} \tau^a_S(x)
% \]
%
% and
$\Sigma^{-}_{a} := \{S\in \Sigma \mid a\notin S\}$.
%Intuitively, the translation says that we should be able to divide the description given by $\dbnfofo{\Sigma}$ in two parts: (1) $\nabla^{-}_p(\Gamma)$ gives a complete (existential and universal) description with respect to colours which are not $a$ and (2) an existential description where $a$ can only appear as a positive constraint.

\bigskip
From the construction it is clear that $\varphi^\tcont \in \cont{\ofo}{a}(A)$ and therefore the right-to-left direction of the lemma is immediate by Lemma~\ref{lem:cofoiscont}. For the left-to-right direction assume that $\varphi$ is continuous in $a$, we have to prove that $(D,\val) \models \varphi$ iff $(D,\val) \models \varphi^\tcont$, for every one-step model $(D,\val)$. We will take a slightly different but equivalent approach.

It is easy to prove that $(D,\val) \equiv_\fo (D\times \omega,\val_\pi)$ where $D\times\omega$ has countably many copies of each element in $D$ and $\val_\pi(a) := \{(d,k) \mid d\in \val(a), k\in\omega\}$.
%
Moreover, as $\varphi$ is continuous in $a$ there is $U \subseteq_\omega \val_\pi(a)$ such that $\val'_\pi := \val[a \mapsto U]$ satisfies $(D\times\omega,\val_\pi) \models \varphi$ iff $(D\times\omega,\val'_\pi) \models \varphi$.
%
Therefore, it is enough to prove that $(D\times\omega,\val'_\pi) \models \varphi$ iff $(D\times\omega,\val'_\pi) \models \varphi^\tcont$. % Assume again that $\varphi = \bigvee \dbnfofo{\Sigma}$.

\bigskip
\noindent \fbox{$\Rightarrow$}
Let $(D\times\omega,\val'_\pi) \models \mondbnfofo{\Sigma}{a}$, we show that $(D\times\omega,\val'_\pi) \models \mondgbnfofo{\Sigma}{\Sigma^{-}_{a}}{a}$. The existential part of $\mondgbnfofo{\Sigma}{\Sigma^{-}_{a}}{a}$ is trivially true. We have to show that every element of $(D\times\omega,\val'_\pi)$ realises an $a$-positive type in $\Sigma^{-}_{a}$. Take $(d,k) \in D\times\omega$ and let $T$ be the (full) type of $(d,k)$. If $a\notin T$ then trivially $T\in \Sigma^{-}_{a}$ and we are done. Suppose $a\in T$. Observe that in $D\times\omega$ we have infinitely many copies of $d\in D$. However, as $\val'_\pi(a)$ is finite, there must be some $(d,k')$ with type $T' := T\setminus\{a\}$.
%\fcwarning{Write better}
For $\mondbnfofo{\Sigma}{a}$ to be true we must have $T'\in \Sigma$ and hence $T'\in \Sigma^{-}_{a}$. It is easy to see that $(d,k)$ realises the $a$-positive type $T'$.

\bigskip
\noindent \fbox{$\Leftarrow$}
Let $(D\times\omega,\val'_\pi) \models \mondgbnfofo{\Sigma}{\Sigma^{-}_{a}}{a}$, we show that $(D\times\omega,\val'_\pi) \models \mondbnfofo{\Sigma}{a}$. The existential part is trivial. For the universal part just observe that $\Sigma^{-}_{a} \subseteq \Sigma$.
\end{proof}

Putting together the above lemmas we obtain Theorem~\ref{thm:ofocont}. Moreover, a careful analysis of the translation gives us the following corollary, providing normal forms for the continuous fragment of $\ofo$.

\begin{corollary}\label{cor:ofocontinuousnf}
	Let $\varphi \in \ofo(A)$, the following hold:
	\begin{enumerate}[(i)]
		\item The formula $\varphi$ is continuous in $a \in A$ iff it is equivalent to a formula $\bigvee \mondgbnfofo{\Sigma}{\Sigma^{-}_{a}}{a}$ for some types $\Sigma \subseteq \wp A$, where $\Sigma^{-}_{a} := \{S\in \Sigma \mid a\notin S\}$.
		%
		\item If $\varphi$ is monotone in $A$ (i.e., $\varphi\in\ofo^+(A)$) then $\varphi$ is continuous in $a \in A$ iff it is equivalent to a formula in the basic form $\bigvee \posdgbnfofo{\Sigma}{\Sigma^{-}_{a}}$ for some types $\Sigma \subseteq \wp A$, where $\Sigma^{-}_{a} := \{S\in \Sigma \mid a\notin S\}$.
	\end{enumerate}
\end{corollary}

%%
\subsubsection{Continuous fragment of $\ofoei$}

\index{fragment!continuous!$\ofoei$}
\index{$\ofoei$!$\cont{}{A'}$}
\begin{theorem}\label{thm:ofoeicont}
A formula of $\ofoei(A)$ is continuous in $ A' \subseteq A$ iff it is equivalent to a sentence given by:
\[
\varphi ::= \psi \mid a(x) \mid \exists x.\varphi \mid \varphi \land \varphi \mid \varphi \lor \varphi \mid \wqu x.(\varphi,\psi)
\]
where $a\in A'$ and $\psi \in \ofoei(A\setminus A')$. Recall from Definition~\ref{def:contmufoei} that $\wqu x.(\varphi,\psi)$ is defined as $\forall x.(\varphi(x) \lor \psi(x)) \land \dqu x.\psi(x)$. We denote this fragment as $\cont{\ofoei}{A'}(A)$.
\end{theorem}

\noindent The theorem will follow from the next two lemmas and Remark~\ref{rem:contprodeach}.

\begin{lemma}\label{lem:cofoeiiscont}
Every $\varphi \in \cont{\ofoei}{a}(A)$ is continuous in $a$.
\end{lemma}
\begin{proof}
Observe that monotonicity of $\varphi$ is guaranteed by Theorem~\ref{thm:ofoeimonot}.
We show, by induction, that any formula of the fragment (which may not be a sentence) satisfies, for every one-step model $(D,\val)$ and assignment ${\ass:\fovar\to D}$,
%
\[
\text{if } (D,\val),\ass \models \varphi \text{ then } \exists U \subseteq_\omega \val(a) \text{ such that } (D,\val[a\mapsto U]),\ass \models \varphi.
\]
%
We focus on the inductive case of the new quantifier. Let $\varphi' = \wqu x.(\varphi,\psi)$, i.e., %In other words,
%
\[\varphi' = \forall x.\underbrace{(\varphi(x) \lor \psi(x))}_{\alpha(x)} \land \underbrace{\dqu x.\psi(x)}_\beta.\]
%
Let $(D,\val),\ass \models \varphi'$. By induction hypothesis,
for every $\ass_d := \ass[x\mapsto d]$ which satisfies $(D,\val),\ass_d \models \alpha(x)$ there is $U_d \subseteq_\omega \val(a)$ such that $(D,\val[a\mapsto U_d]),\ass_d \models \alpha(x)$. The crucial observation is that because of $\beta$, %part of $\varphi'$ we know that
only finitely many elements of $d$ make $\psi(d)$ false. Let $U := \bigcup \{U_d \mid (D,\val), \ass_d \not\models \psi(x) \}$. Note that $U$ is a finite union of finite sets, hence finite.
%
\begin{claimfirst}
	Let $\val_U := \val[a\mapsto U]$; then we have $(D,\val_U),\ass \models \varphi'$.
\end{claimfirst}
%
\begin{pfclaim}
	It is clear that $(D,\val_U),\ass \models \beta$ because $\psi$ is $a$-free. To show that the first conjunct is true we have to show that $(D,\val_U),\ass_d \models \varphi(x) \lor \psi(x)$ for every $d\in D$. We consider two cases: (i) if $(D,\val),\ass_d \models \psi(x)$ we are done, again because $\psi$ is $a$-free; (ii) if the former is not the case then $U_d \subseteq U$; moreover, we knew that $(D,\val[a\mapsto U_d]),\ass_d \models \alpha(x)$ and by monotonicity of $\alpha(x)$ we can conclude that $(D,\val_U),\ass_d \models \alpha(x)$.
\end{pfclaim}
%
This finishes the proof of the lemma.
\end{proof}



\begin{lemma}\label{lem:ofoeictrans}
	There is a translation $(-)^\tcont:\monot{\ofoei}{a}(A) \to \cont{\ofoei}{a}(A)$ such that
a formula $\varphi \in \monot{\ofoei}{a}(A)$ is continuous in $a$ if and only if $\varphi\equiv \varphi^\tcont$.
\end{lemma}
\begin{proof} We assume that $\varphi$ is in basic normal form, i.e., $\varphi = \bigvee \mondbnfofoei{\vlist{T}}{\Pi}{\Sigma}{a}$.
% where
% \[
% \mondbnfofoei{\vlist{T}}{\Pi}{\Sigma}{a} = \mondbnfoe{\vlist{T}}{\Pi \cup \Sigma}{a} \land
% \bigwedge_{S\in\Sigma} \qu y.\tau^a_S(y) \land \dqu y.\bigvee_{S\in\Sigma} \tau^a_S(y) .
% \]
For the translation let $(\bigvee \mondbnfofoei{\vlist{T}}{\Pi}{\Sigma}{a})^\tcont := \bigvee \mondbnfofoei{\vlist{T}}{\Pi}{\Sigma}{a}^\tcont$ where
\[
\mondbnfofoei{\vlist{T}}{\Pi}{\Sigma}{a}^\tcont :=
\begin{cases}
	\bot &\text{ if } a\in \bigcup \Sigma\\
	\mondbnfofoei{\vlist{T}}{\Pi}{\Sigma}{a} &\text{ otherwise}.
\end{cases}
\]

First we prove the right-to-left direction of the lemma. By Lemma~\ref{lem:cofoeiiscont} it is enough to show that $\varphi^\tcont \in \cont{\ofoei}{a}(A)$. We focus on the disjuncts of $\varphi^\tcont$. The interesting case is when $a\notin \bigcup \Sigma$. If we rearrange $\mondbnfofoei{\vlist{T}}{\Pi}{\Sigma}{a}$ and define the formulas $\varphi', \psi$ as follows:
%
\begin{align*}
\mondbnfofoei{\vlist{T}}{\Pi}{\Sigma}{a} \equiv \exists \vlist{x}.\Big(& \arediff{\vlist{x}} \land \bigwedge_i \tau^a_{T_i}(x_i)\ \land \forall z.(\underbrace{\lnot\arediff{\vlist{x},z} \lor \bigvee_{S\in \Pi} \tau^a_S(z)}_{\varphi'(\vlist{x},z)} \lor \underbrace{\bigvee_{S\in \Sigma} \tau^a_S(z)}_{\psi(z)})\ \land \\
& \dqu y.\underbrace{\bigvee_{S\in\Sigma} \tau^a_S(y)}_{\psi(y)} \Big) \land \bigwedge_{S\in\Sigma} \qu y.\tau^a_S(y),
\end{align*}
%
then we get that
{\small
\[
\mondbnfofoei{\vlist{T}}{\Pi}{\Sigma}{a} \equiv \exists \vlist{x}.\Big(\arediff{\vlist{x}} \land \bigwedge_i \tau^a_{T_i}(x_i) \land \wqu z.(\varphi'(\vlist{x},z),\psi(z)) \Big) \land \bigwedge_{S\in\Sigma} \qu y.\tau^a_S(y)
\]}
%
which, because $a\notin \bigcup \Sigma$, is in the required fragment.

For the left-to-right direction of the lemma we have to prove that $\varphi \equiv \varphi^\tcont$.

\bigskip
\noindent\fbox{$\Leftarrow$} Let $(D,\val) \models \varphi^\tcont$. The only difference between $\varphi$ and $\varphi^\tcont$ is that some disjuncts may have been replaced by $\bot$. Therefore this direction is trivial.

\bigskip
\noindent\fbox{$\Rightarrow$} Let $(D,\val) \models \varphi$. Because $\varphi$ is continuous in $a$ we may assume that $\val(a)$ is finite. Let $\mondbnfofoei{\vlist{T}}{\Pi}{\Sigma}{a}$ be a disjunct of $\varphi$ such that $(D,\val) \models \mondbnfofoei{\vlist{T}}{\Pi}{\Sigma}{a}$. If $a \notin \bigcup\Sigma$ we trivially conclude that $(D,\val) \models \varphi^\tcont$ because the disjunct remains unchanged. Suppose now that $a\in \bigcup\Sigma$, then there must be some $S\in\Sigma$ with $a\in S$. Because $(D,\val) \models \mondbnfofoei{\vlist{T}}{\Pi}{\Sigma}{a}$ we have, in particular, that $(D,\val) \models \qu y.\tau^a_S(x)$ and hence $\val(a)$ must be infinite which is absurd.
\end{proof}

Putting together the above lemmas we obtain Theorem~\ref{thm:ofoeicont}. Moreover, a careful analysis of the translation gives us the following corollary, providing normal forms for the continuous fragment of $\ofoei$. Point \ref{pt:ofoeifunctionalcontinuous} below is tailored for applications to parity automata, see Theorem~\ref{PROP_facts_finConstrwmso}. We call a formula $\varphi  \in \ofoei(A)$ \emph{functionally continuous} in $B \subseteq A$ when, given a model $(D,V),g$ where $\varphi$ is true, a restriction $V'$ of $V$ can be found that both witnesses continuity in each $b \in B$ and also witnesses functionality in each $b \in B$, in the sense of Definition \ref{def:functionalsentenceofoe}.

\begin{corollary}\label{cor:ofoeicontinuousnf}
Let $\varphi \in \ofoei(A)$, the following hold:
	\begin{enumerate}[(i)]
		\item The formula $\varphi$ is continuous in $a \in A$ iff it is equivalent to a formula in the basic form $\bigvee \mondbnfofoei{\vlist{T}}{\Pi}{\Sigma}{a}$ for some types $\Sigma\subseteq\Pi \subseteq \wp A$ and $T_i \subseteq A$ such that $a\notin \bigcup\Sigma$. 	\label{pt:ofoeifcontinuous}	%
				\item The formula $\varphi$ is functionally continuous in $B \subseteq A$ iff it is equivalent to a formula in the basic form $\bigvee \mondbnfofoei{\vlist{T}}{\Psi \cup \Sigma}{\Sigma}{+}$ for some types $\Sigma\subseteq \pw A$, $\Psi \subseteq \pw B$ and $T_i \subseteq B$ such that (1) for all $b \in B$, $b\notin \bigcup\Sigma$ and (2) $T_1,\dots,T_k$ and each $S \in \Psi$ are either empty or singletons. 	\label{pt:ofoeifunctionalcontinuous}	%
		\item If $\varphi$ is monotone in every element of $A$ (i.e., $\varphi\in{\ofoei}^+(A)$) then $\varphi$ is continuous in $a \in A$ iff it is equivalent to a formula in the basic form $\bigvee \posdbnfofoei{\vlist{T}}{\Pi}{\Sigma}$ for some types $\Sigma\subseteq\Pi \subseteq A$ and $T_i \subseteq A$ such that $a\notin \bigcup\Sigma$. \label{pt:ofoeimonotone}
		%
	\end{enumerate}
\end{corollary}
\begin{proof}
	For \ref{pt:ofoeifcontinuous} and \ref{pt:ofoeimonotone}, the observation is that in order to obtain $\Sigma\subseteq\Pi$ in the above normal forms it is enough to use Proposition~\ref{prop:bfofoei-sigmapi} before applying the translation. For \ref{pt:ofoeifunctionalcontinuous}, fix a model $(D,V)$ where $\bigvee \mondbnfofoei{\vlist{T}}{\Psi \cup \Sigma}{\Sigma}{+}$ is true. This means that a certain disjunct $\mondbnfofoei{\vlist{T}}{\Psi \cup \Sigma}{ \Sigma}{+}$, by definition unfolding as
	\begin{equation*}%\label{eq:basicformolque}
	\begin{aligned}
%&\underbrace{\exists \vlist{x}.\Big( \arediff{\vlist{x}} \land \bigwedge_i \tau^+_{T_i}(x_i)\ \land 
% \forall z.(\lnot\arediff{\vlist{x},z} \lor \bigvee_{S\in \Psi \cup \Sigma} \tau^+_S(z) \lor \bigvee_{S\in \Sigma} \tau^+_S(z)}_{\mondbnfofoe{\vlist{T}}{\Psi \cup \Sigma}{+}}  \\
%& \land \underbrace{\dqu y.\bigvee_{S\in\Sigma} \tau^+_S(y) \Big) \land \bigwedge_{S\in\Sigma} \qu y.\tau^+_S(y)}_{\psi},
&\exists \vlist{x}.\Big( \arediff{\vlist{x}} \land \bigwedge_i \tau^+_{T_i}(x_i)\ \land 
 \forall z.(\lnot\arediff{\vlist{x},z} \lor \!\!\!\bigvee_{S\in \Psi \cup \Sigma}\!\!\! \tau^+_S(z) 
 \land \dqu y. \!\bigvee_{S\in\Sigma}\! \tau^+_S(y) \Big) \land \bigwedge_{S\in\Sigma} \qu y.\tau^+_S(y),
\end{aligned}
\end{equation*}
is true. Because by assumption $B \cap \Sigma = \emptyset$, point \ref{pt:ofoeifcontinuous} yields a restriction $V'$ of $V$ such that $(D,V') \models \mondbnfofoei{\vlist{T}}{\Psi \cup \Sigma}{ \Sigma}{+}$ and $V'$ witnesses continuity of $\varphi$ in $B$. %For functionality, observe that $\mondbnfofoei{\vlist{T}}{\Pi}{\Psi \cup \Sigma}{+}$ is the conjunction of sub-formulas $\mondbnfofoe{\vlist{T}}{\Psi \cup \Sigma}{+}$ (first line of \eqref{eq:basicformolque}) and $\psi$ (second line). 

For functionality, the syntactic shape of $\mondbnfofoei{\vlist{T}}{\Psi \cup \Sigma}{ \Sigma}{+}$ implies that $(D,V')$ can be partitioned in three sets $D_1$, $D_2$ and $D_3$ as follows: $D_1$ contains elements $s_1, \dots, s_k$ witnessing types $\tau^+_{T_1},\dots, \tau^+_{T_k},$ respectively; among the remaining elements, there are infinitely many witnessing $\tau^+_S$ for each $S\in \Sigma$ (these form $D_2$), and finitely many not witnessing any such $\tau^+_S$ but witnessing $\tau^+_R$ for some $R \in \Psi$ (these form $D_3$). If we now prune any other assignment of predicates to elements of $D$ but for the types they witness according to this description, we obtain a restriction $V''$ of $V'$ which still makes $\mondbnfofoei{\vlist{T}}{\Psi \cup \Sigma}{ \Sigma}{+}$ true. Moreover because $T_1,\dots,T_k$ and each $S \in \Psi$ are empty or singleton subsets of $B$ and $\Sigma \cap B = \emptyset$, it shows that $\mondbnfofoei{\vlist{T}}{\Psi \cup \Sigma}{ \Sigma}{+}$ is functional in $B$, and still witnesses its continuity because $V'$ does and $V''$ is a restriction of $V'$.  %Then, because $T_1,\dots,T_k,\Psi$ are empty or singletons, Proposition \ref{lemma:functionalsentenceofoe} yields a restriction $V''$ of $V'$ such that $(D,V'') \models \mondbnfofoe{\vlist{T}}{\Psi \cup \Sigma}{+}$ and $V''$ witnesses functionality in $B$ of $\mondbnfofoe{\vlist{T}}{\Psi \cup \Sigma}{+}$. In order for such $V''$ to witness continuity in $B$ and make $\mondbnfofoei{\vlist{T}}{\Psi \cup \Sigma}{\Sigma}{+}$ true, it must preserve the condition imposed by $\psi$, i.e. that infinitely many nodes are marked with each $S \in \Sigma$ and only finitely many are not marked with any $S \in \Sigma$. It can be checked by inspection of $\mondbnfofoe{\vlist{T}}{\Pi}{\Psi \cup \Sigma}{+}$ that such a $V''$ is definable by pruning the valuation $V'$ only for finitely many elements of $D$. Therefore, $\psi$ remains true, $(D,V'')\models \mondbnfofoei{\vlist{T}}{\Psi \cup \Sigma}{ \Sigma}{+}$ and thus also $(D,V'') \models \varphi$ as required. 
\end{proof}
