In this subsection we
%recall the notions of co-continuity and complete multiplicativity, which are dual to continuity and complete additivity, respectively. Next, we
give syntactic characterizations of the co-continuous  fragment of several one-step logics. This notion is dual to continuity.
%
% \bigskip
% Recall from Definition~\ref{def:os-continuity} that a formula $\varphi\in\llang(A)$ is \emph{continuous in $a\in A$} if $\varphi$ is monotone in $a$ and additionally, for every $(D,\val)$ and assignment $\ass:\fovar\to D$,
% \[
% \text{if } (D,\val),\ass \models \varphi \text{ then } \exists U \subseteq_\omega \val(a) \text{ such that } (D, \val[a \mapsto U]),\ass \models \varphi.
% \]
% %
% We say that $\varphi$ is \emph{co-continuous in $a\in A$} if the Boolean dual $\dual{\varphi}$ of $\varphi$ (\emph{cf.}~Definition~\ref{d:bdual1}) is continuous in $a\in A$.
%
%To define syntactic fragments for the dual notions
We first give a concrete definition of the dualisation operator of Definition~\ref{d:bdual1}.% and then show that the one-step language $\ofoei$ is closed under Boolean duals.

\index{dual!$\ofoei$}
\begin{definition}\label{def:concreteduals} 
The \emph{dual} $\varphi^{\delta} \in {\ofoei}(A)$ of $\varphi\in {\ofoei}(A)$ is given by:
\begin{align*}
 (a(x))^{\delta} & :=  a(x) 
 & (\lnot a(x))^{\delta} & :=  \lnot a(x) 
\\ (\top)^{\delta} & :=  \bot 
  & (\bot)^{\delta} & :=  \top 
\\  (x \approx y)^{\delta} & :=  x \not\approx y 
  & (x \not\approx y)^{\delta}& :=  x \approx y 
\\ (\varphi \wedge \psi)^{\delta} &:=  \varphi^{\delta} \vee \psi^{\delta} 
  &(\varphi \vee \psi)^{\delta}& :=  \varphi^{\delta} \wedge \psi^{\delta}
\\ (\exists x.\psi)^{\delta} &:=  \forall x.\psi^{\delta} 
  &(\forall x.\psi)^{\delta} &:=  \exists x.\psi^{\delta} 
\\ (\exists^{\infty} x.\psi)^{\delta} &:= \forall^{\infty} x.\psi^{\delta} 
  &(\forall^{\infty} x.\psi)^{\delta} &:=  \exists^{\infty} x.\psi^{\delta}
\end{align*}
\end{definition}

\begin{remark}
	Observe that if $\varphi \in \llang(A)$ for $\llang\in\{\ofo,\ofoe,\ofoei\}$ then $\varphi^{\delta} \in \llang(A)$. Moreover, the operator preserves positivity of the predicates, that is, if $\varphi \in \llang^+(A)$ then $\varphi^{\delta} \in \llang^+(A)$.
\end{remark}

\noindent The proof of the following proposition is a routine check.

\begin{proposition}\label{props:duals}
For every $\varphi \in \ofoei(A)$, $\varphi$ and $\varphi^{\delta}$ are Boolean duals.
\end{proposition}

We are now ready to give the syntactic definition of the dual fragments for the one-step logics into consideration.

\begin{definition}\label{def:cocontfrag}\label{def:multfrag}
The fragment $\cocont{\ofoei}{A'}(A)$ is given by the sentences generated by:
\[
\varphi ::= \psi \mid a(x) \mid \forall x.\varphi \mid \dqu x.\varphi \mid \varphi \lor \varphi \mid \varphi \land \varphi
\]
where $a\in A'$ and $\psi \in \ofoei(A\setminus A')$. Observe that the equality is included in $\psi$. The fragment $\cocont{\ofo}{A'}(A)$ is defined as $\cocont{\ofoei}{A'}(A)$ but without the clause for $\dqu$ and with $\psi\in \ofo(A\setminus A')$.
\end{definition}

The following proposition states that the above fragments are actually the duals of the fragments defined earlier in this chapter.

\begin{proposition}\label{prop:newfragsduals}
The following hold:
	%
	\begin{align*}
		\cocont{\ofoei}{A'}(A) &= \{\varphi \mid \varphi^\delta \in \cont{\ofoei}{A'}(A)\} \\
		\cocont{\ofo}{A'}(A) &= \{\varphi \mid \varphi^\delta \in \cont{\ofo}{A'}(A)\}.
	\end{align*}
	%
\end{proposition}
\begin{proof}
	Easily proved by induction.
\end{proof}

\noindent As a corollary, we get a characterisation for co-continuity.

\begin{corollary}~ \label{cor:cocontinuity}
	Let $\llang \in \{\ofo,\ofoei\}$. A formula $\varphi \in \llang(A)$ is co-continuous in $a\in A$ if and only if it is equivalent to some $\varphi' \in \cocont{\llang}{a}(A)$.
\end{corollary}
\begin{proof}
	Consequence of Proposition~\ref{prop:newfragsduals} and~\ref{props:duals}.
\end{proof}
