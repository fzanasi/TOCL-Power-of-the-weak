
\begin{definition}
Given a finite set $A$
we define a \emph{one-step model} to be a tuple $\osmodel = (D,\val)$
consisting of a set $D$,  which we call the domain of $\osmodel$, 
and a valuation $\val: A \to \wp D$.

Depending on context, elements of $A$ will be called \emph{monadic predicates}, \emph{names}
or \emph{propositional variables}. The class of all one-step models will be denoted by $\umods$.


A \emph{one-step language} is a map $\llang$ assigning to each finite set $A$, a set $\llang(A)$ of objects called \emph{one-step formulas} over $A$.
We require that $\llang(\bigcap_{i} A_{i}) = \bigcap_{i} \llang(A_{i})$,
so that for each $\varphi \in \llang(A)$ there is a smallest $A_{\varphi} \subseteq A$ such
that $\varphi \in \llang(A_{\varphi})$; this $A_{\varphi}$ is the set of names that \emph{occur} in $\varphi$.

We assume that one-step languages come with a \emph{truth}
relation: given a one-step model $\osmodel$, a formula $\varphi \in \llang$
is either \emph{true} or \emph{false} in $\osmodel$, denoted by,
respectively, $\osmodel \models \varphi$ and $\osmodel \not\models \varphi$.
%

We also assume that $\llang$ has a \emph{positive fragment} $\llang^+$
characterising monotonicity. We say that a formula $\varphi \in \llang(A)$ is
monotone in $a\in A$ iff $(\osmoddom,\val) \models \varphi$ implies $(\osmoddom,\val[a\mapsto E]) \models \varphi$ whenever $\val(a) \subseteq E$. Hence, we require that $\varphi \in \llang(A)$ is
monotone in all $a\in A$ iff it is equivalent to a formula $\varphi' \in \llang^+(A)$.
\end{definition}


Observe that every valuation $\val: A \to \wp (D)$ can equivalently be seen as a marking (or coloring) $\val^\natural:D \to \wp(A)$ given by $\val^\natural(d) := \{a \in A \mid d \in \val(a)\}$ and as a relation $Z_\val := \{ (a,d) \mid d\in \val(a)\}$.
We will use these perspectives interchangeably.


\begin{definition}
The set $\ofoe(A)$ of one-step first-order sentences (with equality) is given by the sentences formed by
\[
\varphi ::=
\top \mid \bot 
\mid a(x)
%\mid \neg a(x)
\mid x \foeq y
%\mid \neg x \foeq y
\mid \neg \varphi
\mid \varphi \lor \varphi
%\mid \varphi \land \varphi
\mid \exists x.\varphi
%\mid \forall x.\varphi
\]
where $x,y\in \fovar$, $a \in A$. The one-step logic $\ofo(A)$ is as $\ofoe(A)$ but without equality. 
The set $\ofoei(A)$ of one-step first-order sentences with generalized quantifier $\qu$ (with equality)
%and the set $\olqu(A)$ of one-step first-order senteces with generalized quantifier $\qu$ (without equality)
is defined analogously by just adding the clauses for the generalised quantifiers $\qu x. \varphi$ and $\dqu x. \varphi$.
\end{definition}

\begin{remark}
	The elements $\top$ and $\bot$ are added for technical reasons. Even though they are already definable in $\ofoe(A)$, this will not necessarily be the case in other fragments that will be defined later.
\end{remark}


\begin{definition}
	Let $\varphi \in \ofoei(A)$ be a formula, $\osmodel = (\osmoddom,\val)$ be a one-step model and $\ass:\fovar\to \wp(D)$ be an assignment. The semantics of $\ofoei(A)$ is given as follows:
	%
	\begin{align*}
	    \osmodel,\ass \models a(x) & \quad\text{iff}\quad \ass(x) \in \val(a),\\
	    \osmodel,\ass \models x \foeq y & \quad\text{iff}\quad \ass(x) = \ass(y),\\
	    %
	    \osmodel,\ass \models \exists x.\varphi & \quad\text{iff}\quad \osmodel,\ass[x\mapsto d] \models \varphi \text{ for some $d\in D$},\\
	    %
	    \osmodel,\ass \models \qu x.\varphi & \quad\text{iff}\quad \osmodel,\ass[x\mapsto d] \models \varphi \text{ for infinitely many distinct $d\in D$},
	\end{align*}
	%
	while the Boolean connectives are defined as expected.
\end{definition}

Recall that $\dqu x.\varphi$ expresses that there are \emph{at most finitely many} elements \emph{falsifying} the formula $\varphi$.

\index{rank, quantifier}
% \index{$qr$}
\begin{definition}
The quantifier rank $qr(\varphi)$ of a formula $\varphi \in \ofoei$ (hence also for $\ofo$ and $\ofoe$) is defined as follows
%
\begin{itemize}
	\itemsep 0 pt
	\item If $\varphi$ is atomic then $qr(\varphi) = 0$,
	\item If $\varphi = \lnot\psi$ then $qr(\varphi) = qr(\psi)$,
	\item If $\varphi = \psi_1 \land \psi_2$ or $\varphi = \psi_1 \lor \psi_2$ then $qr(\varphi) = \max\{qr(\psi_1),qr(\psi_2)\}$,
	\item If $\varphi = Qx.\psi$ for $Q \in \{\exists,\forall,\qu,\dqu\}$ then $qr(\varphi) = 1+qr(\psi)$.
\end{itemize}
%
% \index{$\equiv_k^{\llang}$}
Given a one-step logic $\llang$ we write $\osmodel \equiv_k^{\llang} \osmodel'$ to indicate that the one-step models $\osmodel$ and $\osmodel'$ satisfy exactly the same formulas $\varphi \in \llang$ with $qr(\varphi) \leq k$. The logic $\llang$ will be omitted when it is clear from context.
\end{definition}

% The following observation will allow us to
% work with the (single-sorted) language $\ofoe$ instead of the (two-sorted) language $OWMSO$.\fcwarning{owmso?}

% \begin{fact}[\cite{vaananen77}]
% $OWMSO(A) \equiv \ofoei(A)$.
% \end{fact}

% In the following subsections we provide a detailed model theoretic analysis of the one-step logics that we use in this article, specifically, we give
% %
% \begin{itemize}
% 	\itemsep 0 pt
% 	\item Normal forms for arbitrary formulas of $\ofo$, $\ofoe$ and $\ofoei$.
% 	%
% 	\item Strong forms of syntactic characterizations for the monotone and continuous fragments of several of the mentioned logics. Namely, for $\llang \in \{\ofo,\ofoe,\ofoei\}$ we provide
% 		\begin{enumerate}[(a)]
% 			%\itemsep 0 pt
% 			\item A fragment $\monot{\llang}{a}$ and a translation $(-)^\tmono:\llang(A)\to\monot{\llang}{a}(A)$ such that for every $\varphi \in\llang$ we have $\varphi\equiv\varphi^\tmono$ iff $\varphi$ is monotone in $a \in A$,
% 			%
% 		\end{enumerate}
% 		%
% 		for $\llang \in \{\ofo,\ofoei\}$ we provide
% 		%
% 		\begin{enumerate}[(a)]
% 			\item[(b)] A fragment $\cont{\llang}{a}$ and a translation $(-)^\tcont:\llang(A)\to\cont{\llang}{a}(A)$ such that for every $\varphi \in\llang$ we have $\varphi\equiv\varphi^\tcont$ iff $\varphi$ is continuous in $a \in A$.\fcwarning{The real cont-translation is from $\monot{\llang}{a}(A)$}
% 		\end{enumerate}
% 		%
% 		Moreover, we show that the latter translation also restricts to the fragment $\llang^+_1$, i.e.,
% 		%
% 		\begin{enumerate}[(a)]
% 			\item[(c)] The restriction $(-)^\tcont_{+}:\llang^+_1(A)\to\cont{\llang^+_1}{a}(A)$ of $(-)^\tcont$ is such that for every $\varphi \in\llang^+_1$ we have $\varphi\equiv\varphi^\tcont_+$ iff $\varphi$ is continuous in $a \in A$.
% 		\end{enumerate}
% 	%
% 	\item Syntactic characterizations of the co-continuous fragments of $\ofo$ and $\ofoei$.
% 	%
% 	\item Normal forms for the monotone and continuous fragments.
% \end{itemize}

\index{isomorphism, partial}
% \index{$f: [d_1,\dots,d_k] \mapsto [d'_1,\dots,d'_k]$}
\index{$f:\vlist{d} \mapsto \vlist{d'}$}
A \emph{partial isomorphism} between two one-step models %$\osmodel = (D,\val)$ and $\osmodel' = (D',\val')$
$(D,\val)$ and $(D',\val')$ is a \emph{partial} function $f: D \to D'$ which is injective and satisfies that $d \in \val(a) \Leftrightarrow f(d) \in \val'(a)$ for all $a\in A$ and $ d\in \Dom(f)$.

Given two sequences $\vlist{d} \in D^k$ and $\vlist{d'} \in {D'}^k$ 
we use $f:\vlist{d} \mapsto \vlist{d'}$ to denote the partial function $f:D\pto D'$ defined as $f(d_i) := d'_i$. We explicitly avoid cases where there exist $d_i,d_j$ such that $d_i = d_j$ but $d'_i \neq d'_j$.

% Given two sequences $[d_1,\dots,d_k] \in D^k$ and $[d'_1,\dots,d'_k] \in {D'}^k$ we use $f: [d_1,\dots,d_k] \mapsto [d'_1,\dots,d'_k]$ to denote the partial function $f:D\to D'$ defined as $f(d_i) := d'_i$. We explicitly avoid situations where there exist $d_i,d_j$ such that $d_i = d_j$ but $d'_i \neq d'_j$.
