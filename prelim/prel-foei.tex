

In this subsection we introduce an extension of first-order logic with so called generalised quantifiers.
% Our interest in this extension stems from the fact that it will allow us to define a variant of first-order logic that is expressively equivalent to weak monadic second order (see Section~\ref{sec:onestep}) and has nice technical features such as a normal form theorem.
%
Mostowski~\cite{Mostowski1957} defined unary generalised quantifiers as follows: a unary generalised quantifier ${\mathcal Q}$ is a collection of pairs $(J, X)$ with $X \subseteq J$, and satisfying the following condition
%
\[
\text{If } \big( (J,X)\in {\mathcal Q}, \ |X|=|Y| \ \land \ | J \setminus X|=|K \setminus Y|\big) \text{ then } (K,Y)\in {\mathcal Q}.
\]

\noindent The semantics of ${\mathcal Q}$ is then defined by the following condition
\[
\npmodel,\ass \models {\mathcal Q}x. \phi(x) \quad\text{iff}\quad (\npmoddom,\{s\in\npmoddom \mid \npmodel,\ass[x\mapsto s] \models \phi(x) \}) \in {\mathcal Q},
\]
%
for every model $\npmodel$ and assignment $\ass$.

In this work we will only focus on the generalised quantifier $\qu$ expressing that there exist infinitely many elements satisfying a certain condition. Formally, it is defined as:
%
\[ \qu := \{(J,X) \mid |X| \geq \aleph_0\}.\]
%
The dual of $\qu$ is $\dqu =\{(J,X) \mid |J\setminus X| < \aleph_0\}$. It is worth observing what is the intended meaning of this quantifier: $\dqu x.\varphi$ expresses that there are \emph{at most finitely many} elements \emph{falsifying} the formula $\varphi$.


\begin{definition}
The extension of first-order logic with equality ($\foe$) 
obtained by adding $\qu$ to the corresponding first-order language is denoted $\foei$.
\end{definition}
