

In this subsection we give an extension of $\foe$ with a unary fixed point operator. This extension is known in the literature as FO(LFP$^1$) but we will call it $\mufoe$ %. In this dissertation we focus on \unfolfp, which we denote as $\mufoe$
to keep a consistent notation for fixpoint extensions (e.g., we use $\mu\llang$ for a base logic $\llang$) and fragments thereof (e.g.~$\mu_X\llang$ where $X$ is some restriction on the fixpoint).
%
% \shtbs{PTIME by Immerman, import more things from the PDLJW paper}
% Arity hierarchy of LFP is strict~\cite{grohefphierarchy}.

%As usual with (extensions of) first-order logic, $\mufoe$ will be interpreted over models with an assignment. See Section~\ref{sec:wcl} ($2\wcl$ \textsl{vs.} $\wcl$) for a discussion on how languages with individual variables fit in our setting.
Because of the presence of individual variables, the syntax and semantics of the fixpoint operator is considerably more involved than for the modal $\mu$-calculus.


\begin{definition}
The language of \emph{first-order logic with equality and unary fixpoints} ($\mufoe$) on a set of predicates $\props$, actions $\acts$ and individual variables $\fovar$ is given by:\[
\varphi ::= q(x) \mid R_\aact(x,y) \mid x \foeq y \mid \exists x.\varphi \mid \lnot\varphi \mid \varphi \lor \varphi \mid [\lfp_{p{:}x}.\varphi(p,x)](z)
\]
where $p,q\in\props$, $\aact\in\acts$ and $x,y\in\fovar$.
%The formula $\varphi(p,x)$ should also satisfy that $p$ occurs only positively and if $p(y)$ is a free occurrence then $x=y$.
Observe that $z$ is free in the fixpoint clause and the fixpoint operator binds the designated variables $x$ and $p$.
\end{definition}

The free variables $\FV(\varphi)$ of a formula $\varphi\in\mufoe$ are obtained by extending the standard definition of $\FV$ for $\foe$ with the clause 
\[\FV([\lfp_{p{:}x}.\varphi(p,x)](z)) := (\FV(\varphi) \setminus \{x\}) \cup \{z\}.\]


The semantics of the fixpoint formula $[\lfp_{p{:}x}.\varphi(p,x)](z)$ is the expected one (as introduced in~\cite{Chandra1982,Moschovakis2008,MoschovakisOrig}). First we give a slightly more general definition than we need right now (which will be useful later). For every model $\npmodel$, assignment $\ass$, and predicates (propositions) $\qprops$, the map $\ffunc{\varphi}{\qprops{:}x}:\wp(\npmoddom)\to \wp(\npmoddom)$ is given as:
\[
% \ffunc{\varphi}{p{:}x}(Y) := \{t \in \npmoddom \mid \npmodel[p \mapsto Y],\ass[x\mapsto t] \models \varphi(p, x) \}.
\ffunc{\varphi}{\qprops{:}x}(\vlist{Y}) := \{t \in \npmoddom \mid \npmodel[\qprops \mapsto \vlist{Y}],\ass[x\mapsto t] \models \varphi(\qprops, x) \}.
\]
%
The formula $\npmodel,\ass \models [\lfp_{p{:}x}.\varphi(p,x)](z)$ is then defined to hold iff $\ass(z) \in \lfp(\ffunc{\varphi}{p{:}x})$. That is, if $\ass(z)$ is in the least fixpoint of the map $\ffunc{\varphi}{p{:}x}$.

\begin{remark}\label{rem:parameters}
	
	Suppose that $\varphi \in \mufoe$ has free variables $FV(\varphi) = \{x,\vlist{y}\}$. If we consider the fixpoint formula $\psi := [\lfp_{p{:}x}.\varphi(p,x)](z)$ then $\psi$ would have as free variables $FV(\psi) = \{z,\vlist{y}\}$. The free variables of $\varphi$ which are not bound by the fixpoint (in this case~$\vlist{y}$) are called the \emph{parameters} of the fixpoint.

	Parameters can always be avoided at the expense of increasing the arity of the fixpoint~\cite[p.~184]{Libkin2004}. That is, for example, taking the fixpoint over a relation $P(x_1,\dots,x_n)$ instead of just a predicate $p$. However, in this dissertation we will only consider fixpoints over unary predicates, and therefore we will allow the use of parameters unless explicity stated.
\end{remark}

The language of $\mufoe$ can also be further extended with a \emph{greatest} fixpoint operator $[\gfp_{p{:}x}.\varphi(p,x)](z)$ whose semantics are given by $\npmodel,\ass \models [\gfp_{p{:}x}.\varphi(p,x)](z)$ iff $\ass(z) \in \gfp(\ffunc{\varphi}{p{:}x})$. However, this extension does not add expressive power, since it is possible to prove that $[\gfp_{p{:}x}.\varphi(p,x)](z) \equiv \lnot[\lfp_{p{:}x}.\lnot\varphi(\lnot p,x)](z)$.
