% !TEX root = main.tex

% Given the diversity of the objects considered in this article, we have strived to make the notation as consistent and informative as possible.
The following table works as a summary of the most used notation in this dissertation. It should be taken as a set of general rules from which we try to divert as little as possible.

% \begin{figure}[H]
% \centering
\begin{center}
\begin{tabular}{|l|l|}
	\hline
	\textbf{Concept} & \textbf{Notation}\\
	\hline
	\hline
	Transition system (pointed model) & $\model = \tup{\moddom,R_{\aact\in\acts},\tscolors,s_I}$\\
	Tree (pointed tree) & $\tmodel = \tup{\tmoddom,R_{\aact\in\acts},\tscolors,s_I}$\\
	Model & $\npmodel = \tup{\npmoddom,R_{\aact\in\acts},\tscolors}$\\
	One-step model & $\osmodel = (D,\val:A\to\wp(D))$\\
	Automaton & $\aut,\baut,\dots$\\
	\hline
	\hline
	Formula & $\varphi, \psi, \alpha, \beta, \xi, \chi,\dots$ $\Phi,\Psi,\dots$\\
	Set & $A, B, C, D,\dots$ $X, Y, Z, W,\dots$\\
	Sequence of objects & $\vlist{x}, \vlist{y}, \dots$ $\vlist{a}, \vlist{b}, \dots$ $\vlist{X}, \vlist{Y}, \dots$\\
	Propositional variable & $p, q, r, \dots$\\
	Individual (first-order) variable & $x, y, z, w, \dots$\\
	Second-order (set) variable & $X,Y,Z,W,\dots$ $p,q,r,\dots$\\
	\hline
	\hline
	Assignment (of individual variables) & $\ass:\fovar \to \npmoddom$\\
	Valuation (of names/propositions) & $\val:A\to \wp(D)$, $\tsval:\props\to\wp(\moddom)$\\
	Marking/coloring & $\val^\natural:D\to \wp(A)$, $\tscolors:\moddom\to\wp(\props)$\\
	\hline
\end{tabular}
\end{center}
% \caption{Notational convention}
% \end{figure}
