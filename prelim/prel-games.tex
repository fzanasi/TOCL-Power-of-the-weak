We introduce some terminology and background on infinite games.
All the games that we consider involve two players called \emph{\'Eloise}
($\exists$) and \emph{Abelard} ($\forall$).
In some contexts we refer to a player $\Pi$ to specify a
a generic player in $\{\exists,\forall\}$.
%
Given a set $A$, by $A^*$ and $A^\omega$ we denote respectively the set of
words (finite sequences) and streams (or infinite words) over $A$.

A \emph{board game} $\mc{G}$ is a tuple $(G_{\exists},G_{\forall},E,\win)$,
where $G_{\exists}$ and $G_{\forall}$ are disjoint sets whose union
$G=G_{\exists}\cup G_{\forall}$ is called the \emph{board} of $\mc{G}$,
$E\subseteq G \times G$ is a binary relation encoding the \emph{admissible
moves}, and $\win \subseteq G^{\omega}$ is a \emph{winning condition}.
An \emph{initialized board game} $\mc{G}@u_I$ is a tuple
$(G_{\exists},G_{\forall},u_I, E,\win)$ where
%$(G_{\exists},G_{\forall},E,\win)$ is a board game and
$u_I \in G$ is the
\emph{initial position} of the game.
When $\win$ is  given by a parity function
$\pmap: G \to \omega$ we say that $\mc{G}$ is a parity game and sometimes
simply write $\mc{G}=(G_{\exists},G_{\forall},E,\pmap)$.

Given a board game $\mc{G}$, a \emph{match} of $\mc{G}$ is simply a path
through the graph $(G,E)$; that is, a sequence $\pi = (u_i)_{i< \alpha}$ of
elements of $G$, where $\alpha$ is either $\omega$ or a natural number,
and $(u_i,u_{i+1}) \in E$ for all $i$ with $i+1 < \alpha$.
A match of $\mc{G}@u_{I}$ is supposed to start at $u_{I}$.
Given a finite match $\pi = (u_i)_{i< k}$ for some $k<\omega$, we call
$\m{last}(\pi) := u_{k-1}$ the \emph{last position} of the match; the
player $\Pi$ such that $\m{last}(\pi) \in G_{\Pi}$ is supposed to move
at this position, and if $E[\m{last}(\pi)] = \emptyset$, we say that
$\Pi$ \emph{got stuck} in $\pi$.
%
A match $\pi$ is called \emph{total} if it is either finite, with one of the
two players getting stuck, or infinite. Matches that are not total are called
\emph{partial}.
Any total match $\pi$ is \emph{won} by one of the players:
If $\pi$ is finite, then it is won by the opponent of the player who gets stuck.
Otherwise, if $\pi$ is infinite, the winner is $\exists$ if $\pi \in
\win$, and $\forall$ if $\pi \not\in \win$.

Given a board game $\mc{G}$ and a player $\Pi$, let $\pmatches{G}{\Pi}$ denote
the set of partial matches of $\mc{G}$ whose last position belongs to player
$\Pi$.
A \emph{strategy for $\Pi$} is a function $f:\pmatches{G}{\Pi}\to G$.
A match $\pi  = (u_i)_{i< \alpha}$ of $\mc{G}$ is
\emph{$f$-guided} if for each $i < \alpha$ such that $u_i \in G_{\Pi}$ we
have that $u_{i+1} = f(u_0,\dots,u_i)$.
%
Let $u \in G$ and a $f$ be a strategy for $\Pi$.
We say that $f$ is a \emph{surviving strategy} for $\Pi$ in $\mc{G}@u$ if
%
\begin{enumerate}
  \item[(i)] For each $f$-guided partial match $\pi$ of $\mc{G}@u$, if $\m{last}(\pi)$
  is in $G_{\Pi}$ then $f(\pi)$ is legitimate, that is,
  $(\m{last}(\pi),f(\pi)) \in E$.
\end{enumerate}
%
We say that $f$ is a \emph{winning strategy} for $\Pi$ in $\mc{G}@u$ if, additionally, %the following condition is met
%We say that $u$ is a \emph{winning position} for $\Pi$ in $\mc{G}$ if, additionally, the following condition is met
%
\begin{enumerate}
  \item[(ii)] $\Pi$ wins each $f$-guided total match of $\mc{G}@u$.
\end{enumerate}
%
If $\Pi$ has a winning winning strategy for $\mc{G}@u$ then $u$ is called a \emph{winning position} for $\Pi$ in $\mc{G}$.
The set of positions of $\mc{G}$ that are winning for $\Pi$ is denoted by $\win_{\Pi}(\mc{G})$.
A strategy $f$ is called \emph{positional} if $f(\pi) = f(\pi^{\prime})$ for each $\pi,\pi^{\prime} \in \Dom(f)$ with $\m{last}(\pi) = \m{last}(\pi^{\prime})$.
A board game $\mc{G}$ with board $G$ is \emph{determined} if $G = \win_{\exists}(\mc{G}) \cup \win_{\forall}(\mc{G})$, that is, each $u \in G$ is a winning position for one of the two players.

\begin{fact}[Positional Determinacy of Parity Games~\cite{EmersonJ91,Mostowski91Games}]
\label{THM_posDet_ParityGames}
For each parity game $\mc{G}$, there are positional strategies $f_{\exists}$
and $f_{\forall}$ respectively for player $\exists$ and $\forall$, such that
for every position $u \in G$ there is a player $\Pi$ such that $f_{\Pi}$ is a
winning strategy for $\Pi$ in $\mc{G}@u$.
\end{fact}
%
From now on, we always assume that each strategy we work with in parity games
is positional. Moreover, we think of a positional strategy $f_\Pi$ for player $\Pi$
as a function $f_\Pi:G_\Pi\to G$.