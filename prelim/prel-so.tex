Given a set of propositional letters, or predicates, $\prop$, we define three variants of monadic second-order logic on it:
\emph{(standard) monadic second-order logic} ($\mso(\prop)$),
\emph{weak monadic second-order logic} ($\wmso(\prop)$) and
\emph{noetherian monadic second-order logic} ($\nmso(\prop)$).
We omit  $\prop$ when the set of proposition letters is clear from context. 
These logics share the same syntax.
\begin{definition}\label{def:mso}
The formulas of the \emph{monadic second-order
language} on a set of predicates $\prop$ are defined by the following grammar:
%
\begin{eqnarray*}\label{EQ_mso}
  \varphi ::= \here{p} \mid p \inc q \mid R(p,q) \mid \lnot\varphi \mid \varphi\lor\varphi \mid \exists p.\varphi,
\end{eqnarray*}
where $p$ and $q$ are letters from $\prop$.
We  adopt the standard convention that no letter is both free and bound in
$\varphi$.
\end{definition}

The three logics are distinguished by their semantics.
Let  $\model = \tup{T,R,\tscolors, s_I}$ be a LTS, the interpretation of the
atomic formulas and the boolean connectives is fixed and standard, e.g.:
\begin{align*}
\model \models \here{p} & \quad\text{ iff }\quad  \tsval(p) = \compset{s_I} \\
\model \models p \inc q & \quad\text{ iff }\quad  \tsval(p) \subseteq \tsval(q) \\
\model \models R(p,q) & \quad\text{ iff }\quad  \text{for every $s\in \tsval(p)$ there exists $t\in \tsval(q)$ such that $sRt$} 
\end{align*}

The interpretation of the existential quantifier is that

\begin{align*}
\model \models\ \exists p. \varphi  & \quad\text{ iff }\quad  \model[p \mapsto X] \models \phi \,
\left.\begin{cases}
 \text{for some }   & {\bf (\mso)} \\
  \text{for some \emph{finite} }   & {\bf (\wmso)} \\
    \text{for some \emph{noetherian} }   & {\bf (\nmso)} 
 \end{cases}\right\}\,
 X \subseteq T.
\end{align*}
%if and only if
%\begin{description}
%%[\IEEEsetlabelwidth{$\alpha\omega \pi\theta\mu\varphi$}\IEEEusemathlabelsep]
%\item[$(\Wmso)$] $\model[p \mapsto X] \models \phi$ for some finite $S \subseteq_\omega T$
%\item[$(\Nmso)$] $\model[p \mapsto X] \models \phi$ for some noetherian
%    $X \subseteq T$.
%\end{description}

Let $\varphi\in \mso$ be a formula. We denote with $\|\varphi \|_P$ the set
of $C$-transition structures $\model$ such that $\model\models \varphi$.
The subscript $P$ is omitted when the set $P$ of proposition letters is clear
from the context.
A class $\mc{L}$ of transition systems is $\mso$\emph{-definable} if there
is a formula $\varphi \in \mso$ such that $\| \varphi \| = \mc{L}$.
We define the analogous notions for $\wmso$ and $\nmso$ in the same way.



\begin{remark}
The reader may have expected a more standard two-sorted language for second-order logic, for example given by
%
$$
\varphi ::= p(x)
%\mid X(y)
\mid R(x,y)
\mid x \foeq y
\mid \neg \varphi
\mid \varphi \lor \varphi
\mid \exists x.\varphi
\mid \exists p.\varphi
$$%
where $p \in \prop$, $x,y \in \fovar$ (individual variables), %$X \in \sovar$ (second-order variables)
and $\foeq$ is the symbol for equality.
Both definitions can be proved to be equivalent, however, we choose to keep Definition~\ref{def:mso} as it will be better suited to work with in the context of automata.
\end{remark}
