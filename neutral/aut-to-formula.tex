% !TEX root = ../00CFVZ_TOCL.tex
\subsection{Automata and formulas}

\newcommand{\ytr}{\mathtt{tr}}

It is well-known that there are effective translations from automata to formulas
and vice versa~\cite{xxxx}.
The result on $\llang$-automata that we need in this paper is the following.

\begin{theorem}\label{t:autofor}
There is an effective procedure that, given an automaton $\bbA$ in 
$\Aut(\llang)$, returns a formula $\xi_{\bbA} \in \mu\llang$ which satisfies
the following properties:

(1) $\xi_{\bbA}$ is equivalent to $\bbA$;

(2) $\xi_{\bbA} \in \mu_{D}\llang$ if $\bbA \in \AutW(\llang)$;

(3) $\xi_{\bbA} \in \mu_{C}\llang$ if $\bbA \in \AutWC(\llang)$.
\end{theorem}

In the remainder of this subsection
% (which is not used in other parts of the paper) 
we focus on the proof of this theorem, which is (a refinement of)
a variation of the standard proof showing that any modal automaton can be 
translated into an equivalent formula in the modal $\mu$-calculus (see for
instance~\cite[Section 6]{Ven08}). 
For this reason we will not go into the details of showing that $\bbA$ and 
$\xi_{\bbA}$ are equivalent, but we will provide a detailed definition of the 
translation, and pay special attention to showing that the translations of weak
and of weak-continuous $\llang$-automata land in the right fragments of 
$\mu\llang$. 

The definition of $\xi_{\bbA}$ is by induction on the number of clusters of
$\bbA$, with a subinduction based on the number of states in the top cluster 
of $\bbA$.
For this inner induction we need to widen the class of $\llang$-automata, and
it will also convenient to introduce the notion of a preautomaton (which is  
basically an automaton without initital state).

\begin{definition}
A \emph{preautomaton} based on $\llang$ and $\props$, or briefly: a
\emph{preautomaton}, is a triple $\bbA = \tup{A,\tmap,\pmap}$ such that $A$ is
a (possibly empty) finite set of states, $\tmap: A\times \wp(\props) \to 
\llang^+(A)$ and $\pmap: A \to \nat$.

Given a set $X$ of propositional variables, a \emph{generalized preautomaton} 
over $\props$ and $X$ is a triple $\bbA = \tup{A,\tmap,\pmap}$ such that $\pmap:
A \to \nat$ is a priority map on the finite state set $A$, while the transition 
map is of the form $\tmap: A\times \wp(\props) \to \llang^+(A\cup X)$.
\end{definition}

\btbs
\item
define \& explain acceptance game
\item
Note that the difference with an ordinary preautomaton is that at a position of 
the form $(a,s) \in A \times S$, where $\eloise$ has to come up with a valuation 
to satisfy the formula $\tmap(a,\tscolors(s))$ in a model based on the set $R[s]$ of
successors of $s$, the interpretation of all propositional variables $x \in X$ 
is fixed (viz., as XXXX), so that $\eloise$ is only free to interpret the 
variables from the set $A$.
\etbs

Before giving the full translation we focus on the following somewhat technical
aspect.

\begin{definition}
Consider, for some preautomaton $\bbA = \tup{A,\tmap,\pmap}$, some state $a \in
A$, and some colour $c \in \wp(\props)$, the one-step formula $\tmap(a,c)\in 
\llang(A)$.
Suppose that for some subset $B \sse A$ we have a collection of 
$\mu\llang$-formulas $\{ \phi_{b} \mid b \in B \}$.
Without loss of generality we may write $\tmap(a,b) = 
\al(a_{1},\ldots,a_{m},b_{1},\ldots,b_{n})$, where the $a_{i}$ and $b_{j}$ 
belong to $A\setminus B$ and $B$ respectively.
Then we will denote the $\mu\llang$-formula 
$\nxt{\al}(a_{1},\ldots,a_{m},\phi_{1},\ldots,\phi_{n})$ as follows:
\[
\nxt{\tmap(a,c)}(\phi_{b}/b \mid b \in B)
\isdef \nxt{\al}(a_{1},\ldots,a_{m},\phi_{1},\ldots,\phi_{n}).
\]
\end{definition}

\noindent
We can now define the desired translation from $\llang$-automata to 
$\mu\llang$-formulas.

\btbs
\item
assume that we have properly defined the notion of substitution --- 
in preliminaries?
\etbs

\begin{definition}
\label{d:tr}
By induction on the number of clusters of a preautomaton $\bbA = \tup{A,\tmap,
\pmap}$ we define a map 
\[
\ytr_{\bbA}: A \to \mu\llang(P).
\]
Based on this definition, for an automaton $\bbA = \tup{A,\tmap,\pmap, a_{I}}$ 
we put
\[
\xi_{\bbA} \isdef \ytr_{\tup{A,\tmap,\pmap}}(a_{I}).
\]

In the base case of the definition of $\ytr$ the preautomaton $\bbA$ has no 
clusters at all, which means in particular that $A = \nada$.
In this case we let $\ytr_{\bbA}$ be the empty map.

In the inductive case we assume that $\bbA = \tup{A,\tmap,\pmap, a_{I}}$ does 
have clusters. 
Let $B \neq \nada$ be the highest cluster, and let $\bbA^{-}$ denote the 
preautomaton with carrier $A \setminus B$, obtained by restricting the maps 
$\tmap$ and $\pmap$ to the set $A \setminus B$.
Then inductively we may assume a translation $\ytr_{\bbA^{-}}: (A \setminus B)
\to \mu\llang(P)$, and we will define
\[
\ytr_{\bbA}(a) \isdef \ytr_{\bbA^{-}}(a), \quad\text{ if } a \in A \setminus B.
\]

To extend this definition to the states in $B$, we make a case 
distinction.
If $B$ is a degenerate cluster, that is, $B = \{ b \}$ for some state $b$ 
which is not active in itself, then we define
\[
\ytr_{\bbA}(b) \isdef
   \bigvee_{c \in \wp{\props}}
   \nxt{\tmap(b,c)}(\ytr_{\bbA^{-}}(a)/a \mid a \in A \setminus B).
\]
The main case of the definition is where $B$ is not degenerate.
Fix an enumeration $b_{1},\ldots,b_{n}$ of $B$ such that $i \leq j$ implies 
$\pmap(b_{i}) \leq \pmap(b_{j})$.
Let $\bbA_{k}$ be the generalized preautomaton obtained from $\bbA$ by 
restricting\footnote{Note that XXXX.
   }
the transition and parity map to the set
\[
A_{k} \isdef (A\setminus B) \cup \{ b_{1},\ldots,b_{k} \},
\]
so that $\bbA^{0} = \bbA^{-}$ and $\bbA^{n} = \bbA$.
Where $B_{k} \isdef \{ b_{1},\ldots,b_{k} \}$, we now define, by induction on 
$k$, a map 
\[
\ytr^{k}: B \to \mu\llang(P \cup (B \setminus B_{k})).
\]
In the base case of this definition we set
\[
\ytr^{0}(b) \isdef 
   \bigvee_{c \in \wp{\props}} 
   \nxt{\tmap(b,c)}(\ytr_{\bbA^{-}}(a)/a \mid a \in A \setminus B),
\]
and in the inductive case we first define $\eta_{k+1} \isdef \mu$ if
$\pmap(b_{k+1})$ is odd, and $\eta_{k+1} \isdef \nu$ if $\pmap(b_{k+1})$ is 
even, and then set
\[\begin{array}{llll}
     \ytr^{k+1}(b_{k+1}) &\isdef &
   \eta_{k+1} b_{k+1}. \ytr^{0}(b_{k+1})[\ytr^{k}(b_{i})/b_{i} \mid 1 \leq i\leq k]
\\ \ytr^{k+1}(b_{i}) &\isdef &
   \ytr^{k}(b_{i})[\ytr^{k+1}(b_{k+1})/b_{k+1}]
   & \text{ for } i \neq k+1.
\end{array}\]
Finally, we complete the definition of $\ytr_{\bbA}$ by putting
\[
\ytr_{\bbA}(b) \isdef \ytr^{n}(b),
\]
for any $b \in B$.
\end{definition}

In the proof of Theorem~\ref{t:autofor} we will need the following closure 
property of the fragments $\noe{\mu\llang}{\qprops}$ and 
$\cocont{\mu\llang}{\qprops}$.

\begin{proposition}
\label{p:comp}
Let $\rprops \subseteq \qprops$ be sets of proposition letters, and let $\phi$ 
and $\phi_{q}$, for each $q \in \qprop$, be formulas in $\mu\llang$.

(1) If $\phi$ and each $\phi_{q}$ belongs to 
    $\noe{\mu\llang}{\qprops\setminus\rprops}$ 
    ($\conoe{\mu\llang}{\qprops\setminus\rprops}$), 
   then so does $\phi[\phi_{q}/q \mid q \in \qprops]$.

(2) If $\phi$ and each $\phi_{q}$ belongs to
   $\cont{\mu\llang}{\qprops\setminus\rprops}$
   ($\cocont{\mu\llang}{\qprops\setminus\rprops}$), 
   then so does $\phi[\phi_{q}/q \mid q \in \qprops]$.
\end{proposition}

Both items of this proposition can be proved by a straightforward formula 
induction --- we omit the details.


\begin{proofof}{Theorem~\ref{t:autofor}}
As mentioned, the verification of the equivalence of $\xi_{\bbA}$ and $\bbA$
% for any $\llang$-automaton $\bbA$, 
is a standard exercise in the theory of parity automata and mu-calculi, and so
we omit the details.
We also skip the proof of item (2), completely focussing on item (3).

To prove this item, it suffices to take an arbitrary continuous-weak 
$\llang$-preautomaton $\bbA = \tup{A,\tmap,\pmap}$, and to show that 
\begin{equation}
\label{eq:tr1}
\ytr_{\bbA}(a) \in \mu_{C}\llang(\props)
\end{equation}
for all $a \in A$.
We will prove this by induction on the number of clusters of $\bbA$.

Since there is nothing to prove in the base case of the proof, we immediately
move to the inductive case.
Let $B$ be the highest cluster of $\bbA$.
By the induction hypothesis we have $\ytr_{\bbA}(a) = \ytr_{\bbA^{-}}(a) \in 
\mu_{C}\llang(\props)$ for all $a \in A \setminus B$, where 
$\bbA^{-}$ is as in Definition~\ref{d:tr}.
To show that \eqref{eq:tr1} also holds for all $b \in B$, we distinguish cases.

If $B$ is a degenerate cluster, say, $B = \{ b \}$, then for all $c \in 
\wp(\props)$, the variables occurring in the formula $\tmap(b,c) \in \llang(A)$
are all from $A \setminus \{ b \}$.
Given the definition of $\ytr_{\bbA}(b)$, it suffices to show that all formulas 
of the form $\nxt{\tmap(b,c)}(\ytr_{\bbA}(a)/a \mid a \in A \setminus \{ b \})$
belong to the set $\mu_{C}\llang(\props)$, but this is immediate by the 
induction hypothesis and the definition of the language.

If, on the other hand, $B$ is nondegenerate, let $b_{1},\ldots,b_{n}$ enumerate
$B$, and let, for $0\leq k \leq n$, the map $\ytr^{k}: B \to \mu\llang$ be as in
Definition~\ref{d:tr}.
We only consider the case where $B$ is an odd cluster, i.e., $\pmap(b)$ is odd
for all $b \in B$.
Our key claim here is that 
\begin{equation}
\label{eq:tr2}
\ytr^{k}(b_{i}) \in \mu_{C}\llang(\props \cup \{ b_{k+1},\ldots,b_{n}\}) \cap 
\cont{\mu\llang}{\{ b_{k+1},\ldots,b_{n}\}},
\end{equation}
for all $k$ and $i$ with $0 \leq k \leq n$ and $0 < i \leq n$.
We will prove this statement by induction on $k$ --- this is the `inner' 
induction that we announced earlier on.

In the base case of this inner induction we need to show that $\ytr^{0}(b_{i})$ 
belongs to both $\mu_{C}\llang(\props \cup B)$ and $\cont{\mu\llang}{B}$.
Showing the first membership relation is straightforward; for the second, the 
key observation is that by our assumption on $\bbA$, every one-step formula of
the form $\tmap(b_{i},c)$ is syntactically continuous in every variable $b \in 
B$.
Furthermore, by the outer inductive hypothesis we have $\ytr_{\bbA^{-}}(a) \in
\mu_{C}\llang(\props) \subseteq \cont{\mu\llang}{B}$,
for every $a \in A \setminus B$, and we trivially have that every variable $b 
\in B$ belongs to the set $\cont{\mu\llang}{B}$.
But then it is immediate by the definition of the fragment $\cont{\mu\llang}{B}$
that  the formula $\nxt{\tmap(b_{i},c)}(\ytr_{\bbA^{-}}(b)\mid b \in B)$ belongs
to it, and since this fragment is closed under taking disjunctions, we find that 
$\ytr^{0}(b_{i}) \in \cont{\mu\llang}{B}$ indeed.
This finishes the proof of the base case of the inner induction.

For the induction step we fix a $k$ and assume that
% $\ytr^{k}(b_{i}) \in \mu_{C}\llang(\props  \cup \{ b_{k+1},\ldots,b_{n}\}) 
% \cap \cont{\mu\llang}{\{ b_{k+1},\ldots,b_{n}\}}$,
\eqref{eq:tr2} holds for this $k$ and for all $i$ with $0 < i \leq n$.
We will prove that
\begin{equation}
\label{eq:tr3}
\ytr^{k+1}(b_{i}) \in \mu_{C}\llang(\props) \cap 
    \cont{\mu\llang}{\{ b_{k+2},\ldots,b_{n}\}}.
\end{equation}
first for $i = k+1$, and then for an arbitrary $i \neq k+1$.
To prove \eqref{eq:tr3} for the case $i = k+1$, first note that
% \begin{equation}
% \label{eq:tr3}
% \ytr^{k+1}(b_{k+1}) \in \mu_{C}\llang(\props) \cap 
%     \cont{\mu\llang}{\{ b_{k+2},\ldots,b_{n}\}}.
% \end{equation}
% To see this, 
% \[\begin{array}{lcl}
% \ytr^{0}(b_{k+1}) &\in& \mu_{C}\llang(\props\cup \{ b_{k+1},\ldots,b_{n}\}) 
%    \cap \cont{\mu\llang}{\{ b_{k+1},\ldots,b_{n}\}},
% \end{array}\]
\[\begin{array}{lcl}
\ytr^{0}(b_{k+1}) &\in& \mu_{C}\llang(\props\cup B) 
   \cap \cont{\mu\llang}{B},
\end{array}\]
as we just saw in the base case of the inner induction.
But then it is immediate by Proposition~\ref{p:comp} and the induction 
hypothesis on the formulas $\ytr^{k}(b_{i})$ that 
\[
\ytr^{0}(b_{k+1})[\ytr^{k}(b_{i})/b_{i} \mid 1 \leq i\leq k]
\in \cont{\mu\llang}{\{ b_{k+1},\ldots,b_{n}\}},
\]
and from this it easily follows by the definition of 
$\cont{\mu\llang}{\{ b_{k+1},\ldots,b_{n}\}}$ that 
\[
\mu b_{k+1} b_{k+1}. \ytr^{0}(b_{k+1})[\ytr^{k}(b_{i})/b_{i} \mid 1 \leq i\leq k]
\in \cont{\mu\llang}{\{ b_{k+2},\ldots,b_{n}\}},
\]
that is, 
\[
\ytr^{k+1}(b_{k+1}) \in \cont{\mu\llang}{\{ b_{k+2},\ldots,b_{n}\}}.
\]
This is the crucial step in proving \eqref{eq:tr3} for the case $i=k+1$,
the proof that $\ytr^{k+1}(b_{k+1}) \in \mu_{C}\llang(\props)$ is easy.

Second, to prove \eqref{eq:tr3} for the case $i \neq k+1$, we first 
% claim that
% \begin{equation}
% \label{eq:tr4}
% \ytr^{k+1}(b_{i}) \in \mu_{C}\llang(\props) \cap 
%     \cont{\mu\llang}{\{ b_{k+2},\ldots,b_{n}\}}.
% \end{equation}
% To see this,
recall that by the induction hypothesis we have
\[
\ytr^{k}(b_{i}) \in \mu_{C}\llang(\props) \cap 
    \cont{\mu\llang}{\{ b_{k+1},\ldots,b_{n}\}},
\]
while we just saw that 
$\ytr^{k+1}(b_{k+1}) \in \cont{\mu\llang}{\{ b_{k+2},\ldots,b_{n}\}}$.
But from the latter two statements it is immediate by Proposition~\ref{p:comp} 
that 
\[
\ytr^{k+1}(b_{i}) = \ytr^{k}(b_{i})[\ytr^{k+1}(b_{k+1})/b_{k+1}] 
\in \cont{\mu\llang}{\{ b_{k+2},\ldots,b_{n}\}}
\]
so that we have indeed proved \eqref{eq:tr2} for the case $i \neq 
k+1$.
This finishes the proof of the inner induction.

Finally, it follows from \eqref{eq:tr2}, instantiated with $k = n$, 
that for all $b \in B$ we have
\[
\ytr_{\bbA}(b) = \ytr^{n}(b) \in \mu_{C}\llang(\props),
\]
as required for proving the outer induction step.
In other words, we are finished with the proof of \eqref{eq:tr1}, and 
hence, finished with the proof of the theorem.
\end{proofof}


