% !TEX root = ../00CFVZ_TOCL.tex

In this subsection we focus on the other meaning preserving translation from formula to automata.

\begin{theorem}\label{t:fortoaut}
There is an effective procedure that, given a formula $\xi  \in \mu\llang$ returns an automaton $\bbA_{\xi}$ in 
$\Aut(\llang)$, which satisfies
the following properties:

(1) $\bbA_\xi$ is equivalent to $\xi$;

(2)  $\bbA_\xi \in \AutW(\llang)$ if $\xi \in \mu_{D}\llang$;

(3)  $\bbA_\xi \in \AutWC(\llang)$ if $\xi \in \mu_{C}\llang$.
\end{theorem}


As before, the proof of this theorem, presented in the remainder of this subsection, is  also (a refinement of)
a variation of the standard proof showing that any fixpoint modal formula can be 
translated into an equivalent modal automaton (see for
instance~\cite[Section 6]{Ven08}). 
Hence, we will not go too much into the details of showing that ${\bbA}_\xi$ and 
$\xi$ are equivalent, but we will provide a detailed definition of the 
translation, and pay special attention to showing that the translations of weak
and of weak-continuous fragments of 
$\mu\llang$ land in the right subclass of  $\llang$-automata.

We first gather the observation that, without loss of generality, fixpoint variables can be assumed to be guarded, that is in a scope of a modal operator.
 
\begin{definition}
An occurrence of a bound variable $p$ in $\xi  \in \mu\llang$ is called \emph{guarded} if there is a modal operator between its binding definition and the variable itself. An occurrence is weakly guarded if there is another fixpoint quantifier between its binding definition and the variable itself. A formula $\xi  \in \mu\llang$ is called guarded if every occurrence of every bound variable is guarded.
\end{definition}

\begin{proposition}
For every $\xi  \in \mu\llang$ we can effectively construct a guarded formula $\xi^\flat  \in \mu\llang$ such that $\xi \equiv \xi^\flat$. Moreover

(1)  $\xi^\flat \in \mu_{D}\llang$ if $\xi \in \mu_{D}\llang$;

(2)  $\xi^\flat \in \mu_{C}\llang$ if $\xi \in \mu_{C}\llang$.
\end{proposition}
\btbs
\item
For this result, reference to "Orna Kupferman, Moshe Y. Vardi, and Pierre Wolper. An automata-theoretic approach to branching-time model checking."
\etbs

Let $\xi$ be formula. By the previous result, without loss of generality we can assume that $\xi$ is guarded. 
We therefore define the automaton $\bbA_{\xi}=\tup{A,\tmap,\pmap,a_I}$ such that 

\begin{itemize}
\item $A:=\{ \fsub{\psi} \mid \psi  \subf \xi\}$,
\item $a_I:= \fsub{\xi}$
\item $\pmap: A \to \nat$ is the parity map where 
%$\pmap(\hat{\psi}):= 
%\begin{cases} 
%0 & \text{if $\psi$ is a $\nu$-subformula of $\xi$} \\ 
%1 & \text{ otherwise.}\\
%\end{cases}
%$
\btbs
\item
Give definition, compatible with weakness condition (cf. Wilke)
\etbs
\end{itemize}
The critical point is to define the transition map $\tmap: A\times \wp(\props) \to \llang^+(A)$.

We do it in three steps.
Firstly define a map $\Delta': A \to \mu\llang(A \cup  \props)$ by simply stating
\begin{align*}
		\tmap'(\fsub{\psi \lor \gamma}) &:= \fsub{\psi} \lor \fsub{\gamma} &
		\tmap'(\fsub{\psi \land \gamma}) &:= \fsub{\psi} \land \fsub{\gamma} \\
		\tmap'(\fsub{\nxt{\al}(\psi_{1},\ldots,\psi_{n})}) &:= \nxt{\al}(\fsub{\psi_{1}},\ldots,\fsub{\psi_{n}}) &
		\tmap'(\fsub{\somefp_p p.\psi}) &:= \fsub{\psi}
	\end{align*}
	\vspace{-2mm}
	\begin{align*}	
		\tmap'(\fsub{\gamma}) &:= \gamma \quad \text{for $\gamma \in \{\top,\bot,p,\lnot p\}$ with \emph{unbound} $p$,} \\
		\tmap'(\fsub{p}) &:= \fsub{\delta_p} \quad \text{for bound $p$.}
	\end{align*}
	
Secondly, for every $\fsub{\psi}\in A$, we define a new term $\underline{\tmap}(\fsub{\psi})$. To begin with, we set $\underline{\tmap}(\fsub{\psi}) := \tmap'(\fsub{\psi})$. Next, we replace every non-guarded occurrence of every $\fsub{\gamma}\in A$ in $\underline{\tmap}(\fsub{\psi})$ by $\tmap'(\fsub{\gamma})$. We repeat this process until every occurrence of every $\fsub{\gamma} \in A$ is of the form $\nxt{\al}(\fsub{\psi_{1}},\ldots,\fsub{\psi_{n}})$. It is crucial to observe that this process will eventually converge because $\xi$ was originally \emph{guarded}. This means that while running this process, for every branch of $\underline{\tmap}(\fsub{\psi})$ we will always go through some modality before unfolding some state that was already unfolded.

	Observe that $\underline{\tmap}(\fsub{\psi})$ belongs to the modal sublanguage of $\mu\llang(A \cup  \props)$ whose formulas have modal depth one and are given by
	\[
\varphi ::= 
   p \mid \neg p \mid \top \mid \bot 
   \mid \varphi\lor\varphi \mid \varphi\land\varphi 
   \mid \nxt{\al}(\fsub{\psi_{1}},\ldots,\fsub{\psi_{n}})
\]
where $p \in\props$, $\fsub{\psi_{1}},\ldots,\fsub{\psi_{n}} \in A$. Now, notice that every $\underline{\tmap}(\fsub{\psi})$ is equivalent to a formula $\bigvee_{{\sf Q}\in \wp{(\props)}} (\tau_{{\sf Q}} \land \psi_{{\sf Q}})$, where $\psi_{{\sf Q}}$ is either $\bot$, $\top$ or $\nxt{\al}(\fsub{\psi_{1}},\ldots,\fsub{\psi_{n}})$, for some $\fsub{\psi_{1}},\ldots,\fsub{\psi_{n}} \in A$.

Finally, we state $\tmap: (\fsub{\psi}, {\sf Q}) \mapsto 
\begin{cases}
\top & \text{ if $\psi_{{\sf Q}}= \top$} \\
\bot & \text{ if $\psi_{{\sf Q}}= \bot$} \\
\al(\fsub{\psi_{1}},\ldots,\fsub{\psi_{n}}) & \text{ if $\psi_{{\sf Q}}=\nxt{\al}(\fsub{\psi_{1}},\ldots,\fsub{\psi_{n}})$} \\
\end{cases}
\in \llang^+(A)$.


\begin{proofof}{Theorem~\ref{t:fortoaut}}
It is straightforward to show that $\bbA_\xi$ is equivalent to $\xi$. The crux of the matter is thence to verify that the translation lands in the right class of automata.
%Recall that  $a \ord b$ in the automaton $\bbA_\xi$ if $b$ can be reached from $a$ in the graph structure induced by the transition map of $\bbA_\xi$
Before doing this, we gather another easy observation on the structural properties of formulas without alternation of fixpoints. 
We say that a subformula 
$\psi  \subf \xi$  is a $\mu$-subformula of $\xi$ (resp. $\nu$-subformula) if there is a subformula $\mu p. \psi' \supf \psi$ (resp. $\nu p. \psi' \supf \psi$) of $\xi$ and $p$ is free in $\psi$. 
Hence, let $\xi$ be alternation-free, we immediately obtain that:

\begin{claim}
Every subformula $\psi$ of $\xi$ belongs to exactly one of the following categories: (a) no free variable of $\psi$ is bound in $\xi$; or (b) $\psi$ is a $\mu$-subformula; or (c) $\psi$ is a $\nu$-subformula.
\end{claim}

From this, given $\xi$ alternation-free, it  therefore holds that, if $\fsub{\psi}$ and $\fsub{\gamma}$ belongs to the same cluster, modulo switching the order of the states, one of the following cases holds:
		\begin{align*}
			\text{(a) } \somefp_p p.\delta_p & \supf \psi \supf \gamma \supf p &
			\text{(b) } \somefp_p p.\delta_p & \supf \psi \supf p &
			\text{(c) } \somefp_q q.\delta_q \supf \somefp_p p.\delta_p & \supf \psi \supf q \\
								   &&& \supf \gamma \supf p
								   && \supf \gamma \supf p,
		\end{align*}
		where $p$ is both free in $\psi$ and $\gamma$  in cases (a) and (b); and $q$ is free in $\psi$ and $p$ is free in $\gamma$ in (c). In particular, both $\psi$ and $\gamma$ are $\somefp$-subformulas of $\xi$ of the same type $\somefp$, thus $\somefp_p=\somefp_q$ in (c). From this, we obtain that $\bbA_\xi \in \AutW(\llang)$, whenever $\xi  \in \mu_{D}\llang$.

Next, we verify that the translation preserve continuity. We do it only for the case of a cluster $M$ corresponding to a least fixpoint, the other case being dual. Thus, let $\fsub{\psi} \in M$. We need to check that for every  $\qprops$, the formula $\tmap(\fsub{\psi}, \qprops)$ is continuous in $M$, or equivalently that $\tmap(\fsub{\psi}, \qprops)$ is continuous in each $\fsub{\gamma} \in M$. 

		We consider the relative positioning of $\psi$ and $\gamma$ in the formula graph of $\xi$, assuming that $\somefp_p=\somefp_q=\mu$. In cases (b) and (c) above, it is obvious that $\tmap(\fsub{\psi}, \qprops)$ is continuous in $\fsub{\gamma}$, since $\fsub{\gamma}$ does not occur in $\tmap(\fsub{\psi},  \qprops)$. For case (a) we reason as follows. 
		Let us assume for simplicity that $p$ occurs only once in $\delta_p$. Thus let $\pi:= \pi(0)\dots \pi(n)$ be the unique path from $\delta_p$ to  $p$ in the formula tree of $\xi$ (hence $\pi(0)=\delta_p$,  $\pi(n)=p$, $\phi(i)=\psi$ and $\phi(k)=\gamma$ for some $i<k$). Note that $\phi(\ell) \in \cont{\mu\llang}{p}$, for $\ell \leq n$. We  verify that $\tmap(\fsub{\phi(\ell)}, \qprops)$ is continuous in $M_\ell:=\{\fsub{ \phi(i)} \mid \ell \leq i \leq n\}$. We only consider the case when $\phi(\ell)$ is of the form $\nxt{\al}(\psi_{1},\ldots,\psi_{m}, \phi(\ell+1))$, the other cases being immediate. Notice that $p$ occurs free in $\phi(\ell+1)$  but does not occurs in any $\psi_i$. Since $\phi(\ell) \in \cont{\mu\llang}{p}$, it holds that $\al(a, b_1, \dots, b_m)  \in 
\cont{\llang}{a}$, for $b_1=\fsub{\psi_{1}},\ldots,b_m=\fsub{\psi_{m}}, a=\fsub{\phi(\ell+1)}$. A fortiori $\al(\fsub{\psi_{1}},\ldots,\fsub{\psi_{m}}, \fsub{\phi(\ell+1)}) $ is $M_\ell$-continuous, and therefore it is not difficult to see that  $\tmap(\fsub{\phi(\ell)}, \qprops)$ is also $M_\ell$-continuous. 
		
%Let $\qprops:=\{ q \in \props \mid \fsub{\mu q. \phi} \in M\}$, and let $\pi:= \pi(0)\dots \pi(n)$ the path from$\delta_p$ to  $p$ in the formula tree of $\xi$ (hence $\pi(0)=\delta_p$ and $\pi(n)=p$). For the initial step, $\gamma = p$, and thus $\gamma$ is continuous on  

\btbs
\item Check last paragraph. Correct? Enough?
%\item Remember to add that Section from Aut to For holds also for FO$_1$
\etbs
\end{proofof}
