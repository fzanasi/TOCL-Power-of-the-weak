% \section{Introduction}\label{sec:intro}

\subsection{Expressiveness modulo bisimilarity.}
%
This paper concerns the relative expressive power of some languages used for
describing properties of pointed labelled transitions systems, or Kripke
models.
The interest in such expressiveness questions stems from applications where
these structures model computational processes, and bisimilar pointed
structures represent the \emph{same} process.
Seen from this perspective, properties of transition structures are relevant
only if they are invariant under bisimilarity.
This explains the importance of bisimulation invariance results of the form
\begin{equation*}
%\label{eq:1}
%\eqno{(*)}
M \equiv L / {\bis} \text{ (over $K$)}
\end{equation*}
stating that,  if one restricts attention to a certain class $K$ of transition
structures, one language $M$ is expressively complete with respect to the
relevant (i.e., bisimulation-invariant) properties that can be formulated in
another language $L$.
In this setting, generally $L$ is some rich yardstick formalism such as
first-order or monadic second-order logic, and $M$ is some modal-style
fragment of $L$, usually displaying much better computational behavior
than the full language $L$.

A seminal result in the theory of modal logic is van Benthem's Characterisation
Theorem~\cite{vanBenthemPhD}, stating that every bisimulation-invariant
first-order formula $\alpha(x)$ is actually equivalent to (the standard
translation of) a modal formula:
\begin{equation*}
%\label{eq-vB}
\ML \equiv \fo/{\bis} \text{ (over the class of all LTSs)}.
\end{equation*}
Over the years, a wealth of variants of the Characterisation Theorem have been
obtained.
For instance, Rosen proved that van Benthem's theorem is one of the few
preservation results that transfers to the setting of finite
models~\cite{rose:moda97}; for a recent, rich source of van Benthem-style
characterisation results, see Dawar \& Otto~\cite{DawarO09}.

\subsection{Second-order logics and modal $\mu$-calculi}

In this paper we are mainly interested is the work of Janin \&
Walukiewicz~\cite{Jan96}, who extended van Benthem's result to the setting
of fixpoint logics, by proving that the modal $\mu$-calculus ($\MC$) is the
bisimulation-invariant fragment of monadic second-order
logic ($\mso$):
\begin{equation*}
%\label{eq-JW}
\MC \equiv \mso/{\bis} \text{ (over the class of all LTSs)}.
\end{equation*}

\subsection{Bisimulation invariance for $\wmso$.}
The yardstick logic that we consider in this paper is \emph{weak} monadic
second-order logic ($\wmso$), a variant of monadic second-order logic where
the second-order quantifiers range over \emph{finite} subsets of the
transition structure rather than over arbitrary ones.
Our target will be to identify the bisimulation-invariant fragment of this
logic $\wmso$.

Before moving on, we should stress the role of the ambient class $K$ in
bisimulation-invariance results.
Of particular importance in the setting of weak monadic second-order logic is
the difference between structures of finite versus arbitrary branching degree.
In the case of finitely branching models, it is not very hard to show that
$\wmso$ is a (proper) fragment of $\mso$, and it seems to be folklore that
$\wmso/{\bis}$ corresponds to $\afmc$, the alternation-free fragment of the
modal $\mu$-calculus.
For binary trees, this result was proved by Arnold \& Niwi{\'n}ski in
\cite{ArnoldN01}.
In the case of structures of arbitrary branching degree, however, $\wmso$
and $\mso$ have \emph{incomparable} expressive power.
The fact that, in particular, $\wmso$ does not correspond to a fragment of
$\mso$, is witnessed by the class of infinitely branching structures, which
is clearly $\wmso$-definable, cannot be defined in $\mso$, since every
$\mso$-definable class of trees contains a finitely branching
tree.\footnote{As remarked in \cite{CateF11}, this follows from the automata characterisation of MSO given in \cite{Walukiewicz96}.} %\afnote{This follows implicitly from Igor's automata characterization of MSO, cf. props. 2 in my paper with Balder.}
For this reason, the relative expressive power of $\wmso/{\bis}$ and
$\mso/{\bis}$ is not a priori clear.
However, it is reasonable to think that $\wmso/{\bis}$ is strictly \emph{weaker} than \afmc: the class of well-founded trees, which is definable in $\afmc$ by
the simple formula $\mu p. \Box p$, is not definable in \wmso.\footnote{This follows from the fact that $\wmso$ can only define properties of trees that, from a topological point of view, are Borel, which is not the case of the class of trees defined by $\mu p. \Box p$--- see e.g. \cite{CateF11}.}
% \begin{comment}
% What is clear, however, is that $\wmso/{\bis}$ is strictly \emph{weaker}
% \fzwarning{How do we know even that the bis. inv. fragment of WMSO is weaker than $\MC$?}
% than \afmc: the class of well-founded trees, which is definable in \afmc by
% the simple formula $\mu p. \Box p$, is not definable in \wmso.
% (CITATION NEEDED)\fznote{Alessandro, was the (CITATION NEEDED) first stated in your thesis \cite{FacchiniPhD}?}.
% \end{comment}
Incidentally, the question whether, conversely, there is a natural logic of
which the
bisimulation-invariant fragment corresponds to $\afmc$ was answered positively
by three of the present authors in~\cite{DBLP:conf/lics/FacchiniVZ13}, where they introduced
another variant of $\mso$, called well-founded $\mso$ (\nmso), and proved
that $\nmso/{\bis} \equiv \afmc$ (over the class of all LTSs).

The main result that we shall prove in this paper states that the
bisimulation-invariant fragment of $\wmso$ is equivalent to a certain,
fragment $\mucML$ of the modal $\mu$-calculus.
\begin{equation}
\label{eq-main}
\mucML \equiv \wmso/{\bis}  \text{ (over the class of all LTSs)}.
\end{equation}
This fragment $\mucML$, which is strictly weaker than the alternation-free fragment
of $\muML$, is characterised by a certain restriction on the application of
fixpoint operators, which involves the notion of \emph{(Scott) continuity}.

Continuity, an interesting property that features naturally in the semantics
of many (fixpoint) logics, in fact plays a key role throughout this paper.
For its definition, we consider how the meaning $\haak{\phi}^{\model} \sse T$
of a formula $\phi$ in some structure $\model$ (with domain $T$) depends on the meaning of a fixed
proposition letter or monadic predicate symbol $p$.
This dependence can be formalised as a map $\phi^{\model}_{p}: \wp(T) \to
\wp (T)$, and if this map satisfies the condition
\begin{equation}
\label{eq-Sc}
\phi^{\model}_{p}(X) = \bigcup \Big\{ \phi^{\model}_{a}(X') \mid X'
\text{ is a finite subset of } X \Big\},
\end{equation}
we say that $\phi$ is \emph{continuous in $p$}.
The topological terminology stems from the observation that \eqref{eq-Sc}
expresses the continuity of the map $\phi^{\model}_{p}$ with respect to the Scott
topology on $\pw(T)$. % and $\pw(T)$.
If we look at concrete cases, this definition can be given a different reading:
if $\phi$ is a formula of the
modal $\mu$-calculus, \eqref{eq-Sc} means that $\phi$ holds at some state $s$
of $\model$ iff we can shrink the interpretation of the proposition letter $p$
to some finite subset of the original interpretation, in such a way that
$\phi$ holds at $s$ in the modified version of $\model$.

%\btbs
%\item
%mention nice properties?
%\etbs

A syntactic \emph{characterization} of this property for the modal $\mu$-calculus
was obtained by Fontaine~\cite{Fontaine08,FV12}, and the definition of our fragment $\mucML$
uses this characterization as follows:
whereas in the full language of $\muML$ the only syntactic condition on the
formation of a formula $\mu p. \phi$ is that $\phi$ is \emph{positive} in $p$,
for the fragment $\mucML$ this condition is strengthened to the requirement that
$\phi$ is (syntactically) \emph{continuous} in $p$.
More precisely, the fragment $\mucML$ is defined as follows:

\begin{definition}
For each set $\qprops$ of
proposition letters, the fragment $\cont{\MC}{\qprops}$ of $\MC$ which is \emph{continuous in $\qprops$}
is given by the simultaneous induction
\begin{equation*}
   \varphi ::= q
   \mid \psi
   \mid \varphi \lor \varphi
   \mid \varphi \land \varphi
   \mid \Diamond \varphi
   \mid \mu p.\alpha
\end{equation*}
where $p\in\props$, $q \in \qprops$, $\psi$ is a $\qprops$-free $\MC$-formula, and
$\alpha \in \cont{\MC}{\qprops\cup\{p\}}$.
%
The formulas of the fragment $\mucML$ are then given by the following induction:
\begin{equation*}
   \varphi ::= p \mid \lnot \varphi
    \mid \varphi \lor \varphi
    \mid  \Diamond \varphi
    \mid \mu p.\alpha
\end{equation*}
where $p \in \props$, and $\alpha \in \cont{\MC}{p}$.
\end{definition}

In fact we will prove, analogous to the result by Janin \& Walukiewicz,
the following strong version of the characterization result~\eqref{eq-main},
which provides an explicit translation, mapping any
bisimulation-invariant formula $\phi$ in $\wmso$ to an equivalent formula
$\phi^{\bullet}$ in $\mucML$.

%\afnote{Enumerate in the theorem to be more readable?}
\begin{theorem}
\label{t:m1}
There are effective translations $(-)^{\bullet}: \wmso \to \mucML$ and
$(-)_{\bullet}: \mucML \to \wmso$ such that 
\begin{enumerate}
\item A formula $\phi$ of $\wmso$ is
bisimulation invariant if and only if $\phi \equiv \phi^{\bullet}$, and
\item $\psi \equiv \psi_{\bullet}$ for every formula $\psi \in \mucML$.
\end{enumerate}
\end{theorem}

To see how this theorem implies \eqref{eq-main}, observe that part (i)
shows that $\wmso/{\bis} \leq \mucML$.
Part (ii) states that $\mucML \leq \wmso$, so combined with the fact that
every formula in $\mucML \sse \MC$ is bisimulation invariant, this gives the
converse, $\mucML \leq \wmso/{\bis}$.



\subsection{Automata for $\wmso$.}
%
As usual in this research area, our proof will be automata-theoretic in
nature.
More specifically, as the second main contribution of this paper, we
introduce a new class of parity automata that exactly captures the expressive
power of $\wmso$ over the class of tree models of arbitrary branching degree.

Before we turn to a description of these automata, we first have a look at the
automata, introduced by Walukiewicz~\cite{Walukiewicz96}, corresponding to $\mso$
(over tree models).
Fixing the set of proposition letters of our models as $\props$, we think of
$\wp(\props)$ as an \emph{alphabet} or set of \emph{colors}.
We can then define an $\mso$-automaton as a tuple $\aut = \tup{A,\tmap,\pmap,a_I}$, where $A$ is a finite set of states, $a_I$ an initial state, and $\pmap:
A \to \bbN$ is a parity function.
%
The transition function $\tmap$ maps a pair $(a,c) \in A \times \pw(\props)$ to a
sentence in the first-order language (with equality) $\ofoe(A)$, of which the
state space $A$ provides the set of (monadic) predicates.
%
For a more precise definition, let $\ofoe^+(A)$ denote the set of
those sentences in $\ofoe(A)$ where all predicates in $A$ occur only positively;
% the formulas of this language can be given by the following syntax:
% \[
% \varphi \mathrel{::=}
% a(x)
% %\mid \neg a(y)
% \mid x \foeq y
% \mid \neg (x \foeq y)
% %\mid \neg \varphi
% \mid \varphi \lor \varphi
% \mid \varphi \land \varphi
% \mid \exists x.\varphi
% \mid \forall x.\varphi
% \]
% where $a \in A$ and $x,y$ represent individual first-order variables.
we require that $\tmap: A \times \wp(\props) \to \ofoe^+(A)$.

We shall refer to $\ofoe$ as the \emph{one-step language} of $\mso$-automata,
and denote the class of $\mso$-automata with $\Aut(\ofoe)$.
%, because of the key role of $\ofoe(A)$ in
The automata that we consider in this article run on labelled transition systems
and decide wether to accept or reject them. To take such decision we associate
an acceptance game for an
$\mso$-automaton $\bbA$ and a transition system $\model$.
A match of this game consists of two players, $\eloise$ and $\abelard$, moving a
token from one position to another.
When such a match arrives at a so-called \emph{basic} position, i.e., a
position of the form $(a,t) \in A \times T$, the players consider the
sentence $\tmap(a,c_{t}) \in \ofoe^+(A)$, where $c_{t} \in \wp(\props)$ is the color
of $t$ (that is, the set of proposition letters true at $t$).
At this position $\eloise$ has to turn the set $R[t]$ of successors of
$s$ into a \emph{model} for the formula $\tmap(a,c_{t})$ by coming up with an
interpretation $I$ of the monadic predicates $a \in A$ as subsets of $R[s]$,
so that the resulting first-order structure $(R[s],I)$ makes the
formula $\tmap(a,c_{t})$ true.

% \btbs
% \item
% explain role of $\abelard$?
% \etbs

Walukiewicz's key result linking $\mso$ to $\Aut(\ofoe)$ states that
\begin{equation}
\mso \equiv \Aut(\ofoe)
 \text{ (over tree models)},
\end{equation}
and the proof of this result proceeds by inductively showing that every formula
$\phi$ in $\mso$ can be effectively transformed into an equivalent
automaton $\bbA_{\phi} \in \Aut(\ofoe)$.
For the details of this construction, a fairly intricate analysis of the
one-step logic $\ofoe$ is required, crucially involving various normal
forms of the sentences of $\ofoe(A)$.

In order to adapt this approach to the setting of \wmso, observe that by
K\"onig's lemma, a subset of a tree $\model$ is finite iff it is both a subset of
a finitely branching subtree of $\model$ and \emph{noetherian}, that is, a subset
of a subtree of $\model$ that has no infinite branches.
This suggests that we may change the definition of $\mso$-automata into one
of $\wmso$-automata via two kinds of modifications, roughly speaking
corresponding to a horizontal and a vertical `dimension' of trees.

For the `vertical modification' we may turn to the literature on weak automata~\cite{MullerSaoudiSchupp92}.
The acceptance condition $\pmap$ of a parity automaton $\bbA =
\tup{A, \tmap, \pmap, a_I}$ is \emph{weak} if $\pmap(a) = \pmap(a')$ whenever
the states $a$ and $a'$ belong to the same strongly connected component
(SCC) of the automaton. To see that the notion of connected component is well-defined
observe that for $\bbA$ we can associate a directed graph on $A$
such that $a,b \in A$ are connected iff $b$ occurs in $\tmap(a,c)$ for some $c \in \wp(\props)$.
%
Let $\AutW(\ofoe)$ denote the set of $\mso$-automata with a weak parity
condition.
It was proved in \cite{Zanasi:Thesis:2012} (see also \cite{DBLP:conf/lics/FacchiniVZ13}) that
\begin{equation*}
%\label{eq-wfmso}
\nmso \equiv \AutW(\ofoe) \text{ (over the class of all trees)},
\end{equation*}
with $\nmso$ denoting the earlier mentioned variant of $\mso$ %/$\wmso$
where second-order quantification is restricted to noetherian subsets of trees.
From this it easily follows that
\begin{equation*}
%\label{eq-wfmso2}
\wmso \equiv \AutW(\ofoe) \text{ (over the class of finitely
branching trees)},
\end{equation*}
since the noetherian subsets of a finitely branching trees correspond to the
finite ones.
Over the class of all tree models, however, $\wmso$ is \emph{not} equivalent
to $\AutW(\ofoe)$, %$ \equiv \nmso$
as is witnessesed by the earlier mentioned
class of well-founded trees, which can be defined in $\afmc \leq \nmso$,
but not in $\wmso$.

The hurdle to take, in order to find automata for WMSO on trees of
\emph{arbitrary} branching degree, concerns the horizontal dimension; the
main problem lies in finding the right one-step language for
$\wmso$-automata.
An obvious candidate for this language would be weak monadic second-order logic
itself, or more precisely, its variant $\owmso$ over the signature of monadic
predicates (corresponding to the automata states).
A very helpful observation, made by V\"a\"an\"anen~\cite{vaananen77}, states that
\[
\owmso \equiv \ofoei,
\]
where $\ofoei$ is the extension of $\ofoe$ with the generalized quantifier
$\qu$, where $\qu x. \phi$ meaning that there are \emph{infinitely} many
objects satisfying $\phi$.
Taking the \emph{full} language of $\owmso$ or $\ofoei$ as our one-step language
would give too much expressive power: since $\ofoei$ extends $\ofoe$,
we would find that, over tree models, $\AutW(\ofoei)$ extends
$\AutW(\ofoe)$, whereas we already saw that $\AutW(\ofoe) \equiv
\nmso$ is incomparable to $\wmso$.
%
It is here that we will crucially involve the notion of \emph{continuity}.
The automata corresponding to $\wmso$ will be of the form $\bbA = \tup{A, \tmap, \pmap, a_I}$,
where the transition map $\tmap: A \times \wp(\props) \to
{\ofoei}^+(A)$ is subject to the following two constraints, for all $a,a' \in A$
belonging to the same strongly connected component of $A$:
\begin{description}
\itemsep 0 pt
\item[(weakness)] $\pmap(a) = \pmap(a')$, and
\item[(continuity)]
if $\pmap(a)$ is odd (resp. even), then for each colour $c\in \wp(\props)$,
   $\tmap(a,c)$ is continuous (resp. co-continuous) in $a'$,
\end{description}
where co-continuity is a dual notion to continuity.
The class of these automata is denoted by $\AutWC(\ofoei)$.
%
Consequently, for a proper definition of these automata we need a
\emph{syntactic} characterization of the $\ofoei(A)$-sentences that are
(co-)continuous in one (or more) monadic predicates of $A$.

For this purpose, we conduct a fairly detailed model-theoretic study of the
logic $\ofoei$ which we consider to be the third main
contribution of our work.
Similar to the results for $\ofoe$, we provide normal forms for the
sentences of $\ofoei(A)$, and syntactic characterizations of the fragments
whose sentences are monotone (respectively continuous) in some monadic predicate $a \in A$.

% \begin{theorem}
% \label{t:m2}
% A sentence of $\ofoei(A)$ is monotone in ${a\in A}$ iff it is equivalent to
% a sentence in the language given by the following syntax:
% \begin{equation}
% \label{eq-pofoei}
% \varphi ::= \psi \mid a(x) \mid \exists x.\varphi(x) \mid \forall x.\varphi(x)
% \mid \varphi \land \varphi \mid \varphi \lor \varphi
% \mid \qu x.\varphi(x) \mid \dqu x.\varphi(x)
% \end{equation}
% where $\psi \in \ofoei(A\setminus \{a\})$.

% A sentence of $\ofoei(A)$ is continuous in ${a\in A}$ iff it is equivalent to
% a sentence in the language given by the following syntax:
% \begin{equation}
% \label{eq-cofoei}
% \varphi ::=
% \psi \mid a(x) \mid \exists x.\varphi(x) \mid \varphi \land \varphi
%    \mid \varphi \lor \varphi \mid \wqu x.(\varphi,\psi)
% \end{equation}
% where $\psi \in \ofoei(A\setminus \{a\})$ and
% $\wqu x.(\varphi,\psi) :=
%    \forall x.(\varphi(x) \lor \psi(x)) \land \dqu x.\psi(x)$.
% \end{theorem}

% Using these characterizations, we can now provide the definition of a
% \wmso-automaton.

% \begin{definition}
% We denote by $\yvmonot{\ofoei}(A)$ the set of sentences that are positive
% in each $a \in A$ (i.e., belong to \eqref{eq-pofoei} for all $a \in A$), and
% we let $\AutW{\ofoei}$ denote the class of structures of the form
% $\bbA = \struc{A, a_I, \tmap, \pmap}$ such that $A$ is a finite set of states,
% $a_I \in A$ is the initial state of $\bbA$, $\tmap: A \times C \to
% \yvmonot{\ofoei}{a}(A)$ is the transition map, and $\pmap: A\to \bbN$ is the
% parity map of $\bbA$.

% Such a structure is a \emph{\wmso-automaton} if $\tmap$ and $\pmap$ satisfy,
% for all states $a,a'$ belonging
% to the same strongly connected component, the following two
% conditions:
% \begin{itemize}
% \item
% (weakness) $\pmap(a) = \pmap(a')$, and
% \item
% (continuity)
% if $\pmap(a')$ is even, then $\tmap(a',c)$
% belongs to the fragment \eqref{eq-cofoei}, for each $c \in \wp(\props)$
% \\ (and a dual condition applies in case $\pmap(a')$ is odd).
% \end{itemize}
% The class of these automata is denoted by $AutWC(\ofoei)$.
% \end{definition}

To finish, we give constructions transforming \wmso-formulas to
\wmso-automata and vice-versa, witnessing that
%
\begin{equation}
\label{eq:m3}
\wmso \equiv \AutWC(\ofoei) \text{ (over tree models)}.
\end{equation}

\subsection{Proof of main result.}

To conclude our introduction we briefly sketch the proof of our main result, 
Theorem~\ref{t:m1}(1).
Roughly speaking, we follow the bisimulation-invariance proof by Janin \& 
Walukiewicz, which revolves around relating two distinct types of automata, 
which correspond, respectively, to the logics $\mso$ and $\MC$.
More precisely, these two automaton types are given as $\AutWC(\ofoe)$ and
$\AutWC(\ofo)$, where the one-step languages are first-order 
logic respectively with and without equality.
What we will add to their proof is the insight from~\cite{Venxx} that the
required relation between $\AutWC(\ofoe)$ and $\AutWC(\ofo)$ already follows
from results relating the respective one-step languages.

In our setting, we need to identify automata corresponding to the fragment
$\mucML$.
For this purpose we introduce the class $\AutWC(\ofo)$ consisting of those
automata in $\AutWC(\ofo)$ that satisfy similar weakness and continuity 
conditions as the ones in $\AutWC(\ofoei)$:
\begin{equation}
\label{eq:autF}
\mucML \equiv \AutWC(\ofo) \text{ (over the class of all LTSs)}.
\end{equation}

As the key step in our proof then, we will provide a translation 
$(-)^{\bullet}: \ofoei \to \ofo$ which naturally induces a transformation 
$(-)^{\bullet}: \AutWC(\ofoei) \to \AutWC(\ofo)$.
As a consequence of the nice model-theoretic properties of the translation at 
the one-step level, the automaton transformation satisfies, for all 
transition systems $\bbT$:
\begin{equation}
\label{eq:crux-i}
\bbA^{\bullet} \text{ accepts } \bbT \text{ iff } \bbA \text{ accepts 
} \omegaunrav{\bbT}
\end{equation}
where $\omegaunrav{\bbT}$ is the $\omega$-unravelling of $\bbT$.
It easily follows from \eqref{eq:crux-i} that a \wmso-automaton $\bbA$
is bisimulation invariant iff $\bbA\equiv \bbA^{\bullet}$, and so 
Theorem~\ref{t:m1}(1) follows by \eqref{eq:m3} and \eqref{eq:autF}.

\subsection{Overview of paper.}
In the next section we give a precise definition of the preliminaries required to understand this article. In Section~\ref{sec:onestep} we define the one-step logics that will be used through the paper and give normal forms and syntactic characterizations of their monotone and (co-)continuous fragments. In Section~\ref{sec:aut} we formally define \wmso-automata and show that from every \wmso-formula we can construct an equivalent \wmso-automaton. In Section~\ref{sec:aut-to-formula_wmso} we prove the converse, that is, for every \wmso-automaton we can construct an equivalent \wmso-formula, this finishes the automata characterization of \wmso over tree models. Finally, in Section~\ref{sec:char} we prove the main result of the paper, namely that the fragment $\mucML$ is the bisimulation-invariant fragment of \wmso.


