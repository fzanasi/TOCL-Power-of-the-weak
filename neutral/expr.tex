% !TEX root = ../00CFVZ_TOCL.tex
% \section{Expressiveness modulo bisimilarity}\label{sec:expresso}

In this Section we use the tools developed in the previous parts to prove  the
main results of the paper, namely that both $M=\mucML, L=\wmso$ and  $M=\AFMC,
L=\nmso$ are  solutions of the equation $M \equiv L / {\bis}$, that is:

\begin{eqnarray}
% \[\begin{array}{rcl}
%\label{eq:1}
%\eqno{(*)}
   \AFMC &\equiv& \nmso / {\bis} 
   \label{eq:z11}
\\[1mm] \mucML &\equiv& \wmso / {\bis} 
   \label{eq:z12}
% \end{array}\]
\end{eqnarray}
% As explained in the introduction, 
The structure of the proof is the same for both statements. 

\begin{proofof}{\eqref{eq:z11} and \eqref{eq:z12}}
In both cases, we will need three steps to establish a link between the modal 
language on the left hand side of the equation to the bisimulation-invariant 
fragment of the second-order logic on the right hand side.

The first step is to connect the fragments $\AFMC$ and $\mucML$ of the modal 
$\mu$-calculus to, respectively, the weak and the continuous-weak automata for
first-order logic without equality.
That is, in Theorem~\ref{t:mlaut} below we prove the following:
\begin{eqnarray}
   \AFMC &\equiv& \AutW(\ofo)
   \label{eq:z21}
\\[1mm] \mucML &\equiv&  \AutWC(\ofo)
   \label{eq:z22}
\end{eqnarray}

Second, the main observations that we shall make in this section is that 
\begin{eqnarray}
\AutW(\ofo)  &\equiv& \AutW(\ofoe)/{\bis} 
   \label{eq:z31}
\\[1mm] 
\AutWC(\ofo) &\equiv&  \AutWC(\ofoei)/ {\bis} 
   \label{eq:z32}
\end{eqnarray}
That is, for \eqref{eq:z31} we shall see in Theorem~\ref{t:bi-aut} below that a
weak $\ofoe$-automaton $\bbA$ is bisimulation invariant iff it is equivalent to
a weak $\ofo$-automaton $\bbA^{\tmod}$ (effectively obtained from $\bbA$);
% which can be effectively obtained from $\bbA$ via a one-step translation applied
% to the transition map of $\bbA$
and similarly for \eqref{eq:z32}.

Finally, we use the automata-theoretic characterisations of $\nmso$ and $\wmso$
that we obtained in earlier sections:
\begin{eqnarray}
\AutW(\ofoe)   &\equiv&  \nmso
   \label{eq:z41}
\\[1mm] 
\AutWC(\ofoei) &\equiv&  \wmso 
   \label{eq:z42}
\end{eqnarray}

Then it is obvious that the equation \eqref{eq:z11} follows from \eqref{eq:z21},
\eqref{eq:z31} and \eqref{eq:z41}, while similarly 
\eqref{eq:z12} follows from \eqref{eq:z22}, \eqref{eq:z32} and \eqref{eq:z42}.
\end{proofof}

We finally verify equations \eqref{eq:z21} and \eqref{eq:z22}.

\begin{theorem}
\label{t:mlaut}
\begin{enumerate}
\item 
There is an effective construction transforming a formula $\phi \in \muML$ into
an equivalent automaton in $\Aut(\ofo)$, and vice versa.
\item 
There is an effective construction transforming a formula $\phi \in \AFMC$ into
an equivalent automaton in $\AutW(\ofo)$, and vice versa.
\item 
There is an effective construction transforming a formula $\phi \in \mucML$ into
an equivalent automaton in $\AutWC(\ofo)$, and vice versa.
\end{enumerate}
\end{theorem}
\begin{proof}
From Theorems \ref{t:autofor} and \ref{t:fortoaut}, we know that both $\mu\ofo \equiv \Aut(\ofo)$, $\mu_{D}\ofo \equiv \AutW(\ofo)$ and  $\mu_{C}\ofo \equiv  \AutWC(\ofo)$. It is therefore enough to verify that $\mu\ML \equiv \mu\ofo$, $\AFMC \equiv \mu_{D}\ofo$ and  $\mucML \equiv \mu_{C}\ofo$. This is an immediate consequence of a straightforward connection between $\ofo^+(A)$ and the one-step modal language $\oml^+(A)$ given by the following clauses:
\begin{align*}
	\varphi &::= \top \mid \bot \mid \varphi\land\varphi \mid \varphi\lor\varphi \mid \Diamond\alpha \mid \square \alpha \\
	\alpha &::= a \in A \mid \alpha\land\alpha \mid \alpha\lor\alpha.
\end{align*}
%where $a\in A$.

The formulas of $\oml^+(A)$ are interpreted over transition systems. A key observation is that, as the formulas of $\oml^+(A)$ have at most \emph{one level} of modalities, then we can restrict the valuation $\val$ to the successors of $s_I$, in the sense that, for every formula $\varphi\in\oml^+(A)$, $\model,\val \mmodels \varphi$ with $\val:A\to\wp(\moddom)$ if and only if there is $\val':A\to\wp(R[s_I])$ such that $\model,\val' \mmodels \varphi$.
 
To conclude the proof of theorem is therefore enough to verify the following claim.
\begin{claim}\label{claim:oml2ofo}
	There are translations $(-)^t:\oml^+(A) \to \ofo^+(A)$ and
	$(-)_t:\ofo^+(A) \to \oml^+(A)$ such that for every labeled transition system $\model$,
	valuation $\val:A \to \wp(R[s_I])$, and formulas $\varphi\in\oml^+(A)$ and $\psi \in \ofo^+(A)$ we have:
	%
	\begin{eqnarray}
		\model,\val \mmodels \varphi &\text{iff}& (R[s_I],\val) \models \varphi^t\label{eq:oml2ofo:1} \\
		\model,\val \mmodels \psi_t &\text{iff}& (R[s_I],\val) \models \psi.\label{eq:oml2ofo:2}
	\end{eqnarray}
\end{claim}
\begin{pfclaim}
	For $(-)^t:\oml^+(A) \to \ofo^+(A)$ we define $a^t := a(x)$ for the propositions, $(\Diamond\alpha)^t := \exists x.\alpha^t$ and $(\square\alpha)^t := \forall x.\alpha^t$ 
	for the modalities, and the remaining connectives as expected.

	The other translation is slightly more complicated, and we use the normal form for $\ofo$. Assume, by Theorem~\ref{th:onesteplogics-normalforms} that $\psi = \bigvee \posdbnfofo{\Pi}$ and recall that $\posdbnfofo{\Pi} = \bigwedge_{S\in\Pi} \exists x. \tau^+_S(x) \land \forall x. \bigvee_{S\in\Pi} \tau^+_S(x)$. We define the translation $(-)_t:\ofo^+(A) \to \oml^+(A)$ homomorphically on disjunction and conjunction and
	%
	\[
		(\posdbnfofo{\Pi})_t := \bigwedge_{S\in\Pi} \Diamond \bigwedge_{a\in S} a \land \square \bigvee_{S\in\Sigma} \bigwedge_{a\in S} a.
	\]
	%
	It is not difficult to see that these translations satisfy the above equations.
\end{pfclaim}
\end{proof}



\subsection{Bisimulation invariance, one step at a time}
\label{ss:bisinv}

In this subsection we will show how the bisimulation invariance results in this
paper can be proved by automata-theoretic means.
Following Janin \& Walukiewicz~\cite{jani:xxxxxx}, 
we will define a construction that, for $\llang \in \{{\ofoe},{\olque}\}$, 
transforms an arbitrary $\llang$-automaton $\bbA$ into an $\ofo$-automaton 
$\bbA^{\tmod}$ such that $\bbA$ is bisimulation invariant iff it is equivalent
to $\bbA^{\tmod}$.
In addition, we will make sure that this transformation preserves both the
weakness and the continuity condition.
The operation $(\cdot)^{\tmod}$ is completely determined by the following 
translation at the one-step level.
\begin{definition}
Recall from XXXX that any formula in ${\ofoe}^+(A)$ is equivalent to a 
disjunction of formulas of the form $\posdbnfofoe{\vlist{T}}{\Sigma}$, whereas
by XXXX any formula in ${\olque}^+(A)$ is 
equivalent to a disjunction of formulas of the form 
$\posdbnfolque{\vlist{T}}{\Pi}{\Sigma}$. 
Based on these normal forms, for both one-step languages $\llang={\ofoe}$ and 
$\llang={\olque}$, we define the translation 
$(-)^{\tmod} : {\llang}^+(A) \to \ofo^+(A)$ by setting
% \[
% \Big( \posdbnfofoe{\vlist{T}}{\Sigma} \Big)^{\tmod} = 
% \Big( \posdbnfolque{\vlist{T}}{\Pi}{\Sigma} \Big)^{\tmod} 
% \df
% \bigwedge_{i} \exists x_i. \tau^+_{T_i}(x_i) \land 
% \forall x. \bigvee_{S\in\Sigma} \tau^+_S(x),
% \]
\[
\left.\begin{array}{l}
   \Big( \posdbnfofoe{\vlist{T}}{\Sigma} \Big)^{\tmod} 
\\ \Big( \posdbnfolque{\vlist{T}}{\Pi}{\Sigma} \Big)^{\tmod} 
\end{array}\right\}
\df \bigwedge_{i} \exists x_i. \tau^+_{T_i}(x_i) \land 
\forall x. \bigvee_{S\in\Sigma} \tau^+_S(x),
\]
and for $\al = \bigvee_{i} \al_{i}$ we define $\al^{\tmod} \df \bigvee 
\al_{i}^{\tmod}$.
\end{definition}

\noindent
This definition propagates to the level of automata in the obvious way.

\begin{definition}
Let $\llang\in \{{\ofoe},{\olque}\}$ be a one-step language.
Given an automaton $\bbA = \tup{A,\tmap,\pmap,a_{I}}$ in $\Aut(\llang)$, define 
the automaton $\bbA^{\tmod} \df \tup{A,\tmap^{\tmod},\pmap,a_{I}}$ in 
$\Aut(\ofo)$ by putting, for each $(a,c) \in A \times C$:
\[
\tmap^{\tmod}(a,c) \df (\tmap(a,c))^{\tmod}.
\]
\end{definition}

The main result of this section is the theorem below.
For its formulation, recall that $\bbS^{\om}$ is the $\om$-unravelling of 
the model $\bbS$ (as defined in the preliminaries).
As an immediate corollary of this result, we see that \eqref{eq:z31} and
\eqref{eq:z32} hold indeed.

\begin{theorem}
\label{t:bi-aut}
Let $\llang\in \{{\ofoe},{\olque}\}$ be a one-step language and let $\bbA$ be an
$\llang$-automaton.

\begin{enumerate}
\item
The automata $\bbA$ and $\bbA^{\tmod}$ are related as follows, for every model $\bbS$:
\begin{equation}
\label{eq:crux}
\bbA^{\tmod} \text{ accepts } \bbS \text{ iff } \bbA \text{ accepts
} \bbS^{\om}.
\end{equation}
\item
The automaton $\bbA$ is bisimulation invariant iff $\bbA \equiv \bbA^{\tmod}$.
\item
If $\bbA\in \AutW(\llang)$ then $\bbA^{\tmod}\in \AutW(\ofo)$, and 
if $\bbA\in \AutWC(\ofoei)$ then $\bbA^{\tmod}\in \AutWC(\ofo)$.
\end{enumerate}
\end{theorem}


The remainder of this section is devoted to the proof of 
Theorem~\ref{t:bi-aut}.
The key proposition is the following observation on the one-step translation,
that we take from the companion paper~\cite{xxxxxx}.

\begin{proposition}
\label{p-1P}
Let $\llang\in \{{\ofoe},{\olque}\}$.
For every one-step model $(D,V)$ and every $\al \in \llang^+(A)$ we have
\begin{equation}
\label{eq-1cr}
(D,V) \models \alpha^{\tmod} \text{ iff } (D\times \om,V_\pi) \models \alpha,
\end{equation}
where $V_{\pi}$ % =  f^{-1} \circ V$
 is the induced valuation given by 
$V_{\pi}(a) \df \{ (d,k) \mid d \in V(a), k\in\omega\}$.
\end{proposition}

% \begin{proof}
% We prove the claim for $\llang={\olque}$, the other case being similar.
% Clearly it suffices to prove \eqref{eq-1cr} for formulas of the form
% $\al = \posdbnfolque{\vlist{T}}{\Pi}{\Sigma}$.
% \smallskip
% 
% \noindent\fbox{$\Rightarrow$} 
% Assume $(D,\val) \models \alpha^{\tmod}$, we will show that 
% $(D\times \omega,\val_\pi) \models \posdbnfolque{\vlist{T}}{\Pi}{\Sigma}$.
% Let $d_i$ be such that $\tau_{T_i}^+(d_i)$ in $(D,\val)$. 
% It is clear that the $(d_i,i)$ provide \emph{distinct} elements satisfying 
% $\tau_{T_i}^+((d_i,i))$ in $(D\times\omega,\val_{\pi})$ and therefore the 
% first-order existential part of $\alpha$ is satisfied. 
% With a similar but easier argument it is straightforward that the existential 
% generalized quantifier part of $\alpha$ is also satisfied.
% For the universal parts of $\posdbnfolque{\vlist{T}}{\Pi}{\Sigma}$ it is enough
% to observe that, because of the universal part of $\alpha^\circ$, \emph{every}
% $d\in D$ realizes a positive type in $\Sigma$. 
% By construction, the same applies to $(D\times\omega,\val_{\pi})$, 
% therefore this takes care of both universal quantifiers.
% \medskip
% 		
% \noindent\fbox{$\Leftarrow$} 
% Assuming that $(D\times \omega,\val_\pi) \models 
% \posdbnfolque{\vlist{T}}{\Pi}{\Sigma}$,
% we will show that $(D,\val) \models \alpha^\circ$. 
% The existential part of $\alpha^{\tmod}$ is trivial. 
% For the universal part we have to show that every element of $D$ realizes the 
% positive part of a type in $\Sigma$. 
% Suppose not, and let $d\in D$ be such that $\lnot\tau_S^+(d)$ for all $S\in 
% \Sigma$. 
% Then we have $(D\times\omega,\val_\pi) \not\models \tau_S^+((d,k))$ for all $k$.
% That is, there are infinitely many elements not realizing the positive part of 
% any type in $\Sigma$. 
% Hence we have $(D\times\omega,\val_\pi) \not\models \dqu y.\bigvee_{S\in\Sigma} 
% \tau_S^+(y)$. 
% Absurd, because that is part of $\posdbnfolque{\vlist{T}}{\Pi}{\Sigma}$.
% \end{proof}

\begin{proofof}{Theorem~\ref{t:bi-aut}}
The proof of the first part is based on a fairly routine comparison, based on
Proposition~\ref{p-1P}, of the acceptance games 
$\mathcal{A}(\bbA^{\tmod},\bbS)$ and $\mathcal{A}(\bbA,\bbS^{\om})$.
(In a slightly more general setting, the details of this proof can be found 
in~\cite{Venxx}.)

For part~2, the direction from right to left is immediate by Theorem \ref{t:mlaut}.
%the observation  by XXXX~\cite{xxxxxxxx} that $\muML \equiv \Aut{(\ofo)}$.
The opposite direction follows from the following equivalences:
\begin{align*}
\bbA \text{ accepts } \bbS
  & \text{ iff } \bbA \text{ accepts } \bbS^{\om}
  & \tag{$\bbA$ bisimulation invariant}
\\ & \text{ iff } \bbA^{\tmod} \text{ accepts } \bbS
  & \tag{equivalence~\eqref{eq:crux}}
\end{align*}

It remains to be checked that the construction $(-)^{\tmod}$, which has
been defined for arbitrary automata in $\Aut(\llang)$, transforms 
both $\wmso$-automata and $\nmso$-automata into automata of the right kind.
This can be verified by a straightforward inspection at the one-step level.
\end{proofof}

\begin{remark}{\rm
% As a corollary of the previous two propositions we find that 
% \begin{itemize}
% 	\itemsep 0 pt
% 	\item $\AutW(\ofo) \equiv \AutW(\ofoe)/{\bis}$, and
% 	\item $\AutWC(\ofo) \equiv \AutWC(\olque)/{\bis}$.
% \end{itemize}
In fact, we are dealing here with an instantiation of a more general phenomenon 
that is essentially coalgebraic in nature.
In~\cite{Venxx} it is proved that if $\llang$ and $\llang'$ are two one-step
languages that are connected by a translation $(-)^{\tmod}: \llang' \to 
\llang$ satisfying a condition similar to \eqref{eq-1cr}, then we find that 
$\Aut(\llang)$ corresponds to the bisimulation-invariant fragment of 
$\Aut(\llang')$: $\Aut(\llang) \equiv \Aut(\llang')/{\bis}$.
This subsection can be generalized to prove similar results relating
$\AutW(\llang)$ to $\AutW(\llang')$, and $\AutWC(\llang)$ to 
$\AutWC(\llang')$.
}\end{remark}
