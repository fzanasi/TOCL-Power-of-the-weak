We now see how to associate, with each one-step language $\llang$, the following
variant of the modal $\mu$-calculus.

\begin{definition}
Given a one-step language $\llang$, we define the language $\mu\llang$ of the 
\emph{$\mu$-calculus over $\llang$}  by the following grammar:
% \[
% \varphi ::= q \mid \neg\varphi \mid \varphi\lor\varphi 
%    \mid \nxt{\al}(\varphi_{1},\ldots,\varphi_{n})
%    \mid \mu p. \varphi',
% \]
\[
\varphi ::= 
   q \mid \neg q 
   \mid \varphi\lor\varphi \mid \varphi\land\varphi 
   \mid \nxt{\al}(\varphi_{1},\ldots,\varphi_{n})
   \mid \mu p. \varphi'    \mid \nu p. \varphi',
\]
where $p,q \in\props$, $\al(a_{1},\ldots,a_{n}) \in \llang^{+}$ and $\varphi'$ 
is monotone in $p$.

We will freely use standard syntactic concepts and notations related to this 
language, such as the sets $\FV(\phi)$ and $\BV(\phi)$ of \emph{free} and 
\emph{bound} variables of $\phi$, and the collection $\Sfor(\phi)$ of subformulas
of $\phi$.
\end{definition}

Observe that the language $\mu\llang$ generally has a wealth of modalities:
one for each one-step formula in $\llang$.

The semantics of this language is given as follows.

\begin{definition}
Let $\model$ be a transition system.
The satisfaction relation $\mmodels$ is defined in the standard way, with the 
following clause for the modality $\nxt{\alpha}$:
$$
\model \mmodels \nxt{\al}(\varphi_{1},\ldots,\varphi_{n})
\quad\text{iff}\quad 
(R[s_{I}],V_{\overline{\varphi}}) \models \al(a_{1},\ldots,a_{n}),
$$
where $V_{\overline{\varphi}}$ is the one-step valuation given by 
\[
V_{\overline{\varphi}}(a_{i}) \isdef 
  \{ t \in R[s_{I}] \mid \model.t \mmodels \varphi_{i}\}.
\]
\end{definition}

% \begin{example}
% 
% \end{example}

\btbs
\item
give some examples ($\Diamond$, $\Box$, $\Diamond^{\geq 2}$, 
$\Diamond^{\infty}$, \ldots)
\item
remark that $\nxt{\alpha\land\beta}(\ol{\phi})$ is equivalent to
$\nxt{\alpha}(\ol{\phi}) \land \nxt{\beta}(\ol{\phi})$
\item
remark about nnf \& closure under negation in case the one-step 
language is closed under taking boolean duals.
\etbs

Alternatively but equivalently, one may interpret the language
game-theoretically.

\begin{definition}
Given a $\mu\llang$-formula $\phi$ and a model $\model$ we define the 
\emph{evaluation game} $\egame(\varphi,\model)$ as the two-player infinite
game of which the rules are given in the table below.
% Table~\ref{tab:EGL}.
For the admissible moves at a position of the form 
$(\nxt{\al}(\varphi_{1},\ldots,\varphi_{n}),s)$, we consider the valuation 
$V^{*}_{Z}: \{ a_{1}, \ldots, a_{n} \} \to \wp(R[s])$, given by
$V^{*}_{Z}(a_{i}) \isdef \{ t \in R[s] \mid (\phi_{i},t) \in Z \}$.
%
\begin{table}[htb]
\centering
\begin{tabular}{|l|c|l|c|}
\hline
Position & Player & Admissible moves
\\\hline
    $(q,s)$, with $q \in \FV(\phi) \cap \tscolors(s)$ 
  & $\abelard$ 
  & $\emptyset$
\\  $(q,s)$, with $q \in \FV(\phi) \setminus \tscolors(s)$ 
  & $\eloise$ & $\emptyset$
\\  $(\lnot q,s)$, with $q \in \FV(\phi) \cap \tscolors(s)$ 
  & $\eloise$ 
  & $\emptyset$
\\  $(\lnot q,s)$, with $q \in \FV(\phi) \setminus \tscolors(s)$ 
  & $\abelard$ 
  & $\emptyset$
\\ $(\psi_1 \lor \psi_2,s)$ 
  & $\eloise$ 
  & $\{(\psi_1,s),(\psi_2,s) \}$ 
\\  $(\psi_1 \land \psi_2,s)$ 
  & $\abelard$ 
  & $\{(\psi_1,s),(\psi_2,s) \}$ 
\\  $(\nxt{\al}(\varphi_{1},\ldots,\varphi_{n}),s)$ 
  & $\eloise$ 
  & $\{ Z \sse \{ \varphi_{1},\ldots,\varphi_{n} \} \times R[s]
     \mid (R[s],V^{*}_{Z}) \models \al(\ol{a}) \}$ 
\\  $Z \sse  \Sfor(\phi) \times S$
  & $\abelard$
  & $\{ (\psi, s) \mid (\psi,s) \in Z \}$
\\  $(\mu p.\varphi,s)$ & $-$ & $\{(\varphi,s) \}$ 
\\  $(\nu p.\varphi,s)$ & $-$ & $\{(\varphi,s) \}$ 
\\  $(p,s)$, with $p \in \FV(\phi)$ & $-$ & $\{(\delta_p,s) \}$ \\
  \hline
\end{tabular}
% \caption{Evaluation game for $\mu\llang$}
% \caption{}
\label{tab:EGL}
\end{table}
The winning conditions of $\egame(\varphi,\model)$ are standard: $\eloise$ wins
those infinite matches of which the highest variable that is unfolded infinitely
often during the match is a $\mu$-variable.
\end{definition}

The following proposition, 
stating the adequacy of the evaluation game for the semantics of $\mu\llang$,
is formulated explicitly for future reference.
We omit the proof, which is completely routine.

\begin{fact}[Adequacy]
\label{f:adeqmu}
For any formula $\phi \in \mu\llang$ and any model $\model$ the following 
equivalence holds:
\[
\model \mmodels \phi
\quad\text{iff}\quad 
(\phi,s_{I}) \text{ is a winning position for $\eloise$ in } 
\egame(\varphi,\model).
\]
\end{fact}

We will be specifically interested in two fragments of $\mu\llang$, associated 
with the properties of being noetherian and continuous, respectively, and with 
the associated variants of the $\mu$-calculus $\mu\llang$ where the use of the 
fixpoint operator $\mu$ is restricted to formulas belonging to these two
respective fragments.

\begin{definition}
Let $\qprops$ be a set  of proposition letters.
We first define the fragment $\noe{\mu\llang}{\qprops}$ of $\mu\llang$ of 
formulas that are syntactically \emph{noetherian} in $\qprops$ by the following 
grammar:
\begin{equation*}
   \varphi ::= q
   \mid \psi
   \mid \varphi \lor \varphi
   \mid \varphi \land \varphi
   \mid \nxt{\al}(\varphi_{1},\ldots,\varphi_{n})
   \mid \mu p.\phi'
\end{equation*}
where $q \in \qprops$, $\psi$ is a $\qprops$-free $\MC$-formula,
$\al(a_{1},\ldots,a_{n}) \in \llang^{+}$ and 
$\phi' \in \noe{\mu\llang}{\qprops\cup\{p\}}$. 
The \emph{co-noetherian} fragment $\conoe{\mu\llang}{Q}$ is defined dually.

Similarly, we define the fragment $\cont{\mu\llang}{\qprops}$ of 
$\mu\llang$-formulas that are syntactically \emph{continuous} in $\qprops$ as
follows:
\begin{equation*}
   \varphi ::= q
   \mid \psi
   \mid \varphi \lor \varphi
   \mid \varphi \land \varphi
   \mid 
   \nxt{\al}(\varphi_{1},\ldots,\varphi_{k},\psi_{1},\ldots,\psi_{m})
   \mid \mu p.\phi'
\end{equation*}
where $p\in\props$, $q \in \qprops$, $\psi$, $\psi_{i}$ are $\qprops$-free 
$\MC$-formula, $\al(a_{1},\ldots,a_{k},b_{1},\ldots,b_{m}) \in 
\cont{\llang}{\ol{a}}(\ol{a},\ol{b})$,
and $\phi' \in \cont{\mu\llang}{\qprops\cup\{p\}}$. 
The \emph{co-continuous} fragment $\cocont{\mu\llang}{Q}$ is defined dually.
\end{definition}

Based on this we can now define the mentioned variants 
% $\mu_{D}\llang$ and $\mu_{C}\llang$ 
of the $\mu$-calculus $\mu\llang$ where the use of the least (greatest) 
fixpoint operator can only be applied to formulas that belong to, 
respectively, the noetherian (co-noetherian) and continuous (co-continuous)
fragment of the language that we are defining.

\begin{definition}
The formulas of the \emph{alternation-free} $\mu$-calculus $\mu_{D}\llang$ 
are defined by the following grammar:
\begin{equation*}
   \varphi ::= 
      q \mid \neg q 
   \mid \varphi\lor\varphi \mid \varphi\land\varphi 
   \mid \nxt{\al}(\varphi_{1},\ldots,\varphi_{n})
   \mid \mu p. \varphi'    
   \mid \nu p. \varphi'',
\end{equation*} 
where $\al(a_{1},\ldots,a_{n}) \in \llang^{+}$,
$\phi' \in \mu_{D}\llang \cap \noe{\mu\llang}{p}$
and dually $\phi'' \in \mu_{D}\llang \cap \conoe{\mu\llang}{p}$.

Similarly, the formulas of the \emph{continuous} $\mu$-calculus $\mu_{C}\llang$
are given by the grammar
\begin{equation*}
   \varphi ::= 
      q \mid \neg q 
   \mid \varphi\lor\varphi \mid \varphi\land\varphi 
   \mid \nxt{\al}(\varphi_{1},\ldots,\varphi_{n})
   \mid \mu p. \varphi'    
   \mid \nu p. \varphi'',
\end{equation*} 
where $\al(a_{1},\ldots,a_{n}) \in \llang^{+}$,
$\phi' \in \mu_{C}\llang \cap \cont{\mu\llang}{p}$
and dually $\phi'' \in \mu_{C}\llang \cap \cocont{\mu\llang}{p}$.
\end{definition}


%%%
