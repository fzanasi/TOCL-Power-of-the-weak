% \newpage

\subsection{Fixpoints of continuous maps}

It is well-known that continuous functionals are \emph{constructive}.
That is, if we construct the least fixpoint of a continuous functional $F: 
\wp(S) \to \wp(S)$ using the ordinal approximation $\nada, F\nada, F^2\nada, 
\ldots, F^{\al}\nada, \ldots$, then we reach convergence after at most $\omega$
many steps, implying that $\LFP. F = F^{\omega}\nada$.
We will see now that this fact can be strengthened to the following observation,
which is the crucial result needed in the proof of Proposition~\ref{p:keyfix}.

\begin{theorem}
\label{t:fixcont}
Let $F: \wp(S)\to \wp(S)$ be a continuous functional.
Then for any $s \in S$:
\begin{equation}
\label{eq:fixcont}
s \in \LFP. F \text{ iff }
s \in \LFP.F\rst{X}, \text{ for some finite } X \subseteq S.
\end{equation}
\end{theorem}

\begin{proof}
The direction from right to left of \eqref{eq:fixcont} is a special case of 
Proposition~\ref{p:rstfix}.

Clearly, by definition of $F\rst{X}$ it holds that $F\rst{X}(U) \subseteq F(U)$, 
for all subsets $U \subseteq X$.
Then a routine proof by transfinite induction shows that 
$(F\rst{X})^{\alpha}(\nada) \subseteq F^{\alpha}(\nada)$, for all ordinals 
$\alpha$.
But from this it is immediate that $\LFP.F\rst{X} \subseteq \LFP. F$.

For the opposite direction of \eqref{eq:fixcont} a bit more work is needed.
Assume that $s \in \LFP. F$; we claim that there are sets $U_{1},\ldots,U_{n}$,
for some $n \in \omega$, such that $s \in U_{n}$, $U_{1} \sse_{\omega} F(\nada)$,
and $U_{i+1} \sse_{\omega} F(U_{i})$, for all $i$ with $1 \leq i < n$.

To see this, first observe that since $F$ is continuous, we have $\LFP. F = 
F^{\omega}(\nada) = \bigcup_{n\in\omega}F^{n}(\nada)$, and so we may take $n$ to
be the least natural number such that $s \in F^{n}(\nada)$.
By a downward induction we now define sets $U_{n},\ldots,U_{1}$, with $U_{i} 
\sse F^{i}(\nada)$ for each $i$. 
We set up the induction by putting $U_{n} \mathrel{:=} \{ s \}$, then $U_{n}
\sse F^{n}(\nada)$ by our assumption on $n$.
For $i<n$, we define $U_{i}$ as follows.
Using the inductive fact that $U_{i+1} \sse_{\omega} F^{i+1}(\nada) = 
F(F^{i}(\nada))$, it follows by continuity of $F$ that for each $u \in U_{i+1}$
there is a set $V_{u} \sse_{\omega} F^{i}(\nada)$ such that $u \in F(V_{u})$.
We then define $U_{i} \mathrel{:=} \bigcup \{ V_{u} \mid u \in U_{i+1} \}$,
so that clearly $U_{i+1} \sse_{\omega} F(U_{i})$ and $U_{i} \sse_{\omega}
F^{i}(\nada)$.
Continuing like this, ultimately we arrive at stage $i=1$ where we find
$U_{1} \sse F(\nada)$ as required.

Finally, given the sequence $U_{n},\ldots,U_{1}$, we define 
\[
X \mathrel{:=} \bigcup_{0<i\leq n} U_{i}.
\]
It is then straightforward to prove that $U_{i} \sse \LFP. F\rst{X}$, for each 
$i$ with $0<i\leq n$, and so in particular we find that $s \in U_{n} \sse \LFP.
F\rst{X}$.
This finishes the proof of the implication from left to right in 
\eqref{eq:fixcont}.
\end{proof}

\noindent
As an almost immediate corollary of this result we obtain the second part of 
Proposition~\ref{p:keyfix}.

\begin{proofof}{Proposition~\ref{p:keyfix}(2)}
Take an arbitrary formula $\mu p. \phi \in \mu_{C}\ofoei$, then by definition 
we have $\phi \in \cont{\mu_{C}\ofoei}{p}$.
But it follows from a routine inductive proof that every formula $\psi \in 
\cont{\mu_{C}\ofoei}{\qprops}$ is continuous in each variable in $\qprops$.
Thus $\phi$ is continuous in $p$, and so the result is immediate by 
Theorem~\ref{t:fixcont}.
\end{proofof}

% \newpage