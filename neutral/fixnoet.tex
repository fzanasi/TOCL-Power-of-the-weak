\subsection{Fixpoints of noetherian maps}

We will now see how to prove Proposition~\ref{p:keyfix}(1), which is the key 
result that we need to embed alternation-free $\mu$-calculi such as 
$\mu_{D}\ofoe$ and $\mu_{D}\ML$ into noetherian second-order logic.
Perhaps suprisingly, this case is slightly more subtle than the characterisation of
fixpoints of continuous maps.

%We start with an easy observation concerning bundles.
%
%\begin{proposition}
%\label{p:bundle}
%Let $\model$ be an LTS. If $\mathcal{B}$ is a collection of $s$-bundles, 
%then $\bigcup \mathcal{B}$ is also an $s$-bundle.
%\end{proposition}

We start with stating some auxiliary definitions and results on monotone 
functionals, starting with a game-theoretic characterisation of their least
fixpoints~\cite{Ven08}.

\begin{definition}
\label{d:unfgame}
Given a monotone functional $F: \wp(S)\to \wp(S)$ we define the 
\emph{unfolding game} $\UG_{F}$ as follows:
\begin{itemize}
\item at any position $s \in S$, $\eloise$ needs to pick a set $X$ such that 
$s \in FX$;
\item at any position $X \in \wp(S)$, $\abelard$ needs to pick an element of 
$X$
\item all infinite matches are won by $\abelard$.
\end{itemize}
A positional strategy $\ystrat: S \to \wp(S)$ for $\eloise$ in $\UG_{F}$ is 
\emph{descending} if, for all ordinals $\alpha$,
\begin{equation}
\label{eq:dec}
s \in F^{\alpha+1}(\nada) \text{ implies } \ystrat(s) \sse F^{\alpha}(\nada).
\end{equation}
\end{definition}

It is not the case that \emph{all} positional winning strategies for $\eloise$ 
in $\UG_{F}$ are descending, but the next result shows that there always is one.

\begin{proposition}
\label{p:unfg}
Let $F: \wp(S)\to \wp(S)$ be a monotone functional.
Then
\\(1) for all $s \in S$, $s \in \win_{\eloise}(\UG_{F})$ iff $s \in \LFP. F$;
\\(2) if $s \in \LFP. F$, then \eloise has a descending winning strategy in 
$\UG_{F}@s$.
\end{proposition}

\begin{proof}
Point (1) corresponds to \cite[Theorem 3.14(2)]{Ven08}.
For part (2) one can simply take the following strategy.
Given $s \in \LFP.F$, let $\alpha$ be the least ordinal such that $s \in 
F^{\alpha}(\nada)$; it is easy to see that $\alpha$ must be a successor ordinal,
say $\alpha = \beta + 1$. 
Now simply put $\ystrat(s) := F^{\beta}(\nada)$.
\end{proof}

\begin{definition}
\label{d:str-tree}
Let $F: \wp(S)\to \wp(S)$ be a monotone functional, let $\ystrat$ be a 
positional winning strategy for $\eloise$ in $\UG_{F}$, and let $r \in S$. 
Define $T_{\ystrat,r} \sse S$ to be the set of states in $S$ that are 
$\ystrat$-reachable in $\UG_{F}@r$.
This set has a tree structure induced by the map $\ystrat$ itself, where the 
children of $s \in T_{\ystrat,r}$ are given by the set $\ystrat(s)$; we will
refer to $T_{\ystrat,r}$ as the \emph{strategy tree} of
$\ystrat$.
\end{definition}
Note that a strategy tree $T_{\ystrat,r}$ will have no infinite paths, since we
define the notion only for a \emph{winning} strategy $\ystrat$.

\begin{proposition}
\label{p:afmc-2}
Let $F: \wp(S)\to \wp(S)$ be a monotone functional, let $r \in S$, and let 
$\ystrat$ be a descending winning strategy for $\eloise$ in $\UG_{F}$.
Then
\begin{equation}
\label{eq:afmc3}
r \in \LFP. F \text{ iff } r \in \LFP. F\rst{T_{\ystrat,r}}.
\end{equation}
\end{proposition}

\begin{proof}
Let $F,r$ and $\ystrat$ be as in the formulation of the proposition.
The direction from right to left in \eqref{eq:afmc3} is immediate by
Proposition~\ref{p:rstfix}.

For the opposite direction, it clearly suffices to show that for all ordinals 
$\alpha$ we have
\begin{equation}
\label{eq:unf1}
F^{\al}(\nada) \cap T_{\ystrat,r} \sse (F\rst{T_{\ystrat,r}})^{\alpha}(\nada).
\end{equation}
We will prove \eqref{eq:unf1} by transfinite induction.
The base case, where $\alpha = 0$, and the inductive case where $\alpha$ is a 
limit ordinal are straightforward, so we focus on the case where $\alpha$ is a 
successor ordinal, say $\alpha = \beta +1$.
Take an arbitrary state $u \in F^{\beta+1}(\nada) \cap T_{\ystrat,r}$, then we 
find $\ystrat(u) \sse F^{\beta}(\nada)$ by our assumption \eqref{eq:dec}, and 
$\ystrat(u) \sse T_{\ystrat,r}$ by definition of $T_{\ystrat,r}$.
Then the induction hypothesis yields that 
$\ystrat(u) \sse (F\rst{T_{\ystrat,r}})^{\beta}(\nada)$, and so we have 
$\ystrat(u) \sse (F\rst{T_{\ystrat,r}})^{\beta}(\nada) \cap T_{\ystrat,r}$.
But since $\ystrat$ is a positional winning strategy, and $u$ is a winning 
position for $\eloise$ in $\UG_{F}$ by Claim~\ref{p:unfg}(i), $\ystrat(u)$ is a
legitimate move for $\eloise$, and so we have $u \in F(\ystrat(u))$.
Thus by monotonicity of $F$ we obtain $u \in 
F((F\rst{T})^{\beta}(\nada) \cap T_{\ystrat,r})$, and since $u \in T_{\ystrat,r}$ 
by assumption, this means that $u \in (F\rst{T_{\ystrat,r}})^{\beta+1}(\nada)$ as 
required.
\end{proof}

We now turn to the specific case where $F$ is induced by some least fixpoint 
formula $\phi(p) \in \mu\mathcal{L}_1$ on some LTS $\model$.   
By Proposition \ref{p:unfg} and Fact~\ref{f:adeqmu}, we have that $\eloise$ has
a winning strategy in  $\egame(\mu p.\phi(p),\model)@(\mu p.\phi(p),s)$ if and only if she has one in $\UG_{F}@s$ too, where $F := \phi_{p}^{\bbS}$ is the monotone functional defined by $\phi(p)$. The next Proposition  makes this correspondence explicit when $\mathcal{L}_1=\foe$. 

First, we need to introduce some auxiliary concepts and notations.
Given  a winning strategy   $\ystrat$ for $\eloise$ in $\egame(\mu p. \phi,\model)@(\mu p. \phi,s)$, 
let $B(\ystrat)$ be the set of all finite $\ystrat$-guided, possibly partial, 
matches in  $\egame(\psi,\model)@(\psi,s)$ in which no position of the form 
$(\nu q. \psi, r)$ is visited. 
Hence, $\eloise$ is said to have \emph{compatible} positional winning strategies
$\chi$ in $\UG_F@s$ and  $\chi'$  in $\egame(\mu p.\phi,\bbS)@(\mu p.\phi,s)$ if
each point in $T_{\chi,s}$ lies on some path belonging to $B^\model(\chi')$. 

\begin{proposition}\label{p:unfold=evalgame2}
%Player $\exists$ has a winning strategy $\chi$ in $\UG_F@(s,\phi(p))$ if and only she has winning strategy $\chi'$ in 
%$\egame(\mu p.\phi,\bbS)@(s, \mu p.\phi)$. 
%Moreover it holds that each $r \in  T_{\chi,s}$ lies in some path in $B^\model(\chi')$.
Let %$s \in \mng{\mu p.\phi(p)}^{\model}$, with 
$\phi(p) \in \mu_{D}\foe{p}$. Then 
\\(i) for every positional winning strategy $\chi$ for $\eloise$ in $\UG_F@s$,
there is a compatible positional winning strategy $\chi'$ for her in 
$\egame(\mu p.\phi,\bbS)@(\mu p.\phi,s)$;
\\(ii) for every positional winning strategy $\chi'$ for $\eloise$ in 
$\egame(\mu p.\phi,\bbS)@(\mu p.\phi,s)$, there is a compatible  positional 
winning strategy $\chi'$ for her in $\UG_F@s$.
\end{proposition}

\begin{proof}
For point (i) we reason as follows. 
Let $\chi: S \to \wp{(S)}$  be a positional winning strategy for $\eloise$ in $\UG_F@s$. 
Hence, for every $t \in \win_{\eloise}(\UG_{F})$, it holds that $t \in F(\chi(t))$ 
and therefore that there is a positional winning strategy $\chi_t$ for $\eloise$ 
in the game $\egame(\phi,\bbS[p \mapsto \chi(s)])$ starting at $(\phi,t)$. 
%\btbs
%\item
%YV: I get confused with the T,t and $\chi, s$.
%\etbs
Let $\chi'$ be the following strategy for $\eloise$ in
$\egame(\mu p.\phi,\bbS)@(\mu p.\phi,s)$:
\begin{enumerate}
\item after the initial automatic move, the position of the match is $(\phi,s)$; 
$\eloise$ first plays her strategy $\chi_s$;
\item each time a position $(p,t)$ is reached, 
\begin{enumerate}
%necessarily 
\item if $t \in \win_{\eloise}(\UG_{F})$, then $\eloise$ continues with $\chi_t$;
\item if $t \notin \win_{\eloise}(\UG_{F})$, then $\eloise$ continues with a random positional strategy.
\end{enumerate}
\end{enumerate}
We verify that this positional strategy is actually a compatible positional winning strategy for her.

First of all, notice that whenever by applying $\chi'$ a position of the form $(p,t)$ is reached,  necessarily $t \in \win_{\eloise}(\UG_{F})$. This can be proved by induction on the number of position of the form $(p,t)$ visited during a $\chi'$-guided match. For the inductive step, assume $w \in \win_{\eloise}(\UG_{F})$. Hence $\chi_w$ is winning for $\eloise$ in  $\egame(\phi,\bbS[p \mapsto \chi(w)])@(\phi,w)$. This means that if a position of the form $(p, t)$ is reached, the variable $p$ must be true at $t$ in the model $\bbS[p \mapsto \chi(w)]$, meaning that it belongs to the set $\chi(w)$. By assumption $\chi$ is a winning strategy for $\eloise$ in $\UG_F$, and therefore any element of $\chi(w)$ is again a member of $ \win_{\eloise}(\UG_{F})$. 
Finally, let $\pi$ any $\chi'$-guided match $\egame(\phi,\bbS[p \mapsto \chi(w)])@(\phi,w)$. We verify that $\pi$  is winning for $\eloise$. First notice that since $\chi$ is winning for her in $\UG_F@s$, the fixpoint variable $p$ is unfolded only finitely many often in $\pi$. 
Let $(p,t)$ be the last basic position in $\pi$ where $p$ occurs. Then from now on $\chi'$ and $\chi_t$ coincide, yielding  that the match is winning for $\eloise$. Clearly $\chi'$ and $\chi$ are compatible.


For point (ii), let $\chi'$ be a winning strategy for $\eloise$ in
$\egame(\mu p.\phi(p),\model)@(\mu p.\phi(p),s)$.  Let $B(\chi')$ be the set of all finite $\chi'$-guided partial matches in $\egame$, $\pi\in B(\chi')$ and define \[P_{\last(\pi)}:=\{ (p,w) \mid \exists \pi' \in B(\chi'), \pi'=\pi \rho(p,w), p\text{ does not occur in }\rho\}.\]

Remember that for every $\pi, \pi' \in B(\chi')$ such that $\last(\pi)=\last(\pi')$,  by positionality of $\chi'$ it holds that $\pi\rho \in B(\chi')$ if and only if $\pi'\rho \in B(\chi')$, for every finite match $\rho$.

Hence, consider  the positional strategy $\ystrat$ in $\UG_{F}@s$ 
defined by \[\chi(t)= \{ w \mid (p,w) \in P_{\last(\pi)}, \pi\in B(\chi'), \last(\pi)=(\mu p. \phi(p), t)\}.\]


%While playing the unfolding game $\UG$, $\eloise$ builds an 
%$\chi'$-guided `shadow match' $\egame(\mu p.\phi(p),\model)@(\mu p.\phi(p),s)$ 
%such that the positions in the $\UG$-match of the form $t$ exactly correspond to
%the positions of the form $(\phi(p),t)$ in the evaluation game $\egame$.

Compatibility being immediate, it suffices to verify that $\ystrat$ is winning for $\eloise$. First of all, since $\chi'$ is winning, $B(\chi')$ does not contain an infinite 
ascending chain of $\chi'$-guided matches, and thence any  $\chi$-guided match in $\UG_{F}@s$ is finite. It therefore remains to verify that for every $\chi$-guided match $\pi$ in $\UG_{F}@s$ such that $\last(\pi)$ is an $\eloise$ position, she can always move. But this is immediate since 
\begin{itemize}
\item $(\mu p. \phi(p), s) \in \last(B(\chi'))$, 
\item if $(\mu p. \phi(p), t) \in \last(B(\chi'))$, then %$(\phi(p), t) \in \last(B(\chi'))$ and  
$\chi'$ is (naturally induces) a winning strategy for $\eloise$ in $\egame(\phi(p),\bbS[p \mapsto \chi(t)])@(\phi(p),t)$, and thus in particular $t \in F(\chi(t))$,
\item if $(p, t) \in \last(B(\chi'))$ then $(\mu p. \phi(p), t) \in \last(B(\chi'))$,
\end{itemize}
%\btbs
%\item YV: some more detail here?
%\etbs
where $\last(B(\chi')):=\{\last(\pi) \mid \pi \in B(\chi')\} \subseteq \win_{\eloise}(\egame(\mu p.\phi(p),\model))$.
\end{proof}

%Notice that the evaluation game for the modal $\mu$ calculus can be easily adapted to $\mu_{D}\foe$ and provide a corresponding Adequacy Theorem (Proposition \ref{p:unfold=evalgame}). In particular, we would have that 
%$\eloise$ has a winning strategy in  $\egame(\mu p.\phi(p),\model)@(\mu p.\phi(p),s)$ if and only if $s \in \mng{\mu p.\phi(p)}^{\model}$, with $\phi(p) \in \mu_{D}\foe{p}$.

\begin{proofof}{Proposition~\ref{p:keyfix}(1)}
%    
%\btbs
%\item
%the argument below needs to be thoroughly checked.
%\etbs
Let $\model$ be an LTS and $\phi(p) \in \mu_{D}\ofoe{p}$. 

The right-to-left direction of \eqref{eq:foe-d} being proved by 
Proposition \ref{p:rstfix}, we check the left-to-right direction.
We first verify that winning strategies in evaluation games for noetherian fixpoint formulas naturally induce bundles. More precisely:

%Given  a winning strategy   $\ystrat$ for $\eloise$ in $\egame(\mu p. \phi,\model)@(\mu p. \phi,s)$,  let $B(\ystrat)$ be the set of all finite $\ystrat$-guided, possibly partial, matches in  $\egame(\psi,\model)@(\psi,s)$ in which a position of the form $(\nu q. \psi, r)$ is not visited.

\begin{claim}\label{p:strategybundledEv}
Let $B^\model(\ystrat)$ be the projection of $B(\ystrat)$ on $S$, that is the set of all paths in $\model$ that are a projection on $S$ of a $\ystrat$-guided (partial) match in $B(\ystrat)$. Then $B^\model(\ystrat)$ is a bundle.
\end{claim}
\begin{pfclaim}
Assume towards a contradiction that $B^\model(\ystrat)$ contains an infinite 
ascending chain $\pi_{0} \sqsubset \pi_{1} \sqsubset \cdots$. 
Let $\pi$ be its limit and consider the set of elements in $B(\ystrat)$ that,
projected on $S$, are prefixes of $\pi$. 
By  K\"{o}nig's Lemma, this set contains an infinite ascending chain whose 
limit is an infinite $\ystrat$-guided match in $\egame(\mu p. \phi,\model)$
which starts at $(\mu p. \phi,s)$, and of which $\pi$ is the projection on $S$.
By definition of $B(\ystrat)$,  the highest bound variables of $\mu p. \phi$ 
that get unravelled infinitely often in $\rho$ is a $\mu$-variable, meaning that
the match is winning for $\abelard$, a contradiction.
\end{pfclaim}

Then, let $F := \phi_{p}^{\bbS}$ be the monotone functional defined by $\phi(p)$.
Assume that $s \in \mng{\mu p.\phi}^{\model}$. This means that $s \in \LFP. F$, and by Proposition~\ref{p:afmc-2}, we obtain that $s \in \LFP. F\rst{T_{\ystrat,r}}$. We verify that $T_{\chi, s}$ is noetherian.
Applying Proposition~\ref{p:unfg} to the fact that $s \in \LFP. F$ yields that $\eloise$ has a descending winning strategy $\chi$
in $\UG_{F}@s$. 
By Proposition \ref{p:unfold=evalgame2}, $\eloise$ has a winning strategy $\chi'$ in $\egame(\mu p.\phi,\bbS)@(\mu p.\phi,s)$ compatible with $\chi$. Because of compatibility, every node in $T_{\chi, s}$ lies on some path of $B^\model(\ystrat)$. From Claim \ref{p:strategybundledEv}, we known that $B^\model(\ystrat)$ is a bundle, meaning that  $T_{\chi, s}$ is noetherian as required.   
%\btbs
%\item
%YV: some more detail here?
%\etbs
\end{proofof}
