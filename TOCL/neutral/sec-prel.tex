\section{Preliminaries}
   \label{sec:prel}

We assume the reader to be familiar with the syntax and (game-theoretic)
semantics of the modal $\mu$-calculus and with the automata-theoretic 
perspective on this logic.
For background reading we refer to~\cite{ALG02,Ven08}; the purpose of this 
section is to fix some notation and terminology. 

\subsection{Transition systems and trees} 
\label{ssec:prelim_trees}

Throughout this article we fix a set $\Prop$ of elements that will be called
\emph{proposition letters} and denoted with small Latin letters $p, q, \ldots$ .
We will often focus on a finite subset $\pprop \subseteq_{\om} \Prop$, and
denote with $C$ the set $\pow (\pprop)$ of \emph{labels} on $\pprop$; it will
be convenient to think of $C$ as an \emph{alphabet}.
Given a binary relation $R \subseteq X \times Y$, for any element $x \in X$,
we indicate with $R[x]$ the set $\{ y \in Y \mid (x,y) \in R \}$ while 
$R^+$ and $R^{*}$ are defined respectively as the transitive closure of~$R$ and
the reflexive and transitive closure of~$R$. 
The set $\Ran(R)$ is defined as $\bigcup_{x\in X}R[x]$.

A \emph{$\pprop$-labeled transition system} (LTS) is a tuple $\bbS = 
\tup{T,R,\tscolors,s_I}$ where $T$ is the universe or domain of $\bbS$, 
$\tscolors:T\to\pow(\pprop)$ is a colouring (or marking),
$R\subseteq T^2$ is the accessibility relation and $s_I \in T$ is a 
distinguished node.
% We use $|\bbS|$ to denote the domain of $\bbS$.
We call $\tscolors(s)$ the colour, or type, of node $s \in T$.
Observe that the colouring %marking 
${\tscolors:T\to\pow(\pprop)}$ can be seen as a 
valuation $\tsval:\pprop\to\pow (T)$ given by $\tsval(p) \isdef \{s \in T \mid
p\in \tscolors(s)\}$. 
%
A \emph{$\pprop$-tree} is a $\pprop$-labeled LTS in which every node can
be reached from $s_I$, and every node except $s_I$ has a unique predecessor;
the distinguished node $s_I$ is called the \emph{root} of $\bbS$.
Each node $s \in T$ uniquely defines a subtree of $\bbS$ with carrier
$R^{*}[s]$ and root $s$. We denote this subtree by ${\bbS.s}$.

The \emph{tree unravelling} of an LTS $\bbS$ is given by $\unravel{\bbS} 
\isdef \tup{T_P,R_P,\tscolors',s_I}$ where $T_P$ is the set of finite paths in 
$\bbS$ stemming from $s_I$, $R_P(t,t')$ iff $t'$ is a one-step extension of $t$ 
and the colour of a path $t\in T_P$ is given by the colour of its last node in
$T$.
The \emph{$\omega$-unravelling} $\omegaunrav{\bbS}$ of $\bbS$ is an 
unravelling which has $\omega$-many copies of each node different from the root.

A \emph{$p$-variant} of a transition system $\bbS = \tup{T,R,\tscolors,s_I}$
is a $\pprop\cup\{p\}$-transition system $\tup{T,R,\tscolors',s_I}$
such that $\tscolors'(s)\setminus\{p\} = \tscolors(s) \setminus \{p \}$ for all 
$s \in T$.
Given a set $S \subseteq T$, we let $\bbS[p\mapsto S]$ denote the $p$-variant
where $p \in \tscolors'(s)$ iff $s \in S$.

Let $\phi \in \yvlang$ be a formula of some logic $\yvlang$,
we use $\Mod_{\yvlang}(\phi) = \{\bbS \mid \bbS \models \phi\}$ to denote 
the class of transition systems that make $\phi$ true.
The subscript $\yvlang$ will be omitted when $\yvlang$ is clear from context.
A class $\mathsf{C}$ of transition systems is said to be 
\emph{$\yvlang$-definable} if there is a formula $\phi \in \yvlang$ such that
$\Mod_{L}(\phi) = \mathsf{C}$.
We use the notation $\phi \equiv \psi$ to mean that $\Mod_{L}(\phi) = 
\Mod_{L}(\psi)$ and given two logics $\yvlang, \yvlang'$ we use $\yvlang \equiv 
\yvlang'$ when the $\yvlang$-definable and $\yvlang'$-definable classes of 
models coincide.

% \textcolor{red}{%
% \textit{Convention.}
% Throughout this paper, we will only consider transition systems $\bbS$
% in which $R[s]$ is non-empty for every node $s \in T$.
% \marginpar{\fbox{check!}}
% In particular this means that every tree we consider is \emph{leafless}.
% All our results, however, can easily be lifted to the general case.
% }

\subsection{Games}

We introduce some terminology and background on infinite games.
All the games that we consider involve two players called \emph{\'Eloise}
($\exists$) and \emph{Abelard} ($\forall$).
In some contexts we refer to a player $\Pi$ to specify a
a generic player in $\{\exists,\forall\}$.
%
Given a set $A$, by $A^*$ and $A^\omega$ we denote respectively the set of
words (finite sequences) and streams (or infinite words) over $A$.

A \emph{board game} $\mc{G}$ is a tuple $(G_{\exists},G_{\forall},E,\win)$,
where $G_{\exists}$ and $G_{\forall}$ are disjoint sets whose union
$G=G_{\exists}\cup G_{\forall}$ is called the \emph{board} of $\mc{G}$,
$E\subseteq G \times G$ is a binary relation encoding the \emph{admissible
moves}, and $\win \subseteq G^{\omega}$ is a \emph{winning condition}.
An \emph{initialized board game} $\mc{G}@u_I$ is a tuple
$(G_{\exists},G_{\forall},u_I, E,\win)$ where
%$(G_{\exists},G_{\forall},E,\win)$ is a board game and
$u_I \in G$ is the
\emph{initial position} of the game.
In a \emph{parity game}, the set $\win$ is  given by a \emph{parity function},
that is, a map $\pmap: G \to \omega$ of finite range, in the sense that a
sequence $(a_{i})i<\om$ belongs to $\win$ iff the maximal value $n$ that 
is reached as $n = \pmap(a_{i})$ for infinitely many $i$, is even.
% we say that $\mc{G}$ is a parity game and sometimes
% simply write $\mc{G}=(G_{\exists},G_{\forall},E,\pmap)$.

Given a board game $\mc{G}$, a \emph{match} of $\mc{G}$ is simply a path
through the graph $(G,E)$; that is, a sequence $\pi = (u_i)_{i< \alpha}$ of
elements of $G$, where $\alpha$ is either $\omega$ or a natural number,
and $(u_i,u_{i+1}) \in E$ for all $i$ with $i+1 < \alpha$.
A match of $\mc{G}@u_{I}$ is supposed to start at $u_{I}$.
Given a finite match $\pi = (u_i)_{i< k}$ for some $k<\omega$, we call
$\mathit{last}(\pi) \isdef u_{k-1}$ the \emph{last position} of the match; the
player $\Pi$ such that $\mathit{last}(\pi) \in G_{\Pi}$ is supposed to move
at this position, and if $E[\mathit{last}(\pi)] = \emptyset$, we say that
$\Pi$ \emph{got stuck} in $\pi$.
%
A match $\pi$ is called \emph{total} if it is either finite, with one of the
two players getting stuck, or infinite. Matches that are not total are called
\emph{partial}.
Any total match $\pi$ is \emph{won} by one of the players:
If $\pi$ is finite, then it is won by the opponent of the player who gets stuck.
Otherwise, if $\pi$ is infinite, the winner is $\exists$ if $\pi \in
\win$, and $\forall$ if $\pi \not\in \win$.

Given a board game $\mc{G}$ and a player $\Pi$, let $\pmatches{G}{\Pi}$ denote
the set of partial matches of $\mc{G}$ whose last position belongs to player
$\Pi$.
A \emph{strategy for $\Pi$} is a function $f:\pmatches{G}{\Pi}\to G$.
A match $\pi  = (u_i)_{i< \alpha}$ of $\mc{G}$ is
\emph{$f$-guided} if for each $i < \alpha$ such that $u_i \in G_{\Pi}$ we
have that $u_{i+1} = f(u_0,\dots,u_i)$.
%
Let $u \in G$ and a $f$ be a strategy for $\Pi$.
We say that $f$ is a \emph{surviving strategy} for $\Pi$ in $\mc{G}@u$ if
for each $f$-guided partial match $\pi$ of $\mc{G}@u$, if $\mathit{last}(\pi)$
is in $G_{\Pi}$ then $f(\pi)$ is legitimate, that is, $(\mathit{last}(\pi),
f(\pi)) \in E$.
We say that $f$ is a \emph{winning strategy} for $\Pi$ in $\mc{G}@u$ if,
additionally, $\Pi$ wins each $f$-guided total match of $\mc{G}@u$.
%
If $\Pi$ has a winning strategy for $\mc{G}@u$ then $u$ is called a 
\emph{winning position} for $\Pi$ in $\mc{G}$.
The set of positions of $\mc{G}$ that are winning for $\Pi$ is denoted by
$\win_{\Pi}(\mc{G})$.

A strategy $f$ is called \emph{positional} if $f(\pi) = f(\pi')$ for
each $\pi,\pi'\in \Dom(f)$ with $\mathit{last}(\pi) = 
\mathit{last}(\pi')$.
A board game $\mc{G}$ with board $G$ is \emph{determined} if $G = \win_{\exists}(\mc{G}) \cup \win_{\forall}(\mc{G})$, that is, each $u \in G$ is a winning position for one of the two players.
%
The next result states that parity games are positionally determined.

\begin{fact}[\cite{EmersonJ91,Mostowski91Games}]
\label{THM_posDet_ParityGames}
For each parity game $\mc{G}$, there are positional strategies $f_{\exists}$
and $f_{\forall}$ respectively for player $\exists$ and $\forall$, such that
for every position $u \in G$ there is a player $\Pi$ such that $f_{\Pi}$ is a
winning strategy for $\Pi$ in $\mc{G}@u$.
\end{fact}
%
In the sequel we will often assume, without notification, that strategies in
parity games are positional. 
Moreover, we think of a positional strategy $f_\Pi$ for player $\Pi$ as a 
function $f_\Pi:G_\Pi\to G$.

\subsection{The Modal $\mu$-Calculus and some of its fragments.}
\label{subsec:mu}

The language of the modal $\mu$-calculus ($\muML$) is given by the following 
grammar:
%
\begin{equation*}
    \phi\ \isbnf  q \mid \neg q \mid \phi \land \phi \mid
    \phi \lor \phi \mid  \Diamond \phi \mid \Box \phi \mid
    \mu p.\phi \mid \nu p.\phi
\end{equation*}
% \begin{equation*}
%     \phi\ \isbnf  q \mid \phi \land \phi \mid \lnot\phi
%     \mid  \Diamond \phi \mid
%     \mu p.\phi
% \end{equation*}
%
where $p,q \in \Prop$ and $p$ is positive in $\phi$ (i.e., $p$ is not
negated).
We will freely use standard syntactic concepts and notations related to this
language, such as the sets $\FV(\phi)$ and $\BV(\phi)$ of \emph{free} and 
\emph{bound} variables of $\phi$, and the collection $\Sfor(\phi)$ of subformulas
of $\phi$.
We use the standard convention that no variable is both free and bound in a
formula and that every bound variable is fresh.
We let $\muML(\pprop)$ denote the collection of formulas $\phi$ with $\FV(\phi)
\sse \pprop$.
Sometimes we write $\psi \subf \phi$ to denote that $\psi$ is a subformula
of $\phi$.
%
For a bound variable $p$ occurring in some formula $\phi \in \muML$, we use
$\delta_p$ to denote the binding definition of $p$, that is, the unique formula
such that either $\mu p.\delta_p$ or $\nu p.\delta_p$ is a subformula of 
$\phi$.

We need some notation for the notion of \emph{substitution}.
Let $\phi$ and $\{ \psi_{z} \mid z \in Z \}$ be modal fixpoint formulas, where
$Z \cap \BV{(\phi)} = \nada$.
Then we let $\phi[\psi_{z}/z \mid z \in Z]$
denote the formula obtained from $\phi$ by simultaneously substituting each
formula $\psi_{z}$ for $z$ in $\phi$ (with the usual understanding that no 
free variable in any of the $\psi_{z}$ will get bound by doing so).
In case $Z$ is a singleton $z$, we will simply write $\phi[\psi_{z}/z]$, or 
$\phi[\psi]$ if $z$ is clear from context.
\smallskip

The semantics of this language is completely standard. 
Let $\bbS = \tup{T,R,\tscolors, s_I}$ be a transition system and $\phi \in 
\muML$. 
We inductively define the \emph{meaning} $\ext{\phi}^{\bbS}$ which includes
the following clauses for the least $(\mu)$ and greatest ($\nu$) fixpoint 
operators:
%
\begin{align*}
  \ext{\mu p.\psi}^{\bbS}  & \isdef   \bigcap \{S \subseteq T \mid S \supseteq \ext{\psi}^{\bbS[p\mapsto S]} \}  \\
  \ext{\nu p.\psi}^{\bbS}  & \phi   \bigcup \{S \subseteq T \mid S \subseteq \ext{\psi}^{\bbS[p\mapsto S]} \}
\end{align*}
%
We say that $\phi$ is \emph{true} in $\bbS$ (notation $\bbS \mmodels \phi$) iff $s_I \in \ext{\phi}^{\bbS}$.% As for the case of $\wmso$, $\ext{\phi}$ denotes the class of transition systems where $\phi$ is true.\fcwarning{Pointed or not?}

We will now describe the semantics defined above in game-theoretic terms. 
That is, we will define the evaluation game $\egame(\phi,\bbS)$ associated with
a formula $\phi \in \muML$ and a transition system $\bbS$. 
This game is played by two players (\eloise and \abelard) moving through 
positions $(\xi,s)$ where $\xi \subf \phi$ and $s \in T$.
%

%\begin{table}
%\centering
%\begin{tabular}{|l|c|l|c|}
%\hline
%Position & Player & Admissible moves
%\\ \hline
%   $(\psi_1 \vee \psi_2,s)$   & $\exists$ & $\{(\psi_1,s),(\psi_2,s) \}$ 
%\\ $(\psi_1 \wedge \psi_2,s)$ & $\forall$ & $\{(\psi_1,s),(\psi_2,s) \}$ 
%\\ $(\Diamond\phi,s)$         & $\exists$ & $\{(\phi,t)\ |\ t \in R[s] \}$ 
%\\ $(\Box\phi,s)$             & $\forall$ & $\{(\phi,t)\ |\ t \in R[s] \}$ 
%\\ $(\mu p.\phi,s)$           & $-$       & $\{(\phi,s) \}$ 
%\\ $(\nu p.\phi,s)$           & $-$       & $\{(\phi,s) \}$ 
%\\ $(p,s)$ with $p \in \BV(\phi)$ & $-$ & $\{(\delta_p,s) \}$ 
%\\ $(\lnot q,s)$ with $q \in \FV(\phi)$ and $q \notin \tscolors(s)$ 
%   & $\forall$ & $\emptyset$
%\\ $(\lnot q,s)$ with $q \in \FV(\phi)$ and $q \in \tscolors(s)$ 
%   & $\exists$ & $\emptyset$
%\\ $(q,s)$ with $q \in \FV(\phi)$ and $q \in \tscolors(s)$ 
%   & $\forall$ & $\emptyset$
%\\ $(q,s)$ with $q \in \FV(\phi)$ and $q \notin \tscolors(s)$ 
%   & $\exists$ & $\emptyset$
%\\ \hline
%\end{tabular}
%\caption{}
%\label{egame_mucalc}
%\end{table}

%\begin{acmtable}{\textwidth}
%\centering
%\begin{tabular}{|l|c|l|c|}
%\hline
%Position & Player & Admissible moves
%\\ \hline
%   $(\psi_1 \vee \psi_2,s)$   & $\exists$ & $\{(\psi_1,s),(\psi_2,s) \}$ 
%\\ $(\psi_1 \wedge \psi_2,s)$ & $\forall$ & $\{(\psi_1,s),(\psi_2,s) \}$ 
%\\ $(\Diamond\phi,s)$         & $\exists$ & $\{(\phi,t)\ |\ t \in R[s] \}$ 
%\\ $(\Box\phi,s)$             & $\forall$ & $\{(\phi,t)\ |\ t \in R[s] \}$ 
%\\ $(\mu p.\phi,s)$           & $-$       & $\{(\phi,s) \}$ 
%\\ $(\nu p.\phi,s)$           & $-$       & $\{(\phi,s) \}$ 
%\\ $(p,s)$ with $p \in \BV(\phi)$ & $-$ & $\{(\delta_p,s) \}$ 
%\\ $(\lnot q,s)$ with $q \in \FV(\phi)$ and $q \notin \tscolors(s)$ 
%   & $\forall$ & $\emptyset$
%\\ $(\lnot q,s)$ with $q \in \FV(\phi)$ and $q \in \tscolors(s)$ 
%   & $\exists$ & $\emptyset$
%\\ $(q,s)$ with $q \in \FV(\phi)$ and $q \in \tscolors(s)$ 
%   & $\forall$ & $\emptyset$
%\\ $(q,s)$ with $q \in \FV(\phi)$ and $q \notin \tscolors(s)$ 
%   & $\exists$ & $\emptyset$
%\\ \hline
%\end{tabular}
%\caption{\label{egame_mucalc}Rules of the evaluation game.}
%\end{acmtable}
%

In an arbitrary position $(\xi,s)$ it is useful to think of \eloise trying to
show that $\xi$ is true at $s$, and of \abelard of trying to convince her that 
$\xi$ is false at $s$. 
The rules of the evaluation game are given in  the following table.%Table~\ref{egame_mucalc}.
%\begin{table}
\begin{center}
\begin{tabular}{|l|c|l|c|}
\hline
Position & Player & Admissible moves
\\ \hline
   $(\psi_1 \vee \psi_2,s)$   & $\exists$ & $\{(\psi_1,s),(\psi_2,s) \}$ 
\\ $(\psi_1 \wedge \psi_2,s)$ & $\forall$ & $\{(\psi_1,s),(\psi_2,s) \}$ 
\\ $(\Diamond\phi,s)$         & $\exists$ & $\{(\phi,t)\ |\ t \in R[s] \}$ 
\\ $(\Box\phi,s)$             & $\forall$ & $\{(\phi,t)\ |\ t \in R[s] \}$ 
\\ $(\mu p.\phi,s)$           & $-$       & $\{(\phi,s) \}$ 
\\ $(\nu p.\phi,s)$           & $-$       & $\{(\phi,s) \}$ 
\\ $(p,s)$ with $p \in \BV(\phi)$ & $-$ & $\{(\delta_p,s) \}$ 
\\ $(\lnot q,s)$ with $q \in \FV(\phi)$ and $q \notin \tscolors(s)$ 
   & $\forall$ & $\emptyset$
\\ $(\lnot q,s)$ with $q \in \FV(\phi)$ and $q \in \tscolors(s)$ 
   & $\exists$ & $\emptyset$
\\ $(q,s)$ with $q \in \FV(\phi)$ and $q \in \tscolors(s)$ 
   & $\forall$ & $\emptyset$
\\ $(q,s)$ with $q \in \FV(\phi)$ and $q \notin \tscolors(s)$ 
   & $\exists$ & $\emptyset$
\\ \hline
\end{tabular}
\end{center}
%\caption{}
%\label{egame_mucalc}
%\end{table}
Every finite match of this game is lost by the player that got stuck. 
To give a winning condition for an infinite match let $p$ be, of the bound 
variables of $\phi$ that get unravelled infinitely often, the one such that 
$\delta_{p}$ the highest subformula in the syntactic tree of $\phi$. 
The winner of the match is \abelard if $p$ is a $\mu$-variable and \eloise if 
$p$ is a $\nu$-variable.
We say that $\phi$ is true in $\bbS$ iff \eloise has a winning strategy in 
$\egame(\phi,\bbS)$.

\begin{proposition}[Adequacy Theorem]\label{p:unfold=evalgame}
Let $\phi = \phi(p)$ be a formula of $\muML$ in which all occurrences of $p$ are positive, $\bbS$ be a LTS and $s \in T$. Then:
\begin{equation}
\label{eq:adeq3}
s \in \ext{\mu p.\phi}^{\bbS} %\iff s \in \Win_{\eloi}(\UG(\phi_{x}^{\bbS}))
\iff (\mu p.\phi,s) \in \win_{\eloise}(\egame(\mu p.\phi,\bbS)).
\end{equation}
\end{proposition}

\bigskip
Formulas of the modal $\mu$-calculus may be classified according to their
\emph{alternation depth}, which roughly is given as the maximal length of
a chain of nested alternating least and greatest fixpoint operators~\cite{Niwinski86}.
The \emph{alternation-free fragment} of the modal $\mu$-calculus~($\AFMC$) is 
usually defined as the collection of $\muML$-formulas without nesting of least
and greatest fixpoint operators. 
It can also be also given a more standard grammatical definition as follows.

\begin{definition}
Given a set $\qprop$ of propositional variables, we define the fragment 
$\noe{\mu\ML}{\qprop}$ of \muML-formulas that are (syntactically) 
\emph{noetherian} in $\qprop$, by the following grammar:
\begin{equation*}
   \phi \isbnf  q
   \mid \psi
   \mid \phi \lor \phi
   \mid \phi \land \phi
     \mid \Diamond \phi
       \mid \Box \phi
   \mid \mu p.\phi'
\end{equation*}
where $q \in \qprop$, $\psi$ is a $\qprop$-free $\muML$-formula, and 
$\phi' \in \noe{\mu\ML}{\qprop\cup\{p\}}$. 
The \emph{co-noetherian} fragment $\conoe{\mu\ML}{Q}$ is defined dually.
\end{definition}

The alternation-free $\mu$-calculus can be defined as the fragment of the full
language where we restrict the application of the least fixpoint operator $\mu
p$ to formulas that are noetherian in $p$ (and apply a dual condition to the 
greatest fixpoint operator).

\begin{definition}
The formulas of the \emph{alternation-free} $\mu$-calculus $\AFMC$ 
are defined by the following grammar:
\begin{equation*}
   \phi \isbnf  
      q \mid \neg q  
   \mid \phi\lor\phi \mid \phi\land\phi 
      \mid \Diamond \phi
       \mid \Box \phi
   \mid \mu p. \phi'    
   \mid \nu p. \phi'',
\end{equation*} 
where $p,q \in \Prop$, $\phi' \in \AFMC \cap \noe{\mu\ML}{p}$
and dually $\phi'' \in \AFMC \cap \conoe{\mu\ML}{p}$.
\end{definition}

It is then immediate to verify that the above definition indeed captures exactly
all formulas without alternation of least and greatest fixpoints.
One may prove that 
% \begin{fact}
a formula $\phi \in \muML$ belongs to the fragment $\AFMC$ iff for all 
subformulas $\mu p.\psi_1$ and $\nu q.\psi_2$ it holds that $p$ is not free in
$\psi_2$ and $q$ is not free in $\psi_1$.
% \end{fact}
%
Over arbitrary transition systems, this fragment is less expressive than the 
whole $\muML$~\cite{Park79}. 

In order to properly define the fragment $\mucML \subseteq \AFMC$ which is of
critical importance in this article, we are particularly interested in the 
\emph{continuous} fragment of the modal $\mu$-calculus. 
As observed in Section~\ref{sec:intro}, the abstract notion of continuity can 
be given a concrete interpretation in the context of $\mu$-calculus.
%
\begin{definition}
Let $\phi \in \muML$, and $q$ be a propositional variable. 
We say that \emph{$\phi$ is continuous in $q$} iff for every transition 
system $\bbS$ there exists some finite $S \subseteq_\omega \tsval(q)$ such that
$$
\bbS \mmodels \phi \quad\text{iff}\quad \bbS[q \mapsto S] \mmodels \phi.
$$
\end{definition}

We can give a syntactic characterisation of the fragment of $\muML$ that captures
this property. 

\begin{definition}
Given a set $\qprop$ of propositional variables, we define the fragment of \muML 
\emph{continuous} in $\qprop$, denoted by $\cont{\muML}{\qprop}$, by induction 
in the following way
\begin{equation*}
   \phi \isbnf  q
   \mid \psi
   \mid \phi \lor \phi
   \mid \phi \land \phi
   \mid \Diamond \phi
   \mid \mu p.\phi'
\end{equation*}
%
where $q,p \in \qprop$, $\psi$ is a $\qprop$-free $\muML$-formula and 
$\phi' \in \cont{\muML}{\qprop\cup\{p\}}$.

The  \emph{co-continuous} fragment $\cocont{\mu\ML}{Q}$ is defined dually. 
\end{definition}

\begin{proposition}[\cite{Fontaine08,FV12}]\label{prop:FVcont}
A $\muML$-formula is continuous in $q$ iff it is equivalent to a formula in the
fragment $\cont{\muML}{q}$.
\end{proposition}

Finally, we define $\mucML$ to be the fragment of $\muML$ where the use of the
least fixed point operator is restricted to the continuous fragment. 

\begin{definition}
Formulas of the fragment $\mucML$ are given by:% the following induction:
\begin{equation*}
   \phi \isbnf  q \mid \lnot q
    \mid \phi \lor \phi
        \mid \phi \land \phi
    \mid \Diamond \phi
     \mid \Box \phi \mid
    \mu p.\phi' 
    \mid \nu p.\phi''
    \end{equation*}
%
where $p,q \in \Prop$,  $\phi' \in \cont{\muML}{p} \cap \mucML$, and dually 
$\phi'' \in \cocont{\muML}{p} \cap \mucML$.
\end{definition}

% \begin{proposition}
% The following hold for any formula $\phi \in \mucML$:
% \begin{enumerate}[(1)]
% \itemsep 0pt
% \item $\phi$ is an $\AFMC$-formula,
% \item 
% Every $\mu$-variable in $\phi$ is existential (i.e., is only in the scope of
% diamonds), and dually every $\nu$-variable in $\phi$ is universal (i.e., is 
% only in the scope of boxes).
% \end{enumerate}
% \end{proposition}
% \begin{proof}
% Both points are proved by an easy induction on the complexity of a formula. 
% For the first one,  it is enough to notice that if $\phi \in \cont{\muML}{q} 
% \cap \AFMC$, then $\mu q. \phi \in \AFMC$ by definition of $\cont{\muML}{q} $.
% \end{proof}

It is easy to verify that $\mucML \sse \mudML$.
Characteristic about $\mucML$ is that in a formula $\mu p. \phi \in \mucML$,
all occurrences of $p$ are \emph{existential} in the sense that they may be 
in the scope of a diamond but not of a box.
Furthermore, as an immediate consequence of Proposition \ref{prop:FVcont} we 
may make the following observation.

\begin{corollary}\label{cor:cont}
For every $\mucML$-formula $\mu p. \phi$, $\phi$ is continuous in $p$.
\end{corollary}

\subsection{Bisimulation}
% Bisimulation is a notion of behavioral equivalence between processes.
For the case of  transition systems, it is formally defined as follows.

\begin{definition}
Let $\model = \tup{T, R, \tscolors, s_I}$ and
$\model' = \tup{T', R', \tscolors', s'_I}$ be transition systems.
A \emph{bisimulation} is a relation $Z \subseteq T \times T'$
such that for all $(t,t^{\prime}) \in Z$ the following holds:
\begin{description}
  \itemsep 0 pt
  \item[(atom)] $p \in \tscolors(t)$ iff $p \in \tscolors^{\prime}(t')$ for all $p\in\prop$;
  \item[(forth)] for all $s \in R[t]$ there is $s^{\prime} \in R^{\prime}[t^{\prime}]$ such that $(s,s^{\prime}) \in Z$;
  \item[(back)] for all $s^{\prime} \in R^{\prime}[t^{\prime}]$ there is $s \in R[t]$ such that $(s,s^{\prime}) \in Z$.
\end{description}
%
Two pointed transition systems $\model$ and $\model^{\prime}$ are
\emph{bisimilar} (denoted $\model \bis \model^{\prime}$) if there is a
bisimulation $Z \subseteq T \times T^{\prime}$ containing $(s_I,s'_I)$.
\end{definition}

The following fact about tree unravellings will allow us to provide a proof of
Theorem~\ref{t:m1} by just proving it for tree languages.

\begin{fact}\label{prop:tree_unrav}
$\model$ and $\model^e$ are bisimilar, for every transition system $\model$.
\end{fact}

A class of transition systems $\mclass$ is \emph{bisimulation closed} if $\model
\bis \model^{\prime}$ implies that $\model \in \mclass$ iff $\model^{\prime}
\in \mclass$, for all $\model$ and $\model^{\prime}$.
A formula $\varphi \in \llang$ is \emph{bisimulation-invariant} if $\model \bis
\model^{\prime}$ implies that $\model \mmodels \varphi$ iff $\model^{\prime}
\mmodels \varphi$, for all $\model$ and $\model^{\prime}$.
%An analogous definition can be given for $\wmso$.

\begin{fact}
Each $\MC$-definable class of transition systems is bisimulation closed.
\end{fact}


Bisimulation is a notion of behavioral equivalence between processes.
For the case of  transition systems, it is formally defined as follows.

\begin{definition}
Let $\bbS = \tup{T, R, \tscolors, s_I}$ and $\bbS' = \tup{T', R', \tscolors', 
s'_I}$ be $\pprop$-labeled transition systems.
A \emph{bisimulation} is a relation $Z \subseteq T \times T'$ such that for all 
$(t,t') \in Z$ the following holds:
\begin{description}
  \itemsep 0 pt
  \item[(atom)] 
  $\tscolors(t) = \tscolors'(t')$;
  \item[(forth)] 
  for all $s \in R[t]$ there is $s'\in R'[t']$ such
  that $(s,s') \in Z$;
  \item[(back)] 
  for all $s'\in R'[t']$ there is $s \in R[t]$ such
  that $(s,s') \in Z$.
\end{description}
%
Two pointed transition systems $\bbS$ and $\bbS'$ are
\emph{bisimilar} (denoted $\bbS \bis \bbS'$) if there is a
bisimulation $Z \subseteq T \times T'$ containing $(s_I,s'_I)$.
\end{definition}

The following observation about tree unravellings is the key to understand
the importance of tree models in the setting of invariance modulo bismilarity
results.

\begin{fact}
    \label{prop:tree_unrav}
$\bbS$, $\unravel{\bbS}$ and $\bbS^{\om}$ are bisimilar, for every transition
system $\bbS$.
\end{fact}

A class $\mathsf{C}$ of transition systems is \emph{bisimulation closed} if 
$\bbS \bis \bbS'$ implies that $\bbS \in \mathsf{C}$ iff $\bbS'
\in \mathsf{C}$, for all $\bbS$ and $\bbS'$.
A formula $\phi \in \yvlang$ is \emph{bisimulation-invariant} if $\bbS \bis
\bbS'$ implies that $\bbS \mmodels \phi$ iff $\bbS'
\mmodels \phi$, for all $\bbS$ and $\bbS'$.
%An analogous definition can be given for $\wmso$.

\begin{fact}
Each $\muML$-definable class of transition systems is bisimulation closed.
\end{fact}


\subsection{Monadic second-order logics}
\label{sec:prel-so}
% Given a set of propositional letters, or predicates, $\prop$, we define three variants of monadic second-order logic on it:
\emph{(standard) monadic second-order logic} ($\mso(\prop)$),
\emph{weak monadic second-order logic} ($\wmso(\prop)$) and
\emph{noetherian monadic second-order logic} ($\nmso(\prop)$).
We omit  $\prop$ when the set of proposition letters is clear from context. 
These logics share the same syntax.
\begin{definition}\label{def:mso}
The formulas of the \emph{monadic second-order
language} on a set of predicates $\prop$ are defined by the following grammar:
%
\begin{eqnarray*}\label{EQ_mso}
  \varphi ::= \here{p} \mid p \inc q \mid R(p,q) \mid \lnot\varphi \mid \varphi\lor\varphi \mid \exists p.\varphi,
\end{eqnarray*}
where $p$ and $q$ are letters from $\prop$.
We  adopt the standard convention that no letter is both free and bound in
$\varphi$.
\end{definition}

The three logics are distinguished by their semantics.
Let  $\model = \tup{T,R,\tscolors, s_I}$ be a LTS, the interpretation of the
atomic formulas and the boolean connectives is fixed and standard, e.g.:
\begin{align*}
\model \models \here{p} & \quad\text{ iff }\quad  \tsval(p) = \compset{s_I} \\
\model \models p \inc q & \quad\text{ iff }\quad  \tsval(p) \subseteq \tsval(q) \\
\model \models R(p,q) & \quad\text{ iff }\quad  \text{for every $s\in \tsval(p)$ there exists $t\in \tsval(q)$ such that $sRt$} 
\end{align*}

The interpretation of the existential quantifier is that

\begin{align*}
\model \models\ \exists p. \varphi  & \quad\text{ iff }\quad  \model[p \mapsto X] \models \phi \,
\left.\begin{cases}
 \text{for some }   & {\bf (\mso)} \\
  \text{for some \emph{finite} }   & {\bf (\wmso)} \\
    \text{for some \emph{noetherian} }   & {\bf (\nmso)} 
 \end{cases}\right\}\,
 X \subseteq T.
\end{align*}
%if and only if
%\begin{description}
%%[\IEEEsetlabelwidth{$\alpha\omega \pi\theta\mu\varphi$}\IEEEusemathlabelsep]
%\item[$(\Wmso)$] $\model[p \mapsto X] \models \phi$ for some finite $S \subseteq_\omega T$
%\item[$(\Nmso)$] $\model[p \mapsto X] \models \phi$ for some noetherian
%    $X \subseteq T$.
%\end{description}

Let $\varphi\in \mso$ be a formula. We denote with $\|\varphi \|_P$ the set
of $C$-transition structures $\model$ such that $\model\models \varphi$.
The subscript $P$ is omitted when the set $P$ of proposition letters is clear
from the context.
A class $\mc{L}$ of transition systems is $\mso$\emph{-definable} if there
is a formula $\varphi \in \mso$ such that $\| \varphi \| = \mc{L}$.
We define the analogous notions for $\wmso$ and $\nmso$ in the same way.



\begin{remark}
The reader may have expected a more standard two-sorted language for second-order logic, for example given by
%
$$
\varphi ::= p(x)
%\mid X(y)
\mid R(x,y)
\mid x \foeq y
\mid \neg \varphi
\mid \varphi \lor \varphi
\mid \exists x.\varphi
\mid \exists p.\varphi
$$%
where $p \in \prop$, $x,y \in \fovar$ (individual variables), %$X \in \sovar$ (second-order variables)
and $\foeq$ is the symbol for equality.
Both definitions can be proved to be equivalent, however, we choose to keep Definition~\ref{def:mso} as it will be better suited to work with in the context of automata.
\end{remark}


Three variants of monadic second-order logic feature in our work:
\emph{standard}, \emph{weak}, and \emph{noetherian} monadic second-order 
logic, and for each of these three variants, we consider a one-sorted and 
a two-sorted version.
As we will see later, the one-sorted version fits better in the 
automata-theoretic framework, whereas it is more convenient to use the 
two-sorted approach when translating $\mu$-calculi into second order languages.
In both the one-sorted and the two-sorted version, the syntax of the three 
languages is the same, the difference lying in the semantics, more specificaly,
in the type of subsets over which the second-order quantifiers range.
In the case of standard and weak monadic second-order logic, these quantifiers 
range over all, respectively, all finite subsets of the model.
In the case of \nmso we need the concept of a \emph{noetherian} subset of an LTS.

\begin{definition}
\label{d:bundle1}
Let $\bbS = \tup{T,R,\tscolors, s_I}$ be an LTS, and let $B$ be a non-empty set 
of finite paths that all share the same starting point $s$; we call $B$ a 
\emph{bundle rooted at} $s$, or simply an $s$-\emph{bundle}, if $B$ does not
contain an infinite ascending chain $\pi_{0} \sqsubset \pi_{1} \sqsubset 
\cdots$, where $\sqsubset$ denotes the (strict) initial-segment relation on 
paths.
A \emph{bundle} is simply an $s$-bundle for some $s \in T$.

A subset $X$ of $T$ is called \emph{noetherian} if there is a bundle $B$ such
that each $t \in X$ lies on some path in $B$.
\end{definition}

Notice that in a tree model, the noetherian subsets coincide with those that
are included in a well-founded subtree.

\subsubsection*{One-sorted monadic second-order logics}
%
\begin{definition}\label{def:mso}
The formulas of the \emph{(one-sorted) monadic second-order language} are
defined by the following grammar:
%
\begin{eqnarray*}\label{EQ_mso}
  \phi \isbnf  \here{p} \mid p \inc q \mid R(p,q) \mid \lnot\phi 
     \mid \phi\lor\phi \mid \exists p.\phi,
\end{eqnarray*}
where $p$ and $q$ are letters from $\Prop$.
We  adopt the standard convention that no proposition letter is both free and
bound in $\phi$.
\end{definition}


As mentioned, the three logics $\smso$, $\wmso$ and $\nmso$ are distinguished by
their semantics. 
Let  $\bbS = \tup{T,R,\tscolors, s_I}$ be an LTS.
The interpretation of the atomic formulas is fixed:
\begin{align*}
\bbS \models \here{p} & \quad\text{ iff }\quad  \tsval(p) = \{s_I\} \\
\bbS \models p \inc q & \quad\text{ iff }\quad  \tsval(p) \subseteq \tsval(q) \\
\bbS \models R(p,q) & \quad\text{ iff }\quad  \text{for every $s\in \tsval(p)$ there exists $t\in \tsval(q)$ such that $sRt$} 
\end{align*}
Furthermore, the interpretation of the boolean connectives is standard.
The interpretation of the existential quantifier is where the logics diverge:

\begin{align*}
\bbS \models\ \exists p. \phi  & \quad\text{ iff }\quad  \bbS[p \mapsto X] \models \phi \,
\left.\begin{cases}
   \text{for some }                   & (\smso)
\\ \text{for some \emph{finite} }     & (\wmso) 
\\ \text{for some \emph{noetherian} } & (\nmso)
\end{cases}\right\}\,
 X \subseteq T.
\end{align*}

Observe that for a given monadic second-order formula $\phi$, the classes 
$\Mod_{\smso}(\phi)$, $\Mod_{\wmso}(\phi)$ and $\Mod_{\nmso}(\phi)$ will 
generally be different.

\subsubsection*{Two-sorted monadic second-order logics}
The reader may have expected to see the following more standard language for
second-order logic.
\begin{definition}
\label{def:2mso}
Given a set $\fovar$ of individual (first-order) variables, we define the 
formulas of the \emph{two-sorted monadic second-order language} by the following
grammar:
\[
\phi \isbnf  p(x)
%\mid X(y)
\mid R(x,y)
\mid x \foeq y
\mid \neg \phi
\mid \phi \lor \phi
\mid \exists x.\phi
\mid \exists p.\phi
\]
where $p \in \Prop$, $x,y \in \fovar$ and $\foeq$ is the symbol for equality.   
\end{definition}

Formulas are interpreted over an LTS $\bbS = \tup{T,R,\tscolors, s_I}$ with a
variable assignment $g: \fovar \to T$, and the semantics of the language is
completely standard. 
Depending on whether second-order quantification ranges over all subsets, over 
finite subsets or over noetherian subsets, we obtain the three two-sorted 
variants denoted respectively as $2\smso$, $2\wmso$ and $2\nmso$.

%\begin{remark}
\subsubsection*{Equivalence of the two versions}
In each variant, the one-sorted and the two-sorted versions can be proved to
be equivalent, but there is a sublety due to the fact that our models have a 
distinguished state.
In the one-sorted language, we use the downarrow $\here$ to access this
distinguished state; in the two-sorted approach, we will use a \emph{fixed}
variable $v$ to refer to the distinguished state, and given a formula 
$\phi(v)$ of which $v$ is the only free individual variable, we write 
$\bbS \models \phi[s_{I}]$ rather than $\bbS[v \mapsto s_{I}] \models \phi$.
As a consequence, the proper counterpart of the one-sorted language $\smso$ is
the set $2\smso(v)$ of those $2\smso$-formulas that have precisely $v$ 
as their unique free variable.

More in particular, with $L \in \{\smso, \wmso, \nmso\}$, we say that $\phi \in
L$ is \emph{equivalent to} $\psi(v) \in L(v)$ if
\[
\bbS \models \phi \text{ iff } \bbS \models \psi[s_{I}]
\]
for every model $\bbS = \tup{T,R,\tscolors, s_I}$.
We can now state the equivalence between the two approaches to 
monadic second-order logic as follows.

\begin{fact}
\label{fact:msovs2mso}
Let $L \in \{\smso, \wmso, \nmso\}$ be a monadic second-order logic.
\begin{enumerate}
\item
There is an effective construction transforming a formula $\phi \in L$ into
an equivalent formula $\phi^{t} \in 2L(v)$.
\item
There is an effective construction transforming a formula $\psi \in 2L(v)$ into
an equivalent formula $\psi^{o} \in L$.
\end{enumerate}
\end{fact}

Since it is completely straightforward to define a translation $(\cdot)^{t}$ as 
required for part (1) of Fact~\ref{fact:msovs2mso}, we only discuss the proof 
of part (2). 
The key observation here is that a single-sorted language can interpret the 
corresponding two-sorted language by encoding every individual variable $x \in 
\fovar$ as a set variable $p_x$ denoting a singleton, and that it is easy to 
write down a formula stating that a variable indeed is interpreted by a 
singleton.
As a consequence, where $2\yvlang(\pprop,\mathsf{X})$ denotes the set of
$2L$-formulas with free second-order variables in $\pprop$ and free first-order
variables in $\mathsf{X}$, it is not hard to formulate a translation 
$(\cdot)^{m} : 2\yvlang(\pprop,\mathsf{X}) \to \yvlang(\pprop \uplus 
\{ p_{x} \mid x \in \mathsf{X} \})$
such that, for every model $\bbS$, every variable assignment $g$ and every
formula $\psi \in 2\yvlang(\Prop,\mathsf{X})$:
\[
\bbS,g \models \psi \quad\text{iff}\quad 
\bbS[p_{x} \mapsto \{g(x)\} \mid x \in \mathsf{X}] \models \psi^m.
\]
From this it is immediate that any $\psi \in 2L(v)$ satisfies
\[
\bbS \models \psi[s_{I}]
\quad\text{iff}\quad 
\bbS \models \exists p_{v} (\here{p_{v}} \land \psi^{m}),
\]
so that we may take $\psi^{o} \isdef \exists p_{v} (\here{p_{v}} \land 
\psi^{m})$.

% \begin{fact}
% Let $\yvlang \in \{\mso, \wmso, \nmso\}$ and $\Prop_X\isdef  \{p_x\mid x \in \fovar\}$.
% The translation $(\cdot)^{m} : 2\yvlang(\pprop) \to \yvlang(\pprop \uplus
% \Prop_X)$ defined as
% \begin{itemize}
% \item $(p(x))^m\isdef   p_x \sqsubset p$,
% \item $(R(x,y))^m\isdef  R(p_x, p_y)$,
% \item $ (x \foeq y)^m\isdef    p_x \sqsubset p_y \land p_y \sqsubset p_x$,
% \item $(\exists p. \phi)^m \isdef   \exists p. \phi^m$,
% \item $(\exists x. \phi)^m\isdef   \exists p_x. \texttt{singleton}(p_x) \land \phi^m$,
% \end{itemize}
% where  
% \begin{align*}
%    \texttt{empty}(p) & \isdef \forall q. (p \inc q) 
% \\ \texttt{singleton}(p) & \isdef \neg\texttt{empty}(p) \land 
%    \forall q. (q \inc p \to (\texttt{empty}(q) \lor p\inc q))
% \end{align*}
% is such that
%   \[
%   \bbS,g \models \phi \quad\text{iff}\quad 
%   \bbS[p_{x} \mapsto \{g(x)\}] \models \phi^m).
%   \]
% \end{fact}

