\subsection{$\mu$-Calculi}
\label{sec:onestep-to-mc}

We now see how to associate, with each one-step language $\oslang$, the 
following variant $\mu\oslang$ of the modal $\mu$-calculus.
These logics are of a fairly artificial nature; their main use is to smoothen
the translations from automata to second-order formulas further on.

\begin{definition}
Given a one-step language $\oslang$, we define the language $\mu\oslang$ of the 
\emph{$\mu$-calculus over $\oslang$}  by the following grammar:
% \[
% \varphi \isbnf  q \mid \neg\varphi \mid \varphi\lor\varphi 
%    \mid \nxt{\al}(\varphi_{1},\ldots,\varphi_{n})
%    \mid \mu p. \varphi',
% \]
\[
\varphi \isbnf  
   q \mid \neg q 
   \mid \varphi\lor\varphi \mid \varphi\land\varphi 
   \mid \nxt{\al}(\varphi_{1},\ldots,\varphi_{n})
   \mid \mu p. \varphi'    \mid \nu p. \varphi',
\]
where $p,q \in\Prop$, $\al(a_{1},\ldots,a_{n}) \in \oslang^{+}$ and $\varphi'$ 
is monotone in $p$.

As in the case of the modal $\mu$-calculus $\muML$, we will freely use standard
syntactic concepts and notations related to this language.
\end{definition}

Observe that the language $\mu\oslang$ generally has a wealth of modalities:
one for each one-step formula in $\oslang$.

The semantics of this language is given as follows.

\begin{definition}
Let $\bbS$ be a transition system.
The satisfaction relation $\mmodels$ is defined in the standard way, with the 
following clause for the modality $\nxt{\alpha}$:
\begin{equation}\label{eq:mumod}
\bbS \mmodels \nxt{\al}(\varphi_{1},\ldots,\varphi_{n})
\quad\text{iff}\quad 
(R[s_{I}],V_{\overline{\varphi}}) \models \al(a_{1},\ldots,a_{n}),
\end{equation}
where $V_{\overline{\varphi}}$ is the one-step valuation given by 
\begin{equation}\label{eq:valmod}
V_{\overline{\varphi}}(a_{i}) \isdef 
  \{ t \in R[s_{I}] \mid \bbS.t \mmodels \varphi_{i}\}.
\end{equation}
\end{definition}

\begin{example}
\label{ex:mul1}
\begin{enumerate}[(1)]
\item
If we identify the modalities $\nxt{\Diamond a}$ and $\nxt{\Box a}$ of the basic
modal one-step language $\oml$ (cf.~Definition~\ref{d:oml}) with the standard
$\Diamond$ and $\Box$ operators, we may observe that $\mu(\oml)$ corresponds to
the standard modal $\mu$-calculus: $\mu(\oml) = \muML$.
\item
Consider the one-step formulas 
$\al = \exists x (a_{1}(x) \land \forall y\, a_{2}(y))$,
$\beta = \exists x y (x \not\foeq y \land a_{1}(x) \land a_{1}(y))$, and
$\gamma = \wqu x (a_{1}(x), a_{2}(x))$.
Then $\nxt{\al}(\phi_{1},\phi_{2})$ is equivalent to the modal formula
$\Diamond \phi_{1} \land \Box \phi_{2}$ and 
$\nxt{\beta}(\phi)$ expresses that the current state has at least two 
successors where $\phi$ holds.
The formula $\nxt{\gamma}(\phi_{1},\phi_{2})$ holds at a state $s$ if all 
its successors satisfy $\phi_{1}$ or $\phi_{2}$, while at most finitely
many successors refute $\phi_{2}$.
Neither $\nxt{\beta}$ nor $\nxt{\gamma}$ can be expressed in standard modal 
logic.
\item
If the one-step language $\oslang$ is closed under taking disjunctions  
(conjunctions, respectively), it is easy to see that 
$\nxt{\al\lor\beta}(\ol{\phi}) \equiv \nxt{\al}(\ol{\phi}) \lor 
\nxt{\beta}(\ol{\phi})$ 
($\nxt{\al\land\beta}(\ol{\phi}) \equiv \nxt{\al}(\ol{\phi}) \land 
\nxt{\beta}(\ol{\phi})$, respectively).
\end{enumerate}
\end{example}

Alternatively but equivalently, one may interpret the language
game-theoretically.

\begin{definition}
Given a $\mu\oslang$-formula $\phi$ and a model $\bbS$ we define the 
\emph{evaluation game} $\egame(\varphi,\bbS)$ as the two-player infinite
game %of which the 
whose rules are given in the next table.
% Table~\ref{tab:EGL}.%
%\begin{table}[htb]
\begin{center}
\begin{tabular}{|l|c|l|c|}
\hline
Position & Player & Admissible moves
\\\hline
    $(q,s)$, with $q \in \FV(\phi) \cap \tscolors(s)$ 
  & $\abelard$ 
  & $\emptyset$
\\  $(q,s)$, with $q \in \FV(\phi) \setminus \tscolors(s)$ 
  & $\eloise$ & $\emptyset$
\\  $(\lnot q,s)$, with $q \in \FV(\phi) \cap \tscolors(s)$ 
  & $\eloise$ 
  & $\emptyset$
\\  $(\lnot q,s)$, with $q \in \FV(\phi) \setminus \tscolors(s)$ 
  & $\abelard$ 
  & $\emptyset$
\\ $(\psi_1 \lor \psi_2,s)$ 
  & $\eloise$ 
  & $\{(\psi_1,s),(\psi_2,s) \}$ 
\\  $(\psi_1 \land \psi_2,s)$ 
  & $\abelard$ 
  & $\{(\psi_1,s),(\psi_2,s) \}$ 
\\  $(\nxt{\al}(\varphi_{1},\ldots,\varphi_{n}),s)$ 
  & $\eloise$ 
  & $\{ Z \sse \{ \varphi_{1},\ldots,\varphi_{n} \} \times R[s]
     \mid (R[s],V^{*}_{Z}) \models \al(\ol{a}) \}$ 
\\  $Z \sse  \Sfor(\phi) \times S$
  & $\abelard$
  & $\{ (\psi, s) \mid (\psi,s) \in Z \}$
\\  $(\mu p.\varphi,s)$ & $-$ & $\{(\varphi,s) \}$ 
\\  $(\nu p.\varphi,s)$ & $-$ & $\{(\varphi,s) \}$ 
\\  $(p,s)$, with $p \in \BV(\phi)$ & $-$ & $\{(\delta_p,s) \}$ \\
  \hline
\end{tabular}
\end{center}
 %\caption{Evaluation game for $\mu\oslang$}
%\caption{}
%\label{tab:EGL}
%\end{table}
For the admissible moves at a position of the form 
$(\nxt{\al}(\varphi_{1},\ldots,\varphi_{n}),s)$, we consider the valuation 
$V^{*}_{Z}: \{ a_{1}, \ldots, a_{n} \} \to \pow(R[s])$, given by
$V^{*}_{Z}(a_{i}) \isdef \{ t \in R[s] \mid (\phi_{i},t) \in Z \}$.
The winning conditions of $\egame(\varphi,\bbS)$ are standard: $\eloise$ wins
those infinite matches of which the highest variable that is unfolded infinitely
often during the match is a $\mu$-variable.
\end{definition}

The following proposition, 
stating the adequacy of the evaluation game for the semantics of $\mu\oslang$,
is formulated explicitly for future reference.
We omit the proof, which is completely routine.

\begin{fact}[Adequacy]
\label{f:adeqmu}
For any formula $\phi \in \mu\oslang$ and any model $\bbS$ the following 
equivalence holds:
\[
\bbS \mmodels \phi
\quad\text{iff}\quad 
(\phi,s_{I}) \text{ is a winning position for $\eloise$ in } 
\egame(\varphi,\bbS).
\]
\end{fact}

We will be specifically interested in two fragments of $\mu\oslang$, associated 
with the properties of being noetherian and continuous, respectively, and with 
the associated variants of the $\mu$-calculus $\mu\oslang$ where the use of the 
fixpoint operator $\mu$ is restricted to formulas belonging to these two
respective fragments.

\begin{definition}
Let $\qprop$ be a set  of proposition letters.
We first define the fragment $\noe{\mu\oslang}{\qprop}$ of $\mu\oslang$ of 
formulas that are syntactically \emph{noetherian} in $\qprop$ by the following 
grammar:
\begin{equation*}
   \varphi \isbnf  q
   \mid \psi
   \mid \varphi \lor \varphi
   \mid \varphi \land \varphi
   \mid \nxt{\al}(\varphi_{1},\ldots,\varphi_{n})
   \mid \mu p.\phi'
\end{equation*}
where $q \in \qprop$, $\psi$ is a $\qprop$-free $\muML$-formula,
$\al(a_{1},\ldots,a_{n}) \in \oslang^{+}$ and 
$\phi' \in \noe{\mu\oslang}{\qprop\cup\{p\}}$. 
The \emph{co-noetherian} fragment $\conoe{\mu\oslang}{Q}$ is defined dually.

Similarly, we define the fragment $\cont{\mu\oslang}{\qprop}$ of 
$\mu\oslang$-formulas that are syntactically \emph{continuous} in $\qprop$ as
follows:
\begin{equation*}
   \varphi \isbnf  q
   \mid \psi
   \mid \varphi \lor \varphi
   \mid \varphi \land \varphi
   \mid 
   \nxt{\al}(\varphi_{1},\ldots,\varphi_{k},\psi_{1},\ldots,\psi_{m})
   \mid \mu p.\phi'
\end{equation*}
where $p\in\Prop$, $q \in \qprop$, $\psi$, $\psi_{i}$ are $\qprop$-free 
$\mu\oslang$-formula, $\al(a_{1},\ldots,a_{k},b_{1},\ldots,b_{m}) \in 
\cont{\oslang}{\ol{a}}(\ol{a},\ol{b})$,
and $\phi' \in \cont{\mu\oslang}{\qprop\cup\{p\}}$. 
The \emph{co-continuous} fragment $\cocont{\mu\oslang}{Q}$ is defined dually.
\end{definition}

Based on this we can now define the mentioned variants 
% $\mu_{D}\oslang$ and $\mu_{C}\oslang$ 
of the $\mu$-calculus $\mu\oslang$ where the use of the least (greatest) 
fixpoint operator can only be applied to formulas that belong to, 
respectively, the noetherian (co-noetherian) and continuous (co-continuous)
fragment of the language that we are defining.

\begin{definition}
The formulas of the \emph{alternation-free} $\mu$-calculus $\mu_{D}\oslang$ 
are defined by the following grammar:
\begin{equation*}
   \varphi \isbnf  
      q \mid \neg q 
   \mid \varphi\lor\varphi \mid \varphi\land\varphi 
   \mid \nxt{\al}(\varphi_{1},\ldots,\varphi_{n})
   \mid \mu p. \varphi'    
   \mid \nu p. \varphi'',
\end{equation*} 
where $\al(a_{1},\ldots,a_{n}) \in \oslang^{+}$,
$\phi' \in \mu_{D}\oslang \cap \noe{\mu\oslang}{p}$
and dually $\phi'' \in \mu_{D}\oslang \cap \conoe{\mu\oslang}{p}$.

Similarly, the formulas of the \emph{continuous} $\mu$-calculus $\mu_{C}\oslang$
are given by the grammar
\begin{equation*}
   \varphi \isbnf  
      q \mid \neg q 
   \mid \varphi\lor\varphi \mid \varphi\land\varphi 
   \mid \nxt{\al}(\varphi_{1},\ldots,\varphi_{n})
   \mid \mu p. \varphi'    
   \mid \nu p. \varphi'',
\end{equation*} 
where $\al(a_{1},\ldots,a_{n}) \in \oslang^{+}$,
$\phi' \in \mu_{C}\oslang \cap \cont{\mu\oslang}{p}$
and dually $\phi'' \in \mu_{C}\oslang \cap \cocont{\mu\oslang}{p}$.
\end{definition}

\begin{example}
Following up on Example~\ref{ex:mul1}, it is easy to verify that 
$\mu_{D}\oml = \mudML$ and $\mu_{C}\oml = \mucML$.
\end{example}
%%%
