\begin{abstract}
A landmark result in the study of logics for formal verification is Janin \& 
Walukiewicz's theorem, stating that the modal $\mu$-calculus ($\muML$) is 
equivalent modulo bisimilarity to standard monadic second-order logic (here 
abbreviated as $\smso$), over the class of labelled transition systems (LTSs
for short). 
Our work proves two results of the same kind, one for the alternation-free 
or \emph{noetherian} fragment of $\muML$ ($\mudML$) and one for weak $\mso$
($\wmso$). 
Whereas it was known in the setting of binary trees that $\wmso$ is equivalent 
to an appropriate version of $\mudML$, our analysis shows that the picture 
radically changes once we reason over arbitrary LTSs.
The first theorem that we prove is that, over LTSs, $\mudML$ is equivalent
modulo bisimilarity to \emph{noetherian} $\mso$ ($\nmso$), a newly introduced 
variant of $\smso$ where second-order quantification ranges over ``well-founded''
subsets only. 
Our second theorem starts from $\wmso$, and proves it equivalent modulo 
bisimilarity to a fragment of $\mudML$ defined by a notion of continuity. 
Analogously to Janin \& Walukiewicz's result, our proofs are automata-theoretic
in nature: as another contribution, we introduce classes of parity automata 
characterising the expressiveness of $\wmso$ and $\nmso$ (on tree models) and 
of $\mucML$ and $\mudML$ (for all transition systems).
\end{abstract}
