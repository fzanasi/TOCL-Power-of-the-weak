% !TEX root = ../00CFVZ_TOCL.tex

\section{Automata for $\wmso$}
   \label{sec:autwmso}

In this section we start looking at the automata-theoretic characterisation of
$\wmso$.
That is, we introduce the following automata, corresponding to this version of
monadic second-order logic; these \emph{$\wmso$-automata} are the continuous-weak
automata for the one-step language $\ofoei$, cf.~Definition~\ref{d:ctwk}.

\begin{definition}
A \emph{$\wmso$-automaton} is a continuous-weak automaton for the one-step
language $\ofoei$.
\end{definition}

Recall that our definition of continuous-weak automata is syntactic in nature,
i.e., if $\bbA = \tup{A,\tmap,\pmap,a_I}$ is a $\wmso$-automaton, then for any
pair of states $a,b$ with $a \ord b$ and $b \ord a$, and any $c\in C$, we have
$\tmap(a,c) \in \cont{{\ofoei(A)}^+}{b}$ if $\pmap(a)$ is odd and $\tmap(a,c)
\in \cocont{{\ofoei(A)}^+}{b}$ if $\pmap(a)$ is even.

The main result of this section states one direction of the automata-theoretic
characterisation of $\wmso$.

\begin{theorem}
\label{t:wmsoauto}
There is an effective construction transforming a $\wmso$-formula $\phi$
into a $\wmso$-automaton $\bbA_{\phi}$ that is equivalent
to $\phi$ on the class of trees.
% \models {\phi}$.
\end{theorem}

{\color{blue} The proof of Theorem~\ref{t:wmsoauto}, provided at the end of this section, goes by induction on the complexity of
$\phi$. The case $\phi = \exists p. \psi$ requires most of the work. First, we need to define a closure operation on classes of 
tree models corresponding to the semantics of $\wmso$ quantification.}

\begin{definition}\label{def:tree_finproj-w}
Fix a set $\pprop$ of proposition letters, a proposition letter $p \not\in P$ 
and a language $\mathsf{C}$ of $\pprop\cup\{p\}$-labeled trees.
The \emph{finitary projection} of $\mathsf{C}$ over $p$ is the language of 
$\pprop$-labeled trees defined as 
\[
{\finexists} p.\mathsf{C} \df \{\bbT \mid
\text{ there is a finite $p$-variant } \bbT' \text{ of } \bbT \text{ with }
\bbT' \in \mathsf{C}\}.
\]
%
A collection of classes of tree models is \emph{closed under finitary 
projection over $p$} if it contains the class ${{\finexists} p}.\mathsf{C}$ 
whenever it contains the class $\mathsf{C}$ itself.
\end{definition}

{\color{blue} The case $\phi = \exists p. \psi$ of the proof of Theorem \ref{t:wmsoauto} will require a \emph{projection construction} that, given a $\wmso$-automaton
$\bbA$, provides one recognising ${\finexists p}.\TMod(\bbA)$. In other words, this will  prove that 
the collection of classes that are recognisable by $\wmso$-automata is closed under finitary projection. The next subsection is devoted to a preliminary result, allowing to put $\wmso$-automata in a suitable shape for the projection construction.} %As a preliminary step, this will require a simulation  we  subsection \ref{sec:simulationwmso} to a simulation theorem that puts $\wmso$-automata in a suitable shape for the projection construction}

\subsection{Simulation theorem for $\wmso$-automata}
\label{sec:simulationwmso}

\noindent
For $\smso$-automata, the analogous projection construction {\color{blue}(mimicking $\smso$ quantification)} crucially uses the following
\emph{simulation theorem}: every $\smso$-automaton $\bbA$ is equivalent to a
\emph{non-deterministic} automaton $\bbA'$ \cite{Walukiewicz96}.
Semantically, non-determinism yields the appealing property that every node of
the input model $\bbT$ is associated with at most one state of $\bbA'$ during
the acceptance game--- that means, we may assume $\eloise$'s strategy $f$ in
$\agame(\bbA',\bbT)$ to be \emph{functional} (\emph{cf.}
Definition \ref{def:StratfunctionalFinitary} below).
This is particularly helpful in case we want to define a $p$-variant of $\bbT$
that is accepted by the projection construct on $\bbA'$: our decision whether
to label a node $s$ with $p$ or not, will crucially depend on the value
$f(a,s)$, where $a$ is the unique state of $\bbA'$ that is associated with $s$.
Now, in the case of $\wmso$-automata we are interested in guessing
\emph{finitary} $p$-variants, which requires $f$ to be functional only on a
\emph{finite} set of nodes.
Thus the idea of our simulation theorem is to turn a $\wmso$-automaton $\bbA$
into an equivalent one $\bbA^{\f}$ that behaves non-deterministically on a
\emph{finite} portion of any accepted tree.

For $\smso$-automata, the simulation theorem is based on a powerset construction:
if the starting automaton has set of states $A$, the resulting non-deterministic
automaton is based on ``macro-states'' from the set $\shA$.
Analogously, for $\wmso$-automata we will associate the non-deterministic
behaviour with macro-states.
However, as explained above, the automaton $\bbA^{\f}$ that we construct has to
be non-deterministic just on finitely many nodes of the input and may behave as
$\bbA$ (i.e. in ``alternating mode'') on the others.
To this aim, $\bbA^{\f}$ will be ``two-sorted'', roughly consisting of a copy of
$\bbA$ (with set of states $A$) together with a variant of its powerset construction,
based both on $A$ and $\shA$.
For any accepted $\bbT$, the idea is to make any match $\pi$ of
$\mc{A}(\bbA^{\f},\bbT)$ consist of two parts:
\begin{description}
\item[(\textit{Non-deterministic mode})] For finitely many rounds $\pi$ is
  played on macro-states, i.e. positions belong to the set $\shA \times T$.
  In her strategy player $\exists$ assigns macro-states (from $\shA$) only to
  \emph{finitely many} nodes, and states (from $A$) to the rest.
  Also, her strategy is functional in $\shA$, i.e. it assigns \emph{at most one
  macro-state} to each node.
\item[(\textit{Alternating mode})] At a certain round, $\pi$ abandons
  macro-states and turns into a match of the game $\mc{A}(\bbA,\bbT)$, i.e. all
  subsequent positions are from $A \times T$ (and are played according to a
  not necessarily functional strategy).
\end{description}
Therefore successful runs of $\bbA^{\noet}$ will have the property of processing 
only a \emph{finite} amount of the input with $\bbA^{\noet}$ being in a 
macro-state and all the rest with $\bbA^{\noet}$ behaving exactly as $\bbA$.
We now proceed in steps towards the construction of $\bbA^{\noet}$. 
First, recall from Definition \ref{def:basicform-ofoe} that a \emph{$A$-type} is
just a subset of $A$. 
We now define a notion of liftings for sets of types, which is instrumental in
translating the transition function from states on macro-states.

\begin{definition}
The \emph{lifting} of a type $S \in \pow A$ is defined as the following
$\pow A$-type:
\[
\lift{S} \isdef 
\begin{cases} \{ S \} & \text{ if } S \neq \nada \\
\nada & \text{ if } S = \nada.
\end{cases}
\]
This definition is extended to sets of $A$-types by putting $\lift{\Sigma} 
\isdef \{ \lift{S} \mid S \in \Sigma \}$.
\end{definition}
The distinction between empty and non-empty elements of $\Sigma$ is to ensure 
that the empty type on $A$ is lifted to the empty type on $\pow A$. 
Notice that the resulting set $\lift{\Sigma}$ is either empty or 
contains exaclty one $\pow A$-type.
This property is important for functionality, see below.

Next we define a translation on the sentences associated with the transition
function of the original $\wmso$-automaton.
Following the intuition given above, we want to work with sentences that can be
made true by assigning macro-states (from $\shA$) to finitely many nodes in the
model, and ordinary states (from $A$) to all the other nodes.
Moreover, each node should be associated with \emph{at most one} macro-state,
because of functionality. 
These desiderata are expressed for one-step formulas as \emph{$\shA$-continuity} 
and \emph{$\shA$-separability}, see the Definitions~\ref{def:semnotions} 
and~\ref{d:sep}.
For the language $\ofoei$, Theorem \ref{t:osnf} and Proposition~\ref{p:sep}
guarantee these properties when formulas are in a certain syntactic shape.
The next definition will provide formulas that conform to this particular shape.

\begin{definition}\label{DEF_finitary_lifting}
Let $\phi \in {\ofoei}^+(A)$ be a formula of shape 
$\posdbnfofoei{\vlist{T}}{\Pi}{\Sigma}$ for some $\Pi,\Sigma \subseteq \shA$ 
and $\vlist{T} = \{T_1,\dots,T_k\} \subseteq \shA$. 
We define $\phi^{\fin} \in {\ofoei}^+(A \cup \shA)$ as the formula
$\posdbnfofoei{\lift{\vlist{T}}}{\lift{\Pi} \cup \lift{\Sigma}}{\Sigma}$, that
means,
\begin{equation}\label{eq:unfoldingNablaolque}
\begin{aligned}
\phi^{\fin} \df\ &
    \exists \vlist{x}.\Big(\arediff{\vlist{x}}
      \land \bigwedge_{0 \leq i \leq n} \tau^+_{\lift{T}_i}(x_i)
\land
    \forall z.(\arediff{\vlist{x},z} \to
    \bigvee_{S\in \lift{\Pi} \cup \lift{\Sigma} \cup \Sigma}
       \tau^+_S(z))\Big)
\\ &
    \land \bigwedge_{P\in\Sigma} \qu y.{\tau}^{+}_P(y)
 \land
    \dqu y.\bigvee_{P\in\Sigma} {\tau}^{+}_P(y)
    \end{aligned}
\end{equation}
\end{definition}


We combine the previous definitions to form the transition function for
macro-states.

\begin{definition}\label{PROP_DeltaPowerset}
Let $\bbA = \tup{A,\tmap,\pmap,a_I}$ be a $\wmso$-automaton. 
Fix $c \in C$ and $Q \in \shA$. 
By Theorem \ref{t:osnf}, for some $\Pi,\Sigma \subseteq
\shA$ and $T_i \subseteq A$, there is a sentence $\Psi_{Q,c} \in {\ofoei}^+(A)$
in the basic form $\bigvee \posdbnfofoei{\vlist{T}}{\Pi}{\Sigma}$ such that
$\bigwedge_{a \in Q} \tmap(a,c) \equiv \Psi_{Q,c}$. 
By definition $\Psi_{Q,c}$ is of the form $\bigvee_{i}\phi_i$, with each
$\phi_{i}$ of shape $\posdbnfofoei{\vlist{T}}{\Pi}{\Sigma}$. 
We put $\shDe(Q,c) \isdef \bigvee_{i}\phi_i^{\fin}$, where the translation
$(-)^{\fin}$ is given as in Definition~\ref{DEF_finitary_lifting}. 
Observe that $\shDe(Q,c)$ is of type ${\ofoei}^+(A \cup \shA)$.
\end{definition}

We have now all the ingredients to define our two-sorted automaton.

\begin{definition}\label{def:finitaryconstruct}
Let $\bbA = \tup{A,\tmap,\pmap,a_I}$ be a {\wmso-automaton}.
We define the \emph{finitary construct over $\bbA$} as the automaton
$\bbA^{\fin} = \tup{A^{\fin},\tmap^{\fin},\pmap^{\fin},a_I^{\fin}}$ given by
% \begin{gather*}
%       % \nonumber to remove numbering (before each equation)
%         A^{\fin} \ \df \  A \cup \shA \quad\quad\quad a_I^{\fin} \ \df \  \{(a_I,a_I)\} \quad\quad\quad \pmap^{\fin}(a) \ \df \  \pmap(a) \quad\quad\quad \pmap^{\fin}(R) \ \df \  1 \\
%         \tmap^{\fin}(a,c) \ \df \  \tmap(a,c) \qquad \qquad \qquad
%         \tmap^{\fin}(Q,c) \ \ \df \ \  \shDe(Q,c) \vee \! \! \! \! \bigwedge_{a \in \Ran(Q)} \! \! \! \tmap(a,c).
% \end{gather*}
\[
\begin{array}{lll}
   A^{\fin}   &\df&  A \cup \shA
\\ a_I^{\fin} &\df&  \{a_I\}
\end{array}
%
\hspace*{5mm}
%
\begin{array}{lll}
   \pmap^{\fin}(a) &\df& \pmap(a)
\\ \pmap^{\fin}(R) &\df& 1
\end{array}
%
\hspace*{5mm}
%
\begin{array}{lll}
   \tmap^{\fin}(a,c) &\df& \tmap(a,c)
\\ \tmap^{\fin}(Q,c) &\df&
  \shDe(Q,c) \vee \bigwedge_{a \in Q} \! \! \tmap(a,c).
\end{array}
\]
\end{definition}

\begin{remark} 
In the standard powerset construction of non-deterministic parity automata
(\cite{Walukiewicz02}, see also \cite{Ven08,ArnoldN01})
macro-states are required to be \emph{relations} rather than sets in order to
determine whether a run through macro-states is accepting. 
This is not needed in our construction: macro-states will never be visited
infinitely often in accepting runs, thus they may simply be assigned the
priority $1$.
\end{remark}

The idea behind this definition is that $\bbA^{\fin}$ is enforced to process only a finite portion of any accepted tree while in the non-deterministic mode. This is encoded in game-theoretic terms through the notion of functional and finitary strategy.

\begin{definition}\label{def:StratfunctionalFinitary}
Given a $\wmso$-automaton $\bbA = \tup{A,\tmap,\pmap,a_I}$ and transition system $\bbT$, a strategy $f$ for \eloise in $\mathcal{A}(\bbA,\bbT)$ is \emph{functional in $B \subseteq A$} (or simply functional, if $B=A$) if for each node $s$ in $\bbT$ there is at most one $b \in B$ such that $(b,s)$ is a reachable position in an $f$-guided match. Also $f$ is \emph{finitary} in $B$ if there are only finitely many nodes $s$ in $\bbT$ for which a position $(b,s)$ with $b \in B$ is reachable in an $f$-guided match.
\end{definition}



The next proposition establishes the desired properties of the finitary
construct.

\begin{theorem}[Simulation Theorem for $\wmso$-automata]
\label{PROP_facts_finConstrwmso}
Let $\bbA$ be a $\wmso$-automaton and $\bbA^{\fin}$ its finitary construct.
\begin{enumerate}[(1)]
  \itemsep 0 pt
\item $\bbA^{\fin}$ is a $\wmso$-automaton.\label{point:finConstrAut-w}
\item 
For any tree model $\bbT$, if $(a_I^{\fin},s_I)$ is a winning position for 
$\eloise$ in $\agame(\bbA^{\fin},\bbT)$ , then she has a winning strategy that
is both functional and finitary in $\shA$.
  \label{point:finConstrStrategy}
\item $\bbA \equiv \bbA^{\fin}$. \label{point:finConstrEquiv}
\end{enumerate}
\end{theorem}

\begin{proof}
\begin{enumerate}[(1)]
\item
Observe that any cluster of $\bbA^{\fin}$ involves states of exactly one sort,
either $A$ or $\shA$.
For clusters on sort $A$, weakness and continuity of $\bbA^{\fin}$ follow by 
the same properties of $\bbA$.
For clusters on sort $\shA$, weakness follows by observing that all macro-states
in $\bbA^{\fin}$ have the same priority.
Concerning continuity, by definition of $\tmap^{\fin}$ any macro-state can only
appear inside a formula of the form $\phi^{\fin} =
\posdbnfofoei{\lift{\vlist{T}}}{\lift{\Pi} \cup \lift{\Sigma}}{\Sigma}$ as in
\eqref{eq:unfoldingNablaolque}.
Because $\shA \cap \bigcup\Sigma = \nada$, by Theorem \ref{t:osnf-cont}
$\phi^{\fin}$ is continuous in each $Q \in \shA$.

\item
Let $f$ be a (positional) winning strategy for $\eloise$ in $\mathcal{A}
(\bbA^{\fin},\bbT)@(a_I^{\fin},s_I)$.
We define a strategy $f'$ for $\eloise$ in the same game as follows:
\begin{enumerate}[label=(\alph*),ref=\alph*]
\item 
\label{point:stat2point1}
On basic positions of the form $(a,s) \in A\times T$, let $V: A \to \pow R[s]$
be the valuation suggested by $f$.
We let the valuation suggested by $f'$ be the restriction $V'$ of $V$ to $A$.
Observe that, as no predicate from $A^{\fin}\setminus A =\shA$ occurs in
$\tmap^{\fin}(a,\V(s)) = \tmap(a,\V(s))$, then $V'$ also makes that sentence
true in $\R{s}$.

\item
For winning positions of the form $(R,s) \in \shA \times T$, let $V_{R,s}: 
(\pow A \cup A) \to \pow R[s]$ be the valuation suggested by $f$.
As $f$ is winning, $\tmap^{\fin}(R,\V(s))$ is true in the model $V_{R,s}$.
If this is because the disjunct $\bigwedge_{a \in R} \tmap(a,\V(s))$ is made
true, then we can let $f'$ suggest the restriction to $A$ of $V_{R,s}$,
for the same reason as in \eqref{point:stat2point1}.
  
Otherwise, the disjunct $\shDe(R,\V(s)) = \bigvee_{i}\phi_i^{\fin}$ is made
true.
This means that, for some $i$, $(R[s], V_{R,s}) \models \phi_i^{\fin}$.
Now, by construction of $\phi_i^{\fin}$ as in \eqref{eq:unfoldingNablaolque},
we have $\shA \cap \bigcup\Sigma = \nada$.
By Theorem \ref{t:osnf-cont}, this implies that $\phi_i^{\fin}$ is continuous
in $\shA$.
Thus we have a restriction $V_{R,s}'$ of $V_{R,s}$ that verifies $\phi_i^{\fin}$ 
and assigns only finitely many nodes to predicates from $\shA$. 
Moreover, by construction of $\phi_i^{\fin}$, for each $S \in
\{\lift{T}_1,\dots,\lift{T}_k\}\cup \in \lift{\Pi} \cup \lift{\Sigma}$,
$S$ contains at most one element from $\shA)$. 
Thus, by Proposition~\ref{p:sep}, $\phi_i^{\fin}$ is $\shA$-separable. 
But then we may find a separating valuation $V_{R,s}''\leq_{\shA} V_{R,s}''$ 
such that $V_{R,s}''$ verifies $\phi_i^{\fin}$. 
Separation means that $V_{R,s}''$ associates with each node at most one
predicate from $\shA$, and the fact that $V_{R,s}''\leq_{\shA} V_{R,s}''$,
combined with the $\shA$-continuity of $V_{R,s}'$ ensures $\shA$-continuity of 
$V_{R,s}''$. 
In this case we let $f'$ suggest $V_{R,s}''$ at position $(R,s)$.
\end{enumerate}
% \bigskip\hrule\bigskip
The strategy $f'$ defined as above is immediately seen to be surviving for
$\eloise$. 
% It is also winning, because the set of basic positions on which $f'$
% is defined is a subset of the one of the winning strategy $f$.
It is also winning, since at every basic winning position for $\eloise$, the set
of possible next basic positions offered by $f'$ is a subset of those 
offered by $f$.
By this observation it also follows that any $f'$-guided match visits basic
positions of the form $(R,s) \in \shA \times C$ only finitely many times, as
those have odd parity.
By definition, the valuation suggested by $f'$ only assigns finitely many nodes
to predicates in $\shA$ from positions of that shape, and no nodes from other
positions. It follows that $f'$ is finitary in $\shA$. Functionality in $\shA$ 
also follows immediately by definition of $f'$.
\item 
For the direction from left to right, it is immediate by definition of 
$\bbA^{\fin}$ that a winning strategy for $\eloise$ in $\mc{G} = 
\mathcal{A}(\bbA,\bbT)@(a_I,s_I)$ is also winning for $\eloise$ in 
$\mc{G}^{\fin} = \mathcal{A}(\bbA^{\fin},\bbT)@(a_I^{\fin},s_I)$.
\smallskip

For the direction from right to left, let $f$ be a winning strategy for 
$\eloise$ in $\mc{G}^{\fin}$.
The idea is to define a strategy $f'$ for $\eloise$ in stages, while playing a
match $\pi'$ in $\mc{G}$. 
In parallel to $\pi'$, a shadow match $\pi$ in $\mc{G}^{\fin}$ is maintained, 
where $\eloise$ plays according to the strategy $f$. 
For each round $z_i$, we want to keep the following relation between the two 
matches:
\smallskip
\begin{center}
\fbox{\parbox{12cm}{
Either
\begin{enumerate}[label=(\arabic*),ref=\arabic*]
  \item positions of the form $(Q,s) \in \shA \times T$ and $(a,s) \in A \times T$ occur respectively in $\pi$ and $\pi'$, with $a \in Q$,
\end{enumerate}
or
\begin{enumerate}[label=(\arabic*),ref=\arabic*]
  \item[(2)] the same position of the form $(a,s) \in A \times T$ occurs in both matches.
\end{enumerate}
}}\hspace*{0.3cm}($\ddag$)
\end{center}
\smallskip
The key observation is that, because $f$ is winning, a basic position of the 
form $(Q,s) \in \shA \times T$ can occur only for finitely many initial rounds
$z_0,\dots,z_n$ that are played in $\pi$, whereas for all successive rounds 
$z_n,z_{n+1},\dots$ only basic positions of the form $(a,s) \in A \times T$ are
encountered. 
Indeed, if this was not the case then either $\eloise$ would get stuck or the
highest priority occurring infinitely often would be odd, since states from 
$\shA$ all have priority $1$.

It follows that enforcing a relation between the two matches as in ($\ddag$) suffices to prove that the defined strategy $f'$ is winning for $\eloise$ in $\pi'$. For this purpose, first observe that $(\ddag).1$ holds at the initial round, where the positions visited in $\pi'$ and $\pi$ are respectively $(a_I,s_I) \in A \times T$ and $(\{a_I\},s_I) \in A^{\fin} \times T$. Inductively, consider any round $z_i$ that is played in $\pi'$ and $\pi$, respectively with basic positions $(a,s) \in A \times T$ and $(q,s) \in A^{\fin} \times T$. To define the suggestion of $f'$ in $\pi'$, we distinguish two cases.
\begin{itemize}

\item 
First suppose that $(q,s)$ is of the form $(Q,s) \in \shA\times T$. 
By ($\ddag$) we can assume that $a$ is in $Q$. 
Let $V_{Q,s} :A^{\fin} \rightarrow \pow(\R{s})$ be the valuation suggested by 
$f$, verifying the sentence $\tmap^{\fin}(Q,\V(s))$. 
We distinguish two further cases, depending on which disjunct of 
$\tmap^{\fin}(Q,\V(s))$ is made true by $V_{Q,s}$.
\begin{enumerate}[label=(\roman*), ref=\roman*]
\item 
\label{point:valuation1}
If $(\R{s},V_{Q,s})\models \bigwedge_{b \in Q} \tmap(b,\V(s))$, then we let 
$\eloise$ pick the restriction to $A$ of the valuation $V_{Q,s}$. 
\item 
\label{point:valuation2}
If $(\R{s},V_{Q,s})\models \shDe(Q,\V(s))$, we let $\eloise$ pick a valuation
$V_{a,s}:A \rightarrow \p (\R{s})$ defined by putting, for each $b \in A$:
\begin{align*}
V_{a,s}(b) \isdef \bigcup_{b \in Q'} \{t \in \R{s} \mid t \in V_{Q,s}(Q')\}
               \cup  \{t \in \R{s} \mid t \in V_{Q,s}(b)\} .
\end{align*} 
\end{enumerate}
It can be readily checked that the suggested move is legitimate for $\eloise$
in $\pi$, i.e. it makes $\tmap(a,\V(s))$ true in $\R{s}$.

For case \eqref{point:valuation2}, observe that the nodes assigned to $b$ by
$V_{Q,s}$ have to be assigned to $b$ also by $V_{a,s}$, as they may be necessary
to fulfill the condition, expressed with $\qu$ and $\dqu$ in $\shDe$, that 
infinitely many nodes witness (or that finitely many nodes do not witness) 
some type.

We now show that $(\ddag)$ holds at round $z_{i+1}$. 
If \eqref{point:valuation1} is the case, any next position $(b,t)\in A \times T$
picked by player $\forall$ in $\pi'$ is also available for $\forall$ in $\pi$, 
and we end up in case $(\ddag .2)$. 
Suppose instead that \eqref{point:valuation2} is the case. 
Given a move $(b,t) \in A \times T$ by $\forall$, by definition of $V_{a,s}$ 
there are two possibilities. 
First, $(b,t)$ is also an available choice for $\forall$ in $\pi$, and we end up
in case $(\ddag .2)$ as before. 
Otherwise, there is some $Q' \in \shA$ such that $b$ is in $Q'$ and $\forall$ 
can choose $(Q',t)$ in the shadow match $\pi$. By letting $\pi$ advance at round
$z_{i+1}$ with such a move, we are able to maintain $(\ddag .1)$ also in
$z_{i+1}$.
\item 
In the remaining case, inductively we are given the same basic position $(a,s)
\in A\times T$ both in $\pi$ and in $\pi'$. 
The valuation $V$ suggested by $f$ in $\pi$ verifies $\tmap^{\fin}(a,\V(s)) = 
\tmap(a,\V(s))$, thus we can let the restriction of $V$ to $A$ be the valuation
chosen by $\eloise$ in the match $\pi'$. 
It is immediate that any next move of $\forall$ in $\pi'$ can be mirrored by the
same move in $\pi$, meaning that we are able to maintain the same position 
--whence the relation $(\ddag.1)$-- also in the next round.
\end{itemize}
In both cases, the suggestion of strategy $f'$ was a legitimate move for 
$\eloise$ maintaining the relation $(\ddag)$ between the two matches for any
next round $z_{i+1}$.
It follows that $f'$ is a winning strategy for $\eloise$ in $\mc{G}$.
\end{enumerate}
\end{proof}





\subsection{From formulas to automata}

In this subsection we conclude the proof of Theorem~\ref{t:wmsoauto}.
We first focus on the case of projection with respect to finite sets, which
exploits our simulation result, Theorem~\ref{PROP_facts_finConstrwmso}.
The definition of the projection construction is formulated more generally for
parity automata, as it will be later applied to classes other than
$\AutWC(\ofoei)$.
It clearly preserves the weakness and continuity conditions.

%\subsection{Closure under Finitary Projection}

\begin{definition}\label{DEF_fin_projection}
Let $\bbA = \tup{A, \tmap, \Omega, a_I}$ be a parity automaton on alphabet 
$\p(\pprop \cup \{p\})$.
We define the automaton ${{\exists} p}.\bbA = \tup{A, \tmapProj, \Omega, a_I}$ 
on alphabet $\p\pprop$ by putting
\begin{equation*}
% \nonumber to remove numbering (before each equation)
  \tmapProj(a,c) \ \df \ \tmap(a,c) \qquad \qquad
  \tmapProj(Q,c) \ \df \ \tmap(Q,c) \vee \tmap(Q,c\cup\{p\}).
\end{equation*}
The automaton ${{\exists} p}.\bbA$ is called the \emph{finitary projection
construct of $\bbA$ over $p$}.
\end{definition}


\begin{lemma}\label{PROP_fin_projection}
Let $\bbA$ be a $\wmso$-automaton on alphabet $\p (\pprop \cup \{p\})$.
Then $\bbA^{\fin}$ is a $\wmso$-automaton on alphabet $\p\pprop$ which satisfies
$$\TMod({{\exists} p}.\bbA^{\fin}) \equiv
{{\finexists} p}.\TMod(\bbA).$$
\end{lemma}

\begin{proof}
Unraveling definitions, we need to show that for any tree $\bbT = \tup{T,R,\V \:
\pprop \to \pow T,s_I}$:
$${{\exists} p}.\bbA^{\fin} \text{ accepts } \bbT \text{ iff } \text{there is a
finite }p \text{ -variant }\bbT' \text{of } \bbT \text{  such that } \bbA 
\text{  accepts } \bbT'.
$$
For the direction from left to right, by the equivalence between $\bbA$ and 
$\bbA^{\fin}$ it suffices to show that if ${{\exists} p}.\bbA^{\fin}$ accepts
$\bbT$ then there is a finite $p$-variant $\bbT'$ of $\bbT$ such that 
$\bbA^{\fin}$ accepts $\bbT'$. 
First, we first observe that the properties stated by 
Theorem~\ref{PROP_facts_finConstrwmso}, which hold for $\bbA^{\fin}$ by 
assumption, by construction hold for ${{\exists} p}.\bbA^{\fin}$ as well. 
Thus we can assume that the given winning strategy $f$ for $\eloise$ in 
$\mc{G_{\exists}} = \mc{A}({\finexists p}.\bbA^{\fin},\bbT)@(a_I^{\fin},s_I)$ 
is functional and finitary in $\shA$. 
Functionality allows us to associate with each node $s$ either none or a unique 
state $Q_s \in \shA$ such that $(Q_s,s)$ is winning for $\eloise$. 
We now want to isolate the nodes that $f$ treats ``as if they were labeled with 
$p$''. 
For this purpose, let $V_{s}$ be the valuation suggested by $f$ from a position
$(Q_s,s) \in \shA \times T$. As $f$ is winning, $V_{s}$ makes 
$\tmapProj(Q,\tscolors(s))$ true in $\R{s}$. 
We define a $p$-variant $\bbT' = \tup{T,R,\V' \: \pprop\cup\{p\} \to \pow T,s_I}$
of $\bbT$ by defining $\tscolors' \isdef \tscolors[p \mapsto X_{p}]$, that is, 
by colouring with $p$ all nodes in the following set:
\begin{equation}\label{eq:X_p}
% \nonumber to remove numbering (before each equation)
X_p \isdef \{s \in T\mid (\R{s},V_{s}) \models 
\tmap^{\f}(Q_s,\tscolors(s)\cup\{p\})\}.
\end{equation}
The fact that $f$ is finitary in $\shA$ guarantees that $X_p$ is finite, whence 
$\bbT'$ is a finite $p$-variant.
It remains to show that $\bbA^{\fin}$ accepts $\bbT'$: we claim that $f$ itself
is winning for $\eloise$ in $\mc{G} = (\bbA^{\fin},\bbT')@(a_I,s_I)$. 
In order to see that, let us construct in stages an $f$-guided match $\pi$ 
of $\mc{G}$ and an $f$-guided shadow match $\tilde{\pi}$ of $\mc{G_{\exists}}$.
The inductive hypothesis we want to bring from one round to the next is that the 
same basic position occurs in both matches, as this suffices to prove that $f$ 
is winning for $\eloise$ in $\mc{G}$.

First we consider the case of a basic position $(Q,s) \in A^{\fin} \times T$ 
where $Q \in \shA$. 
By assumption $f$ provides a valuation $V_s$ that makes $\tmapProj(Q,\V(s))$ true
in $\R{s}$. 
Thus $V_s$ verifies either $\tmap^{\fin}(Q,\V(s))$ or $\tmap^{\fin}(Q,\V(s)\cup 
\{p\})$. 
Now, the match $\pi^{\fin}$ is played on the $p$-variant $\bbT'$, where the 
labeling $\V'(s)$ is decided by the membership of $s$ to $X_p$. 
According to \eqref{eq:X_p}, if $V_s$ verifies 
$\tmap^{\fin}(Q,\V(s)\cup \{p\})$ then $s$ is in $X_p$, meaning that it is
labeled with $p$ in $\bbT'$, i.e. $\V'(s) = \V(s)\cup \{p\}$. 
Therefore $V_s$ also verifies $\tmap^{\fin}(Q,\V'(s))$ and it is a 
legitimate move for $\eloise$ in match $\pi^{\fin}$. 
In the remaining case, $V_s$ verifies $\tmap^{\fin}(Q,\V(s))$ but 
falsifies $\tmap^{\fin}(Q,\V(s)\cup \{p\})$, implying by definition that $s$ 
is not in $X_p$. 
This means that $s$ is not labeled with $p$ in $\bbT'$, i.e. $\V'(s) = \V(s)$.
Thus again $V_s$ verifies $\tmap^{\fin}(Q,\V'(s))$ and it is a 
legitimate move for $\eloise$ in match $\pi^{\fin}$.

It remains to consider the case of a basic position $(a,s) \in A^{\fin} \times T$
with $a \in A$ a state. 
By definition $\tmapProj(a,\V(s))$ is just $\tmap^{\fin}(a,\V(s))$. 
As $(a,s)$ is winning, we can assume that no position $(Q,s)$ with $Q$ a 
macro-state is winning according to the same $f$, as making $\tmapProj$-sentences
true never forces $\eloise$ to mark a node both with a state and a macro-state. 
Therefore, $s$ is not in $X_p$ either, meaning that it it is not labeled with 
$p$ in the $p$-variant $\bbT'$ and thus $\V'(s) = \V(s)$. 
This implies that $f$ makes $\tmap^{\fin}(a,\V'(s)) = \tmap^{\fin}(a,\V(s))$
true in $\R{s}$ and its suggestion is a legitimate move for $\eloise$ in match
$\pi^{\fin}$.
In order to conclude the proof, observe that for all positions that we consider
the same valuation is suggested to $\eloise$ in both games: this means that any 
next position that is picked by player $\abelard$ in $\pi^{\fin}$ is also 
available for $\abelard$ in the shadow match $\tilde{\pi}$.
\medskip

We now show the direction from right to left of the statement. Let $\bbT'$ be a finite $p$-variant of
$\bbT$, with labeling function $\tscolors'$, and $g$ a winning strategy for $\exists$ in $\mc{G} = \mathcal{A}(\bbA,\bbT')@(a_I,s_I)$. Our goal is to define a strategy $g'$ for $\exists$ in $\mc{G_{\exists}}$. 
As usual, $g'$ will be constructed in stages, while playing a match $\pi'$ in $\mc{G_{\exists}}$. In parallel to $\pi'$, a \emph{bundle} $\mc{B}$ of $g$-guided shadow matches in $\mc{G}$ is maintained, with the following condition enforced for each round $z_i$:
\smallskip
\begin{center}
\fbox{\parbox{13.5cm}{
\begin{enumerate}
  \item If the current basic position in $\pi'$ is of the form $(Q,s) \in \shA \times T$, then for each $a \in Q$ there is an $g$-guided (partial) shadow match $\pi_a$ at basic position $(a,s) \in A\times T$ in the current bundle $\mc{B}_i$. Also, either $\bbT'_s$ is not $p$-free (i.e., it does contain a node $s'$ with $p \in \tscolors'(s')$) or $s$ has some sibling $t$ such that $\bbT'_t$ is not $p$-free.
  \item Otherwise, the current basic position in $\pi'$ is of the form $(a,s) \in A \times T$ and $\bbT'_s$ is $p$-free. Also, the bundle $\mc{B}_i$ only consists of a single $g$-guided match $\pi_a$ whose current basic position is also $(a,s)$.
\end{enumerate}
}}\hspace*{0.3cm}($\ddag$)
\end{center}
\smallskip
We recall the idea behind ($\ddag$). Point ($\ddag.1$) describes the part of match $\pi'$ where it is still possible to encounter nodes which are labeled with $p$ in $\bbT'$. As $\tmapProj$ only takes the letter $p$ into account when defined on macro-states in $\shA$, we want $\pi'$ to visit only positions of the form $(Q,s) \in \shA \times T$ in that situation. Anytime we visit such a position $(Q,s)$ in $\pi'$, the role of the bundle is to provide one $g$-guided shadow match at position $(a,s)$ for each $a \in Q$.
Then $g'$ is defined in terms of what $g$ suggests from those positions.

 Point ($\ddag.2$) describes how we want the match $\pi'$ to be
 played on a $p$-free subtree: as any node that one might encounter has the same label in $\bbT$ and $\bbT'$,
it is safe to let ${\finexists p}.\bbA^{\fin}$ behave as $\bbA$ in such situation. Provided that the two matches visit the same basic positions, of the form $(a,s)\in A \times T$, we can let $g'$ just copy $g$.

The key observation is that, as $\bbT'$ is a \emph{finite} $p$-variant of $\bbT$, nodes labeled with $p$ are reachable only for finitely many rounds of $\pi'$. This means that, provided that ($\ddag$) hold at each round, ($\ddag.1$) will describe an initial segment of $\pi'$, whereas ($\ddag.2$) will describe the remaining part. Thus our proof that $g'$ is a winning strategy for $\exists$ in $\mc{G}_{\exists}$ is concluded by showing that ($\ddag$) holds for each stage of construction of $\pi'$ and $\mc{B}$.

For this purpose, we initialize $\pi'$ from position $(\shai,s) \in \shA\times T$ and the bundle $\mc{B}$ as $\mc{B}_0 = \{\pi_{a_I}\}$, with $\pi_{a_I}$ the partial $g$-guided match consisting only of the position $(a_I,s)\in A\times T$. The situation described by ($\ddag .1$) holds at the initial stage of the construction.
Inductively, suppose that at round $z_i$ we are given a position $(q,s) \in A^{\f} \times T$ in $\pi^{\f}$ and a bundle $\mc{B}_i$ as in ($\ddag$). To show that ($\ddag$) can be maintained at round $z_{i+1}$, we distinguish two cases, corresponding respectively to situation ($\ddag.1$) and ($\ddag.2$) holding at round $z_i$.
\begin{enumerate}[label = (\Alph*), ref = \Alph*]
%\yvwarning{Notation `$q$' is confusing, see $V'(q)$ below FZ: I corrected $q$ into $q'$ below}
  \item If $(q,s)$ is of the form $(Q,s) \in \shA \times T$, by inductive hypothesis we are given with $g$-guided shadow matches $\{\pi_a\}_{a \in Q}$ in $\mc{B}_i$. For each match $\pi_a$ in the bundle, we are provided with a valuation $V_{a,s}: A \rightarrow \p (\R{s})$ making $\tmap(a,\tscolors'(s))$ true. Then we further distinguish the following two cases.
\begin{enumerate}[label = (\roman*), ref = \roman*]
  \item \label{point:TsNotPFree} Suppose first that $\bbT'_s$ is not $p$-free. We let the suggestion $V' \: A^{\f} \to \p (\R{s})$ of $g'$ from position $(Q,s)$ be defined as follows:
       \begin{align*}
       % \nonumber to remove numbering (before each equation)
       %\widetilde{V}_{Q,s}(Q') \ \df \  \bigcup_{a \in \Ran(Q),\ b \in \Ran(Q')}\{t\ \in \R{s}|\ t \in V_{a,s}(b)\}.
       V'(q') \isdef \begin{cases}
               \bigcap\limits_{\substack{(a,b) \in q',\\ a \in Q}}\{t\ \in \R{s} \mid t \in V_{a,s}(b)\}               & q' \in \shA \\[2em]
               \bigcup\limits_{a \in Q} \{t\ \in \R{s} \mid t \in V_{a,s}(q') \text{ and }\bbT'.t\text{ is $p$-free}\}              & q' \in A.
               %\\[1.5em]               \hspace{.6cm}\nada & \text{otherwise.}
           \end{cases}
       \end{align*}
The definition of $V'$ on $q' \in \shA$ is standard 
(\emph{cf.}~\cite[Prop. 2.21]{Zanasi:Thesis:2012}) and guarantees a 
correspondence between the states assigned by the valuations 
$\{V_{a,s}\}_{a \in Q}$ and the macro-states assigned by $V'$. 
The definition of $V'$ on $q' \in A$ aims at fulfilling the conditions, 
expressed via $\qu$ and $\dqu$, on the number of nodes in $\R{s}$ witnessing 
(or not) some $A$-types. 
Those conditions are the ones that $\shDe(Q,\tscolors'(s))$ --and thus also
$\tmap^{\f}(Q,\tscolors'(s))$-- ``inherits'' by $\bigwedge_{a \in R} 
\tmap(a,\tscolors'(s))$, by definition of $\shDe$. 
Notice that we restrict $V'(q')$ to the nodes $t \in V_{a,s}(q')$ such that 
$\bbT'.t$ is $p$-free.
As $\bbT'$ is a \emph{finite} $p$-variant, only \emph{finitely many} nodes in
$V_{a,s}(q')$ will not have this property.
Therefore their exclusion, which is crucial for maintaining condition ($\ddag$) 
(\emph{cf.}~case \eqref{point:ddag2CardfromMacro} below), does not influence 
the fulfilling of the cardinality conditions expressed via $\qu$ and $\dqu$ in 
$\shDe(Q,\tscolors'(s))$.

       On the base of these observations, one can check that $V'$ makes $\shDe(Q,\tscolors'(s))$--and thus also $\tmap^{\f}(Q,\tscolors'(s))$--true in $\R{s}$. In fact, to be a legitimate move for $\exists$ in $\pi'$, $V'$ should make $\tmapProj(Q,\tscolors(s))$ true: this is the case, for $\tmap^{\f}(Q,\tscolors'(s))$ is either equal to $\tmap^{\f}(Q,\tscolors(s))$, if $p \not\in \tscolors'(s)$, or to $\tmap^{\f}(Q,\tscolors(s)\cup\{p\})$ otherwise. In order to check that we can maintain $(\ddag)$, let $(q',t) \in A^{\f} \times T$ be any next position picked by $\forall$ in $\pi'$ at round $z_{i+1}$. As before, we distinguish two cases:
       \begin{enumerate}[label = (\alph*), ref = \alph*]
         \item If $q'$ is in $A$, then, by definition of $V'$, $\forall$ can choose $(q',t)$ in some shadow match $\pi_a$ in the bundle $\mc{B}_i$. We dismiss the bundle --i.e. make it a singleton-- and bring only $\pi_a$ to the next round in the same position $(q',t)$. Observe that, by definition of $V'$, $\bbT'.t$ is $p$-free and thus ($\ddag.2$) holds at round $z_{i+1}$. \label{point:ddag2CardfromMacro}
         \item Otherwise, $q'$ is in $\shA$. The new bundle $\mc{B}_{i+1}$ is given in terms of the bundle $\mc{B}_i$: for each $\pi_a \in \mc{B}_i$ with $a\in Q$, we look if for some $b \in q'$ the position $(b,t)$ is a legitimate move for $\forall$ at round $z_{i+1}$; if so, then we bring $\pi_a$ to round $z_{i+1}$ at position $(b,t)$ and put the resulting (partial) shadow match $\pi_b$ in $\mc{B}_{i+1}$. Observe that, if $\forall$ is able to pick such position $(q',t)$ in $\pi'$, then by definition of $V'$ the new bundle $\mc{B}_{i+1}$ is non-empty and consists of an $g$-guided (partial) shadow match $\pi_b$ for each $b \in q'$. In this way we are able to keep condition ($\ddag.1$) at round $z_{i+1}$.
       \end{enumerate}
    \item Let us now consider the case in which $\bbT'_s$ is $p$-free. We let $g'$ suggest the valuation $V'$ that assigns to each node $t \in \R{s}$ all states in $\bigcup_{a \in Q}\{b \in A\ |\ t \in V_{a,s}(b)\}$. It can be checked that $V'$ makes $\bigwedge_{a \in Q} \tmap(a,\tscolors'(s))$ -- and then also $\tmap^{\f}(Q,\tscolors'(s))$ -- true in $\R{s}$. As $p \not\in \tscolors(s)=\tscolors'(s)$, it follows that $V'$ also makes $\tmapProj(Q,\tscolors(s))$ true, whence it is a legitimate choice for $\exists$ in $\pi'$. Any next basic position picked by $\forall$ in $\pi'$ is of the form $(b,t) \in A \times T$, and thus condition ($\ddag.2$) holds at round $z_{i+1}$ as shown in (i.a). %\eqref{point:ddag2CardfromMacro}
  \end{enumerate}
  \item In the remaining case, $(q,s)$ is of the form $(a,s) \in A \times T$ and by inductive hypothesis we are given with a bundle $\mc{B}_i$ consisting of a single $f$-guided (partial) shadow match $\pi_a$ at the same position $(a,s)$. Let $V_{a,s}$ be the suggestion of $\exists$ from position $(a,s)$ in $\pi_a$. Since by assumption $s$ is $p$-free, we have that $\tscolors'(s) = \tscolors(s)$, meaning that $\tmapProj(a,\tscolors(s))$ is just $\tmap(a,\tscolors(s)) = \tmap(a,\tscolors'(s))$. Thus the restriction $V'$ of $V$ to $A$ makes $\tmap(a,\tscolors'(t))$ true and we let it be the choice for $\exists$ in $\tilde{\pi}$. It follows that any next move made by $\forall$ in $\tilde{\pi}$ can be mirrored by $\forall$ in the shadow match $\pi_a$.
%Version with minimality:
%      It follows that $\tmapProj(a,\tscolors(t))$ is just $\tmap(a,\tscolors(t)) = \tmap(a,\tscolors'(t))$ and the same valuation suggested by $f$ in $\pi_a$ is a legitimate choice for $\exists$ in $\tilde{\pi}$. By letting $\exists$ choose such valuation, it follows that any next move made by $\forall$ in $\tilde{\pi}$ can be mirrored by $\forall$ in the shadow match $\pi_a$.
\end{enumerate}
%As explained above, since $\bbT'$ is a noetherian $p$-variant, then ($\ddag .1$) holds for finitely many stages of construction of $\tilde{\pi}$, whereas ($\ddag .2$) holds for all the remaining stages, by construction of $\tilde{f}$. It follows that this strategy is winning for $\exists$ in $\tilde{G}$.
\end{proof}

%%%%%%
%%%%%% BOOLEANS
%%%%%%

\subsubsection{Closure under Boolean operations}

Here we show that the collection of $\Aut(\wmso)$-recognizable classes of tree
models is closed under the Boolean operations.
%
For union, we use the following result, leaving the straightforward proof as an 
exercise to the reader.

\begin{lemma}
\label{t:cl-dis}
Let $\bbA_{0}$ and $\bbA_{1}$ be $\wmso$-automata.
Then there is a $\wmso$-automaton $\bbA$ such that $\TMod(\bbA)$ is the
union of $\TMod(\bbA_{0})$ and $\TMod(\bbA_{1})$.
\end{lemma}

For closure under complementation we reuse the general results established in 
Section \ref{sec:parityaut} for parity automata.

\begin{lemma}
\label{t:cl-cmp}
Let $\bbA$ be an $\wmso$-automaton.
Then the automaton $\overline{\bbA}$ defined in Definition~\ref{d:caut} is a
$\wmso$-automaton recognizing the complement of $\TMod(\bbA)$.
\end{lemma}

\begin{proof} 
It suffices to check that Proposition \ref{prop:autcomplementation} restricts 
to the class $\AutWC(\ofoei)$ of $\wmso$-automata. 
First, the fact that $\ofoei$ is closed under Boolean duals 
(Definition~\ref{def:concreteduals}) implies that it holds for the class
$\Aut(\ofoei)$. 
It then remains to check that the dual automata construction 
$\overline{(\cdot)}$ preserves weakness and continuity. 
But this is straightforward, given the self-dual nature of these properties.
\end{proof}


%%%%
%%%% PROOF THEOREM
%%%%

We are now finally able to conclude the direction from formulas to automata of 
the characterisation theorem.

\begin{proof}[of Theorem \ref{t:wmsoauto}] The proof is by induction on $\phi$.
\begin{itemize}

\item 
For the base case, we consider the atomic formulas $\here{p}$, $p \inc q$ and 
$R(p,q)$.

\begin{itemize}
\item
The $\wmso$-automaton $\bbA_{\here{p}} = \tup{A,\tmap,\Omega,a_I}$ is given 
by putting
\begin{eqnarray*}
        A  \  \df \  \{a_0,a_1\}  \qquad \qquad  
	a_I  \   \df  \  a_0      \qquad \qquad   
	\Omega(a_0)  \  \df \  0  \qquad \qquad
        \Omega(a_1)  \  \df \  0 
\\ \tmap(a_0,c)  \  \df \  \left\{
	\begin{array}{ll}
           \forall x. a_1(x)  &  \mbox{if } p \in c
	\\ \bot               &  \mbox{otherwise.}
	\end{array}
\right. \qquad 
  \tmap(a_1,c)  \  \df \  \left\{
	\begin{array}{ll}
           \forall x. a_1(x)  &  \mbox{if } p \not\in c
	\\ \bot &  \mbox{otherwise.}
	\end{array}
\right.
\end{eqnarray*}
\item
% If $\phi = p \inc q$, as the corresponding $\wmso$-automaton we can take
% $\bbA_{p\inc q} = \tup{A,\tmap,\Omega,a_I}$, where
The $\wmso$-automaton $\bbA_{p\inc q} = \tup{A,\tmap,\Omega,a_I}$ is given by
$A \isdef \{ a \}$, $a_{I} \isdef a$, $\pmap(a) \isdef 0$ and 
$\tmap(a,c) \isdef \forall x\, a(x)$ if $p \not\in c$ or $q\in c$, and
$\tmap(a,c) \isdef \bot$ otherwise.

\item
The $\wmso$-automaton $\bbA_{R(p,q)} = \tup{A,\tmap,\Omega,a_I}$ is given below:
\begin{eqnarray*}
        A  \  \df \  \{a_0,a_1\}  \qquad \qquad  
	a_I  \   \df  \  a_0      \qquad \qquad   
	\Omega(a_0)  \  \df \  0  \qquad \qquad
        \Omega(a_1)  \  \df \  1 
\\ \tmap(a_0,c)  \  \df \  \left\{
	\begin{array}{ll}
           \exists x. a_1(x) \wedge \forall y. a_0(y)  &  \mbox{if }p \in c
	\\ \forall x\ (a_0(x))  &  \mbox{otherwise.}
	\end{array}
\right. \qquad 
  \tmap(a_1,c)  \  \df \  \left\{
	\begin{array}{ll}
        \top  &  \mbox{if }q \in c \\
        \bot  &  \mbox{otherwise}
	\end{array}
\right.
\end{eqnarray*}
% Note that the $\smso$-automaton for $R(p,q)$ provided in
% \cite[Ex. 2.5]{Zanasi:Thesis:2012} is \emph{not} a $\wmso$-automaton, as the
% continuity property does not hold.
\end{itemize}

\item
For the Boolean cases, where $\phi = \psi_1 \vee \psi_2$ or $\phi = \neg\psi$
we refer to the Boolean closure properties that we just established in the
Lemmas~\ref{t:cl-dis} and~\ref{t:cl-cmp},
respectively.

\item
The case $\phi = \exists p. \psi$ follows by the following chain of
equivalences, where $\bbA_{\psi}$ is given by the inductive hypothesis and
${\finexists p}.\bbA_{\psi}$ is constructed according to
Definition~\ref{DEF_fin_projection}:
\begin{alignat*}{2}
{\finexists p}.\bbA_{\psi} \text{ accepts }\mb{T}
   & \text{ iff }
     \bbA_{\psi} \text{ accepts } \mb{T}[p \mapsto X],
     \text{ for some } X \sse_{\om} T
   & \quad\text{(Lemma~\ref{PROP_fin_projection})}
\\ & \text{ iff }
     \mb{T}[p \mapsto X] \models \psi,
     \text{ for some } X \sse_{\om} T
   & \quad\text{(induction hyp.)}
\\ & \text{ iff }
    \mb{T} \models \exists p. \psi
   & \quad\text{(semantics $\wmso$)}
\end{alignat*}
\end{itemize}
\end{proof}

%\newpage 