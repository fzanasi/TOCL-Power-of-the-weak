% !TEX root = ../00CFVZ_TOCL.tex
% \section{Expressiveness modulo bisimilarity}\label{sec:expresso}

\section{Expressiveness modulo bisimilarity}
   \label{sec:expresso}

In this Section we use the tools developed in the previous parts to prove the
main results of the paper on expressiveness modulo bisimilarity, viz., 
Theorem~\ref{t:11} stating
\begin{eqnarray}
% \[\begin{array}{rcl}
%\label{eq:1}
%\eqno{(*)}
   \AFMC &\equiv& \nmso / {\bis} 
   \label{eq:z11}
\\[1mm] \mucML &\equiv& \wmso / {\bis} 
   \label{eq:z12}
% \end{array}\]
\end{eqnarray}

\begin{proofof}{Theorem~\ref{t:11}}
The structure of the proof is the same for the statements \eqref{eq:z11} and
\eqref{eq:z12}.
In both cases, we will need three steps to establish a link between the modal 
language on the left hand side of the equation to the bisimulation-invariant 
fragment of the second-order logic on the right hand side.

The first step is to connect the fragments $\AFMC$ and $\mucML$ of the modal 
$\mu$-calculus to, respectively, the weak and the continuous-weak automata for
first-order logic without equality.
That is, in Theorem~\ref{t:mlaut} below we prove the following:
\begin{eqnarray}
   \AFMC &\equiv& \AutW(\ofo)
   \label{eq:z21}
\\[1mm] \mucML &\equiv&  \AutWC(\ofo)
   \label{eq:z22}
\end{eqnarray}

Second, the main observations that we shall make in this section is that 
\begin{eqnarray}
\AutW(\ofo)  &\equiv& \AutW(\ofoe)/{\bis} 
   \label{eq:z31}
\\[1mm] 
\AutWC(\ofo) &\equiv&  \AutWC(\ofoei)/ {\bis} 
   \label{eq:z32}
\end{eqnarray}
That is, for \eqref{eq:z31} we shall see in Theorem~\ref{t:bi-aut} below that a
weak $\ofoe$-automaton $\bbA$ is bisimulation invariant iff it is equivalent to
a weak $\ofo$-automaton $\bbA^{\tmod}$ (effectively obtained from $\bbA$);
% which can be effectively obtained from $\bbA$ via a one-step translation applied
% to the transition map of $\bbA$
and similarly for \eqref{eq:z32}.

Finally, we use the automata-theoretic characterisations of $\nmso$ and $\wmso$
that we obtained in earlier sections:
\begin{eqnarray}
\AutW(\ofoe)   &\equiv&  \nmso
   \label{eq:z41}
\\[1mm] 
\AutWC(\ofoei) &\equiv&  \wmso 
   \label{eq:z42}
\end{eqnarray}

Then it is obvious that the equation \eqref{eq:z11} follows from \eqref{eq:z21},
\eqref{eq:z31} and \eqref{eq:z41}, while similarly 
\eqref{eq:z12} follows from \eqref{eq:z22}, \eqref{eq:z32} and \eqref{eq:z42}.
\end{proofof}

It is left to prove the equations \eqref{eq:z21} and \eqref{eq:z22}, and 
\eqref{eq:z31} and  \eqref{eq:z32}; this we will take care of in the two 
subsections below.

\subsection{Automata for $\AFMC$ and $\mucML$}

In this subsection we consider the automata corresponding to the 
continuous and the alternation-free $\mu$-calculus. 
That is, we verify the equations \eqref{eq:z21} and \eqref{eq:z22}.

\begin{theorem}
\label{t:mlaut}
\begin{enumerate}
\item 
There is an effective construction transforming a formula $\phi \in \muML$ into
an equivalent automaton in $\Aut(\ofo)$, and vice versa.
\item 
There is an effective construction transforming a formula $\phi \in \AFMC$ into
an equivalent automaton in $\AutW(\ofo)$, and vice versa.
\item 
There is an effective construction transforming a formula $\phi \in \mucML$ into
an equivalent automaton in $\AutWC(\ofo)$, and vice versa.
\end{enumerate}
\end{theorem}

\begin{proof}
In each of these cases the direction from left to right is easy to verify,
so we omit details.
For the opposite direction, we focus on the hardest case, that is, we will 
only prove that $\AutWC(\ofo) \leq \mucML$.
By Theorem~\ref{t:autofor} it suffices to show that $\mu_{C}\ofo \leq \mucML$,
and we will in fact provide a direct, inductively defined, truth-preserving 
translation $(\cdot)^{t}$ from $\mu_{C}\ofo(\pprop)$ to $\mucML(\pprop)$.
Inductively we will ensure that, for every set $\qprop \sse \pprop$:
\begin{equation}
\label{eq:zz1}
\phi \in \cont{\mu\ofo}{\qprop} % \cap \mu_{C}\ofo
\text{ implies } \phi^{t} \in 
\cont{\muML}{\qprop} % \cap \mucML,
\end{equation}
and that the dual property holds for cocontinuity.

Most of the clauses of the definition of the translation $(\cdot)^{t}$ are 
completely standard: for the atomic clause we take $p^{t} \isdef p$ and
$(\neg p)^{t} \isdef \neg p$, for the boolean connectives we define 
$(\phi_{0}\lor\phi_{1})^{t} \isdef \phi_{0}^{t} \lor \phi_{1}^{t}$ and 
$(\phi_{0}\land\phi_{1})^{t} \isdef \phi_{0}^{t} \land \phi_{1}^{t}$, and 
for the fixpoint operators we take $(\mu p. \phi)^{t} \isdef \mu p. \phi^{t}$ 
and $(\nu p. \phi)^{t} \isdef \nu p. \phi^{t}$ ---
to see that the latter clauses indeed provide formulas in $\mucML$ we use
\eqref{eq:zz1} and its dual.
In all of these cases it is easy to show that \eqref{eq:zz1} holds (or remains
true, in the inductive cases).

The only interesting case is where $\phi$ is of the form 
$\nxt{\al}(\phi_{1},\ldots,\phi_{n})$.
By definition of the language $\mu_{C}\ofo$ we may assume that 
$\al(a_{1},\ldots,a_{n}) \in \cont{\ofo(A)}{B}$, where 
$A = \{ a_{1},\ldots,a_{n} \}$ and $B = \{ a_{1}, \ldots, a_{k} \}$,
that for each $1 \leq i \leq k$ the formula $\phi_{i}$ belongs to the set 
$\cont{\mu_{C}\ofo}{\qprop}$ and that for each $k+1\leq j \leq n$ the formula 
$\phi_{j}$ is $\qprop$-free.
It follows by the induction hypothesis 
that $\phi_{l} \equiv \phi_{l}^{t} \in \mucML$ for each $l$, 
that $\phi_{i}^{t} \in \cont{\muML}{\qprop}$ for each $1 \leq i \leq k$,
and that  the formula $\phi_{j}^{t}$ is $\qprop$-free for each $k+1\leq j \leq n$.
The key observation is now that by Theorem~\ref{t:osnf-cont} we may without 
loss of generality assume that $\al$ is in \emph{normal form}; that is, a 
disjunction of formulas of the form $\al_{\Sigma,\Pi} = \posdgbnfofo{\Sigma}{\Pi}$,
where every $\Sigma$ and $\Pi$ is a subset of $\pow (A)$, $B \cap \bigcup\Pi =
\nada$ for every $\Pi$, and 
\[
\posdgbnfofo{\Sigma}{\Pi} \isdef 
\bigwedge_{S\in\Sigma} \exists x \bigwedge_{a \in S} a(x) 
\;\land\; \forall x \bigvee_{S\in\Pi} \bigwedge_{a \in S} a(x) 
\]
We now define
\[\begin{array}{rll}
%\bigvee \big(\nxt{\al_{\Sigma}}(\ol{\phi})\big)^{t} 
 \big(\nxt{\al_{\Sigma,\Pi}}(\ol{\phi})\big)^{t} 
& \isdef &
    \bigwedge_{S\in\Sigma} \Diamond \bigwedge_{a_{l} \in S} \phi_{l}^{t}
     \;\land\; 
     \Box \bigvee_{S\in\Pi} \bigwedge_{a_{j} \in S} \phi_{j}^{t} 
\\ \phi^{t} & \isdef & {\bigvee} \big(\nxt{\al_{\Sigma,\Pi}}(\ol{\phi})\big)^{t}
\end{array}\]
It is then obvious that $\phi$ and $\phi^{t}$ are equivalent, so it remains to 
verify \eqref{eq:zz1}.
But this is immediate by the observation that all formulas $\phi_{j}^{t}$
in the scope of the $\Box$ are associated with an $a_{j}$ belonging to a set 
$S \sse A$ that has an empty intersection with the set $B$; that is, each 
$a_{j}$ belongs to the set $\{ a_{k+1}, \ldots, a_{n}\}$ and so $\phi_{j}^{t}$
is $\qprop$-free.
\end{proof}

\subsection{Bisimulation invariance, one step at a time}
\label{ss:bisinv}

In this subsection we will show how the bisimulation invariance results in this
paper can be proved by automata-theoretic means.
Following Janin \& Walukiewicz~\cite{Jan96}, 
we will define a construction that, for $\oslang \in \{{\ofoe},{\ofoei}\}$, 
transforms an arbitrary $\oslang$-automaton $\bbA$ into an $\ofo$-automaton 
$\bbA^{\tmod}$ such that $\bbA$ is bisimulation invariant iff it is equivalent
to $\bbA^{\tmod}$.
In addition, we will make sure that this transformation preserves both the
weakness and the continuity condition.
The operation $(\cdot)^{\tmod}$ is completely determined by the following 
translation at the one-step level.

\begin{definition}
Recall from Theorem~\ref{t:osnf} that any formula in ${\ofoe}^+(A)$ is 
equivalent to a disjunction of formulas of the form 
$\posdbnfofoe{\vlist{T}}{\Sigma}$, whereas any formula in ${\ofoei}^+(A)$ is 
equivalent to a disjunction of formulas of the form 
$\posdbnfolque{\vlist{T}}{\Pi}{\Sigma}$. 
Based on these normal forms, for both one-step languages $\oslang={\ofoe}$ and 
$\oslang={\ofoei}$, we define the translation 
$(\cdot)^{\tmod} : {\oslang}^+(A) \to \ofo^+(A)$ by setting
% \[
% \Big( \posdbnfofoe{\vlist{T}}{\Sigma} \Big)^{\tmod} = 
% \Big( \posdbnfolque{\vlist{T}}{\Pi}{\Sigma} \Big)^{\tmod} 
% \df
% \bigwedge_{i} \exists x_i. \tau^+_{T_i}(x_i) \land 
% \forall x. \bigvee_{S\in\Sigma} \tau^+_S(x),
% \]
\[
\left.\begin{array}{l}
   \Big( \posdbnfofoe{\vlist{T}}{\Sigma} \Big)^{\tmod} 
\\ \Big( \posdbnfolque{\vlist{T}}{\Pi}{\Sigma} \Big)^{\tmod} 
\end{array}\right\}
\df \bigwedge_{i} \exists x_i. \tau^+_{T_i}(x_i) \land 
\forall x. \bigvee_{S\in\Sigma} \tau^+_S(x),
\]
and for $\al = \bigvee_{i} \al_{i}$ we define $\al^{\tmod} \df \bigvee 
\al_{i}^{\tmod}$.
\end{definition}

\noindent
This definition propagates to the level of automata in the obvious way.

\begin{definition}
Let $\oslang\in \{{\ofoe},{\ofoei}\}$ be a one-step language.
Given an automaton $\bbA = \tup{A,\tmap,\pmap,a_{I}}$ in $\Aut(\oslang)$, define 
the automaton $\bbA^{\tmod} \df \tup{A,\tmap^{\tmod},\pmap,a_{I}}$ in 
$\Aut(\ofo)$ by putting, for each $(a,c) \in A \times C$:
\[
\tmap^{\tmod}(a,c) \df (\tmap(a,c))^{\tmod}.
\]
\end{definition}

The main result of this section is the theorem below.
For its formulation, recall that $\bbS^{\om}$ is the $\om$-unravelling of 
the model $\bbS$ (as defined in the preliminaries).
As an immediate corollary of this result, we see that \eqref{eq:z31} and
\eqref{eq:z32} hold indeed.

\begin{theorem}
\label{t:bi-aut}
Let $\oslang\in \{{\ofoe},{\ofoei}\}$ be a one-step language and let $\bbA$ be an
$\oslang$-automaton.

\begin{enumerate}
\item
The automata $\bbA$ and $\bbA^{\tmod}$ are related as follows, for every model $\bbS$:
\begin{equation}
\label{eq:crux}
\bbA^{\tmod} \text{ accepts } \bbS \text{ iff } \bbA \text{ accepts
} \bbS^{\om}.
\end{equation}
\item
The automaton $\bbA$ is bisimulation invariant iff $\bbA \equiv \bbA^{\tmod}$.
\item
If $\bbA\in \AutW(\oslang)$ then $\bbA^{\tmod}\in \AutW(\ofo)$, and 
if $\bbA\in \AutWC(\ofoei)$ then $\bbA^{\tmod}\in \AutWC(\ofo)$.
\end{enumerate}
\end{theorem}


The remainder of this section is devoted to the proof of Theorem~\ref{t:bi-aut}.
The key proposition is the following observation on the one-step translation,
that we take from the companion paper~\cite{carr:mode18}.

\begin{proposition}
\label{p-1P}
Let $\oslang\in \{{\ofoe},{\ofoei}\}$.
For every one-step model $(D,V)$ and every $\al \in \oslang^+(A)$ we have
\begin{equation}
\label{eq-1cr}
(D,V) \models \alpha^{\tmod} \text{ iff } (D\times \om,V_\pi) \models \alpha,
\end{equation}
where $V_{\pi}$ % =  f^{-1} \circ V$
 is the induced valuation given by 
$V_{\pi}(a) \df \{ (d,k) \mid d \in V(a), k\in\omega\}$.
\end{proposition}

% \begin{proof}
% We prove the claim for $\oslang={\ofoei}$, the other case being similar.
% Clearly it suffices to prove \eqref{eq-1cr} for formulas of the form
% $\al = \posdbnfolque{\vlist{T}}{\Pi}{\Sigma}$.
% \smallskip
% 
% \noindent\fbox{$\Rightarrow$} 
% Assume $(D,V) \models \alpha^{\tmod}$, we will show that 
% $(D\times \omega,V_\pi) \models \posdbnfolque{\vlist{T}}{\Pi}{\Sigma}$.
% Let $d_i$ be such that $\tau_{T_i}^+(d_i)$ in $(D,V)$. 
% It is clear that the $(d_i,i)$ provide \emph{distinct} elements satisfying 
% $\tau_{T_i}^+((d_i,i))$ in $(D\times\omega,V_{\pi})$ and therefore the 
% first-order existential part of $\alpha$ is satisfied. 
% With a similar but easier argument it is straightforward that the existential 
% generalized quantifier part of $\alpha$ is also satisfied.
% For the universal parts of $\posdbnfolque{\vlist{T}}{\Pi}{\Sigma}$ it is enough
% to observe that, because of the universal part of $\alpha^\circ$, \emph{every}
% $d\in D$ realizes a positive type in $\Sigma$. 
% By construction, the same applies to $(D\times\omega,V_{\pi})$, 
% therefore this takes care of both universal quantifiers.
% \medskip
% 		
% \noindent\fbox{$\Leftarrow$} 
% Assuming that $(D\times \omega,V_\pi) \models 
% \posdbnfolque{\vlist{T}}{\Pi}{\Sigma}$,
% we will show that $(D,V) \models \alpha^\circ$. 
% The existential part of $\alpha^{\tmod}$ is trivial. 
% For the universal part we have to show that every element of $D$ realizes the 
% positive part of a type in $\Sigma$. 
% Suppose not, and let $d\in D$ be such that $\lnot\tau_S^+(d)$ for all $S\in 
% \Sigma$. 
% Then we have $(D\times\omega,V_\pi) \not\models \tau_S^+((d,k))$ for all $k$.
% That is, there are infinitely many elements not realizing the positive part of 
% any type in $\Sigma$. 
% Hence we have $(D\times\omega,V_\pi) \not\models \dqu y.\bigvee_{S\in\Sigma} 
% \tau_S^+(y)$. 
% Absurd, because that is part of $\posdbnfolque{\vlist{T}}{\Pi}{\Sigma}$.
% \end{proof}

\begin{proofof}{Theorem~\ref{t:bi-aut}}
The proof of the first part is based on a fairly routine comparison, based on
Proposition~\ref{p-1P}, of the acceptance games $\mathcal{A}(\bbA^{\tmod},\bbS)$
and $\mathcal{A}(\bbA,\bbS^{\om})$.
(In a slightly more general setting, the details of this proof can be found 
in~\cite{Venxx}.)

For part~2, the direction from right to left is immediate by Theorem \ref{t:mlaut}.
%the observation  by XXXX~\cite{xxxxxxxx} that $\muML \equiv \Aut{(\ofo)}$.
The opposite direction follows from the following equivalences, where we use
the bisimilarity of $\bbS$ and $\bbS^{\om}$ (Fact~\ref{prop:tree_unrav}):
\begin{align*}
\bbA \text{ accepts } \bbS
  & \text{ iff } \bbA \text{ accepts } \bbS^{\om}
  & \tag{$\bbA$ bisimulation invariant}
\\ & \text{ iff } \bbA^{\tmod} \text{ accepts } \bbS
  & \tag{equivalence~\eqref{eq:crux}}
\end{align*}

It remains to be checked that the construction $(\cdot )^{\tmod}$, which has
been defined for arbitrary automata in $\Aut(\oslang)$, transforms 
both $\wmso$-automata and $\nmso$-automata into automata of the right kind.
This can be verified by a straightforward inspection at the one-step level.
\end{proofof}

\begin{remark}{\rm
% As a corollary of the previous two propositions we find that 
% \begin{itemize}
% 	\itemsep 0 pt
% 	\item $\AutW(\ofo) \equiv \AutW(\ofoe)/{\bis}$, and
% 	\item $\AutWC(\ofo) \equiv \AutWC(\ofoei)/{\bis}$.
% \end{itemize}
In fact, we are dealing here with an instantiation of a more general phenomenon 
that is essentially coalgebraic in nature.
In~\cite{Venxx} it is proved that if $\oslang$ and $\oslang'$ are two one-step
languages that are connected by a translation $(\cdot )^{\tmod}: \oslang' \to 
\oslang$ satisfying a condition similar to \eqref{eq-1cr}, then we find that 
$\Aut(\oslang)$ corresponds to the bisimulation-invariant fragment of 
$\Aut(\oslang')$: $\Aut(\oslang) \equiv \Aut(\oslang')/{\bis}$.
This subsection can be generalized to prove similar results relating
$\AutW(\oslang)$ to $\AutW(\oslang')$, and $\AutWC(\oslang)$ to 
$\AutWC(\oslang')$.
}\end{remark}
