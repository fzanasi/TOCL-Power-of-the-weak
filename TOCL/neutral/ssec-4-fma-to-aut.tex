% !TEX root = ../00CFVZ_TOCL.tex
\subsection{From Formulas to Automata}
   \label{sec:mc-to-parity}

In this subsection we focus on the meaning preserving translation in the
opposite direction, viz., from formula to automata.
In our set-up we need the one-step language to be closed under conjunctions and
disjunctions.

\begin{theorem}\label{t:fortoaut}
Let $\oslang$ be a on-step language that is closed under taking conjunctions and 
disjunctions.
Then there is an effective procedure that, given a formula $\xi \in \mu\oslang$
returns an automaton $\bbA_{\xi}$ in $\Aut(\oslang)$, which satisfies the 
following properties:
\begin{enumerate}[(1)]
\item $\bbA_\xi$ is equivalent to $\xi$;
\item  $\bbA_\xi \in \AutW(\oslang)$ if $\xi \in \mu_{N}\oslang$;
\item  $\bbA_\xi \in \AutWC(\oslang)$ if $\xi \in \mu_{C}\oslang$.
\end{enumerate}
\end{theorem}

As in the case of the translation from automata to formulas, the proof of part
(1) of this theorem, is a straightforward variation of the standard proof showing
that any fixpoint modal formula can be translated into an equivalent modal 
automaton (see for instance~\cite[Section 6]{Ven08}). 
The point is to show that this standard construction transforms formulas from 
$\mu_{N}\oslang$ and $\mu_{C}\oslang$ into automata of the right kind.
In fact, we will confine our attention to proving the (second and) third part
of the theorem; the fact that the input formula belongs to the alternation-free
fragment of $\mu\oslang$ enables a slightly simplified presentation of the 
construction.

Our first observation is that without loss of generality we may confine 
attention to \emph{guarded} formulas.

\begin{definition}
An occurrence of a bound variable $p$ in $\xi  \in \mu\oslang$ is called 
\emph{guarded} if there is a modal operator between its binding definition and 
the variable itself. 
A formula $\xi  \in \mu\oslang$ is called \emph{guarded} if every occurrence of
every bound variable is guarded.
\end{definition}

There is a standard construction, going back to \cite{Koz83}, which transforms
any formula $\xi$ in $\mu\oslang$ into an equivalent guarded $\xi^\flat \in 
\mu\oslang$, and it is easily verified that the construction $(\cdot)^{\flat}$
restricts to the fragments $\mu_{N}\oslang$ and $\mu_{C}\oslang$.
It therefore suffices to show that any guarded formula in $\mu_{N}\oslang$
($\mu_{C}\oslang$, respectively) can be transformed into an equivalent
continuous-weak $\oslang$-automaton.

In the remainder of this section we will show that any guarded formula $\xi$ 
in $\mu_{C}\oslang$ can be transformed into an equivalent continuous-weak 
$\oslang$-automaton $\bbA_{\xi}$; the analogous result for $\mu_{N}\oslang$ will
be obvious from our construction.

We will prove the result by induction on the so-called weak alternation depth
of the formula $\xi$.
We only consider the $\mu$-case of the inductive proof step; that is, we 
assume that $\xi$ can be obtained as a formula in the following grammar, for 
some set $\qprop$ of variables:
\begin{equation}
\label{eq:gr}
\phi \isbnf 
   q \mid \psi \mid \phi\lor\phi \mid \phi\land\phi \mid
   \nxt{\al}(\ol{\phi},\ol{\psi}) \mid \mu q. \phi,
\end{equation}
where $q \in \qprop$, every $\psi,\psi_{j}$ is $\qprop$-free, and 
$\al(\ol{a},\ol{b}) \in \cont{\oslang}{\ol{a}}$.
We write $\phi \subf' \xi$ if $\phi$ is a subformula of $\xi$ according to 
the grammar \eqref{eq:gr} (that is, we consider $\qprop$-free formulas $\psi$
to be atomic).
Furthermore, we inductively assume that for every $\qprop$-free formula $\psi
\subf' \xi$ we have already constructed an equivalent continuous-weak 
$\oslang$-automaton $\bbA_{\psi} = \tup{A_{\psi},a_{\psi},\tmap_{\psi},
\pmap_{\psi}}$.
(Observe that every subformula of a guarded formula it itself guarded, so that 
we are justified to apply the induction hypothesis.)

Define
$\Psi \isdef \{ \psi \subf' \xi \mid \psi\text{ $\qprop$-free} \}$, let $\Phi$
consist of all formulas $\phi \not\in \Psi$ that occur as some $\phi = \phi_{i}$
in a formula $\nxt{\al}(\ol{\phi},\ol{\psi}) \subf' \xi$, and set $A_{\Psi} 
\isdef \{ a_{\psi} \mid \psi \in \Psi \}$ and $A_{\Phi} \isdef \{ a_{\phi} \mid 
\phi \in \Phi \}$.
It now follows by guardedness of $\xi$ that there is a unique map $(\cdot)^{o}$ 
assigning a one-step formula $\chi^{o} \in \oslang(A_{\Psi} \cup A_{\Phi})$
to every formula $\chi \subf' \xi$ such that 
\[\begin{array}{lll}
   q^{o}    & = & \delta_{q}^{o}
\\ \psi^{o} & = & a_{\psi}
\\ (\phi_{0}\lor\phi_{1})^{o}  & = & \phi_{0}^{o} \lor \phi_{1}^{o}
\\ (\phi_{0}\land\phi_{1})^{o} & = & \phi_{0}^{o} \land \phi_{1}^{o}
\\ \nxt{\al}(\phi_{1},\ldots,\phi_{n},\psi_{1},\ldots,\psi_{k})^{o} & = &
   \al(a_{\phi_{1}},\ldots,a_{\phi_{n}},a_{\psi_{1}},\ldots,a_{\psi_{k}})
\\ (\mu q. \delta_{q})^{o}     & = & \delta_{q}^{o},
\end{array}\]
where $\delta_{q}$ is the unique formula $\delta$ such that $\mu q.\delta 
\subf' \xi$.
% \bigskip\hrule\bigskip

It is easy to verify that for every $\chi \subf' \xi$, the formula $\chi^{o}$ 
is continuous in $A_{\Phi}$.
We may thus assume without loss of generality that every $\chi^{o}$ belongs to
the syntactic fragment $\cont{\oslang}{A_{\Phi}}$.
(Should this not be the case, then by Theorem~\ref{t:osnf-cont} we may replace
$\chi^{o}$ with an equivalent formula that does belong to this fragment.)
We are now ready to define the automaton $\bbA_{\xi} = \tup{A_{\xi},a_{I},
\tmap_{\xi},\pmap_{\xi}}$ by putting 
\[\begin{array}{lll}
A_{\xi} &\isdef& \{ a_{\xi} \} \cup A_{\Phi} \cup \bigcup_{\psi\subf'\xi}A_{\psi}
\\ a_{I}    &\isdef& a_{\xi}
\\ \tmap_{\xi}(a,c) &\isdef& 
   \left\{\begin{array}{ll}
    \chi^{o} & \text{if } a = a_{\chi} \text{ with } \chi \in \{ \xi \} \cup \Phi
   \\ \tmap_{\psi}(a,c) & \text{if } a \in A_{\psi}.
   \end{array}\right.
\\ \pmap_{\xi}(a) &\isdef& 
   \left\{\begin{array}{ll}
   1 & \text{if } a \in \{ a_{\xi} \} \cup A_{\Phi}
   \\ \pmap_{\psi}(a) & \text{if } a \in A_{\psi}.
   \end{array}\right.
\end{array}\]
It is then straightforward to check that $\bbA_{\xi}$ is a continuous-weak
$\oslang$-automaton.
Proving the equivalence of $\xi$ and $\bbA_{\xi}$ is a routine exercise.


